\chapter{प्रकरणग्रन्थाः }
प्रस्थानत्रयान्तर्गतास्सर्वे ग्रन्था व्याख्योपव्याख्यासहिताः निरूपिताः । प्रस्थानत्रयानन्तर्गता अद्वैतसिद्धान्तप्रतिपादकाश्च मार्कि पञ्चशताधिकाः सव्याख्याः निर्व्याख्याश्च विभिन्नाचार्यकृता ग्रन्था उपलभ्यन्ते । ते च सिद्धान्तमात्रप्रदर्शनपराः, खण्डनमण्डनपराः, अनुभवात्मकभावाविष्करणपराः काव्यशैल्या सिद्धान्तप्रदर्शनपराश्चेति चतुर्धा विभज्यन्ते । एते सर्वेऽपि ग्रन्थाः -
शास्त्रैकदेशसम्बद्धं शास्त्रकार्यान्तरे स्थितम् ।
आहुः प्रकरणं नाम ग्रन्थभेदं विपश्चितः ।। 
इति लक्षणलक्षिता इति प्रकरणप्रस्थान एव प्रतिपादिताः । चतुर्विधा एते ग्रन्था असंख्या इति प्रकरणेऽस्मिन् प्रकरणग्रन्थानां प्रकरणग्रन्थकर्तृणाञ्च नामनिर्देशः परं क्रियते । तत्तत्प्रकरणग्रन्थप्रतिपादिता विषयाः, प्रकाशिताः? उत अप्रकाशिता इत्यादय इतरे विषया अद्वैताचार्यपरिच्छेदे प्रतिपादयिष्यन्ते । 
		ग्रन्थाः 										कर्तारः
1. अज्ञानतिमिरदीपकः 					कृष्णानन्दसरस्वती
2. अज्ञानध्वान्तचण्डभास्करः 			अमरेश्वरशास्त्री
3. अणुवेदान्तः 							रामशास्त्री
4. अद्वैतग्रन्थः								अप्पय्यदीक्षितः
5. अद्वैतग्रन्थः								महादेवसरस्वती
6. अद्वैतचन्द्रिका (वाराणस्यां मुद्रिता)	सुदर्शनाचार्यः
7. अद्वैतचिन्तामणिः						कुमारभवस्वामी
8. अद्वैतचिन्तामणिः						रङ्गोजिभट्ट
9. अद्वैतचिन्तामणिः						श्रीदेवः
10. अद्वैतचिन्तामणिः						सुन्दरेशः
11. अद्वैतचूडामणिः						चिद्धनानन्दगुरुशिष्यः
12. अद्वैतजलजातः						पाण्डुरङ्गः
13. अद्वैतजलजातम्						अच्युतशर्मामोडकः
14. अद्वैततत्वदीपः						नित्यानन्दाश्रमः 
15. अद्वैततत्वप्रबोधिनी					साधुशान्तिनाथः
16. अद्वैततरङ्गिणी						रामेश्वरभट्टः
17. अद्वैतदर्पणम् 							अप्पकविः
18. अद्वैतदर्पणम् 							भजनानन्दः
19. अद्वैतदर्पमव्याख्या-भावप्रकाशिका	भजनानन्दः
20. अद्वैतदीपिका						कामाक्षी
21. अद्वैतदीपिका (वाराणस्यां मुद्रिता)	गोपालशास्त्री
22. अद्वैतदीपिका 						रामेश्वरभट्टः
23. अद्वैतदीपिका 						नृसिंहाश्रमी
24. अद्वैतदीपिकाव्याख्या-विवरणम्	नारायणाश्रमः 
25. अद्वैतदीपिकाव्याख्या				सदानन्दव्यासः
26. अद्वैतदीपिकाव्याख्या प्रकाशः		सुन्दरराजः
27. अद्वैतनवनीतम् 						कृष्णावधूतः 
28. अद्वैतनिर्णयः							अच्युतमुनिः
29. अद्वैतनिर्णयसंग्रहः					रामानन्दतीर्थः
30. अद्वैतषद्यभाष्यम् 					सोमनाथव्यासः
31. अद्वैतपारिजातम् 						नीलकण्ठतीर्थः
32. अद्वैतपञ्चरत्नम् (सोपानपञ्चकम्) 	शङ्कराचार्यः
33. अद्वैतपञ्चरत्नव्याख्या 				विमलभूधरः
34. अद्वैतपञ्चरत्नव्याख्या-किरणावली	बालकृष्णानन्दः 
35.अद्वैतपञ्चरत्नविवृत्तिः				नित्यानन्दसरस्वती
	  निर्वाणपञ्चकविवृतिरित्यपरनामायं ग्रन्थः वाराणस्यां मुद्रितः ।
	  
36. अद्वैतप्रकाशः						दुर्गाप्रसादयतिः 
37. अद्वैतप्रकाशः 						महादेवसरस्वती
38. अद्वैतप्रकाशकः						वासुदेवज्ञानयतिः
39. अद्वैतप्रकाशिका						सोमनाथव्यासः
40. अद्वैतब्रह्मतत्वप्रकाशिका				वीरराघवयज्वा
	  ग्रन्थोयं नेल्लूर नगरे तेलुगुलिप्यां मुद्रितः । 
41. अद्वैतब्रह्मसुधा						योगीन्द्रशान्ताश्रमशिष्यः
42. अद्वैतमतप्रकाशः						अय्यण्णादीक्षितः
43. अद्वैतमतसर्वस्वम् 					वेङ्कटेशशास्त्री 
44. अद्वैतबोधदीपिका					चिदम्बरयोगी
45. अद्वैतबोधदीपिका					नृसिम्हभट्टः
46. अद्वैतब्रह्मसुधाकारिका 				गोविन्दानन्दसरस्वती 
47. अद्वैतभूषणम्							महादेवसरस्वती
48. अद्वैतभूषणव्याख्या-आनन्ददीपिका वासुदेवेन्द्रः
49. अद्वैतमकरन्दः						लक्ष्मीधरः
50. अद्वैतमकरन्दव्याख्या-रसाभिव्यञ्जिका स्वयम्प्रकाशः
51. अद्वैतमकरन्दः 						सदाशिवब्रह्मेन्द्रसरस्वती
	 ग्रन्थोऽयं कुम्मधोणस्थशाङ्करमठे उपलभ्यते । अस्या व्याख्यापि विज्ञानदीपिकानाम्नी अमुद्रिता विद्यते ।।
52. अद्वैतमतैक्यप्रकाशः 				ऐयणाचार्यः
53. अद्वैतमार्ताण्डः 						ब्रह्मानन्दतीर्थस्वामी 
	 मुद्रितोऽयं ग्रन्थः। अस्य कर्ता पेरुमङ्गलग्रामवासी दक्षिणदेशीयः आधुनिकश्चेति ज्ञायते ।। 
54. अद्वैतमुक्तासरः 						लोकनाथाध्वरी
55. अद्वैतमुक्तासरव्याख्या-कान्तिः		लोकनाथाध्वरी
56. अद्वैतरत्नप्रकाशः					अमरेश्वरशास्त्री
57. अद्वैतरत्नाकरः 						अनन्तभट्टः
58. अद्वैतरत्नाकरः						नारायणतीर्थः
59. अद्वैतरत्नाकरव्याख्या-रत्नप्रभा	अमरदासः
60. अद्वैतरसमञ्जरी 						नल्लासुधीः
61. अद्वैतरहस्यम् 						रामानन्दतीर्थः
62. अद्वैतवाक्यार्थः						त्र्यम्बकभट्टः
63. अद्वैतवादः							जगन्नाथसरस्वती
64. अद्वैतविद्यातिलकम् 					समरपुङ्गवदीक्षितः 
65. अद्वैतविद्यातिलकव्याख्या-दर्पणम्	धर्मय्यदीक्षितः  
66. अद्वैतविद्यामुकुरः					रङ्गराजाध्वरी
67. अद्वैतविद्याविजयः 					कृष्णशास्त्री
68. अद्वैतशतकम्						गङ्गाधरः
69. अद्वैतशास्त्रार्थविचारः 				हरियशश्शर्मा
	  ग्रन्थोऽयं पञ्चापसूच्यां दृश्यते 
70. अद्वैतशास्त्रोद्धारः						रङ्गोजिभट्टः
71. अद्वैतसाम्राज्यम्						कृष्णानन्दसरस्वती
72. अद्वैतसारः 							सुन्दरमूर्तिः
73. अद्वैतसिद्धान्तगुरुचन्द्रिका 			चन्द्रिकाचार्यः 
74. अद्वैतसिद्धान्तगुरुचन्द्रिकाव्याख्या 	अमृतरसझरी-चन्द्रिकाचार्यः
75. अद्वैतसिद्धान्तगुरुचन्द्रिकासारबोधः माधवतीर्थः
76. अद्वैतसिद्धान्तविद्योतनम् 			ब्रह्मानन्दसरस्वती
77. अद्वैतसिद्धान्तवैजयन्ती				त्र्यम्बकभट्टः	
78. अद्वैतसिद्धान्तसारसंग्रहः				नारायणाश्रयः
79. अद्वैतसुधा							नारायणसरस्वती
80. अद्वैतसुधाबिन्दुः 					कृष्णाः
81. अद्वैतसुधासारः 						ज्ञानानन्दः
82. अद्वैतसंग्रहः							रामभद्रदीक्षितः 
83. अद्वैतस्तवः 							रघुनाथसूरिः
84. अद्वैतस्तवव्याख्या-ज्ञानञ्चनशलाका पाण्डुरङ्गशास्त्री
	  सव्याख्योयं ग्रन्थः मध्यप्रान्तीयबरार्ग्रन्थसूच्यां दृश्यते ।
85. अद्वैतविद्याविनोदः 					अच्युतशर्मा
86. अद्वैतागमहृदयसंग्राहकश्लोकाः 		शान्त्यानन्दः
87. अद्वैतानन्दलहरी						अद्वैतानन्दतीर्थः
88. अद्वैतानन्दसागरः					रघूत्तमतीर्थः
ग्रन्थोऽयं जयपुरपोटीकानासूच्यां (222 Vol II) दृश्यते । अस्य कर्ता पुरुषोत्तमतीर्थ - ब्रह्मानन्दसरस्वती - स्वयम्प्रकाशतीर्थशिष्यः रघूत्तमतीर्थ इति ज्ञायते । अस्य प्रतिलेखनकालः 1650 श 1683 A D इति ज्ञायते।।
89. अद्वैतनन्दानुभूतिः 					सीतारामशास्त्री 
90. अद्वैतानुभवोल्लासः					सुब्रह्मण्येन्द्रः
91. अद्वैतानुभूत्यष्टकम् 					जीवन्मुक्तभिक्षुः	
92. अद्वैतानुभूत्यष्टकव्याख्या-भावार्थदीपिका-जीवन्मुक्तभिक्षुः 
	  सव्याख्योऽयं ग्रन्थः लन्दननगरस्थभारतकार्यालयमुद्रितग्रन्थसूच्यां 
	  (IOL, Vol I दृश्यते )।।
93. अद्वैतानुभूतिः						शङ्कराचार्यः
94. अद्वैतानुसन्धानम् 					नृसिम्हाश्रमी
95. अद्वैतानुसन्धानम् 					रामब्रह्मेन्द्रशिष्यः
	  ग्रन्थोऽयं सरस्वतीमहालयस्थसूच्यां दृश्यते ।। 
96. अद्वैतामृतम् 							जगन्नाथसरस्वती
97. अद्वैतामृतव्याख्या-तरङ्गिणी 		जगन्नाथसरस्वती
98. अद्वैतामृतमञ्चरी						अच्युतशर्मा
99. अद्वैतामृतसारः						द्विजेन्द्रलालपुरकायस्थः 
100. अद्वैतामृतसारकम् 					आदिनारायणः
101. अद्वैतामोदः							वासुदेवशास्त्री अभ्यङ्करः
102. अधिकरणकुञ्चिकाः					अप्पय्यदीक्षितः
103. अदिकरणञ्जरी						चित्सुखाचार्यः
104. अधिकरणसङ्गतिः					चित्सुखाचार्यः
105. अधिष्ठानविवेकः					पूर्णप्रकाशः
106. अधिष्ठानविवेकः					प्रकाशानन्दः
107. अध्यात्मवासुदेवः					रमणीदासः
	   ग्रन्थोऽयममुद्रितः रायल आसियाटिक सोसाइटि बम्बई पुस्तकालयसूच्यां दृश्यते ।। 
108. अध्यात्मोपदेशविधिः 				शङ्कराचार्यः
109. अध्यारोपणम् 						वासुदेवशिष्यः
110. अध्यारोपणप्रकरणम् 				पुरुषोत्तमसरस्वतीशिष्यः
111. अनात्मश्रीविगर्हणम् 					शङ्कराचार्यः
112. अनुभवामृतम् 						ज्ञानदेवः
113. अनुभवामृतम् 						बालकृष्णानन्दः
114. अनुभवविलासः						हरिहरपरमहंसः 
115. अनुभववेदान्तप्रकरणम् 			हस्तामलकः
116. अनुभूतिप्रकाशः						विद्यारण्यः
117. अनुभूतिलेशः 						वामनपण्डितः
118. अनुभूतिविवरणम् सव्याख्यम्		भास्करानन्दः
119. अपरोक्षानुभूतिः						शङ्कराचार्यः
120. अपरोक्षानुभूतिविवरणम् 			नित्यानन्दशिष्यः
121. अपरोक्षानुभूतिदीपिका				विद्यारण्यः
122. अवधूतसम्प्रदायपञ्चरत्नावली 		शुकानन्दयोगी
	   ग्रन्थोऽयं बरोडापुस्तकालये लभ्यते 
123. अविदितसुखदुःखेतिपद्यटीका 		नन्दीशः
124. असङ्गात्मप्रकरणम् 				शङ्करभारतीतीर्थः
125. असङ्गात्मप्रकाशिका				गोविन्देन्द्रः
126. आत्मज्ञानप्रकरणम्					शङ्कराचार्यः
127. आत्मज्ञानप्रकरणदीपिका			पूर्णानुभवः
128. आत्मज्ञानविवरणव्याख्या-सम्प्रदायतिलकम्-अनन्तराममुनिः
129. आत्मज्ञानोपदेशविधिः				शङ्कराचार्यः
130. आत्मज्ञानोपदेशविधिटीका 		रामचन्द्रसरस्वती
ग्रन्थोऽयं रायल आसियाटिक सोसाइटी कल्कत्ता पुस्तकालये (3.G.67) लभ्यते ।
131. आत्मज्ञानोपदेशविधिटीका 		आनन्दगिरिः
132. आत्मतत्वम् 						राघवः
	   ग्रन्थोऽयं बम्बई विश्वविद्याल यपुस्तकालयसूच्यां दृश्यते । 
133. आत्मतत्वविवेकसारः 				रामानन्दसरस्वती
134. आत्मदर्शनयोगः					सच्चिदानन्दस्वामी 
135. आत्मपरीक्षा						भास्करदीक्षितः 
136. आत्मपुराणम्						शङ्करानन्दः
137. आत्मपुराणव्याख्या					रामकृष्णः
138. आत्मपञ्चकम् 						नीलकण्ठः
	   ग्रन्थोऽयं लन्दननगरस्थपुस्तकालये लभ्यते ।
139. आत्मपक्राशिका					नन्दरामतर्कवागीशः 
140. आत्मप्रकाशिकाव्याख्या 			काशीरामः
141. आत्मबोधः (ग्रन्थोऽयं उज्जैन् सूच्यां दृश्यते) ईश्वरकृष्णः 
142. आत्मबोधः							शङ्कराचार्यः
143. आत्मबोधव्याख्य-दीपिका 		विश्वेश्वरपण्डितः 
144. आत्मबोधटीका						चित्सुखशिष्यः
145. आत्मबोधव्याख्या					पद्भपादाचार्यः
146. आत्मबोधव्याख्या					बोघेन्द्रसरस्वती
147. आत्मबोधव्याख्या					ब्रह्मानन्दयोगी
148. आत्मबोधव्याख्या-बालबोधिनी	स्वयम्प्रकाशनारायणः
	  अमुद्रितोऽयं आत्मबोधिन्यपरनामायं ग्रन्थः जयपुर पोटीखानासूच्यां दृश्यते ।
149. आत्मबोधव्याख्या					सच्चिदानन्दयोगी
150. आत्मबोधव्याख्या					मधुसूदनसरस्वती
151. आत्मबोघप्रकरणम् 					वासुदेवयतिः
152. आत्मबोधलहरी					चिदानन्दयोगी
153.	आत्मलाभः (अद्वैतसारः)			त्यागराजदीक्षितः
154.	आत्मविद्याविलासः				सदाशिवब्रह्मेन्द्रः
155.	आत्मविवेचनिका					कुबेरानन्दवर्णिः
156.	आत्मसोपानम् 					केशवशास्त्री
157.	आत्मानात्मविवेक					शङ्कराचार्यः
158. आत्मानात्मविवेकः					विश्वेश्वरः
159. आत्मानात्मविवेकः					सदाशिवब्रह्म
160. आत्मानात्मविवेकः					स्वयम्प्रकाशयोगी
161. आत्मानात्मविवेकप्रकाशिका 		सदाशिवब्रह्म
162. आत्मानात्मविवेकसंग्रहः			परमहंसः
163. आत्मानात्मविवेचनम् 				महेश्वरानन्दः
164.	आत्मानसन्धानम् 				सदाशिवब्रह्म
165.	आत्मानुभवस्तुतिः				बालब्रह्मानन्दः
166.	आत्मार्कबोधः						गोविन्दभट्टः
167. आनन्ददीपिकाव्याख्या विशुद्धदृष्टिः रामनाथविद्वान् 
168. आनन्ददायिनी						नरसिम्हभट्टः
169. आनन्दरससागरः					आनन्दाश्रमः
170. आनन्दलहरी चन्द्रिका-व्याख्यासहिता अप्पय्यदीक्षितः
171. ईश्वरप्रतिपत्तिप्रकाशः				मधुसूदनसरस्वती
172. उदासीनसाधुस्तोत्रम् देवतीर्थस्वामी । ग्रन्थोऽयं निर्णयसागरमुद्रणालये बम्बईनहरे मुद्रितः ।
173. उपदेशसारः 						विश्वनाथः
174. उपदेशपञ्चदशीव्याख्या सत्यनारायणशर्मा । ग्रन्थोऽयं चैखाम्वा मुद्रणालये मुद्रितः ।
175. उपदेशसाहस्री						शङ्कराचार्यः
176. उपदेशसाहस्रीव्याख्या पदयोजनिका रामतीर्थः
177. उपदेशसाहस्रीव्याख्या-विवरण्म 	बोधनिधिः
178. उपदेशसाहस्रीव्याक्या-गूढार्थदीपिका	अखण्डधामा
179. उपदेशसाहस्त्रीव्याख्या				आनन्दगिरिः
180.	उपदेशशिखामणिः					त्यागराजः
181.	उपनिषत्सारसंग्रहः				सुब्रह्मण्यसूरिः
182.  उपासनाप्रयोगः					वेङ्कटेश्वरदीक्षितः
		ग्रन्थोऽयमुद्रितः तिरुविडमरुदूरदेवालयपुस्तकालये लभ्यते । 
183. एकश्लोकः							शङ्कराचार्यः
184. एकश्लोकव्याख्या स्वात्मदीपनम्	स्वयम्प्रकाशानन्दः
185.	एकश्लोकप्रकरणम् 				लीलाविभूतिः
अमुद्रितोऽयं ग्रन्थः लन्दननगरपुस्तकालये लभ्यते । 	
186. कप्यासकौमुदी 					रामशास्त्री
187.	कर्मांकर्मविवेकः					रामचन्द्रेन्द्रः
188. कर्माकर्मविवेकनौका 				वासुदेवेन्द्रः
189.	काशीपञ्चकम्						शङ्कराचार्यः
190.	कैवल्यगाथा						कृष्णानन्दसरस्वती
191.	कैवल्यदीपिका					कृष्णाः
192. कैवल्यदीपिकाव्याख्या				कृष्णाः
193.	कैवल्यदीपिका					बौधानन्दः
194.	कैवल्यदीपिकाव्याख्या-स्नेहः	नारायणप्रियः
195. कैवल्यनवनीतम् 					शङ्कुकविः
        ग्रन्थोऽयं कल्पात्तिमुद्रणालये मुद्रितः ।
196. कैवल्यरत्नम् 						वासुदेवज्ञानमुनिः
197. कौपीनपञ्चकम्						शङ्कराचार्यः
198. गुणत्रयविवेकः 						स्वयम्प्रकाशः
199. ख्यातिवादः						शङ्करचैतन्यभारती
200. चतुर्ग्रन्थिसंग्रहः					अनन्तकृष्णशास्त्री
201. चतुर्मतसारसङ्ग्रहः				अप्पय्यदीक्षितः
202. चतुर्वेदमहावाक्यचिन्तामणिः		आदिनारायणः
203. चित्सुधार्या(स्वाराज्यसर्वस्वम्)	नीलकण्ठतीर्थः
204. चिदचिदग्रन्थिविवेकः				खयम्प्रकाशः 
205. चिदद्वैतकल्पतरूः 				चिन्मयमुनिः
206. चिदद्वैतकल्पतरुव्याख्या-परिमल चिन्मयमुनिः
207. चिदचिच्छारीरकब्रह्मसिद्धिः		जगदीश्वरशास्त्री
208. चैतन्यप्रकरणम् 					ब्रह्मवित्प्रवरदासः 
209. जगन्मिथ्यात्वदीपिका				रामचन्द्रबुधः
210. जगन्मिथ्यात्वदीपिका				रामेन्द्रयोगी
211. जीवन्मुक्ततरङ्गिणी					देवदत्तशर्मा
	   गन्थोऽयममुद्रितः लन्दननगरपुस्तकालये लभ्यते।
212. जीवन्मुक्तलक्षणम् 					दत्तात्रेयः
	   ग्रन्थोऽयं सरस्वतीमहालयवर्णनात्मकपुस्तकसूच्याः त्रयोदशे भागे (D.C. TSML Vol XIII) मुद्रितः।
213. जीवन्मुक्तिकल्याणम् 				नल्लाध्वरिः
	   ग्रन्थोऽयं वाणीविलासमुद्रणालये मुद्रितः।
214. जीवन्मुक्तिविवेकः 					विद्यारण्यः
215. जीवन्मुक्तिविवेकव्याख्या पूर्णानन्दकौमुदी अच्युतशर्मा
216. जीवन्मुक्तिविवेकव्याख्या-दीपिका	पूर्णानन्दः
	   ग्रन्थोऽयममुद्रितः बन्दरकारप्राच्यभाषासंशोधनालयपुस्तकालये (B.O.R.I) लभ्यते । 
217. जीवन्मुक्तिविवेकसारसंग्रहः			सदेकानन्दः
218. ज्ञानतारावलिः						चिद्रूपानन्दनाथः
	   ग्रन्थोऽयममुद्रितः मद्रासराजकीयपुस्तकालये (MGOML) लभ्यते ।। 
219. ज्ञानबोधः							शुकयोगी
	   ग्रन्थोऽयं सरस्वतीमहालये अडयारपुस्तकालये च लभ्यते ।।
220. ज्ञानविलासकाव्यम् 				जगन्नाथः
	   ग्रन्थोऽयममुद्रितस्सरस्वतीमहालये लभ्यते ।।
221. ज्ञानाङ्कुरम् 							कैपललक्ष्मीनृसिम्हः
	   मद्रासराजकीयकोशागारे लभ्यते (MGOML) । 
222. ज्ञानाङ्कुशम् 						शङ्कराचार्यः
223. ज्ञानाङ्कुशः सव्याख्यः				नीलकण्ठदीक्षितः
224. तत्वत्रयनिरूपणम् 					नारायणाश्रमः 
225. तत्वदीपः							वल्लभदीक्षितः
226. तत्वदीपनम् 						नृसिम्हाश्रमी
227. तत्वनिरूपणम् 					त्र्यम्बकभट्टः
228. तत्वप्रदीपिका						चित्सुखाचार्यः
229. तत्वप्रदीपिकाव्याख्या-मानसनयनप्रसादिनी प्रत्यक्स्वरूपः
230. तत्वप्रदीपिकाव्याख्या-भावद्योतनिका सुखप्रकाशः
231. तत्वबोधः							लक्ष्मीनारायणदासः
	   ग्रन्थोऽयं कल्कत्तासंकृतकलाशालापुस्तकालये लभ्यते ।
232. तत्वबोधः 							वासुदेवेन्द्रः 
233. तत्वबोधः							मुकुन्दमुनिः
234. तत्वबोधः							महादेवसरस्वती
235. तत्वबोधः							तत्वबोधभगवान् 
236. तत्वबोधप्रकरणम्					रामचन्द्रबुधः
237. तत्वमस्यखण्डार्थनिरूपणम् 		रामानन्दसरस्वती
238. तत्वमुक्तावलिः						स्वयम्प्रकाशः
239. तत्वशुद्धिः							ज्ञानघनः
240. तत्वशुद्धिव्याख्या 					उत्तमज्ञयतिः
241. तत्वसुधा (दक्षिणामूर्त्यष्टकव्याख्या) स्वयम्प्रकाशः
242. तत्वसिद्धान्तबिन्दुः				अनन्तरामः
243. तत्वम्पदार्थविवेकः					पूर्णानन्दः
244. तत्वम्पदार्थविवेकः					पूर्णप्रकाशः
245.	तत्वम्पदार्थशोधनप्रकारः			नृसिम्हाश्रमी
246. तत्वम्पदार्थलक्ष्यैकशतकम् 		रामचन्द्रेन्द्रः
247. तत्वम्पदार्थलक्ष्यैकशतकव्याख्या-तरङ्ग रामचन्द्रेन्द्रः
248. तत्वानुभवः						गोविन्देन्द्रः
249. तत्वानुसन्धानम् 					महादेवसरस्वती
250. तत्वानुसन्धानव्याख्या-अद्वैतचिन्ताकौस्तुभः महादेवसरस्वती
251. तत्वानुसन्धानव्याख्या				स्वयम्प्रकाशयोगी
252. तत्वालोकः						आनन्दगिरिः
253. तत्वालोकव्याख्या-तत्वप्रकाशिका प्रज्ञानानन्दः
254. तर्कसङ्गहः							आनन्दगिरिः
255. तात्पर्यदीपिका						राघवानन्दः 
	    ग्रन्थोऽयं अण्णामलैविश्वविद्यालये मुद्रितः ।
256. तिमिोरोद्धाटनम् 					कृष्णानन्दसरस्वती
257. त्रिपात्तत्वविवेकः					रामचन्द्रेन्द्रः
258. त्रिपात्तत्वादिसप्तप्रकरणी			उपनिषद्ब्रह्म
259. दशंहससूत्रटीका					विट्ठलबुधः
260. दशश्लोकी							शङ्कराचार्यः
261. वशश्लोकीव्याख्या-सिद्धान्तबिन्दः मधुसूदनसरस्वती
262. दशश्लोकीव्याख्या सिद्धाान्तविन्दुसारः तारानाथशर्मा
263. दशश्लोकी-सिद्धान्तबिन्दुलघुटीका नारायणयतिः
264. दशश्लोकी-सिद्धान्तबिन्दुसन्दीपनम् पुरुषोत्तमसरस्वती
265. दशश्लोकी-सिद्धान्तबिन्दु-तत्वम्पदार्थविवेकः पूर्णानन्दसरस्वती 
266. दशश्लोकी-सिद्धान्तबिन्दु-न्यायरत्नावली ब्रह्मानन्दसरस्वती
267. दशश्लोकी-न्यायरत्नावलीप्रदीपिका कृष्णकान्तः
268. दशश्लोकी-सिद्धान्तबिन्दुव्याख्या सच्चिदानन्दः
269. दशश्लोकीसिद्धान्तबिन्दुव्याख्या 	शिवलालशर्मा
270. दशोपनिषद्रहस्यम् 				रामचन्द्रपण्डितः
271. दहरविद्याप्रकाशः					परमशिवेन्द्रः
272. द्दग्दृश्यविवेकः (वाक्यसुधा)		भारतीतीर्थः
273. दृग्दृश्यविवेकव्याख्या 				ब्रह्मानन्दभारती
274. दृश्योन्मार्जनिकाप्रकरणम् 		शङ्कुशास्त्री
275. दुर्जनोक्तिनिरासः 					त्यागराजशास्त्री
	   राजुशास्त्रीत्यपरनामायं त्यागराजशास्त्री। ग्रन्थोऽयं श्रीविद्यामुद्रणालये कुम्भघोणे मुद्रितः । 
276. देहचतुष्टयम् 						साक्षात्कारप्रकाशः
	    ग्रन्थोऽयममुद्रितः बरोडा ओरियण्टल पुस्तकालये (BRD.) लभ्यते । 
277. द्वादशमञ्जरिकास्तोत्रम् 			शङ्कराचार्यः
278. द्वादशमञ्जरिकाव्याख्या मकरन्दः	स्वयम्प्रकाशानन्दः
279. द्वादशमहावाक्यसिद्धान्तः			आनन्दः
	   अमुद्रितोऽयं ग्रन्थः कल्कत्ता संस्कृतकलाशालापुस्तकालये लभ्यते । 
280. नयमञ्जरी 							अप्पय्यदीक्षितः
281. नवमणिमाला						सदाशिवब्रह्म
282. नामविवेकः						लीलाविभूतिः
283. नामविवेकव्याख्या					उपनिषद्ब्रह्मेन्द्रः
	   समूलग्रन्थोऽयं व्याख्याग्रन्थः बरोडापुस्तकालये उपनिषद्ब्रह्मेन्द्रमठे च लभ्यते ।
284. निगमान्तार्थचन्द्रिका 				नारायणाश्रमः 
285. निजतत्वामृतसारः (रसः)		परमेश्वरभारती
286. निर्वाणाष्टकम् 						शङ्कराचार्यः
287. निर्वाणाष्टकव्याख्या				गङ्गाधरेन्द्रः
288. नृसिम्हविज्ञापना					नृसिम्हाश्रमी
289. नैष्कर्म्यसिद्धिः						सुरेश्वराचार्यः
290. नैष्कर्म्यसिद्धिविवरणम् 			अखिलात्मा
291. नैष्कर्म्यसिद्धिभावप्रकाशिका		चित्सुखाचार्यः
292. नैष्कर्म्यसिद्धिचन्द्रिका 			ज्ञानोत्तमः 
293. नैष्कर्म्यसिद्धिव्याख्या-विद्यासुरभिः ज्ञानामृतयतिः 
294. नैष्कर्म्यसिद्धिसारार्थः				रामतीर्थः
295.	नैष्कर्म्यसिद्धिव्याख्या-सारथिः	रामदत्तः
		ग्रन्थोऽयं बन्दरकारप्राच्यभाषासंशोधनालयमुद्रिते नैष्कर्म्यसिद्धिग्रन्थे निर्दिष्टः। 
296. न्यायचूडामणिः					माधवसरस्वती
297. न्यायचूडामणिव्याख्या वेदान्तमन्दाकिनी नारायणसरस्वती
298. न्यायप्रमाणमञ्जरी सव्याख्या		राघवेन्द्रशिष्यः
299. पञ्चकोशविमर्शिनी					त्यागराजः
300. पञ्चकोशविवेकः					भारतीतीथेः
301. पञ्चदशी								विद्यारण्यः
302. पञ्चदशीव्याख्या-तात्पर्यबोधिनी	रामकृष्णः
303. पञ्चदशीव्याख्या-कल्याणपीयृषः	लिङ्गनसोमयजी 
304. पञ्चदशीव्याख्या-पूर्णानन्देन्दुकौमुदी अच्युतशर्मा
305. पञ्चदशीव्याख्या-वृत्तिप्रभाकरः 	निश्चलदासः
306. पञ्चदशीव्याख्या-विशुद्धदृष्टिः		रामानन्दसरस्वती
307. पञ्चदशी							रामब्रह्मेन्द्रः
308.	पञ्चप्रकरणी						रामदासः
309. पञ्चप्रक्रिया							सर्वज्ञात्मा
310. पञ्चप्रक्रियाव्याख्या					आनन्दगिरिः
311. पञ्चप्रक्रियाव्याख्या					पूर्णविद्यः
312. पञ्चब्रह्माख्यविवरणम्				लीलानन्दः
		ग्रन्थोऽयममुद्रितस्सरस्वतीमहालये लभ्यते ।
313. पञ्चरत्नमालिका					शङ्कराचार्यः
314. पञ्चरत्नकारिकाः					सदाशिवः
315. पञ्चरत्नव्याख्या प्रकाशः			पाण्डुरङ्गः
316. पञ्चरत्नव्याख्या कल्पवल्ली			अभिनवनारायणेन्द्रः
317. पञ्चरत्नप्रकाशः						सुब्रह्मण्यः
318. पञ्चरत्नविवृत्तिः 					वासुदेवेन्द्रशिष्यः 
		ग्रन्थोऽयं बरोडा ओरियण्डल इन्स्टिटयूट पुस्तकालये लभ्यते -
319. पञ्चावस्थाविवेकः 					वासुदेवयतिः
320. पञ्चीकरणम् 						अभिनवसदाशिवः
321. पञ्चीकरणम्							शङ्कराचार्यः
322. पञ्चीकरणवार्तिकम् 				सुरेश्वराचार्यः
323. पञ्चीकरणवार्तिकव्याख्या-वार्तिकाभरणम् अभिनवनारायणः
324. पञ्चीकरणवार्तिकविवरणदीपिका	शिवनारायणानन्दः
325. पञ्चीकरणवार्तिकम् 				गौडपादः
326. पञ्चीकरणवार्तिकपाठः				उपेन्द्रदत्तः
327. पञ्चीकरणव्याख्या अद्वैतागमहृदयम् शान्त्यानन्दः
328. पञ्चीकरणविवरणम् 				आनन्दगिरिः
329. पञ्चीकरणविवरणव्याख्या			आनन्दगिरिः
330. पञ्चीकरणविवरणम् 				स्वयम्प्रकाशः
331. पञ्चीकरणविवरणम् 				प्रज्ञानानन्दः
332.	परब्रह्मतत्वनिरूपणम् 				एकोजीराजः
333. परमसिद्धान्तसारः					स्वयम्प्रकाशानन्दः
334. परमसिद्धान्तसारः					स्वयम्प्रकाशशिष्यः
335. परमहंसचर्या						सदाशिवब्रह्म
336.	परमहंससंहिता					लक्ष्मणपण्डितः
337. परमाक्षरविवेकः					रामचन्द्रबुधः
338. परमाद्वैतदर्शनम् 					रामचन्द्रबुधः
339. परमाद्वैतदर्शनम्					लीलाविभूतिः
340. परमाद्वैतदर्शनव्याख्या				उपनिषद्ब्रह्म
		समूलग्रन्थोऽयं व्याख्याग्रन्थः बरोडापुस्तकालये लभ्यते । 
341. परमाद्वैतसिद्धान्तपरिभाषा			उपनिषद्ब्रह्म
		ग्रन्थोेऽयं मद्रासराजकीयपुस्तकालयेऽमुद्रित उपलभ्यते । 
342. परमार्थबोधः						मुकुन्दमुनिः 
343.	परमार्थसारः						आदिशेषः
344. परमार्थसारव्याख्या-विवरणम् 	राघवानन्दः
345. परमार्थसारप्रकाशिका				वासुदेवयतिः
346. परमामृतम् 						महादेवसरस्वती
347. परमामृतम् 						मुकुन्दराजः
348. पूरुषार्थबोधः 						ब्रह्मानन्दसरस्वती
349. पुरुषार्थरत्नाकरः					पुरुषोत्तमतीर्थः
		ग्रन्थोऽयं मद्रासराजकीयप्राचीनपुस्तकालये लभ्यते । 
350. पुरुषार्थरत्नाकरः					रङ्गनाथसूरिः	
351.	पूर्णपुरुषार्थचन्द्रोदयः				जातवेदस्
		ग्रन्थोऽयं मद्रासराजकीयहस्तलिखितपुस्तकालये लभ्यते ।। 
352. प्रचण्डराहूदयम् 					घनश्यामः
353. प्रणवकल्पप्रकाशः					गङ्गाघरेन्द्रः
354. प्रणवदीपिका						ब्रह्मानन्दयोगी
355. प्रणवार्थकारिकाः					सुरेश्वराचार्यः
356. प्रणवार्थप्रकाशिका 				विज्ञानात्मा
357. प्रणवार्थप्रकाशिका					ब्रह्मानन्दयोगी
358. प्रणावार्थप्रकाशिका सव्याख्या 	कैवल्यानन्दः 
359. प्रत्यक्तत्वचिन्तामणिः				सदानन्दव्यासः
360. प्रत्यक्तत्वप्रकाशिका				वासुदेवेन्द्रः
361. प्रत्यक्त्वस्वप्रकाशवादः			कृष्णः
362. प्रपञ्चमिथ्यात्वम् 					गोतमशङ्करः
		ग्रन्थोऽयं बन्दरकार ओरियण्टलपुस्तकालये लभ्यते । 
363. प्रबोधचन्द्रिका						ब्रह्मेन्द्रसरस्वती
364. प्रबोधचन्द्रोदयः 					कृष्णामिश्रः
365. प्रबोधन्द्रोदयव्याख्या				चण्डिदासः
		ग्रन्थोऽयं मद्रासराजकीयहस्तलिखितपुस्तकालये लभ्यते ।। 
366. प्रबोधचन्द्रोदयव्याख्या प्रबोधप्रकाशः सुब्रह्मण्यपाण्डरिः
		ग्रन्थोऽयं सरस्वतीमहालये मद्रासराजकीयपुस्तकालये च लभ्यते । 
367. प्रबोधचन्द्रोदयव्याख्या सञ्जीविनी	घनश्यामः
		ग्रन्थोऽयं सरस्वतीमहालये मद्रासराजकीयपुस्तकालये च लभ्यते । 
368. प्रबोधचन्द्रोदय प्रकाशः			रामदासः
		ग्रन्थोऽयं लन्दननगरस्थभारतकार्यालयपुस्तकालये, सरस्वतीमहालयपुस्तकालये रायल आसियाटिक सोसाइटि बाम्बेनगरे च लभ्यते । निर्णयसागरमुद्रणालये मुद्रितः । 
369. प्रबोधचन्द्रोदयव्याख्या चन्द्रिका 	नन्दिल्लगोपमन्त्रिशेखरः
		ग्रन्थोऽयं निर्णयसागरमुद्रणालये मुद्रितः।
370. प्रबोधचन्द्रोदयी चिच्चिन्द्रिका 	गणेशः 
371. प्रबोधचन्द्रोदयव्याख्य 				महेश्वरन्यायालङ्कारः
		ग्रन्थोऽयं रायल आसियाटिक सोसाइटि कलकत्तानगरे लभ्यते ।
372. प्रबोधचन्द्रोदयव्याख्या नाटकाभरणम्  गोविन्दामृतः
		अमुद्रितोयं ग्रन्थ मद्रासराजकीयहस्तलिखितप्राचीनपुस्तकालये लभ्यते । 
373. प्रबोधशतकम् 						ब्रह्मानन्दसरस्वती
374. प्रबोधसुधाकरः					शङ्कराचार्यः
375. प्रबोधसुधाकरः					सूर्यसूरिः
376. प्रबोधामृतम् 						श्रीरामः
		ग्रन्थोऽयं शङ्कराचार्यमठपुस्तकालये लभ्यते ।
377.	प्रमाणतत्वम् 						त्र्यम्बकभट्टः
378. प्रमाणमाला (प्रमाणरत्नमाला)	आनन्दबोधः
379. प्रमाणरत्नमालाव्याख्या निबन्धनम् 	अनुभूतिस्वरूपः
380. प्रमाणरत्नमाला सम्बन्धोक्तिः		चित्सुखाचार्यः
		तात्पर्यदीपिका इत्यपि नामान्तरमस्य दृश्यते ।। 
381. प्रमाणलक्षणम् 						सर्वज्ञात्मा
382. प्रमाणविभागश्लोकाः				स्वयम्प्रकाशयोगी
383. प्रमाणप्रवृत्तिनिर्णयः				विमुक्तात्मा 				
384. प्रमेयरत्नावली 					बलदेवविद्याभूषणः
385. प्रश्नावलिः							जडभरतः
		बन्दरकार ओरियण्टलपुस्तकालये लभ्यते । 
386. प्रश्नोत्तररत्नमालिका 				शङ्कराचार्यः
387. प्रस्थानभेदः						मधुसूदनसरस्वती
388. प्रौढानुभूतिप्रकरणम् 				शङ्कराचार्यः
389. बिम्बदृष्टिः							अमरेश्वरशास्त्री
390. बृहद्वाक्यवृत्तिः						वेदोत्तमभट्टारकः
391. बृहद्वाक्यवृत्तिवाक्यदीपिका		आनन्दस्वरूपः
392. बृहद्वाक्यवृत्तिव्याख्या				आनन्दज्ञानः
393.	बोधसारः							नरहरिः
394. बोधसारव्याख्या					दिवाकरः
395. बोधार्या							सदाशिवबोधेन्द्रः
396. बोधार्याव्याख्या-स्वात्मानन्द्रप्रकाशिका 	प्रज्ञानाश्रमः
397. बोधैक्यसिद्धिः						अच्युतशर्मा
398. बोधैक्यसिद्धि व्याख्या				अच्युतशर्मा
399. ब्रह्मतत्वसुवोधिनी					कृष्णानन्दसरस्वती
400. ब्रह्मतत्वसुबोधिनी					वेङ्कटस्वामी	
401. ब्रह्मतत्वसुबोधिनी					गोपालानन्दाश्रमः
402. ब्रह्मतारकषोडशसमाधिः			रामचन्द्रेन्द्रः
403. ब्रह्मप्रणवदीपिका					रामचन्द्रेन्द्रः
404. ब्रह्मभावनानिर्णयः 					पूर्णानन्दः
405. ब्रह्मविचाराधिकार निरूपणम्		रामाशास्त्री
406. ब्रह्मविदाशीर्वादपद्धतिः				विद्यारण्यः
407. ब्रह्मविद्यातरङ्गिणी					नारायणयोगी
408. ब्रह्मविद्यातरङ्गिणीव्याख्या			त्यागराजदीक्षितः 
409. ब्रह्मविद्यासुधार्णवः					परमानन्दतीर्थः
410. ब्रह्मविन्निधिः						वेङ्कटयोगी
411. ब्रह्मसिद्धान्तः						कात्यायनः
412. ब्रह्मसिद्धान्तव्याख्या-तत्वबोधिनी	 कात्यायनः
413. ब्रह्माद्वैतप्रकाशिका					भाववागीश्वरः
414. ब्रह्मानन्दप्रकाशिका				जगज्जीवनः
415. ब्रह्मानन्दप्रदीपिका					नारायणः
416. ब्रह्मानन्दविलासः					शाश्वतानन्दः
417. ब्रह्मानन्दविलासः					स्वामियतिः
418. ब्रह्मानुचिन्तनम् 					शङ्कराचार्यः
419. ब्रह्मामृतम्							जयकृष्णतीर्थः
420. ब्रह्मावबोधः						मुकुन्दमुनिः
421. ब्रह्माह्निकम् 						वासुदेवब्रह्मेन्द्रसरस्वती
		सङ्कलनात्मकोऽयं ग्रन्थः आनन्दसागरमुद्रणालये मायूरक्षेत्रेच मुद्रितः ।
422. ब्रह्मोत्तरतत्वमाला					शङ्करमिश्रसुकविः
423. भावनापुरुषोत्तमनाटकम् 			रत्नखेट - श्रीनिवासः़
		अमुद्रितोऽयं ग्रन्थः सरस्वतीमहालये लभ्यते ।
424. भावाज्ञानप्रकाशिका				शिवरामः
425. भावाज्ञानप्रकाशिका				नृसिम्हाश्रमी
426. भिक्ष्वष्टकम् 						सच्चिदानन्दस्वामी
427. भेदतमोमार्ताण्डशतकम् 			रामचन्द्रेन्द्रः
428. मतत्रितयसर्वस्वम् 				वेङ्कटेशशास्त्री
		ग्रन्थोऽयं मद्रासराजकीयहस्तलिखितपुस्तकालये लभ्यते ।
429. मनीषापञ्चकम् 					शङ्कराचार्यः 
430. मनीषापञ्चकतात्पर्यदीपिका 		सदाशिवब्रह्म
431. मनीषापञ्चकव्याख्या				सदाशिवब्रह्मशिष्यः
432. मनीषापञ्चकव्याख्या 				शिवयोगीन्द्रशिष्यः
433. मनीषापञ्चकमधुमञ्जरी 			गोपालबालयतिः
434. मनीषापञ्चकव्याख्या मधुमञ्जरी	नृहिम्हाश्रमी
435. मनीषापञ्चकव्याख्या पञ्चरत्नविवृतिः वासुदेवेन्द्रः
436. मनीषापञ्चकव्याख्या				विमलबुधाकरः
		सिद्धान्तपञ्चकव्याख्या इत्यपरनामायं ग्रन्थः पञ्चाबविश्वविद्यालये लभ्यते । 
437. मनोनियमनम् 						सदाशिवब्रह्म
		वाणीविलासमुद्रणालये श्रीरङ्गक्षेत्रे मुद्रितः ।
438. महावाक्यनिरूपणप्रक्रिया			सुब्रह्मण्यमणेरीकरः
439. महावाक्यार्थमञ्चरी 				महेश्वरानन्दः
440. महावाक्यर्थमञ्जरीव्याख्या परिमलः महेश्वरानन्दः
441. महावाक्यप्रकरणम् 				विज्ञानेश्वरः
442. महावाक्यादर्शः					जयरामः
443. महावाक्यार्थदर्पणम्				कृष्णानन्दभारती
444. महावाक्यरत्नाुवलिः				रामचन्द्रेन्द्रः
445. महावाक्यरत्नावली प्रभा			रामचन्द्रेन्द्रः
446. महावाक्यरत्नावली प्रभा			त्रिलोकीनाथमिश्रः
		ग्रन्थोऽयं वाराणस्यां मुद्रितः ।
447. महावाक्यरत्नावलीप्रभालोचनम् 	उपनिषद्ब्रह्मेन्द्रः
448. महावाक्यरत्नावलीव्याख्या		देवकीनन्दनः
		ग्रन्थोऽयं भारतकार्यालयपुस्तकालये लन्दननगरे लभ्यते । 
449. महावाक्यविवरणम्				विद्यारण्यः
450. महावाक्यवृत्तिदीपः 				अद्वैतयोगी
		ग्रन्थोऽयममुद्रितः रायलआसियाटिककल्कत्तापुस्तकालये लभ्यते । 
451. महावाक्यरत्नावली किरणावलिः	उपनिषद्व्रह्मेन्द्रः
452. महावाक्यरत्नावलीविवरणम् 		उपनिषद्ब्रह्मेन्द्रः
453. महावाक्यार्थमञ्जरी					अच्युतशर्मा
454. मानदीपिका						बालब्रह्मानन्दः
455. मानदीपिकास्मृतिसारसंग्रहः		बालब्रह्मानन्दः
456. मानमाला							अच्युतकृष्णानन्दः
457. मानमालाविवरणम् 				रामानन्दभिक्षुः
458. मानसोल्लासः						सुरोेश्वराचार्यः
459. मनसोल्लासवृत्तान्तविलासः		रामतीर्थः
460. मायापञ्चकम् 						शङ्कराचार्यः
461. मिथ्यापवादविध्वंसनम् 			चन्द्रशेखरसूरिः
		ग्रन्थोऽयं अडयारपुस्तकालये लभ्यते । 
462. मुक्तिपरिणयनाटकम्				सुन्दरदेवः
		अमुद्रितोयं ग्रन्थः सरस्वतीमहालये लभ्यते ।
463. मुक्तिस्वयंवरनाटकम् 				शेषाश्रपपण्डितः
		अमुद्रितोऽयं ग्रन्थस्सरस्वतीभवनपुस्तकालये वाराणसीनगरस्थे लभ्यते ।
464. मुक्तोपदेशः							हरिभास्करः
465. मूलाविद्यानिरासः					सुब्रह्मण्यशर्मा
466. मोक्षधर्मदीपिका 					नन्दनार्यः
467. मोक्षनिर्णयः						सुरेश्वराचार्यः
468. मोक्षसिद्धिः							कृष्णगिरिः
469. मोक्षोपायः							अभिनन्दः
		निर्णयसागरमुद्रणालये मुद्रितः ।
470. लघुवाक्यवृत्तिप्रकरणम् 			शङ्कराचार्यः
471. लघुवाक्यवृत्ति प्रकाशिका			रामानन्दसरस्वती
472. लघुवार्तिकम्						उत्तमश्लोकः
473. लघुवार्तिकव्याख्या- न्यायसुधा	उत्तमश्लोकः
474. लघुवासुदेवमननम्					वासुदेवेन्द्रः
475. लिङ्गभङ्गमुक्तिशतकम् 				उपनिषद्ब्रह्म
476. वाक्यप्रकरणम् 					अवधूतशिवयोगी
477. वाक्यवृत्तिः						शङ्कराचार्यः
478. वाक्यवृत्तिव्याख्या					आनन्दगिरिः
479. वाक्यसुधाव्याख्या					विश्वेश्वरपण्डितः
480. वाक्यसुधाव्याख्या					भीमदासभूपालः
481. वाक्यसुधाटीका					रामचन्द्रयतिः
482. वाक्यसुधाव्याख्या 				विश्वेश्वरः
483. वाक्यामृतम् 						विश्वेश्वरः
484. वाक्यार्थदर्पणम्					रामतीर्थः
485. वाक्यार्थप्रकाशिका				विश्वनाथः
486. वाचारम्भणप्रकरणम्				नृसिम्हाश्रमी
487. वासुदेवमननम्						वासुदेवपरमहंसः
488. वासुदेवमननसंग्रहसारः			वासुदेवशिष्यः
489. विज्ञानदीपिका						पद्मपादः
490. विज्ञाननौका (स्वरूपानुसन्धानम्) शङ्कराचार्यः
491. विज्ञाननौका पदव्याख्या			श्रीकुदः
492. विज्ञानामृतम्						विज्ञानयतिः
		अमुद्रितोऽयं ग्रन्थस्सरस्वतीभवनपुस्तकालये वाराणसीनागरे लभ्यते । 
493. विद्वदनुभवः 						शङ्करानन्दसरस्वती
		अमुद्रितोऽयं ग्रन्थस्सरस्वतीभवनपुस्तकालये वाराणसीनगरस्थे लभ्यते ।
494. विद्वन्मोदतरङ्गिणी					चिरञ्जीविभट्टाचार्यः
495. विदेहमुक्तिप्रकरणम् 				रामचन्द्रेन्द्रः
496. विद्यापरिणयनाटकम् 				आनन्दरायमखी
		निर्णयसागरमुद्रणालये मुद्रितोऽयं ग्रन्थः ।
497. विद्याप्राकाशः (उपनिषत्कारिकाः) विद्यारण्यः
498. बिभ्रमविवेकः						मण्डनमिश्रः
499. विवेकचूडामणिः					शङ्कराचार्यः
500. विवेकचूडामणिः					वासुदेवेन्द्रः
501. विवेकमञ्जरी						हस्तामलकः
502. विवेकमञ्जरीव्याख्या				आनन्दप्रकाशः
503. विवेकमञ्जरीभाष्यम् हस्तामलकीयभाष्यम् 	शङ्कराचार्यः
504. विवेकमञ्जरीव्याख्या हस्तमलकीयव्याख्या	स्वयम्प्रकाशमुनिः
505. विवेकमार्ताण्डः					विश्वरूपदेवः
506. विवेकमुकुरः						नृसिम्हसरस्वती 
507. विवेकसारः						वासुदेवशिष्यः
508. विवेकसिन्धुपरमार्थबोधः			मुकुन्दमुनिः
		अमुद्रितोऽयं ग्रन्थस्सरस्वतीमहालये बरोडापुस्तकालये लन्दननगरपुस्तकालये च लभ्यते । 
509. विवेकामृतम् 						गोपालः
510. वृत्तिदीपिका 						कृष्णभट्टः
511. वेदान्तकतकः						नीलकण्ठतीर्थः
512. वेदान्तकौमुदी						रामाद्वयः
513. वेदान्तकौमुदी व्याख्या				रामाद्वयः
514. वेदान्तकल्पलतिका				मधुसूदनसरस्वती
515. वेदान्तडिण्डिमः 					नृसिम्हसरस्वती
516. वेदान्ततत्वसारः					विद्येन्द्रसरस्वती
		अमुद्रितोऽयं ग्रन्थस्सरस्वतीमहालये लभ्यते ।
517. वेदान्ततत्वोदयः					नित्यानन्दः
		अमुद्रितोऽथं ग्रन्थः भारतकार्यालयपुस्तकालये लन्दननगरे लभ्यते ।
518. वेदान्ततात्पर्यनिवेदनम्				गोविन्दभट्टः
519. वेदान्ततात्पर्यनिवेदनव्याख्या		मुकुन्दः
520. वेदान्तार्थविभावना					मुकुन्दः
521. वेदान्तार्थविभावना					नारायणतीर्थः
		ग्रन्थोऽयं बरोडापुस्तकालये लभ्यते । 
522. वेदान्तवादार्थः						कृष्णानन्दः
		ग्रन्थोऽयं बरोडापुस्तकालये लभ्यते । 
523. वेदान्ततात्पर्यविचारः				हरियशोमिश्रः
524. वेदान्तदीपिका						वासुदेवशिष्यः
525. वेदान्तदीपिका 					चोक्कनाथदीक्षितः 
526. वेदान्तनामरत्नसहस्रव्याख्या 		परमशिवेन्द्रः
527. वेदान्तन्यायसंग्रहः (अद्वैतचिन्तामणिः) सुन्दरेशः
528. वेदान्तपरिभाषा					ब्रह्मेन्द्रसरस्वती
529. वेदान्तपरिभाषा					काशीनाथशास्त्री
530. वेदान्तपरिभाषा					धर्मराजध्वरिः 
531. वेदान्तपरिभाषाव्याख्या प्रकाशिका पेद्दादीक्षितः
532. वेदान्तपरिभाषाव्याख्या प्रकाशिका अनन्तकृष्णशास्त्री
533. वेदान्तपरिभाषाव्याख्या शिखामणिः रामकृष्णदीक्षितः
534. वेदान्तपरिभाषाव्याख्या अर्थदीपिका  शिवदत्तः
535. वेदान्तपरिभाषाव्याख्या भूषणम् 	नारायणविद्वान् 
536. वेदान्तपरिभाषाव्याख्या अर्थदीपिका धनपतिसूरिः
537. वेदान्तपरिभाषाव्याख्या अर्थदीपिका जीवानन्दविद्यासागरः
538. वेदान्तपरिभाषाव्याख्या तत्वबोधिनी  वेदाद्रिसूरिः
539. वेदान्तपरिभाषाव्याख्या आशुबोधिनी  कृष्णनाथः
540. वेदान्तपरिभाषाव्याख्या मणिप्रभा 	अमरदासः 
541. वेदान्तपरिभाषासङ्ग्रहः				रामवर्मा
542. वेदान्तप्रकरणम् 						वासुदेवयतिः
543. वेदान्तमञ्जरी							अद्वैतेन्द्रसरस्वती
544. वेदान्तमननम् 							रामचन्द्रेन्द्रशिष्य़ः
545. वेदान्तरत्नमाला						देवनारायणः
546. वेदान्तरहस्यम् 						वेदान्तवागीशभट्टः
547. वेदान्तलहरी							सच्चिदानन्दस्वामी
548. वेदान्तवार्तिकम् 						योगस्फुरणानुभवः
549. वेदान्तवादसंग्रहः						त्यागराजशास्त्री
550. वेदान्तसप्तशती							विश्वानुभवः
551. वेदान्तसारः							शङ्कराचार्यः
552. वेदान्तसारसङ्ग्रहः 					अद्वयानन्दः
		अमुद्रितोऽयं ग्रन्थ अडयार पुस्तकालये लभ्यते ।
553. वेदान्तसारसङ्ग्रहः					शिवरामभट्टः
554. वेदान्तसारसङ्ग्रहव्याख्या आत्मबोधामृतम् शिवरामभट्टः 
555. वेदान्तसारसङ्ग्रहप्रकरणम्			सीतारामः
		अमुद्रितोऽयं ग्रन्थः बरोडापुस्तकालये लभ्यते ।
556. वेदान्तसारपञ्चीकरणम् 				कैवल्याश्रमी
557. वेदान्तसारसद्रत्नावलिः 				जगज्जीवनः
558. वेदान्तसारसङ्ग्रह मननम् 			चिद्धनभारती 
559. वेदान्तसारः							सदानन्दसरस्वती
560. वेदान्तसारव्याख्या बालबोधिनी		आपदेवः 
561. वेदान्तसारव्याख्या सुबोधिनी			नृसिंहसरस्वती 
562. वेदान्तसारव्याख्या विद्वन्मनोरञ्जिनी	रामतीर्थः
563. वेदान्तसारव्याख्या ब्रह्मबोधिनी		रामचन्द्रानन्दः
564. वेदान्तसारव्याख्या भावबोधिनी		रामशरणशास्त्री
565. वेदान्तसारसर्वस्वम् 					माधवसरस्वती
566. वेदान्तसारसङ्ग्रहः					पापयाराध्यः
567. वेदान्तसारवार्तिक राजसङ्ग्रहः		सुरेश्वराचार्यः
568. वेदान्तसिद्धान्तकारिकामञ्जरी			चित्सुखाचार्यः
569. वेदान्तसिद्धान्तकल्पवल्ली				सदाशिवब्रह्म
570. वेदान्तसिद्धान्तकल्पवल्ली व्याख्या केसरावली सदाशिवब्रह्म
571. वेदान्तसिद्धान्तचन्द्रिका 				रामानन्दसरस्वती
		मुद्रितोऽयं ग्रन्थः गोपालनारायणमु्द्रणालये बाम्बे नगरे । 
572. वेदान्तसिद्धान्तचूडामणिः				शिवरामानन्दशिष्यः
573. वेदान्तसिद्धान्तनिःश्रेणिः				विट्टलबुधाकरः
574. वेदान्तसिद्धान्तरत्नमाला 				विश्वनाथः 
575. वेदान्तसिद्धान्तरहस्यम् 				रामचन्द्रः
576. वेदान्तसिद्धान्तरहस्यप्रकाशः			कल्याणरामः
577. वेदान्तसिद्धान्तसारः					उमामहेश्वरः
578. वेदान्तसिद्धान्त सारसग्रङ्हः			सदानन्दः
579. वेदान्तसिद्धान्तसूक्तिमञ्जरी			गङ्गाधरेन्द्रः
580. वेदान्तसिद्धान्तादर्शः 					मोहनलालः
581. वेदान्तसङ्गग्रहः							स्वयम्प्रकाशानन्दः
582. वेदान्तसङ्ग्रहः						अच्युताश्रमी
583. वेदान्तसङ्ग्रहव्याख्या					महादेवः
584. वेदान्तसंज्ञाप्रकरणम् 					आदित्यपूर्णः
		गुजराती मुद्रणालये बाम्बे नगरे मुद्रितः । 
585. वेदान्तहृदयम् 							वरदपण्डितः
586. वेदान्तामृतम् 							गोपालानन्दः
587. वेदान्तार्थनिरूपणम् 					श्रीवत्सशर्मा
588. वेदान्तार्थविवेचन भाष्यम् 				मुकुन्दः 
589. वेदान्तार्थविभावना						नागयणतीर्थः
590. वेदान्तार्थसारसङ्ग्रहः					सीतारामः
591. वेदार्थतत्वनिर्णयः						लिङ्गाध्वरी
592. वैदिकसिद्धान्तसङ्ग्रहः				सुखचिद्रूपभारती
593. वैदिकसिद्धान्तसङ्गग्रहः				नृसिम्हाश्रमी
594. वैराग्यतरङ्गम् 							श्रीनाथः
595. वैराग्यपञ्चकम् 							वासुदेवेन्द्रः
596. व्यामोहविद्रावणम्						रामसुब्रह्मण्यशास्त्री
597. शतप्रश्नोत्तरी							अयोध्याप्रसादः
		ग्रन्थोऽयं बरोडापुस्तकालये लभ्यते । 
598. शतश्लोकी 							शङ्कराचार्यः
599. शतश्लोकी								ईश्वरतीर्थः
600. शाब्दनिर्णयः							प्रकाशात्मा
601.	शाब्दनिर्णयव्याख्या					आनन्दबोधः
		अमुद्रितोऽयं ग्रन्थः तिरुवनन्तपुरपुस्तकालये मद्रासराजकीयहस्तलिखितपुस्तकालये च लभ्यते । 
602. शारीरक संक्षेपविवृत्तिः				महादेवसरस्वती 
603. शारीरक भाष्यवार्तिकम् 				दिव्यसिम्हमिश्रः
		(एकेनद्वे इत्यादिपद्यव्याख्या)
604. शारीरकव्याख्याप्रस्थानानि			गुरुस्वामिशास्त्री
605. शास्त्रसिद्धान्तलेशसङ्ग्रहः				वासुदेवब्रह्मेन्द्रः
606. शास्त्राकूतप्रकाशः						कृष्णानन्दसरस्वती
607. शास्त्रारम्भः							राजचूडामणिदीक्षितः 
608. शास्त्रारम्भसमर्थनम् 					त्र्यम्बकभट्टः
609. शुकाष्टकम् 							शुकाचार्यः
610. शुकाष्टकाव्याख्या						गङ्गाधरेन्द्रः
611. श्रुतिमतप्रकाशः							त्र्यम्बकभट्टः 
612. श्रुतिसारः								पूर्णप्रकाशः
613. श्रुतिसारसमुच्चयः						पूर्णानन्दः
614. श्रुतिसारसमुच्चयः						ब्रह्मनिराकारयोगी
615. श्रुतिसारसमुद्धरणम् 					तोटकाचार्यः
616. श्रुतिसारसमुद्धरणव्याख्या 'तत्वदीपिका' सच्चिदानन्दः
617. श्रुतिसारसमुद्धरण व्याख्या				पूर्णात्मकृष्णः
618. श्रुतिसारोद्धारः							परमहंसः
619. श्रुतिस्मृतिसारसंग्रहः					बालबह्मानन्दः
620. श्लोकत्रयम्							रामचन्द्रेन्द्रः
621. श्रौताखण्डार्थसिद्धिः					रामानन्दसरस्वती 
622. षट्त्रिंशदद्वैततत्वमालिका				आदिनारायणः
623. षट्पदी									शङ्कराचार्यः
624. षट्पदीव्याख्या 						कविसरोजभिक्षुः
625. षट्पदीव्याख्या 						वैकुण्ठशिष्यः 
626. षट्पदीव्याख्या मञ्जरी					शङ्करानन्दतीर्थः
627. सच्चिदनुभवप्रकाशिका				वासुदेवब्रह्म
628. सत्सुखानुभवः							इच्छारामस्वामी 
629. सत्तासाम्यविवेकः						रामचन्द्रबुधः 
630. सदाचारप्रकरणम् 						शङ्कराचार्यः
631. सदाचारप्रकरणव्याख्या शुद्धधर्मपद्धतिः अच्युतशर्मा
632. सद्विद्याविलासः						त्यागराजशास्त्री
633. सप्तदशप्रकरणम् 						महेश्वरानन्दः
634. समन्वयसाम्राज्यसमर्थनम् 			हरिहरानन्दसरस्वती
		श्रीकरपात्रीजीति प्रसिद्धेनानेन कृतः ग्रन्थोऽयं "एम् एल जे" मुद्रणालये मद्रासनगरे मुद्रितः । 
635. समाधिप्रकिया 						अनन्तराममुनिः
636. सर्वसिद्धान्तसङग्रहः					बोधनिधिः
		ग्रन्थोऽयं विरुवनन्तपुरपुस्तकालये लभ्यते ।
637. सर्वसिद्धान्तसङ्ग्रहदीपिका			विट्ठलबुधाकरः 
638. सिद्धान्तचन्द्रिका						रामभद्रानन्दः
639. सिद्धान्तचन्द्रिकाव्याख्या उद्गारः		गङ्गाधरेन्द्रः 
640. सिद्धान्ततत्वम् 						अनन्तदेवः
641. सिद्धान्ततत्वव्याख्या समप्रदायनिरूपणम् अनन्तदेवः
642.सिद्धान्तदीपः							हयग्रीवाश्रमी
		वेदान्तसिद्धान्तदीपापरनामायं ग्रन्थः रायल आसियाटिक सोसाइटि पुस्तकालये कल्कत्ता नगरस्थे लभ्यते । 
643. सिद्धान्तदीपिका 						भवानीशङ्करः
		ग्रन्थोऽयं मद्रासहस्तालिखितराजकीयपुस्तकालये लभ्यते । 
644. सिद्धान्तपञ्जरम् 						विनायकः 
645. सिद्धान्तपरिभाषानिरूक्तिः				श्रीनिवासः
		ग्रन्थोऽयं मद्रासराजकीयपुस्तकालये लभ्यते । 
646. सिद्धान्तमुक्तावलिः					प्रकाशानन्दः
647. सिद्धान्तमुक्तावलीव्याख्या सिद्धान्तदीपिका नामादीक्षितः
648. सिद्धान्तरत्नम् 						बलदेवविद्यामूषणः 
649. सिद्धान्तरत्नमाला 					श्रीवत्सलाञ्च्छनशर्मा 
		ग्रन्थोऽयं जयपुरसूच्यां मद्रासराजकीयपुस्तकालये च लभ्यते । 
650. सिद्धान्तलेशसङग्रहः 					अप्पय्यदीक्षितः
651. सिद्धान्तलेशसङग्रहव्याख्या-कृष्णलङ्कारः अच्युतकृष्णानन्दः
652. सिद्धान्तलेशसङ्ग्रह - सिद्धान्तकौमुदी राघवानन्दः
653. सिद्धान्तलेशसङ्ग्रह सिद्धान्तसूक्तिमञ्जरी रामचन्द्रपूज्यपादः
		वेदान्तसिद्धान्तसूक्तिमञ्जर्यपरनामायं ग्रन्थः चौखाम्बामुद्रणालये मुद्रितः । 
654. सिद्धान्तलेशसङ्ग्रहव्याख्या			विश्वनाथः
655. सिद्धान्तलेशसङ्ग्रहसारः				वासुदेवब्रह्म 
		ग्रन्थोऽयं हिन्दीप्रचारसभामुद्रणालये मद्रासनगरे मुद्रितः ।
656. सिद्धान्तश्लोकत्रयम् 					रामचन्द्रेन्द्रः 
		ग्रन्थोऽयं मद्रासराजकीयपुस्तकालये लभ्यते । 
657. सिद्धान्तसर्वस्वम् 						लक्ष्मणपण्डितः
658. सिद्धान्तसङ्ग्रहः						रामः
659. सिद्धान्तसङ्ग्रहः						रुचिकरः 
660. सिद्धान्तामृतम् 						व्यङ्कटनाथः
661. सुज्ञानविंशतिः							मुकुन्दः
662. सोपाधिकब्रह्मविद्यादर्पणम् 			स्वयम्प्रकाशब्रह्मानन्दः
		ग्रन्थोऽयं मद्रासनगरे मुद्रितः ।
663. सर्ववेदान्ततात्पर्यसारसङ्ग्रहः			सुन्दररामशास्त्री
664. सर्ववेदान्तसङ्ग्रहः						सच्चिदानन्दस्वामी
665. सर्ववेदान्तसिद्धान्तसारसंग्रहः 		शङ्कराचार्यः 
666. सर्वसिद्धान्तसंग्रहव्याख्या				शेषगोविन्दः
		अमुद्रितोऽयं ग्रन्थ मद्रासराजकीयपुस्तकालये लभ्यते । 
667. स्वप्नोदितम् 							सदासिवब्रह्म
		ग्रन्थोऽयं सरस्वतीमहायसूच्यास्त्रयोदशे भागेे DCTSML Vol XIII मुद्रितः । 
668. स्वरूपदर्शनसिद्धाञ्जनम् 				रामब्रह्मेन्द्रः
669. स्वरूपब्रह्मभावना						स्वयम्प्रकाशः
670. स्वरूपविमर्शिनी सव्याख्या			चिदानन्दस्वामी
671. स्वरूपानुभवः							पद्मपादः
		ग्रन्थोऽयं ब्रह्मानन्दचिद्रत्ननाम्ना शङ्कराचार्यकृतत्वेन वाणीविलासमुद्रणालये श्रीरङ्गक्षेत्रे मुद्रितः । 
672. स्वरूपानुसन्धानम्					गौरीशङ्कर ओझा 
		भारतीयकार्यालयमुद्रितपुस्तकसूच्यां दृश्यतेऽयं ग्रन्थः लन्दन नगरे । 
673. स्वरूपानुसन्धानम् (विज्ञाननौका)	शङ्कराचार्यः
674. स्वरूपविवरणम् 						आनन्दगिरिः
675. स्वात्मनिरूपणम् 						शङ्कराचार्यः
676. स्वात्मनिरूपणव्याख्या				सच्चिदानन्दस्वामी
677. स्वात्मप्रकाशकम् 						सदानन्दः
		ग्रन्थोऽयं रायल असियाटिक सोसाइटि कल्कत्तानगर पुस्तकालये अमुद्रित उपलभ्यते । 
678. स्वात्मयोगप्रदीपः						अमरानन्दः
679. स्वात्मयोगप्रदीप प्रबोधिनी			अमरानन्दः
680. स्वात्मस्फूर्तिविलासः					त्यागराजदीक्षितः
681. स्वात्मस्फूर्तिविलासः व्याख्या			त्यागराजशास्त्री
682. स्वात्मादर्शः							शिवानन्देन्द्रः
683. स्वात्मानन्दचन्द्रिका					स्वानन्दयोगी
684. स्वात्मानन्दस्तोत्रम् 					विमलब्रह्म
685. स्वानुभवतरङ्गम् 						अद्वैतेन्द्रः
686. स्वानुभवविवेकसारमननम्			शिवरामयतिः 
687. स्वानुभवादर्शः							माधवाश्रमः
688. स्वानुभवादर्श-अर्थप्रकाशिका			माधवाश्रमः
689. स्वात्मानुभूतिमणिदर्पणः				वेङ्कटेशशास्त्री
690. स्वानुभूतिप्रकाशः						सदाशिवब्रह्म
691. स्वानुभूतिविलासः						कृष्णानन्दसरस्वती
692. स्वाराज्यसिद्धिः						गङ्गाधरेन्द्रः
693. स्वाराज्यसिद्धि कैवल्यकल्पद्रुमः		गङ्गाधरेन्द्रः
694. स्वाराज्यसिद्धि कैवल्यकल्पद्रुम-परिमलः कृष्णशास्त्री
695. स्वाराज्यसिद्धिव्याख्या				भास्करानन्दः
696. संक्षेपशारीरकम् 						सर्वज्ञात्मा
697. संक्षेपशारीरकव्याख्या सुबोधिनी		अग्निचित्पुरुषोत्तमः 
698. संक्षेपशारीरक तत्वबोधिनी			नृसिम्हाश्रमी
699. संक्षेपशारीरक विद्यामृतवर्षिणी 		राघवानन्दः			
700. संक्षेपशारीरक अन्वयार्थप्रकाशिका	रामतीर्थः
701. संक्षेपशारीरक सिद्धान्तदीपः 			विश्ववेदः 
702. संक्षेपशारीरक सम्बन्धोक्तिः			वेदानन्दः
703. संक्षेपशारीरकव्याख्या					प्रत्यग्विष्णुः
704. संक्षेपशारीरकसारसङ्ग्रहः				मधुसूदनसरस्वती 
705. हरिहराद्वैतभूषणम् 						बोधेन्द्रसरस्वती
		"हरिहरभेदधिक्कारः," "हरिहराद्वैत भूषणकारिका" इत्यपि नामान्तराण्यस्यैव ग्रन्थस्य दृश्यन्ते। अस्य कर्ता कुत्रचित् विश्वाधिकगुरुशिष्य इति दृश्यते । परन्तु विश्वाधिकगुरुशिष्यः बोधेन्द्रसरस्वत्येवेति ज्ञेयम् । 
