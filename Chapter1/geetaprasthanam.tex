\chapter{गीताप्रस्थानम् }
एवं असंख्याभिरुपनिषद्भिरद्वैतवेदान्तभावा प्रतिपादिताः । एवमपि उपनिषत्समय एव क्वचित् वैदिकधर्मविरोधिनां विभिन्नानां दर्शनानां सूक्ष्मस्सद्भावः कठोपनिषत्स्थात् - "येयं प्रेते विचिकित्सा मनुष्येऽस्तीत्येके नायमस्तीतिचैके" 1-20 मन्त्रात् ज्ञायते । एवं "नत्थि किरियं नत्थि कम्म नत्थि विरियम् " इत्यादिना सर्वशून्यत्ववादः निष्क्रियवादश्च वात्यावेगेन प्रचलित आसीदनन्तरभाविषु कालेषु । एतस्य सर्वशून्यवादस्य प्रतिक्षेपणाय जैनमतानुयायिनः " अत्थि उत्थानेति वा कर्मेति वा बलेति वा विरियेति वा पुरिसकारेति वा परक्कमेति वा" (भगवती सूत्रम् 1-3-5) इति सिद्धान्तं प्रकटयामासुः । जैनबौद्धधर्माणां तत्सामयिकेेषु नास्तिकदर्शनेषु सयुक्तिकत्वात् तत्तद्दर्शनाचार्याः समाजे समादृतास्समभूवन् । एवं श्वेताश्वतरोपनिषदि - 
" कालस्स्वभावो नियतिर्यदृच्छा भूतानि योनिः पुरुष इति चिन्त्यम् ।
संयोग एषां न त्वात्मभावात् आत्माप्यनीशस्सुखदुःखहेतोः ।। "(I - 2)
इति मन्त्रश्श्रूयते । एतत्प्रमाणादपि उपनिषत्काल एव विभिन्नसिद्धान्तानां सद्भाव ऊहितुं शक्यते । एतेषां विभिन्नानां सिद्धान्तानुयायिनां प्रभावेण प्रजासु विभिन्ना भावा प्रचालिता प्रचलिताश्चासन् । कारणादस्मात् भारतीयाध्यात्मिकवायुमण्डले लोकायतिकमतोदयस्समभूत् । वैदिकमतानां हानिरिवाभूत् । विभिन्नदर्शनानां खण्डनस्य वैदिकधर्मस्य पुनः प्रतिष्ठायाश्चावश्यकत्वं अनिवार्यं सञ्जातम् । तस्मात् भगवान् व्यासः पञ्चमवेदतुल्यं महाभारतं, महापुराणानि उपपुराणानि च निर्ममे तत्तत्पुराणपठनात् असंख्यानि दर्शनानि क्वचित् सनामनिर्देशं क्वचित् ससिद्धान्तानुवादं खण्डितानि दृश्यन्ते। महाभारतस्थशान्तिपर्वानुशासनिकपर्वणी महाभारतकालिक्याः आध्यात्मिकभावनायाः प्रत्यक्षं निदर्शनम् । महाभारतादिषु इतिहासग्रन्थेषु महापुराणोपपुराणादिषु च वैदिकमतप्रतिष्ठापनाय आध्यात्मिकविज्ञानप्रतिपादकाः निखिलवेदान्तसारभूताश्च गीताः स्थानमलभन्त । तादृश्यः आध्यात्मिकभावनाप्रधानाः गीता अद्वैतवेदान्तसाहित्यस्य द्वितीयं प्रस्थानम् । यद्यपि प्रतिपुराणं बह्व्यः गीताश्श्रूयन्ते तथापि याश्चाद्वैतवेदान्ताविरोधिन्यः, याश्चाद्वैताचार्यैर्व्याख्यातास्ता एव गीता अत्र वर्णमालाक्रमेण निर्दिश्यन्ते । तासु काश्चन गीताः" गीतासमुच्चये " मुद्रिताः, काश्चन पुरणान्तर्गताश्च विद्यन्ते । तास्सर्वा अपि निर्दिश्यन्ते च ।। 
1. अगस्त्यगीता - 
बिब्लियोथिका - इण्डिकाग्रन्थमालायां कल्कत्तानगरे मुद्रितस्य वराहपुराणस्य एकपञ्चाशत्तमादध्यायादारभ्य त्रिपञ्चाशत्तमाध्यायपर्यन्तेयं गीता (51 - 53) अगस्त्यगीतेति कथ्यते । 
भद्राश्व - अगस्त्यसंवादरूपायां अस्यां गीतायां अहङ्कार - अज्ञान - विज्ञानेन्द्रियाणामुत्पत्तिः, जगत्सृष्टिः, ज्ञानस्वरूपमित्येते विषयाः पशुपालकथाव्याजेन निरूपिताः ।। 
2. अनुगीता -
महाभारतान्तर्गतेयं गीता। महाभारतस्याश्वमेधिकपर्वणि षोडशाध्यायादारभ्या प्रवृत्तः कृष्णार्जुनसंवादः अनुगीताशब्देन व्यवहृतः । पूर्वोपदिष्टगीतार्थस्य संग्रामव्यग्रस्य अत एव विस्मृतगीतार्थस्यार्जुनस्य प्रार्थनया भगवान् भगवद्गीतार्थं सङूगृह्णाति । ज्ञानस्याविनाशित्वं, ज्ञानादेव मुक्तिरिति च प्रतिपादयन्तीयं गीता मुद्रिता ।। (S.B.E.8) 
1. गौडपादमुनिकृतम् - अनुगीताभाष्यम् । अमुद्रितोऽयं ग्रन्थः नासिकहस्तलिखितपुस्तकालयसूच्यां (XXVI 35) दृश्यते । गौडपादमुनिश्च माण्डूक्यप्रकरणे उपपदितः । 
अवधूतगीता - 
दत्तगीता, जीवन्मुक्तगीता इत्यादिनाम्ना प्रसिद्धेयं गीता 'आष्टेकर कम्पेनि ' पूनानगरे मुद्रिता । गोरक्षदत्तात्रेयसंवादरूपेऽस्मिन् ग्रन्थे जीवन्मुक्तस्य स्वरूपं, मोक्षस्वरूपं आत्मज्ञानविचारश्च सम्यगुपपादिताः । अस्य कर्ता यदि महाभारतादिषु प्रसिद्धः दत्तात्रेयस्स्यात्तर्हि अत्रेरनसूयायाश्च पुत्र, निमेः पिता, श्रीमतः पितामहः इति महाभारतानुशासनपर्व - सभापर्वप्रमाणात् निर्णेतुं शक्यते । 
1.  परमानन्दतीर्थकृता - अवधूतगीताटीका ।
अमुद्रितोऽयं ग्रन्थ मद्रासराजकीयहस्तलिखितपुस्तकालये (D. 17476 MGOML) तिरुवनन्तपुरपुस्तकालये महीशूरपुस्तकालये च लभ्यते । अस्याः कर्ता परमानन्दतीर्थः विद्यातीर्थप्रशिष्यः भारतीतीर्थशिष्यः चतुर्दशशतकीय (1300 - 1400 A.D.) इति निश्चीयते । 
2. शङ्कराचार्यकृताः - अवधूतगीतासारः । अमुद्रितोऽयं ग्रन्थः (D. 18884 MGOML) लभ्यते । 
3. अज्ञातकर्तृका - अवधूतगीताव्याख्या । अमुद्रितोऽयं ग्रन्थः (R. 5498 MGOML) लभ्यते ।। 
अष्टावक्रगीता - 
अष्टावक्रीयं, अष्टावक्रसूक्तं, अवधूतानुभूतिरित्यादिनामभिः प्रसिद्धेयं गीता एकविशतिभिरध्यायैः पूर्णा अष्टावक्रजनकसंवादरूपेण अद्वैतवेदान्तसिद्धान्तान् ब्रह्मणोऽद्वितीयत्वं चिन्मयत्वञ्च प्रतिपादयति । मुद्रिता चेयं गीता आष्टेकर कम्पेनिपूनानगरे ।
1. पूर्णानन्दकृता - अष्टावक्रगीताव्याख्या । 
ग्रन्थोऽयं मध्यप्रान्तीय वरार्ग्रन्थसूच्यां (327 CCPB) दृश्यते । यद्ययं पूर्णानन्दः ब्रह्मविद्याभरणकाराणां अद्वैतानन्देन्द्राणां शिष्यः, रामानन्दस्य प्रशिष्यः, अच्युतकृष्णानन्दसतीर्थ्यः, भाष्यरत्नप्रभाव्याख्यापूर्णानन्दीयकर्ता स्यात् तर्हि अस्य कालः (1650 - 1750 A.D) सप्तदशशतकीय इति निर्णेतुं शक्यते ।। 
2. भासुरानन्दकृता - अष्टावक्रगीताव्याख्या ।
ग्रन्थोऽयं मध्यप्रान्तीयबरार्ग्रन्थसूच्यां दृश्यते । भासुरानन्दोऽयं काशीवासीसन्नपि चोलदेशानागतः कावेरीतीरवासी सञ्जातः । अस्य शिष्य उमानन्दाख्यः । उमानन्देन नित्योत्सव इति ग्रन्थ कृतः यश्च मद्रासनगरे (R. 2462 MGOML) लभ्यते । भासुरानन्देन रत्नावलीनामा ग्रन्थेऽपि कृत इति ज्ञायते ।। 
3. मुकुन्दमुनिकृता - अष्टावक्रगीताव्याख्या । 
ग्रन्थोऽयमपि मध्यप्रान्तीयवरार्ग्रन्थसूच्यां दृश्यते । अस्य कर्ता मुकुन्दमुनिः हरिनाथप्रशिष्यः रामनाथशिष्यः ब्रह्मामृतवर्षिणीकर्तू रामकिङ्करधर्मस्य गुरुः, महाराष्टदेशीयः, मुकुन्दानन्दमुक्तिनाथमुकुन्दराजापराभिधानस्सुज्ञानविंशतिकाराद्भिन्नः सप्तदशशतकीय (1550 - 1650 A.D.) इति निश्चीयते ।। 
4. विश्वेश्वरपण्डितकृता - अष्टावक्रदीपिका ।
अवधूतानुभूतिदीपिका, अध्यात्मदीपिका, अष्टावक्रदीपिका इत्यपरनामायं ग्रन्थ आष्टेकरकम्पेनि पूनानगरे मुद्रितः । अस्य कर्ता विश्वेश्वरपण्डितः माध्वप्रज्ञशिष्य इति परं ज्ञायते ।। 
ईश्वरगीता - 
कूर्मपुराणान्तर्गतेयं ईश्वरगीता । द्वादशभिरध्यायैः पूर्णेयं गीता ऋषिव्याससंवादमुखेन प्रवृत्ता बिब्लियोथिकासंस्कृतग्रन्थमालायां मुद्रिता । 
1. यज्ञेश्वरसूरिकृता - ईश्वरगीताव्याख्या ।
ईश्वरगींताभाष्यमित्यपरनामायं ग्रन्थ अमुद्रितस्प्तरस्वतीमहालये (8997 DC. TSML) लभ्यते । अस्य कर्ता काश्यपगोत्रजः बहवृचशाखाध्यायी चर्कूरिवंशजः कोण्डुभट्ट - गङ्गाम्बिकयोः पुत्रः, तिरुमलैदीक्षितकनीयान् भ्राता यज्ञेश्वरकृष्णाश्रमयोः शिष्यः चोलदेशीयः, शाहेन्द्रग्रामवासी नृसिम्हाश्रमिण अर्वाचीनस्सप्तदशशतकीयः (1600 - 1700 A.D) शास्त्रदीपिकाव्याख्याप्रभामण्डलकर्ता यज्ञेश्वरदीक्षित इति पञ्चपादिकाविवरणोज्जीविनी (R. 592 MGOML) ईश्वरगीताव्याख्यानाच्च ज्ञायते ।। 
उत्तरगीता - 
कृष्णार्जुनसंवादात्मकोऽयं ग्रन्थः प्रणवोपासनादीनद्वैतवेदान्तप्रकरणग्रन्थप्रतिपादितान् विषयान् त्रिभिरध्यायैः प्रतिपादयन् महाभारतान्तर्गतः वाणीविलासमुद्रणालये कृलकत्तायां बाम्बे नगर्याञ्च मुद्रितः । 
1. काण्डद्वयातीतयोगिकृता -उत्तरगीताव्याख्या 
अमुद्रितोऽयं ग्रन्थस्सरस्वतीमहालयपुस्तकालये (7568 DC TSML) लभ्यते । अस्य कर्ता हारीतगोत्रजः अतिवर्णाश्रमी सन्यासात् पूर्वं काण्डयार्यनीलाम्बयोः पुत्रः सामराजतीर्थशिष्यः काण्डद्वयातीतयोगीति ज्ञायते । यद्ययं सामराजः रतिकल्लोलिनीश्रीदामचरितशृङ्गारामृतलहर्यादिकर्ता नरहरिदीक्षितपुत्रः आनन्दरायकालिकः मधुरावासी सामराजस्स्यात्तर्हि काण्डद्वयातीतयोगिनः कालः 1775 सं. 1719 A. D.  इति निर्णेतुं शक्यते । 
2. गौडपादाचार्यकृता उत्तरगीतादीपिका 
मुद्रितोऽयं ग्रन्थः वाणीविलासमुद्रणालये । अस्य कर्ता गौडपादः किं माण्डूक्यकारिकाकर्ता ? उतान्य इत्यत्र न निश्चयः ।। 
3. परमानन्दतीर्थकृता - उत्तरगीताटीका 
ग्रन्थोऽयं महीशूरपुस्तकसूच्यां दृश्यते । परमानन्दतीर्थोऽयं विद्यातीर्थप्रशिष्यः भारतीतीर्थशिष्यश्चतुर्दशशतकीय (1300 - 1400 A.D.) इति ज्ञायते ।। 
4. अज्ञातकर्तृकः - उत्तरगीतासारः - 
गीतासार इत्यपरनामायं ग्रन्थः महीशूरसूच्यां (Vol. II Page 22) दृश्यते । आनन्दगिरिणा कृता च काचन व्याख्या मुद्रिता विद्यते ।। 
ऋभुगीता - 
शिवरहस्यान्तर्गतेयं गीता सप्तर्विशतिभिरध्यायैः पूर्णा सरस्वतीमहालये अडयारपुस्तकालये महीशूरपुस्तकसूच्याञ्च दृश्यते ।।
कपिलगीता - 
देवहूतिभगवत्संवादरूपेयं गीता आष्टेकरकम्पनीमुद्रितगीतासंग्रहान्तर्गता । पद्मपुराणान्तर्गता अन्या काचन शिवपार्वतीसंवादात्मिका कपिलगीतापि विद्यते । 
गणेशगीता - 
गजाननवरेण्यसंवादमुखेन प्रवृत्तेयं गीता भगवद्गीतां सर्वात्मनानुकरोति । मुद्रिता चेयमुपनिषदानन्दाश्रममुद्रणालये (ASS 52)
1.  नीलकण्ठतीर्थकृता - गणपतिभावदीपिका 
गणेशगीताव्याख्यात्मकोऽयं ग्रन्थस्संक्षेपशारीरकं वार्तिकञ्चोद्धरन् आनन्दाश्रमे (ASS 52) मुद्रितः । अस्य कर्ता चतुर्धरवेशावतंसः काशीवासी गोविन्दसूरिपुत्रः नीलकण्ठतीर्थः षोडशशतकान्तिमात्  समयादारब्धे सप्तदशशतकान्ते काले 1750 सं 1696 A.D. आसीदिति गगनशरगिरीन्दुमिते विक्रमसंवत्सरे ग्रन्थञ्चकारेति च ज्ञायते ।। 
जीवन्मुक्तिगीता - 
उपेन्द्रनाथमुखोपाध्यायेन कल्कत्तायां मुद्रितेयं गीता कृष्णार्जुसंवादमुखेन जीवन्मुक्तस्वरूपं वर्णयति । 
ज्ञानगीता -
अमुद्रितोऽयं ग्रन्थ नासिकसूच्यां (XXIV 22) दृश्यते । अद्वैतसिद्धान्तप्रतिपादिकेयं गीता इति (J.O.R.Vol. 12 Page 114) जर्नल आफल ओरियण्टल पत्रिकाया द्वादशतमात् भागात् ज्ञायते ।।
देवीगीता - 
योगत्रयप्रतिपादिकेयं गीता देवीभागवतसप्तमस्कन्धान्तर्गता । भगवती गीता इत्यपि नामान्तरमस्या श्रूयते । हिमवद्देवीसंवादमुखेन प्रवृत्तेयं गीता आष्टेकरकम्पनि मुद्रितगीतासंग्रहे दृश्यते । 
बोधानन्दगीता 
ईशादिबृहदारण्यकान्तानां दशोपनिषदां भाष्यानुसारिणमर्थं सङ्गगृह्णन्तीयं गीता द्वादशमिः परिच्छेदैः पूर्णाऽमुद्रिता च तिरुवनन्तपुरं पुस्तकालये (316 DCTCL) लभ्यते । 
अस्य कर्ता ब्रह्मानन्दयतिशिष्यः गुरुमूर्तिगुरुः पञ्चनदक्षेत्रवासी (1700 - 1800 A.D.) अष्टादशशतकीयः कैवल्यदीपिकाकारः (R. 2934 MGOML) बोधानन्द इति ज्ञायते ।।
ब्रह्मगीता I 
योगवासिष्ठोत्तरार्घगतनिर्वाणप्रकरणान्तर्गता वसिष्ठरामसंवादमुखेन प्रवृत्तेयं गीता आष्टेकरकम्पनिगीतासंग्रहे मुद्रिता ।।
ब्रह्मगीता II
स्कान्दपुराणान्तर्गतसूतसंहितान्तर्गतेयं गीता गीतासंग्रहे मुद्रिता । शिवभक्ति अद्वैतसिद्धान्तप्रदर्शनप्रधानेयं गीता दशोपनिषदामर्थञ्च प्रतिपादयति । 
1. कृष्णानन्दकृता - चित्प्रकाशिनी 
ब्रह्मगीतव्याख्यापरनामायं ग्रन्थः नासिकसूच्यां 136 दृश्यते । अस्य कर्ता कैवल्यानन्दकृष्णानन्दयोश्शिष्यः वाराणसीवासी एकोनविंशतिशतकीयः (1800 - 1900 A.D.) कृष्णानन्दसरस्वतीति ज्ञायते ।।
2. माधवाचार्यकृता - तात्पर्यदीपिका 
ब्रह्मगीताव्याख्यात्मकोऽयं र्ग्रन्थः मद्रासनगरे बालमनोरमामुद्रणालये मुद्रितः । अस्य कर्ता माधवाचार्यः माधवमन्त्रीत्यपरनामा आङ्गिरसगोत्रजः. चावुण्ड्यात् माचाम्बिकायामुत्पन्नः काशीविलासक्रियाशक्तिशिष्यः परमशिवभक्तोऽपि अद्वैत वेदान्ततत्ववेत्ता प्रचण्डयोद्धा शत्रुमानमर्दनकारी शौर्यसम्पन्नः, उपनिषन्मार्गप्रवर्तकाचार्यविरुदभूषितः, चतुर्दशशतकीयः (1340 - 1391 A.D.) इति ज्ञायते । माधवमन्त्री अयं भारद्वाजगोत्रजात् , मायणश्रीमत्योः पुत्रात् , सायणभोगनाथयोः भ्रातुः, विद्यातीर्थभारतीतीर्थश्रीकण्ठशिष्यात् , पराशरमाधवपञ्चदश्यादिकर्तुः विद्यारण्यापराभिधानात् माधवाचार्यात् भिन्न इति यफिग्राफिका कर्नाटिका (Vol. 7 ) शिकारपुर 281,  एवं - यफिग्राफिकाकर्नाटिका (Vol. 8) प्रमाणाच्च, ज्ञायते । 
3. वेङ्कटेशशास्त्रिकृता - तात्पर्यबोधिनी ।
अमुद्रितोऽयं ब्रह्मगीताव्याख्याग्रन्थः मद्रासराजकीयपुस्तकालये (R. 4065 MGOML) लभ्यते । अस्य कर्ता आत्रेयगोत्रजः कोयम्पुरि (कोयम्पुत्तूर ?) वासी सर्वदेवशिष्यः अय्यात्तुरेशास्त्रीत्यपरनामा सामिशास्त्रिसामयिकः एकोनविं शतिशतकीयः (1790 - 1850 A.D.) वेङ्कटेश्वरशास्त्रीति ज्ञायते ।। 
भगवद्गीता - 
अष्टादशभिरध्यायैः सप्तशतश्लोकैश्च पूर्णेयं गीता निःश्रेयसप्राप्त्युपायान् स्वभावमधुरैः पदविन्यासैः प्रतिपायति । 
"श्लोकैकं धृराष्टस्य नव दुर्योधनस्तथा । 
दात्रिंशत्सञ्जयप्रोक्तं वेदाष्टौ अर्जुनस्य च ।।
तत्वाववोधो वेदाव्धिः पञ्च केशवनिर्मितम् ।
एवं गीताप्रमाणञ्च श्लोकास्सप्तशतानि च ।। "
इत्ययं श्लोकः गीताविषये प्रसिद्धः ।। भगवद्गीता विभिन्नदार्शनिकवादेषु निर्लिप्ता आध्यात्मिकतत्वनिरूपणपरवैदिक धर्ममात्रानुसारिणी राजते । गीतायां आत्मापरोक्षानुभूतिप्रतिपादिकानां उपनिषदां अद्वैतसिद्धान्तः, प्रकृतिपुरुषविवेकज्ञानान्मोक्ष इति प्रतिपादिका सांख्याप्रक्रिया, कर्मानुष्ठानजन्यस्वर्गादिफलदायिकर्ममीमांसासिद्धान्तः, अष्टाङ्गसाधनेन प्रकृतिबन्धनाश्च पाञ्चरात्रसिद्धान्तश्च यत्र तत्र प्रतिपादितः । अत एव गीताया आशयः कुत्रेति न निर्णेतुं शक्यत इति केचिद्वदन्ति । ततदाचार्यैः गीता प्रमाणीक्रियते । शङ्करभगवत्पादैश्च गीताप्रस्तावनाभाष्ये " तदिदं गीताशास्त्रं समस्तवेदार्थसारसंग्रहभूतं दुर्विज्ञेयार्थं " इत्युच्यते । शाङ्करभाष्यानुसारं उपक्रमोपसंहारप्रमाणात् भगवद्गीतायाः ज्ञानयोगे एव तात्पर्यं अन्ये तु तदङ्गा एवेति प्रतिपाद्यते । 
भगवद्गीतायां दृश्यमाणः " पत्रं पुष्पं फलं तोयमित्यादिः श्लोकः (9-26) बोधायनगृह्यसूत्रेषु दृश्यते । बोधायनगृह्यसूत्राणां रचनाकालस्तु (400 BC) इति प्रसिद्धम् । तस्मात् तत्पूर्वतनेयं गीतेति ज्ञायते । बेलबलकरमहाशयः, दासगुप्तमहाशयाश्च बौद्धधर्मग्रन्थात् प्राचीनेयं भगवद्गीतेति वदन्ति । महाभारतादपि प्राचीनेयं भगवद्गीता । भगवद्गीतायाः प्रशस्त्यं सार्वजनीनत्वञ्च दृष्ट्वा महाभारते संयोजितेति तत्वविदः । व्याख्योपेतेयं गीता निर्णयसागर मुद्रणालयेऽन्यत्र बहुषु स्थलेषु मुद्रिता च । अस्य भाष्यादिकं एतत्सम्बद्धाः ग्रन्थाश्च-"
1. शङ्कराचार्यकृतम् - भगवद्गीताभाष्यम् । मुद्रितश्चायं ग्रन्थ आनन्दाश्रमे, वाणीविलासमुद्रणालये, निर्णयसागरमुद्रणालये च ।
(A)अनुभूतिस्वरूपाचार्यकृतम् - गीताभाष्यटिप्पणम् ।
अमुद्रितोऽयं ग्रन्थ अडायारपुस्तकालये (33 AL), तिरुवनन्तपुरपुस्तकालये (231 DC. TCL ) च लभ्यते । अस्य कर्ताऽनुभूतिस्वरूपाचार्यः द्वादशशकीय इत्यादि उपनिषत्प्रस्थाने सूत्रप्रस्थाने चोपपादितम् । 
(B) आनन्दगिरिकृतम् - गीताभाष्यविवेचनम् । 
भगवद्गीताभाष्यव्याख्यात्मकोऽयं ग्रन्थ आनन्दाश्रममुद्रणालये (ASS 34) मुद्रितः । आनन्दगिरिरयं शुद्धानन्दानुभूतिस्वरूपाचार्यशिष्यः त्रयोदशशतकीय इति सूत्रप्रस्थाने उपनिषत्प्रस्थानेऽद्वैताचार्यप्रकरणे चोपपादितम् ।। 
(C) केशवसाक्षिभगवत्पादकृतः - गीताभाष्यसंक्षेपः ।
अमुद्रितोऽयं पूर्णग्रन्थः मद्रासराजकीयपुस्तकालये  (D. 2058 MGOML)  लभ्यते । केशवसाक्षिमगवतानेन विद्यारण्यः नृसिम्हभारती च ग्रन्थे निर्दिष्टौ । तस्मात्तयोरनन्तरवत्रलिक इति परं निर्णीयते ।। 
(D) रामानन्दसरस्वतीकृता - गीताभाष्यव्याख्या - गीताशयः 
शाङ्करभाष्यव्याख्यात्मकोऽयं ग्रन्थः दासगुप्तमहाशयेन (HIP. Vol. II Page 39 ) निर्दिष्टः । बरोडासूच्यां (6939 D. DC. BRD.) दृश्यते च । स्वयम्प्रकाशसरस्वती राममद्रानन्दसरस्वतीप्रशिष्यः, राघवानन्दशिष्यश्चायं रामानन्दस्स्वग्रन्थेषु लघुचन्द्रिकाकारं ब्रह्मानन्दसरस्वती निर्दिशन् सप्तदशशतकापरार्धादारब्धकालवासीति निश्चीयते । अनेन पञ्चदशीव्याख्या विशुद्धदृष्टिनाम्नी कृता च ।। 
(E) भागवतानन्दकृतम् - भाष्यविवेचनम् ।
ग्रन्थोऽयं दासगुप्तेन निर्दिष्टः। सूर्यपण्डितकृता व्याख्या च पूना नगरे मुद्रिता । व्याख्याया नाम परमार्थप्रमा इति दृश्यते । 
रामरायकवि (बेल्लङ्कोण्डा) रचिता अर्कप्रकाशिकानाम्नी व्याख्या अमुद्रिता विद्यते । 
2. उपनिषद्ब्रह्मेन्द्रकृता - भगवद्गीताव्याख्या ।
अस्याः नाम अडयार सूच्यां (34. 18. ग्र 550 AL) अर्थप्रकाशिका इति दृश्यते महीशूरसूच्यान्तु ब्रह्मतत्वप्रकाशिका "इति दृश्यते अमुद्रितेयं व्याख्या । " अस्याः कर्ता उपनिषद्ब्रह्मेन्द्रः, वासुदेवेन्द्रप्रशिष्यः, वासुदेवेन्द्रशिष्यः रामचन्द्रेन्द्रसतीर्थ्यः कृष्णानन्दगुरुः, अष्टादशशतकीयः उपनिषदां विवरणकार इति ज्ञायते ।।
3. कल्याणभट्टकृता - भगवद्गीताटीका । ग्रन्थोऽयममुद्रितः मध्यप्रान्तीय बरार्ग्रन्थसूच्यां (1383 CCPB) दृश्यते 
4. कृष्णकृता - भगवद्गीताभावप्रकाशिका ।
ग्रन्थोऽयं मध्यप्रन्तीयबरार्ग्रन्थसूच्यां (1390 CCPB) दृश्यते यद्यय कृष्णपण्डितः कैवल्यदीपिकाकारस्यात्तर्हि चोलदेशीयः कृष्णानन्दयतिशिष्य अष्टादशशतकीय इति निर्णेतुं शक्याते ।। 
5. कृष्णानन्दसरस्वतीकृतः - भगवद्गीतैकदेशपरामर्शः । 
गीतायाः भेदवादे तात्पर्यं निरस्य अद्वैतब्रह्मवादे तात्पर्यं वर्णयन्नयं ग्रन्थ "गवर्नमेण्ट प्रेस गोण्डारप्रटे नगरे" मुद्रितः ।। अस्य कर्ता सच्चिदानन्दाश्रमस्य वासुदेवेन्द्रयोगितश्च शिष्यः वाराणसीवासी एकोनविंशतिशतकीय  (1825 - 1900 A.D.) कृष्णानन्दसरस्वतीति ज्ञायते । अदसीया अन्ये ग्रन्था अन्यत्र प्रतिपादिताः । 
6. कृष्णानन्दसरस्वतीकृतः - गीतासारोद्धारः। अस्य कर्ता अद्वैतसाम्राज्यकारः कैवल्यानन्दकृष्णानन्दयोः शिष्य एकोनर्विशतिशतकीयः कृष्णानन्दसरस्वतीति ज्ञायते ।। 
7. धनपतिसूरिकृता - गीताभाष्योत्कर्षदीपिका ।
अस्य कर्ता धनपतिसूरिः पञ्जाबन्तर्गतरावलपिण्डीनगरवासी रामकुमारसूरिपुत्रः सारस्वतब्राह्मणः, प्रत्यक्तत्वचिन्तामणिकारस्य सदानन्दव्यासवरस्य जामाता बालगोपालतीर्थशिष्यः, अष्टादशशकापरार्धादारब्धे (1750 - 1850 A.D.) काले आसीदिति ज्ञायते । अदसीयाः वेदान्तपरिभाषाव्याख्या अर्थदीपिकाद्याः अन्यत्र प्रतिपादिताः ।। 
8. नीलकण्ठतीर्थकृतः - गीतार्थप्रकाशकः  
अमुद्रितोऽयं ग्रन्थः मद्रासराजकीयपुस्तकालये (D. 2081 MGOML) लभ्यते । अस्य कर्ता नीलकण्ठतीर्थः गोविन्दसूरिसूनुः बालतीर्थशिष्यः आत्मयोगिनीगुरुः केरलवासी अष्टादशशतकीय (1775 - 1875 A.D.) इति ज्ञायते ।। मुद्रितश्च निर्णयसागरमुद्रणालये । 
9. भारद्वाजकृता - भगवद्गीतासङ्गतिमाला
135 पद्यैः पूर्णोऽयं ग्रन्थः भगवद्गीतायाः ज्ञानयोगे एव तात्पर्यमिति वर्णयति । अज्ञातकर्तृनामधेया व्याख्याप्यस्ति। अमुद्रितोऽयं ग्रन्थ मद्रासराजकीयपुस्तकालये (R. 5336 MGOML) लभ्यते ।
(A) अज्ञातकर्तृनामधेया - भगवद्गीतासङ्गतिमालाव्याख्या 
अमुद्रितोऽयं ग्रन्थ मद्रासराजकीयपुस्तकालये ( R 533 E MGOML) लभ्यते । 
10. मधुसूदनसरस्वतीकृता - गूढार्थदीपिका 
मुद्रितोऽयं ग्रन्थः निर्णयसागरमुद्रणालये । अस्य कर्ता मधुसूदनसरस्वती कमलजनयनापराभिधः राममिश्रवंशोत्पन्नः पुरन्दराचार्यपुत्रः, श्रीनाथ-यादवानन्दवागीशगोस्वामिभ्राता, श्रीरामविश्वेश्वरमाधवशिष्यः, पुरुषोत्तम-बलभद्र-शेषगोविन्दगुरुः, हिन्दीभाषाकवितुलसीदास - भारतसम्राडग्बरसामयिकः षोडशसप्तदशशतकवासी (1565 -  1665 A D) इति विस्तरेणोपपादितं अद्वैताचार्यप्रकरणप्रस्तावे ।। 
11. धर्मदत्तबच्चाशर्मकृता - गूढार्थतत्वालोकः 
निर्णयसागरमुद्रणालये मुद्रितोऽयं ग्रन्थः । नव्यनैय्यायिकशैलीबद्धेयं व्याख्या । अस्याः कर्ता बच्चाशर्मा इति प्रसिद्धः धर्मदत्तः मैथिल एकोनर्विशतिशतकीय व्युत्पत्तिवादव्याख्याता च (1850 - 1920 A.D.) इति ज्ञायते ।। 
12. रघुनाथसूरिकृतम् - पदभूषणम् 
शङ्करपदभूषणमित्यपरनामायं भगवद्गीताव्याख्यात्मकः ग्रन्थः कुत्रात्य इति न ज्ञायते । परन्तु शङ्करशास्त्रिमारुलकारेण ब्रह्मसूत्रवृत्तिशङ्करपादभूषणभूमिकायां Page 101 निर्दिश्यते । ग्रन्थकारोऽयं रघुनाथसूरिः ब्रह्मसूत्रव्याख्याखङ्करपादभूषणकारः रामचन्द्रसूरिपुत्रः रामशास्त्रिपिता राघवाचार्यशिष्यः, एकोनर्विशतिशतकीय (1850 A.D.) इति ज्ञायते ।। 
13. राघवानन्दकृता - तत्वार्थचन्द्रिका । 
अमुद्रितोऽयं ग्रन्थ तिरुवनन्तपुरसूच्यां (282 DC. TCD) महीशूर पुस्तकालये (443) च दृश्यते । अस्य कर्ता राघवानन्दस्स्वयम्प्रकाशप्रशिष्यः, कृष्णानन्दरामभद्रानन्दशिष्यस्सप्तदशशतकीयः (1685 A.D.) परमार्थसारव्याख्याता चेतिज्ञायते ।। 
14. रामचन्द्रानन्दसरस्वतीकृता - तत्त्वदीपिका ।
पदयोजना इत्यपरनामायं ग्रन्थ अमुद्रितः मद्रासराजकीय पुस्तकालये (D. 2068, R. 1921 MGOML) बरोडा - तिरुपति - महीशूर-अडयारपुस्त -कालयेषु च लभ्यते । अस्य कर्ता रामचन्द्रानन्दः नारायणानन्दसरस्वतीप्रशिष्यः लघुचन्द्रिकाकर्तुः ब्रह्मानन्दसरस्वत्याः  शिष्यः, बालकृष्णानन्दसतीर्थ्यस्सप्तदशशतकापरार्धारब्धकालवासीति (1650 - 1920 A.D.) ज्ञायते ।। 
15. रामानन्दसरस्वतीकृतः - गीताशयः ।
भगवद्गीताभाष्यसंग्रहात्मकोऽयं ग्रन्थः व्याख्यारूपश्चामुद्रितः विद्यारण्यपुर सूच्यां (137) दृश्यते । एष एव ग्रन्थः भाष्यव्याख्यात्मक इति बरोडा सूच्यां निर्दिष्ट इति ज्ञायते । रामनन्दोऽयं स्वयम्प्रकाशकृष्णानन्दरामभद्रानन्दानां प्रशिष्यः, राघवानन्दशिष्यः गौडब्रह्मानन्दादर्वाचीन इति पूर्वमुपपादितम् ।
16. लिङ्गोजीपण्डितकृता - व्यासभावप्रकाशिका । भगवद्गीताव्याख्यात्मकोऽयं ग्रन्थ मद्रासराजकीयहस्तलिखितपुस्तकालये (R 2294 MGOML) लभ्यते ।। 
17. वासुदेवशास्त्री अभ्यङ्करकृतः - अद्वैताङ्कुरः । भगवद्गीतायाः प्रथमद्वितीयाध्यायव्याख्यात्मकोऽयं ग्रन्थ आनन्दाश्रममुद्रणालये (ASS 109) मुद्रितः ।
अस्य कर्ता वासुदेवशास्त्री अभयङ्करः नागोजीभट्टप्रशिष्यस्य भास्कराचार्यस्य पौत्रः प्रशिष्यः शिष्यश्च, रामशास्त्रिशिष्यः पुण्यनगरवासी एकोनविंशतिशतकीयः (1850 - 1920 A.D.) अद्वैतामोदकर्ता चेति ज्ञायते ।। 
18. वासुदेवशिष्यकृताः - कुमारकारिकाः । भगवद्गीतार्थसंग्राहकोऽयं ग्रन्थ तिरुवनन्तपुरपुस्तका कलये (280 DCTCD) लभ्यते । अस्य कर्ता पुरुषोत्तमवासु देवयोश्शिष्यः विवेकसारादिकर्ता इति परं ज्ञायते ।
19. वेङ्कटनाथकृता - ब्रह्मानन्दगिरिः । 
भगवद्गीताव्याख्यात्मकोऽयं ग्रन्थ द्वैतविशिष्टाद्वैतखण्डनपरः शाङ्कारभाष्यानुसारी वाणीविलासमुद्रणालये (VVSS. 12) मुद्रितः । ग्रन्थेऽस्मिन् " अद्वैतवज्रपञ्जर" तैत्तरीयोपनिषट्टीका च निर्दिष्टे । अस्य कर्ता वेङ्कटनाथः अभिनवशङ्कराचार्यापरनामकरामब्रह्मानन्दसरस्वतीशिष्य इति परं ज्ञायते । यद्ययं वेङ्कटनाथः धर्मंराजाध्वरिणा निर्दिष्टः वेङ्कटनाथस्स्यात् तर्हि वेलाङ्गुडिवासी धर्मराजाध्वरिगुरुष्षोडशशतकवासीति (1550 - 1650 A.D.) परं निर्णेंतु शक्यते ।। 
20. श्रीधरस्वामिकृता - सुबोधिनी ।
भगवद्गीताशाङ्करभाष्यानुसारिणीयं व्याख्या निर्णयसागरमुद्रणालये आनन्दाश्रममुद्रणालये च मुद्रिता । अस्य कर्ता श्रीधरस्वामी "परमानन्दपादाब्जरजः श्रीधारिणामुना" इति वदन् आत्मानं परमानन्दतीर्थशिष्यं निर्दिशति । यद्ययं परमानन्दः गुरुचन्द्रिकाकर्तुः गौडब्रह्मानन्दस्य गुरोः परमानन्दात् न भिन्नः, यदि च रत्नप्रभाकारस्य रामानन्दसरस्वत्यास्सतीर्थ्यात् स्वयम्प्रकाशगोविन्दानन्दयोश्शिष्यात् परमानन्दात् न भिन्नस्तर्हि अस्य कालः (1550 - 1650 A.D.) इति वक्तुं शक्यते । अस्य प्रतिलिपिकालः (1668 सं 1612 A.D) इति पञ्चावसूच्यां दृश्यते । एवञ्च सप्तदशशतकपूर्वार्धात् प्राचीन इति परं निर्णेतुं शक्यते ।। 
21. शङ्करानन्दकृता - तात्पर्यद्योतिनी । 
भगवद्गीताव्याख्यात्मकोऽयं ग्रन्थः निर्णयसागरमुद्रणालये मुद्रितः । अस्य कर्ता शङ्करानन्द आन्दात्मशिष्य आत्मपुराणादिविविधग्रन्थकर्ता विद्यारण्यगुरुस्त्रयोदशशतकापरार्धारब्धकालवासी (1275 - 1350 A.D.) इति निश्चीयते । अस्यैव विद्यातीर्थः, विद्याशङ्कर इति नामान्तरमिति, शङ्करानन्दस्य शिष्यः प्रथमसदानन्द इत्यादिविविधसिद्धान्तः प्रकरणग्रन्थप्रस्तावे प्रतिपादितः । 
22. सदानन्दव्यासवरकृतः - गीताभावप्रकाशः । 
भगवद्गीताया शाङ्करभाष्यानुसारिणीयं व्याख्या पद्यमयी चौखाम्बामुदणालये मुद्रिता । अस्याः कर्ता सदानन्दव्यासवरः प्रत्यक्तत्वचिन्तामणिकारः, सारस्वतब्राह्मणकुलोत्पन्नः, पञ्चाबदेशान्तर्त्मत रावलपिण्डी जिलान्तर्गतः, नानकसम्प्रदायानुगतवाबारामदयालु प्रेमपात्रमपि शाङ्करसिद्धान्तपक्षपाती शिवलाल प्रेम पात्रं, धनपति सूरिश्वशुरः वासुदेवयोगिनः प्राप्तदीक्षः, अष्टादशशतकीय (1740 -1810 A.D.) इति निश्चयः । अभिनवगुप्ताचार्यकृतः "गीतार्थसंग्रहः" श्च मुद्रितः । अभिनवगुप्तश्च काष्मीरी भट्टेन्दुराज शिष्यः। अस्य जीवनकालः (993 - 1015 A.D.)
23. हनूमत्कृतम् - पैशाचाभाष्यम् । ग्रन्थोऽयं चौखाम्बामुद्रणालये मुद्रितः । 
24. अज्ञातकृर्तृकः - गीतासारः । भगवद्गीतासारोऽयं ग्रन्थः विद्यारयप्पुरसूच्यां (118) दृश्यते । 
25. अष्टादशश्लोकी भगवद्गीता - ग्रन्थोऽयं सरस्वतीमहालये (8939 DC. TSML) दृश्यते । 

% भगवद्गीता 

रामगीता-
तैत्तरीयोपनिषदः, वाजसेनेयश्रुतिञ्च प्रमाणयन्तीयं रामगीता तत्वसारायणान्तर्गता आत्मविद्याग्रन्थमालायां (1) आष्टेकर कम्पनिमुद्रितगीतासंग्रहे च मुद्रिता । रामलक्ष्मणसंवादशैल्या प्रवृत्तोऽयं ग्रन्थः ज्ञानादेव मोक्ष अज्ञाननिवृत्तिश्चेति प्रतिपादयति ।
रुद्रगीता -
बृहद्ब्रह्मसंहितान्तर्गतेयं गीता रुद्रभद्रवाहुसंवादमुखेन प्रवृत्ता अद्वैतवेदान्तविषयिणी आनन्दाश्रममदितायां बृहद्ब्रह्मसंहितायां द्रष्टव्या ।
वासिष्ठगीता -
निर्णयसागरमुद्रणालयमुद्रितयोगवासिष्ठोत्तरार्घगतनिर्वाणप्रकरणान्तर्गता आत्मविश्रान्तिस्वभावविश्रान्तिविषणीयं वासिष्ठगीतेति ज्ञायते ।। 
शिवगीता - 
षोडशभिरध्यायैः पूर्णेयं गीता अगस्त्योपदेशमुखेन प्रवृत्ता आष्टेकरकम्पनीमुद्रणालये मुद्रिता ।। अस्या व्याख्याः -
1.  परमशिवेन्द्रसरस्वतीकृता - शिवगीतातात्पर्यप्रकाशिका । 
मुद्रितोऽयं ग्रन्थः वाणीविलासमुद्रणालये अस्य कर्ता परमशिवेन्द्रः ज्ञानेन्द्रसरस्वतीप्रशिष्यः अभिनवनारायणेन्द्रशिष्यः, सदाशिवब्रह्मेन्द्रगुरुः सप्तदशशतकीय इति (1600 - 1700 A.D.) इति ज्ञायते ।।
2. ब्रह्मानन्दसरस्वतीकृता शिवगीताव्याख्या ।
अमुद्रितोऽयं ग्रन्थः मद्रासराजकीयपुस्तकालये (R. 926 A. MGOML) लभ्यते । अस्याः कर्ता ब्रह्मानन्दस्समराज सरस्वत्याः सदाशिवतीर्थस्य च शिष्य इति परं ज्ञायते ।।
3. लक्ष्मीनरहरिसूनुकृता - बालानन्दिनी । 
शिवगीताव्याख्यात्मकोऽयममुद्रितग्रन्थः सरस्वतीमहालये (9022 DC.TSML) दृश्यते । अस्य कर्ता भट्टोजीदीक्षितादर्वाचीन इति परं ज्ञायते ।।
4. शङ्करानन्दकृता - शिवगीताव्याख्या ।
अमुद्रितोऽयं ग्रन्थः अडयारपुस्तकालये (9 B 42 क 3 AL) लभ्यते । अस्य कर्ता शङ्करानन्दः भगवद्गीताव्याख्यावसरे प्रतिपादितः ।। 
शिवरामगीता 
सुखोदयापरनामकाद्वैतसुधारसान्तर्गतेयं गीताऽमुद्रिता मद्रासराजकीयपुस्तकालये (R. 140 MGOML) अडयारपुस्तकालये (24 F आ 15 AL) महीशूरसूच्याञ्च दृश्यते शिवरामसंवादसरण्यां प्रवृत्तोऽयं ग्रन्थः योग - अद्वैतसिद्धान्तान् प्रतिपादयति ।। 
श्रुतिगीता - भागवतदशमस्कन्धान्तर्गतेयं गीता ब्रह्मस्तुतिरूपा ।
1. शङ्करानन्दकृता - श्रुतिगीताव्याख्या । अमुद्रितोऽयं ग्रन्थः अडयारपुस्तकालये (20 E. 25 ग्र 151 AL) लभ्यते । शङ्करानन्दः उपपादितपूर्वः । 
1. श्रीधरस्वामिकृता - श्रुतिगीताव्याख्या ।
2. अज्ञातकर्तृकः - भावबोधः । श्रुतिगीताव्याख्याभूतोऽयं ग्रन्थ अडयारपुस्तकालये (19 DI 100 AL), सरस्वतीमहालये च (8990 DC. TSML) लभ्यते ।। 
सिद्धगीता - 
योगवासिष्ठोपशमप्रकरणान्तर्गतेयं गीता निर्णयसागरमुद्रणालये मुद्रिता ।
सिद्धान्तगीता - 
अथर्वणवेदरहस्यान्तर्गतेयं गीताऽमुद्रिता अडयारपुस्तकालये (9 H.24, 13 AL), सरस्वतीमहालये (9025 DC. TSML) लभ्यते ।। 
सूतगीता - सूतसंहितान्तर्गतेयं गीता आनन्दाश्रममुद्रणालये (ASS 25) बालमनोरमामुद्रणालये च मुद्रिता । 
1. माधवमन्त्रिकृता - तात्पर्यदीपिका ।
मुद्रिताचेयं व्याख्या पूर्वोक्तस्थले । ग्रन्थकारोऽयं काशीविलासक्रियाशक्तिशिष्यः प्रसिद्धिविद्यारण्यापराभिधमाधवाचार्याद्भिन्न उपनिषन्मार्गप्रवर्तकाचार्यबिरुदभागिति ब्रह्मगीताव्याख्याप्रकरण उपपादितम् । 
सूर्यगीता -
वासिष्ठतत्वसारायणान्तर्गतेयं गीता सूर्यारुणसंवादमुखेन प्रवृत्ता अद्वैतवेदान्तसिद्धान्तप्रतिपादिका आष्टेकरकम्पनिमुद्रितगीतासंग्रहे प्रकाशिता ।
हंसगीता - 
कृष्णोद्धवसंवादमुखेन प्रवृत्तेयं गीता आष्टेकरगीतासंग्रहे प्रकाशिता ।
