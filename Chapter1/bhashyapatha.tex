\chapter{शाङ्करभाष्यपठनक्रमः ।}
श्रीशङ्करभगवत्पादकृतभाष्यपाठारम्भे सम्प्रदायसमागतः शान्तिमन्त्रपठनक्रमः 

शिष्यास्सर्वे वस्त्रावगुण्ठितशरीराः पठेयुः 

शिवनामनि भवितेऽन्तरङ्गे महति ज्योतिषि मानिनीमयार्धे । \\
दुरितान्यपयान्ति दूग्दूरे महुरायान्ति महान्ति मङ्गलानि ।।

स्मृते सकलकल्याणभाजनं यत्र जायते । \\
पुरुषस्तमजं नित्यं व्रजामि शरणं हरिम् ।। 

१. ओं शं नो मित्रः शं वरुणः। शं नो भवत्वर्यमा। शं न इन्द्रो बृहस्पतिः। शं नो विष्णुरुरुक्रमः । नमो ब्रह्मणे । नमस्ते वायो । त्वमेव प्रत्यक्षं ब्रह्मासि त्वमेव प्रत्यक्षं ब्रह्म वदिष्यामि । ऋतं वदिष्यामि । सत्यं वदिष्यामि । तन्मामवतु । तद्वक्तारमवतु । अवतु माम् । अवतु वक्तारम् । ओं शान्तिः शान्तिः शान्तिः । 

२. सह नाववतु । सह नौ भुनक्तु । सह वीर्यं करवावहै। तेजस्वि नावधीतमस्तु मा विद्विषावहे । ओं शान्तिः शान्तिः शान्तिः ।

३. यददृन्दसामृषभो विश्वरूपः । छन्दोभ्योऽध्यमृतात् संबभूव। स मेन्द्रो मेधया स्पृणोतु। अमृतस्य देव धारणो भूयासम् । शरीरं मे विचर्षणम् । जिह्ना मे मधुमत्तग। कर्णाभ्यां भूरि विश्रुवम् । ब्रह्मणः कोशोऽसि मेधयापिहितः । श्रुतं मे गोपाय । ओं शान्तिः शान्तिः शान्तिः ।

४. अहं वृक्षस्य रेरिवा । कीर्तिः पृष्ठं गिरेरिव । ऊर्ध्वपवित्रो वाजिनीव स्वमृतमस्मि। द्रविणँ सवर्चसम् । सुमेधा अमृतोक्षितः । इति त्रिशङ्कोर्वेदानुवचनम् । ओं शान्तिः शान्तिः शान्तिः ।

५. पूर्णमदः पूर्णमिदं पूर्णात् पूर्णमुदच्यते ।
पूर्णस्य पूर्णमादाय पूर्णमेवावशिष्यते ।।
ओं शान्तिः शान्तिः शान्तिः ।

६. आप्यायान्तु ममङ्गानि वाक्याणिचक्षुःश्रोत्रं अथो बलमिन्द्रियाणि सर्वाणि । सर्वं ब्रह्मोपनिषदम् । माहं ब्रह्म निराकुर्याम् । मा मा ब्रह्म निराकरोत् । अनिराकरणमस्त्वनिराकरणं मे अस्तु । तदात्मनि निरते य उपनिषत्सु धर्मास्ते मयि सन्तु ते मयि सन्तु । ओं शान्तिः शान्तिः शान्तिः ।

७. वाड्मे मनसि प्रतिष्ठिता । मनो  मे वाचि प्रतिष्ठितम् । आविरावीर्म एधि । वेदस्य म आणीस्थः । श्रुतं मे मा प्रहासीः । अनेनाधीतेन । अहोरात्रान् सन्दधामि । ऋतं वदिष्मामि । सत्यं वदिप्यामि । तन्मामवतु तद्वक्तारमवतु । अवतु मां अवतु वक्तारमवतु वक्तारम् । ओं शान्तिः शान्तिः शान्तिः ।

८. भद्रं नो अपि वातय मनः । ओं शान्तिः शान्तिः शान्तिः ।

९. भद्रं कर्णेभिःशृणुयाम देवाः । भद्रं पश्येमाक्षभिर्यजत्राः स्थिरैरङ्गै स्तुष्टुवांसस्तनूभिः । व्यशेम देवहितं यदायुः । स्वस्ति न इन्द्रो वृद्धश्रवाः । स्वस्ति नः पूषा विश्ववेदाः । स्वस्ति नस्ताक्ष्यों अरिष्टनेमिः । स्वस्ति नो बृहस्पतिर्दधातु । ओं शान्तिः शान्तिः शान्तिः ।

१०. यो ब्रह्मणं विदधाति पूर्वं यो वै वेदांश्च प्रहिणोति तस्मै । तं ह देवं आत्मबुद्धिप्रकाशं मुमुक्षुर्वै शरणमहं प्रपद्ये । ओं शान्तिः शान्तिः शान्तिः । 

ओं नमो ब्रह्मादिभ्यो ब्रह्मविद्यासंप्रदायकर्तृभ्यो वंशऋषिभ्यो नमो गुरुभ्यः। सर्वोपल्पवरहितः प्रज्ञानघनः प्रत्यगथों ब्रह्मैवाहमस्मि । अधीहि भो भगवः, अधीहि भो भगवः । 

ततः किञ्चिद् भाष्यं पठेयुः ।
अनन्तरं दक्षिणामूर्त्यष्टक देहं प्राणमपीत्यन्तं पठन्तः प्रतिश्लोकं नमस्कुर्युः 

११. मौनव्याख्याप्रकटितपरब्रह्मतत्वं युवानम् \\
वषिंष्ठान्तेवसदृषिगणैरावृतं ब्रह्मनिष्ठैः ।\\
आचार्येन्द्रङ्करकलितचिन्मुद्रमानन्दमूर्तिम् \\
स्वात्मारामं मुदितवदनं दक्षिणामूर्तिमीडे ।। 

१२. विश्वं दर्पणदृश्यमाननगरीतुल्यं निजान्तर्गतम् \\
पश्यन्नात्म निमायया बहिरिवोद्भूतं यथा निद्रया ।\\
यस्साक्षात्कुरुते प्रबोधसमये स्वात्मानमेवाद्वयं \\
तस्मै श्रीगुरुमूर्तये नम इदं श्रीदक्षिणामूर्तये ।। 

१३. बीजस्यान्तरिवाङ्कुरो जगदिदं प्राङ्निर्विकल्पं पुनः \\
मायाकल्पितदेशकालकलनावैचित्र्यचित्रीकृतम् ।\\
मायावीव विजम्भयत्यपि महायोगीव यः स्वेच्छया \\
तस्मै श्रीगुरुमूर्तये नम इदं श्रीदक्षिणामूर्तये ।। 

१४. यस्यैव स्फुरणं सदात्मकमसत्कल्पार्थकं भासते \\
साक्षात्तत्वमसीति वेदवचसा यो बोधयत्याश्रितान् ।\\
यत्साक्षात्करणाद् भवेन्न पुनरावृत्तिर्भवाम्भोनिधौ \\
तस्मै श्रीगुरुमूर्तये नम इदं श्रीदक्षिणामूर्तये ।। 

१५. नानाच्छिद्रघटोदरस्थितमहादीपप्रभाभास्वरम् 
ज्ञानं यस्य तु चक्षुरादिकरणद्वारा बहिः स्पन्दते ।
जानामीति यमेव भान्तमनुभात्येतत्समस्तं जगत् 
तस्मै श्रीगुरुमूर्तये नम इदं श्रीदक्षिणामूर्तये ।। 
१६. देहं प्राणमपीन्द्रियाण्यपि चलां बुद्धिञ्च शून्यं विदुः
स्त्रीबालान्धजडोपमास्त्वहमिति भ्रान्ता भृशं वादिनः ।
मायाशक्तिविलासकल्पितमहाव्यमोहसंहारिणे 
तस्मै श्रीगुरुमूर्तये नम इदं श्रीदक्षिणामृर्तये ।। 
तत इमान् श्लोकान् पठन्तः नमस्कुर्युः - 
१७. श्रुतिस्मृतिपुराणानामालयं करुणालयम् । 
नमामि भगवत्पादं शङ्करं लोकशङ्करम् ।। 
१८. शङ्करं शङ्कराचार्यं केशवं वादरायणम् ।
सूत्रभाष्यकृतौ वन्दे भगवन्तौ पुनः पुनः ।। 
१९. नमः श्रुतिशिरःपद्मषण्डमार्तण्डमूर्तये ।
वादरायणसंज्ञाय मुनये शमवेश्मने ।।
२०. ब्रह्मसूत्रकृते तस्मै वेदव्यासाय वेधसे 
ज्ञानशक्त्यवताराय नमो भगवतो हरेः ।। 
२१. नारायणं पद्मभुवं वसिष्ठं शक्तिञ्च तत्पुत्रपराशरञ्च 
व्यासं शुकं गौडपदं महान्तं गोविन्दयोगीन्द्रमथास्य शिष्यम् । 
श्रीशङ्कराचार्यमथास्य पद्मपादञ्च हस्तामलकञ्च शिष्यम् 
तं तोटकं वार्तिककारमन्यान् अस्मद्गुरून् सन्ततमानतोऽस्मि ।।
२२. शङ्कराश्लेषविलसदानन्दामृतनिर्भराम् ।
विश्वोत्तंसितपादाब्जां ब्रह्मविद्यां विभावये ।।
२३. वेदान्तनिकुरुम्बेण तात्पर्येण प्रकाशितः ।
स्वात्मानन्दैकरस्येन कल्याणाय शिवोऽस्तु नः ।। 
२४. सदाशिवसमारम्भां शङ्कराचार्यमध्यमाम् ।
अस्मदाचार्यपर्यन्तां वन्दे गुरुपरम्पराम् ।। 
ततः भाष्यश्रवणं कर्तव्यम् ।
B
श्रीशङ्करभगवत्पादकृतभाष्यपाठान्ते सम्प्रदायसमागतः शान्तिमन्त्रपठनक्रमः । 

१. शं नो मित्रः शं वरुणः । शं नो भवत्वर्यमा । शं न इन्द्रो बृहस्पतिः । शं नो विष्णुरुरुक्रमः । नमो ब्रह्मणे । नमस्ते गयो । त्वमेव प्रत्यक्षं ब्रह्मसि । त्वामेव प्रत्यक्षं ब्रह्मावादिषम् । ऋतमवादिषम् । सत्यमवादिषम् । तन्मामावीत् । तद्वक्तारमावीत् । आवीन् माम् । आवीद् वक्तारम् । ओं शान्तिः शान्तिः शान्तिः । 

२. ओं सह नाववतु । सह नौ भुनक्तु । सह वीर्यं करवावहै । तेजस्वि नावधीतमस्तु । मा विद्विषावहै । ओं शान्तिः शान्तिः शान्तिः । 

३. ओं यश्चन्दसामृषभो विश्वरूपः । छन्दोभ्योऽध्यमृतात्संबभूव । स मेन्द्रो मेधया स्पृणोतु। अमृतस्य देव धारणो भूयासम् । शरीरं मे विचर्षणम् । जिह्वा मे मधुमत्तमा । कर्णाभ्यां भूरि विश्रुवम् । ब्रह्मणः कोशोऽसि मेधया पिहितः । श्रुतं मे गोपाय । ओं शान्तिः शान्तिः शान्तिः।
 
४. ओं अहं वृक्षस्य रेरिवा । कीर्तिः पृष्ठं गिरेरिव । ऊर्ध्वपवित्रो वाजिनीव स्वमृतमस्मि । द्रविणँ सवर्चसम् । सुमेधा अमृतोक्षितः । इति त्रिशङ्कोर्वेदानुवचनम् । ओं शान्तिः शान्तिः शान्तिः । 

५. ओं पूर्णमदः पूर्णमिदं पूर्णात्पूर्णमुदच्यते । पूर्णस्य पूर्णमादाय पूर्णमेवावशिष्यते । ओं शान्तिः शान्तिः शान्तिः । 

६. ओं आप्यायन्तु ममाङ्गानि वाक् प्राणश्चक्षुःश्रोत्रमथो बलमिन्द्रियाणि च सर्वाणि । सर्वं ब्रह्मौपनिषदम् । माहं ब्रह्म निराकुर्याम् । मा मा ब्रह्म निराकरोदनिराकरणमस्त्वनिराकरणं मे अस्तु । तदात्मनि निरते य उपनिषत्सु धर्मास्ते मयि सन्तु ते मयि सन्तु । ओं शान्तिः शान्तिः शान्तिः । 

७. ओं वाड् मे मनसि प्रतिष्ठिता । मनो मे वाचि प्रतिष्ठितम् । आविरावीर्म एधि । वेदस्य म आणीस्थः । श्रुतं मे मा प्रहासीः । अनेनाधीतेन । अहोरात्रान् सन्दधामि । ऋतं वदिष्यामि । सत्यं वदिष्यामि । तन्मामवतु । तद्वक्तारमवतु । अवतु माम् । अवतु वक्तारम् । अवतु वक्तारम् । ओं शान्तिः शान्तिः शान्तिः । 

८. ओं भद्रं नोऽपि वातय मनः । ओं शान्तिः शान्तिः शान्तिः । 

९. ओं भद्रं कर्णेभिः श्रृणुयाम देवाः । भद्रं पश्येमाक्षभिर्यजत्राः । स्थिरैरङ्गैस्तुष्टुवाँसस्तनूभिः । व्यशेण देवहितं यदायुः । स्वस्ति न इन्द्रो वृद्धश्रवाः । स्वस्ति नः पूषा विश्ववेदाः । स्वस्ति नस्ताक्ष्यों अरिष्टनेमिः । स्वस्ति नो बृहस्पतिर्दधातु । ओं शान्तिः शान्तिः शान्तिः ।

१०. ओं यो ब्रह्माणं विदधाति पूर्वं यो वै वेदांश्च प्रहिणोति तस्मै । तं ह देवमात्मबुद्धिप्रकाशं मुमुक्षुर्वैशरणमहं प्रपद्ये। ओं शान्तिः शान्तिः शान्तिः। 

इतः परं दक्षिणामूर्त्यष्टके अवशिष्टाः श्लोकाः पठनीयाः, प्रणामश्च कर्तव्यः । 

११. राहुग्रस्तदिवाकरेन्दुसदृशो मायासमाच्छादनात् 
सन्मात्रः करणोपसंहरणतो योऽभूत्सुषुप्तः पुमान् ।
प्रागस्वाप्समिति प्रबोधसमये यः प्रत्यभिज्ञायते 
तस्मै श्रीगुरुमूर्तये नम इदं श्रीदक्षिणामूर्तये ।।

१२. बाल्यादिष्वपि जाग्रदादिषु तथा सर्वास्ववस्थास्वपि 
व्यावृत्तास्वनुवर्तमानमहमित्यन्तः स्फुरन्तं सदा ।
स्वात्मानं प्रकटीकरोति भजतां यो मुद्रया भद्रया 
तस्मै श्रीगुरुमूर्तये नम इदं श्रीदक्षिणामूर्तये ।। 

१३. विश्वं पश्यति कार्यकारणतया स्वस्वामिसम्बन्धतः 
शिष्याचार्यतया तथैव पितृपुत्राद्यात्मना भेदतः ।
स्वप्ने जाग्रति वा य एष पुरुषो मायापरिभ्रामितः 
तस्मै श्रीगुरुमूर्तये नम इदं श्रीदक्षिणामूर्तये ।।  

१४. भूरम्भांस्यनलोऽनिलोऽम्बरमहर्नाथो हिमांशुः पुमान् 
इत्याभाति चराचरात्मकमिदं यस्यैव मूर्त्यष्टकम् ।
नान्यत्किञ्चन विद्यते विमृशतां यस्मात्परस्प्राद्विभोः 
तस्मै श्रीगुरुमूर्तये नम इदं श्रीदक्षिणामूर्तये ।। 

१५. सर्वात्मत्वमिति स्फुटीकृतमिदं यस्मादमुष्मिंस्तवे 
तेनास्य श्रवणात्तदर्थमननाद् ध्यानाच्च सङ्कीर्तनात् ।
सर्वात्मत्वमहाविभूतिसहितं स्यादीश्वरत्वं स्वतः 
सिध्येत् तत्पुनरष्टधा परिणतञ्चैश्वर्यमव्याहतम् ।। 

१६. वटविटपिसमीपे भूमिभागे निषण्णम् 
सकलमुनिजनानां ज्ञानदातारमारात् ।
त्रिभुवनगुरुमीशं दक्षिणामूर्तिदेवम् 
जननमरणदुःखच्छेददक्षं नमामि ।। 

१७. चित्रं वटतरोर्मूले वृद्धाश्शिष्या गुरुर्युवा । 
गुरोस्तु मौनं व्याख्यानं शिष्यास्तु छिन्नसंशयाः ।।

१८. अड्गुष्ठतर्जनीयोगमुद्राव्याजेन देहिनाम् ।
श्रुत्यर्थं ब्रह्मजीवैक्यं दर्शयन्नोऽवताच्छिवः ।। 

१९. ओं नमः प्रणवार्थाय शुद्वज्ञानैकमूर्तये ।
निर्मलाय प्रशान्ताय दक्षिणामूर्तये नमः ।।

२०. गुरवे सर्वलोकानां भिषजे भवरोगिणाम् । 
निधये सर्वविद्यानां दक्षिणामूर्तये नमः ।। 
२१. चिद्धनाय महेशाय वटमूलनिवासिने । 
सच्चिदानन्दरूपाय दक्षिणामूर्तये नमः ।। 

C

महाप्रदोषदिने परं यथापूर्वं भाष्यपाठश्रवणं कृत्वा दशशान्तिमन्त्रपाठानन्तरं (B/1-10) श्रीशङ्करभगवत्पादपूजां कृत्वा श्वेतसर्षपेण मधुना च मिश्रं दधि दूर्वातृणानि च निवेदनं कृत्वा एते मङ्गलपाठश्लोकाः वारत्रयं पठनीयाः - 

१. अशुमानि निराचष्टे तनोति शुभसन्ततिम् ।
स्मृतिमात्रेण यत्पुंसां ब्रह्म तन्मङ्गलं परम् ।। (त्रिः)

२. अतिकल्याणरूपत्वात् नित्यकल्याणसंश्रयात् ।
स्मर्तॄणां वरदत्वाच्च ब्रह्म तन्मङ्गलं परम् ।। (त्रिः)

३. ओंकारश्चाथशब्दश्च द्वावेतौ ब्रह्मणः पुरा।
कण्ठं भित्वा विनिर्यातौ तस्मान्माङ्गलिकावुभौ ।। (त्रिः)

ओं अथ ओं अथ ओं अथ ।
अतः परं "राहुग्रस्त" इत्यारब्धाः "दक्षिणामूर्तये नम"
इत्यन्ताः (११ - २१) श्लोकाः सप्रणामं पठनीयाः । 

इति शान्तिपाठसम्प्रदायः । 
