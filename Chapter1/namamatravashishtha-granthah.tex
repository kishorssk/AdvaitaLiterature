\chapter{।। विभिन्नग्रन्थेषु प्रमाणत्वेनोपन्यस्ताः नाममात्रावशिष्टाश्च ग्रन्थाः ।। }
		ग्रन्थाः 										कर्तारः
अद्वयसिद्धिः										श्रीधराचार्यकृता
न्यायकन्दलीकारेण श्रीधराचार्येण (10th cent. A.D.) कृतोऽयं ग्रन्थः न्यायकन्दल्यामेव निर्दिष्टः ।
1. अद्वैतकौस्तुभम् 								रत्नखेटश्रीनिवासदीक्षितः
	ग्रन्थोऽयं राजचूडामणिदीक्षितकृतस्य रुक्मिणीकल्याणस्य बालयज्ञवेदीश्वरकृतायां व्याख्यायां निर्दिष्टः। 
2. अद्वैतचन्द्रिका									नरसिम्हभट्टः
	नृसिम्हाश्रमिकृतस्य भेदाधिक्कारस्य व्याख्यात्मकोऽयं ग्रन्थः दासगुप्तमहाशयेन स्वीयभारतीयदर्शनसाहित्येतिहासे द्वितीये भागे (H.I.P. Val II) निर्दिष्टः । 
3. अद्वैतचिन्तामणिः								कुमारभवस्वामी
	ग्रन्थोऽयं रुक्मिणीकल्याणव्याख्यायां बालयज्ञवेदेश्वरकृतायां अम्बास्तवव्याख्यायाञ्च निर्दिष्टः ।
4. अद्वैतप्रकाशकः								वासुदेवज्ञानमुनिः
	ग्रन्थोऽयं अमुद्रिते मद्रासराजकीयहस्तलिखितपुस्तकालयस्थे कैवल्यरत्नाख्ये (R. 3620 MGOML) ग्रन्थे ग्रन्थकृतैव निर्दिष्टः । 
5. अद्वैतपद्यभाष्यम् 								रामसुब्रह्मण्यशास्त्री
	ग्रन्थोऽयं चिदम्बरनगरे	ग्रन्थलिप्यां मुद्रितस्य न्यायेन्दुशेखरयोगधटनग्रन्थस्य भूमिकायां निर्दिष्टः ।। 
6. अद्वैतवज्रपञ्जरः								वेङ्कटनाथः
	ग्रन्थोऽयं भगवद्गीताव्याख्यायां ब्रह्मानन्दगिर्याख्यायां वाणीविलासमुद्रितायां निर्दिष्टः ।
7. अद्वैतविद्याविजयः								कृष्णशास्त्री
	ग्रन्थोऽयं कृष्णशास्त्रिपुत्रशिवरामकृतायां अमुद्रितायां भावाज्ञानप्रकाशिकायां (54. A. 16. A.L) निर्दिष्टः । 
8. अद्वैतविवेकः									अज्ञातकर्तृकः
	ग्रन्थस्यास्य व्याख्या परं वेङ्कटेश्वरहस्तलिखितपुस्तकालये अमुद्रिता उपलभ्यतचे । व्याख्यायाः कर्ता रामकृष्णः । नायं ग्रन्थः पञ्चदश्यन्तर्गतः। तस्मात् विद्यारण्यो नास्य कर्ता भवितुमर्हति इति रामकृष्णीयायाः व्याख्याया अवगम्यते ।
9. अद्वैतशतकम् 									गङ्गाधरः
	ग्रन्थोऽयं यपिग्राफिका इण्डिका पत्रिकायाः द्वितीये भागे Page 333 निर्दिष्टः।
10. अद्वैतसिद्धान्तमण्डनम् 						रामसुब्रह्मण्यशास्त्री
	ग्रन्थोऽयं चिदम्बरक्षेत्रमुद्रिते "न्यायरक्षामणिभाष्योक्तिविरोध" ग्रन्थे निर्दिष्टः । 
11. अद्वैताधिकरणस्रग्धरा							रामसुब्रह्मण्यशास्त्री
	ग्रन्थोऽयमपि पूर्ववदेव । 
12. अद्वैतामृतकन्दम् 								नारायणसरस्वती
	ग्रन्थोऽयं गद्यमयशारीरकमीमांसाभाष्यवार्तिके कल्कत्तासंस्कृत ग्रन्थमालामुद्रिते C.S. S. I Page 19 निर्दिष्टः । 
13. आनन्ददीपिका								शिवानन्दयति
	एतच्छिष्येण रामनाथविदुषा अस्याः "विशुद्धदृष्टि" नाम्नी काचन व्याख्या रचिता या चामुद्रिताऽपूर्णा च मद्रासराजकीय स्तलिखितपुस्त कालये D 564 MGOML लभ्यते । तस्मादेव मूलग्रन्थसत्ता ज्ञायते ।
14. उपनिषत्संक्षेपवार्तिकम्						भारतीतीर्थः
	ग्रन्थोऽयं वाक्यसुधाटीकायां वाराणसी मुद्रितायां निर्दिष्टः ।
15. ज्ञानसिद्धिः 									ज्ञानोत्तमः
	ग्रन्थोऽयं चित्सुखाचार्यकृततत्वप्रदीपिकायाः व्याख्यायां मानसनयनप्रसादिन्यां (Page 392 N.S.Edn.) निर्दिष्टः । 
16. ज्ञानसुधा										ज्ञानोत्तमः 
	ग्रन्थोऽयं भारतीयैतिहासिकत्रैमासिकपत्रिकायाः चतुर्दशतमे भागे H.I Q. Vol XIV श्रीकण्ठशास्त्रिकृते "अद्वैताचार्य" नामके प्रबन्धे निर्दिष्टः। 
17. तैत्तरीयकटीका 								वेङ्कटनाथः
	ग्रन्थोऽयं भगवद्गीताव्याख्यायां ब्रह्मानन्दगिर्याख्यायां (Page 461 VVP. Edn.) निर्दिष्टः ।
18. दुःखद्रुमकुठारः 								अम्बिकादत्तगौडः
	वाराणसीमुद्रिते मोहनलालकृते वेदान्तसिद्धान्तादर्शे निर्दिष्टः ।
19. नैष्कर्म्यसिद्धिसारार्थः						रामतीर्थः
	ग्रन्थोऽयं दासगुप्तमहाशयेन भारतीयदर्शनसाहित्यचरिते द्वितीयभागे (H.I.P Vol II Page 99) निर्दिष्टः ।
20. नैष्कर्म्यसिद्धिव्याख्या "सारथिः"			रामदत्तः। 
	ग्रन्थोऽयं बन्दरकारप्राच्यभाषासंशोधनालयमुद्रितायां नैष्कर्म्यसिद्धौ (B.O. R. I. Edn.) निर्दिष्टः । 
21. न्यायसुधा									ज्ञानोत्तमः
	ग्रन्थोऽयं चित्सुखाचार्यकृतायां तत्वप्रदीपिकायां 392 पुटे, अप्पय्यदीक्षितैस्सिद्धान्तलेशसङ्ग्रहे 269 - 270 पुटे च निर्दिष्टः। 
22. पदभूषणम् (गीताव्याख्या)					रघुनाथसूरिः
	ग्रन्थोऽयं आनन्दाश्रममुद्रणालयमुद्रिते शङ्करपादभूषणाख्ये ब्रह्मसूत्रव्याख्याने भूमिकायां निर्दिष्टः । 
23. पञ्चपादिकाव्याख्या 							श्रीकृष्णः
	ग्रन्थोऽयं दासगुप्तमहाशयेन भारतीयदर्शनसाहित्येतिहासे द्वितीयभागे (H.I.P. Vol II Page 103) निर्दिष्टः । 
24. पञ्चपादिकाविवरणव्याख्या					रामतीर्थः
	अयमपि ग्रन्थः	दासगुप्तेन भारतीयदर्शनसाहित्येतिहासे (H.I.P. Vol II Page 52) निर्दिष्टः । 
25. प्रमाणवृत्तिनिर्णयः 							विमुक्तात्मा
	ग्रन्थोऽयं इष्टसिद्धौ 37 तमे पुटे (G.O.S. Edn) निर्दिष्टः । 
26 ब्रह्मतत्वसमीक्षा								वाचस्पतिमिश्रः
	ग्रन्थोऽयं न्यायवार्तिकतात्पर्यटीकायां तृतीयेऽधिकरणे द्वितीयाह्विके, भामत्यां "विस्तरतस्तु ब्रह्मतत्वसमीक्षायाम् " इति (Page 64) वाणीविलासमुद्रितायाम् , योगभाष्यव्याख्यातत्ववैशारद्याञ्च निर्दिष्टः। 
27. ब्रह्मप्रकाशिका
	ग्रन्थोऽयं अज्ञातकर्तृनामा प्रकटार्थविवरणे जिज्ञासाधिकरणे (Page 10) मद्रपुरीविश्वविद्यालयसंस्कृतग्रन्थावलीप्रकाशिते (MUSS 9) निर्दिष्टः। 
28. ब्रह्मविद्यातरङ्गिणी							नारायणयोगी
	ग्रन्थोऽयं कुत्रत्य इति न ज्ञायते । परन्तु अस्याः व्याख्या राजुशास्त्रीत्यपरनामकत्यागराजशास्त्रिभिः कृता विद्यते । 
29. ब्रह्मसूत्रसारार्थः								गौडपादाचार्यः 
	ग्रन्थस्यास्य प्राप्तिस्थानं न ज्ञायते । परं श्लोकसहस्रात्मक इति विन्ध्येश्वरी प्रसादद्विवेदी स्वसम्पादितवैशेषिकदर्शनभूमिकायां निर्दिशति । 
30. माण्डूक्य भाष्यार्थसङ्ग्रहः 					राघवानन्दः
	ग्रन्थोऽयं दासगुप्तमहाशयेन (H.I.P. II 78) निर्दिष्टः । 
31. माण्डूक्य भाष्यव्याख्या						मधुरानाथशुक्लः
	ग्रन्थोऽयं दासगुप्तमहाशयेन (H.I.P Vol II 78) निर्दिष्टः 
32. माध्वभ्रान्ति निरासः							सूर्यनारायणशुक्लः
	ग्रन्थोऽयं नारायणाचार्यद्वारा प्रकाशितस्य द्वैतसिद्धान्तमण्डनपरस्य अद्वैत विमर्शाख्यग्रन्थस्य खण्डनपरः अद्वैतमतसाधकः माध्वमुखभङ्गे निर्दिष्टश्च । 
33. वेदान्तपरिभाषा								काशीनाथशास्त्री
	ग्रन्थोऽयं दासगुप्तमहाशयेन (H.I.P. Vol II 54) निर्दिष्टः ।
34. वेदान्तभूषणम् 								परमेष्ठिगुरुः
	ग्रन्थोऽयं अमुद्रिते	अडयारपुस्तकालयस्थे "मध्वमतध्वान्तदिवाकराख्ये" (I.A. 18 A. L. Mss.) निर्दिष्टः। अस्य ग्रन्थस्य कर्तू रामाश्रम इति नामान्तरमिति अस्य प्रशिष्येण माधवाश्रमेण कृतात् स्वानुभवादर्शाज् ज्ञायते । 
35. वेदान्तभूषमम् 								कैवल्येन्द्रः
	ग्रन्थोऽयं कैवल्येन्द्रशिष्येण विद्येन्द्रसरस्वत्या कृते वेदान्ततत्वसाराख्ये अमुद्रिते सरस्वतीमहालयस्ये (7575 DC. TSML) ग्रन्थे निर्दिष्टः ।
36. वेदान्तसाख्याख्या							रामकृष्णदीक्षितः
	ग्रन्थोऽयं दासगुप्तमहाशयेन (H.I.P Vol. II 54) निर्दिष्टः ।
37. सिद्धान्तलेशसङ्‌ग्रहव्याख्या					मधुसूदनसरस्वती
	ग्रन्थोऽयं हरिलीलाविवेकभूमिकायां निर्दिष्टः। सरस्वतीमहालयस्थहस्तलिखितग्रन्थानां वर्णनात्मकसूच्याः त्रयोदशतमे भागे (7535 DC TSML Vol XIII) P.P. शास्त्रिणश्च निर्दिशन्ति । 
38. सुरेश्वरवार्तिकव्याख्या						ज्ञानोत्तममिश्रः
	ग्रन्थोऽयं दासगुप्तप्तहाशयेन (H.I.P. Vol II) निर्दिष्टः । अस्य कालः दशमशतकमिति च निर्दिष्टम् । 
39. सर्वसिद्धान्तरहस्यटीका						बलभट्रः
	शाङ्करसर्वदर्शनसिद्धान्तसङ्ग्रहव्याख्यात्मकोऽयं ग्रन्थः दासगुप्तमहाशयेन (H.I.P. Vol II 55) निर्दिष्टः । 
40. स्वरूपप्रकाशः								सदानन्दकाश्मीरी
	ग्रन्थोऽयं सदानन्दकाष्मीरककृतायां अद्वैतब्रह्मसिद्धौ निर्दिष्टः।
41. स्वानुभूतिप्रकाशः							देवेन्द्रसरस्वती 
	ग्रन्थोऽयं दासगुप्तमहाशयेन (H.I.P Vol. II 55) निर्दिष्टः। नृसिम्हाश्रमसामयिकोऽयमिति च निर्दिश्यते ।। 
