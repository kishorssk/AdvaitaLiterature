\documentclass[twoside,openany]{book}
\usepackage[none]{hyphenat}
\usepackage{polyglossia}
\usepackage{fontspec,graphicx}
\usepackage{setspace}
\usepackage{xstring}
\usepackage[ruled]{manyfoot}
\usepackage{fancyhdr}
\usepackage{bigfoot}

\usepackage[papersize={140mm,215mm},textwidth=10.2cm,
textheight=16cm,headheight=6mm,headsep=4mm,topmargin=2.25mm,botmargin=1.65cm,
leftmargin=19mm,rightmargin=19mm,cropmarks,footskip=0.6cm]{zwpagelayout}

\newcommand{\arabictodevnag}[1]%
{%
  \StrSubstitute{#1}{0}{०}[\num]
  \StrSubstitute{\num}{1}{१}[\num]
  \StrSubstitute{\num}{2}{२}[\num]
  \StrSubstitute{\num}{3}{३}[\num]
  \StrSubstitute{\num}{4}{४}[\num]
  \StrSubstitute{\num}{5}{५}[\num]
  \StrSubstitute{\num}{6}{६}[\num]
  \StrSubstitute{\num}{7}{७}[\num]
  \StrSubstitute{\num}{8}{८}[\num]
  \StrSubstitute{\num}{9}{९}
}



\setmainfont[Script=Devanagari,Ligatures=TeX,AutoFakeBold=3.5]{Sanskrit 2003}
\setdefaultlanguage[numerals=Devanagari]{Devanagari}

\newcommand{\devanagarinumeral}[1]{%
  \devanagaridigits{\number\csname c@#1\endcsname}}

\pagestyle{fancy}
\renewcommand{\headrulewidth}{0pt}

\lhead[\arabictodevnag{\thepage}]{}
\rhead[]{\arabictodevnag{\thepage}}
\chead[]{}

\lfoot[]{}
\rfoot[]{}
\cfoot[]{}



\begin{document}
\tableofcontents
\chapter{Introduction}
The growth of Literature on Advaita Vedanta can be treated in relation to the following periods - Ancient, Mediaeval, the late mediaeval and the present one. Of these the former extends upto the times of Sri Sankara and his contemporaries. In the works of these preceptors the basis for the doctrinal differences within the fold of Advaita such as the Vivarana view and the Bhamati view is found. During this period many independent works on Advaita were composed. The mediaeval period represents the growth of independent works chiefly concerned with a critical analysis of the Nyaya - Vaisesika school. 

According to Advaita, Brahman is the sole reality. Everything, else is false. The entire world appears in Brahman like silver in a piece of shell. Brahman is attributeless and so it can not be primarily signified. The Upanisad-s, therefore, state that Brahman transcends both speech and mind. Attributes like omniscience, etc., too are adventitious in Brahman. Jiva in its essential nature is identical with Brahman. Yet, owing to avidya it appears as if different from Brahman. Everyone has the cognition 'I' and it is due to the limiting adjunct mind. In the state of deep sleep, however, mind provisionally merges in avidya and hence there is no cognition of 'I' in that state. In the same manner, in the state of liberation too there is the absence of the cognition 'I' in view of the absence of mind then. Thus the cognition 'I' is only adventitious. The 'I' is the empirical jiva and in its essential nature it is one and not many. the essential nature of jiva is pure consciousness; it is all - pervasive; it is designated by the term Atman. It is three - fold as Isvara, jiva, and saksi. Ajnana or nescience is the primal cause of the entire world. Atman associated with ajnana is termed Isvara. The latter is designated as Visnu Brahma and Siva owing to the preponderance of the strands of sattva rajas, and tamas respectively. Atman is of the nature of knowledge. it is not an agent. It becomes a knower owing to its association with mind. The world becomes an abject of knowledge only through avidya. Three levels of reality are admitted, absolute, empirical, and apparent. Of these, the apparently real object like shell - silver is removed by the knowledge of the empirically real object - the shell. Pot, etc., are empirically real. And, these are annihilated by the immediate knowledgae of Brahman or atman which is absolutely real. For Brahman there is no sublation at any one of the three divisions of time - past, present, and future. Hence it is absolutely real. Pot, cloth, etc., are real only provisionally. Really they are non-real like the objects of the dream state. The objects of the dream state do not exist in the state of waking. Hence they are non-real. In the same way, pot, etc., do not exist from the stand - point of absolute. Hence athough they appear to exist, they are not real. They are annihilated by the direct knowledge of Brahman which is absolutely real.

The order of creation of the fundamental elements mentioned in one set of Upanisad-s is different from the order mentioned in another set of Upanisad-s. Some texts speak that creation proceeds in the order of space, etc., while certain other texts state that it proceeds in the order of fire, etc. Yet another set of Upanisadic texts states that everything is Brahman. All the texts are valid and hence the apparent contradiction among them must be resolved. It comes to this that Brahman alone is real and the world of objects being non-real, the Upanisadic texts do not have creation as their final import. Moreover, if the world were absolutely real, then the Upanisadic text which states that nothing exists apart from Brahman would become non-significant. The world is real from the stand-point of the ignorant. From the stand-point of the realized soul, it is non-real. The Upanisad not only speaks of the non-reality of the world; it speaks of its non-reality too. 

The Primal cause of the entire world is beginningless nescience. it consists of the three stands of sattva, rajas, and tamas. When the dissolution is about to end, the supreme self, that is, Brahman, associated with avidya and aided by merits and demerits of the individual souls contemplates thus 'Let me become many.' Then proceeds cration in the order of space, etc. Then from the quintuplicated elements spring forth physical body, etc. And, Atman which is pure consciousness gets reflected in the physical body and it is designated as jiva. The latter is not atomic in size; it is pervasive, as there is the experience of happiness and misery present throughout the body. The jiva-hood endures till liberation is attained. the jiva, like the one who has forgotten the golden ornament round one` s neck, experiences transmigration by having lost sight of its identity with Brahman. The three factors namely, nescience, subtle body, and gross body constitute the limiting adjunct; the latter is always proximate to jiva. The limiting adjunct is removed by the knowledge of the Atman, that is, the knowledge of the identity of the true nature of the individual soul and God. This is similar to the removal of ignorance in the case of one who has lost sight of his true nature of being a tenth man by the knowledge arising from the sentence 'you are the tenth man.' Never indeed is it noticed that erroneous knowledge gets itself removed without the true knowledge of the substratum. Thus when avidya is annihilated the so-called jiva becomes liberated. in his case the fire in the form of knowledge removes the accumulated and the future merits and demerits. The fructified merits and demerits alone operate at this stage. The jiva experience the fruits of fructified merits and demerits without any feeling of identity with the body, sense organs, and mind. He is termed jivanmukta. When the fructified merits and demerits are exhausted, the body falls off and he remains as Brahman; and, this is known as videhamukti. The latter alone is the superme form of liberation. Herein, the so-called jiva attains what is known as absolute identity with the supreme self. In other words, it remains non-different from Brahman. The jiva doed not remain as a servant of God as in some other systems. Although jiva is always identical with Brahman, yet owing to the influence of ajnana which is positive in nature, is considers itself to be an experient of happiness, and misery. It is designated as 'bound'. When ajnana is removed the jiva is designated as 'released'.

One has to perform one` s allotted duties as anoffering to God. There results purity of heart. Then one resorts to a preceptor who is devoted to Brahman, and who initiates him into the major-texts of the upanisad-s. He lives the life of an ascetic, gives up the performance of Karma, pursues vedantic study, reflection, and meditation, attains the knowledge of Brahman and is liberated. In order to attain liberation, knowledge is the sole means. Performance of Karma and meditative worship on the conditioned Brahman give rise to purity of heart and concentration of mind which are essential for the rise of the knowledge of self and hence they are the indirect menas to liberation. Combinations of jnana and Karma is not accepted as the means to liberation. Ascertainment of what is real and what is non-real, detachment towards enjoyment of objects here and in a hereafter, acquisition of the qualities of control of mind, control of external senses, etc., and intense desire for liberation-these constitute what is known as the four-fold aid. Then follows enquiry into the nature of Brahman.

On the epistemological side, advaita admits what is known as anirvacaniyakhyati to explain the nature of error. Perception, inference, comparison, verbal testimon, presumption, and non-apprehension are the six proofs. In regard to Brahman which is extra-empirical, verbal testimony is the primal source. Perception, etc. are pramana-s only in regard to empirically real objects. The well-known verse that summarizes the teachings of Advaita is : "Brahman is the reality; the world is non-real; and jiva is not different from Brahman".
The doctrines of Advaita have been virulently attacked by the Nyayavaisesika school which advocated the reality of difference and by the purva-mimamsa school represented by Salikanatha and others. Duing the 13th century Sri Madhvacarya the chief expounder of the Dualisitic school became one of the opponents of Advaita. In order to defend Advaita against the attacks of the above-mentioned schools, works like the Tattvapradipika, Nyaya-makaranda Nyaya-dipavali,. Prancadasi, Khandana khanda khadhya were written. Their chief aim has been to defend Advaita purely on rational grounds. The style of the text however differs. The Tattvapradipika. Nyaya-makaranda and other texts do not involve much of the technicalities of the navya-nyaya. But the text Advaita Siddhi and its commentary famously known as Brahmanandiya abound in logical subtleties to refute the doctrine of difference advaocated by the dualistic school. 

The advocates of the dualistic school raises the powerful objection that if the world were admitted to illusory then it would be an absolute nothing. Just as shell-silver is an absolute nothing, in the same way, the world too would be an absolute nothing. If this were the case, there would be the impossibility of the conduct of worldly activity and also the function of pramana-s. Moreover the knowledge of reality admitted to be derived from scripture would also be illusory. It is because the entire world of objects is admitted to be illusory by the Advaitins. The scripture too is included in the world of objects and hence it is also illusory. It comes to this, that the knowledge derived from the scripture too is illusory. 

The above objection has been carefully analysed and ultimately rejeceted by the Preceptors of Advaita. What exactly is meant by the term asat ? Is it an absolute nothing? Or, non-existence for a given period of time ? It cannot be the former because the world which is hiven in perception, which is adapted to practical needs of life cannot be an absolute nothing. That alone is an absolute nothing which does not come within the range of one` s perception. The world comes within the range of one` s perception and hence it cannot be an absolute nothing. 

Nor can the world be treated as real. Never indeed perceptuality or adaptability to practical needs of life could serve as the criterion of reality. In that case dream objects would become ral as they are perceived, Thus. the world is not an absolute nothing like the horn of a hare; nor is it real like the self. That alone is real which does not leave out its essential nature in the three divisions of time past, present and future. The world does not come within the sphere of the above definition and hence it is not real. 

Mithyatva or illusoriness is sadviviktatva - that which is distinct from reality; or, it is sadasadvilaksanatva - that is, that which is different from being both real and an absolute nothing. In the texts like the Nyayaratna -dipavali, Tattvapradipila, the Tarkasangraha of Anandagiri, the Pramanamala and other texts of Ananda Bodha, the definitions of mithyatva have been set forth, analysed and established on Sound basis. 

In the Brahamanandiya otherwise known as Laghucandrika five definitions of mithyatva are identified as one mentioned in the Pancapadika, the second and the third advocated in the vivarana the fourth ser forth in the Tettvapradipika and the fifth one referred to in the works of Ananda Bodha.

It has already been said that the philosophy of Advaita is based upon the three texts of the Upanisad-s, the Bhagavad - Gita, and the Brahma-Sutra. During the post-Sankara period doctrinal differences within the fold of Advaita developed to suit the needs of the aspirants of different levels of intellect. These doctrinal differences constitute glory to the system of Advaita. In the language of Suresvara any theory within the fold of Advaita is to be taken as valid if it facilitates the easy understanding of the concepts of Advaita ultimately leading to the realization of the non-dual self. Of these different schools within the fold of Advaita, the school of Bhamati and the school of Vivarana are important. 

The Bhamati school maintains that 1) Karma-s are useful for giving rise to the desire to know the self; 2) the realization of Brahman arises through the instrument of mind; 3) there is no injunction in the vedic text 'Atman should be realized'; for that purpose it should be heard, reflected and meditated upon; 4) meditation is the principle factor and vedantic study and reflection are its subsidiaru factors; 5) jiva is the consciousness that is limited by ajnana and Isvara is the consciousness that transcends the limiting adjunct; 6) the locus and content of ajnana is different; 7) the primal nescience is manifold; 8) it is only Brahman that is conditioned by the vrtti that is the content of the direct knowledge of Brahman; 9) the first factor in the fourfold aif in the discrimination between what is real and non-real; 10) the injuctiove text 'One` s own recension of veda must be studied' has for its fruit the knowledge of the meaing of the veda; 11) the world-creation is explained by adapting the theory of triplication; 12) the Omniscience of Brahman is derived from the essential nature of Brahman; 13) mind is a sense organ; and, 14) avidya is located in jiva. 

The Vivarana school on the other hand maintains that 1) Karma is responsible for the rise of the knowledge of Self; 2) the direct knowledge of Brahman arises from major texts of the Upanisad-s; 3) in the text Atman Should be realized, etc.,there is restrictive injuction; 4) vedantic study is the principal factor and reflection and meditation are the subsidiary features; 5) jiva is the reflected image of Brahman in avidya, mind etc.; 6) the locus and content of avidya is the same; 7) the primal nescience is one only; 8) the content of the direct knowledge of Brahman is pure Brahman; 9)the first factor in the fourfold aid is the discrimination between what is eternal and non-eternal; 10) the injunctive text 'One` s own recension of the veda must be studied' has for its fruit the learning of the veda by rote; 11) the world creation is explained on the basis of quintuplication theory; 12) the omniscience of Brahman is based on the modes of avidya;  13) mind is not a sense organ; and, 14) avidya is located in pure consciousness.

It is only pure consciousness which is Brahman when delimited by avidya, attains to the state of jiva. For each and every jiva, there is a distinct limiting adjunct-mind. Hence there is the usage that jiva is limited, ignorant, etc. This is the Bhamati view. According to the vivarana view it is only.

In regard to the nature of jiva, there are three theories known as avaccheda-vada, abhaasa-vada and pratibimba-vada. The first one is advocated by Vacaspati Misra, the second by Suresvera and the third by Sarvajnatman and others in his line of thinking.

Avaccheda means only immanence. That which is immanent is known as avacchinna or conditioned. just as the space immanent in water is stated to be conditioned by water, in the same way pure consciousness which serves as the locus of ajnana is jiva. If jiva is admitted to be consciousness associated with or conditioned by or reflected in avidya which is one, then jiva according to this view is one only. This theory is therefore known as eka-jivavada. The distinct experience of happiness, misery, etc., is due to the multiplicity of mind, the limiting adjunct. If jiva is admitted to be pure consciousness associated with mind, then multiplicity of mind answers for the plurality of souls. Suresvara, Padmapada, Prakasatman, Sarvajnatman and Anubhutisvarypa advocate the plurality of jiva-s. Ananda Bodha in his Nyaya-makaranda examines the view-points of his predecessors, approves specifically the view-point of the Brahma-siddhi and the Ista-siddhi, Shows that the eka-jivavada has the authority of scriptures and proves its soundness on logical grounds. 

In order to explain the rise of the world characterized by duality from Brahman which is non-dual and attributeless, the preceptors of advaita, following the teachings of the Upanisad-s, introduce the principle of maya. Sri Sankara in his commentary on the Brahma-sutra (i, iv, iii) uses the words maya, avidya, ajnana, avyakta, etc., as synonyms. Advaitins of the post-Sankara Isvara and jiva etc. Vacaspatimisra Advocates avacchedavada to explain the nature of jiva and Isvara and also maintains difference between the locus and content of avidya. Prakasatman, the author of the Vivarana advocates the pratibimbavada to explain the nature of jiva and isvara and maintains the identity of the locus and content of avidya. Other preceptors maintained the distinction between maya and avidya. The primal factor which is termed prakrti is termed maya when the pure sattva guna is predominant in it. It is termed avidya when the impure sattvaguna is predominant in it. And the consciousness conditioned by maya is termed Isvara. And that which is conditioned by avidya is jiva. 

There cannot be any activity in Brahman if it is free from relation to avidya. It is avidya or maya that is referred to by the term avyakta; it is the seed of the universe. It is located in Brahman. Just as the power to burn is not separable from fire in the same way the power of maya is not separable from Brahman. As Vidyaranya in his Pancadasi states : Just as the power to transform into pot exists only in the wet clay forming part of earth in the same way the power to transform into the world, that is, maya exist in a part as it were of Brahman. This maya consisting of the three strands of sattava, rajas and tamas is opposed to the knowledge of Brahman. It is positive in nature in the sense that it is not absence of knowledge. Since maya is sublated by the direct knowledge of Brahman it is not real. Since it is given in perception in the form 'I am ignorant' it is not an absolute nothing. An Absolute nothing will never be presented in any cognition. It cannot be real and an absolute nothing at once. For, it is a discrepant notion. Hence avidya or maya is said to be indeterminable either as real or absolute nothing. The very nature of avidya as Suresvara puts it, lies in its not coming within the range of any pramana.

This maya must be admitted to be the transformative material cause of the world. The world is indeterminable either as real or an absolute nothing. It is an effect ; it must, there-fore, have a cause. And, the cause must belong to the same order as that of the effect. The effect is indeterminable. Hence, the cause also must be indeterminable and it is maya or avidya. Sri Sankara in his Vivekacudamani states, 'this power belonging to the supreme self is termed avyakta. It is beginningless; it consists of the three strands of sattva, rajas, and tamas. It is to be inferred from the nature of the effect, namely, the world. 

This maya consists pf a two-fold power of veiling and revealing. By the former power, it  conceals the absolute aspect of Brahman ans by the latter it makes Brahman appera as God, soul and the world. In the work Drgdrsyaviveka attributed to Vidyaranya this distinction is clearly made out. Therein, it is stated that the power of revealing known as viksepa-sakti brings into existence the entire world. The power of veilling known as avarana sakti conceals the distinction between the self and mind etc., and Brahman and the world of objects. In the samksepa sariraka it is stated : avidya owing to the strength of having pure consciousness as its locus and content, comes to have a veiling  and transfiguring faculty. It veils the ever-luminous Brahman and then projects it illusorily in the form of embodied souls, God and the phenomenal world. 

Illusory projection presupposes concealment of the true nature of the substratum. Just as a juggler first conceals the right perception of the onlookers by charm etc., and the manifests illusory objects in the same way, maya too first conceals the true nature of Brahman and then projects it illusorily as God, souls and the world. It might be asked as to how maya which is limited could conceal Brahman which is all pervasive. This is answered by making a reference to the analogy of cloud which though limited conceals the orb of the sun which is more pervasive than that. In the pratyaktattvacintamani it is said, that just as blue colour is ascribed to the sky,snake to the rope, water to the barren land lit by the sun` s rays, in the same way the entire world is superimposed upon Brahman. 

According to Advaita, six factors are admitted to be beginningless. 1. the soul, 2. God, 3. pure consciousness, 4. the distinction between soul and God, 5. Avidya, and, 6. the association of avidya with pure consciousness. 

The concept of maya would be intelligible only if we refer to its locus and content. It is only the pure - conciousness that is the locus and content of avidya and in this aspect it is viewed as the mahat-tattva and from the latter there proceeds the world creation.

The Neo-Advaitins are of the view that the powers of ajnana are jnanasakti and kriyasakti. jnana-sakti is only sattva guna that is not overpowered by rajoguna and tamoguna. The kriyasakti is two-fold as avarana-sakti and viksepasakti. Of these, the former is the tamoguna not overpowered by rajoguna and sattavaguna. The latter is rajoguna not subdued by tamoguna and sattvaguna. The pure consciousness conditioned by ajana associated with the viksepasakti is the material cause of the world. According to this view ajnana associated with the avarana sakti is termed avidya and when it is associated with the viksepa sakti it is termed maya. This word is significant as maya is that by which the world is directly presented to us as real. It termed tamas,, avidya, avyakta, etc. It is termed maya because it is responsible for the manifestation of Brahman as something else. It is termed tamas because it conceals the true nature of Brahman. tamas because it conceals the true nature of Brahman. It is termed avidya because it is annihilated by the direct knowledge of Brahman. It is termed moha because it is the oause of delusion. It is termed asat as it is different from being real. Since it is not clearly manifested like pot, etc., it is termed avyakta. It is not independent like the supreme self. 

In respect of the creation of the world Brahman is the material and the efficient cause of the world only through maya or avidya. The sruti text 'The supreme self assumes mainfold forms through maya which possesses innumerable powers' confirms the above view. In the Bhagavad-Gita, the Lord states, 'Although I am the lord of all beings and not subject to birth yet depending upon maya located in me I appear to be born.' The proves that maya is related to Brahman. In the pancadasi Vidyaranya a follower of the vivarana school brings about a synthesis between some of the views of the vivarana and those of the Bhamati. He states that maya is the primal power in which the pure sattva guna is predominant by not being overpowered by rajo - guna and tamo -guna. Avidya is the primal power wherein the impure sattva -guna is predominant by being overpowered by the rajpguna and tamoguna. This view, vidyaranya sets forth in his vivaranaprameya-sangraha too.

According to the vivarana school, conciousness is threefold as Brahman, jiva and Isvara. Vidyaranya adds one more namely immutable reality, that is, kutastha - caitanya. 

In ragard to the nature of the witness-self some of the preceptors who preceded vidyaranya maintained the view that consciousness which transcend both jiva and Isvara, is the witness-self. vidyaranya however argues that it is the kutastha that is, the witness-self. Kutastha is different from jiva, the latter is being witnessed by the kutastha. vidyaranya explains this on the analogy of a lamp in a theatre. The lamp illumines three objects. the owner of the show, the members of the audience, and the danseuse, and the lamp continues to throw light even when the above three are absent after the show is over. In the same way, the kutastha which is consciousness conditioned by the gross and the subtle body and which is the substratum of jiva illumines three factors. 

1. The jiva who is comparable to the owner of the show.\\
2. Teh objects of the world which are comparable to the members of the audience, \\
3. The mind which is comparable to the danseuse. \\
The kutastha manifests the jiva and its activities, the objects of the world and the mind with all its variations in the state of waking and in that of dream. In the state of deep sleep when the mind provisionally merges in avidya, when jiva is not determinately perceived as 'I' and when the objects of the world are not manifested, the kutastha shines manifesting only avidya. Vidyaranya is of the view that the reflected image of the kutastha in mind is jiva; and, the kutastha which remains as prototype consciousness is the saksi and jiva and saksi are different. Vidyaranya` s contribution to Advaita lies in his synthesis of both the Vivarana and Bhramati views.

Till the end of the 16th century the advaitic preceptors were chiefly concerned with elucidating the view - points of the Vivarana and the Bhamati and also proving the theory of anirvcaniya-khyati and of the impartite sense. After this period preceptors of Advaita had for their main target of attack, the writers on Dvaita, who in their works such as Nyayamra, Nyayamrta-tarangini, Nyaya-Bhaskara and the like vehemently refuted the doctrines of Advaita. Advaitasiddhi of Madhusudana Sarasvati, the Guru-candrika and the Laghucandrika of Brahmananda and the work of Vitthalesa deserve special mention in the history of Advaita literature. Many a writer during the present century wrote refuting the charges levelled against Advaita by the Visistadvaitins in their texts like the Sri Bhasya and satadusani.

The doctrines of Advaita are confirmed not only by the three text of the Upanisads, the Bhagavad-Gita and the Brahmasutra but by texts like pancadasi, and Yoga-vasista and those which critically examine the objections raised by the other schools of thought against Advaita and thereby defend Advaita. 

\section {THE AIM OF THE PRESENT WORK}
In languages other than Sanskrit, such as English and HIndi and other modern languages, there are encyclopaedic works concerning almost every subject providing bibliographical information regarding the number of works available on a particular subject, its earliest beginnings, development, the preceptors responsible for its growth, etc. At a time when foreign countries have made tremendous progress in the fields of science and technology, our country is just opening her eyes like a child as it were. If our country is just opening her eyes like a child as it were. If our country has earned any name and fame among the comity of nations it is chiefly due to her contribution to philosophical thinking through the system of Advaita. The present attempt is to give an authentic and systematic history of the Advaita literature from the earliest beginnings. It is the normal practice among writers of literary history to write accounts of works in a language, in that particular language only. This is the reason why I too have taken keen interest in writing a history of Advaita works in Sanskrit language itself. I must at this point record my grateful thanks to the former Professor of Sanskrit, University of Madras, the late Dr. V Rahavan, Who enthused me first to undertake this work. 

\section{THE METHOD ADOPTED TO INTRODUCE THE SUBJECT MATTER IN THIS WORK}
A history of the Advaita literture should consist not only of the texts on the triple path, the so-called "Prasthanatraya" namely, the Upanisads, the Brahmasutra and the Bhagavadgita but also of all allied works. In fact, apart from the commentaries on the three Prasthanas, the Advaita literature has three more 'Prasthanas' as it were, namely, the ''Vada''-tracts or disquisitions, expositions on "anubhava" (the experience of the oneness of the self), and works couched in literary style. It will therefore be in the fitness of things to call this Advaita system, a fusion of "six" Prasthanas. For convenience, the present literary history is arranged in two parts. part I divided into six chapters in view of the vastness and importance of the Advaita texts that have appeared down the centuries. 

Chapter One is mainly concerned with the Upanisads. It takes into account only those Upanisads that are Advaitic in character, and present an account of the topics dealt with in them, the commentaries extant on them, the commentaries written by teachers of Advaita and their sub-commentaries. Discussion on the date of the Upanisads being outside the purview of the present analysis, the dates given by renowned scholars have been adopted here. 

Chapter Two deals with the Bhagavadgita and its allied literature. In this chapter, an account of the so-called "gita" s found in several puranas that are favourable to Advaita is given. A detailed account of the commentaries on these "gitas" and also on the well-known Bhagavadgita (which forms part of the Mahabharata) also finds a place in this chapter. 
Chapter Three is devoted to third Prasthana, namely, the Brahmasutra. This chapter gives accounts of the Brahmasutra, Sankara` s commentary thereon, texts following the Vivarana school and the Bhamati school, independent treatises, minor commentaries and the like.
Chapter Four deals mainly with independent treaties on Advaita, including their commentaries, which are about 800 in number. Some of the authors of these treatises are also found to have written on all the three Prasthanas. In some cases, a single author is seen to have composed several treatises. These treatises are many in number. So as to avoid repetition we have given in this chapter, only, the names of independent treatises and their commentaries in the order in which their basic texts appear. The topics dealt with in each of these works are to be found in the chapter dealing with writers of the Advaita texts. 

Chapter Five is concerned with works whose titles alone are known to us and which are known only through quotations by other writers. There are about 40 works belonging to these categories. This chapter also provides exact references to these quotations. 

Chapter Six is concerned with anonymous texts on Advaita Vedanta. There are about 500 such anonymous works both in manuscript and print. Of these again, some have the same title and are distributed over different libraries of the country. all the works mentioned in the Descriptive Catalogues of the Government Oriental Manuscripts Library, Madras. Adyar Library and Research centre, Madras, and the Sarasvati Mahal Library, Tanjore, are consulted for the present work. The identification of texts having the same name has been done in some cases; but wherever such identification proved impossible, we had to leave the matter as it is. Thus for instancem Anandabodha in his Nyayamakaranda (p. 170), while discussing the capacity of words in apprehending the relation with an already existing entity, states : "Details of this view may be understood from the Nyayadipika." From the context the Nyayadipila appears to be an independent Advaitic treatise. But there are many works bearing the title "Nyayadipika". in the Dvaita, Purvamimamsa, Jaina and Nyaya schools, which are not, of course, in favour of Advaita. There is a work called Nyayadipika in the manuscripts Libraries at Madras and Trivandrum; but it has turned out to be only a commentary on the sabdanirnaya, but not an independent work on Advaita school. Further. it is known as "Sabdadipika" also. Likewise, what could be done in the case of works deposited in the libraries at palces like Nasik, Ujjain and the Punjab, but which do not have any Descriptive Catalogue or index? Doubts in regard to such works do indeed persist. Hence a separate chapter had to be devoted for works whose authors are not known or identified. Thus in six chapters we have tried to cover works on Advaita which are basic and those which are in the form of commentaries. Already it has been pointed out that some texts are published and some, unpublished. We have also mentioned here the places where these unpublished works could be traced. Excepting for the well-known basic texts and their commentaries, all other works, printed or not, are briefly analysed in this Volume. 

As for the commentaries of the basic texts, whether they are published or not, they are mentioned in the section dealing with the authors, and also while discussing the basic texts. For the benefit of the readers and cursory viewers, an index of the basic texts and their commentaries is also included here. Thus in six chapters of Part I on the whole, 1,076 works of both known and unknown authorship, independent works, basic texts and their commentaries are mentioned. 

Of these, the number of printed texts is 402, Even among the printed texts, only those printed in Grantha and Devanagari characters are covered in the present survey. The total number of commentaries included in this book in 466. 

Part II concerns itself with authors of texts on Advaita. This part also it divided into five chapters. Chapter One deals with the pre-Sankara preceptors of Advaita, whose exact date, place and other biographical details could not be ascertained. Chapter Two attempts a chronological account of some well-known post-Sankara writers who can be identified as preceptors of Advaita. On the basis of available textual data, we have provided such details as the names of these writers, their titles reflection their station in life, the period in which they flourished, the names of their preceptors, grand-preceptors, classmates disciples and grand-disciples, places from which they hailed, names of their contemporaries, royal patrons and the like. This is followed by brief account of important anecdotes or events, of these lives of the preceptors, which are handed down from tradition, and are as such, popular in scholarly circles. There after are stated their settled opinions and specific concepts as given in the accounts of well-known critics of East and West. This is followed by works composed by these writers, along with a brief outline of the topics dealt with therein. We have also provided several pedigress (vamsa-vrksas) or  genealogical tables showing the line of the preceptors, their family descendants, the works handed down as basic texts and their commentaries. the works handed down as basic texts and their commentaries. As for writers who are known to be samnyasins, having the same name and same preceptors, they are identified as one and the same author, but not as several. But those who have identical names but different preceptors, grand-preceptors and so on, are treated as different but not identical. Such authors are indicated in the text by Roman figures (I,II, etc.,) after their names, to mark out their mutual difference.

In the present survey which is devoted to a literary history of Advaita Strictly speaking, only texts bearing on Advaita Should find a place. But, if the works of some non-Advaita writers are also found to yield valuable information in arranging our biography, such works are also consulted and mentioned here. But they are not included in the table of contents. 

In Chapter Three are given the life-accounts and works of those writers who, though well-known authorities in systems other than Advaita, have written a few works in Advaita also. This chapter also contains names of those writers whose dates and other details are either not known, or little knownm, and those who are not known as ascetics (smanyasins).

In Chapter Four are dealt with the works of the post-Sankara authors whose names alone are known to us, but not dates and other details. In Chapter Five is given the list of authors whose names are not known, but who are known through their own references as the disciples or preceptors of others. 

A list of abbreviations is given at the beginning of this volume, and at the end are appended general indexes of the names of works and authors. Letters in bold print used in these lists bring out the difference between texts and authors, as also the distinction among the work of the same title, and authors  of the same name. The general indexes furnish such details as: the other names by which an author is known, his work on other schools of thought, the village/town part of the country from which he hailed, his family, lineage, etc. list of technical termsof the Advaita system, list of meditational exercises (vidyas) list of different theories of knowledge, etc. There is a saying in Sanskrit : "the question of slipping does arise only when one starts walking". ['gacchatah khalu skhalanam']. Errors of commission and omission that have crept into the present volume due to inadvertance or in the process of printing have been rectified and indicated in the Errata. At the end is given a big stemma codicum [vamsa-vrksa] indicating the preceptor-disciple tradition obtaining in the Advaita system. It must be pointed out that only authors of texts on Advaita, their teachers and disciples alone are included in this pedigree, but not the names of those who are known through the records and chronicles of the Advaitic maths as authors of works which are passed on an oral tradition, but not actually recorded on paper or palm-leaf. Also this "tree" does not contain the names of those who worte on Advaita in languages other than sanskrit. 

Some problems encountered in writing this work:
Naturally in a work of this nature and magnitude, there are hound to occur several gaps and loopholes. I have tried, to the best of my ablility, to include in this volume, works both unpublished, and published, of known or unknown authorship, whether they are commentaries on other basic texts, or have formed the basic texts for other sub-commentaries, etc. But, as Nilakantha Diksita has rightly observed, "how many works have been composed so far in this world! How many poets were actually born! How many works are lost to posterity, how many are known and available to us. How many are badly damaged beyond recovery." I do not know as to how many works I could actually lay my hands on for the present work, and as to how many are actually lost or escaped my notice. This is one great problem I had to face. The next one is concerning the identity of several works and authors, in terms of the name and title. This problem is to some extent, settled with the help of the several Descriptive Catalogues available in the Libraries at Madras, Adyar, Tanjore and Trivandrum. But in so far as the works in libraries in distant places in other parts of the country are concerned, there being no Descriptive Catalogues or indexes, nothing could be done because of their inaccessibility, even through correspondence with those in charge of such libraries. So our doubts regrading the identity of a few works are still unresolved. This is the main reason in devoting a separate chapter for works of unknown authorship in the present volume. 

\section{Acknowledgements:} 
It is now my pleasant duty to express to all the members of the sanskrit Department, University of Madras for having laid at my disposal all the Descriptive Catalogues, Indexes and other refernce material needed for my work. I ma grateful to Prof. K. Kunjunni Raja who gave me the benefit of his advice and opinion. I am also thankful to the Curator of the Government Oriental Library, Madras, and the Director of the Adyar Library and Research centre, Madras, for permitting me to consult the unpublished manuscripts in their collections. I am also thankful to the authorities of the Mr. Kuppuswami Sastri Research Institute, Mylapore, Madras, the Sanskrit College, Madras, and the Sanskrit Departments of several local colleges, who graciously allowed me to use their libraries, even relaxing their rules, purely prompted by their love of Sanskrit. 

Although there are many works awaiting publication, the authorities of the University of Madras have magnanimously undertaken my work for printing and publishing under their aegis, out of their regard for the subject-matter and in view of the pains taken by me in preparing this compendium. I am particularly beholden to them for the kind help and encouragement extended to me in this regard. My thanks are also due to the authorities, the University Grants Commission, New Delhi, for their financial assistance in bringing out this volume. I am also thankful to messrs. Rathnam Press, Madras, for their neat and pleasant get-up and printing. I would also like to convey my appreciation and blessings to chis. K. Swaminathan and Babu Rajendran, office assistants of the Sanskrit Department, University of Madras, who have been very helpful in matters connected with the printing of this book, and I also thank Dr. N Veezhinathan Dr. M Narasimhachary for translating the Sanskrit Introduction in to English and Dr. C. S. Sundaram for assistiong in publication. 

I shall feel amply rewarded in my labours if this volume meets the approval of the world of scholars in general, and lovers of Sanskrit literary history in particular. Perhaps I cannot make a better appeal to scholars than in the words of Kumarila Bhatta the great mimamsaka, who, several centuries ago, declared:

"Let Scholars therefore, listen to this (work of mine) with ears in the form of their minds properly and gracefully attuned Good men in fact, will approve, without any taint of jealousy, the words of merit and wisdom."

I shall now, in all humility, offer this work at the lotus feet of the renowned world-teacher, the great Adi Sankara. ]

Madras - 600 005.\\
Siddharthi Varsa\\
Kartikasukla Saptami\\
26-11-1979\\
R THANGASWAMI SARMA


\chapter{प्रास्ताविकम्}

\begin{center}
 \textbf{।। ओं नमो ब्रह्मादिभ्यो ब्रह्मविद्यासंप्रदायकर्तृभ्यो वंशऋषिभ्यो नमो \\ महद्भ्यो नमो गुरुभ्यः ।।}
\end{center}

\begin{center}
\begin{enumerate}
\item
नमः श्रुतिशिरःपद्मषण्डमार्तण्डमूर्तये ।\\
 वादरायणसंज्ञाय मुनये शमवेश्मने ।।
\item
 ब्रह्मसृत्रकृते तस्मै वेदव्यासाय वेधसे ।\\
 ज्ञानशक्त्यवताराय नमो भगवतो हरेः ।।
\item
 शङ्करं शङ्कराचार्यं केशवं वादरायणम् ।\\
 सृत्रभाष्यकृतौ वन्दे भगवन्तौ पुनः पुनः ।।
 \end{enumerate}
\end{center} 

 (४) श्रुतिस्मृतिपुराणानामालयं करुणालयम् ।\\
 नमामि भगवत्पादं शङ्करं लोकशङ्करम् ।। 
 
 (५) वेदान्ताम्भोगभीरा नयमकरकुला ब्रह्मविद्याव्जषण्डा \\
 पाषण्डोत्तुङ्गवृक्षप्रमथननिपुणा मानवीचीतरङ्गा ।\\
 यस्यास्योत्था सरस्वत्यखिलभवभयध्वंसिनी शङ्करस्य \\
 गङ्गा शम्भोः कपर्दादिव निखिलगुरोर्नौमि तत्पादपद्मम् ।। 
 
 (६) वेदान्तार्थं गभीरं ह्यतिसुगमतया बोधयामीति विष्णु-\\
 र्व्यासात्माऽसूत्रयत तद् दुरधिगमममूद् वादिदुर्बुद्घिभेदात् ।\\
 भिन्दन् दुर्बुद्धिभेदं य इह करुणयाऽभाष्ययद् भाष्यमेतत् \\
 तं वन्दे सर्ववन्द्यं त्रिजगति भगवत्पादसंज्ञं महेशम् ।। 
 
 (७) नारायणं पद्मभुवं वसिष्ठं शक्तिं च तत्पुत्रपराशरं च । \\
 व्यासं शुकं गौडपदं महान्तं गोविन्दयोगीन्द्रमथास्य शिष्यम् ।\\
श्रीशङ्कराचार्यमथास्य पद्मपादञ्च हस्तामलकञ्च शिष्यम् ।\\
तं तोटकं वार्तिककारमन्यान् अस्मद्गुरून् सन्ततप्रानतोऽस्मि ।। 

(८) सदाशिवसमारम्भां शङ्कराचार्यमध्यमाम् ।
अस्मदाचार्यपर्यन्तां वन्दे गुरुपदम्पराम् ।।


\section{किञ्चित् प्रास्ताविकम् }
संशोधनाय ग्रन्थरचनाय च स्वीकृतोऽयं विषयः - अद्वैतवेदान्तसाहित्येतिहासकोशः - ( A Bibliographical History of Advaita Vedanta literature ) इति । तत्सम्पादने मयानुमूतस्य कृतस्य च परिश्रमस्य फलरूपोऽय विषयः ग्रन्थरूपेण भवतां सन्निधिमागच्छति । 

सर्वदर्शनोत्तमस्य अद्वैतदर्शनस्य विकासकालः प्राचीन मध्यम - आधुनिकभेदेन त्रेधा विभक्तुं शक्यते । तत्र प्राचीनो भागः दशमशतकान्तः, मध्यमो भागः षोडशशतकान्तः, आधुनिकभागस्तु अद्यावधिक इतः परश्च। प्राचीने काले शाङ्करभाष्यस्य व्याख्यानभूतानां ग्रान्थानाम् , भामतीप्रस्थान्स्य विवरणप्रस्थानस्य च वीजावापः, संख्येयानां स्वतन्त्रग्रन्थानाञ्च आविर्भावो दृश्यते । न्यायशैलीनिब्द्धाः न्यायवैशेषिकादिमतसिद्धभेदवादखण्डनपरा अद्वैतग्रन्थास्स्वतन्त्रभूताः मध्यम एव काले प्रादुरभृवन् ।

\section{अद्वैतसिद्धान्तः}
शाङ्करे अद्वैतमते ब्रह्मैव सत्यम् । अन्यत्सर्वं मिथ्या । जगदादि सर्वं शुक्तौ रजतमिव भासते । ब्रह्म निर्गुणम् , अत एव शब्दैरप्रतिपाद्यम् । अत एव वाड्मनसयोरगोचरमिति श्रुतौ कथ्यते । सर्वज्ञत्वादयो गुणा अपि ब्रह्मणि औपाधिका एव । जीवो ब्रह्मरूपोऽपि अज्ञानात् भिन्न इव भाति। मनोबुद्ध्याद्युपाधिभिस्सर्वोऽपि जनः "अह" मिति प्रत्येति । अत एव सुषुप्तौ मनोबुद्ध्यादिलयेन अहमाकारप्रतीत्यभावः। एवं मोक्षावस्थायामपि उपाधिलयात् अहमाकारप्रतीत्यभावः। एवञ्चाहमिति प्रतीतिरौपाधिकी । अयं प्रातीतिको जीव एक एव न तु नाना । अन्तःकरणभेदात् सुखदुःखानुभवभेदः। चैतन्यं सर्वत्र्यापकं स एव आत्मा इत्यभिधीयते । अयमात्मा ईश्वर - जीव - साक्षीति भेदेन त्रिविधः। सर्वजगन्मूलकारणम्  अज्ञानम् , तस्मिन् चैतन्यान्तर्गते सति स आत्मा ईशअवर इत्यभिधीयते । स एव ईश्वरस्सत्त्वरजस्तमोभिर्विप्णु ब्रह्म - शङ्कराख्याः लभते । आत्मा ज्ञानस्वरूपो न तु ज्ञातृस्वरूपः । तस्य ज्ञातृत्वमहङ्काराद्युपाधिभिः । प्रपञ्चस्यापि ज्ञेयत्वमज्ञानादेव । त्रिविधं सत्यत्वं - प्रातिभासिकं व्यावहारिकं पारमार्थिकञ्च। तत्र प्रातिभासिकस्य शुक्तिरजतादेः व्यावहारिकसत्यरजतेन बाधः। व्यावहारिकसत्यत्वं घटपटादीनाम् , तच्च पारमार्थिकसत्येन ब्रह्मदर्शनेन बाध्यते । ब्रह्मणस्तु कस्यामप्यवस्थायां न बाधः । अत एव तत् परमार्थसत् । धटपटदीनां व्यवहारदशायां सत्यत्वम् । वस्तुतस्तु ते स्वाप्निकपदार्थवत् मिथ्याभूताः । यथा स्वप्नगताः पदार्थाः जाग्रद्दशायां न सन्ति अतो मिथ्या, तथा पारमार्थिकदशायां अभावात् एते व्यवहारदशायां विद्यनाना अपुि घटपटादयो मिथ्या । पारमार्थिकात्मज्ञानेन वाधसम्भवात् ।

कासुचित् श्रुतिषु सृष्टिराकाशक्रमेण, कासुचित् श्रुतिषु तेजाआदिक्रमेण, कुत्रचित् सर्वं ब्रह्मैव नान्यदित्युक्तम् । सर्वासामपि श्रुतानां अवाधितप्रामाण्यात् तत्समन्वयोऽवश्यं कर्तव्यः । तथा च इदमेव सिध्यति यत् ब्रह्मैव सत्यमिति पारमार्थिकदृष्ट्या उक्तम् । आकाशादिकमसृष्टिस्तु व्यावहारिकी । अनेनापि प्रमाणेन जगतो व्यावहारिकत्वम्, न परमार्थसत्यत्वम् । यदि जगत् परमार्थसत् स्यात् तर्हि अन्यत् किमपि नास्तीति प्रतिपादिनी श्रुतिरनर्थिका स्यात् । एवञ्च सत्यानृते मिथुनीकृत्य नैसर्गिकोऽयं लोकव्यवहार इति मन्तव्यम् । अत एव अज्ञानिदृष्ट्या तत्सत्यत्वं भासते । ज्ञानिदृष्ट्या च " तस्य पिता अपिता भवति " इत्युक्तरीत्या तस्य मिथ्यात्वम् । न केवलं सर्वस्य जगतो मिथ्यात्वं श्रुत्या वोध्यते, अपि तु स्वस्यापि मिथ्यात्वं श्रुतिर्निरभिमानितया ब्रूते ।

एतस्य सर्वस्य जगतः मूलस्वरूपमनाद्यविद्या । सा च त्रिगुणात्मिका । प्रलयकालसमाप्तिवेलायां अविद्यया जीवकृतकर्मभिश्च " तदैक्षत बहुस्यां प्रजायेय " इति रीत्या परमात्मा संकल्पयति, तत आकाशादिक्रमेण सृष्ट्युद्गमः । ततश्च पञ्चीकृतमहाभूतेभ्यः शरीरादीनि प्रादुर्मवन्ति । एवं उत्पन्नशरीरे प्रविष्टं चैतन्यं जीव इत्यभिधीयते । स च  न अणुः, किन्तु व्यापकः, सर्वस्मिन् शरीरे सुखदुःखानुभवात् । स च जीवः मुक्तिपर्यन्तं स्थायी, जन्मान्तरीयकर्मजसुखदुःखसम्बन्धात् । जीवः विस्मृतकण्ठस्थचामीकरः पुरुष इव आत्मविस्मृतेः, अज्ञानात् सुखदुःखानुभवभाक् । अज्ञानं, लिङ्गशरीरम्, स्थूलशरीरञ्चेति उपाधिस्तस्य सदा प्रत्यासन्नः। अयमुपाधिः " दशमस्त्वमसीति " ज्ञानेन विस्मृतात्मस्वरूपस्य अज्ञानमिव, आत्मज्ञानेन जीवात्मपरमात्मैक्यज्ञानापरपर्यायेण नश्यति, सत्यज्ञानेन विना मिथ्याभूतदर्शनस्यानिवृतेः । एवमविद्यानाशे जीवो मुक्तो भवति । तस्य ज्ञानाग्निः प्रारब्धेतराणि सञ्चितक्रियमाणानि कर्माणि नाशयति । दशायामस्यां प्रारब्धकर्माणि भुञ्जानः स विगतशरीराद्यभिमानो जीवन्मुक्त इत्यभिवीयते । प्रारब्धकर्मावसाने देहपाते स विदेहमुक्तो भवति। इयमेव परमा मुक्तिः अस्यां जीवः परमात्मसायुज्यं नाम स्वरूपमधिगच्छति । न च मतान्तरवत् अल्पेवापि सेवकादिरूपेण भिन्नस्तिष्ठति।। यद्यपि जीवस्सदा आत्मस्वरूप एव तथापि भावरूपेणाज्ञानेन आत्मानं सुखदुःखभाजं मनुते तदैव वद्ध इत्यभिधीयते, तादृशाज्ञाननिवृत्तौ स एव मुक्त इत्युच्यते ।

वेदविहितकर्मभिश्चित्तशुद्धिः ततो ब्रह्मनिष्ठं गुरुं प्रतिशरणगमनम्, तेन तत्त्वमसीतिज्ञानोपदेशः, तच्छ्रवणमनननिदिध्यासनैः सन्यासाश्रममधिवसन् कर्माणि परित्यजन् स सुस्थिरज्ञानो मोक्षं लभते । मोक्षाप्तौ ज्ञानमेव साक्षात् साधनम् । कर्मोपासने तु चित्तशुद्धिचित्तैकाग्र्यप्रापकत्वात् परम्परितसाधने । ज्ञानकर्मसमुच्चयस्तु नेष्टः । नित्यानित्यवस्तुविवेकः, इहामुत्रार्थभोगविरागः शमदमादिसाधनसम्पत् , मुमुक्षुत्वञ्चेति साधनचतुष्टसम्पत्यनन्तरं ब्रह्मजिज्ञासा । अस्मिन् मते अनि र्वचनीयख्यातिः। प्रत्यक्षानुनानोपमानशाब्दार्थापत्यनुपलब्धि - आख्यानि षट् प्रमाणानि । अलौकिकेऽर्थे आत्मनि शाब्दमेव मुख्यं प्रमाणम् । अन्यदनुमानादि तदवष्टम्भेन प्रमाणम् । मतस्यास्य संग्राहकः श्लोकः " ब्रह्म सत्यं जगन्मिथ्या जीवो ब्रह्मैव नापरः " इति । 

\section{अद्वैतसिद्धान्तविकासः}
 एतादृशस्य श्रुतिसिद्धस्य प्राचीनतमस्य अज्ञविज्ञोपकारकस्य सकलदर्शन शिरोऽलङ्काररत्नभूतस्य अद्वैतवेदान्तदर्शनस्य विशेषतः प्रबलप्रतिवादिनः भेदवादिनः न्यायवैशेषिकादयः, शालिकनाथप्रभृतयः पूर्वमीप्रांसकाः आसन् । त्रयोदशशतका दनन्तरं द्वैतमतप्रवर्तका आनन्दतीर्था अपि प्रतिवादिषु अन्यतमा अभूवन् । एतादृशैस्तर्काभिमानिभिर्वैशेषिकादिभिर्युकत्याभासैः कलुषितस्य आनन्दतीर्थादिभिर्निन्दितस्य अद्वैतसिद्धान्तस्य रक्षणार्थं तर्केणैव समाधानार्थं च प्रवृत्तेषु ग्रन्थेषु चित्सुखीयन्यायमकरन्द न्यायदीपावली - अद्वैतसिद्धि - पञ्चदशीत्यादयः खण्डनखण्डस्वाद्यमित्यादयश्च विशिऽय उल्लेवार्हाः । इतरमतखण्डनपरग्रन्थेषु शैलीद्रयं दृश्यते । तत्र प्राचीना शैली तु युक्तिसहिता अर्थगाम्भीर्यवती श्रुतिमधुरशब्दविन्यासयुक्तः सम्भाषणसमा दृश्यते । नव्यनैय्यायिकैः गङ्गेशोपाध्यायप्रभृतिभिः नवीनतया न्यायशास्त्रपरिवर्तने कृते भेदवादखण्डनार्थं प्रवृत्ताः मधुसूदनसरस्वतीप्रभृतयः ब्रह्मानन्दसरस्वत्यन्ता आचार्याः " यक्षानुरूपो बलि " रिति न्यायेन परिष्कारप्रधानान्येव  वाक्यान्यारचय्य इतरमतखण्डने प्रवृत्ताः। सेयं नवीना शैली। चित्सुख - आनन्दबोधादीनां शैली तु प्राचीना । नैतच्छल्यां काठिन्यं दृश्यते । नापि विषयप्रतिपादने युक्तिकथने शब्दविन्यासे वा मन्दता दृश्यते । चाटुकारीणि दृढयुक्तिकानि न्यायोपबंहितानि च वाक्यानि तेषां वाग्मितां सिद्धान्तविमर्शक्षमताञ्च प्रतिपादयन्ति । तत्र मेदवादिनामयमाक्षेपः प्रवलतरः - यत् - जगतो मिथ्यात्वे मायिकत्वे वा तस्य असत्वप्रसङ्गः । यथा च शुक्तिरजतं असत् भवति तथा जगदपि असत् भवेत् । तथा सति सर्वप्रमाणव्यवहारस्य असम्भवस्त्यात् । किञ्चैवं सति वैदिकत्वज्ञानस्यापि मिथ्यात्वापत्तिः। सर्वदृश्यान्तर्गतस्य वेदस्य मिथ्यात्वेन तदुक्तस्य तत्त्वज्ञानस्य अर्थादेव मिथ्यात्वं स्यात् इति । सोऽयमाक्षेपः विचार्य विकल्प्य दूरीकृत आचार्यैः। असत्वमित्यस्य कोऽर्थः ? किं अत्यन्तासत्वम् ? उत किञ्चित्कालासत्वम् ? । नाद्यः, प्रत्यक्षं दृश्यमानस्य अर्थक्रियाकारिणो जगतः शशश्रृङ्गवत् अत्यन्तासत्वानुपपत्तेः । तद्धि अत्यन्तासत् यत् कदापि केनापि नोपलब्घम् , न तु जगत् तथा, सर्वैरपि प्रत्यक्षमुपलभ्यमानत्वात् । तस्मात् जगत् न असत् । नापि सत्वेन स्वीकर्तुं शक्यते । नहि प्रत्यक्षेण दृश्यमानत्वं अर्थक्रियाकारित्वं वा अत्यन्तसत्वस्य प्रयोजकम्, तथा सति स्वाप्नादार्थानां प्रत्यक्षं अनुमूयमानत्वेन, मनोरथानां अर्थक्रियाकारित्वेन च व्यभिचारात् । तस्मात् न शशश्रृङ्गदिवत् अत्यन्तासत् । नापि आत्मवत् अत्यन्तसत् , यद्रूपेण यत् निश्वितं तद्रूपं न व्यमिचरति तत् सत्यम् इति प्रतिपादितं सत्यत्वं यस्य भवति तदेव सत्यमिति वक्तव्यम् । प्रतिक्षणपरिणामिनः सततवञ्चलस्वभावस्य नियतपरिवर्तनशीलस्य अस्य जगतः तद्गतपदार्थानां वा निरुक्तसत्यत्वलक्षणानाक्रान्तत्वात् । किन्तु सद्विविक्तत्वम् , अथवा सदसद्वि लक्षणत्वं वा मिथ्यात्वमिति वक्तव्यम् । इदमेव मिथ्यात्वलक्षणं
 परिष्कृतं न्यायरत्नदीपवली तत्वप्रदीपिका - तर्कसंग्रह (आनन्दगिरीय) प्रमाणमालादिषु ग्रन्थेषु । लघुचन्द्रिकायाञ्च -
आद्यं स्यात् पञ्चपाद्युक्तं ततो विवरणोदिते । 
चित्सुखीयं चतुर्थं तु अन्त्यमानन्दबोधजम् ।। इति । 
अद्वैतसिद्धान्तस्यास्य उपनिषद् गीता - सूत्ररूपाणि श्रुति - स्मृति - युक्तिताम्ना व्यवहृतानि प्रस्थानानि महाप्रस्थ नानि प्रस्थानत्रयमिति च प्रसिद्धानि । तथा भामतीप्रस्थानं विवरणप्रस्थानमिति च अवान्तरप्रस्थानं सिद्धान्तप्रतिपादनभेदमूलकं प्रसिद्धं विद्यते, एतत् प्रस्थानद्वयमपि - 
यया यया भवेत् पुंपां व्युत्पत्तिः प्रत्यगात्मनि ।
सा सैव प्रक्रियेह स्यात् साध्वी सा चानवस्थिता ।।
इति सुरेश्वरोक्त्या मान्यतां प्राप्नोत्येव । तयोस्सिद्धान्तभेदास्तु एवं दृश्यन्ते - 

भामतीकारः कर्मणां विविदिषार्थत्वं वदति, विवरणकारस्तु कर्मणां विद्यार्थत्वं वदति। भामतीकारः ब्रह्मसाक्षात्कारः मनःकरणकः, न तु शब्दाकरणक इति वदति। विवरणकारस्तु वेदान्तवाक्यारूपशब्दकरणक इति वदति । भामतीकारः श्रोतव्यो मन्तव्य इत्य़ादिवाक्ये विधिर्नास्तीति वदति, विधिरस्तीति विवरणकारः । भामतीकारः निदिध्यासंन अङ्गि, श्रवणमनने अङ्गमूते इति वदति । श्रवणं अङ्गि मनननिदिध्यासने अङ्गभूते इति विवरणकारः । जीवेश्वरविषये भामतीकार अवच्छेइवादी, विवरणकारः प्रतिविम्बवादी भवति । भामतीकार अज्ञानाश्रयस्य अज्ञानविषयस्य च भेदं स्वीकरोति, विवरणकार अज्ञानस्य आश्रयविषयभेदो नास्तीति वदति । भामतीकारः प्रतिजीवं मूलाविद्या नाना इति वदति । विवरणकारः मूलाविद्या एकैवेति वदति । भामतीकार अखण्डाकारवृत्तेरूपहितं ब्रह्म विषय इति वदति। विवरणकार अखण्डाकारवृत्तेः शुद्धं ब्रह्म विषय इति वदति। भामतीकारस्य मते साधनचतुष्टये सत्यासत्यवस्तुविवेकः प्रथमसाधनम् , विवरणकारस्य मते नित्यानित्यवस्तुविवेकः प्रथमसाधनम् । स्वाध्यायोऽध्येतव्य इति विधिरर्थावबोधफलक इति भामती, अक्षरग्रहणफलक इति विवरणम् । भामती त्रिवृत्करणम् , विवरणं पञ्चीकरणं च स्वीकरोति । भामतीकारः ब्रह्मणः सर्वज्ञत्वं स्वरूपचैतन्येनेति वदति, विवरणकारः ब्रह्मणः सर्वज्ञत्वं मायावृत्तिभिरिति । भामतीकारस्य मते मनस इन्द्रियत्वमस्ति, विवरणकारस्य मते मनस इन्द्रियत्वं नास्ति । भामतीमतेे अविद्या जीवाश्रिता भवति, विवरणमते अविद्या ईश्वाराश्रिता भवति ।

	अविद्यया संसक्तं अविद्यारूपिणा अपाधिना सहितञ्च ब्रह्मणः विशुद्धं चैतन्यमेव जीव इत्युच्यते । प्रतिजीवं एकं अन्तःकरणं उपाधिर्भवति । अतश्च जीवः परिच्छिन्न अल्पज्ञ इति व्यवहारः ।
विवरणमते अन्तःकरणतत्संस्कारावच्छिन्नाज्ञानप्रतिविम्बितं चैतन्यं जीव इति, अन्तःकरणे ब्रह्मप्रतिबिम्बमेव जीव इति भवति। जीवस्वरूपविषये अवच्छेदवादः, आभासवादः, प्रतिविम्ववादश्चेति पक्षाःसन्ति - 
वाचस्पतेरवच्छिन्न आभासो वार्तिकस्य च ।
संक्षेपशारीरककृतः प्रतिविम्बं तथेष्यते ।। इति ।। 
तत्र अवच्छेदो नाम अन्तःप्रवेशः । तेन युक्त अवच्छिन्नः । यथा वा जले अन्तःप्रविष्टं आकाशं जलावच्छिन्नमित्युच्यते एवं अज्ञानाश्रयीभूतं शुद्धचैतन्यं जीव इत्युच्यते । येषां मते अविद्यया संयुक्तं चैतन्यं जीवः, स च अवच्छिन्नो वा उपहितो वा प्रतिबिम्बितो वा तेषां मते जीवावस्थायां जीवस्य एकत्वम् । अविद्याया एकत्वात् । अयमेव एकजीववादः । सुखदुःखादिवैचित्र्यन्तु उपाधिभेदात् भवति । अयमेकजीववादः भामतीकाराणां मते । बहुप्रकारया अविद्यया अविद्याकार्यबुद्ध्या वा संयुक्तं चैतन्यं जीवः, स च जीवः अविच्छिन्नो वा, उपहितः, प्रतिविम्बितो वा, भवति इति ये वदन्ति तेषां मते जीवनानात्वम् । वार्तिककारसुरेश्वरः पञ्चपादिकाविवरणकारप्रकाशात्मा संक्षेपशारीरककारसर्वज्ञात्मा प्रकटार्थविवरणकारश्व बहुजीववादिनः । न्यायमकरन्दकारानन्दबोधश्च सर्वेषामाचार्याणां सिद्धान्तं सविमर्शं निरूप्य, विषेषतः ब्रह्मसिद्धि - इष्टसिद्धिसिद्धान्तं मण्डयन् एकजीववादे श्रुतिप्रामाण्यं प्रदर्श्य युक्तियुक्ततां साधयति । 

%शाङ्करभाष्यम् 
 
एवं निर्विशेषात र्न्लिक्षणाच्च स्वयग्प्रकाशात् ब्रह्मणः सविशेषस्य सलक्षणस्य श्च जगतः कथमुत्पत्तिः? एकस्मात् अद्वैतात् ब्रह्मणः नानात्मकस्य जगतः कथं सृष्टिः ? इत्येतेषां प्रश्नानां यथावत्
समाधनं मायास्वरूपवर्णनद्वारा आचार्यैरूपवर्णितम् । शङ्करभगवत्पादैः माया - अविद्या अज्ञान - अव्यक्तदि शब्दानां समानार्थकत्वं सूत्रभाष्ये ( १ -४ -३) उपवर्णितम् । शङ्करभगवत्पादादर्वाचीनेषु आचार्येषु माया - अविद्ययोस्तारतम्यविषये जीवेश्वरस्वरूपविषये अविद्याया आश्रयत्वे मायाया आश्रयत्वे च सिद्धान्तभेदा उद्भाविताः । भावतीकारः जीवेश्वरविषये अवच्छेदवादं अज्ञानाश्रयस्य अज्ञानविषयस्य च भेदं स्वीकुर्वन्ति । विवरणकाराश्च जीवेश्वरविषये प्रतिविम्ववादं अज्ञानाश्रयत्वविषयत्वयोर्भेदाभावं च स्वीकुर्वन्ति । भामतीमते अज्ञानविषयीकृतं चैतन्यं ईश्वरः । अज्ञानाश्रयीभूतं चैतन्यं जीव इति । विवरणप्रस्थाने ईश्वरविषये आभासवादं, जीवविषये प्रतिबिम्बवादं च स्वीकृत्य अज्ञानोपहितं विम्वचैतन्यं ईश्वरः, अन्तःकरणतत्संस्कारावच्छिन्नाज्ञानप्रतिबिम्बितं चैतन्यं जीव इति स्वीक्रियते । एवं माया - अविद्याशब्दयोस्सूक्ष्ममर्थभेदं वर्णयन्ति। जीवत्व - ईश्वरत्वप्रापकोपाधिस्तु भावरूपं त्रिगुणात्मकं सदसद्भ्यां अनिर्वचनीयं अनादि अज्ञानम् । तच्चाज्ञानं माया - अविद्याभेदेन द्विविधम् । शुद्धसत्वप्रधान मायापदवाच्यम् । मलिनसत्वप्रधानं अविद्यापदवाच्यम् । मायोपहिंत चैतन्यं ईश्वरः । अविद्योपहिंत चैतन्यं जीवः ।

मायारहिते परमेश्वरे प्रवृत्तिर्न भवति । मायाशक्तिरहितः परमेश्वरः जगत्सर्जने अशक्तो भवति। मायाशक्तिरेव अविद्यात्मिका बीजशक्तिरव्यक्तनाम्ना व्यवह्रियते । मायेयं परमेराश्रया भवति। 
अग्नेरपृयक्मूता दाहिका शक्तिरिव माया ब्रह्मण अपृथक्मूता शक्तिः। मायेयं ब्रह्मण एकदेशवर्तिनी न तु कृत्स्नवर्तिनी । 

न कृत्स्नवृत्तिः सा शक्तिस्तस्य किन्त्वेकदेशभाक् । \\
घटशक्तिर्यथा भूमौ स्निग्धमृद्येव वर्तते ।। \\
पादोऽस्य विश्वा भूतानि त्रिपादस्ति स्वयम्प्रभः ।\\
इत्येकदेशवृत्तित्वं मायाया वदति श्रुतिः ।। (पञ्चदशी)\\
 
"पादोऽस्य विश्वा भूतानि, त्रिपादस्यामृतं दिवि " इति श्रुतिश्व प्रतिपादयति । 
सत्वरजस्तमोगुणात्मिकेयं माया ज्ञानविरोधिनी भावरूपः पदार्थः। माया न सती । नापि असती । सदसदुभयविलक्षणतया शास्रे सा अनिर्वचनीयेति कथ्यते । ब्रह्मज्ञानेन बाधसम्भवात् माया न सती । त्रिकालावाधितत्वं हि सत्वम् । यदि माया सती स्यात् तर्हि तस्या अवाधितत्वमेव स्यात् । ब्रह्मज्ञानेन तु  माया बाध्यते। तस्मात् सा न सतीति वक्तुं शक्यते । परन्तु तस्याः प्रतीतिरनुमूयते । तस्मात् असतीति च न वक्तुं शक्यते । यतोऽसत् वस्तु न प्रतीयेत। एवञ्च मायायां उभयविरुद्धयोर्बधितत्वप्रतीतिविषयत्वरूपयोः गुणयोरनुभवगम्यत्वात् माया अनिर्वचनीयेति निश्चीयते । मायाभिधाया अविद्यात्वं प्रमाणासहिप्णुत्वमेव । तर्कादिवलेन माया न ज्ञातुं पार्यते । यथा च अन्धकारस्य साहाय्येन अन्धकारस्य प्रतीतिस्तथैव तर्कवलेन मायायाः प्रतीतिः। उक्तञ्चेदं बृहदारण्यकवार्तिके -
अविद्याया अविद्यात्वे इदमेव तु लक्षणम् ।
यत् प्रमाणासहिण्णुत्वं अन्यथा वस्तु सा भवेत् ।। इति । 
सूर्योदयकाले यथा च अन्धकारस्य नाशो दृश्यते तथा ज्ञानोदयकाले मायायाः प्रतीतिर्नश्यति - 
सेयं भ्रान्तिर्निरालम्वा सर्वन्यायविरोधिनी ।
सहते न विचारं सा तमो यद्वद् दिवाकरम् ।। 
इत्युक्तम्  नैष्कर्म्यसिद्धौ । एवञ्च प्रमाणासहिण्णुरूपिणी माया जगत उत्पत्तौ कारणमिति स्वीकर्तव्यम् । सा अव्यक्ता शक्तिः कार्यानुमेया इति विवेकचूडामणौ - 
अव्यक्तनाम्नी परमेशशक्तिरनाद्यविद्या त्रिगुणात्मिका या ।
कार्यानुमेया सुघियैव माया यया जगत्सर्वमिदं प्रसूयते ।। इति । 
मायायाः शक्तिरावरणविक्षेपभेदेन द्विधा भिन्ना भवति । आभ्यामेव शक्तिभ्यां ब्रह्मणः वास्तविकं सद्रूपमाच्छाद्यते । ब्रह्मणि असत असत्यस्य च जगतः प्रतीतिरारोप्यते ।
शक्तिद्वयं हि मायाया विक्षेपावृतिरूपकम् ।
विक्षेपशक्तिर्लिड्गादि ब्रह्माण्डान्तं जगत् सृजेत् ।। 
अन्तर्दृग्दृश्ययोर्भेदं बहिश्च ब्रह्मसर्गयोः ।
आवृणोत्यपरा शक्तिस्सा संसारस्य कारणम् ।। इति दृगदृश्यविवेके । 
एवं -
" आच्छाद्य विक्षिपति संरफुरदात्मरूपम् 
जीवेश्वरत्वजगदाकृतिभिर्मृषैव ।
अज्ञानमावरणविभ्रमशक्तियोगात् 
आत्मत्वमात्रविषयाश्रयताबलेन ।। इति । "
संक्षेपशारीरके च उक्तम् । अधिष्ठानस्य सत्यत्वापलापादनन्तरमेव अधिष्ठाने नृतनधर्मारोपो भवेत् । यथा च ऐन्द्रजालिकस्य इन्द्रजालविद्या द्रष्ट़़ृणां नेत्रेषु वास्तविकीं दर्शनक्षम्तां आच्छाद्य भ्रान्तेरुत्पादनादेव सफला इति लोके दृश्यते, एवमेव मायाया आवरणरूपा शक्तिः ब्रह्मणश्शुद्धस्वरूपमाच्छादयति । यथा च अग्निः स्वविरोधिनि जले साक्षात् प्रवेप्टु शक्तोऽप सूक्ष्मरूपेण पात्रादिद्वारा जले प्रविश्य तदीयं शैत्यं अपहनुत्य तत्र स्वीयं उष्णत्वं प्रदर्शयति तथेयं मायां सूक्ष्मतरेण स्वकीयमू रूपेण ब्रह्मणि प्रविश्य तदीयं निर्विषयं निराश्रयञ्च स्वरूमपह्नुत्य स्वीयं साश्रयत्वसविशेषत्वरूपं तत्र प्रदर्शयति । एवं लघुर्मेघः दर्शकाणां नेत्रं आच्छादयन् विस्तुतस्यादित्यमण्डलस्य यथा आच्छादकरो भवति तथा परिच्छिन्नमज्ञानं अपरिच्छिन्नस्य असंसारिण आत्मन आवारकं भवति । आवरणशक्तिरेव द्रष्टृदृश्ययोरन्तर्भेदम् , बहिर्ब्रह्मजगतोश्च भेदं उत्पादयति । यथा च रज्जौ सर्पभ्रान्तिस्सर्पज्ञानमुद् भावयति तथा मायापि अज्ञानावृते आत्मनि आकाशादिजगतिः ज्ञानमुद्भावयति । उक्तञ्च - 
मायाशक्तिर्निखिलकलनां साम्प्रतं वा विरुद्धाम् 
स्वाधिष्ठाने चितिफलयुता दर्शयत्याविमोक्षम् । 
नैल्यं व्योम्नि स्रजि विषधरो वार्यथा रश्मिपूगे 
तद्वन् मिथ्यात्मनि जगदिदं कल्पितं स्वप्नवच्च ।। 
इति  प्रत्यक्तत्वचिन्तामणौ । इयं माया अविद्यापदवाच्य़ा अज्ञानाख्या कथं जाता ? केन कारणेनास्याः ब्रह्मणा सम्बन्धो जात इति न चोदनीयम् । अविद्यायाः तत्सम्बन्धस्य च अनादित्वाङ्गीकारात् - 
" जीव ईशो विशुद्धा चित् तथा जीवेशयोर्भिदा ।
अविद्या तच्चितोर्योगः षडस्माकमनादयः ।। "
इति। मायासम्बन्धादेव प्रपञ्चोत्पत्तिः। मायेयं साश्रया सविषया च भवति । मायायां विषय ईश्वरः । आश्रयो जीवः । एतदेव साश्रयत्वं सविषयत्वञ्च ज्ञातरूपे ब्रह्मणि तदीयत्वं प्रदर्शयति । साश्रयत्वेन सविषयत्वेन च भासमानं यत् ज्ञानं तदेव महत्तत्वम् । महत्तत्वादेव प्रपञ्चस्योत्पत्तिः । 
माया (अविद्या)

नवीनास्तु अद्वैतिनः अज्ञानस्य शक्तिः - ज्ञानशक्तिः । क्रियाशक्तिस्तु आवरणविक्षेपभेदेन द्विविधा भवति । रजस्सत्वाभ्यामनभिमूतं तम आवरणशक्तिः । सा च अत्र घटो नास्तीति प्रपञ्चत्र्यवहारहेतुः। तमस्सत्वाभ्यामनभिमूतं रजो विक्षेपशक्तिः । सा च आकाशादिप्रपञ्चोत्पतिहेतुः । विक्षेपशक्तिपता अज्ञानेन उपहितस्यैव ईश्वरस्य जगत उपादानकारणता । अत्र "यथोर्णनाभिस्सृजते गृह्णते च" इति श्रुतिः प्रमाणम् । अत्र मते आवरणशक्तिपधानं अज्ञानमविद्या । विक्षेपशक्तिप्रधानमज्ञानं माया इत्युच्यते । मीयते अपरोक्षवत् प्रदर्श्यते अनयेति माया । स्वीयशक्तिवलात् प्रपञ्चमिमं प्रत्यक्षवत् सत्यमिव च प्रदर्शयतीयमित्यस्याः मायाख्या अन्वर्था भवति । इयमेव माया तमः, अविद्या, अव्यक्तमित्यादिशब्देन व्यवह्रियते - इत्यत्र 
अन्यथा भानहेतुत्वात् इयं मायेति कीर्तिता ।
आत्मतत्वतिरस्करात् तम इत्युच्यते बुधैः ।
विद्यानाश्यत्वतोऽविद्या मोहस्तत्कारणत्वतः ।
सद्वैलक्षण्यदृष्ट्रयायं असदित्युच्यते बुधैः ।। 
कार्यवत् व्यक्तताभावात् अव्यक्तमिति गीयते ।
एषा माहेश्वरीशक्तिर्न स्वकतन्त्रा परात्मवत् ।। 
इत्यादिवृद्धवचनानि प्रमाणानि । एवं जगतः सृष्टौ ईश्वरस्य अभिन्ननिमितोपादानत्वमपि मायाद्वारेणैवेत्यत्र - इन्द्रो मायाभिः पुरुरूप ईयते । सर्वं खलु इदं ब्रह्म तज्जलानिति शान्त उपासीत यतो वा इमानि भूतानि इत्यादीनि श्रुतिवाक्यानि, 
अजोऽपि सन्नव्ययात्मा भूतानामीश्वरोऽपि सन् ।
प्रकृतिं स्वां अधिष्ठाय सम्भवाम्यात्ममायया ।। 
निरुपमनिर्गुणेऽप्यखण्डे मयि चिति सर्वविकल्पनादिशून्ये । 
घटयति जगदीशजीवभेदान् अघटितघटनापटीयसी माया ।।
इत्यादिवचतानि च प्रमाणानि । एवञ्च तत्वप्रतिभासप्रतिवन्धेन अतत्वप्रतिभासहेतुः आवरणविक्षेपशक्तिद्वयवृत्ती अविद्या सर्वप्रपञ्चपकृतिरिति प्रतिपादयन्ति । प्रतिपादितञ्च विद्यारण्यस्वामिभिः पञ्चदश्याम् । विद्यारण्यस्वामी तु विवरणप्रस्थानानुयायी सन्नपि विवरणप्रस्थानानुकृलानां सिद्धन्तानां मार्गप्रदर्शकः भामतीविवरणप्रस्यानयोस्समन्वयकारी च विराजते । रजस्तमोऽनभिभूता शूद्धसत्वप्रधाना माया, रजस्तमोऽभिभूता मलिनसत्वप्रधाता अविद्या इति प्रतिपादयन् मायाप्रतिविम्बितं चैतन्यं सर्वज्ञत्वादिगुणविशिष्टं ईश्वर इति, अविद्याप्रतिबिम्बितं जीव इति - 
सत्वशुध्यविशुद्धिभ्यां मायाविद्ये च ते मते । 
मायाबिम्बो वशीकृत्य तां स्यात् सर्वज्ञ ईश्वरः ।
मायाख्यायाः कामघेनोर्वत्सौ जीवेश्वरावुभौ । 
यथेच्छं पिबतां द्वैतं अद्वैतं परमार्थतः ।। इत्यादिना पञ्चदश्यां विवरणप्रमेयसंग्रहे च प्रतिपादयति ।
मायाविषये प्राचीना :-

मायाविषये नवीना  :-

एवं विवरणमते ब्रह्मचैतन्यं ईश्वरचैतन्यं जीवचैतन्यञ्चेति चैतन्यत्रयं स्वीकृतम् । परन्तु कूटस्थचैतन्यं, ब्रह्मचैतन्यं, जीवेश्वरचैतन्यद्वयञ्चेति चतुर्विधं चैतायं स्वीकुर्वन् विद्यारण्यः भामतीविवरणप्रस्थानयोस्समन्वयकारिणं आत्मानं परिचाययति - 
कूटस्थो ब्रह्म जीवेशौ इत्येवं चिच्चतुर्विधा । 
घटाकाशमहाकाशौ जलाकाशाभ्रखे यथा ।। इति । 
साक्षिस्वरूपविषये विद्यारण्यात् प्राचीना वेदान्तिनः जीवेश्वराभ्यां भिन्नं शुद्धं चैतन्यात्मानं साक्षिणं वदन्ति । विद्यारण्यस्तु कूटस्थचैतन्यमेव साक्षीति प्रतिपादयति । यथा नृत्यशालास्थदीपः साक्षी भवति तथा कूटस्थचैतन्यमेव साक्षीति - 
नृत्यशालास्थितो दीपः प्रभुं सभ्यांश्च नर्तकीम् । 
दीपयेदविशेषेण तदभावेऽपि दीप्यते ।।
अहंकारः प्रभुः, सभ्याः विषया नर्तकी मतिः ।
तालादिधारीण्यक्षाणि दीपः साक्ष्यवभासकः ।। इति 
इमानि तत्वानि पञ्चदश्यादिषु ग्रन्थेषु वर्णितानि दृश्यन्ते ।
एवं दशम्शतकादनन्तरमुत्पन्नेषु षोडशशतकान्तेषु ग्रन्थेषु न्यायमकरन्दन्यायदीपावली - वेदान्तकौस्तुभाद्यादिषु ग्रन्थेषु भामतीविवरणप्रस्थानानां विकासः, अनिर्वचनीयख्यातिसाधनम् , अखण्डार्थत्वनिरूपणमित्यादिविषयश्च विकासं प्राप्तो दृश्यते । विशेषतश्च अद्वैतसिद्धान्तखण्डनपरस्य द्वैतसिद्धान्तमण्डनपरस्य व्यासतीर्थकृतन्यायामृतस्य तद्व्याख्यायाः न्यायामृततरङ्गिण्याः न्यायभास्करस्य च खण्डनाय प्रवृत्तानां अद्वैतसिद्धि - गुरु - सघुचन्द्रिका - विट्ठलेशीयादीनां ग्रन्थानाम् , एवं आधुनिके काले विशिष्टाद्वैतखण्डनपराणां शतदूषणीखण्डनपराणां शतभूषण्यादिग्रन्थानाञ्च आविर्भाव अद्वैतसिद्धान्तस्य विकासोन्मुखतामापादयति । 
यद्यपि अद्वैतवेदान्तस्य विकासः प्रस्थानत्रय्येति व्यपदिश्यते तथापि तस्य विकासः मद्दृष्ट्या न केवलं प्रस्थानत्रय्या परन्तु प्रस्थानत्रयीबहिर्मूतैः खण्डनमण्डनपरैर्वादप्रधानैश्च ग्रन्थैः, अद्वैतानन्दानुभाविभिराचार्यैरुपरचितैरनुभवप्रधानभा वाविष्करणात्मकैः - आत्मविद्याविलास - पञ्चदशयादिप्रकरणग्रन्थैः, अद्वैतवेदान्तसिद्धान्तप्रतिपादकैः ब्रह्मनैर्गुण्यवाद - विद्वन्मोदतरङ्गिणीप्रभृतिभिः काव्यैः योगवासिष्ठादिभिर्महाकाव्यैश्च सुतरामद्वैतवेदान्तसाहित्यं विकसितमिति तु नापरोक्षम् । 

\section{एतद्ग्रन्थप्रयोजनम्}
संस्कृतेतरासु विशेषत आङ्गिलहिन्दीप्रभृतिषु प्रसृततरासु आधुनिकासु भाषासु प्रतिविषयमेतादृशाः ग्रन्थास्समुपलभ्यन्ते येषु तत्तद्विषयविशेषसम्बद्धाः कति ग्रन्थाः विद्यन्ते ? तत्तद्विषयस्योत्पत्तिः कुतः ? तत्तद्विषयस्य वृद्धिर्विकासश्च कथम् ? तत्तद्विषयाभिवृद्धौ दत्तचित्ता आचार्याः के के ? इति विशिष्टेतिहासो दरीदृश्यते । भौतिकविज्ञानाभिवृद्धौ वृद्धसमे नितरां विज्ञानवादिनि भारतेतरदेशे भौतिकविज्ञानविकासे स्तनन्धयशिशुकल्पस्य भाग्तवर्षस्य या कीर्तिः, या श्रद्धा,  यश्च गौरवगरिमा तस्य मुख्यं कारणं भारतीयानामद्वैतवेदान्तशास्रमिति तु निस्संशयं विदुषाम् । एतादृशस्याद्वैतवेदान्तसाहित्यस्येतिहासलेखने ममायं विशेष अभिनिवेशस्समुदपद्यत । संस्कृतेतरभाषासु प्रायशस्तत्तद्भाषायामेव तत्तद्विषयविशेषस्य इतिहासः दृश्यते । तस्मान्ममापि संस्कृतभाषायामेव संशोधनपूर्वकेतिहासलेखने महानादरस्समभूत् । तादृश भिनिवेशपूर्तये मद्रासविश्वविद्यालयसंस्कृतविभागभूतपूर्वप्राध्यक्षाः पूज्यतमाः यशश्शरीरमापन्नाः Dr. V. राघवमहोदयाः मां प्रोत्साहितवन्त इति तेषामधमणोंऽस्मि । 

\section{ग्रन्थस्यास्य विषयप्रतिपादनसरणिः}
अद्वैतवेदान्तसाहित्यं न केवलं प्रस्थानत्रय्य परमन्यैरपि ग्रन्थैरिति पूर्वमुपवर्णितम् । तस्मात् अद्वैतवेदान्तसाहित्यस्येतिहासः उपनिषत् - गीता - सूत्रतद्भाष्य - वादप्रधान - अनुभवप्रधान - काव्यात्मकशैलीप्रधानैः ग्रन्थैर्विकसित इति अद्वैतवेदान्तस्य प्रस्थानषट्कमिति कथनमपि अविरुद्धमन्वर्थञ्च भवेत् । अत एवायं ग्रन्थः पूर्वोत्तरभागद्वयेन विभक्तः। तत्र पूर्वभागे ग्रन्थप्राधान्यं मनसिकत्य षट् परिच्छेदाः परिकल्पिताः । 

तत्र प्रथमे उपनिषत्प्रस्थानप्रधाने परिच्छेदे अद्वैतपरा उपनिषदः ग्रन्थप्रतिपाद्यवर्णनपूर्वकं तासां भाप्याणि, अद्वैताचार्यकृताः व्याख्याः, उपव्याख्याश्च निरूपिताः । उपनिषदां कालः विचारनिर्णयव्राह्य इति मत्या विमर्शकवरेण्यानांं सिद्धान्ता एव तत्र तत्र निरूपिताः । 
द्वितीये गीताप्रस्थानप्रधाने परिच्छेदे निखिलपुराणान्तर्गता अद्वैतमताविरोधिन्यः गीताः भगवगद्गीताश्च प्रतिपाद्यविशिष्टा सभाष्य़व्याख्योपव्याख्या निरूपिताः । 

तृतीये सृत्रतद्भाष्यप्रस्थानप्रधाने परिच्छेदे ब्रह्मसृत्राणि तेषां भाष्यम् , पञ्चपादिकाविवरणप्रस्थानानुयायिनः ग्रन्थाः, भामतीप्रस्थानानुयांयिनः ग्रन्थाः भाष्यस्य स्वतन्त्रव्याख्यात्मकाः ग्रन्थाः भाष्यानुसारिण्यः वृत्तयश्च सव्याख्या निरूपिताः । 
चतुर्थः परिच्छेदः प्रकारणग्रन्थप्रधानः । परिच्छेदेऽस्मिन् प्रकरणग्रन्थाः सत्र्याख्याः मार्कि अष्टाशतेभ्य अन्यूना निर्दिष्टः। प्रकारणग्रन्थरचयितारः प्रस्यानत्रयेऽपि ग्रन्थकारा दृश्यन्ते एवं बहूनां प्रकारणग्रन्थानां रचयिता दृश्यते । प्रकरणग्रन्थाश्च असंख्येया वर्तन्ते । तस्मात् पोतरुक्त्यादिदोषपरिहाराय परिच्छेदेऽस्मिन् सव्य ख्यानां प्रकरणग्रन्थानां नामनिर्देशमात्रं मातृकाक्रमेण निर्दिष्टम् । तत्तद्ग्रन्थ प्रतिपादितास्तु विषयाः तत्तद्ग्रन्थकर्तृविचारावसरे (अद्वैतग्रन्थकारपरिच्छेदे ) प्रतिपादिताः । 
पञ्चमे परिच्छेदे उद्धारमात्रज्ञाताः नाममात्रप्रसिद्धाश्च ग्रन्थाः निरूपिताः । ते च ग्रन्थाः कुत्रोद्धुता इत्यादिभिर्विवरणैस्साकं चत्वारिंशदन्यूताः प्रतिपादिताः । 

षष्ठः परिच्छेद अज्ञातकर्तृकाद्वैतग्रन्थप्रधानः । परिच्छेदेऽस्मिन् मुद्रितामुद्रितभेदेन अज्ञातकर्तृकाः पञ्चशतं ग्रन्थाः वर्णिताः । तेष्वपि समाननामानः विभिन्नग्रन्था अमुद्रिता विभिन्नहस्तलिखितपुस्तकालयेषु लभ्यन्ते । ये च मद्रासराजकीय हस्तलिखितपुस्तकालये अडयारपुस्तकालये तञ्जपुरपुस्तकालये च लभ्यन्ते, येषाञ्च वर्णनात्मकविस्तृतग्रन्थसूची विद्यते ते चाधीतास्सन्दृष्टाश्च । तेषाञ्च ग्रन्थानां मिन्नत्वाभिन्नत्वे निश्चिते च । तथापि अनिवार्यं काठिन्यं यत्र तत्र मौनीभाव एव स्वीकर्तव्यः- उदाहरणार्थम् - आनन्दबोधेन न्यायमकरन्दे ( p. No . 170) पदानां सिद्धे सङ्गतिग्रहस्थापनावसरे "विस्तरतस्तु न्यायदीपिकायां अवगन्तव्यम् " इत्युक्तम् । एष ग्रन्थः प्रकरणवशात् स्वतन्त्रो वेदान्तग्रन्थ इत्येव निर्णतुं शक्यते । परं न्यायदीपिकाख्याः बहवो ग्रन्थाः द्वैतपूर्वमीमांसाजैनन्यायशास्त्रेषु विद्यन्ते । न ते अद्वैतसाधकाः। मद्रपुरी तिरुवनन्तपुर हस्तलिखितपुस्तकालययोर्विद्यमानः न्यायदीपिकाख्यः ग्रन्थस्तु शाब्दनिर्णयव्याख्यारूप एव न तु स्वतन्त्रग्रन्थः, तस्य शाब्ददीपिका इत्येव नाम दृश्यते । एवमेव नासिकउज्जैनपञ्चावादिपुस्तकालयस्थेषु वर्णनात्मकविस्तृतसृचीरहितेषु पुस्तकेषु किमस्माभिः कर्तव्यम् । तेषां ग्रन्थानां विषये संशयस्सुस्थ एव । तस्मादज्ञातकर्तृकग्रन्थप्रधानोऽयं परिच्छेदस्सन्दृव्धः । 

एवं प्रस्थानषट्केन विकासितस्यास्य वेदान्तसाहित्यस्य ग्रन्थाः मूलव्याख्यानमेदभिन्नाः परिच्छेदषट्केन प्रतिपादिताः । ग्रन्थाः मुद्रिता उतामुद्रिता इति च निर्दिष्टाः । अमुद्रितग्रन्थानां प्राप्तिस्थानानि च निरूपितानि । प्रसिद्धतमेभ्यः मूलव्याख्याग्रन्थेभ्य ऋते प्रायस्सर्वेषामपि मुद्रितामुद्रितग्रन्थानां विषयाश्च सङ्ग्रहेण प्रदर्शिताः । व्याख्येयग्रन्थानां यावन्त्यः व्याख्या उभयविधास्सन्ति तास्सर्वा अपि व्याख्येयग्रन्थप्रस्तावे अन्यत्र व्याख्यानग्रन्थकर्तृप्रस्तावे च निर्दिष्टाः । पठितृणां प्रेक्षितृणाञ्च सौलभ्याय अनुक्रमणिकापि व्याख्येयव्याख्यान ग्रन्थानुसारं समारचिता। 

एवं षट्सु परिच्छेदेषु मूलव्याख्यानभेदभिन्ना ज्ञाताज्ञातकर्तृकभेदभिन्नाश्च ग्रन्थाः षट्सप्तत्यधिकसहस्रसंख्यापरिमिताः (1076) ग्रन्थेऽस्मिन् निर्दिष्टाः । ग्रन्थेप्वेतेषु मुद्रित्ग्रन्थाः द्वयधिकचतुश्शतसंख्याका (402) निर्दिश्यन्ते स्म अमुद्रितेष्वपि ग्रन्थलिपिदेवनागरीलिप्योर्मुद्रिता एव निर्देष्चटुं पारिताः । व्याख्याग्रन्थाः षट्षष्ठ्यधिकचतुश्शताधिकग्रन्थाः (466) निर्दिष्टाः । 

द्वितीयो भागः 

भागेऽस्मिन् द्वितीये अद्वैतवेदान्तग्रन्थकर्तार उपवर्णिताः । भागोऽयमपि पञ्चभिः परिच्छेदैः परिच्छिन्नः । तत्र प्रथमे परिच्छेदे शङ्करभगवत्पादेभ्यः प्राचीना अनिर्णीतकालादिविषयविशेषाः अद्वैताचार्या वर्ण्यन्ते स्म । द्वितीये परिच्छेद्दे शङ्करादर्वाचीनाः प्रसिद्धतमाः अद्वैताचार्यशब्दव्यपदेश्याः ग्रन्थकारास्तिथिक्रमानुसारं निर्दिष्टाः । प्रत्यद्वैताचार्यं आश्रमभेदेन नामान्तराणि, कालः, आचार्य - प्राचार्यसतीर्थ्याः, शिष्य - प्रशिष्याः, वासमूमिः, सामायेकाः, पोषक - पोष्या राजान इत्यादिकं तत्तत्प्रणीतग्रन्थप्रामाण्यादुपवर्णितम् । पश्चात् सदुपदेशात्मकगुरुशिष्यपरम्परारूपसम्प्रदायसमागता पण्डितमण्डलीमात्रप्रसिद्धा कथा अथवा किंवदन्ती च प्रतिपादिता । अनन्तरं संशोधनधुरीणानां प्राच्यप्रतीच्यभाषाविदुषां सिद्धान्ताश्च न्यरूपिषत । तत्तदाचार्यप्रणीता अद्वैतग्रन्थाः प्रतिपादिताः । ग्रन्थप्रतिपाद्यानां रूपरेखा च समाकृष्टा । तत्र तत्र आवश्यकस्थलेषु गुरुपरम्परावृक्षुः, वंशवृक्षः व्याख्यानव्याख्येयग्रन्थपरम्परावृक्षा इति बहवो वृक्षास्समारोपिताः। ग्रन्थकारा ये सन्यासिनस्समानाचार्याश्च ते अभिन्ना इति निर्णीयन्ते स्म । ये तु समानाभिख्याः भिन्नप्राचार्याश्च त अभिन्ना इति न निर्णीताः परन्तु भिन्ना इति प्रथमद्वितीयसंख्याभिर्विशेषिताः । 

यद्यपि ग्रन्थेऽस्मिन् ग्रन्थकर्तॄणां इतिहासलेखने तत्तत्प्रणीत - अद्वैतग्रन्थनिर्देशेनैवालम् , तथापि यदि तेषां इतिहासलेखने तत्तत्प्रणीतेभ्य अद्वैतेतरग्रन्थेभ्यश्च प्रमाणं लभ्यते तदा तत्तद्ग्रन्थनिर्देशोॄ ऽप्यावश्यकस्मम्पन्नः । तस्मात्तेऽपि ग्रन्था अत्र निर्दिष्टाः । परन्तु न ते अनुक्रमणिकायां स्थानमर्हन्ति। 

तृतीये परिच्छेदे ये च ज्ञातकालादिविषयविशेषाः, ये तु न प्रसिद्धतमाः, ये च न सन्यासिनः, दर्शनान्तरे प्रतिष्ठिता अपि अद्वैतेऽपि कतिपयग्रन्थकर्तारस्तेषां जीवनवृत्तान्तः ग्रन्थाश्च ग्रन्थप्रतिपाद्यसहिताः वर्णमालाक्रमेण निरूपिताः । 
चतुर्थे परिच्छेदे ज्ञातनामान अज्ञातकालादिविषयविशेषाश्शङ्करादर्वक्तना ग्रन्थकारास्तेषां ग्रान्थश्च निरूपिताः। पञ्चमे परिच्छेदे अज्ञातनामानः परन्तु तत्तद्गुरुशिष्यत्वेन मुद्रितामुद्रितग्रन्थेषु आत्मानं निर्दिशन्तो ग्रन्थकाराः तेषां ग्रन्थाश्च निर्दिष्टाः । 
ग्रन्थस्यादौ सड्क्षेपसङ्केतबोधिनी निर्दिष्टा । सग्प्रदायसमागतश्शान्तिपाठकमः, मङ्गलपाठक्रमसहितस्संयाजितः यश्च अद्वैतवेदान्तसाहित्यस्य अनिवार्य आवश्यकः कल्याणकारी च भाग इति ममाभिनिवेशः । ग्रन्थस्यान्ते अद्वैतग्रन्थग्रन्थकर्तृ सामान्यानुक्रमणिकाश्च संयोजिताः तत्र स्थूलाक्षराङ्कितपत्रसंख्याः ग्रन्थानां, ग्रन्थकतॄणां, समानाख्यानां ग्रन्थानां ग्रन्थकर्तॄणाञ्च भिन्नतां वृत्तान्तञ्च बोधयन्ति । सामान्यानुक्रमणिकायां ग्रन्थकर्तुर्विभिन्नानि तथा अन्यानि नामानि, अद्वैतवेदान्तबहिर्भूताः ग्रन्थाः, देशपुरप्रामादयः गोत्रवंशादयः, अद्वैतवेदान्तसम्बद्धाः पारिभाषिकशब्दाः, विद्या, विभिन्ना वादाश्च स्थानं लभन्ते स्म। गच्छतस्स्खलनमिति न्यायात् , मुद्रणालयानवधानाद्वा उत्पन्नाः केचन मुद्रणदेषाः शोधनेन योजनेन दूरीकृतास्सन्ति । क्वचित् क्वचित्  संस्कृत्सुलभीकरणघोषगौरवदानाय केवलं सन्धिविच्छेदस्स्वीकृतः । द्वित्रिस्थलेषु षष्टिश्ब्दस्थाने षष्ठिरिति प्रयोगः, सिंहशब्दस्थाने सिम्ह इति प्रयोगश्च स्वीकृतः । उद्धृतवाक्यानुक्रमणिका च संयोजिता । अन्ते च महानद्वैताचार्यगुरुशिप्यपरम्परावृक्षस्संरोपितः, यस्मिन् अद्वैतग्रन्थकर्तारः गुरवः, शिष्याश्च केवल स्थानमलभन्त, न मठाम्नायादिप्रसिद्धाः, नापि अद्वेंतसम्प्रदायागता अपि अकृतग्रन्थाः । नापि च संरकृतेतरभाषासु ग्रन्थप्रणेतारो वा । 

एतद्ग्रन्थरचनायामनिवार्यं कठिन्यम् ।
एवममुद्रित मुद्रित उद्रधृत - ज्ञातकर्तृक - अज्ञातरकर्तृ-व्याख्यान - व्याख्येयभेदेन भिन्नास्सर्वेऽपि ग्रन्थाः प्रबन्धेऽस्मिन् यथामति यथाशक्ति च व्यलिख्यन्त तथापि 
"कति कृतयः कति कवयः कति लुप्ताः कति चरन्ति कति शिथिलाः । "
इत्यभियुक्तं क्तिमनुसृत्य कति ग्रन्थाः मद्दृष्टिगोचरतामानीता ? कति विलुप्ता इति न ज्ञायन्ते । इदमेकं काठिन्यम् । अपरमेकं काठिन्यमिदम् यत् समाननामानस्समानकर्तृकाः विभिन्नकर्तृकाश्च बहवो ग्रन्थाः विभिन्नहहस्तलिखितपुस्तकालयेषु लक्ष्यन्ते । ये च मद्रासगजकीयअडयार - तञ्जपुर - तिरुवनन्तपुरपुस्तकालयेषु दृश्यन्ते, येषाञ्च वर्णनात्मकविस्तृतसूच्यो विद्यन्ते, तेषां ग्रन्थानां भिन्नात्वाभिन्नत्वे निर्णीते । परन्तु दूरदेशस्थानां वर्णनात्मकसूचीरहितानां पत्रव्यवहारेणापि अवेद्यानां ग्रन्थानां विषये किमस्माभिः कर्तव्यम् ? । तेषां विषये संशयस्सुस्थ एव । अत एवाज्ञातकर्तृकपरिच्छेदस्संदृब्धः । 
कृतज्ञताविष्करणम् 
एतादृशमदीयज्ञानवृद्धिकरस्य कार्यस्य सफलतायै नूतनबृात्सूचीसम्पादनाय मद्रासविश्वविद्यालयसंस्कृतविभागस्थानां निखिलविदेशस्वदेशस्थ स्तलिखितपुस्तकालय वर्णनात्मकविस्तृतसृचीनां प्रदानेन बहूपकृतवतां मद्रासविश्वविद्यालयसंस्कृतविभागस्थानां समेषाम् , अनर्घाभिप्रायोपदेशादिप्रदानेन उपकृतवतां प्राध्यापकवर्याणां Dr. K.K. Raja महोदयानाञ्च अधमर्णतमः कृतज्ञश्चास्मि ।
मदीयं प्रास्ताविकं आङ्गिलभाषायां संक्षिप्य अनूद्य उपकृतवते Dr. N. Veezinathan M.A. Ph.D , \& Vedanta Siromani महाशयाय कृतज्ञतां निवेदये । 
अप्तुदितग्रन्थावलोकनाय हस्तलिखितग्रन्थप्रदानेन उपकारिभ्यः मद्रामराजकीयहस्तलिखितपुस्तकालयाध्यक्षेभ्यः, अडयारपुस्तकालयाधियेभ्यश्च मदीयां कृतज्ञतां विनिवेदये । 
मुद्रितग्रन्थावलोकनाय नियमानपि शिथिलीकृत्य असंख्यग्रन्थप्रदानेन प्रोत्साहितवतां M.M. कुप्पुस्वामिशास्रियुस्तकालयाधिकारिणाम् , मद्रपुरीसंस्कृतकलाशालाधिकारिणाम् , प्रान्तीयकलाशाला (Presidency College ) सस्कृतविभागाध्याक्षाणाञ्च घन्यवादवादी कृतज्ञश्चास्मि । 
बहुषु ग्रन्थेषु मुदापणाय सत्स्वपि विषयविशेषम् , एतादृशग्रन्थनिर्माणपरिश्रमम् , ग्रन्थयोग्यताञ्व परिशील्य मुद्रापणाय प्रकाशनाय च प्रयतितवते उपकृतवते च गुणग्राहिणे मद्रासविश्वविद्यालयाय विश्वविद्यालयानुदानायोगाय (University Grant Commission) च, एवं सुचारुरूपेण मुद्रितवते रत्नमुद्रणलयाय (Rathnam Press) मदीयां कृतज्ञतां विनिवेदये । मुद्रापणकार्यस्यास्य सफलतायै बहूयकृतवद्भ्यां संस्कृतविभागकार्यालयकार्यत्र्यापृताभ्यां चिरञ्जीविस्वामिनाथ - बाबू राजेन्द्मभ्यां सन्तु श्रेयांसि भूयांसीत्याशासे । 
अनेन मदीयेन परिश्रमेण विद्वासस्संस्कृताभिज्ञाश्च नूनं किञ्चिदपि प्रयोजनं प्राप्स्यन्तीति विश्वस्य - 
"तद्विद्वांसोऽनुगृह्णन्तु चित्तश्रोत्रैः प्रसादिभिः । 
सन्तः प्रणयिवाक्यानि गृह्णन्ति ह्यनसूयवः ।। इति "
कुमरिलभट्टवाक्येन सम्प्रार्थ्य ग्रन्थमिमं आचार्य - जगद्गुरुशङ्करभगवत्पादकमलयोस्समर्पये - 

सिद्धार्थिनामसंवत्सरम् 
कार्तिक - शुक्ल सप्तमी
षड्विंशतितमो दिवसः 
26 -11 - 1979 
मद्रास-5

इत्थम् 
विदुषां विधेयः 
R THANGASWAMI SARMA


\chapter{ABBREVIATIONS सङ्क्षेपसङ्केतबोधिनी.}

ABORI: Annals of Bhandarkar Oriental Research Institute, Poona.\\
ALS: Adyar Library Series, Madras.\.\\
AL: Adyar Manuscripts Library, Madras..\\
ALPS: Adyar Library Padmphlet Series..\\
AMSS: Aryamata Samvardhini Sanskrit Series Madras..\\
AMS: Advaita Manjari Series, Kumbakanam.\\
AOR: Annals of Oriental Research, Madras..\\
AUSS: Allahabad Univesity Sanskrit Series.\\
ASS: Anandasrama Sanskrit Series, Poona.\\
AS: Asutosh Mukherjee Series, Calcutta.\\
BMP: Balamanorama Press, Madras - 4.\\
BRD: Gaekwad Oriental Research Institute, Baroda.\\
BSS: Banaras Sanskrit Series.\\
BUL: Bombay University Manuscripts Library..\\
BSP: Bombay Sanskrit and Prakrit Series..\\
CPB: Catalogue of Central Provinces and Berar..\\
COSS: Cochin Sanskrit Series..\\
CSS: Calcutta Sanskrit Series..\\
CU: Calcutta University.\\
DC: Descriptive Catalogues.\\
Edn: Edition..\\
GNPB: Gopalnarayan Press, Bombay.\\
GOML Mysore: Government Oriental Manuscripts Library, Mysore.\\
HIP: History of Indian Philosophy: S. N. Das Gupta..\\
IA: Indian Antiquary..\\
IHQ: Indian Historical Quarterly..\\
IHR: Indian Historical Review..\\
IOL: India Office Library, London.\\
JOR: Journal of Oriental Research, Madras..\\
JRAS: Journal of the Royal Asiatic Society..\\
MGOML: Madras Government Oriental Manuscripts Library..\\
MGOMLS: Madras Government Oriental Manuscripts Library Series..\\
MUSBS: Madras University Sanskrit Bulletin Series.\\
MUSS: Madras University Sanskrit Series..\\
NCC: New Catalogus Catalogorum..\\
NIA: New Indian Autiquary..\\
NSP: Nirnayasagar Press, Bombay.\\
ORISS Mysore: Oriental Research Institute Sanskrit Series, Mysore.\\
PNS: Pandit New Series. Banaras.\\
PSB: Pandit Series Banaras..\\
Q: Quoted.\\
RAS: Royal Asiatic Society.\\
SBC: Catalogues of Saraswathi Bhavan Library Banaras..\\
SBTS: Saraswathi Bhavan Text Series.\\
SBS: Saraswathi Bhavan Series, Banaras.\\
SME: Sankara` s Memorial Edition, Vanivilas Press, Sri Rangam..\\
SVP: Sri Vanivilas Press, Sri Rangam..\\
SVPK: Saradha Vilas Press, Kumbak0nam..\\
TCD: Travancore Curator` s Library Descriptive Catalogue..\\
TCL: Travancore Curator` s Library.\\
TMPL: Travancore Maharaja` s Palace Library.\\
TSML: Tanjore Saraswathi Mahal Library..\\
TSS: Travancore Sanskrit Series..\\
VBS: Viswabharathi, Santiniketan, Calcutta..\\
VNSS: Vizianagaram Sanskrit Series..\\
VORIT: Venkateshwara Oriental Research Institute, Tirupati.\\
VVP: Vanivilas Press (Sri Rangam).\\
VVSS: Vanivilas Sanskrit Series, Sri Rangam.

\chapter{शाङ्करभाष्यपठनक्रमः ।}
श्रीशङ्करभगवत्पादकृतभाष्यपाठारम्भे सम्प्रदायसमागतः शान्तिमन्त्रपठनक्रमः 

शिष्यास्सर्वे वस्त्रावगुण्ठितशरीराः पठेयुः 

शिवनामनि भवितेऽन्तरङ्गे महति ज्योतिषि मानिनीमयार्धे । \\
दुरितान्यपयान्ति दूग्दूरे महुरायान्ति महान्ति मङ्गलानि ।।

स्मृते सकलकल्याणभाजनं यत्र जायते । \\
पुरुषस्तमजं नित्यं व्रजामि शरणं हरिम् ।। 

१. ओं शं नो मित्रः शं वरुणः। शं नो भवत्वर्यमा। शं न इन्द्रो बृहस्पतिः। शं नो विष्णुरुरुक्रमः । नमो ब्रह्मणे । नमस्ते वायो । त्वमेव प्रत्यक्षं ब्रह्मासि त्वमेव प्रत्यक्षं ब्रह्म वदिष्यामि । ऋतं वदिष्यामि । सत्यं वदिष्यामि । तन्मामवतु । तद्वक्तारमवतु । अवतु माम् । अवतु वक्तारम् । ओं शान्तिः शान्तिः शान्तिः । 

२. सह नाववतु । सह नौ भुनक्तु । सह वीर्यं करवावहै। तेजस्वि नावधीतमस्तु मा विद्विषावहे । ओं शान्तिः शान्तिः शान्तिः ।

३. यददृन्दसामृषभो विश्वरूपः । छन्दोभ्योऽध्यमृतात् संबभूव। स मेन्द्रो मेधया स्पृणोतु। अमृतस्य देव धारणो भूयासम् । शरीरं मे विचर्षणम् । जिह्ना मे मधुमत्तग। कर्णाभ्यां भूरि विश्रुवम् । ब्रह्मणः कोशोऽसि मेधयापिहितः । श्रुतं मे गोपाय । ओं शान्तिः शान्तिः शान्तिः ।

४. अहं वृक्षस्य रेरिवा । कीर्तिः पृष्ठं गिरेरिव । ऊर्ध्वपवित्रो वाजिनीव स्वमृतमस्मि। द्रविणँ सवर्चसम् । सुमेधा अमृतोक्षितः । इति त्रिशङ्कोर्वेदानुवचनम् । ओं शान्तिः शान्तिः शान्तिः ।

५. पूर्णमदः पूर्णमिदं पूर्णात् पूर्णमुदच्यते ।
पूर्णस्य पूर्णमादाय पूर्णमेवावशिष्यते ।।
ओं शान्तिः शान्तिः शान्तिः ।

६. आप्यायान्तु ममङ्गानि वाक्याणिचक्षुःश्रोत्रं अथो बलमिन्द्रियाणि सर्वाणि । सर्वं ब्रह्मोपनिषदम् । माहं ब्रह्म निराकुर्याम् । मा मा ब्रह्म निराकरोत् । अनिराकरणमस्त्वनिराकरणं मे अस्तु । तदात्मनि निरते य उपनिषत्सु धर्मास्ते मयि सन्तु ते मयि सन्तु । ओं शान्तिः शान्तिः शान्तिः ।

७. वाड्मे मनसि प्रतिष्ठिता । मनो  मे वाचि प्रतिष्ठितम् । आविरावीर्म एधि । वेदस्य म आणीस्थः । श्रुतं मे मा प्रहासीः । अनेनाधीतेन । अहोरात्रान् सन्दधामि । ऋतं वदिष्मामि । सत्यं वदिप्यामि । तन्मामवतु तद्वक्तारमवतु । अवतु मां अवतु वक्तारमवतु वक्तारम् । ओं शान्तिः शान्तिः शान्तिः ।

८. भद्रं नो अपि वातय मनः । ओं शान्तिः शान्तिः शान्तिः ।

९. भद्रं कर्णेभिःशृणुयाम देवाः । भद्रं पश्येमाक्षभिर्यजत्राः स्थिरैरङ्गै स्तुष्टुवांसस्तनूभिः । व्यशेम देवहितं यदायुः । स्वस्ति न इन्द्रो वृद्धश्रवाः । स्वस्ति नः पूषा विश्ववेदाः । स्वस्ति नस्ताक्ष्यों अरिष्टनेमिः । स्वस्ति नो बृहस्पतिर्दधातु । ओं शान्तिः शान्तिः शान्तिः ।

१०. यो ब्रह्मणं विदधाति पूर्वं यो वै वेदांश्च प्रहिणोति तस्मै । तं ह देवं आत्मबुद्धिप्रकाशं मुमुक्षुर्वै शरणमहं प्रपद्ये । ओं शान्तिः शान्तिः शान्तिः । 

ओं नमो ब्रह्मादिभ्यो ब्रह्मविद्यासंप्रदायकर्तृभ्यो वंशऋषिभ्यो नमो गुरुभ्यः। सर्वोपल्पवरहितः प्रज्ञानघनः प्रत्यगथों ब्रह्मैवाहमस्मि । अधीहि भो भगवः, अधीहि भो भगवः । 

ततः किञ्चिद् भाष्यं पठेयुः ।
अनन्तरं दक्षिणामूर्त्यष्टक देहं प्राणमपीत्यन्तं पठन्तः प्रतिश्लोकं नमस्कुर्युः 

११. मौनव्याख्याप्रकटितपरब्रह्मतत्वं युवानम् \\
वषिंष्ठान्तेवसदृषिगणैरावृतं ब्रह्मनिष्ठैः ।\\
आचार्येन्द्रङ्करकलितचिन्मुद्रमानन्दमूर्तिम् \\
स्वात्मारामं मुदितवदनं दक्षिणामूर्तिमीडे ।। 

१२. विश्वं दर्पणदृश्यमाननगरीतुल्यं निजान्तर्गतम् \\
पश्यन्नात्म निमायया बहिरिवोद्भूतं यथा निद्रया ।\\
यस्साक्षात्कुरुते प्रबोधसमये स्वात्मानमेवाद्वयं \\
तस्मै श्रीगुरुमूर्तये नम इदं श्रीदक्षिणामूर्तये ।। 

१३. बीजस्यान्तरिवाङ्कुरो जगदिदं प्राङ्निर्विकल्पं पुनः \\
मायाकल्पितदेशकालकलनावैचित्र्यचित्रीकृतम् ।\\
मायावीव विजम्भयत्यपि महायोगीव यः स्वेच्छया \\
तस्मै श्रीगुरुमूर्तये नम इदं श्रीदक्षिणामूर्तये ।। 

१४. यस्यैव स्फुरणं सदात्मकमसत्कल्पार्थकं भासते \\
साक्षात्तत्वमसीति वेदवचसा यो बोधयत्याश्रितान् ।\\
यत्साक्षात्करणाद् भवेन्न पुनरावृत्तिर्भवाम्भोनिधौ \\
तस्मै श्रीगुरुमूर्तये नम इदं श्रीदक्षिणामूर्तये ।। 

१५. नानाच्छिद्रघटोदरस्थितमहादीपप्रभाभास्वरम् 
ज्ञानं यस्य तु चक्षुरादिकरणद्वारा बहिः स्पन्दते ।
जानामीति यमेव भान्तमनुभात्येतत्समस्तं जगत् 
तस्मै श्रीगुरुमूर्तये नम इदं श्रीदक्षिणामूर्तये ।। 
१६. देहं प्राणमपीन्द्रियाण्यपि चलां बुद्धिञ्च शून्यं विदुः
स्त्रीबालान्धजडोपमास्त्वहमिति भ्रान्ता भृशं वादिनः ।
मायाशक्तिविलासकल्पितमहाव्यमोहसंहारिणे 
तस्मै श्रीगुरुमूर्तये नम इदं श्रीदक्षिणामृर्तये ।। 
तत इमान् श्लोकान् पठन्तः नमस्कुर्युः - 
१७. श्रुतिस्मृतिपुराणानामालयं करुणालयम् । 
नमामि भगवत्पादं शङ्करं लोकशङ्करम् ।। 
१८. शङ्करं शङ्कराचार्यं केशवं वादरायणम् ।
सूत्रभाष्यकृतौ वन्दे भगवन्तौ पुनः पुनः ।। 
१९. नमः श्रुतिशिरःपद्मषण्डमार्तण्डमूर्तये ।
वादरायणसंज्ञाय मुनये शमवेश्मने ।।
२०. ब्रह्मसूत्रकृते तस्मै वेदव्यासाय वेधसे 
ज्ञानशक्त्यवताराय नमो भगवतो हरेः ।। 
२१. नारायणं पद्मभुवं वसिष्ठं शक्तिञ्च तत्पुत्रपराशरञ्च 
व्यासं शुकं गौडपदं महान्तं गोविन्दयोगीन्द्रमथास्य शिष्यम् । 
श्रीशङ्कराचार्यमथास्य पद्मपादञ्च हस्तामलकञ्च शिष्यम् 
तं तोटकं वार्तिककारमन्यान् अस्मद्गुरून् सन्ततमानतोऽस्मि ।।
२२. शङ्कराश्लेषविलसदानन्दामृतनिर्भराम् ।
विश्वोत्तंसितपादाब्जां ब्रह्मविद्यां विभावये ।।
२३. वेदान्तनिकुरुम्बेण तात्पर्येण प्रकाशितः ।
स्वात्मानन्दैकरस्येन कल्याणाय शिवोऽस्तु नः ।। 
२४. सदाशिवसमारम्भां शङ्कराचार्यमध्यमाम् ।
अस्मदाचार्यपर्यन्तां वन्दे गुरुपरम्पराम् ।। 
ततः भाष्यश्रवणं कर्तव्यम् ।
B
श्रीशङ्करभगवत्पादकृतभाष्यपाठान्ते सम्प्रदायसमागतः शान्तिमन्त्रपठनक्रमः । 

१. शं नो मित्रः शं वरुणः । शं नो भवत्वर्यमा । शं न इन्द्रो बृहस्पतिः । शं नो विष्णुरुरुक्रमः । नमो ब्रह्मणे । नमस्ते गयो । त्वमेव प्रत्यक्षं ब्रह्मसि । त्वामेव प्रत्यक्षं ब्रह्मावादिषम् । ऋतमवादिषम् । सत्यमवादिषम् । तन्मामावीत् । तद्वक्तारमावीत् । आवीन् माम् । आवीद् वक्तारम् । ओं शान्तिः शान्तिः शान्तिः । 

२. ओं सह नाववतु । सह नौ भुनक्तु । सह वीर्यं करवावहै । तेजस्वि नावधीतमस्तु । मा विद्विषावहै । ओं शान्तिः शान्तिः शान्तिः । 

३. ओं यश्चन्दसामृषभो विश्वरूपः । छन्दोभ्योऽध्यमृतात्संबभूव । स मेन्द्रो मेधया स्पृणोतु। अमृतस्य देव धारणो भूयासम् । शरीरं मे विचर्षणम् । जिह्वा मे मधुमत्तमा । कर्णाभ्यां भूरि विश्रुवम् । ब्रह्मणः कोशोऽसि मेधया पिहितः । श्रुतं मे गोपाय । ओं शान्तिः शान्तिः शान्तिः।
 
४. ओं अहं वृक्षस्य रेरिवा । कीर्तिः पृष्ठं गिरेरिव । ऊर्ध्वपवित्रो वाजिनीव स्वमृतमस्मि । द्रविणँ सवर्चसम् । सुमेधा अमृतोक्षितः । इति त्रिशङ्कोर्वेदानुवचनम् । ओं शान्तिः शान्तिः शान्तिः । 

५. ओं पूर्णमदः पूर्णमिदं पूर्णात्पूर्णमुदच्यते । पूर्णस्य पूर्णमादाय पूर्णमेवावशिष्यते । ओं शान्तिः शान्तिः शान्तिः । 

६. ओं आप्यायन्तु ममाङ्गानि वाक् प्राणश्चक्षुःश्रोत्रमथो बलमिन्द्रियाणि च सर्वाणि । सर्वं ब्रह्मौपनिषदम् । माहं ब्रह्म निराकुर्याम् । मा मा ब्रह्म निराकरोदनिराकरणमस्त्वनिराकरणं मे अस्तु । तदात्मनि निरते य उपनिषत्सु धर्मास्ते मयि सन्तु ते मयि सन्तु । ओं शान्तिः शान्तिः शान्तिः । 

७. ओं वाड् मे मनसि प्रतिष्ठिता । मनो मे वाचि प्रतिष्ठितम् । आविरावीर्म एधि । वेदस्य म आणीस्थः । श्रुतं मे मा प्रहासीः । अनेनाधीतेन । अहोरात्रान् सन्दधामि । ऋतं वदिष्यामि । सत्यं वदिष्यामि । तन्मामवतु । तद्वक्तारमवतु । अवतु माम् । अवतु वक्तारम् । अवतु वक्तारम् । ओं शान्तिः शान्तिः शान्तिः । 

८. ओं भद्रं नोऽपि वातय मनः । ओं शान्तिः शान्तिः शान्तिः । 

९. ओं भद्रं कर्णेभिः श्रृणुयाम देवाः । भद्रं पश्येमाक्षभिर्यजत्राः । स्थिरैरङ्गैस्तुष्टुवाँसस्तनूभिः । व्यशेण देवहितं यदायुः । स्वस्ति न इन्द्रो वृद्धश्रवाः । स्वस्ति नः पूषा विश्ववेदाः । स्वस्ति नस्ताक्ष्यों अरिष्टनेमिः । स्वस्ति नो बृहस्पतिर्दधातु । ओं शान्तिः शान्तिः शान्तिः ।

१०. ओं यो ब्रह्माणं विदधाति पूर्वं यो वै वेदांश्च प्रहिणोति तस्मै । तं ह देवमात्मबुद्धिप्रकाशं मुमुक्षुर्वैशरणमहं प्रपद्ये। ओं शान्तिः शान्तिः शान्तिः। 

इतः परं दक्षिणामूर्त्यष्टके अवशिष्टाः श्लोकाः पठनीयाः, प्रणामश्च कर्तव्यः । 

११. राहुग्रस्तदिवाकरेन्दुसदृशो मायासमाच्छादनात् 
सन्मात्रः करणोपसंहरणतो योऽभूत्सुषुप्तः पुमान् ।
प्रागस्वाप्समिति प्रबोधसमये यः प्रत्यभिज्ञायते 
तस्मै श्रीगुरुमूर्तये नम इदं श्रीदक्षिणामूर्तये ।।

१२. बाल्यादिष्वपि जाग्रदादिषु तथा सर्वास्ववस्थास्वपि 
व्यावृत्तास्वनुवर्तमानमहमित्यन्तः स्फुरन्तं सदा ।
स्वात्मानं प्रकटीकरोति भजतां यो मुद्रया भद्रया 
तस्मै श्रीगुरुमूर्तये नम इदं श्रीदक्षिणामूर्तये ।। 

१३. विश्वं पश्यति कार्यकारणतया स्वस्वामिसम्बन्धतः 
शिष्याचार्यतया तथैव पितृपुत्राद्यात्मना भेदतः ।
स्वप्ने जाग्रति वा य एष पुरुषो मायापरिभ्रामितः 
तस्मै श्रीगुरुमूर्तये नम इदं श्रीदक्षिणामूर्तये ।।  

१४. भूरम्भांस्यनलोऽनिलोऽम्बरमहर्नाथो हिमांशुः पुमान् 
इत्याभाति चराचरात्मकमिदं यस्यैव मूर्त्यष्टकम् ।
नान्यत्किञ्चन विद्यते विमृशतां यस्मात्परस्प्राद्विभोः 
तस्मै श्रीगुरुमूर्तये नम इदं श्रीदक्षिणामूर्तये ।। 

१५. सर्वात्मत्वमिति स्फुटीकृतमिदं यस्मादमुष्मिंस्तवे 
तेनास्य श्रवणात्तदर्थमननाद् ध्यानाच्च सङ्कीर्तनात् ।
सर्वात्मत्वमहाविभूतिसहितं स्यादीश्वरत्वं स्वतः 
सिध्येत् तत्पुनरष्टधा परिणतञ्चैश्वर्यमव्याहतम् ।। 

१६. वटविटपिसमीपे भूमिभागे निषण्णम् 
सकलमुनिजनानां ज्ञानदातारमारात् ।
त्रिभुवनगुरुमीशं दक्षिणामूर्तिदेवम् 
जननमरणदुःखच्छेददक्षं नमामि ।। 

१७. चित्रं वटतरोर्मूले वृद्धाश्शिष्या गुरुर्युवा । 
गुरोस्तु मौनं व्याख्यानं शिष्यास्तु छिन्नसंशयाः ।।

१८. अड्गुष्ठतर्जनीयोगमुद्राव्याजेन देहिनाम् ।
श्रुत्यर्थं ब्रह्मजीवैक्यं दर्शयन्नोऽवताच्छिवः ।। 

१९. ओं नमः प्रणवार्थाय शुद्वज्ञानैकमूर्तये ।
निर्मलाय प्रशान्ताय दक्षिणामूर्तये नमः ।।

२०. गुरवे सर्वलोकानां भिषजे भवरोगिणाम् । 
निधये सर्वविद्यानां दक्षिणामूर्तये नमः ।। 
२१. चिद्धनाय महेशाय वटमूलनिवासिने । 
सच्चिदानन्दरूपाय दक्षिणामूर्तये नमः ।। 

C

महाप्रदोषदिने परं यथापूर्वं भाष्यपाठश्रवणं कृत्वा दशशान्तिमन्त्रपाठानन्तरं (B/1-10) श्रीशङ्करभगवत्पादपूजां कृत्वा श्वेतसर्षपेण मधुना च मिश्रं दधि दूर्वातृणानि च निवेदनं कृत्वा एते मङ्गलपाठश्लोकाः वारत्रयं पठनीयाः - 

१. अशुमानि निराचष्टे तनोति शुभसन्ततिम् ।
स्मृतिमात्रेण यत्पुंसां ब्रह्म तन्मङ्गलं परम् ।। (त्रिः)

२. अतिकल्याणरूपत्वात् नित्यकल्याणसंश्रयात् ।
स्मर्तॄणां वरदत्वाच्च ब्रह्म तन्मङ्गलं परम् ।। (त्रिः)

३. ओंकारश्चाथशब्दश्च द्वावेतौ ब्रह्मणः पुरा।
कण्ठं भित्वा विनिर्यातौ तस्मान्माङ्गलिकावुभौ ।। (त्रिः)

ओं अथ ओं अथ ओं अथ ।
अतः परं "राहुग्रस्त" इत्यारब्धाः "दक्षिणामूर्तये नम"
इत्यन्ताः (११ - २१) श्लोकाः सप्रणामं पठनीयाः । 

इति शान्तिपाठसम्प्रदायः । 

\chapter{उपनिषत्प्रस्थानम् }
वैदिकसाहित्ये उपनिषदां स्थितिरन्यादृशी । औपनिषदानां महर्षीणां कार्यात्मकविश्वब्रह्माण्डस्य अभिन्नत्वे अखण्डत्वे च महान् विश्वासः दरीदृश्यते । औपनिषदा मुनयस्सुखदुःखेभ्य उदासीना दृश्यन्ते । सान्तस्यानन्तेन सम्बन्धः औपनिषदानां महर्षीणां रहस्यात्मकवाणीष्वेव प्रथमं अभिव्यक्तः । 
ब्राह्मणग्रन्थास्तु कर्मसु पुरुषं प्रेरयन्ति । उपनिषदस्तु ज्ञाने प्रेरयन्ति । जीवनात् जीवनसम्बद्धविचारे उपनिषदामैदम्पर्यं दृश्यते । वेदकालिका भारतीया ऐहिकस्यैश्वर्यस्य साधने यथा बद्धपरिकरा दृश्यन्ते, यथा च ब्राह्मणकालिका भारतीयास्स्वर्गादिपरलोकेप्सवश्व दृश्यन्ते न तथोपनिषत्कालिका भारतीयाः । परन्तु ते साधका ऐहिकैश्वर्यात् परलोकेच्छायाश्चोदासीनाः मुमुक्षवश्च दृश्यन्ते । गतिशीलैः प्राकृतिकवस्तुभिरेव जनिमतां न सम्बन्धः, परन्तु अनिर्वचनीयेन केनचित् स्थिरतत्वेनैव तेषां सम्बन्ध इत्युपनिषत्कालवर्तिनां भारतीयानां सिद्धान्तः । अतएवोपनिषदां सिद्धान्ताः विचारप्रधाना वर्तन्ते । 
उपनिषदः गम्भीरानर्थान् काव्यसुलभया गद्यपद्यात्मकशैल्या प्रतिपादयन्ति । परम् उपनिषदः नैककर्तृकाः । यत एकस्यामेवोपनिषदि शिक्षकभेदो दृश्यते । 

उपनिषदां संख्याः - 
सवोंपनिषदां मध्ये सारमष्टोत्तरं शतम् ।
सकृच्छ्रवणमात्रेण सर्वाघौघविकृन्तनम् ।। (१ - ४४ ) इति 
अष्टोत्तरशतोपनिषदां सारभूतत्वं प्रतिपादयन्त्या मुक्तिकोपनिषदा अष्टोत्तरशताधिकानां उपनिषदां सत्वे प्रमाणमप्यावेदितं भवति। परन्तु न तास्सर्वा उपनिषद अद्यावधि प्रकाशिता उपलब्धाः वा सन्ति । काश्चनोपनिषद अडयारपुस्तकालयात् प्रकाशतां नीतः। तास्वष्टोत्तरशतोपनिषत्सु दशोपनिषद ऋग्वेदीयाः, एकोनविंशत्युपनिषदश्शुक्लयजुर्वेदीयाः, द्वात्रिंशदुपनिषदः कृष्णयजुर्वेदीयाः, षोडशोपनिषदस्सामवेदीयाः, एकत्रिंशदुपनिषद अथर्ववेदीया इति ज्ञायते । 

अष्टोत्तरशतोपनिषत्स्वपि विषयप्रतिपादनदृष्ट्या त्रयोदशोपनिषद एव प्राचीनतमा इति ज्ञायन्ते । ऐतरेय - कौषीतक्युपनिषद ऋग्वेदीयाः, छान्दोग्याकेनोपनिषदस्मामवेदीयाः, तैत्तरीय नारायण - कठ - श्वेताश्वतर - मैत्रायण्युपनिषदः कृप्णयजुर्वेदीयाः, ईशावास्योपनिषच्छुक्लयजुर्वेदीया बृहदारण्यकोपनिषदपि शुक्लयजुर्वेदीया, मुण्डकमाण्डूक्यप्रश्नोपनिषद अथर्ववेदीया इति च प्रसिद्धाः । एतास्वपि दशोपनिषदामेव शङ्करभगवत्पादैर्भाष्यं कृतमिति ता एवात्र प्रधानतमास्स्वीक्रियन्ते। नृसिम्होत्तरतापिनीकौषीतकी- श्वेताश्वतरोपनिषदामपि व्याख्याः दृश्यन्ते । अतः अद्वैताचार्यैः व्याख्याताः अथवा व्याख्यातत्वेन प्रसिद्धाः सर्वा अप्युपनिषदः निर्देशार्हा इति सामान्यं नियमं मनसि कृत्वा शङ्करैरव्याख्याता अपि नारायणाश्रमि - शङ्करानन्दादिभिर्व्याख्याता अद्वैतमतप्रतिपादकास्सर्वा अप्युपनिषदः निर्दिष्टाः ।

उपनिषदां पौर्वापर्यम् - 
सर्वासूपनिषत्सु का वा उपनिषत् प्राचीना ? का वा नवीना ? इति निश्चेतुं न शक्यते । यतोऽस्मिन् विषये मतभेदास्सप्रमाणाः दृश्यन्ते ।  

१. प्रोफेसर - डायसनस्तु गद्यशैल्यां दृश्यमाना उपनिषद एव प्राचीना इति "फिलासफि आफ द उपनिषत्स् " इति ग्रन्थे अभिप्रैति । 

२. प्रो - रामचन्द्ररानडे तु "ए कंस्ट्रक्टिव सर्वे आफ़ उपनिषदिक फिलासफ़ि " नामके ग्रन्थे उपनिषदां पौर्वापर्यमेवमभिप्रैति - बृहदारण्यक - छान्दोग्यईश - केन - ऐतरेय - तैत्तरीय - कौषीतकी - कठ-मुण्डक - श्वेताश्वतर - प्रश्न - मैत्रायणि - माण्डूक्या इति उपनिषदां उत्तरोत्तरकालोत्पन्नतां प्रतिपादयन्ति। तस्य मतेन उपनिषदः पञ्चभिर्वर्गैः परिगणिताः - प्रथमवर्गे बृहदारण्यकछान्दोग्ये, द्वितीयवर्गे ईशकेनोपनिषदौ, तृतीयवर्गे ऐतरेयतैत्तरीयकौषीतक्यः, चतुर्थवर्गे कठमुण्डकश्वेताश्वतराः, पञ्चमरवर्गे प्रश्नमैत्रीमाण्ड्क्यानि चेति । एषु प्रथमवर्गः प्राचीनतमः, अन्तिमश्चार्वाचीन इति । 
३. बेलवलकर महाशयस्तु - एकस्यामेवोपनिषदि भिन्नकालिकानां रचनाविशेषाणां दर्शनात् , एकस्या एवोपनिषदः केचन भागाः प्राचीनाः, केचनार्वाचीना भवन्तीति वर्णयति । 
४. सर् - राधाकृष्णमहोदयास्तु वेदकालादारभ्य क्रिस्तोः पूर्वं (B.C) षष्ठशतकात्पूर्वावधिकः काल उपनिषदां काल इति निश्चिन्वन्ति। 
प्राचीनतमासु उपनिषत्सु दार्शनिकविचारधारायाः प्राधान्यं, अनन्तरभाविनीषु धार्मिकविचारधारायाः भक्तिधारायाश्च प्रवेशस्सन्दृश्यते । 
उपनिषत्सु शाण्डिल्य - दध्यौच - सनत्कुमार - आरुणि - याज्ञवल्क्य -  उद्दालकरैक्व - प्रतर्दन - अजातशत्रु - जनक - पिप्पलाद् - वरुण - गार्ही - मैत्रेयी - नचिकेतिःप्रभृतीनां बहूनां आचार्याणां महर्षीणाञ्च नाम दृश्यते । केचिदेतेषु सपत्नीकारसापत्याश्च दृश्यन्ते । श्वेतकेतोः पुत्रः आरुणिः, वरुणस्य पुत्रः भृगुः, याज्ञक्ल्क्यः द्विभार्य इत्यादि । औपनिषदास्सिद्धान्तास्सर्वेऽपि पतिपत्नींंसवादशैल्यां पितृपुत्रसंवादशैल्याञ्च प्रतिपादिता इति तु विशेषत अवधेयार्हः विषयः । 
उपनिषच्छब्दनिर्वचनम् - 
कठोपनिषदां प्रस्तावनाभाष्याप्रामाण्यात् , मुण्डकोपनिषदां प्रस्तावनाभाष्यप्रामाण्यात् , केनोपनिषदि ( ४- ३२) "उपनिषदमब्रूत " इति वाक्यस्य भाष्यप्रामाण्यात् , छान्दोग्योपनिषत्स्थाष्टमाध्यायाष्टमखण्ड चतुर्थखण्डिकास्थभाष्यप्रामाण्यात् बृहदारण्यकोपनिषत्प्रस्तावनाभाष्यप्रामाण्याच्च संसारबीजविनाशिनी या विद्या सा उपनिषच्छब्देन मुख्यया वृत्या बोध्यत इति निश्चीयते । तादृशविद्याप्रतिपादकत्वात् लक्षणया ईशावास्यादयो ग्रन्था अप्युपनिषच्छब्देन व्यवह्रियन्ते । 
उपनिषदां सिद्धान्ताः -
प्रायः मुख्यासूपनिषत्सु ब्रह्मस्वरूपं, तत्प्रतिपत्तय उपासनाः, काश्चित् स्वतन्त्राः, काश्चित्कर्माङ्गत्वेन, आख्यायिकासहिताः प्रतिपादिताः । ब्रह्मविद्याप्राप्तेस्साधनानि ब्रह्मविद्याजिज्ञासूनां आवश्यकगुणविशेषाः, कर्ममार्गस्य जटिलत्वम् , ज्ञानमार्गस्य सुगमत्वम् , अनासक्तकर्मपरत्वं वैराग्यञ्चेत्येवमादीनि आत्मज्ञानाप्तेस्साधनानि, न तु तर्कविचारः युक्तिवादा वा इत्यादि प्रतिपादितम्। 
उपनिषत्सु जाग्रत्तत्वानुशीलनपराः नैकविधाः कथाश्श्रूयन्ते । औपनिषदाः मननशीला मुनयः जगतः मूलतत्वानुसन्धाने बद्धश्रद्धाः दृश्यन्ते । प्रापञ्चिकानां नैकविधानां वस्तूनां विभिन्नतायां एकत्वसम्पादकं तत्वं किं स्यात् ? तादृशतत्वलाभोपायः क ? इति शङ्काकुलाः पञ्चभूतानुशीलनमार्गेण आत्मतत्वविचारपराः भूत्वा स्वस्मिन्नेव तादृशं तत्त्वं प्रत्यक्षीचक्रुः । एतादृशस्वानुभवशीलानां तेषां क्रान्तदर्शिन्याः दृष्टेः आन्तरबाह्यजगतोर्न कोऽपि भेदः विषयीबभूव। छान्दोग्योपनिषदीयया इन्द्रविरोचनकथया ( 8 - 7 - 12 ), आरुणिश्वेतकेतुसंवादरूपया न्यग्रोधकथया (6-12) च " तत्त्वमसि "  " अयमात्मा ब्रह्म " इत्येष एव सिद्धान्तः प्रतिपादितः । एवं सप्रपञ्च- निष्प्रपञ्च ब्रह्मस्वरूपर्वर्णना, मनस्तत्वविवेचना, कर्म -  सन्यास - मोक्षसिद्धिान्ताः, भारतीयविविधदर्शनमूलभूतास्सर्वेऽपि सिद्धान्ताः प्रतिपादिताः । 
एवञ्च भारतीयाध्यात्मिकविचारधाराया उपजीव्यत्वात् उपनिषदः प्रस्थानत्रये प्रथमगणनामर्हन्ति । यद्यप्युपनिषद अंसख्यास्तथापि या उपनिषद अद्वैतसिद्धान्तमूलभूताः, याश्च गौडपादशङ्करभगवत्पादादिभिरद्वैताचार्यैः व्याख्यातास्त एवात्र सव्याख्योपव्याख्या वर्णमालाक्रमेण निर्दिश्यन्ते ।। 
१. अथर्वशिखोपनिषत् - 
अस्यामुपनिषदि सकलवेदमूलभूतस्य प्रणवस्य स्वरूपं, तदीयमात्राणां देवतादयः, प्रणववाचकानां ओङ्कारतारकादिपदानां व्युत्पत्तिः, तद्ध्यानध्यातृघध्येयस्वरूपं च इत्येतत् सर्वं निरूप्यते । इयमुपनिषद् आनन्दाश्रममुद्रणालये मुद्रिता । अस्या व्याख्याः - 
नारायणाश्रमिकृता - अथर्वशिखोपनिषद्दीपिका 
अथर्वशिखोपनिषदां व्याख्यात्मकोऽयं ग्रन्थः आनन्दाश्रममुद्रणालये मुद्रितः । अस्य कर्ता श्रीनाथपौत्र रत्नाकरभट्टपुत्रः आनन्दात्मशिष्यः नारायणाश्रम इति ज्ञायते । अनेन विरचितायां " माणडूक्योपनिषद्दीपिकायां " अमुद्रितायां सरस्वतीमहालयपुस्तकालयस्थायां (1556 D.C.T.S.M.L) अानन्दगिरिर्निर्दिष्टः । आनन्दाश्रममुद्रितायां जाबालोपनिषद्दीपिकायां आनन्दात्मा अध्यात्मगुरुरिति निर्दिष्टः। आनन्दात्मा तु शङ्करानन्दस्यापि गुरुरिति नारायणाश्रमस्य कालः त्रयोदशचतुर्दशशतकम् (1275 - 1350 A.D) इति निश्चीयते ।। 
शङ्करानन्दविरचिता - अथर्वशिखोपनिषद्दीपिका 
आनन्दात्मनः विद्यातीर्थस्य च शिष्योऽयं शङ्करानन्दः विद्यारण्यस्य गुरुः त्रयोदशशतकारम्भकालवासी (1275 - 1350 A.D.) इति ज्ञायते । अमुद्रितोऽयं ग्रन्थः सरस्वतीमहालयसूच्यां (1427 TSML) अडयारपुस्तकालये बरोडापुस्तकालये च दृश्यते । शङ्करानन्दमधिकृत्याधिकं सूत्रवृत्तिप्रकरणे प्रकरणग्रन्थप्रकरणे अद्वैताचार्यप्रतिपादनावसरे च प्रतिपाद्यते ।। 
शङ्कराचार्यकृतं - अथर्वशिखोपनिषद्भाष्यम् (१) 
ग्रन्थोऽयं शङ्कराचार्यकृतत्वेन अडयारपुस्तकालयस्थे (30 B 22 ग्र 6 AL) आदर्शपुस्तके दृश्यते । अमुद्रितश्चायं ग्रन्थः । अस्य कर्ता न प्रसिद्धश्शङ्कराचायों भवितुमर्हति । तथा प्रसिद्धेरभावात् । अन्तरङ्गपरीक्षायान्तु कृतायां सर्वथा न शङ्कराचार्यकृतोऽयं ग्रन्थ इत्येव प्रतीयते । यतः - यासामुपनिषदां शङ्कराचार्यैर्व्याख्या कृता तासां न व्याख्या कृता तासां व्याख्याप्रस्तावे नारायणाश्रमिणा " शङ्करोक्त्युपजीविना " इत्युच्यते । यासां न व्याख्या कृता तासां व्याख्याप्रस्तावे " श्रुतिमात्रोपजीविना " इत्येव निर्दिश्यते । दृश्यते चात्र श्रुतिमात्रोपजीविना इति । तस्मादपि कारणात् नेयं व्याख्या शङ्कराचार्यकृता इत्येव प्रतिभाति ।। 
उपनिषद्ब्रह्मेन्द्रकृतं - विवरणम् अडयार पुस्तकालये मुद्रितम् । 
२. अथर्वशिर उपनिषदः - 
अस्यामुपनिषदि देवानां को भवानिति रुद्रं प्रति प्रश्नः, तदुत्तरेण तस्य सर्वात्मकत्वज्ञानेन तन्नतिपूर्वकं बहुधा तस्य स्तुतिः, तत्प्रतिपादकत्वेन प्रसिद्धानां " ओङ्कार - प्रणव - सर्वव्याप्यनन्तादिपदानां निर्वचनं, इत्यादयो विषयाः प्रतिपादिताः । ग्रन्थोऽयमानन्दाश्रममुद्रणालये मुद्रितः । अस्या व्याख्याः - 
नारायणाश्रमकृता - अथर्वशिर उपनिषद्दीपिका 
श्रुतिमात्रोपजीविना नारायणाश्रमिणा रचितेयं दीपिका आनन्दाश्रमुद्रणालये मुद्रिता । अस्य कालादि पूर्ववत् । 
शङ्करानन्दकृता - अथर्वशिर उपनिषद्दीपिका   
ग्रन्थोऽयं आनन्दाश्रममुद्रणालये मुद्रितः । शङ्करानन्दस्य कालादि पूर्ववत् । 
शङ्कराचार्यकृतं - अथर्वशिर उपनिषद् भाष्यम् ?
अमुद्रितोऽयं ग्रन्थः अडयारपुस्तकालये (30 B 22 ग्र 28 AL) लभ्यते । उपनिषद्ब्रह्मेन्द्रकृतं विवरणं अडयार पुस्तकालये मुद्रितम् । 
३. अमृतनादोपनिषत् - 
अस्यामुपनिषदि शास्त्राभ्यासस्य ब्रह्मज्ञानफलकत्वमुपवर्ण्य प्रत्याहारध्यानप्रणायामधारणातर्कसमाध्यभिधेयाङ्गषट्ककस्य योगस्य तत्तदङ्गलक्षणकथनपूर्वकं अभ्यसनप्रकारं फलञ्च निरूप्य प्राणादिवायूतां स्थानवर्णादिकमभिधीयते । मुद्रिता चेयमुपनिषदानन्दाश्रममुद्रणालये । अस्या व्याख्याः - 
(क) शङ्करानन्दविरचिता - अमृतनादोपनिषद्दीपिका 
नारायणविरचिता, उपनिषद्ब्रह्मेन्द्रविरचिताश्च व्याख्यास्सन्ति । मुद्रिता आनन्दाश्रममुद्रणालये अडयार पुस्तकालये च । 
४. अमृतबिन्दूपनिषत् -
अस्यामुपनिषदि मनश्शुद्धिप्रशंसापूर्वकं ब्रह्मज्ञानावाप्तये साधनमुपवर्ण्य मुक्तिस्वरूपं निरूप्य ब्रह्मज्ञानं प्रशस्यते । मुद्रिताचेयमुपनिषदानन्दाश्रममुद्रणालये । अस्या व्याख्याः - 
(क) नारायणाश्रमि विरचिता - अमृतविन्दूपनिषद्दीपिका 
श्रुतिमात्रोपजीविना नारायणाश्रमिणा विराचितेयं दीपिका आनन्दाश्रममुद्रणालये मुद्रिता । अस्य कालादि पूर्ववत् ।। 
(ख) शङ्करानन्दविरचिता - अमृतबिन्दूपनिषद्दीपिका 
व्याख्याचेयं मुद्रिताऽनन्दाश्रममुद्रणालये । कालादि पूर्ववत् । 
(ग) सदाशिवेन्द्रसरस्वतीकृता - अमृतविन्दूपनिषद्दीपिका 
अमुद्रितेयं व्याख्या मद्रासराजकीयप्राचीनहस्तलिखितपुस्तकालये (R 1492 M.G. O. M. L) लभ्यते । अस्य कर्ता कामकोटिपीठाधीशस्य महादेवेन्द्रसरस्वत्याः प्राचार्यः सदाशिवेन्द्रसरस्वती षोडशशतकीयः (1550 - 1650 A.D.) इति ज्ञायते । अनेेन आत्मानात्मविवेकोऽपि रचितः ।। उपनिषद्ब्रह्मयोगिकृतं विवरणमपि मुद्रितम् । 
५. आत्मप्रबोधोपनिषत् - 
अत्र प्रणवस्वरूपं प्रशस्य तत्सहिताष्टाक्षरेण महामन्त्रेण ब्रह्मानुसन्धानं कुर्वत उत्तमलोकाद्यवाप्त्या परमानन्दानुभवप्रकारोऽभिधीयते । ऋक्शाखीयायां अस्यामुपनिषदि आत्माद्वैत - जीवन्मुक्ततादि - प्रतिपादिकाः कारिकाः विशेषत उल्लेखार्हाः । मुद्रिता चेयमुपनिषदानन्दाश्रममुद्रणालये । अस्या व्याख्या :-
(क) नारायणाश्रमिकृता - आत्मप्रबोधोपनिषद्दीपिका 
श्रुतिमात्रोपजीविना नारायणाश्रमिणा रचितेयं व्याख्यानन्दाश्रममुद्रणालये मुद्रिता गद्यभागानामेव विद्यते । 
(ख) शङ्करानन्दविरचिता - आत्मप्रबोधदीपिका 
अमुद्रितोऽयं ग्रन्थः श्रृङ्गगिरिसूच्यां (10. C) दृश्यते ।। उपनिषद्ब्रह्मेन्द्रव्याख्या च मुद्रिता ।
६. आरुणिकोपनिषत् - 
आरुणिप्रजापतिप्रश्नप्रतिवचनरूपायामस्यामुपनिषदि सर्वसङ्गपरित्यागपूर्वकसन्यासाश्रमग्रहणप्रकारमुपवर्ण्य सन्यासिनां धर्मांश्चाभिधायान्ते परमपदावाप्तिरिति प्रतिपादितम् । मुद्रिता चेयमुपनिषदानन्दाश्रममुद्रणालये । अस्य व्याख्याः -
(क) नारायणाश्रमिकृता - आरुणेकोपनिषद्दीपिका 
" नारायणेन रचिता शङ्करानन्दपाठत " इति दीपिकायामस्यां दर्शनात् शङ्करानन्ददीपिकाया अनन्तरं रचितेयं दीपिकेति ज्ञायते । मुद्रिता चेयमानन्दाश्रममुद्रणालये । 
(ख) शङ्करानन्दरचिता - आरुणिकदीपिका 
मुद्रिताचानन्दाश्रमे । उपनिषद्ब्रह्मेन्द्रव्याख्या च मुद्रिता । 
७. ईशावास्योपनिषत् - 
अस्यामुपनिषदि चिदचित्स्वरूपस्य जगतः परमात्माधीनस्वरूपस्थित्यादिमत्वं, आदेहपातं यथाशक्ति ब्रह्मविद्याङ्गभूतकर्मयोगस्यानुष्ठेयत्वं, अविदुषो निन्दनं, परमात्मनः विचित्रानन्तशक्तिमत्वं, ब्रह्मात्मकजगदनुसन्धानस्य फलं, ईशेशितव्यवेदिनः ज्ञानयोगाद्युपदेशः केवलकर्मयोगावलम्बिनां विनिन्दनं भगवद्भक्तिनिष्ठस्यावश्यानुसन्धेयोपदेश इत्यादिकमुपवर्णितं दृश्यते । यजुर्वेदीयेयमुपनिषत् आनन्दाश्रममुद्रणालये (ASS 5) मुद्रिता । अस्या उपनिषदः रचनाकालः क्रिस्तोः पूर्वं सप्तमशतकम् (700 BC) इति विमर्शकाः। अस्याः व्याख्याः, उपव्याख्याश्च- 
(क) शङ्कराचार्यकृतं - ईशावास्योपनिषद्भाष्यम् 
मुद्रितञ्चेदं भाष्यमानन्दाश्रममुद्रणालये (ASS 5)। शङ्कराचार्यकालादि सविस्तरं प्रकरणग्रन्थप्रस्तावे अद्वैताचार्यप्रस्तावे च प्रतिपाद्यते । 
(A) आनन्दगिरिकृता - ईशावास्यभाष्यव्याख्या
व्याख्यायामस्यां "तत्वालोकः " निर्दिष्टः । भास्करमतं खण्डितञ्च । मुद्रितश्चायं ग्रन्थ आनन्दाश्रममुद्रणालये (ASS 5) ।
व्याख्याया अस्याः कर्ता आनन्दज्ञानापराभिध आनन्दगिरिः । सन्यासस्वीकारात्पूर्वं जनार्दन इत्यस्यैव नाम । गुजरातप्रान्तजोऽयं द्वारकास्थ शङ्करपीठाधीश आसीदिति प्रसिद्धिः। आन्ध्रदेशज इति साम्प्रदायिकाः । अनुभूतिस्वरूपाचार्यशुद्धानन्दयोशिशष्योऽयं अखण्डानन्दस्य प्रज्ञानानन्दस्य च गुरुः, कलिङ्गदेशाधिपतेर्नृसिम्हदेवस्य सामयिकस्त्रयोदशशतकीय (1260 - 1320 A.D) इति ज्ञायते । अदसीया अन्ये ग्रन्थाः प्रकरणप्रस्तावे अद्वैताचार्यप्रस्तावे च प्रतिपाद्यन्ते ।। 
(B) शिवानन्दयतिकृतं - ईशावास्यभाष्यटिप्पणम् 
अमुद्रितोऽयं पूर्णग्रन्थः मद्रासराजकीयहस्तलिखितपुस्तकालये (R 3882 M.G.O.M.L) लभ्यते ।
अस्य कर्ता रामनाथविदुष आचार्यस्सप्तदशाष्टादशशतकमध्यवर्ती (1650 - 1750 A.D) शिवानन्दयतिरिति ज्ञायते । अनेनरचित आनन्ददीपाख्य. प्रकरणग्रन्थः अन्यत्र निरूपितः ।। 
(ख) अनन्ताचार्यकृता - वेदार्थदीपिका 
समग्रवेदभागस्य व्याख्यात्मकोऽयं ग्रन्थः । प्रकृतिप्रत्ययविवेचनपूर्वकं सप्रमाणं प्रक्रियां निरूपयन्नयं ग्रन्थ आनन्दाश्रममुद्रणालये मुद्रितः । 
अस्य कर्त काण्वशाखीयः नागदेवभट्टपुत्रः, समग्रवेदभागव्याख्याता अनन्ताचार्यः। अनेन विधानपारिजाताख्यः ग्रन्थः (1625 A.D) काले रचितः । 
(ग) आनन्दभट्टकृतम् - ईशावास्यभाष्यम् ।
ग्रन्थोऽयमानन्दाश्रममुद्रणालये मुद्रितः । ग्रन्थेऽस्मिन् शङ्करानन्दः निर्दिष्टः। अस्य कर्ता जातवेदभट्टजाह्नव्योः पुत्रः वासुदेवपुरी - आत्मावासपूज्यपादशिष्य आनन्दभट्टश्शङ्करानन्दादर्वाचीन इति परं ज्ञायते । 
(घ) उपनिषद्ब्रह्मेन्द्रकृतम् - ईशावास्यविवरणम् 
भावेन वाक्यविन्यासेन च शाङ्करं भाष्यं पूर्णतयानुसरदिदं विवरणं अडयारपुस्तकालये मुद्रितम् ।
अस्य कर्ता प्रथमवासुदेवेन्द्रप्रशिष्यः द्वितीयवासुदेवेन्द्रशिष्यः रामचन्द्रेन्द्रसतीर्थ्यः, कृष्णानन्दगुरुः, अष्टादशशतकापरार्घकालवासी (1765 - 1850 A.D) उपनिषब्रह्मेन्द्रापरनामा रामचन्द्रेन्द्र इति ज्ञायते । 
(ङ) उवटाचार्यकृतम् - ईशावास्यभाष्यम् 
ईशावास्योपनिषदां विवरणात्मकः माध्यन्दिनशाखान्तर्गतस्य समग्रवेदभागस्य व्याख्यात्मकश्चायं ग्रन्थ आनन्दाश्रप्तमुद्रणालये मुद्रितः । अस्य कर्तां आनन्दपुरवास्तव्यस्य वज्रटभदृस्य सूनुरवन्तीपुरवासी भोजराजसामयिकः उवटाचार्य इति शुक्लयजुर्वेदसंहिताभाष्यप्रमाणाज्ज्ञायते । यद्ययं भोजराजः धारानगराधीशः सरस्वतीकण्ठाभरणशृङ्गारप्रकाशकारस्यात् तर्हि तस्य शासनकालः (1010 - 1062 A.D) इति यफिग्राफिका इण्डिकापत्रिकायाः प्रथमभाग (230 page) प्रमाणात् ज्ञायते । तस्मादुवटाचायोंऽपि एकादशशतकीय इति निर्णेतु शक्यते ।। 
(च) गोपालानन्दकृता - ईशावास्यटीका 
अमुद्रितोऽयं ग्रन्थः बरोडापुस्तकसृच्यां (4527 DC BRD) दृश्यते । अस्य कर्ता सहजानन्दशिष्यः गोपालानन्दः माकिं सप्तदशएकोनर्विशतिशतकमध्यवर्तीति ज्ञायते ।। 
(छ) नरसिम्हभट्टकृता - ईशावास्यटीका 
अमुद्रितोऽयं ग्रन्थः मध्यप्रान्तीयबरार्ग्रन्थसूच्यां (481 CCPB) दृश्यते । अस्य कर्ता अद्वैतचन्द्रिकाकारः, नागेश्वररामभद्राश्रमयोश्शिष्यः रघुनाथभट्टपुत्रः किम्मिडि (खिमुण्डिः वंशजस्य जगन्नाथनृपतेस्सामयिकः मिथिलावासी अष्टादशशतकीयः नरसिम्हभट्ट इति ज्ञायते ।।
(ज) ब्रह्मानन्दसरस्वतीकृतम् - ईशावास्यरहस्यम् 
आनुष्ठुभेण छन्दसा घटितैः पद्यै रचितोऽयं ग्रन्थ आनन्दाश्रममुद्रणालये मुद्रितः। अस्य कर्ता ब्रह्मानन्दसरस्वती। यद्ययं गुरुचन्द्रिकाकारस्स्यात्तर्हि नारायणतीर्थंशिष्यस्सप्तदशशतकीय (1600 - 1700 A.D) इति निर्णेतु शक्यते ।।
(झ) भास्करानन्दसरस्वतीकृता - ईशावास्यव्याख्या 
शाङ्करभाष्यानुसारिणीयं व्याख्या भारतीजीवनमुद्रणालये वाराणस्यां मुद्रिता। अस्य कर्ता शाण्डिल्यगोत्रजः कान्यगुब्जब्राह्मणः, गोभिलसूत्री, सामवेदी, कौथुमशाखीयः, पूर्वाश्रमे वेङ्कटमिश्रनामा, मिश्रीलालमिश्रपुण्यमत्योः पुत्रः अनन्तरामपण्डितशिष्यः, पूर्णानन्दस्वामिनो लब्धदीक्षः, मणिराजचौबेदैहित्रः मैथिलोऽपि काशीवासी भास्करानन्दसरस्वती एकोनर्विशतिशतकीयः (1800 -1900 A.D) इति ज्ञायते ।
(ञ) रामचन्द्रपण्डितकृता - ईशावास्यरहस्यविवृतिः 
प्रतिमन्त्र अर्थसंग्राहकश्लोकेन साकं भाष्यार्थसंग्राहकोऽयं ग्रन्थ आनन्दाश्र मुद्रणालये मुद्रितः । अस्य कर्ता माध्यन्दिनशाखाध्यायी, आत्रेयगोत्रजः, श्रीसिद्धपुत्रः सिद्धेश्वरशिष्यः एकोनविंशतिशतकवासी (1817 A.D) रामचन्द्रपण्डित इति ज्ञायते ।। 
(ट) श्रीधरानन्दकृतः - ईशावास्यविवेकः 
अमुद्रितोऽयं ग्रन्थः अडयारपुस्तकालये (33 H.L आ 2 A.L.) लभ्यते । 
(ठ) सच्चिदानन्दाश्रमिकृता - ईशावास्यदीपिका 
अमुद्रितोऽयं ग्रन्थः बरोडापुस्तकसूच्यां (1969) दृश्यते । अस्याः कर्ता नृसिम्हाश्रमिशिष्यस्सच्चिदान्दाश्रमीति ज्ञायते । यद्ययं नृसिम्हाश्रमी अद्वैतदीपिकाकारस्स्यात्तर्हि षोडशशतकीयोऽयमिति निर्णेतुं शक्यते ।। 
(ड) शङ्करानन्दकृता - ईशावास्यदीपिका 
मुद्रितश्चायं ग्रन्थ आनन्दाश्रममुद्रणालये । एनमधिकृत्य विस्तरेण अद्वैताचार्यप्रकरणे पूर्वञ्चोपपादितम् । 
(ढ) सदानन्दकृतः - ईशावास्यचिन्तामणिः
अमुद्रितोऽयं ग्रन्थ उज्जैनूसूच्यां (1947 ) दृश्यते । अस्य कर्ता सदानन्द इति ज्ञायते । कोऽयं सदानन्दः? किं काश्मीरी सदानन्दः ? उत सदानन्दसरस्वती ? आहोस्वित् सदानन्दव्यासवर इत्यत्र न किमपि प्रमाणमुपलभामहे ।। 
%ईशावस्योपनिषद् 
ऐतरेयोपनिषत् - 
ऋग्ब्राह्मणान्तर्गतेयमुपनिषत् । अध्यायत्रिकरूपाया अस्या उपनिषदः खण्डषट्कं विद्यत इति कृत्वोपनिषदियम् आत्मषट्क मित्यपि प्रसिद्धा । अस्यामुपनिषदि त्रिष्वध्यायेषु सर्वशरीरस्य ब्रह्मणः जगत्सृष्ट्यादिवर्णनम्, बद्धस्य जीवात्मनो वैराग्योदयाय गर्भप्रवेशदुःखानुभवादिनिरूपणं, निश्श्रेयसावाप्त्यर्थं परमात्मोपासनञ्च प्रतिपाद्यते । अस्या रचनाकाल 600 - 500 B.C इति विमर्शकाः । मुद्रिता चेयमुपनिषदानन्दाश्रममुद्रणालये । अस्या व्याख्याः - 
१. शङ्कराचार्यकृतम् - ऐतरेयमाष्यम् 
मुद्रितश्चायं ग्रन्थ आनन्दाश्रममुद्रणालये वाणीविलासमुद्रणालये च । 
(क) अभिनवनारायणेन्द्रकृता - भाष्यव्याख्या 
अमुद्रितोऽयं ग्रन्थ मद्रासराजकीयहस्तलिखितपुस्तकालये (R. 1475 MGOML) अडयारपुस्तकालये च लभ्यते । 
अस्याः कर्ता कैवल्येन्द्रसरस्वतीप्रशिष्यः ज्ञानेन्द्रसरस्वतीशिष्य अग्निहोत्रभट्टसतीर्थ्यः परमशिवेन्द्रसरस्वतीगुरुः षोडशसप्तदशशतकमध्यवासी 1550 - 1650 A.D अभिनवनारायणेन्द्रसरस्वतीति विज्ञायते । अदसीया अन्येऽपि ग्रन्थास्तत्तत्प्रकरणे वर्णिताः ।।
(ख) आनन्दगिरिकृता - भाष्यटिप्पणी 
मुद्रितश्चांय ग्रन्थ आनन्दाश्रममुद्रणालये । ग्रन्थेऽस्मित् ( Page 28) विद्यारण्यकृता दीपिका निर्दिष्टा । विद्यारण्यश्चानन्दगिरेरर्वाचीन इति प्रसिद्धिः । 
तस्मात् न प्रसिद्धोऽयमानन्दगिरिरस्य कर्ता । परन्तु यः कोऽप्यन्य एव स्यात् । मुद्रितग्रन्थे शुद्धानन्दादिगुरुवन्दनापि न दृश्यते ।। 
(ग) ज्ञानामृतयतिकृता - भाष्यटिप्पणी 
अमुद्रितोऽयं ग्रन्थः मद्रासराजकीयहस्तलिखितपुस्तकालये (D. 332 MGOML) लभ्यते । ग्रन्थादस्मात् सायणाचार्यसामयिकोऽयमिति ज्ञायते । उत्तमामृत - आनन्दाख्ययोश्शिष्यः, सायणाचार्यसामयिकः नैष्कर्म्यसिद्धिव्याख्याविद्यासुरभिकर्ता चतुर्दशशतकीय (1350 A.D.) ज्ञानामृतयतिरिति ज्ञायते ।। नृसिम्हाश्रमकृता भाष्यव्याख्या चास्तीती श्रूयते । सीतानाथतर्कभूषणेन आधुनिकेन कृता 'शङ्करकृपा' नाम्नी व्याख्या च मुद्रिता ।  
(घ) उपनिषद्ब्रह्मेन्द्रकृतम् - भाष्यविवरणम् ।
अमुद्रितोऽयं ग्रन्थ अडयारपुस्तकालये (36 F. 14 ग्र 79 A.L) लभ्यते अस्य कर्ता उपनिषद्ब्रह्मेन्द्रः वासुदेवेन्द्रशिष्यः अष्टादशशतकीय इति प्रतिपादि तमन्यत्र ।। 
२. भास्करानन्दकृता - ऐतरेयदीपिका (व्याख्या)
मुद्रितश्चायं ग्रन्थः भारतीजीवनमुद्रणालये वाराणस्याम् । अस्याः कर्ता एकोनविंशतिशतकीयः भास्करानन्दः पूर्वमुपपादितः ।
३. विद्यातीर्थकृता - ऐतरेयदीपिका 
अमुद्रितोऽयं ग्रन्थः वेङ्कटेश्वरपुस्तकालये तत्सूच्याञ्च लक्ष्यते । अस्याः कर्ता रुद्रप्रश्नभाष्यकर्ता नृसिम्हतीर्थशिष्यः भारतीतीर्थविद्यारण्यगुरुः, कृष्णानन्दभारती - ब्रह्मानन्दभारती-विद्यारण्यानां प्रगुरुः, विद्याशङ्करश्शङ्करानन्द इत्यपरनामा त्रयोदशशतकीयः ( 1228 - 1333 A.D) विद्यातीर्थ इति ज्ञायते । विद्यारण्यकृताया दीपिकाया दीपिकाया इयं भिद्यते न वा इति न निश्चेतुं शक्यते । 
४. विद्यारण्यकृता - ऐतरेयदीपिका 
मुद्रितश्चायं ग्रन्थ आनन्दाश्रममुद्रणालये । अस्य कर्ता विद्यारण्यः श्रीकण्ठाचार्यशङ्करानन्दभारतीतीर्थानां शिष्यः, विद्यातीर्थप्रशिष्यः, रामकृष्ण - कृष्णानन्दभारतीब्रह्मानन्दभारतीगुरुः पञ्चदश्यादिग्रन्थप्रणेता चतुर्दशशतकीयः (1300 - 1400 A.D) इति ज्ञायते । ग्रन्थस्यास्य भाष्यमिति नाम सरस्वतीमहालयस्थे पुस्तके (1451 TSML) दृश्यते । 
५. शङ्करानन्दकृता - ऐतरेयदीपिका । ग्रन्थोऽयं शृङ्गगिरिसृच्यां दृश्यते ।। 
%एेतरेयोपनिषद् 
काठकोपनिषत -
नचिकेतोमृत्युंसवादरूपायामस्यामुपनिषदि प्रत्येकं वल्लित्रितययुते अध्यायद्वितये सर्वमेधाख्ये यज्ञे दत्तसर्वस्वदक्षिणस्य वास्रवसनाम्नो महर्षेः पुत्रेण नचिकेतोनाम्ना निजनिर्बन्धादेव पित्रा मृत्यवे प्रदत्तेन मृत्युलोकमधिगम्य त्रिस्रो रात्रीरुपोषितवता मृत्युप्रसादादेव निजपितुश्शान्तचित्तत्वं, नचिकेता इति स्वनामधेयोपेतस्याग्नेः ब्रह्मविद्याङ्गतया विज्ञानं सविस्तरं मृत्युनैवोपदिष्टं प्रत्यगात्मपरमात्मनोस्स्वरूपज्ञानं च इत्येतद्वरत्रितयं पुत्रपौत्रचिरायुस्सम्पत्यादिकं प्रत्याचक्षाणेन समधिगतमिति प्रतिपाद्यते । अस्य उपनिषदो रचनाकालः 500 - 400 BC इति विमर्शकः । मुद्रिता चेयमुपनिषदानन्दाश्रममुद्रणालये । अस्या व्याख्याः - 
1. शङ्कराचार्यकृतम् - कठोपनिषद्भाष्यम् 
सव्याख्योऽयं ग्रन्थः आनन्दाश्रममुद्रणालये मुद्रितः ।
(क) अच्युतकृष्णानन्दकृता - भाष्यटीका 
अमुद्रितेयं काठकशाङ्करभाष्यटीका महीशूरहस्तलिखितपुस्तकालये (1278 ग्र 22 प My GML) लभ्यते ।  
अस्याः कर्ता रामानन्दप्रशिष्यस्स्वयम्प्रकाशशिष्य अद्वैतानन्दसरस्वत्याश्च शिष्यः तैत्तरीयव्याख्यावनमालाकारः सप्तदशाष्टादशशतकीय (1650 - 1750 A.D,) अच्युतकृष्णानन्द इति ज्ञायते ।
(ख) आनन्दगिरिकृता - भाष्यटीका । मुद्रितश्चायं ग्रन्थ आनन्दाश्रममुद्रणालये । 
(ग) गोपालबालयतिकृतम् - कठवल्लीभाष्यविवरणम् । मुद्रिता चेयं व्याख्या आनन्दाश्रममुद्रणालये ।।
अस्याः कर्ता बालगोपालयतिरित्यपरनामा जगन्नाथाश्रमिशिष्यः, नृसिम्हाश्रमिसतीर्थ्यः, स्वयम्प्रकाशगुरुः, मनीषापञ्चकव्याख्यामधुमञ्जरीकारः षोडशशतकीयः 1500 - 1600 A.D. गोपालबालयतिरिति ज्ञायते ।।
(घ) शिवानन्दयतिकृता - कठभाष्यटिप्पणी 
अमुद्रितोऽयं ग्रन्थ मद्रासराजकीयहस्तलिखितपुस्तकालये ( R 3882 MGOML) लभ्यते । अस्य कर्ता रामनाथविदुष आचार्यः सप्तदश - अष्टादशशतकमध्यवर्ती 1650 - 1751 A.D. शिवानन्द इति पूर्वमुपपादितम् ।
(ङ) श्रीधरशास्रिपाठककृता - भाष्यव्याख्या - बालबोधिनी 
मुद्रिताचेयं व्याख्या पूनानगरे । अस्याः कर्ता त्र्यम्बकशास्त्रिपुत्रः डेक्कान संस्कृतकलाशालाप्रधानाध्यापकः विंशतिशतकीयः 1850 - 1920 A.D. श्रीधरशास्त्रीति ज्ञायते ।।
2. अभिनवनारायणेन्द्रकृता - कठोपनिषद्व्याख्या 
अमुद्रितोऽयं ग्रन्थ औधसूच्यां दृश्यते । अस्य कर्ता कैवल्येन्द्र ज्ञानेन्द्र योशिशष्य अग्निहोत्रभट्टसतीर्थ्यः षोडशशतकीयः 1550 - 1650 A.D. अभिनवनारायणेन्द्र इति ज्ञायते । 
3. उपनिषद्ब्रह्मेन्द्रकृता - कठवल्लीव्याख्या 
मुद्रिताचेय व्याख्या अडयारपुस्तकालये । अस्य कर्तारं उपनिषद्ब्रङ्मेन्द्रमधिकृत्य पूर्वमेवोपपादितम् ।। 
4. दिगम्बरानुचरकृता - कठवल्लीव्याख्या - अर्थप्रकाशिका 
मुद्रिता चेयंव्याख्या आनन्दाश्र मुदणालये । अस्याः कर्ता एकनाथसामयिकः महाराष्ट्रीयसाधुर्दिगम्बरानुचर इत्येव ज्ञायते । 
5. नारायणगजपतिराजकृता - कठोपनिषद्व्याख्या - द्विमतप्रकाशिका
अमुद्रितोऽयं ग्रन्थः शाङ्करभाष्यानुसारिणी व्याख्यां द्वयोर्मतयोरन्यतमाञ्च स्वीकरोति। ग्रन्थोऽयं बरोडासूच्यां (10058 BRD) दृश्यते । अस्य कर्ता नाराराववेङ्काम्बयोः पुत्रः जमदग्निगोत्रजः क्षत्रियः नारायणगजपतिराय इति परं ज्ञायते ।।
6. नारायणाश्रमिकृता - कठवल्लीदीपिका । मुद्रितश्चायं ग्रन्थ आनन्दाश्रममुद्रणालये ।  
7. भास्करानन्दसरस्वतीकृता - कठवल्लीव्याख्या 
मुद्रितश्चायं ग्रन्थः भारतीजीवनमुद्रणाले वाराणस्याम् । भास्करानन्दश्च पूर्वमुपपादितः ।।
8. शङ्करानन्दकृता - कठवल्लीदीपिका 
मुद्रितश्चायं ग्रन्थः आनन्दाश्रममुद्रणालये । शङ्करानन्दश्च उपपादितपूर्वः । 
%कठोपनिषद

केनोपनिषत् - 
खण्डचतुष्टयेन विभक्तायामस्यां केनोपनिषदि प्रथमखण्डे मनःप्रभृतिकरणप्रवृत्तेश्चेतनायत्तत्वेन तत्प्रवर्तकचेतनविशेषस्य परब्रह्मत्वमभिधीयते । द्वितीयखण्डे ब्रह्मणोऽपरिच्छिन्नत्वात् तस्य कार्त्स्न्येन विज्ञातुमशक्यत्वेन योग्यैकदेशज्ञानादेव इष्टावाप्तिरित्यभिधीयते । तृतीयखण्डे - परं ब्रह्म कार्त्स्न्येन ज्ञातुमशक्यमित्यस्मिन्नर्थे ब्रह्मानुप्रवेशलब्धासुरविजयानां स्वेषामेव जेतृत्वाभिमानवतां देवानां गर्वापनोदनमुखेन स्वमाहात्म्यं यक्षरूपधरेण ब्रह्मणा प्रकाशितमित्याख्यायिकाभिधीयते । चतुर्थखण्डे यक्षरूपधरे ब्रह्मणि देवानामुपदेशाय हैमवतों देवीमाकाशेऽवस्थाप्या अन्तर्हिते सति तया देव्या इन्द्रः, तस्माच्चान्येऽपि आघिदैवताध्यात्मात्मरूपां उपास्याब्रह्मणो रूपमजानन् । अस्या रचनाकाल (500 -400 B.C) इति विमर्शशीलाः । मुद्रिता चेयमुपनिषत् सव्याख्याऽनन्दाश्रममुद्रणालये ।। 
अस्या व्याख्या :-
 (1) शङ्कराचार्यकृतम् -
 1. पदभाष्यम् 
 2. वाक्यभाष्यम् 
 मुद्रितञ्चेदं भाष्यद्वयमानन्दाश्रममुद्रणालये । किमेतत् भाष्यद्वयमपि शङ्कराचार्यकृतम् ? उत नेति विमृश्यते - 
 भाष्यद्वयस्यापि एककर्तृकत्वे अर्थैक्यं अभिप्रायसाम्यञ्चापेक्ष्यते । तत्तु नात्रोपलभ्यते । वैलक्षण्यस्य दर्शनात् । वैलक्षण्यानि - 
(क) पदभाष्ये "नाहं मन्ये सुवेदेति" 2-10  मन्त्रे अहमिति पाठं गृहीत्वा व्याख्यातम् । वाक्याभाष्ये तु अवधारणार्थकं अह इति पाठं गृहीत्वा व्याख्यातम् । यद्युभयोर्भाष्ययोरेकः कर्ता स्यात्तर्हि मया पदभाष्येऽयं पाठो व्याख्यात इत्याद्युच्येत । न तथोक्तमिति कर्तुर्वैलक्षण्यं स्पष्टम् ।
(ख) पदभाष्ये 1-1 मन्त्रे " प्रेषित " इत्यस्यार्थः प्रेरितमिति वर्णितः। वाक्यभाष्ये तु तस्यैव " प्रेषितमिव " इति पूर्वविलक्षणोपमार्थ उपवर्णितः । 
(ग) पदभाष्ये " श्रोत्रस्य श्रोत्रम् " 1-2 मन्त्रे श्रोत्रमाद्याः प्रथमास्सन्तीति साधितम् । वाक्यभाष्ये तु तत्रैव प्रथमास्सन्तीति साधितम् । वाक्यभाष्ये तु तत्रैव प्रथमाद्वितीयेति विभक्तिद्वयं साध्यते ।
(घ) 1-3 मन्त्रे, 2-9 मन्त्रे, च अवतरणिकाभेदस्सन्दृश्यते । 1-4 मन्त्रे पदवाक्यभाष्ययोरक्षरार्थसाम्येऽपि अभिप्रायभेदो दृश्यते । "भूतेषु भूतेषु विचित्य" 2-13 मन्त्रे विचित्येत्यस्य पदवाक्यभाष्ययोर्भिन्नार्थता च दृश्यते ।   
एवमादिभिर्वैलक्षण्यदर्शनैः पदभाष्यवाक्यभाष्ययोर्निर्मातारौ भिन्नाविति प्रतीयते । एवं भिन्नकर्तृकत्वे सिद्धे कतरत् पूर्वम् किं पदभाष्यम् ? उत वाक्यभाष्यम् ? इति विमर्शः कर्तव्यः।
केनोपनिषदां पदभाष्यं भगवत्पूज्यपादानां आदिशङ्कराणां कृतिः। वाक्यभाष्यन्तु तदनुयायिनां श्रीशङ्करपीठमधिष्ठितानां केषाञ्चिदाचार्याणामित्यस्माकं मतिः। तत्रेमानि कारणानि - 
पदभाष्यापेक्षया वाक्यभाष्ये श्रुत्यन्तराणां प्रमाणार्थनिर्देशोऽल्पीयान् । यत्र तत्र पूर्वपक्षखण्डनेश्वरसिध्यादिर्लेखो भूयान् दृश्यते । आद्याचायों हि उन्मूलमप्रासङ्गिकं उपनिषद्भाष्ये न कदापि लिखति । उपनिषदां बहूनि वाक्यानि आद्याचार्याणां ग्रन्थे सम्भवन्ति। तस्मात् बाक्यभाष्यं आद्याचार्येतरकृतमिति भाति । 
"प्रतिबोधविदितम् " 2 - 22 इति मन्त्रे वाक्यभाष्ये प्रतिबोधमित्यस्यार्थत्रितयं वर्णितम् । तत्र द्वितीयव्याख्या सद्योमुक्तिं प्रत्याययति । सद्योमुक्तिश्च न शास्त्रीया । एतादृशमशास्त्रीयं मतमाद्याचार्याः कदापि न प्रतिपादयन्तीति वाक्यभाष्यमाद्याचार्ये तरकृतम् । पदभाष्ये यथाक्रमं शब्दशो विवरणम् । वाक्यभाष्ये तु सुगमशब्दविशिष्टो मन्त्रो न व्याख्यायते । आद्याचार्याणां सर्वत्रेयं परिपाठी यत् ते यथाक्रमं सर्वान् शब्दान्‌ निर्दिशन्ति, विशिष्टशब्दान् व्याख्यान्ति च ।
वाक्यभाष्ये चतुर्थखण्डे षष्ठमन्त्रे "सेवन्ते स्म" इति पदं निरर्थकं प्रयुक्तम् । एवं निरर्थकशब्दप्रयोग आद्याचार्याणां ग्रन्थे न सञ्जाघटीति इदं तत्कृत न सम्भवति । 
अत्र केचन साम्प्रदायिका इत्थं समादधते यत् - यथाऽन्यासामुपनिषदां मन्त्रान् विषयवाक्यत्वेन गृहीत्वा तेषां व्याख्यानं आचार्यैः सूत्रभाष्ये कृतं तथा केनोपनिषत्स्थमन्त्राणां न कृतम् । तत्रत्यानां मन्त्राणां विषयवाक्यत्वाभावात् । तेन पदभाष्यं कृत्वाऽपरितुष्यन्तो भगवत्पादा पुनर्वाक्यभाष्यमकुर्वन्निति । अतएव "पदशो व्याख्यायापि न तुतोष भगवान् भष्यकार" इत्यानन्दज्ञानोक्तिस्सङ्गच्छत इति च । परं तन्न विचारसहम् । यतो यथा केनोपनिषत्स्थमन्त्राः न विषयवाक्यत्वेन सूत्रभाष्ये व्याख्याता तथा ईशावास्यमन्त्रा अपि विषयवाक्यत्वेन ब्रह्मसूत्रे न व्याख्याताः । ततश्च ईशावास्योषनिषदि भगवत्पूज्यपादैः भाष्यान्तरं कृतं स्यात्, तत्तु नोपलभ्यते । तेन पूर्वोक्तं समाधानं स्थवीयः। किञ्चैतत्समाधानं पूवोंक्तवैलक्षण्यविवरणं न सहत इति तु अन्यदेतत् । तस्मात् पदभाष्यं प्राचीनं, वाक्यभाष्यं पश्चात्तनमिति सिध्यति । 
अयं वाक्यभाष्यकृत् शङ्करानन्दात्प्राक्तनः। शङ्करानन्दो विद्यारण्यगुरुः । तेन शङ्करानन्दकालः त्रयोदशशतकं भवितुमर्हति । शङ्करानन्देन च "व्याकरिष्ये पदाध्वना" इति तत्कृतदीपिकायां कथ्यते । आनन्दगिरिणा तु ऐतरेयोपनिषद् भाष्यव्याख्यायां विद्यारण्यीयैतरेयोपनिषद्दीपिकास्थवाक्यस्य निर्देशात् शङ्करानन्दादर्वाचीनेन भाव्यम् । तस्मात् भाष्यद्वये व्याख्याकरणं सुसङ्गतमेव । 'अपरितुष्य' न्निति अवतरणिकाप्रदानमपि तन्मत्यनुसारं न विरुध्यते च । 
तस्मात् - शाङ्करपीठाधिष्ठितेनान्येन केनचित् एकं भाष्यं केनोपनिषदि कृतम् । तस्य वाक्यभाष्यमिति सज्ञा कृता । परं कालवशात् अज्ञात्वा तत्कर्तारं भाष्याध्यायिभिरस्य वाक्यसंज्ञां दृष्ट्वा पूर्वभाष्यस्य पदसज्ञा कृता भवेत्  इति निश्चीयते ।।
भाष्यव्याख्याः - 
(क) अभिनवनारायणेन्द्रकृता - भाष्यटीका 
अमुद्रितोऽयं ग्रन्थ औंधसूच्यां XXI 26 दृश्यते । अभिनवनारायणेन्द्रः पूर्वमुपपादितः । 
(ख) आनन्दगिरिकृता - माष्यटिप्पणी । मुद्रितश्चायं ग्रन्थ आनन्दाश्रममुद्रणालये । उपपादितपूर्वश्चायमानन्दगिरिः ।
(ग) शिवानन्दयतिकृता - भाष्यटिप्पणी 
अमुद्रितोऽयं ग्रन्थः मद्रासराजकीयहस्तलिखितग्रन्थागारे D 392 लभ्यते । सरस्वतीमहालयसूच्याञ्च दृश्यते । अस्य कर्ता शिवानन्दयतिः पूर्वमुपपादितः ।। 
(घ) श्रीधरशास्त्रिकृता - भाष्यव्याख्या बालबोधिनी 
मुद्रितश्चायं ग्रन्थः पूनानगरे । अस्य कर्ता श्रीधरशात्री डेक्कानसंस्कृतकलाशालाप्रधानाध्यापकः त्र्यम्बकशास्त्रिपुत्रः विंशतिशतकीय इति पूर्वमुपपादितम् ।। आधुनिकेन सच्चिदानन्देन्द्रसरस्वत्या कृता काचन भाष्यव्याख्या मुद्रिता च विद्यते ।
(2) उननिषद्व्रह्मेन्द्रकृतम् - केनोपनिषद्विवरणम् । मुद्रितश्चायं ग्रन्थ अदृयार पुस्तकालये । अस्य कर्ता उपनिषद्बह्मेन्द्र पूर्वमुपपादितः ।
3.  कृष्णलीलाशुककृता - केनोपनिषद्व्याख्या, शङ्करहृदयङ्गमा मुद्रितश्चायं ग्रन्थः मद्रासराजकीयहस्तलिखितपुस्तकालयसंस्कृतग्रन्थमालायाम् । ग्रन्थेऽस्मिन् क्षीरस्वामी बुद्धाः शङ्कराचार्याश्च निर्दिष्टाः । 
अस्याःकर्ता कृष्णलीलाशुकः । अयमेव बिल्वमङ्गलाचार्य इति प्रसिद्धः। यद्येवमस्य पिता दामोदरः । माता नीलीनाम्नी । ईशानदेवशिष्योऽयं कृष्णलीलाशुकः केरलदेशवासी कोच्चिनूसाम्राज्यवर्तिनि तिरुच्चूरन्तर्गते तेक्कमठे लब्धवासः 
1220 - 1300 A.D. काले उवासेति प्राच्यभाषामहासम्मेलनपत्रिकायाः नवमे भागे दृश्यते ।
यद्ययं कृष्णलीलाशुकः सरस्वतीकण्ठाभरणव्याख्यापुरुषकार - कृष्णलीलामृतादिकर्ता स्यात् तर्हि अस्य कालः द्वादशत्रयोदशशतकोत्तरार्धपूर्वार्धावधिक इति निर्णेतुं शक्यते । यतः पुरुषकारे हेमचन्द्र उद्धृतः । हेमचन्द्रकालस्तु 1166 - 1220 A.D. । पुरुषकारस्य पंक्तयः देवराजकृतायां निघण्डुटीकायां उद्धृताः । देवराजकालस्तु 1293 - 1343 A.D. । एवञ्च तयोर्मध्यपाती  पुरुषकारकर्तायं कृष्णलीलाशुकः 1168 - 1293 A.D. काले आसीदिति निर्णेतुं शक्यते ।
सीतारामजयराम जोशी तु कृष्णलीलाशुकं एकादशशतकीयं 1100 A.D. स्वीये संक्षिप्तसाहित्येतिहासे वदति । 
श्रीचिह्ननामा कश्चन ग्रन्थः कृष्णलीलाशुकेन रचितः। तत्रायं श्लोकस्सन्दृश्यते । "श्रीपद्मपादमुनिवर्यविनेयवर्गश्रीभूषणं मुनिरसौ कविसार्वभौमः" इति । पद्यादस्मात् पद्मपादशिष्यः कृष्णलीलाशुक इति वक्तुं शक्यते । केनोपनिषद्व्याख्यायां (Page 10)  दृश्यमाना "ब्रह्म भूयं गते पूर्वे शंकरे कृत्स्नवेदिनि । पूर्वे च तादृशे" इति श्लोकीया "पूर्वे च तादृशे " इति पंक्तिरपि पद्मपादाचार्यं निर्दिशतीव लक्ष्यते । एवञ्च वेदान्ते पद्मपादाचार्यशिष्योऽयं कृष्णलीलाशुकः नवमशतकीय इति सिध्यति । अत एव "इण्डियन् हिस्टारिकल रेव्यू" नामकपत्रिकायां सप्तमे भागे (I.H.R VII Page 334) कृष्णलीलाशुकः नवमशतकीय इति प्रतिपादितम् । केचित्तु वङ्गदेशीयं वदन्ति। 
4.  नारायणाश्रमिकृता - केनोपनिषद्दीपिका 
मुद्रितश्चायं ग्रन्थ आनन्दाश्रममुद्रणालये । बालकृष्णानन्दसरस्वत्या कृता काचन व्याख्या अमुद्रिता लन्दननगरकार्यालयपुस्तकालये विद्यते ।
5. भास्करानन्दकृता - केनोपनिषद्व्याख्या । मुद्रितश्चायं ग्रन्थः भारतजीवनमुद्रणालये वाराणस्याम् । 
6. शङ्करानन्दकृता - केनोपनिषद्दीपिका । मुद्रितश्चायमपि ग्रन्थ आनन्दाश्रममुद्रणालये ।। 
% केनोपनिषद् 

कैवल्योपनिषत् - 
आश्वलायनपरमेष्ठिप्रश्नोत्तररूपायामस्यामुपनिषदि ब्रह्मज्ञानेनाविद्याविनाशः, सन्यासिनां प्रशंसना, सगुणतिर्गुणोपासनाक्रमश्च प्रतिपाद्यन्ते। मुद्रिता चेयमुपनिषदानन्दाश्रममुद्रणालये । अस्याः व्याख्याः - 
1. उपनिषद्ब्रह्मेन्द्रकृतम् - कैवल्योपनिषद्विवरणम् । मुद्रितोऽयं ग्रन्थ अडयार पुस्तकालये । 
2. नारायणाश्रमिकृता - कैवल्योपनिषद्दीपिका 
श्रुतिमात्रोपजीविना नारायणाश्रमिणा कृतेयं दीपिका आनन्दाश्रममुद्रणालये मुद्रिता । 
3. शङ्करानन्दकृता - कैवल्योपनिषद्दीपिका 
शङ्करानन्दकृतत्वेन मुद्रितेयं दीपिका आनन्दाश्रममुद्रणालये । परन्तु बरोडा-अडयार-नासिकादिपुस्तकालयस्थेषु आदर्शपुस्तकेषु विद्यारण्यकृता इति ग्रन्थान्तपुष्पिकाया ज्ञायते ।। अस्या तैलनाम्नी व्याख्या (Mad: uni RAS. 12 C.) विद्यते ।
4. सदाशिवब्रह्मेन्द्रसरस्वतीकृता - कैवल्योपनिषद्दीपिका 
अमुद्रितोऽयं ग्रन्थः मद्रासराजकीयहस्तलिखितपुस्तकालये लभ्यते ( R 1492d MGOML) ।
अस्याः कर्ता परमशिवेन्द्रसरस्वतीशिष्यः, ब्रह्मतत्त्वप्रकाशिका - आत्मविद्याविलासादिकारः, नल्लादीक्षितगुरुरष्टादशशतकीयः सदाशिवब्रह्मेन्द्रसरस्वतीति सूत्रवृत्तिप्रकरणे प्रकरणग्रन्थप्रस्तावे चोपपादितम् ।। 
कौषीतक्युपनिषत् - 
ऋक्शाखीयायामस्यामुपनिषदि फलसङ्गाद्यभिसन्धियुतानां धूमादिमार्गेण चन्द्रमसं प्राप्य पुनरावृत्तानां संसृतौ भ्रमणम् , विद्याकर्मविहीनानां कीटपतङ्गाद्यात्मना क्लेशानुभूतिः, गुर्वनुग्रहाधिगतविद्यानां ब्रह्मचारिगृहिप्रभृतीनां अर्चिरादिमार्गेण उत्तमां गर्ति प्रपन्नानां ब्रह्मणस्सारूप्यर्माधगतानां ब्रह्मालङ्कारपरिमण्डितानां परब्रह्मानुभवप्रकारः, इन्द्रप्रतर्दनाख्यायिकया स्वदेहभोगस्य प्रारब्धपरिक्षयैकपर्यवसायित्वकथनम् ,प्राणोपासनाभिधानम् , अजातशत्र्त्वाख्यायिकया आदित्य - चन्द्र - अग्निविद्युत् - स्तनयित्वादिषु उपासनासु विसंवादप्रदर्शनपूर्वकं परमात्मस्वसूपतद्विज्ञानादिप्रकाशनमित्येतत्सर्वं प्रतिपाद्यते । अस्या रचनाकालः 200 - 100 BC  इति विमर्शकाः। मुद्रिता चेयमुपनिषत् सव्याख्याऽनन्द्राश्रममुद्रणालये ।। अस्या व्याख्या :-
1. उपनिषद्ब्रह्मेन्द्रकृतम् - कौषीतकिविवरणम् । अडयारपुस्तकालये मुद्रितोऽयं ग्रन्थः। 
2. नागरनारायणकृता - ज्ञानमाला । रामेन्द्रसरस्वतीशिष्येण कृतेेयं व्याख्या अमुद्रिता बरोडापुस्तकालये तत्सूच्यां (3827 BRD) च दृश्यते । 
3. नारायणाश्रमिकृता - दीपिका 
4. शङ्करानन्दकृता - दीपिका । मुद्रिते चेमे दीपिके आनन्दाश्रममुद्रणालये ।।
क्षुरिकोपनिषत् - 
यजुश्शाखीयायामस्यामुपनिषदि स्वयम्मूपदिष्टंं योगाभ्यसनक्रममनुष्ठातुर्जातवैराग्यस्य योगिनः प्राणायामपूर्वकनाडीभेदनद्वारा उत्तमलोकावाप्तिरभिधीयते ।। मुद्रिता चेयमुपनिषदुपनिषदां समुच्चये आनन्दाश्रममुद्रणालये ।। अस्या व्याख्या :-
1. उपनिषद्ब्रह्मेन्द्रकृतम् - विवरणम् । अडयार पुस्तकालये मुद्रितम् ।
2. नारायणाश्रमिकृता - दीपिका । आनन्दाश्रममुद्रणालये मुद्रिता ।
3. शङ्करानन्दकृता - दीपिका । आनन्दाश्रममुद्रणालये मुद्रिता । 
गोपालतापिन्युपनिषत् - 
इयमुपनिषदाथर्वणी पूर्वॉत्तरभागद्वयेन विभक्ता दृश्यते । तत्र पूर्वभागे - मुनिब्राह्मणप्रश्नोत्तररूपेण मृत्योरपि भयजनकः परमो देवश्श्रीकृष्ण इति प्रतिपाद्य तदुपासनोपयोगिनस्संसारोत्तारकाः तद्देवताकाः केचिन्मन्त्रा आम्नाताः उत्तरभागे श्रीकृष्णस्य नित्यब्रह्मचारित्वं दुर्वाससो मुनेर्दूर्वाशित्वं प्रतिपाद्य दुर्वाससा गान्धर्वी गाम्न्याः उपदेशव्याजेन श्रीकृष्णस्य परब्रह्मत्ववर्णनपूर्वकं तन्माहात्म्यममभिधाय सप्तपुरीमध्यगतब्रह्मगोपालपुरीवर्णनपूर्वकं श्रीभगवन्माहात्म्यं  तदुपासनाप्रकारश्च प्रतिपाद्यन्ते । अस्या व्याख्याः - 
1. उपनिषद्ब्रहेन्द्रकृतम् - विवरणम् । अडयारपुस्तकालये मुद्रितम् । 
2. नारायणाश्रमिकृता - दीपिका (ASS 29) 
3. विश्वेश्वरकृता - दीपिका 
स्वयम्प्रकाशशिष्योऽयं विश्वेश्वर इति परं ज्ञायते । अमुद्रितोऽयं ग्रन्थस्सरस्वतीमहालये तत्सृच्यां (1474 DCTSML), बरोडापुस्तकालये तत्सूच्यां (234 DCBRD) च दृश्यते ।।
छान्दोग्योपनिषत् - 
सामशाखीयायामस्यां छान्दोग्योपनिषदि प्रथमप्रपाठके त्रयोपदशभिः खण्डैः उद्गीथस्यानेकधा उपासनानि. स्तोभाक्षरोपासनम् , तत्फलम् , सामादिज्ञानं तत्फलञ्च प्रतिपाद्यते । द्वितीयप्रपाठके चतुर्विंशतिखण्डैः - साम्नः पञ्चधा सप्तधा चोपासनम् , तत्फलञ्च कथ्यते । तृतीयप्रपाठके एकोनविंशतिखण्डैः आदित्योपासनम् , मधुविद्या, गायत्रीविद्या, शाण्डिल्यविद्या, तासां फलानि चाभिधीयन्ते । चतुर्थप्रपाठके सप्तदशखण्डैः विद्यादानग्रहणविचारः उपकोसलविचारः , ब्रह्मविदो गतिविचारश्च कृतः । पञ्चमप्रपाठके त्रयोविंशतिख्णडैः - प्राणस्य मुख्यत्वविचारः, पञ्चाग्निविद्या, सगुणब्रह्मविद्याफलं आख्यायिकामुखेन वैराग्योदयाय संसारगतिः, वैश्वानरविद्या चाभिधीयते । षष्ठप्रपाठके षो़डशभिः खण्डैराख्यायिकामुखेन ब्रह्मविद्या विचारिता । अनेकधा आत्मतत्वविचार उपदेशश्च कृतः। सप्तमप्रपाठके पञ्चदशभिः खण्डैः भूमविद्या प्रपञ्च्यते । अष्टमप्रपाठके पञ्चदशखण्डैर्दहरविद्या तत्फलं, प्रत्यगात्मविद्या तत्फलं च प्रकाश्यते । अस्या रचनाकालः 700 BC इति विमर्शकाः । सव्याख्येयमुपनिषन्मुद्रिता आनन्दाश्रममुद्रणालये । अस्या व्याख्या :- 
1. द्रविडाचार्यकृतम् - छान्दोग्यभाष्यम् 
2. ब्रह्मनन्दिकृतम् - छान्दोग्यवाक्यम् 
इमौ द्वावपि ग्रन्थौ नोपलभ्येते, तथापि तत्र विभिन्नग्रन्थेषु उद्धरणात् तयोर्ग्रन्थयोस्सत्ता ऊह्यते । एतावधिकृत्य विस्तरेण प्रतिपादितं अद्वैताचार्यप्रकरणे । 
3. शङ्कराचार्यकृतम् - छान्दोग्यभाष्यम् (ASS 14) 
(क) अभिनवनारायणेन्द्रकृता - भाष्यदीका । अमुद्रितोऽयं ग्रन्थ मद्रासराजकीयहस्तलिखितपुस्तकालये (R 1662 MG OML) अडयारपुस्तकालये च लभ्यते ।।
(ख) आनन्दगिरिकृता - भाष्यव्याख्या (ASS 14)
4. उपनिषद्ब्रह्मेन्द्रकृतम् - छान्दोग्यऋजुविवरणम् । अडयारपुस्तकालये मुद्रितोऽयं ग्रन्थः ।  
5.  त्यागराजशास्त्रिकृतः- सद्विद्याविलासः सव्याख्यः 
(राजुशास्त्रिकृत ) (रसानुभूतिरिति नामान्तरं व्याख्यायाः) 
छान्दोग्यषष्ठाध्यायसंग्रहात्मकः पद्यात्मकश्चायं ग्रन्थः भाष्यतद्व्याख्यानानुसारं विस्पष्टार्थप्रकाशकः चिदम्बरनगरे ग्रन्थाक्षरेषु मुद्रितः । 
अस्य कर्ता आधुनिकोऽयं प्रकाण्डविद्वान् न्यायेन्दुशेस्वरकर्ता मन्नार्गुडि राजुशास्रीति विख्यातः, त्यागराजपौत्रः. अप्पादीक्षितपुत्रः, भारद्वाजगोत्रजः, यज्ञस्वामिदीक्षितपिता, नारायणसरस्वती - रघुनाथशास्रि - गोपालशास्रिशिष्यः एकोनविंशतिशतकीयः (1815 - 1904 A.D.) त्यागराजदीक्षितः । अदसीया अन्ये ग्रन्था अन्यत्र प्रतिपादिताः ।। 
6. नित्यानन्दाश्रमिकृता -  मिताक्षरा 
मुद्रिता चेयं मिताक्षरानाम्नी छान्दोग्यव्याख्या बम्बर्इसंस्कृतमुद्रणालये । क्वचिदादर्शपुस्तकेषु बरो़डा - अडयारपुस्तकालयस्थेषु ग्रन्थस्यास्य "अर्थप्रकाशिका" इत्यपि नामान्तरं दृश्यते । 
अस्याः कर्ता पुरुषोत्तमाश्रमशिष्यः, नित्यानन्दाश्रमीति ज्ञायते । एनमधिकृत्य नाधिकं ज्ञातुं पार्यते । मुद्रितपुस्तकात् कालनिर्णययोग्यानि प्रमाणानि नेपलभ्यन्ते । परं तु उज्जयिनीपुस्तकालयस्थे 1373  संख्याके आदर्शपुस्तके 1746  संवत् इति दृश्यते । यदि स लेखककालस्स्यात् तर्हि सप्तदशशतकात् (1690 AD)  पूर्वतन इति सिध्यति । यदि वा स एव ग्रन्थकर्तुः कालस्स्यात्तर्हि सुवर्णे सौगन्ध्यायितम् । 
7.  शङ्करानन्दकृता - छान्दोग्यदीपिका । मुद्रिता चेयमानन्दाश्रममुद्रणालये ।। 
%छान्दोग्योपनिषद 
जाबालोपनिषत् - 
पैप्पलादिजाबालप्रश्नोत्तररूपायामस्यामुपनिषदि पशुपदार्थनिरूपणपूर्वकं पशुपतिशब्दार्थं प्रकाश्य विभूतिधारणात् ज्ञांन मोक्षसाधनं भवतीत्युक्त्वा तत्प्रसङ्गात् भस्ममहिमानमुपवर्ण्य भस्मधारणप्रकारञ्च प्रकाश्य भस्मधारणप्राशस्त्यं च प्रतिपादितम् । सामशाखीयेयमुपनिषदानन्दाश्रममुद्रणालये मुद्रिता । अस्या व्याख्या :- 
1. नारायणाश्रमिकृता - जाबालोपनिषद्दीपिका 
व्याख्यायामस्यां नारायणाश्रमस्य गुरुः "आनन्दात्मा" इति निर्दिष्टः । "आनन्दात्मानमध्यात्मगुरुं देवं नतोऽस्म्यहम् " Page 275 इति दृश्यते । 
श्रुतिमात्रोपजीविना नारायणाश्रमिणा रचितेयं दीपिका आनन्दाश्रममुद्रणालये मुद्रिता । अनेन बृहदारण्यकस्यापि दीपिका कृता इति जाबालोपनिषद्दीपिकाया Page 275  ज्ञायते । नारायणाश्रमी श्रीनाथपौत्रः भट्टरत्नाकरपुत्रः आनन्दगिरेरर्वाचीन आनन्दात्मशिष्य इति पूर्वंमपि प्रतिपादितम् । 
2. वल्लभेन्द्रसरस्वतीकृता - जाबालोपनिषद्व्याख्या मोक्षलक्ष्मीविलासः ।
असुद्रितोऽयं ग्रन्थः अडयारपुस्तकालये तत्सूच्यां (38 E 38 दे 152 AL) बरोडासूच्यां (1701 BRD) लन्दननगरस्थभारतकार्यालयपुस्तकालये तत्सूच्याञ्च ( 2433 DCIOL) दृश्यते । अस्य कर्ता वासुदेवेन्द्रशिष्यः काशीवासी वल्लभेन्द्रसरस्वती (1801 AD) कालात्प्राचीन इति ज्ञायते । 
3. अमुद्रितोऽयं ग्रन्थस्सरस्वतीमहालयसूच्यां (1484 DCTSML) दृश्यते । आनन्दाश्रमे मुद्रिता च ।  
4. श्रीनिवासशास्त्रिकृतम् - जाबालोपनिषद्भाष्यम्  
जाबालोपनिषद्दीपिकापरनामायं ग्रन्थः ब्रह्मविद्यामुद्रणालये चिदम्बरक्षेत्रे ग्रन्थलिप्यां मुद्रितः । अस्य कर्ता राजुशास्त्र्यपरनामकस्य त्यागराजशास्त्रिणः शिष्यः रामयज्वसीताम्बयोः पुत्रः कौण्डिन्यगोत्रजः द्रविडदेशवासी चोलदेशीयः एकोनविंशतिशतकवासी (1894 AD) श्रीनिवासशास्त्रीति ज्ञायते ।। 
तैत्तरीयोपनिषत् - 
तैत्तरीयोपनिषद्यस्यां संहित्याख्यायां शिक्षावल्यां अङ्गभूतोपासनोपदेशपूर्वकं परविद्या प्रस्तूयते । आनन्दवल्यां परमतत्वहितपुरुषार्थप्रतिपादनं दृश्यते । वारुण्याख्यायां भृगुवल्यां तपोवधूतकल्मषमनसो ब्रह्मप्रतिपत्तिरित्यभिधीयते । अस्याः रचनाकाल (600 - 500 BC) इति विमर्शकाः । आनन्दाश्रममुद्रणालये मुद्रिताच । अस्या व्याख्या :- 
1. शङ्कराचार्यकृतम् - तैत्तरीयभाष्यम् (ASS 12)
क. अच्युतकृष्णानन्दकृता - भाष्यव्याख्या वनमाला 
वनमालाख्या तैत्तरीयशाङ्करभाष्यव्याख्याऽनन्दाश्रमे वाणीविलासमुद्रणालये च मुद्रिता । अस्याः कर्ता अद्वैतानन्दस्वयम्प्रकाशसरस्वतीशिष्यः रामानन्दप्रशिष्यः सिद्धान्तलेशसंग्रहव्याख्याकृष्णालङ्कारकर्ता सप्तदशशतकापरार्धवासी (1650 - 1750 AD) अच्युतकृष्णानन्दतीर्थः ।
(ख) अभिनवनारायणेन्द्रकृता - भाष्यटिप्पणी । अमुद्रितोऽयं ग्रन्थः महीशरपुस्तकालये लभ्यते । 
(ग) आनन्दगिरिकृता - भाष्यटिप्पणी (ASS 12) ज्ञानामृतयतिकृतं व्याख्यानं आनन्दगिरिव्याख्यायाः अस्तीति ज्ञायते ।
(घ) सुरेश्वराचार्यकृतम् - तैत्तरीयभाष्यवार्तिकम् 
शाङ्करतैत्तरीयकभाष्यस्य पद्यमयी वार्तिकाख्या व्याख्या आनन्दाश्रममुद्रणालये (ASS 13) वाराणस्याञ्च मुद्रिता । अस्याःकर्ता शङ्कराचार्यशिष्येषु अन्यतमः पद्मपादादिसतीर्थ्यः अष्टमशतकवासी (800 AD) सुरेश्वराचार्यः । 
पूर्वाश्रमे शोणानदीतिरवासी पञ्चगौडान्तर्गतः कुमरिलभट्टजामाता पूर्वकाण्डप्रवर्तकः मण्डनमिश्र इति ख्यातः विश्वरूप एव सन्यासस्वीकारादनन्तरं सुरेश्वर इति प्रसिद्ध इति साम्प्रदायिकाः वदन्ति । 
"जागोप् महाशयेन नैष्कर्म्यसिद्धिभूमिकायां मण्डनमिश्रसुरेश्वरविश्वरूपाणामैक्यं स्वीक्रियते । सप्तदशशतकीयेन बालकृष्णानन्दसरस्वत्या कृते शारीरकमीमांसाभाष्यवार्तिके मण्डनमिश्रसुरेश्वरविश्वरूपाणामैक्यमेवोपवर्णितम् । विद्यारण्यै र्विववरणप्रमेयसंग्रहे बृहदारण्यकभाष्यवार्तिकादुद्धरणं दत्तम् । तत्रापि विश्वरूपशब्देन सुरेश्वर एव निर्दिष्टः ।।"
दासगुप्तस्तु सुरेश्वरविश्वरूपावभिन्नौ मण्डनमिश्रस्तु भिन्न इति (HIP Vol .II) वदति । हिरियण्णामहाशयस्तु (J. R. A. S. 1924)  जर्नल आफ रायल आसियारिक सोसाइटि पत्रिकायां सुरेश्वरः मण्डनादन्य इति प्रतिपादयति । 
कुप्पुस्वामिशास्रिणस्तु ब्रह्मसिद्धिभूमिकायां सुरेश्वरब्रह्मसिद्धिकारयोस्सिद्धान्तगतभेदमुपवर्ण्य ब्रह्मसिद्धिकारः सुरेश्वरादन्य इति प्रतिपादयन्ति। 
संक्षेपशारीरककर्ता सर्वज्ञात्मा सुरेश्वरशिष्य इति तु प्रसिद्धिः । सुरेश्वरश्च देवेश्वरशब्देन सर्वज्ञात्मना निर्दिष्ट इति तु पण्डितपरम्परागता वार्ता । 
श्रीकण्ठशास्त्री (I.H.Q. val. XIV) एवं (J.O.R. 1937) पत्रिकायां नायं सर्वज्ञात्मगुरुस्सुरेश्वर इति प्रतिवादयति । देवेश्वरस्त्वन्य इति च प्रतिपादयति । चिन्तामणिमहोदयेन च अङ्गीक्रियते । निरूपितञ्चैतदस्माभिर्विस्तरेण सर्वज्ञात्मप्रसङ्गे । 
श्रीकण्ठशास्त्रिणा प्रदर्शितायां शृङ्गगिरिगुरुपरम्परायां नित्यबोधघनाभिधः नित्यबोधाचार्य एव सुरेश्वरशिष्य इति निर्दिश्यते । नित्यबोधाचार्यकालश्च 773 - 848  इति च निर्दिश्यते । तस्मात् सर्वज्ञात्मन एव नित्यबोधाचार्य इति नामान्तरस्वीकारोऽपि सुष्ठु लगति । 
कुप्पुस्वामिशास्त्रिसिद्धान्तानुसारं, श्रीकण्ठशास्त्रिप्रदर्शितशृङ्गगिरिपरम्परानुसारञ्च सुरेश्वरकालः सप्तमाष्टमशतकमिति (620 - 777 A.D) इति सिध्यति । 
(A)  आनन्दगिरिकृता - तैत्तरीयवार्तिकव्याख्या । ( ASS. 13) 
(B)  लिङ्गनसोमयाजिकृतम् - कल्याणविवरणम् 
शाङ्करभाष्यार्थप्रकाशकः वार्तिकभावं सुलभं बोधयन् अयं ग्रन्थः शारदामुद्रणालये भटनविल्लीनगरे मुद्रितः । 
अस्य कर्ता श्रीरमणराज्यलक्ष्म्योः पुत्रः कल्याणानन्दभारतीशिष्यः आत्रेयगोत्रजः गुण्टूराख्यान्ध्रदेशवासी पञ्चदशीव्याख्याता विंशतिशतकीयः  (1900 - 1950 A.D.) लिङ्गनसोमयाजीति ज्ञायते ।। 
(C) विश्वानुभवकृता - तैत्तरीयभाष्यवार्तिकसङ्गतिः 
अमुद्रितोऽयं ग्रन्थः मद्रासराजकीयहस्तलिखितपुस्तकालये (R 2929 M.G. O. M. L.) लभ्यते । अस्य कर्ता प्रत्यग्बोधभगवतशिशष्यः विश्वनुभव इति परं ज्ञायते ।।
2. अद्वैतानन्दतीर्थकृता - तैत्तरीयदीपिका 
"तैत्तरीयोपनिषत्तात्पर्यदीपिका" नाम्ना प्रसिद्धोऽयं ग्रन्थः ब्रह्मविद्यामुद्रणालये मुद्रितः । अस्य कर्ता दक्षिणदेशवासी सदानन्दतीर्थशिष्यः, पूर्वश्रमे माध वसूरि-महालक्ष्म्योः पुत्रः हारीतगोत्रजः, आधुनिकोऽयं अद्वैतानन्द इति ज्ञायते । अदसीया ब्रह्मसूत्रवृत्तिरपि प्रतिपादिता ।। 
3. उपनिषद्ब्रह्मेन्द्रकृतम् - तैत्तरीयविवरणम् ।  (A. L. S.)
4. तारकब्रह्माश्रमिकृताः - तैत्तरीयोपनिषत्सारसंग्रहः
अमुद्रितोऽयं ग्रन्थ अडयार पुस्तकालये लभ्यते । अस्य रचयिता दक्षिणदेशीयः रामचन्द्राश्रमिशिष्यः, कल्पतरुपरिमलसंग्रहकर्ता अष्टादशशतकीयः (1700 - 1800 A.D.) तारकब्रह्माश्रमीति ज्ञायते ।। 
5. नारायणाश्रमिकृता - दीपिका । (ASS. 12)  
6. बालकृष्णानन्दसरस्वतीकृतम् - तैत्तरीयविवरणम् 
अमुद्रितोऽयं ग्रन्थः मद्रासराजकीयहस्तलिखितपुस्तकालये (R 383 M.G.O.M.L.) लभ्यते । अस्य कर्ता पद्यमयशारीरकमीमांसाभाष्यवार्तिककर्ता काञ्चीमण्डलान्तर्गतवेदपुरीवासी अभिनवद्रविडाचार्यबिरुदभूषितः, श्रीधरानन्दसरस्वत्याः प्राप्तदीक्षः, गौडब्रह्मानन्दसरस्वत्याश्शिष्यः, महादेवकैलासेशगुरुः सप्तदशशतकीयः (1600 - 1700 A.D.) बालकृष्णानन्दसरस्वतीति ज्ञायते ।
7. भास्करानन्दकृता - तैत्तरीयकव्याख्या । मुद्रितश्चायं ग्रन्थः भारतीजीवनमुद्रणालये वाराणस्याम् ।।
8.  विज्ञानात्मकृता - तैत्तरीयकविवृतिः 
अमुद्रितोऽयं ग्रन्थः मद्रासराजकीयपुस्तकालये (R. 3208 M.G.O.M.L) लभ्यते । अस्य कर्ता परमानन्दमस्करीत्यपरनामा ज्ञानोत्तमस्य शिष्यः चित्सुखाचार्यसतीर्थ्यः द्वादशशतकवासी (1100 - 1200 A.D.) विज्ञानात्मा इति ज्ञायते ।।
9. विद्यारण्यकृता - लघुदीपिका । अमुद्रितोऽयं ग्रन्थः मद्रासरजकीयपुस्तकालये (R. 1968 M.G.O.M.L) बरोडासूच्याञ्च लभ्यते । अस्य कर्ता विवरणप्रमेयसंग्रहकारः पञ्चदशीनिर्माता विद्यारण्यः ।।
(क) कृष्णानन्दश्रीरामशिष्यकृता - लघुदीपिकाटीका 
विद्यारण्यकृतलघुदीपिकासारसंग्रहकारी ग्रन्थोऽयं अमुद्रितः सरस्वतीमहालये (1494 T.S.M.L) मद्रासराजकीयपुस्तकालये (D. 515 M.G.O.M.L) च लभ्यते । 
10. वेङ्कटनाथकृता - तैत्तरीयकटीका । ग्रन्थोऽयं वेङ्कटनाथकृतायां ब्रह्मानन्दगिरिनाम्न्यां भगवद्गीताव्याख्यायां (461 V.V.P. Edn.) ग्रन्थकारेणैव निर्दिष्टंः । 
11. शङ्करानन्दकृता - दीपिका । (ASS 12)
12. सीतारामकृतम् - तैत्तरीयकव्याख्यानम् आगमामृतम् 
अमुद्रितोऽयं ग्रन्थः मद्रासराजकीयहस्तलिखितपुस्तकालये (D.514 M.G.O.M.L) लभ्यते । ग्रन्थोऽयं शिक्षावल्याः परं लभ्यते । अस्य कर्ता उत्तरमायूर क्षेत्रवासी कौण्डिन्यगोत्रजः वीरमाम्बागर्भजः अच्चन्नासूरिपुत्रः शिष्यश्च सप्तदशशतकीयस्तीतारामशास्त्रीति ज्ञायते ।। 
13. अज्ञातकर्तृका - तैत्तरीयकव्याख्या । ग्रन्थोऽयं मद्रासराजकीयहस्तलिखितपुस्तकालये (D 508 M.G.O.M.L) लभ्यते ।। 

%तैत्तरीयोपनिषद् 

नारायणोपनिषत् -
अस्यामुपनिषदि भगवतो नारायणस्य सङ्कल्पात् प्राणेन्द्रियभूतानामुत्पत्तिः, तस्मादेव ब्रह्मरुद्रेन्द्रादीनां स्वस्वकार्येषु प्रवृत्यादिकञ्च ऋग्वेदशिरोधीतमभिहितम् । तथा यजुर्वेदशिरोधीतं नारायणस्य ब्रह्मशिवाद्यन्तर्यामित्वजगद्वयाप्त्यादिकम् , सामवेदशरोधीतं नारायणाष्टक्षरमन्त्रस्वरूपतदध्ययनफलादिकं, अथर्वशिरोधीतं प्रणवोपासनाभ्यां दिव्यलोकगमनादिकं प्रतिपादितम् । यद्यपीयमुपनिषद्विशिष्टाद्वैतसिद्धान्तमृलमिति प्रथा वर्तते तथापि अद्वैतिभिरियमुपनिषदद्वैतसिद्धान्तप्रदर्शिनी व्याख्याता ।। 
1. उपनिषद्ब्रह्मेन्द्रकृतम् - विवरणम् (ALS)
2. नारायणाश्रमिकृता - दीपिका । ग्रन्थो बरोडासूच्यां दृश्यते (11529 BRD)
3. माधवाचार्यकृतम् - भाष्यम् । सरस्वतीमहालये (1504 D.C.T.S.M.L) दृश्यते ।
4. विज्ञानात्मकृतम् - विवरणम् । सरस्वतीमहालये तत्सूच्यां (1505 D.C.T.S.M.L) दृश्यते ।
5. शङ्करानन्दकृता - दीपिका । सरस्वतीमहालये तत्सूच्यां(1508 D.C.T.S.M.L) बरोडासूच्याञ्च (98191 BRD) दृश्यते ।।
नृसिम्हतापनीयोपनिषत् - 
पूर्वोत्तरभागद्वयेन विभक्तायामस्यामुपनिषदि पूर्वभागे श्रीनृसिम्हविद्यायाः साख्यायिकाप्रदर्शनमवताणम् , सामसम्बन्धित्वेन पृथिव्यादिलोकानामुपासनम् , साङ्गसामोपासनम् , सामाङ्गदेवतानामग्न्यादिदेवतानामुपासनम् , आरूढयोगस्य नरकेसरिणः उपासनाकारविशेषादिनिरूपणपूर्वकं सामचतुष्टयोद्धारप्रकाशनम् , साख्यायिकमधिकारिविशेषणान्तरकथनम् , प्रणवचतुर्मात्राव्यूहोपासनम् , हृदयाद्यङ्गपञ्चकोपन्याससहितम् , पादाक्षरसंख्यापूर्वककृत्स्नमूलमन्त्राक्षरसंख्यानिरूपणम् , मूलमन्त्रगतपदोद्धारपूर्वकं तदर्थप्रकाशनम् , मूलमन्त्रजपोपयोगिशक्तिबीजादिनिरूपणम् , मूलमन्त्राङ्गमन्त्राणामुपदेशः, प्रणवसावित्रीमहालक्ष्मीनृसिम्हगायत्रीरूपाणां अङ्गमन्त्राणां व्याख्यानपूर्वकं द्वात्रिंशन्नृसिम्हव्यूहस्तुतिमन्त्राणां प्रदर्शनम् , नृसिम्ह विद्याफलञ्च प्रतिपादितम् । 
उत्तरस्मिन् भागे प्रणवमहिमानुवर्णनपूर्वकं प्रणवोपासनाप्रकारं प्रकाश्यतन्मुखेन ब्रह्मणस्स्वरूपादिकं विस्तरेणोपपादितं दृश्यते ।। मुद्रिता चेयमुपनिषदानन्दाश्रममुद्रणालये अडयार पुस्तकालये च । 
1. उपनिषद्ब्रह्मेन्द्रकृतम् - विवरणम् (ALS)
2. गौडपादमुनिकृतम् - नृसिम्हतापिनीभाष्यम् 
अमुद्रितोऽयं ग्रन्थः मद्रासराजकीयहस्तलिखितपुस्तकालये (D. 581 D. 582 MGOML) उज्जैन् सूच्यां नासिकसूच्यां च दृश्यते । किमयं माण्डूक्याकारिकाकार उत अन्य इति न ज्ञायते ।
गौडपादमधिकृप्य माण्डूक्योपनिषत्प्रस्तावे उपपादयिष्यते ।। 
3. नारायणाश्रमिकृता - दीपिका । अमुद्रितोऽयं ग्रन्थः बरोडासूच्यां (11490 BRD) लभ्यते ।
4. विद्यारण्यकृता - दीपिका । उत्तरतापिन्याः परं विद्यारण्यकृता दीपिका अमुद्रिता मद्रासराजकीयपुस्तकालये (R. 3615 MGOML) बरोडापुस्तकालये च लभ्यते । विद्यारण्यः उपपादितचरः ।। 
5. शङ्कराचार्यकृतम् - नृसिम्हतापिनीभाष्यम् 
ग्रन्थोऽयं मद्रासराजकीयपुस्तकालये (D. 581-583 MGOML) सरस्वती महालये (1509 DC. TSML) बरोडासूच्यां 269 च लभ्यते । शङ्कराचार्येण नृसिम्हतापनीयस्य व्याख्या कृता इत्यत्र माध्वाचार्येण अथवा अभिनवकालिदासेन कृतं शङ्करदिग्विजयकाव्यमपि प्रमाणं भवति । ग्रन्थोऽयं आनन्दश्रामे (ASS 30) मुद्रितः ।
6. शङ्करानन्दकृताच दीपिका विद्यते । 
प्रश्नोपनिषत् - 
अस्यामुपनिषदि सृष्टिप्रकारः, प्राणस्य प्राधान्यम् , प्राणविद्योपासनम् , परविद्याविचारः, प्रणवोपासनम् , षोडशकलपुरुषविचारश्व विविच्य प्रतिपादिताः । अर्थर्ववेदीया इयमुपनिषत् । अस्या रचनाकाल (500 - 400 B.C.) इति विमर्शकाः । मुद्रिता चानन्दाश्रममुद्रणालये । 
1. शङ्कराचार्यकृतम् - प्रश्नोपनिषद्भाष्यम् (ASS 8) वाणीविलासमुद्रणालये आनन्दाश्रमे च मुद्रितम् । 
(क) अभिनवनारायणसरस्वतीकृता - भाष्यव्याख्या 
अमुद्रितेयं व्याख्या मद्रासराजकीयहस्तलिखितपुस्तकालये (D. 621 MGOML)  अडयारपुस्तकालये बरोडासूच्याञ्च (6944 DCBRD) दृश्यते । 
(ख) आनन्दगिरिकृतम् - भाष्यटिप्पणम् ।
यद्यपि आनन्दगिरिकृता इति व्याख्येयमानन्दाश्रमे मुद्रिता तथापि आनन्दगिरेरर्वाचीनाया विद्यारण्यकृतदीपिकायास्तत्रोल्लोखात् नेयं प्रसिद्धानन्दगिरिकृता भवितुमर्हति । शुद्धानन्दादिगुरुस्मरणमपि न तत्र दृश्यते । मुद्रितग्रन्थस्य अन्तिमपुष्पिकायां विद्यमानः " कैवल्येन्द्रशिष्यज्ञानेन्द्रगुरुचरणसेविनारायणेन्द्रसरस्वतीविरचितं प्रश्नोपनिषद्भाष्यविवरणं समाप्तम् " इति पाठभेदोऽपि आनन्दगिरिकृतत्वे संशयमुत्पादयति ।। 
(ग) शिवानन्दयतिकृतम् - भाष्यटिप्पणम् । अमुद्रितोऽयं ग्रन्थः मद्रासराजकीयहस्तलिखितपुस्तकालये (D.389) अडयारपुस्तकालये च लभ्यते ।।
(घ) अज्ञातकर्तृकम् - भाष्यटिप्पणम् । अमुद्रितोऽयं ग्रन्थः मद्रासपुस्तकालये (D. 620 MGOML) लभ्यते । 
2. उपनिषद्ब्रह्मेन्द्रकृतम् - प्रश्नोपनिषद्विवरणम् । (ALS)
3. नारायणाश्रमिकृता - दीपिका । (ASS 8)
4. शङ्करानन्दकृता - दीपिका । (ASS 8)

%प्रश्नोपनिषद् 

बृहदारण्यकोपनिषत् - 
शुक्लयजुश्शाखान्तर्गतायां षडध्यायीपरिमितायां अस्यां उपनिषदि षड्भिर्ब्राह्मणैर्विभक्ते प्रथमेऽध्याये अश्वमेधीयाश्वस्य ब्रह्मदृष्ट्यौपयिकविश्वरूपत्वोत्कीर्तनम् , उद्गातरि मुख्यप्राणदृष्टिकथनम् , नारायणतदुपासनतत्प्राप्तीनां परमतत्वहितपुरुषार्थत्वकथनम् , मनुष्यादिसृष्टिकथनम् कृत्स्नब्रह्मोपासनाभिधानम् , सर्वविज्ञानप्रश्नोत्तराभिधानम् , आत्मलोकोपासनकथनम् , पांक्तयज्ञादिकथनम् , अविद्याकार्योपसंहार इत्येतत्सर्वं दृश्यते । 
ब्राह्मणषट्कपरिमिते द्वितीयेऽध्याये कार्यकरणात्मकमूतस्वरूपनिर्धारणं मूर्तामूर्तब्रह्मस्वरूपाभिधानम् , "आत्मैवेदं सर्व" मिति प्रतिज्ञातार्थनिर्वहणम् , विद्याप्रवर्तकाचार्यवंशकथनम् इतीमे विषयाः प्रतिपाद्यन्ते । 
नवभिर्ब्राह्मणैर्विभक्ते तृतीयेऽध्याये याज्ञवल्क्यमहिमानुवर्णनपूर्वकस्तेन सह कहोलगार्ग्युद्दालकशाकल्यादिसंवादः परिदृश्यते ।
ब्राह्मणषट्कपरिमिते चतुर्थाध्याये जनकं प्रति याज्ञवल्क्येन जीवस्वरूपदेहान्तराप्तिक्रमतदुपेयपरमफलतत्साधनादिकमभिहितं दृश्यते । 
पञ्चदशभिर्ब्रह्मणैर्विभक्ते पञ्चमाध्याये प्रणवोपासना सर्वोपास्त्यङ्गभूतशमदमादयः, ब्रह्मोपासनाङ्गं हृदयोपासनम् , विद्युदादिषु ब्रह्मदृष्ट्योपासनानि, ब्रह्मविदां गतिः, तपसःउपासनम् , अन्नप्राणयोर्ब्रह्मत्वेनोपासनम् , उक्थजुस्सामदृष्ट्या प्राणोपासनानि, गायत्र्युपहितब्रह्मोपासनम् , आतिवाहिकादित्याग्न्योः प्रार्थनामन्त्राश्च अभिधीयन्ते ।
ब्राह्मणैः पञ्चभिरुपेते षष्ठेऽध्याये वाक्च्क्षुःश्रोत्रमनोरेत आत्मकप्राणोपासनानि, वागादीनामुत्क्रमणप्रवेशक्रमः, प्रवहणनाम्ना राज्ञा गौतमाय पञ्चाग्निविद्योपदेशः, पञ्चाग्निविदां तदविदाञ्च गतिः, कर्मानुष्ठातुः सति दोषे प्रायश्चित्तानां कामनाविशेषेषु कर्तव्यानाञ्च विधिः ओदनपाककालजातकर्मनामकरणानि पुत्रवत्वप्रशंसा, एतद्विद्याप्रवचनवंशश्चाभिधीयन्ते ।
शुक्लयजुश्शाखीयशतपथब्राह्मणान्तर्गतमिदं उपनिषत्काण्डं बृहत्वात् आरण्यपाठ्यत्वात् ब्रह्मविद्याप्रतिपादकत्वाच्च बृहदारण्यकोपनिषदिति व्यपदिश्यते । अध्यायष्टकपरिमितमप्येतत् प्रथमद्वितीययोर्ब्रह्माप्रतिपादकत्वात् तृतीयस्याश्वमेधविषयकत्वेऽपि ब्रह्मदृष्टिविधिरूपतया ब्रह्मात्मकत्वप्ततिपादनपरतया च ब्रह्मसम्बन्धित्वेन तृतीयाध्यायप्रभृति अध्यायषट्कपरिमितमेव उपनिषदिति व्यवह्नियते ।
अस्या रचनाकालः (600 B.C)  इति विमर्शकाः । मुद्रिता चेयनुपनिषदानन्दाश्रममुद्रणालयेऽन्यत्र च । एतत्सम्बद्धाः व्याख्यानुव्याख्यारूपाः ग्रन्थाः - 
1. शङ्कराचार्यकृतम् - बृहदारण्यकभाष्यम् (ASS 16)
(क) आनन्दगिरिकृता - भाष्याव्याख्या (ASS 16) व्याख्यायामस्यां भर्तृप्रपञ्चकृता बृहदारण्यकव्याख्या निर्दिष्टा या चाद्याविधि न लभ्यते ।। 
(ख) महादेवेन्द्रसरस्वतीकृता - भाष्यतात्पर्यदीपिका ।
द्वितीयाध्यायचतुर्थब्राह्मणसारात्मकोऽयं ग्रन्थस्सरस्वतीमहालये ( 1539 - 1540 DC. TSML) लभ्यते । अस्य कर्ता गोपालबालयोगिप्रशिष्यः स्वयम्प्रकाशानन्दसरस्वत्याः सतीर्थ्यः सुदर्शनेन्द्रेत्यपरनामा महादेवसरस्वती षोडशसप्तदशशतकवासी (1600 - 1700 A.D.) इति ज्ञायते । एनमधिकृत्याधिकं प्रकरणग्रन्थनिरूपणवसरे प्रतिपादितम् ।
(ग) शिवानन्दयतिकृतम् - भाष्यटिप्पणम् । अमुद्रितोऽयं ग्रन्थः मद्रासराजक्रीयपुस्तकालये (R. 3882 MGOML) लभ्यते । 
(घ) सुरेश्वराचार्यकृतम् - बृहदारण्यकभाष्यवार्तिकम् । मुद्रितश्चायं ग्रन्थ आनन्दाश्रममुद्रणालये (ASS 16) सव्याख्याः । 
(A) आनन्दगिरिकृता - भाष्यवार्तिकव्याख्या । (ASS 16)  शास्त्रप्रकाशिकानामायं ग्रन्थः आनन्दाश्रमे (ASS 16) मुदितः । सुरेश्वराश्चोपपादितपूर्वः । 
(B) आनन्दपूर्णविद्यासागरकृता - भाष्यवार्तिकव्याख्या। 
न्यायकल्पलतिका नाम्नीयं व्याख्या अमुद्रिता मद्रासराजकीयपुस्तकालये (R. 5283 MGOML) बरोडासूच्याञ्च (8938 B.C.BRD) दृश्यते । अस्य कर्ता खण्डनखण्डखाद्यव्याख्याता श्वेतगिर्यभयानन्दशिष्यः, पुरुषोत्तमानन्दगुरुः विद्यासागरापरनामा आनन्दपूर्णश्चतुर्दशशतकीय इति ज्ञायते । तिरुपति केन्द्रीय विद्यापीठेन ग्रन्थस्यास्य भागद्वयं प्रकाशितम् । एनमधिकृत्याधिकमन्यत्र निरूपितम् ।। 
(C) ज्ञानोत्तममिश्रकृता - भाष्यवार्तिकव्याख्या ।
ग्रन्थोऽयं दशमशतकीयेन ज्ञानोत्तममिश्रेण कृत इति दासगुप्तमहाशयेन (HIP. Vol. II) निर्दिश्यते । परन्तु ग्रन्थस्य प्राप्तिस्थानं न लभ्यते । 
(D) नृसिम्हप्रज्ञकृतम् - भाष्यवार्तिकविवरणम् ।
न्यायतत्वविवरणमित्यपरनामायं ग्रन्थ अमुद्रितः मद्रासराजकीयहस्तलिखितपुस्तकालये (R. 3327 MGOML) अडयारपुस्तकालये अनन्तशयनपुस्तकालये शृङ्गगिरिमठसूच्यां वेङ्कटेश्वरपुस्तकालयसूच्याञ्च दृश्यते । अस्य कर्ता कुलशेखरपुरीशिष्यः नृसिम्हप्रज्ञ इति परं ज्ञायते ।।
(E) विश्वानुभवकृता - भाष्यवार्तिकसम्बन्धोक्तिः । 
अनादिरनन्तश्चायं ग्रन्थस्तृतीयाध्यायचतुर्थाध्याययोः परं अमुद्रितः मद्रासराजकीयपुस्तकालये (R. 4435 MGOML) अडयारपुस्तकालये च लभ्यते । अस्य कर्ता प्रत्यग्बोधशिष्यः विश्वानुभव इति परं ज्ञायते ।।
(F) विद्यारण्यकृतः - वार्तिकसारः । मुद्रितश्चायं ग्रन्थः वाराणसीग्रन्थमालायाम् । अस्य कर्ता प्रसिद्धः विद्यारण्यः पञ्चदश्यादिकर्ता ।
1.  अज्ञातकर्तृका - वार्तिकसारव्याख्या । अमुद्रितोऽयं ग्रन्थः तिरुवनन्तपुरसूच्यां (397 TCD) दृश्यते ।
2. महेश्वरतीर्थकृतः - वार्तिकसारसंग्रहः ।
ग्रन्थोऽयं वारणसीसंस्कृतग्रन्थमालायां मुद्रितः। अस्य कर्ता चिदानन्दघनस्य विद्यारण्यस्य च शिष्यः महेश्वरतीर्थः । स्वग्रन्थे आनन्दगिरिं निर्दिशन्नयं ग्रन्थकारः विद्यारण्यशिष्यः चतुर्दशशतकीय (1350 - 1450 A.D) इति निश्चेतुं अर्हते ।। 
3.  अज्ञातकर्तृका - वेदान्तोपनिषत् । वार्तिकसारात्मकोऽयं ग्रन्थः (7612 TSML) सरस्वतीमहालये लभ्यते ।
4. उपनिषद्ब्रह्मेन्द्रकृतम् - बृहदारण्यकविवरणम् ।। (ALS)
5.  द्रविडाचार्यकृतम् - बृहदारण्यकभाष्यम् । 
भाष्यमिदं नोपलभ्यते । तथापि शङ्करात् पूर्वतनेन द्रविडाचार्येण कृतमिति भाष्यादिषु उद्धृतत्वात् ज्ञायते । एतच्च द्रविडाचार्यप्रकरणे प्रतिपादयिष्यते । 
6. नारायणाश्रमिकृता - दीपिका । यद्यपीयं दीपिका नोपलभ्यते तथापि जाबालोपनिषद्दीपिकायाः 275 तमे पुटे ग्रन्थकृतैव निर्दिष्टा ।। 
7. नित्यानन्दाश्रमकृता - मिताक्षरा । (ASS 31) 
8. वासुदेवब्रह्मकृता - प्रकाशिका । अमुद्रितोऽयं बरोडासूच्यां दृश्यते । अस्य कर्ता अनिरूद्धपुत्रः हृषीकेशशिष्यः वासुदेवब्रह्म एकोनविंशतिशतकीय (1800 - 1909 A.D.) इति ज्ञायते । 
9. विद्यारण्यकृता - बृहदारण्यकव्याख्या । ग्रन्थोऽयममुद्रितः नासिकसूच्यां दृश्यते ।
10. शङ्करानन्दकृता - दीपिका। ग्रन्थोऽयं सरस्वतीमहालयसूच्यां दृश्यते ।। 

% बृहदारण्यकोपनिषद् 

ब्रह्मविद्योपनिषत् -
अस्यामुपनिषदि प्रणवस्वरूपविवरणं कृतम् । प्रणवोपासनाप्रकारा अनेकविधा विद्याश्च प्रदर्शिताः । तासु हंसविद्याया मुख्यत्वं प्रदर्शितम् । निष्कलसकल भावविवेचनं ज्ञानमाहात्म्यप्रदर्शनञ्चावधेयार्हविषयाः । मुद्रिता चेयमुपनिषदानन्दाश्रममुद्रणालये (ASS 29) । 
1. उपनिषद्ब्रह्मेन्द्रकृतम् - विवरणम् । (ALS)
2. नारायणाश्रमिकृता - दीपिका । (ASS 29)
3. शङ्करानन्दकृता - दीपिका । (ASS 29)
ब्रह्मोपनिषत् - 
अस्यामुपनिषदि आत्माभिव्यक्तिस्थानानि ब्रह्मस्वरूपं, ब्रह्मोपासनाक्रमः, तेषां फलञ्च प्रतिपादितम् । याजुषीयमुपनिषदानन्दाश्रमे मुद्रिता  (ASS 29) । 
1. उपनिषद्ब्रह्मेन्द्रकृतम् - विवरणम् । (ALS)
2. नारायणाश्रमिकृता - दीपिका । (ASS 29)
3. शङ्करानन्दकृता - दीपिका । (ASS 29)
महोपनिषत् - 
अध्यायषट्कपरिच्छिन्नायामस्यामुपनिषदि सृष्ट्यादौ ब्रह्मा निर्दिश्य तस्मात् सृष्टिक्रमः, पञ्चविंशतितत्वोत्पत्तिः, नारायणात् रुद्रादीनामुत्पत्तिः, नारायणस्य सर्वात्मकत्वप्रतिपादनं, शुकजनकसंवादमुखेन ब्रह्मविद्याप्रशंसापूर्वकं शुकस्येत्तमां गतिमभिधाय, निदाघऋभुसंवादेन चिदचित्स्वरूपं प्रकाश्य जीवन्मुक्तत्वस्वरूपप्रकाशनपूर्वकं वैराग्यं प्रशस्य ब्रह्मज्ञानतत्प्राप्तिप्रकारं च निरूप्य एतत्पाठकस्य फलञ्चाभिहितम् ।
1. उपनिषद्ब्रह्मेन्द्रकृतम् - विवरणम् । (ALS)
2. नारायणाश्रमिकृता - दीपिका । (ASS 29)
3. शङ्करानन्दकृता - दीपिका । (ASS 29)
माण्डूक्योपनिषत् - 
नवधाथर्वण इति व्याकरणमहाभाष्यवाक्यात् अथर्वणवेदः नवशाखात्मकः । तस्यान्यतमा शाखा माण्ड्क्यशाखा । दुर्दैवपरिपाकात् तस्यास्संहिताभागः, ब्राह्मणभागो वा साम्प्रतं नोपलभ्यते । केवलं द्वादशमन्त्रात्मकं उपनिषन्मात्रं उपलभ्यते । तच्च माण्डूक्योपनिषदिति व्यवह्नियते । 
मण्डूको नाम मन्त्रद्रष्टा कश्चित् ऋषिः । मण्डूकस्य गोत्रापत्यं माण्डूकः । स एव माण्डूक्यः । माण्डूक्यस्य ऋषिपरत्वे प्रमाणन्तु बृहदारण्यकवंशब्राह्मणम् । तत्र हि 6-5-2 " जायन्तीपुत्रो माण्डूकायनिपुत्रात् , माण्डूकायनिपुत्रो माण्डूकीपुत्रात् , माण्डूकीपुत्रो शाण्डिलीपुत्रात् " इति दृश्यते । तेन प्रोक्ता उपनिषत् माण्डूक्योपनिषत् । अस्या उपनिषदः प्रशंसा मुक्तिकोपनिषदि 1-26, 29 मन्त्रैः कृता । अस्या उपनिषदः द्वादश मन्त्राः । एषु सप्त मन्त्राः नृसिम्हतापिन्युपनिषदि रामतापिन्युपनिषदि च दृश्यन्ते । तथा च तयोरूपनिषदोरर्वाचीनत्वं ज्ञायते । अत एव नृसिम्हतापनीयोपनिषदोऽपि गौडपादेन व्याख्या कृता इत्युक्तिरम्शतस्सत्यतामुपैति । 
अस्या उपनिषदः प्रथमे मन्त्रे चित्तशुद्धये उपास्यस्य प्रणवस्य सर्वात्मकता प्रदर्शिता । द्वितीये मन्त्रे ब्रह्मणश्शुद्धचिद्रूपस्य सर्वात्मकता वर्णिता । सा च ब्रह्मणस्सर्वात्मकता दुग्धदधिवन्न परिणामवामाश्रित्य, निरवयवे कूटस्थब्रह्मणि तदभावात् , किन्त रज्जुसर्पवत् विवर्तवादमाश्रित्य । अन्ते आत्मनश्चतुष्पात्वं प्रतिज्ञातम् । ततश्चतुर्भिः 3,4,5,6 मन्त्रैः "अध्यारोपापवादाभ्यां निष्प्रपञ्च आत्मा प्रपञ्च्यते " इति रीत्या विश्ववैश्वानरयोस्तैजसहिरण्यगर्भयोः प्राज्ञाव्याकृतयोर्व्यष्टिसमष्टिरूपयोस्तुरीये चिद्रूपे आरोपः कृतः । सप्तमेन नान्तःप्रज्ञमिति मन्त्रेण विश्वादीनामपवादेन तुरीयोऽवशेषितः । 
ये तु विशुद्धचित्ता उत्तमाधिकारणस्तेषां कृते नान्तःप्रज्ञमित्यादि मुख्योपदेशः। ये तु मध्यमाधिकारिणस्तेषां चित्तशुद्धये क्रममुक्तये च प्रणवमात्राणां अकार - उकार - मकार - तुरीयाणां आत्मनो ब्रह्मणः पादैः विश्वतैजसप्राज्ञतुरीयैः व्यष्टिभिः समष्टिभिश्च वैश्वानरहिरण्यगर्भईश्वरतुरीयैरभेदभावना  8,9,10,11 मन्त्रैरूपदिष्टा । द्वादशेन मन्त्रेण प्रणवोपासनाफलमुक्तम् ।
उपनिषदां औपनिषदानाञ्च ऋषीणामनादिः काल इति सम्प्रदायः। परन्तु पाश्चात्यानां विमर्शकानां मत माण्डूक्योपनिषदां कालः (200 - 100 B.C) इति प्रतिपादयति। मुद्रिता चेयमुपनिषदानन्दाश्रममुद्रणालयेऽन्यत्र च । 
अस्या व्याख्या एतत्सम्बद्धाः कृतयश्च - 
1.  गौडपादाचार्यकृता - माण्डूक्यकारिका ।  (ASS 10 ) 
गौडपादकारिकाभिधेऽस्मिन् माण्डूक्योपनिषदां व्याख्यात्मके पद्यबद्धे ग्रन्थे चत्वारि प्रकरणानि सन्ति । तत्र प्रथमे भागे आगमाख्ये एकोनत्रिंशत् द्वितीये वैतथ्याख्ये अष्टत्रिंशत् , तृतीये अद्वैताख्ये अष्टाचत्वारिंशत् , चतुर्थे अलातशान्तिप्रकरणे शतमिति संहत्या 215 कारिकास्सन्ति । मुद्रितश्चायं ग्रन्थ आनन्दाश्रममुद्रणालये (ASS 10) 
आगमप्रकरणम् - इदमागमप्रकरणं माण्डूक्योपनिषदां भावार्थरूपम् । आगममूलकत्वादन्वर्थं नाम । प्रकरणेऽस्मिन् अकार-उकारणकारैः प्रतिपादितेभ्यः वैश्वानर-हिरण्यर्गभ-ईश्वरेभ्यः जाग्रत् - स्वप्न- सुषुप्ति - अवस्थाभ्यश्च भिन्नं तदनुगतं साक्षिरूपञ्च परमात्मतत्वं तुरीय इति नाम्ना वर्णितम् । 
वैतथ्यप्रकरणम् - 
द्वितीयेऽस्मिन् वैतथ्याख्ये प्रकरणे दृश्यप्रपञ्चस्य मायामयत्वं, मिथ्यात्वञ्च सयुक्तिकं साधितम् । आत्मा एक एव नित्यः, तस्मिन् विविधकल्पनावशात् प्रपञ्चस्योत्पत्तिरिव विकल्पो भवति । अस्य मूलकारणं माया । मायाकल्पितजगतः गन्धर्वनगरवत् असत्यत्वमिति प्रतिपाद्य " न निरोधो न चोत्पत्ति " रित्यादिना अखण्डचिद्धनानन्दात्मतत्वादन्यस्य असत्वं साधितम् ।
अद्वैतप्रकरणम् -
तृतीयेऽस्मिन् प्रकरणे अनेकाभिस्सुदृढाभिर्युक्तिभिः अद्वैततत्वं साधितम् । आत्मनि सुखदुःखभावना नितरामसङ्गता यथा बालाः धूलिधूमादिंससर्गेण आकाशं मलिनमामनन्ति वस्तुतः यथा चाकाशो मालिन्यशून्यः तथैवात्मनोऽपि सुखित्वदुःखित्वकथनं बालबुद्धिविलासतुल्यमिति प्रतिपादितम् । असङ्गोह्यात्मा । माया हि द्वैतकल्पनायाः कारणम् । अमृतस्य मर्त्यत्वं मर्त्यस्यामृतत्वञ्चासङ्गतम् । अत अमृतस्यात्मनो यद्युत्पत्तिस्स्वीक्रियते तर्हि मर्त्यत्वधर्म आपद्येत इति आत्मनः उत्पत्तिः - जातिः नास्तीति प्रतिपादितम् । अयमेव गौडपादाचार्याणामजातिवादः सोऽयं वादः 
स्वप्नमाये यथा दृष्टे गन्धर्वनगरं यथा ।
तथा विश्वमिदं दृष्टं वेदान्तेषु विचक्षणैः ।। 2-31 
न निरोधो न चोत्पत्तिर्न बद्धो न च साधकः । 
न मुमुक्षुर्नवै मुक्त इत्येषा परमार्थता ।। 2 - 32 
प्रपञ्चो यदि विद्येत निवर्तेत न संशयः । 
मायामात्रमिदं सर्वमद्वैतं परमार्थतः ।। 1-17
न किञ्चिज्जायते जीवसम्भवोऽस्य न विद्यते ।
एतत्तदुत्तमं सत्यं यत्र किञ्चिन्न जायते ।। 3-4. 
इत्यादिषु तत्र तत्र प्रतिपादितः ।
अयमजातिवादः गौडपादात्प्राचीनस्य बौद्धाचार्यस्य दिङ्नागस्य माध्यमिकवृत्तौ, एवं दिङ्नागात् प्राचीनेषु पालीभाषाप्रणीतबौद्धग्रन्थेषु दर्शनात् ततो गृहीत इति विधुशेखरभट्टाचार्याः प्रवदन्ति। पालीभाषाया अपि प्राचीनेषु उपनिषद्ग्रन्थेषु " अजायमानो बहुधा व्याजायत " इत्यादिदर्शनात् , " नासतो विद्यते भाव " इति गीतायाञ्च दर्शनात् पूर्वोक्तोक्त्यनुपपत्तौ भारतीयंसस्कृतिभक्ताः प्रमाणम् । 
अलातशान्तिप्रकरणम् - 
चतुर्थऽलातशान्त्याख्ये प्रकरणे " अलाते भ्रामिते सति यथा गोलाकार प्रतीतिर्जायते परन्तु सा गोलाकृतिः भ्रमणजन्यैव, न वस्तुतः, एवं जगदादि मायाकल्पितमेव । मनसः व्यापारादेवोत्पत्तिस्तस्य मनसो निरोधे च स नास्त्येवेति। यथा च भ्रमणादिक्रियाशान्तौ गोलाकारकालातप्रतीतिशान्तिः, एवं मनस अमनीभावात् जगतश्शान्तिः। जगदुत्पत्तिलयौ प्रतीत्यप्रतीती उभेऽपि भ्रान्तिजनिते परमार्थतः परमात्मतत्वं पारमार्थिकमिति प्रतिपादितम् ।"
अद्वैतवेदान्तस्य प्राणभूता अनिर्वचनीयख्यातिरपि प्रकरणेऽस्मिन्नेव -
विपर्यासात् यथा जाग्रत् अचिन्त्यात् भूतवत् सृजेत् । 
तथा स्वप्ने विवपर्यासात् धर्मान् तत्रैव पश्वति ।। 4 - 41 
न निर्गतास्ते विज्ञानात् द्रव्यत्वाभावयोगतः ।
कार्यकारणताभावात् यतोऽचिन्त्यास्सदैव ते ।। 4-52. 
उभे ह्यन्योन्यं दृश्येते किं तदस्तीति नोच्यते । 
लक्षणाशून्यमुभयं तन्मतेनैव गृह्यते ।।  4 -67.
इत्यादिभिः कारिकाभिः प्रदर्शिता । 
प्रकरणस्यास्य भाषा "विज्ञप्ति" रित्यादिपारिभाषिकशब्दैः पूर्णा । ते च शब्दाः बुद्धग्रन्थेषु दृश्यन्त इति बुद्धमतमेव गौडपादः वेदान्तापदेशेन प्रतिपादयतीति, बुद्धप्रभावः गौडपादे दृश्यत इति, पच्छन्नबौद्धा अद्वैतिन इति व केचिद्वदन्ति। 
परन्तु शब्दसाम्यं नात्र प्रमाणमकिञ्चित्करञ्च । यतः - अध्यात्मशास्त्रस्य पारिभाषिकशब्दाः न केवलं बौद्धानां स्वम् । परन्तु ते सर्वदर्शनसामान्याः । तेषां प्रयोगे यथा गौडपादस्य तथा बौद्धानाम् यथा बौद्धानां तथा गौडपादस्यापीति समानस्वातन्त्र्यादिसत्वात् तादृशी युक्तिरसङ्गतैव । 
अस्याः कारिकायाः कर्ता गौडपादाचार्यः शुकमुनीन्द्रशिष्य इति वेदान्तसम्प्रदायप्रतिपादकात् "नारायणं पद्मभुवं वसिष्ठ " मितित्यादिश्लोकात् जायते । एवं नृसिम्हतापनीयोपनिषदां गौडपादकृते व्याख्याने ( D. 581, 582 MGOML)  दृश्यते । वेताश्वतरोपनिषदां शाङ्करे भाष्ये तथा च शुकशिष्यो गौडपादाचार्यः (Page 30 ASS Edn 17) दृश्यते । लक्ष्मणशास्त्रिकृते गुरुवंशकाव्येऽपि (Page 12) दृश्यते ।। 
गौडपादस्य स्थानं नाद्यापि निश्चितम् । विषयेऽस्मिन् विचाराःभिन्ना एव दृश्यन्त । परन्तु माण्डूक्यकारिकाशाङ्करभाष्यव्याख्याने आनन्दगिरीये अलातशान्तिप्रकरणस्थस्य " तं वन्दे द्विपां वरम् " इति पद्यांशस्य व्याख्याने एवं दृश्यते -
"आचार्यो हि पुरा बदरिकाश्रमे नरनारायणाधिष्ठिते नारायणं भगवन्त मभिप्रेत्य तपो महदतप्यत " (Page 157 ASS Edn) इति । एवञ्च बदरिकाश्रम एव गौडपादस्य स्थानमिति ज्ञायते । 
सप्तशतशतकीयेन बालकृष्णानन्दसरस्वत्या स्वीये शारीरकमीमांसाभाष्यवार्तिके - 
"गौडचरणाः कुरुक्षेत्रगताः हीरारावतीनदीतीरभवगौडजातिश्रेष्ठा देशविशेषभवनाम्नैव प्रसिद्धाः द्वापरयुगमारभ्यैव समाधिनिष्ठत्वेन आधुनिकैर्जनैरपरिज्ञातविशेषाभिधानाः सामान्यनाम्नैव लोकविख्याता " इति प्रतिपादितम् (ASI Page 6)। एवञ्च गौडपादः कुरुक्षेत्रवासी गौडजात्युत्पन्न इति गौडपादीयनामान्तरापरिज्ञाने च कारणं सूचितम् , केचित्तु गौडपादं बंगालदेशानां उत्तररणागवर्तिगौडदेशोद्भवमामनन्ति । 
भारतीयाद्वैतवेदान्तपरम्पराप्रामाण्यात् गौडपादश्शुकशिष्यः, शङ्करश्च गौडपादशिष्यः गौडपादेनानुगृहीतश्चेति शङ्करात्पूर्वतनो वा शङ्करकालपर्यन्तजीवीति वा निश्चीयते । यदि वयं कुप्पुस्वामिशास्त्रिणां सिद्धान्तमनुसृत्य ( 632 - 664 A.D.) कालवर्तिनं शङ्कराचार्यमभ्युपगच्छामस्तर्हि गौडपादकाल (520 - 620 A.D.) इति, इच्छामात्रशरीरत्यागिनां गौडपादानां कालश्शङ्कराचार्यानुग्रहपर्यन्तमिति वा स्वीकर्तव्यम् । 
विधुशेखरभट्टाचार्यास्तु - आगमशास्त्रस्य भूमिकायां द्वितीयशतकादारब्धानां चतुर्थशतकपर्यन्तानां बौद्धापण्डितानां ग्रन्थस्य गौडपादकारिकायाश्च शब्दसाम्यदर्सनात् गौडपादस्तदर्वाग्भव इति गौडपादः पञ्चमशतकीय ( 500 A.D.) इति प्रतिपादयन्ति । 
गौडपादस्यातिप्राचीनत्वेन प्रसिद्धेः, गौडपादकारिकास्था एव शब्दाः द्वितीयशतकादारब्धचतुर्थशातकान्तकालवर्तिभिः बैद्धपण्डितैस्स्वीकृताः , बौद्धैरेव गौडपादोऽनुसृतः, न तु गौडपादेन बौद्धपण्डिता इति च सम्प्रदायविदां सिद्धान्तः । गौडपादकारिकाया व्याख्या :-
(क) शङ्कराचार्य कृतम् - माण्डूक्यकारिकाभाष्यम् (ASS 10) 
1.  अनुभूतिस्वरूपाचार्यकृतम् - गौडपादीयभाष्यटिप्पणम् ।
अमुद्रितोऽयं ग्रन्थः मद्रासराजकीयपुस्तकालये (R. 2911 MGOML)  अडयारपुस्तकालये च लभ्यते । अस्य कर्तारं अनुभूतिस्वरूपाचार्यं अधिकृत्य प्रकटार्थविवरणस्थले उपपादयिष्यते ।
2. आनन्दगिरिकृता - माण्डूक्यकारिकाभाष्यव्याख्या । मुद्रितश्चायं ग्रन्थस्समूल आनन्दाश्रममुद्रणालये (ASS 10) । शुद्घानन्दकृता शाङ्कर कारिकाभाष्यस्य व्याख्या चास्तीति श्रूयते । 
(ख) स्वयम्प्रकाशयोगिकृता - मिताक्षरा (BSS 48)  
गौडपादीयकारिकाव्याख्यात्मकोऽयं ग्रन्थः वाराणसीसंस्कृतग्रन्थमालायां मुद्रितः । अस्य कर्ता स्वयम्प्रकाशयोगी यदि तत्वानुसन्धानव्याख्याता स्यात्तर्हि तस्य कालः (1700 - 1900 A.D.) समयान्तर्गतो भवति। नान्यदत्र प्रमाणं दृश्यते ।। 
 (ग) स्वामियतिकृता - मिताक्षरा । 
 माण्डूक्यकारिकाव्याख्यात्मकोऽयं ग्रन्थः चौखाम्बामुद्रणालये मुद्रितः । अस्य कर्ता स्वामियतिर्विरूपाक्षपुत्रः धारणकोटिकुलोत्पन्नः ब्रह्मानन्दशिष्यः रसरामशैलशशिगे शकवत्सरे 1736 शक (1813 A.D.) स्वग्रन्थं चकारेति ज्ञायते ।।
 (घ) उपनिषद्ब्रह्मेन्द्रकृता - कारिकाव्याख्या । अमुद्रितोऽयं ग्रन्थ अडयारपुस्तकालये लभ्यते ।।
 (ङ) अज्ञातकर्तृकः - गौडपादीयविवेकः । अमुद्रितोऽयं ग्रन्थः मद्रासराजकीयपुस्तकालये ( R. 3882 MGOML ) लभ्यते । माण्डूक्योपनिषद्भाष्यादि -
 2. शङ्कराचार्यकृतम् - माण्डूक्योपनिषद्भाष्यम् ।
 (क) आनन्दगिरिकृता - माण्डूक्यभाष्यव्याख्या । उभावपीमौ ग्रन्थौ आनन्दाश्रमे मुद्रितो । (ASS 10) 
 (ख) मधुरानाथशुक्लकृता - माण्डूक्यभाष्यव्याख्या । ग्रन्थोऽयं दासगुप्तमहाशयेन (H.I.P Vol. II Page 78) निर्दिष्टः । 
 (ग) अज्ञातकर्तृका - पदार्थविवृतिः । माण्डूक्यभाष्यव्याख्यात्मकोऽयममुद्रितः ग्रन्थः मद्रासराजकीयहस्तलिखितपुस्तकालये (D. 17021 MGOML) लभ्यते । 
 (घ) राघवानन्दकृतः - माण्डूक्यभाष्यार्थसंग्रहः ।
 ग्रन्थेऽयं दासगुप्तमहाशयेन (H.I.P. Vol. II Page 78)  निर्दिष्टः । अस्य कर्ता विश्वेश्वरसरस्वतीप्रशिष्यः अद्वियभगवत्पादशिष्यः राघवानन्द इति ज्ञायते । 
 3. उपनिषद्ब्रह्मेन्द्रकृतम् - माण्डूक्यविवरणम् (ALS) 
 4. नारायणाश्रमिकृता - दीपिका । अमुद्रितोऽयं ग्रन्थ अडयारपुस्तकालये सरस्वतीमहालये (1556 TSML) च लभ्यते ।  
 5. भास्करानन्दकृता - माण्डूक्यव्याख्या । 
 6. शङ्करानन्दकृता - दीपिका । (1552 TSML) मद्रासराजकीयहस्तलिखितपुस्तकालये (D. 707 MGOML) बरोडापुस्तकालये च लभ्यते ।। 
 
 % माण्डूक्योपनिषद् 
 
मुण्डकोपनिषत् -
अथर्वशाखान्तर्गतायां प्रत्येकं खण्डद्वितययुतमुण्डकाख्यप्रकरणत्रितयविभक्तायामस्यां उपनिषदि ब्रह्मविद्योपदेशपरम्पराप्रदर्शनपूर्वकं ब्रह्मज्ञानसाधनभूतपरापरविद्यास्वरूपादिकमभिधाय परविद्यावेद्याब्रह्मणः स्वरूपं प्रकाश्य तस्मादेव जगतस्संग्रहेण प्रादुर्भावमुपवर्ण्य मुमुक्ष्वमुमुक्षुकर्तव्यानि च संगृह्य ब्रह्मणो जगत्सृष्टिं विस्तरतः, प्रतिपाद्य प्रणवाख्याशरेण ब्रह्मात्मकलक्ष्यभेदनप्रकारमुक्त्वा प्राप्यस्वरूपतत्प्राप्तिप्रकारतत्साधनादिकं ब्रह्मविदाऽङ्गीरसा शौनकायोपदिष्टं दृश्यते ।। अस्या रचनाकालः 500-400 BC इति विमर्शकाः । मुद्रिता चेयमुपनिषदानन्दाश्रममुद्रणालये ।
अस्या व्याख्यादि - 
1. शङ्कराचार्यकृतम् - मुण्डकोपनिषद्भाष्यम् । (ASS 9) 
(क अभिनवनारायणेन्द्रकृता - मुण्डकभाष्यटीका । अमुद्रितोऽयं ग्रन्थ "औध " सूच्यां (XXI 26)  दृश्यते । 
(ख) आनन्दगिरिकृता - मुण्डकभाष्यव्याख्या । (ASS 9)
(ग) शिवानन्दयतिकृता - मुण्डकभाष्यटिप्पणी । 
अमुद्रितोऽयं ग्रन्थः सरस्वतीमहालयसूच्यां (1563 DCTSML) मद्रासराजकीयपुस्तकालये (D. 722 MGOML) च लभ्यते । 
2. उपनिषद्ब्रह्मेन्द्रकृतम् - मुण्डकोपनिषद्विवरणम् ( ALS)
3. नारायणाश्रणिकृता - मुण्डकोपनिषद्दीपिका । (ASS 9) 
4. भास्करानन्दकृता - मुण्डकोपनिषद्दीपिका ।
5. शङ्करानन्दकृता - मुण्डकोपनिषद्दीपिका । अमुद्रितोऽयं ग्रन्थस्सरस्वती महालयसूच्यां (1561 DCTSML) लभ्यते ।

% मुण्डकोपनिषद् 

रामतापिन्युपनिषत् -
पूर्वोत्तरतापिनीभेदेन द्विधा विभक्तेयमुपनिषत् । अत्र पूर्वतापिन्यां रामशब्दार्थः परं ब्रह्म इत्युक्त्वा तदुपासनोपयोगि मन्त्रमभिधाय रामख्याबीजात् जगदुत्पत्तिमभिवर्ण्य रामायणाख्यायिकाश्रवणफलान्युक्त्वा रामयन्त्रोद्धारकमश्च प्रतिपादितः । उत्तरतापिन्यां तारकशब्दनिर्वचनपूर्वकं तारकमन्त्रस्योपदेष्टारं स्थानमभिधाय तद्देवतानमस्कारमन्त्रकथनपूर्वकं तारकमन्त्रस्य महात्म्यं फलञ्च निरूपितम् । मुद्रिता चेयमुपनिषदानन्दाश्रममुद्रणालये । 
1. उपनिषद्ब्रह्मेन्द्रकृतम् - रामतापिनीविवरणम् । (ALS) 
2. नागेश्वरसूरिकृता - रामतापिनीवृत्तिः । अमुद्रिता चेयं मद्रासराजकीयपुस्तकालये (D. 763 MGOML) लभ्यते । 
3. नारायणाश्रमिकृता - रामतापिनीदीपिका । (ASS 29)
4. भट्ट मुद्गलसूरिकृता - रामोत्तरतापिनीव्याख्या । अमुद्रितेयं व्याख्या (D. 726 MGOML) लभ्यते । 
5. रामयतिकृता - पदयोजनिका  
अमुद्रितोऽयं ग्रन्थः (D. 764 MGOML) मद्रासराजकीयपुस्तकालये, अडयारपुस्तकालये, बरोडासूच्याचञ्च दृश्यते । गोविन्दानन्द शिष्योऽयं रामयतिरिति परं ज्ञायते । 
6. विश्वेश्वरकृता - रामतापिनीपूर्वखण्डव्याख्या । 
अमुद्रितोऽयं ग्रन्थः (D 762. MGOML) अडयारपुस्तकालये बरोडापुस्तकालये च लभ्यते । स्वयम्प्रकाशशिष्योऽयं विश्वेश्वरः ।।
7. सुरेश्वराश्रमिकृता - रामचन्द्रज्योत्स्ना । 
रामोत्तरतापिन्याः परं व्याख्यात्मकोऽयं ग्रन्थ अमुद्रितः बरोडापुकालये (246 DC BRD) दृश्यते । अस्य कर्ता रघुरामतीर्थशान्ताश्रमिशिष्य इति परंज् ज्ञायते । 
8. अज्ञातकर्तृका - रामतापिनीव्याख्या । ग्रन्थोऽयं सरस्वतीमहालयसृच्यां (1564 DC TSML) दृश्यते ।। 
श्वेताश्वतरोपनिषत् - 
अस्यामुपनिषदि क्षराक्षरस्वरूपकथनपूर्वंकं योगक्रममुपपाद्य योगगम्यस्य परमात्मनस्स्वरूपादिकं तस्य पुरुषसूक्तप्रतिपाद्यात्वमोक्षप्रदत्वादिकं चोपवर्ण्य, प्रकृतिस्वरूपं विविच्य परमात्मनो गायत्रीप्रतिपाद्यत्वमुक्त्वा क्षराक्षरविद्याफलभेदादिकं प्रकाश्य परमात्मस्वरूपतदुपासनातन्महिमादिकं प्रपञ्च्यते । श्वेताश्वतरविद्वदुपदिष्टत्वादियं श्वेताश्वतरोपनिषदिति कथ्यते । अस्या रचनाकालः (200 - 100 BC)  इति विमर्शकाः । मुद्रिता चेयमानन्दाश्रममुद्रणालये । (ASS 17) 
1. शङ्कराचार्यकृतम् - श्वेताश्वतरभाष्यम् । (ASS 17)
शङ्कराचार्यकृतत्वेनेदं भाष्यं मुद्रितमानन्दाश्रममुद्रणालये । सरस्वतीमहालयस्थादर्शग्रन्थान्ते (1565 TSML) च शङ्कराचार्यकृतमिति दृश्यते । परन्तु शङ्कराचार्यैः श्वेताश्वतराणां भाष्यं न कृतमिति ज्ञायते । अथवा नेदं भाष्यं शङ्कराचार्यकृतमिति निश्चीयते । अत्रेमानि कारणानि । 
(1) विद्यारण्यप्रणीतशङ्करविजये षष्ठसर्गे आचार्यशङ्करकृतोपनिषद्भाष्यादिनामनिर्देशे श्वेताश्वतरोपनिषद्भाष्यनामानिर्देशः । तत्रहि - 
"करतलकलिताद्वयात्मतत्वं क्षपितदुरन्तचिरन्तनप्रमोहम् ।
उपचितमुदितोदितैर्गुणौघैः उपनिषदामयमुज्जहार भाष्यम् ।।"
अत्र धनपतिसूरिकृतडिण्डिमाख्यटीकायां उपनिषदामित्येतत्पदव्याख्याने " ईशादिबृहदारण्यकान्तानां दशानामेवोपनिषदां संग्रहः कृतः। यदि श्वेताश्वतरोपनिषदामपि भाष्यं शङ्कराचार्यैः प्रणीतं स्यात्तर्हि डिण्डिमकारेणास्या उपनिषदोऽप्युल्लेखः कृतो भवेत् । "
(2) यासामुपनिषदां भाष्यं शङ्कराचार्येण कृतं तासाम् , यासामुपनिषदां शङ्कराचार्येण भाष्यं न कृतं तासामपि दीपिकानाम्नी व्याख्या नारायणाश्रमिणा रचिता आनन्दाश्रममुद्रणालये मुद्रिता च । तासु शङ्कराचार्यकृतत्वेन निश्चितभाष्याणानुपनिषदां व्याख्यावसरे नारायणाश्रमिणा " शङ्करोक्त्युपजीविना " इत्युच्यते । तासुदीपिकासु दृश्यते वाक्यतोऽपि साम्यं भाष्येण । शङ्कराव्याख्यातानामुपनिषदां व्याख्यावसरे "श्रुतिमात्रोपजीविना " इत्येव दीपीकायां निर्दिश्यते । श्वेताश्वतरदीपिकापि श्रुतिमात्रोपजीविना नारायणाश्रमिणैव कृता न तु शङ्करोक्त्युपजीविना । तस्मात् श्वेताश्वतरोपनिषदां शाङ्करं भाष्यं नास्तीत्येव ज्ञायते । 
(3) नारायणविरचिते दीपिकाख्ये व्याख्याने षष्ठेऽध्याये - " यदा चर्मवदाकाशं " इत्यादिविंशतितमर्क्व्याख्यानावसरे " अयमर्थ आचार्यसम्मतः । चर्मवदाकाशवेष्टनासम्भववत् अविदुषो मोक्षासम्भववत् अविदुषो मोक्षासम्भवश्रुतेरिति सर्वधरर्मान् परित्यज्येति श्लोके शाङ्करगीताभाष्ये उक्तत्वादिति " विद्यते । यदि श्वेताश्वतरोपनिषदः भाष्यं कृतं स्यात्तर्हि गीताभाष्ये उक्तत्वादिति हेतुरसङ्गतः । एतद्भाष्ये एव आचार्यैरुक्तत्वादित्येव वक्तुं शक्याम् । 
(4) आचार्यशङ्करप्रणीतदशोपनिषद्भाष्येषु दृश्यमाणाः पदलालित्य गाम्भीर्य सरलतादयो गुणाः अस्या भाष्ये न दृश्यते । तस्मात् शङ्करपीठाधिष्ठितैः अन्यैर्वा शङ्कराचार्याणां नाम्ना केनापि विदुषा वा कृतं स्यादित्येव निर्णीयते । शाङ्करभाष्याणां सर्वेषामपि आनन्दगिरिणा व्याख्या कृता श्रूयते । परन्तु श्वेताश्वतरोपनिषद्भाष्यस्य आनन्दगिरिकृता व्याख्या नाद्यापि लक्ष्यते च ।
2. उपनिषद्ब्रह्नेन्द्रकृतम् - श्वेताश्वतरविवरण् । (ALS) 
3. नारायणाश्रमिकृता - दीपिका । (ASS 17)
4. विज्ञानभगवत्कृतम् - श्वेताश्वतरविवरणम् । (ASS 17) 
मुद्रितञ्चेदं विवरणमानन्दाश्रममुद्रणालये (ASS 17)। अस्य कर्ता विज्ञानात्मा ज्ञानोत्तमषिष्यः, चित्सुखाचार्यसतीर्थ्यः, द्वादशशतकवासी (1100 - 1200 A.D.) इति प्रकृतग्रन्थपरामर्शात् श्रीकण्ठशास्त्रिकृतोपन्यासात् (IHQ Vol XIV) च ज्ञायते । 
5. शङ्करानन्दकृता - दीपिका । (ASS 17)
6. अज्ञातकर्तृका - दीपिका । सरस्वतीमहालयपुस्तकालये (1567 DCTSML) दृश्यते ।। 
हंसोपनिषत् - 
अस्यामुपनिषदि सर्वव्यापिनस्त्रिमात्रस्य हंसाख्यपरमात्मनः ध्यानक्रमं निरूप्य तस्य नादे विलयप्रकारमुक्त्वा ऋषिच्छन्दोदेवतान्यासपूवर्कं तन्मन्त्रस्वरूपादिकं संगृह्य तस्य दशविधनादस्वरूपतायां फलविशेषाश्च संगृहीताः । मुद्रिता चेयमानन्दाश्रम मुद्रणालये ।।
1. उपनिषद्ब्रह्मेन्द्रकृतम् - विवरणम् । (ALS)
2. नारायणाश्रमिकृता - दीपिका । अमुद्रितेयं दीपिका बरोडासूच्यां ( 11529 BRD) दृश्यते । 
3. शङ्करानन्दकृता - दीपिका । (ASS 27) आनन्दाश्रममुद्रणालये मुद्रिता । शङ्करानन्दः पूर्वमुपपादितः ।। 



\end{document}




 ।। ओं नमो ब्रह्मादिभ्यो ब्रह्मविद्यासंप्रदायकर्तृभ्यो वंशऋषिभ्यो नमो महद्भ्यो नमो गुरुभ्यः ।।
 (१) नमः श्रुतिशिरःपद्मषण्डमार्तण्डमूर्तये ।
 वादरायणसंज्ञाय मुनये शमवेश्मने ।।
 (२) ब्रह्मसृत्रकृते तस्मै वेदव्यासाय वेधसे ।
 ज्ञानशक्त्यवताराय नमो भगवतो हरेः ।।
 (३) शङ्करं शङ्कराचार्यं केशवं वादरायणम् ।
 सृत्रभाष्यकृतौ वन्दे भगवन्तौ पुनः पुनः ।।
 (४) श्रुतिस्मृतिपुराणानामालयं करुणालयम् ।
 नमामि भगवत्पादं शङ्करं लोकशङ्करम् ।। 
 (५) वेदान्ताम्भोगभीरा नयमकरकुला ब्रह्मविद्याव्जषण्डा 
 पाषण्डोत्तुङ्गवृक्षप्रमथननिपुणा मानवीचीतरङ्गा ।
 यस्यास्योत्था सरस्वत्यखिलभवभयध्वंसिनी शङ्करस्य 
 गङ्गा शम्भोः कपर्दादिव निखिलगुरोर्नौमि तत्पादपद्मम् ।। 
 (६) वेदान्तार्थं गभीरं ह्यतिसुगमतया बोधयामीति विष्णु-
 र्व्यासात्माऽसूत्रयत तद् दुरधिगमममूद् वादिदुर्बुद्घिभेदात् ।
 भिन्दन् दुर्बुद्धिभेदं य इह करुणयाऽभाष्ययद् भाष्यमेतत् 
 तं वन्दे सर्ववन्द्यं त्रिजगति भगवत्पादसंज्ञं महेशम् ।। 
 (७) नारायणं पद्मभुवं वसिष्ठं शक्तिं च तत्पुत्रपराशरं च । 
 व्यासं शुकं गौडपदं महान्तं गोविन्दयोगीन्द्रमथास्य शिष्यम् ।
श्रीशङ्कराचार्यमथास्य पद्मपादञ्च हस्तामलकञ्च शिष्यम् ।
तं तोटकं वार्तिककारमन्यान् अस्मद्गुरून् सन्ततप्रानतोऽस्मि ।। 
(८) सदाशिवसमारम्भां शङ्कराचार्यमध्यमाम् ।
अस्मदाचार्यपर्यन्तां वन्दे गुरुपदम्पराम् ।।
किञ्चित् प्रास्ताविकम् 
संशोधनाय ग्रन्थरचनाय च स्वीकृतोऽयं विषयः - अद्वैतवेदान्तसाहित्येतिहासकोशः - ( A Bibliographical History of Advaita Vedanta literature ) इति । तत्सम्पादने मयानुमूतस्य कृतस्य च परिश्रमस्य फलरूपोऽय विषयः ग्रन्थरूपेण भवतां सन्निधिमागच्छति । 
सर्वदर्शनोत्तमस्य अद्वैतदर्शनस्य विकासकालः प्राचीन मध्यम - आधुनिकभेदेन त्रेधा विभक्तुं शक्यते । तत्र प्राचीनो भागः दशमशतकान्तः, मध्यमो भागः षोडशशतकान्तः, आधुनिकभागस्तु अद्यावधिक इतः परश्च। प्राचीने काले शाङ्करभाष्यस्य व्याख्यानभूतानां ग्रान्थानाम् , भामतीप्रस्थान्स्य विवरणप्रस्थानस्य च वीजावापः, संख्येयानां स्वतन्त्रग्रन्थानाञ्च आविर्भावो दृश्यते । न्यायशैलीनिब्द्धाः न्यायवैशेषिकादिमतसिद्धभेदवादखण्डनपरा अद्वैतग्रन्थास्स्वतन्त्रभूताः मध्यम एव काले प्रादुरभृवन् ।
अद्वैतसिद्धान्तः 
शाङ्करे अद्वैतमते ब्रह्मैव सत्यम् । अन्यत्सर्वं मिथ्या । जगदादि सर्वं शुक्तौ रजतमिव भासते । ब्रह्म निर्गुणम् , अत एव शब्दैरप्रतिपाद्यम् । अत एव वाड्मनसयोरगोचरमिति श्रुतौ कथ्यते । सर्वज्ञत्वादयो गुणा अपि ब्रह्मणि औपाधिका एव । जीवो ब्रह्मरूपोऽपि अज्ञानात् भिन्न इव भाति। मनोबुद्ध्याद्युपाधिभिस्सर्वोऽपि जनः "अह" मिति प्रत्येति । अत एव सुषुप्तौ मनोबुद्ध्यादिलयेन अहमाकारप्रतीत्यभावः। एवं मोक्षावस्थायामपि उपाधिलयात् अहमाकारप्रतीत्यभावः। एवञ्चाहमिति प्रतीतिरौपाधिकी । अयं प्रातीतिको जीव एक एव न तु नाना । अन्तःकरणभेदात् सुखदुःखानुभवभेदः। चैतन्यं सर्वत्र्यापकं स एव आत्मा इत्यभिधीयते । अयमात्मा ईश्वर - जीव - साक्षीति भेदेन त्रिविधः। सर्वजगन्मूलकारणम्  अज्ञानम् , तस्मिन् चैतन्यान्तर्गते सति स आत्मा ईशअवर इत्यभिधीयते । स एव ईश्वरस्सत्त्वरजस्तमोभिर्विप्णु ब्रह्म - शङ्कराख्याः लभते । आत्मा ज्ञानस्वरूपो न तु ज्ञातृस्वरूपः । तस्य ज्ञातृत्वमहङ्काराद्युपाधिभिः । प्रपञ्चस्यापि ज्ञेयत्वमज्ञानादेव । त्रिविधं सत्यत्वं - प्रातिभासिकं व्यावहारिकं पारमार्थिकञ्च। तत्र प्रातिभासिकस्य शुक्तिरजतादेः व्यावहारिकसत्यरजतेन बाधः। व्यावहारिकसत्यत्वं घटपटादीनाम् , तच्च पारमार्थिकसत्येन ब्रह्मदर्शनेन बाध्यते । ब्रह्मणस्तु कस्यामप्यवस्थायां न बाधः । अत एव तत् परमार्थसत् । धटपटदीनां व्यवहारदशायां सत्यत्वम् । वस्तुतस्तु ते स्वाप्निकपदार्थवत् मिथ्याभूताः । यथा स्वप्नगताः पदार्थाः जाग्रद्दशायां न सन्ति अतो मिथ्या, तथा पारमार्थिकदशायां अभावात् एते व्यवहारदशायां विद्यनाना अपुि घटपटादयो मिथ्या । पारमार्थिकात्मज्ञानेन वाधसम्भवात् ।
कासुचित् श्रुतिषु सृष्टिराकाशक्रमेण, कासुचित् श्रुतिषु तेजाआदिक्रमेण, कुत्रचित् सर्वं ब्रह्मैव नान्यदित्युक्तम् । सर्वासामपि श्रुतानां अवाधितप्रामाण्यात् तत्समन्वयोऽवश्यं कर्तव्यः । तथा च इदमेव सिध्यति यत् ब्रह्मैव सत्यमिति पारमार्थिकदृष्ट्या उक्तम् । आकाशादिकमसृष्टिस्तु व्यावहारिकी । अनेनापि प्रमाणेन जगतो व्यावहारिकत्वम्, न परमार्थसत्यत्वम् । यदि जगत् परमार्थसत् स्यात् तर्हि अन्यत् किमपि नास्तीति प्रतिपादिनी श्रुतिरनर्थिका स्यात् । एवञ्च सत्यानृते मिथुनीकृत्य नैसर्गिकोऽयं लोकव्यवहार इति मन्तव्यम् । अत एव अज्ञानिदृष्ट्या तत्सत्यत्वं भासते । ज्ञानिदृष्ट्या च " तस्य पिता अपिता भवति " इत्युक्तरीत्या तस्य मिथ्यात्वम् । न केवलं सर्वस्य जगतो मिथ्यात्वं श्रुत्या वोध्यते, अपि तु स्वस्यापि मिथ्यात्वं श्रुतिर्निरभिमानितया ब्रूते ।
एतस्य सर्वस्य जगतः मूलस्वरूपमनाद्यविद्या । सा च त्रिगुणात्मिका । प्रलयकालसमाप्तिवेलायां अविद्यया जीवकृतकर्मभिश्च " तदैक्षत बहुस्यां प्रजायेय " इति रीत्या परमात्मा संकल्पयति, तत आकाशादिक्रमेण सृष्ट्युद्गमः । ततश्च पञ्चीकृतमहाभूतेभ्यः शरीरादीनि प्रादुर्मवन्ति । एवं उत्पन्नशरीरे प्रविष्टं चैतन्यं जीव इत्यभिधीयते । स च  न अणुः, किन्तु व्यापकः, सर्वस्मिन् शरीरे सुखदुःखानुभवात् । स च जीवः मुक्तिपर्यन्तं स्थायी, जन्मान्तरीयकर्मजसुखदुःखसम्बन्धात् । जीवः विस्मृतकण्ठस्थचामीकरः पुरुष इव आत्मविस्मृतेः, अज्ञानात् सुखदुःखानुभवभाक् । अज्ञानं, लिङ्गशरीरम्, स्थूलशरीरञ्चेति उपाधिस्तस्य सदा प्रत्यासन्नः। अयमुपाधिः " दशमस्त्वमसीति " ज्ञानेन विस्मृतात्मस्वरूपस्य अज्ञानमिव, आत्मज्ञानेन जीवात्मपरमात्मैक्यज्ञानापरपर्यायेण नश्यति, सत्यज्ञानेन विना मिथ्याभूतदर्शनस्यानिवृतेः । एवमविद्यानाशे जीवो मुक्तो भवति । तस्य ज्ञानाग्निः प्रारब्धेतराणि सञ्चितक्रियमाणानि कर्माणि नाशयति । दशायामस्यां प्रारब्धकर्माणि भुञ्जानः स विगतशरीराद्यभिमानो जीवन्मुक्त इत्यभिवीयते । प्रारब्धकर्मावसाने देहपाते स विदेहमुक्तो भवति। इयमेव परमा मुक्तिः अस्यां जीवः परमात्मसायुज्यं नाम स्वरूपमधिगच्छति । न च मतान्तरवत् अल्पेवापि सेवकादिरूपेण भिन्नस्तिष्ठति।। यद्यपि जीवस्सदा आत्मस्वरूप एव तथापि भावरूपेणाज्ञानेन आत्मानं सुखदुःखभाजं मनुते तदैव वद्ध इत्यभिधीयते, तादृशाज्ञाननिवृत्तौ स एव मुक्त इत्युच्यते ।
वेदविहितकर्मभिश्चित्तशुद्धिः ततो ब्रह्मनिष्ठं गुरुं प्रतिशरणगमनम्, तेन तत्त्वमसीतिज्ञानोपदेशः, तच्छ्रवणमनननिदिध्यासनैः सन्यासाश्रममधिवसन् कर्माणि परित्यजन् स सुस्थिरज्ञानो मोक्षं लभते । मोक्षाप्तौ ज्ञानमेव साक्षात् साधनम् । कर्मोपासने तु चित्तशुद्धिचित्तैकाग्र्यप्रापकत्वात् परम्परितसाधने । ज्ञानकर्मसमुच्चयस्तु नेष्टः । नित्यानित्यवस्तुविवेकः, इहामुत्रार्थभोगविरागः शमदमादिसाधनसम्पत् , मुमुक्षुत्वञ्चेति साधनचतुष्टसम्पत्यनन्तरं ब्रह्मजिज्ञासा । अस्मिन् मते अनि र्वचनीयख्यातिः। प्रत्यक्षानुनानोपमानशाब्दार्थापत्यनुपलब्धि - आख्यानि षट् प्रमाणानि । अलौकिकेऽर्थे आत्मनि शाब्दमेव मुख्यं प्रमाणम् । अन्यदनुमानादि तदवष्टम्भेन प्रमाणम् । मतस्यास्य संग्राहकः श्लोकः " ब्रह्म सत्यं जगन्मिथ्या जीवो ब्रह्मैव नापरः " इति । 
 अद्वैतसिद्धान्तविकासः 
 एतादृशस्य श्रुतिसिद्धस्य प्राचीनतमस्य अज्ञविज्ञोपकारकस्य सकलदर्शन शिरोऽलङ्काररत्नभूतस्य अद्वैतवेदान्तदर्शनस्य विशेषतः प्रबलप्रतिवादिनः भेदवादिनः न्यायवैशेषिकादयः, शालिकनाथप्रभृतयः पूर्वमीप्रांसकाः आसन् । त्रयोदशशतका दनन्तरं द्वैतमतप्रवर्तका आनन्दतीर्था अपि प्रतिवादिषु अन्यतमा अभूवन् । एतादृशैस्तर्काभिमानिभिर्वैशेषिकादिभिर्युकत्याभासैः कलुषितस्य आनन्दतीर्थादिभिर्निन्दितस्य अद्वैतसिद्धान्तस्य रक्षणार्थं तर्केणैव समाधानार्थं च प्रवृत्तेषु ग्रन्थेषु चित्सुखीयन्यायमकरन्द न्यायदीपावली - अद्वैतसिद्धि - पञ्चदशीत्यादयः खण्डनखण्डस्वाद्यमित्यादयश्च विशिऽय उल्लेवार्हाः । इतरमतखण्डनपरग्रन्थेषु शैलीद्रयं दृश्यते । तत्र प्राचीना शैली तु युक्तिसहिता अर्थगाम्भीर्यवती श्रुतिमधुरशब्दविन्यासयुक्तः सम्भाषणसमा दृश्यते । नव्यनैय्यायिकैः गङ्गेशोपाध्यायप्रभृतिभिः नवीनतया न्यायशास्त्रपरिवर्तने कृते भेदवादखण्डनार्थं प्रवृत्ताः मधुसूदनसरस्वतीप्रभृतयः ब्रह्मानन्दसरस्वत्यन्ता आचार्याः " यक्षानुरूपो बलि " रिति न्यायेन परिष्कारप्रधानान्येव  वाक्यान्यारचय्य इतरमतखण्डने प्रवृत्ताः। सेयं नवीना शैली। चित्सुख - आनन्दबोधादीनां शैली तु प्राचीना । नैतच्छल्यां काठिन्यं दृश्यते । नापि विषयप्रतिपादने युक्तिकथने शब्दविन्यासे वा मन्दता दृश्यते । चाटुकारीणि दृढयुक्तिकानि न्यायोपबंहितानि च वाक्यानि तेषां वाग्मितां सिद्धान्तविमर्शक्षमताञ्च प्रतिपादयन्ति । तत्र मेदवादिनामयमाक्षेपः प्रवलतरः - यत् - जगतो मिथ्यात्वे मायिकत्वे वा तस्य असत्वप्रसङ्गः । यथा च शुक्तिरजतं असत् भवति तथा जगदपि असत् भवेत् । तथा सति सर्वप्रमाणव्यवहारस्य असम्भवस्त्यात् । किञ्चैवं सति वैदिकत्वज्ञानस्यापि मिथ्यात्वापत्तिः। सर्वदृश्यान्तर्गतस्य वेदस्य मिथ्यात्वेन तदुक्तस्य तत्त्वज्ञानस्य अर्थादेव मिथ्यात्वं स्यात् इति । सोऽयमाक्षेपः विचार्य विकल्प्य दूरीकृत आचार्यैः। असत्वमित्यस्य कोऽर्थः ? किं अत्यन्तासत्वम् ? उत किञ्चित्कालासत्वम् ? । नाद्यः, प्रत्यक्षं दृश्यमानस्य अर्थक्रियाकारिणो जगतः शशश्रृङ्गवत् अत्यन्तासत्वानुपपत्तेः । तद्धि अत्यन्तासत् यत् कदापि केनापि नोपलब्घम् , न तु जगत् तथा, सर्वैरपि प्रत्यक्षमुपलभ्यमानत्वात् । तस्मात् जगत् न असत् । नापि सत्वेन स्वीकर्तुं शक्यते । नहि प्रत्यक्षेण दृश्यमानत्वं अर्थक्रियाकारित्वं वा अत्यन्तसत्वस्य प्रयोजकम्, तथा सति स्वाप्नादार्थानां प्रत्यक्षं अनुमूयमानत्वेन, मनोरथानां अर्थक्रियाकारित्वेन च व्यभिचारात् । तस्मात् न शशश्रृङ्गदिवत् अत्यन्तासत् । नापि आत्मवत् अत्यन्तसत् , यद्रूपेण यत् निश्वितं तद्रूपं न व्यमिचरति तत् सत्यम् इति प्रतिपादितं सत्यत्वं यस्य भवति तदेव सत्यमिति वक्तव्यम् । प्रतिक्षणपरिणामिनः सततवञ्चलस्वभावस्य नियतपरिवर्तनशीलस्य अस्य जगतः तद्गतपदार्थानां वा निरुक्तसत्यत्वलक्षणानाक्रान्तत्वात् । किन्तु सद्विविक्तत्वम् , अथवा सदसद्वि लक्षणत्वं वा मिथ्यात्वमिति वक्तव्यम् । इदमेव मिथ्यात्वलक्षणं
 परिष्कृतं न्यायरत्नदीपवली तत्वप्रदीपिका - तर्कसंग्रह (आनन्दगिरीय) प्रमाणमालादिषु ग्रन्थेषु । लघुचन्द्रिकायाञ्च -
आद्यं स्यात् पञ्चपाद्युक्तं ततो विवरणोदिते । 
चित्सुखीयं चतुर्थं तु अन्त्यमानन्दबोधजम् ।। इति । 
अद्वैतसिद्धान्तस्यास्य उपनिषद् गीता - सूत्ररूपाणि श्रुति - स्मृति - युक्तिताम्ना व्यवहृतानि प्रस्थानानि महाप्रस्थ नानि प्रस्थानत्रयमिति च प्रसिद्धानि । तथा भामतीप्रस्थानं विवरणप्रस्थानमिति च अवान्तरप्रस्थानं सिद्धान्तप्रतिपादनभेदमूलकं प्रसिद्धं विद्यते, एतत् प्रस्थानद्वयमपि - 
यया यया भवेत् पुंपां व्युत्पत्तिः प्रत्यगात्मनि ।
सा सैव प्रक्रियेह स्यात् साध्वी सा चानवस्थिता ।।
इति सुरेश्वरोक्त्या मान्यतां प्राप्नोत्येव । तयोस्सिद्धान्तभेदास्तु एवं दृश्यन्ते - भामतीकारः कर्मणां विविदिषार्थत्वं वदति, विवरणकारस्तु कर्मणां विद्यार्थत्वं वदति। भामतीकारः ब्रह्मसाक्षात्कारः मनःकरणकः, न तु शब्दाकरणक इति वदति। विवरणकारस्तु वेदान्तवाक्यारूपशब्दकरणक इति वदति । भामतीकारः श्रोतव्यो मन्तव्य इत्य़ादिवाक्ये विधिर्नास्तीति वदति, विधिरस्तीति विवरणकारः । भामतीकारः निदिध्यासंन अङ्गि, श्रवणमनने अङ्गमूते इति वदति । श्रवणं अङ्गि मनननिदिध्यासने अङ्गभूते इति विवरणकारः । जीवेश्वरविषये भामतीकार अवच्छेइवादी, विवरणकारः प्रतिविम्बवादी भवति । भामतीकार अज्ञानाश्रयस्य अज्ञानविषयस्य च भेदं स्वीकरोति, विवरणकार अज्ञानस्य आश्रयविषयभेदो नास्तीति वदति । भामतीकारः प्रतिजीवं मूलाविद्या नाना इति वदति । विवरणकारः मूलाविद्या एकैवेति वदति । भामतीकार अखण्डाकारवृत्तेरूपहितं ब्रह्म विषय इति वदति। विवरणकार अखण्डाकारवृत्तेः शुद्धं ब्रह्म विषय इति वदति। भामतीकारस्य मते साधनचतुष्टये सत्यासत्यवस्तुविवेकः प्रथमसाधनम् , विवरणकारस्य मते नित्यानित्यवस्तुविवेकः प्रथमसाधनम् । स्वाध्यायोऽध्येतव्य इति विधिरर्थावबोधफलक इति भामती, अक्षरग्रहणफलक इति विवरणम् । भामती त्रिवृत्करणम् , विवरणं पञ्चीकरणं च स्वीकरोति । भामतीकारः ब्रह्मणः सर्वज्ञत्वं स्वरूपचैतन्येनेति वदति, विवरणकारः ब्रह्मणः सर्वज्ञत्वं मायावृत्तिभिरिति । भामतीकारस्य मते मनस इन्द्रियत्वमस्ति, विवरणकारस्य मते मनस इन्द्रियत्वं नास्ति । भामतीमतेे अविद्या जीवाश्रिता भवति, विवरणमते अविद्या ईश्वाराश्रिता भवति ।
	अविद्यया संसक्तं अविद्यारूपिणा अपाधिना सहितञ्च ब्रह्मणः विशुद्धं चैतन्यमेव जीव इत्युच्यते । प्रतिजीवं एकं अन्तःकरणं उपाधिर्भवति । अतश्च जीवः परिच्छिन्न अल्पज्ञ इति व्यवहारः ।
विवरणमते अन्तःकरणतत्संस्कारावच्छिन्नाज्ञानप्रतिविम्बितं चैतन्यं जीव इति, अन्तःकरणे ब्रह्मप्रतिबिम्बमेव जीव इति भवति। जीवस्वरूपविषये अवच्छेदवादः, आभासवादः, प्रतिविम्ववादश्चेति पक्षाःसन्ति - 
वाचस्पतेरवच्छिन्न आभासो वार्तिकस्य च ।
संक्षेपशारीरककृतः प्रतिविम्बं तथेष्यते ।। इति ।। 
तत्र अवच्छेदो नाम अन्तःप्रवेशः । तेन युक्त अवच्छिन्नः । यथा वा जले अन्तःप्रविष्टं आकाशं जलावच्छिन्नमित्युच्यते एवं अज्ञानाश्रयीभूतं शुद्धचैतन्यं जीव इत्युच्यते । येषां मते अविद्यया संयुक्तं चैतन्यं जीवः, स च अवच्छिन्नो वा उपहितो वा प्रतिबिम्बितो वा तेषां मते जीवावस्थायां जीवस्य एकत्वम् । अविद्याया एकत्वात् । अयमेव एकजीववादः । सुखदुःखादिवैचित्र्यन्तु उपाधिभेदात् भवति । अयमेकजीववादः भामतीकाराणां मते । बहुप्रकारया अविद्यया अविद्याकार्यबुद्ध्या वा संयुक्तं चैतन्यं जीवः, स च जीवः अविच्छिन्नो वा, उपहितः, प्रतिविम्बितो वा, भवति इति ये वदन्ति तेषां मते जीवनानात्वम् । वार्तिककारसुरेश्वरः पञ्चपादिकाविवरणकारप्रकाशात्मा संक्षेपशारीरककारसर्वज्ञात्मा प्रकटार्थविवरणकारश्व बहुजीववादिनः । न्यायमकरन्दकारानन्दबोधश्च सर्वेषामाचार्याणां सिद्धान्तं सविमर्शं निरूप्य, विषेषतः ब्रह्मसिद्धि - इष्टसिद्धिसिद्धान्तं मण्डयन् एकजीववादे श्रुतिप्रामाण्यं प्रदर्श्य युक्तियुक्ततां साधयति । 
शाङ्करभाष्यम् 
 
 
एवं निर्विशेषात र्न्लिक्षणाच्च स्वयग्प्रकाशात् ब्रह्मणः सविशेषस्य सलक्षणस्य श्च जगतः कथमुत्पत्तिः? एकस्मात् अद्वैतात् ब्रह्मणः नानात्मकस्य जगतः कथं सृष्टिः ? इत्येतेषां प्रश्नानां यथावत्
समाधनं मायास्वरूपवर्णनद्वारा आचार्यैरूपवर्णितम् । शङ्करभगवत्पादैः माया - अविद्या अज्ञान - अव्यक्तदि शब्दानां समानार्थकत्वं सूत्रभाष्ये ( १ -४ -३) उपवर्णितम् । शङ्करभगवत्पादादर्वाचीनेषु आचार्येषु माया - अविद्ययोस्तारतम्यविषये जीवेश्वरस्वरूपविषये अविद्याया आश्रयत्वे मायाया आश्रयत्वे च सिद्धान्तभेदा उद्भाविताः । भावतीकारः जीवेश्वरविषये अवच्छेदवादं अज्ञानाश्रयस्य अज्ञानविषयस्य च भेदं स्वीकुर्वन्ति । विवरणकाराश्च जीवेश्वरविषये प्रतिविम्ववादं अज्ञानाश्रयत्वविषयत्वयोर्भेदाभावं च स्वीकुर्वन्ति । भामतीमते अज्ञानविषयीकृतं चैतन्यं ईश्वरः । अज्ञानाश्रयीभूतं चैतन्यं जीव इति । विवरणप्रस्थाने ईश्वरविषये आभासवादं, जीवविषये प्रतिबिम्बवादं च स्वीकृत्य अज्ञानोपहितं विम्वचैतन्यं ईश्वरः, अन्तःकरणतत्संस्कारावच्छिन्नाज्ञानप्रतिबिम्बितं चैतन्यं जीव इति स्वीक्रियते । एवं माया - अविद्याशब्दयोस्सूक्ष्ममर्थभेदं वर्णयन्ति। जीवत्व - ईश्वरत्वप्रापकोपाधिस्तु भावरूपं त्रिगुणात्मकं सदसद्भ्यां अनिर्वचनीयं अनादि अज्ञानम् । तच्चाज्ञानं माया - अविद्याभेदेन द्विविधम् । शुद्धसत्वप्रधान मायापदवाच्यम् । मलिनसत्वप्रधानं अविद्यापदवाच्यम् । मायोपहिंत चैतन्यं ईश्वरः । अविद्योपहिंत चैतन्यं जीवः ।
	मायारहिते परमेश्वरे प्रवृत्तिर्न भवति । मायाशक्तिरहितः परमेश्वरः जगत्सर्जने अशक्तो भवति। मायाशक्तिरेव अविद्यात्मिका बीजशक्तिरव्यक्तनाम्ना व्यवह्रियते । मायेयं परमेराश्रया भवति। 
अग्नेरपृयक्मूता दाहिका शक्तिरिव माया ब्रह्मण अपृथक्मूता शक्तिः। मायेयं ब्रह्मण एकदेशवर्तिनी न तु कृत्स्नवर्तिनी । 
न कृत्स्नवृत्तिः सा शक्तिस्तस्य किन्त्वेकदेशभाक् । 
घटशक्तिर्यथा भूमौ स्निग्धमृद्येव वर्तते ।। 
पादोऽस्य विश्वा भूतानि त्रिपादस्ति स्वयम्प्रभः ।
इत्येकदेशवृत्तित्वं मायाया वदति श्रुतिः ।। (पञ्चदशी) 
"पादोऽस्य विश्वा भूतानि, त्रिपादस्यामृतं दिवि " इति श्रुतिश्व प्रतिपादयति । 
सत्वरजस्तमोगुणात्मिकेयं माया ज्ञानविरोधिनी भावरूपः पदार्थः। माया न सती । नापि असती । सदसदुभयविलक्षणतया शास्रे सा अनिर्वचनीयेति कथ्यते । ब्रह्मज्ञानेन बाधसम्भवात् माया न सती । त्रिकालावाधितत्वं हि सत्वम् । यदि माया सती स्यात् तर्हि तस्या अवाधितत्वमेव स्यात् । ब्रह्मज्ञानेन तु  माया बाध्यते। तस्मात् सा न सतीति वक्तुं शक्यते । परन्तु तस्याः प्रतीतिरनुमूयते । तस्मात् असतीति च न वक्तुं शक्यते । यतोऽसत् वस्तु न प्रतीयेत। एवञ्च मायायां उभयविरुद्धयोर्बधितत्वप्रतीतिविषयत्वरूपयोः गुणयोरनुभवगम्यत्वात् माया अनिर्वचनीयेति निश्चीयते । मायाभिधाया अविद्यात्वं प्रमाणासहिप्णुत्वमेव । तर्कादिवलेन माया न ज्ञातुं पार्यते । यथा च अन्धकारस्य साहाय्येन अन्धकारस्य प्रतीतिस्तथैव तर्कवलेन मायायाः प्रतीतिः। उक्तञ्चेदं बृहदारण्यकवार्तिके -
अविद्याया अविद्यात्वे इदमेव तु लक्षणम् ।
यत् प्रमाणासहिण्णुत्वं अन्यथा वस्तु सा भवेत् ।। इति । 
सूर्योदयकाले यथा च अन्धकारस्य नाशो दृश्यते तथा ज्ञानोदयकाले मायायाः प्रतीतिर्नश्यति - 
सेयं भ्रान्तिर्निरालम्वा सर्वन्यायविरोधिनी ।
सहते न विचारं सा तमो यद्वद् दिवाकरम् ।। 
इत्युक्तम्  नैष्कर्म्यसिद्धौ । एवञ्च प्रमाणासहिण्णुरूपिणी माया जगत उत्पत्तौ कारणमिति स्वीकर्तव्यम् । सा अव्यक्ता शक्तिः कार्यानुमेया इति विवेकचूडामणौ - 
अव्यक्तनाम्नी परमेशशक्तिरनाद्यविद्या त्रिगुणात्मिका या ।
कार्यानुमेया सुघियैव माया यया जगत्सर्वमिदं प्रसूयते ।। इति । 
मायायाः शक्तिरावरणविक्षेपभेदेन द्विधा भिन्ना भवति । आभ्यामेव शक्तिभ्यां ब्रह्मणः वास्तविकं सद्रूपमाच्छाद्यते । ब्रह्मणि असत असत्यस्य च जगतः प्रतीतिरारोप्यते ।
शक्तिद्वयं हि मायाया विक्षेपावृतिरूपकम् ।
विक्षेपशक्तिर्लिड्गादि ब्रह्माण्डान्तं जगत् सृजेत् ।। 
अन्तर्दृग्दृश्ययोर्भेदं बहिश्च ब्रह्मसर्गयोः ।
आवृणोत्यपरा शक्तिस्सा संसारस्य कारणम् ।। इति दृगदृश्यविवेके । 
एवं -
" आच्छाद्य विक्षिपति संरफुरदात्मरूपम् 
जीवेश्वरत्वजगदाकृतिभिर्मृषैव ।
अज्ञानमावरणविभ्रमशक्तियोगात् 
आत्मत्वमात्रविषयाश्रयताबलेन ।। इति । "
संक्षेपशारीरके च उक्तम् । अधिष्ठानस्य सत्यत्वापलापादनन्तरमेव अधिष्ठाने नृतनधर्मारोपो भवेत् । यथा च ऐन्द्रजालिकस्य इन्द्रजालविद्या द्रष्ट़़ृणां नेत्रेषु वास्तविकीं दर्शनक्षम्तां आच्छाद्य भ्रान्तेरुत्पादनादेव सफला इति लोके दृश्यते, एवमेव मायाया आवरणरूपा शक्तिः ब्रह्मणश्शुद्धस्वरूपमाच्छादयति । यथा च अग्निः स्वविरोधिनि जले साक्षात् प्रवेप्टु शक्तोऽप सूक्ष्मरूपेण पात्रादिद्वारा जले प्रविश्य तदीयं शैत्यं अपहनुत्य तत्र स्वीयं उष्णत्वं प्रदर्शयति तथेयं मायां सूक्ष्मतरेण स्वकीयमू रूपेण ब्रह्मणि प्रविश्य तदीयं निर्विषयं निराश्रयञ्च स्वरूमपह्नुत्य स्वीयं साश्रयत्वसविशेषत्वरूपं तत्र प्रदर्शयति । एवं लघुर्मेघः दर्शकाणां नेत्रं आच्छादयन् विस्तुतस्यादित्यमण्डलस्य यथा आच्छादकरो भवति तथा परिच्छिन्नमज्ञानं अपरिच्छिन्नस्य असंसारिण आत्मन आवारकं भवति । आवरणशक्तिरेव द्रष्टृदृश्ययोरन्तर्भेदम् , बहिर्ब्रह्मजगतोश्च भेदं उत्पादयति । यथा च रज्जौ सर्पभ्रान्तिस्सर्पज्ञानमुद् भावयति तथा मायापि अज्ञानावृते आत्मनि आकाशादिजगतिः ज्ञानमुद्भावयति । उक्तञ्च - 
मायाशक्तिर्निखिलकलनां साम्प्रतं वा विरुद्धाम् 
स्वाधिष्ठाने चितिफलयुता दर्शयत्याविमोक्षम् । 
नैल्यं व्योम्नि स्रजि विषधरो वार्यथा रश्मिपूगे 
तद्वन् मिथ्यात्मनि जगदिदं कल्पितं स्वप्नवच्च ।। 
इति  प्रत्यक्तत्वचिन्तामणौ । इयं माया अविद्यापदवाच्य़ा अज्ञानाख्या कथं जाता ? केन कारणेनास्याः ब्रह्मणा सम्बन्धो जात इति न चोदनीयम् । अविद्यायाः तत्सम्बन्धस्य च अनादित्वाङ्गीकारात् - 
" जीव ईशो विशुद्धा चित् तथा जीवेशयोर्भिदा ।
अविद्या तच्चितोर्योगः षडस्माकमनादयः ।। "
इति। मायासम्बन्धादेव प्रपञ्चोत्पत्तिः। मायेयं साश्रया सविषया च भवति । मायायां विषय ईश्वरः । आश्रयो जीवः । एतदेव साश्रयत्वं सविषयत्वञ्च ज्ञातरूपे ब्रह्मणि तदीयत्वं प्रदर्शयति । साश्रयत्वेन सविषयत्वेन च भासमानं यत् ज्ञानं तदेव महत्तत्वम् । महत्तत्वादेव प्रपञ्चस्योत्पत्तिः । 
माया (अविद्या)

नवीनास्तु अद्वैतिनः अज्ञानस्य शक्तिः - ज्ञानशक्तिः । क्रियाशक्तिस्तु आवरणविक्षेपभेदेन द्विविधा भवति । रजस्सत्वाभ्यामनभिमूतं तम आवरणशक्तिः । सा च अत्र घटो नास्तीति प्रपञ्चत्र्यवहारहेतुः। तमस्सत्वाभ्यामनभिमूतं रजो विक्षेपशक्तिः । सा च आकाशादिप्रपञ्चोत्पतिहेतुः । विक्षेपशक्तिपता अज्ञानेन उपहितस्यैव ईश्वरस्य जगत उपादानकारणता । अत्र "यथोर्णनाभिस्सृजते गृह्णते च" इति श्रुतिः प्रमाणम् । अत्र मते आवरणशक्तिपधानं अज्ञानमविद्या । विक्षेपशक्तिप्रधानमज्ञानं माया इत्युच्यते । मीयते अपरोक्षवत् प्रदर्श्यते अनयेति माया । स्वीयशक्तिवलात् प्रपञ्चमिमं प्रत्यक्षवत् सत्यमिव च प्रदर्शयतीयमित्यस्याः मायाख्या अन्वर्था भवति । इयमेव माया तमः, अविद्या, अव्यक्तमित्यादिशब्देन व्यवह्रियते - इत्यत्र 
अन्यथा भानहेतुत्वात् इयं मायेति कीर्तिता ।
आत्मतत्वतिरस्करात् तम इत्युच्यते बुधैः ।
विद्यानाश्यत्वतोऽविद्या मोहस्तत्कारणत्वतः ।
सद्वैलक्षण्यदृष्ट्रयायं असदित्युच्यते बुधैः ।। 
कार्यवत् व्यक्तताभावात् अव्यक्तमिति गीयते ।
एषा माहेश्वरीशक्तिर्न स्वकतन्त्रा परात्मवत् ।। 
इत्यादिवृद्धवचनानि प्रमाणानि । एवं जगतः सृष्टौ ईश्वरस्य अभिन्ननिमितोपादानत्वमपि मायाद्वारेणैवेत्यत्र - इन्द्रो मायाभिः पुरुरूप ईयते । सर्वं खलु इदं ब्रह्म तज्जलानिति शान्त उपासीत यतो वा इमानि भूतानि इत्यादीनि श्रुतिवाक्यानि, 
अजोऽपि सन्नव्ययात्मा भूतानामीश्वरोऽपि सन् ।
प्रकृतिं स्वां अधिष्ठाय सम्भवाम्यात्ममायया ।। 
निरुपमनिर्गुणेऽप्यखण्डे मयि चिति सर्वविकल्पनादिशून्ये । 
घटयति जगदीशजीवभेदान् अघटितघटनापटीयसी माया ।।
इत्यादिवचतानि च प्रमाणानि । एवञ्च तत्वप्रतिभासप्रतिवन्धेन अतत्वप्रतिभासहेतुः आवरणविक्षेपशक्तिद्वयवृत्ती अविद्या सर्वप्रपञ्चपकृतिरिति प्रतिपादयन्ति । प्रतिपादितञ्च विद्यारण्यस्वामिभिः पञ्चदश्याम् । विद्यारण्यस्वामी तु विवरणप्रस्थानानुयायी सन्नपि विवरणप्रस्थानानुकृलानां सिद्धन्तानां मार्गप्रदर्शकः भामतीविवरणप्रस्यानयोस्समन्वयकारी च विराजते । रजस्तमोऽनभिभूता शूद्धसत्वप्रधाना माया, रजस्तमोऽभिभूता मलिनसत्वप्रधाता अविद्या इति प्रतिपादयन् मायाप्रतिविम्बितं चैतन्यं सर्वज्ञत्वादिगुणविशिष्टं ईश्वर इति, अविद्याप्रतिबिम्बितं जीव इति - 
सत्वशुध्यविशुद्धिभ्यां मायाविद्ये च ते मते । 
मायाबिम्बो वशीकृत्य तां स्यात् सर्वज्ञ ईश्वरः ।
मायाख्यायाः कामघेनोर्वत्सौ जीवेश्वरावुभौ । 
यथेच्छं पिबतां द्वैतं अद्वैतं परमार्थतः ।। इत्यादिना पञ्चदश्यां विवरणप्रमेयसंग्रहे च प्रतिपादयति ।
मायाविषये प्राचीना :-

मायाविषये नवीना  :-

एवं विवरणमते ब्रह्मचैतन्यं ईश्वरचैतन्यं जीवचैतन्यञ्चेति चैतन्यत्रयं स्वीकृतम् । परन्तु कूटस्थचैतन्यं, ब्रह्मचैतन्यं, जीवेश्वरचैतन्यद्वयञ्चेति चतुर्विधं चैतायं स्वीकुर्वन् विद्यारण्यः भामतीविवरणप्रस्थानयोस्समन्वयकारिणं आत्मानं परिचाययति - 
कूटस्थो ब्रह्म जीवेशौ इत्येवं चिच्चतुर्विधा । 
घटाकाशमहाकाशौ जलाकाशाभ्रखे यथा ।। इति । 
साक्षिस्वरूपविषये विद्यारण्यात् प्राचीना वेदान्तिनः जीवेश्वराभ्यां भिन्नं शुद्धं चैतन्यात्मानं साक्षिणं वदन्ति । विद्यारण्यस्तु कूटस्थचैतन्यमेव साक्षीति प्रतिपादयति । यथा नृत्यशालास्थदीपः साक्षी भवति तथा कूटस्थचैतन्यमेव साक्षीति - 
नृत्यशालास्थितो दीपः प्रभुं सभ्यांश्च नर्तकीम् । 
दीपयेदविशेषेण तदभावेऽपि दीप्यते ।।
अहंकारः प्रभुः, सभ्याः विषया नर्तकी मतिः ।
तालादिधारीण्यक्षाणि दीपः साक्ष्यवभासकः ।। इति 
इमानि तत्वानि पञ्चदश्यादिषु ग्रन्थेषु वर्णितानि दृश्यन्ते ।
एवं दशम्शतकादनन्तरमुत्पन्नेषु षोडशशतकान्तेषु ग्रन्थेषु न्यायमकरन्दन्यायदीपावली - वेदान्तकौस्तुभाद्यादिषु ग्रन्थेषु भामतीविवरणप्रस्थानानां विकासः, अनिर्वचनीयख्यातिसाधनम् , अखण्डार्थत्वनिरूपणमित्यादिविषयश्च विकासं प्राप्तो दृश्यते । विशेषतश्च अद्वैतसिद्धान्तखण्डनपरस्य द्वैतसिद्धान्तमण्डनपरस्य व्यासतीर्थकृतन्यायामृतस्य तद्व्याख्यायाः न्यायामृततरङ्गिण्याः न्यायभास्करस्य च खण्डनाय प्रवृत्तानां अद्वैतसिद्धि - गुरु - सघुचन्द्रिका - विट्ठलेशीयादीनां ग्रन्थानाम् , एवं आधुनिके काले विशिष्टाद्वैतखण्डनपराणां शतदूषणीखण्डनपराणां शतभूषण्यादिग्रन्थानाञ्च आविर्भाव अद्वैतसिद्धान्तस्य विकासोन्मुखतामापादयति । 
यद्यपि अद्वैतवेदान्तस्य विकासः प्रस्थानत्रय्येति व्यपदिश्यते तथापि तस्य विकासः मद्दृष्ट्या न केवलं प्रस्थानत्रय्या परन्तु प्रस्थानत्रयीबहिर्मूतैः खण्डनमण्डनपरैर्वादप्रधानैश्च ग्रन्थैः, अद्वैतानन्दानुभाविभिराचार्यैरुपरचितैरनुभवप्रधानभा वाविष्करणात्मकैः - आत्मविद्याविलास - पञ्चदशयादिप्रकरणग्रन्थैः, अद्वैतवेदान्तसिद्धान्तप्रतिपादकैः ब्रह्मनैर्गुण्यवाद - विद्वन्मोदतरङ्गिणीप्रभृतिभिः काव्यैः योगवासिष्ठादिभिर्महाकाव्यैश्च सुतरामद्वैतवेदान्तसाहित्यं विकसितमिति तु नापरोक्षम् । 
एतद्ग्रन्थप्रयोजनम् 
संस्कृतेतरासु विशेषत आङ्गिलहिन्दीप्रभृतिषु प्रसृततरासु आधुनिकासु भाषासु प्रतिविषयमेतादृशाः ग्रन्थास्समुपलभ्यन्ते येषु तत्तद्विषयविशेषसम्बद्धाः कति ग्रन्थाः विद्यन्ते ? तत्तद्विषयस्योत्पत्तिः कुतः ? तत्तद्विषयस्य वृद्धिर्विकासश्च कथम् ? तत्तद्विषयाभिवृद्धौ दत्तचित्ता आचार्याः के के ? इति विशिष्टेतिहासो दरीदृश्यते । भौतिकविज्ञानाभिवृद्धौ वृद्धसमे नितरां विज्ञानवादिनि भारतेतरदेशे भौतिकविज्ञानविकासे स्तनन्धयशिशुकल्पस्य भाग्तवर्षस्य या कीर्तिः, या श्रद्धा,  यश्च गौरवगरिमा तस्य मुख्यं कारणं भारतीयानामद्वैतवेदान्तशास्रमिति तु निस्संशयं विदुषाम् । एतादृशस्याद्वैतवेदान्तसाहित्यस्येतिहासलेखने ममायं विशेष अभिनिवेशस्समुदपद्यत । संस्कृतेतरभाषासु प्रायशस्तत्तद्भाषायामेव तत्तद्विषयविशेषस्य इतिहासः दृश्यते । तस्मान्ममापि संस्कृतभाषायामेव संशोधनपूर्वकेतिहासलेखने महानादरस्समभूत् । तादृश भिनिवेशपूर्तये मद्रासविश्वविद्यालयसंस्कृतविभागभूतपूर्वप्राध्यक्षाः पूज्यतमाः यशश्शरीरमापन्नाः Dr. V. राघवमहोदयाः मां प्रोत्साहितवन्त इति तेषामधमणोंऽस्मि । 
ग्रन्थस्यास्य विषयप्रतिपादनसरणिः 
अद्वैतवेदान्तसाहित्यं न केवलं प्रस्थानत्रय्य परमन्यैरपि ग्रन्थैरिति पूर्वमुपवर्णितम् । तस्मात् अद्वैतवेदान्तसाहित्यस्येतिहासः उपनिषत् - गीता - सूत्रतद्भाष्य - वादप्रधान - अनुभवप्रधान - काव्यात्मकशैलीप्रधानैः ग्रन्थैर्विकसित इति अद्वैतवेदान्तस्य प्रस्थानषट्कमिति कथनमपि अविरुद्धमन्वर्थञ्च भवेत् । अत एवायं ग्रन्थः पूर्वोत्तरभागद्वयेन विभक्तः। तत्र पूर्वभागे ग्रन्थप्राधान्यं मनसिकत्य षट् परिच्छेदाः परिकल्पिताः । 
तत्र प्रथमे उपनिषत्प्रस्थानप्रधाने परिच्छेदे अद्वैतपरा उपनिषदः ग्रन्थप्रतिपाद्यवर्णनपूर्वकं तासां भाप्याणि, अद्वैताचार्यकृताः व्याख्याः, उपव्याख्याश्च निरूपिताः । उपनिषदां कालः विचारनिर्णयव्राह्य इति मत्या विमर्शकवरेण्यानांं सिद्धान्ता एव तत्र तत्र निरूपिताः । 
द्वितीये गीताप्रस्थानप्रधाने परिच्छेदे निखिलपुराणान्तर्गता अद्वैतमताविरोधिन्यः गीताः भगवगद्गीताश्च प्रतिपाद्यविशिष्टा सभाष्य़व्याख्योपव्याख्या निरूपिताः । 
तृतीये सृत्रतद्भाष्यप्रस्थानप्रधाने परिच्छेदे ब्रह्मसृत्राणि तेषां भाष्यम् , पञ्चपादिकाविवरणप्रस्थानानुयायिनः ग्रन्थाः, भामतीप्रस्थानानुयांयिनः ग्रन्थाः भाष्यस्य स्वतन्त्रव्याख्यात्मकाः ग्रन्थाः भाष्यानुसारिण्यः वृत्तयश्च सव्याख्या निरूपिताः । 
चतुर्थः परिच्छेदः प्रकारणग्रन्थप्रधानः । परिच्छेदेऽस्मिन् प्रकरणग्रन्थाः सत्र्याख्याः मार्कि अष्टाशतेभ्य अन्यूना निर्दिष्टः। प्रकारणग्रन्थरचयितारः प्रस्यानत्रयेऽपि ग्रन्थकारा दृश्यन्ते एवं बहूनां प्रकारणग्रन्थानां रचयिता दृश्यते । प्रकरणग्रन्थाश्च असंख्येया वर्तन्ते । तस्मात् पोतरुक्त्यादिदोषपरिहाराय परिच्छेदेऽस्मिन् सव्य ख्यानां प्रकरणग्रन्थानां नामनिर्देशमात्रं मातृकाक्रमेण निर्दिष्टम् । तत्तद्ग्रन्थ प्रतिपादितास्तु विषयाः तत्तद्ग्रन्थकर्तृविचारावसरे (अद्वैतग्रन्थकारपरिच्छेदे ) प्रतिपादिताः । 
पञ्चमे परिच्छेदे उद्धारमात्रज्ञाताः नाममात्रप्रसिद्धाश्च ग्रन्थाः निरूपिताः । ते च ग्रन्थाः कुत्रोद्धुता इत्यादिभिर्विवरणैस्साकं चत्वारिंशदन्यूताः प्रतिपादिताः । 
षष्ठः परिच्छेद अज्ञातकर्तृकाद्वैतग्रन्थप्रधानः । परिच्छेदेऽस्मिन् मुद्रितामुद्रितभेदेन अज्ञातकर्तृकाः पञ्चशतं ग्रन्थाः वर्णिताः । तेष्वपि समाननामानः विभिन्नग्रन्था अमुद्रिता विभिन्नहस्तलिखितपुस्तकालयेषु लभ्यन्ते । ये च मद्रासराजकीय हस्तलिखितपुस्तकालये अडयारपुस्तकालये तञ्जपुरपुस्तकालये च लभ्यन्ते, येषाञ्च वर्णनात्मकविस्तृतग्रन्थसूची विद्यते ते चाधीतास्सन्दृष्टाश्च । तेषाञ्च ग्रन्थानां मिन्नत्वाभिन्नत्वे निश्चिते च । तथापि अनिवार्यं काठिन्यं यत्र तत्र मौनीभाव एव स्वीकर्तव्यः- उदाहरणार्थम् - आनन्दबोधेन न्यायमकरन्दे ( p. No . 170) पदानां सिद्धे सङ्गतिग्रहस्थापनावसरे "विस्तरतस्तु न्यायदीपिकायां अवगन्तव्यम् " इत्युक्तम् । एष ग्रन्थः प्रकरणवशात् स्वतन्त्रो वेदान्तग्रन्थ इत्येव निर्णतुं शक्यते । परं न्यायदीपिकाख्याः बहवो ग्रन्थाः द्वैतपूर्वमीमांसाजैनन्यायशास्त्रेषु विद्यन्ते । न ते अद्वैतसाधकाः। मद्रपुरी तिरुवनन्तपुर हस्तलिखितपुस्तकालययोर्विद्यमानः न्यायदीपिकाख्यः ग्रन्थस्तु शाब्दनिर्णयव्याख्यारूप एव न तु स्वतन्त्रग्रन्थः, तस्य शाब्ददीपिका इत्येव नाम दृश्यते । एवमेव नासिकउज्जैनपञ्चावादिपुस्तकालयस्थेषु वर्णनात्मकविस्तृतसृचीरहितेषु पुस्तकेषु किमस्माभिः कर्तव्यम् । तेषां ग्रन्थानां विषये संशयस्सुस्थ एव । तस्मादज्ञातकर्तृकग्रन्थप्रधानोऽयं परिच्छेदस्सन्दृव्धः । 
एवं प्रस्थानषट्केन विकासितस्यास्य वेदान्तसाहित्यस्य ग्रन्थाः मूलव्याख्यानमेदभिन्नाः परिच्छेदषट्केन प्रतिपादिताः । ग्रन्थाः मुद्रिता उतामुद्रिता इति च निर्दिष्टाः । अमुद्रितग्रन्थानां प्राप्तिस्थानानि च निरूपितानि । प्रसिद्धतमेभ्यः मूलव्याख्याग्रन्थेभ्य ऋते प्रायस्सर्वेषामपि मुद्रितामुद्रितग्रन्थानां विषयाश्च सङ्ग्रहेण प्रदर्शिताः । व्याख्येयग्रन्थानां यावन्त्यः व्याख्या उभयविधास्सन्ति तास्सर्वा अपि व्याख्येयग्रन्थप्रस्तावे अन्यत्र व्याख्यानग्रन्थकर्तृप्रस्तावे च निर्दिष्टाः । पठितृणां प्रेक्षितृणाञ्च सौलभ्याय अनुक्रमणिकापि व्याख्येयव्याख्यान ग्रन्थानुसारं समारचिता। 
एवं षट्सु परिच्छेदेषु मूलव्याख्यानभेदभिन्ना ज्ञाताज्ञातकर्तृकभेदभिन्नाश्च ग्रन्थाः षट्सप्तत्यधिकसहस्रसंख्यापरिमिताः (1076) ग्रन्थेऽस्मिन् निर्दिष्टाः । ग्रन्थेप्वेतेषु मुद्रित्ग्रन्थाः द्वयधिकचतुश्शतसंख्याका (402) निर्दिश्यन्ते स्म अमुद्रितेष्वपि ग्रन्थलिपिदेवनागरीलिप्योर्मुद्रिता एव निर्देष्चटुं पारिताः । व्याख्याग्रन्थाः षट्षष्ठ्यधिकचतुश्शताधिकग्रन्थाः (466) निर्दिष्टाः । 
द्वितीयो भागः 
भागेऽस्मिन् द्वितीये अद्वैतवेदान्तग्रन्थकर्तार उपवर्णिताः । भागोऽयमपि पञ्चभिः परिच्छेदैः परिच्छिन्नः । तत्र प्रथमे परिच्छेदे शङ्करभगवत्पादेभ्यः प्राचीना अनिर्णीतकालादिविषयविशेषाः अद्वैताचार्या वर्ण्यन्ते स्म । द्वितीये परिच्छेद्दे शङ्करादर्वाचीनाः प्रसिद्धतमाः अद्वैताचार्यशब्दव्यपदेश्याः ग्रन्थकारास्तिथिक्रमानुसारं निर्दिष्टाः । प्रत्यद्वैताचार्यं आश्रमभेदेन नामान्तराणि, कालः, आचार्य - प्राचार्यसतीर्थ्याः, शिष्य - प्रशिष्याः, वासमूमिः, सामायेकाः, पोषक - पोष्या राजान इत्यादिकं तत्तत्प्रणीतग्रन्थप्रामाण्यादुपवर्णितम् । पश्चात् सदुपदेशात्मकगुरुशिष्यपरम्परारूपसम्प्रदायसमागता पण्डितमण्डलीमात्रप्रसिद्धा कथा अथवा किंवदन्ती च प्रतिपादिता । अनन्तरं संशोधनधुरीणानां प्राच्यप्रतीच्यभाषाविदुषां सिद्धान्ताश्च न्यरूपिषत । तत्तदाचार्यप्रणीता अद्वैतग्रन्थाः प्रतिपादिताः । ग्रन्थप्रतिपाद्यानां रूपरेखा च समाकृष्टा । तत्र तत्र आवश्यकस्थलेषु गुरुपरम्परावृक्षुः, वंशवृक्षः व्याख्यानव्याख्येयग्रन्थपरम्परावृक्षा इति बहवो वृक्षास्समारोपिताः। ग्रन्थकारा ये सन्यासिनस्समानाचार्याश्च ते अभिन्ना इति निर्णीयन्ते स्म । ये तु समानाभिख्याः भिन्नप्राचार्याश्च त अभिन्ना इति न निर्णीताः परन्तु भिन्ना इति प्रथमद्वितीयसंख्याभिर्विशेषिताः । 
यद्यपि ग्रन्थेऽस्मिन् ग्रन्थकर्तॄणां इतिहासलेखने तत्तत्प्रणीत - अद्वैतग्रन्थनिर्देशेनैवालम् , तथापि यदि तेषां इतिहासलेखने तत्तत्प्रणीतेभ्य अद्वैतेतरग्रन्थेभ्यश्च प्रमाणं लभ्यते तदा तत्तद्ग्रन्थनिर्देशोॄ ऽप्यावश्यकस्मम्पन्नः । तस्मात्तेऽपि ग्रन्था अत्र निर्दिष्टाः । परन्तु न ते अनुक्रमणिकायां स्थानमर्हन्ति। 
तृतीये परिच्छेदे ये च ज्ञातकालादिविषयविशेषाः, ये तु न प्रसिद्धतमाः, ये च न सन्यासिनः, दर्शनान्तरे प्रतिष्ठिता अपि अद्वैतेऽपि कतिपयग्रन्थकर्तारस्तेषां जीवनवृत्तान्तः ग्रन्थाश्च ग्रन्थप्रतिपाद्यसहिताः वर्णमालाक्रमेण निरूपिताः । 
चतुर्थे परिच्छेदे ज्ञातनामान अज्ञातकालादिविषयविशेषाश्शङ्करादर्वक्तना ग्रन्थकारास्तेषां ग्रान्थश्च निरूपिताः। पञ्चमे परिच्छेदे अज्ञातनामानः परन्तु तत्तद्गुरुशिष्यत्वेन मुद्रितामुद्रितग्रन्थेषु आत्मानं निर्दिशन्तो ग्रन्थकाराः तेषां ग्रन्थाश्च निर्दिष्टाः । 
ग्रन्थस्यादौ सड्क्षेपसङ्केतबोधिनी निर्दिष्टा । सग्प्रदायसमागतश्शान्तिपाठकमः, मङ्गलपाठक्रमसहितस्संयाजितः यश्च अद्वैतवेदान्तसाहित्यस्य अनिवार्य आवश्यकः कल्याणकारी च भाग इति ममाभिनिवेशः । ग्रन्थस्यान्ते अद्वैतग्रन्थग्रन्थकर्तृ सामान्यानुक्रमणिकाश्च संयोजिताः तत्र स्थूलाक्षराङ्कितपत्रसंख्याः ग्रन्थानां, ग्रन्थकतॄणां, समानाख्यानां ग्रन्थानां ग्रन्थकर्तॄणाञ्च भिन्नतां वृत्तान्तञ्च बोधयन्ति । सामान्यानुक्रमणिकायां ग्रन्थकर्तुर्विभिन्नानि तथा अन्यानि नामानि, अद्वैतवेदान्तबहिर्भूताः ग्रन्थाः, देशपुरप्रामादयः गोत्रवंशादयः, अद्वैतवेदान्तसम्बद्धाः पारिभाषिकशब्दाः, विद्या, विभिन्ना वादाश्च स्थानं लभन्ते स्म। गच्छतस्स्खलनमिति न्यायात् , मुद्रणालयानवधानाद्वा उत्पन्नाः केचन मुद्रणदेषाः शोधनेन योजनेन दूरीकृतास्सन्ति । क्वचित् क्वचित्  संस्कृत्सुलभीकरणघोषगौरवदानाय केवलं सन्धिविच्छेदस्स्वीकृतः । द्वित्रिस्थलेषु षष्टिश्ब्दस्थाने षष्ठिरिति प्रयोगः, सिंहशब्दस्थाने सिम्ह इति प्रयोगश्च स्वीकृतः । उद्धृतवाक्यानुक्रमणिका च संयोजिता । अन्ते च महानद्वैताचार्यगुरुशिप्यपरम्परावृक्षस्संरोपितः, यस्मिन् अद्वैतग्रन्थकर्तारः गुरवः, शिष्याश्च केवल स्थानमलभन्त, न मठाम्नायादिप्रसिद्धाः, नापि अद्वेंतसम्प्रदायागता अपि अकृतग्रन्थाः । नापि च संरकृतेतरभाषासु ग्रन्थप्रणेतारो वा । 
एतद्ग्रन्थरचनायामनिवार्यं कठिन्यम् ।
एवममुद्रित मुद्रित उद्रधृत - ज्ञातकर्तृक - अज्ञातरकर्तृ-व्याख्यान - व्याख्येयभेदेन भिन्नास्सर्वेऽपि ग्रन्थाः प्रबन्धेऽस्मिन् यथामति यथाशक्ति च व्यलिख्यन्त तथापि 
"कति कृतयः कति कवयः कति लुप्ताः कति चरन्ति कति शिथिलाः । "
इत्यभियुक्तं क्तिमनुसृत्य कति ग्रन्थाः मद्दृष्टिगोचरतामानीता ? कति विलुप्ता इति न ज्ञायन्ते । इदमेकं काठिन्यम् । अपरमेकं काठिन्यमिदम् यत् समाननामानस्समानकर्तृकाः विभिन्नकर्तृकाश्च बहवो ग्रन्थाः विभिन्नहहस्तलिखितपुस्तकालयेषु लक्ष्यन्ते । ये च मद्रासगजकीयअडयार - तञ्जपुर - तिरुवनन्तपुरपुस्तकालयेषु दृश्यन्ते, येषाञ्च वर्णनात्मकविस्तृतसूच्यो विद्यन्ते, तेषां ग्रन्थानां भिन्नात्वाभिन्नत्वे निर्णीते । परन्तु दूरदेशस्थानां वर्णनात्मकसूचीरहितानां पत्रव्यवहारेणापि अवेद्यानां ग्रन्थानां विषये किमस्माभिः कर्तव्यम् ? । तेषां विषये संशयस्सुस्थ एव । अत एवाज्ञातकर्तृकपरिच्छेदस्संदृब्धः । 
कृतज्ञताविष्करणम् 
एतादृशमदीयज्ञानवृद्धिकरस्य कार्यस्य सफलतायै नूतनबृात्सूचीसम्पादनाय मद्रासविश्वविद्यालयसंस्कृतविभागस्थानां निखिलविदेशस्वदेशस्थ स्तलिखितपुस्तकालय वर्णनात्मकविस्तृतसृचीनां प्रदानेन बहूपकृतवतां मद्रासविश्वविद्यालयसंस्कृतविभागस्थानां समेषाम् , अनर्घाभिप्रायोपदेशादिप्रदानेन उपकृतवतां प्राध्यापकवर्याणां Dr. K.K. Raja महोदयानाञ्च अधमर्णतमः कृतज्ञश्चास्मि ।
मदीयं प्रास्ताविकं आङ्गिलभाषायां संक्षिप्य अनूद्य उपकृतवते Dr. N. Veezinathan M.A. Ph.D , & Vedanta Siromani महाशयाय कृतज्ञतां निवेदये । 
अप्तुदितग्रन्थावलोकनाय हस्तलिखितग्रन्थप्रदानेन उपकारिभ्यः मद्रामराजकीयहस्तलिखितपुस्तकालयाध्यक्षेभ्यः, अडयारपुस्तकालयाधियेभ्यश्च मदीयां कृतज्ञतां विनिवेदये । 
मुद्रितग्रन्थावलोकनाय नियमानपि शिथिलीकृत्य असंख्यग्रन्थप्रदानेन प्रोत्साहितवतां M.M. कुप्पुस्वामिशास्रियुस्तकालयाधिकारिणाम् , मद्रपुरीसंस्कृतकलाशालाधिकारिणाम् , प्रान्तीयकलाशाला (Presidency College ) सस्कृतविभागाध्याक्षाणाञ्च घन्यवादवादी कृतज्ञश्चास्मि । 
बहुषु ग्रन्थेषु मुदापणाय सत्स्वपि विषयविशेषम् , एतादृशग्रन्थनिर्माणपरिश्रमम् , ग्रन्थयोग्यताञ्व परिशील्य मुद्रापणाय प्रकाशनाय च प्रयतितवते उपकृतवते च गुणग्राहिणे मद्रासविश्वविद्यालयाय विश्वविद्यालयानुदानायोगाय (University Grant Commission) च, एवं सुचारुरूपेण मुद्रितवते रत्नमुद्रणलयाय (Rathnam Press) मदीयां कृतज्ञतां विनिवेदये । मुद्रापणकार्यस्यास्य सफलतायै बहूयकृतवद्भ्यां संस्कृतविभागकार्यालयकार्यत्र्यापृताभ्यां चिरञ्जीविस्वामिनाथ - बाबू राजेन्द्मभ्यां सन्तु श्रेयांसि भूयांसीत्याशासे । 
अनेन मदीयेन परिश्रमेण विद्वासस्संस्कृताभिज्ञाश्च नूनं किञ्चिदपि प्रयोजनं प्राप्स्यन्तीति विश्वस्य - 
"तद्विद्वांसोऽनुगृह्णन्तु चित्तश्रोत्रैः प्रसादिभिः । 
सन्तः प्रणयिवाक्यानि गृह्णन्ति ह्यनसूयवः ।। इति "
कुमरिलभट्टवाक्येन सम्प्रार्थ्य ग्रन्थमिमं आचार्य - जगद्गुरुशङ्करभगवत्पादकमलयोस्समर्पये - 

सिद्धार्थिनामसंवत्सरम् 
कार्तिक - शुक्ल सप्तमी
षड्विंशतितमो दिवसः 
26 -11 - 1979 
मद्रास-5

इत्थम् 
विदुषां विधेयः 
R THANGASWAMI SARMA

ABBREVIATIONS 
सङ्क्षेपसङ्केतबोधिनी
ABORI : Annals of Bhandarkar Oriental Research Institute, Poona.
ALS : Adyar Library Series, Madras.
AL : Adyar Manuscripts Library, Madras. 
ALPS : Adyar Library Padmphlet Series. 
AMSS : Aryamata Samvardhini Sanskrit Series Madras. 
AMS : Advaita Manjari Series, Kumbakanam.
AOR : Annals of Oriental Research, Madras. 
AUSS : Allahabad Univesity Sanskrit Series.
ASS : Anandasrama Sanskrit Series, Poona.
AS : Asutosh Mukherjee Series, Calcutta.
BMP : Balamanorama Press, Madras - 4.
BRD : Gaekwad Oriental Research Institute, Baroda.
BSS : Banaras Sanskrit Series.
BUL : Bombay University Manuscripts Library. 
BSP : Bombay Sanskrit and Prakrit Series. 
CPB : Catalogue of Central Provinces and Berar. 
COSS : Cochin Sanskrit Series. 
CSS : Calcutta Sanskrit Series. 
CU : Calcutta University 
DC : Descriptive Catalogues.
Edn : Edition. 
GNPB : Gopalnarayan Press, Bombay.
GOML Mysore : Government Oriental Manuscripts Library, Mysore.
HIP : History of Indian Philosophy : S. N. Das Gupta. 
IA : Indian Antiquary. 
IHQ : Indian Historical Quarterly. 
IHR : Indian Historical Review. 
IOL : India Office Library, London.
JOR : Journal of Oriental Research, Madras. 
JRAS : Journal of the Royal Asiatic Society. 
MGOML : Madras Government Oriental Manuscripts Library. 
MGOMLS : Madras Government Oriental Manuscripts Library Series. 
MUSBS : Madras University Sanskrit Bulletin Series.
MUSS : Madras University Sanskrit Series. 
NCC : New Catalogus Catalogorum. 
NIA : New Indian Autiquary. 
NSP : Nirnayasagar Press, Bombay.
ORISS Mysore : Oriental Research Institute Sanskrit Series, Mysore .
PNS : Pandit New Series. Banaras.
PSB : Pandit Series Banaras. 
Q : Quoted.
RAS : Royal Asiatic Society.
SBC : Catalogues of Saraswathi Bhavan Library Banaras. 
SBTS : Saraswathi Bhavan Text Series.
SBS : Saraswathi Bhavan Series, Banaras.
SME : Sankara` s Memorial Edition, Vanivilas Press, Sri Rangam. 
SVP : Sri Vanivilas Press, Sri Rangam. 
SVPK : Saradha Vilas Press, Kumbak0nam. 
TCD : Travancore Curator` s Library Descriptive Catalogue. 
TCL : Travancore Curator` s Library.
TMPL : Travancore Maharaja` s Palace Library.
TSML : Tanjore Saraswathi Mahal Library. 
TSS : Travancore Sanskrit Series. 
VBS : Viswabharathi, Santiniketan, Calcutta. 
VNSS : Vizianagaram Sanskrit Series. 
VORIT : Venkateshwara Oriental Research Institute, Tirupati.
VVP : Vanivilas Press (Sri Rangam) 
VVSS : Vanivilas Sanskrit Series, Sri Rangam.
 
शाङ्करभाष्यपठनक्रमः ।
श्रीशङ्करभगवत्पादकृतभाष्यपाठारम्भे सम्प्रदायसमागतः शान्तिमन्त्रपठनक्रमः 
शिष्यास्सर्वे वस्त्रावगुण्ठितशरीराः पठेयुः 
शिवनामनि भवितेऽन्तरङ्गे महति ज्योतिषि मानिनीमयार्धे । 
दुरितान्यपयान्ति दूग्दूरे महुरायान्ति महान्ति मङ्गलानि ।।
स्मृते सकलकल्याणभाजनं यत्र जायते । 
पुरुषस्तमजं नित्यं व्रजामि शरणं हरिम् ।। 
१. ओं शं नो मित्रः शं वरुणः। शं नो भवत्वर्यमा। शं न इन्द्रो बृहस्पतिः। शं नो विष्णुरुरुक्रमः । नमो ब्रह्मणे । नमस्ते वायो । त्वमेव प्रत्यक्षं ब्रह्मासि त्वमेव प्रत्यक्षं ब्रह्म वदिष्यामि । ऋतं वदिष्यामि । सत्यं वदिष्यामि । तन्मामवतु । तद्वक्तारमवतु । अवतु माम् । अवतु वक्तारम् । ओं शान्तिः शान्तिः शान्तिः । 
२. सह नाववतु । सह नौ भुनक्तु । सह वीर्यं करवावहै। तेजस्वि नावधीतमस्तु मा विद्विषावहे । ओं शान्तिः शान्तिः शान्तिः ।
३. यददृन्दसामृषभो विश्वरूपः । छन्दोभ्योऽध्यमृतात् संबभूव। स मेन्द्रो मेधया स्पृणोतु। अमृतस्य देव धारणो भूयासम् । शरीरं मे विचर्षणम् । जिह्ना मे मधुमत्तग। कर्णाभ्यां भूरि विश्रुवम् । ब्रह्मणः कोशोऽसि मेधयापिहितः । श्रुतं मे गोपाय । ओं शान्तिः शान्तिः शान्तिः ।
४. अहं वृक्षस्य रेरिवा । कीर्तिः पृष्ठं गिरेरिव । ऊर्ध्वपवित्रो वाजिनीव स्वमृतमस्मि। द्रविणँ सवर्चसम् । सुमेधा अमृतोक्षितः । इति त्रिशङ्कोर्वेदानुवचनम् । ओं शान्तिः शान्तिः शान्तिः ।
५. पूर्णमदः पूर्णमिदं पूर्णात् पूर्णमुदच्यते ।
पूर्णस्य पूर्णमादाय पूर्णमेवावशिष्यते ।।
ओं शान्तिः शान्तिः शान्तिः ।
६. आप्यायान्तु ममङ्गानि वाक्याणिचक्षुःश्रोत्रं अथो बलमिन्द्रियाणि सर्वाणि । सर्वं ब्रह्मोपनिषदम् । माहं ब्रह्म निराकुर्याम् । मा मा ब्रह्म निराकरोत् । अनिराकरणमस्त्वनिराकरणं मे अस्तु । तदात्मनि निरते य उपनिषत्सु धर्मास्ते मयि सन्तु ते मयि सन्तु । ओं शान्तिः शान्तिः शान्तिः ।
७. वाड्मे मनसि प्रतिष्ठिता । मनो  मे वाचि प्रतिष्ठितम् । आविरावीर्म एधि । वेदस्य म आणीस्थः । श्रुतं मे मा प्रहासीः । अनेनाधीतेन । अहोरात्रान् सन्दधामि । ऋतं वदिष्मामि । सत्यं वदिप्यामि । तन्मामवतु तद्वक्तारमवतु । अवतु मां अवतु वक्तारमवतु वक्तारम् । ओं शान्तिः शान्तिः शान्तिः ।
८. भद्रं नो अपि वातय मनः । ओं शान्तिः शान्तिः शान्तिः ।
९. भद्रं कर्णेभिःशृणुयाम देवाः । भद्रं पश्येमाक्षभिर्यजत्राः स्थिरैरङ्गै स्तुष्टुवांसस्तनूभिः । व्यशेम देवहितं यदायुः । स्वस्ति न इन्द्रो वृद्धश्रवाः । स्वस्ति नः पूषा विश्ववेदाः । स्वस्ति नस्ताक्ष्यों अरिष्टनेमिः । स्वस्ति नो बृहस्पतिर्दधातु । ओं शान्तिः शान्तिः शान्तिः ।
१०. यो ब्रह्मणं विदधाति पूर्वं यो वै वेदांश्च प्रहिणोति तस्मै । तं ह देवं आत्मबुद्धिप्रकाशं मुमुक्षुर्वै शरणमहं प्रपद्ये । ओं शान्तिः शान्तिः शान्तिः । 
ओं नमो ब्रह्मादिभ्यो ब्रह्मविद्यासंप्रदायकर्तृभ्यो वंशऋषिभ्यो नमो गुरुभ्यः। सर्वोपल्पवरहितः प्रज्ञानघनः प्रत्यगथों ब्रह्मैवाहमस्मि । अधीहि भो भगवः, अधीहि भो भगवः । 
ततः किञ्चिद् भाष्यं पठेयुः ।
अनन्तरं दक्षिणामूर्त्यष्टक देहं प्राणमपीत्यन्तं पठन्तः प्रतिश्लोकं नमस्कुर्युः 
११. मौनव्याख्याप्रकटितपरब्रह्मतत्वं युवानम् 
वषिंष्ठान्तेवसदृषिगणैरावृतं ब्रह्मनिष्ठैः ।
आचार्येन्द्रङ्करकलितचिन्मुद्रमानन्दमूर्तिम् 
स्वात्मारामं मुदितवदनं दक्षिणामूर्तिमीडे ।। 
१२. विश्वं दर्पणदृश्यमाननगरीतुल्यं निजान्तर्गतम् 
पश्यन्नात्म निमायया बहिरिवोद्भूतं यथा निद्रया ।
यस्साक्षात्कुरुते प्रबोधसमये स्वात्मानमेवाद्वयं 
तस्मै श्रीगुरुमूर्तये नम इदं श्रीदक्षिणामूर्तये ।। 
१३. बीजस्यान्तरिवाङ्कुरो जगदिदं प्राङ्निर्विकल्पं पुनः 
मायाकल्पितदेशकालकलनावैचित्र्यचित्रीकृतम् ।
मायावीव विजम्भयत्यपि महायोगीव यः स्वेच्छया 
तस्मै श्रीगुरुमूर्तये नम इदं श्रीदक्षिणामूर्तये ।। 
१४. यस्यैव स्फुरणं सदात्मकमसत्कल्पार्थकं भासते 
साक्षात्तत्वमसीति वेदवचसा यो बोधयत्याश्रितान् ।
यत्साक्षात्करणाद् भवेन्न पुनरावृत्तिर्भवाम्भोनिधौ 
तस्मै श्रीगुरुमूर्तये नम इदं श्रीदक्षिणामूर्तये ।। 
१५. नानाच्छिद्रघटोदरस्थितमहादीपप्रभाभास्वरम् 
ज्ञानं यस्य तु चक्षुरादिकरणद्वारा बहिः स्पन्दते ।
जानामीति यमेव भान्तमनुभात्येतत्समस्तं जगत् 
तस्मै श्रीगुरुमूर्तये नम इदं श्रीदक्षिणामूर्तये ।। 
१६. देहं प्राणमपीन्द्रियाण्यपि चलां बुद्धिञ्च शून्यं विदुः
स्त्रीबालान्धजडोपमास्त्वहमिति भ्रान्ता भृशं वादिनः ।
मायाशक्तिविलासकल्पितमहाव्यमोहसंहारिणे 
तस्मै श्रीगुरुमूर्तये नम इदं श्रीदक्षिणामृर्तये ।। 
तत इमान् श्लोकान् पठन्तः नमस्कुर्युः - 
१७. श्रुतिस्मृतिपुराणानामालयं करुणालयम् । 
नमामि भगवत्पादं शङ्करं लोकशङ्करम् ।। 
१८. शङ्करं शङ्कराचार्यं केशवं वादरायणम् ।
सूत्रभाष्यकृतौ वन्दे भगवन्तौ पुनः पुनः ।। 
१९. नमः श्रुतिशिरःपद्मषण्डमार्तण्डमूर्तये ।
वादरायणसंज्ञाय मुनये शमवेश्मने ।।
२०. ब्रह्मसूत्रकृते तस्मै वेदव्यासाय वेधसे 
ज्ञानशक्त्यवताराय नमो भगवतो हरेः ।। 
२१. नारायणं पद्मभुवं वसिष्ठं शक्तिञ्च तत्पुत्रपराशरञ्च 
व्यासं शुकं गौडपदं महान्तं गोविन्दयोगीन्द्रमथास्य शिष्यम् । 
श्रीशङ्कराचार्यमथास्य पद्मपादञ्च हस्तामलकञ्च शिष्यम् 
तं तोटकं वार्तिककारमन्यान् अस्मद्गुरून् सन्ततमानतोऽस्मि ।।
२२. शङ्कराश्लेषविलसदानन्दामृतनिर्भराम् ।
विश्वोत्तंसितपादाब्जां ब्रह्मविद्यां विभावये ।।
२३. वेदान्तनिकुरुम्बेण तात्पर्येण प्रकाशितः ।
स्वात्मानन्दैकरस्येन कल्याणाय शिवोऽस्तु नः ।। 
२४. सदाशिवसमारम्भां शङ्कराचार्यमध्यमाम् ।
अस्मदाचार्यपर्यन्तां वन्दे गुरुपरम्पराम् ।। 
ततः भाष्यश्रवणं कर्तव्यम् ।
B
श्रीशङ्करभगवत्पादकृतभाष्यपाठान्ते सम्प्रदायसमागतः शान्तिमन्त्रपठनक्रमः । 
१. शं नो मित्रः शं वरुणः । शं नो भवत्वर्यमा । शं न इन्द्रो बृहस्पतिः । शं नो विष्णुरुरुक्रमः । नमो ब्रह्मणे । नमस्ते गयो । त्वमेव प्रत्यक्षं ब्रह्मसि । त्वामेव प्रत्यक्षं ब्रह्मावादिषम् । ऋतमवादिषम् । सत्यमवादिषम् । तन्मामावीत् । तद्वक्तारमावीत् । आवीन् माम् । आवीद् वक्तारम् । ओं शान्तिः शान्तिः शान्तिः । 
२. ओं सह नाववतु । सह नौ भुनक्तु । सह वीर्यं करवावहै । तेजस्वि नावधीतमस्तु । मा विद्विषावहै । ओं शान्तिः शान्तिः शान्तिः । 
३. ओं यश्चन्दसामृषभो विश्वरूपः । छन्दोभ्योऽध्यमृतात्संबभूव । स मेन्द्रो मेधया स्पृणोतु। अमृतस्य देव धारणो भूयासम् । शरीरं मे विचर्षणम् । जिह्वा मे मधुमत्तमा । कर्णाभ्यां भूरि विश्रुवम् । ब्रह्मणः कोशोऽसि मेधया पिहितः । श्रुतं मे गोपाय । ओं शान्तिः शान्तिः शान्तिः। 
४. ओं अहं वृक्षस्य रेरिवा । कीर्तिः पृष्ठं गिरेरिव । ऊर्ध्वपवित्रो वाजिनीव स्वमृतमस्मि । द्रविणँ सवर्चसम् । सुमेधा अमृतोक्षितः । इति त्रिशङ्कोर्वेदानुवचनम् । ओं शान्तिः शान्तिः शान्तिः । 
५. ओं पूर्णमदः पूर्णमिदं पूर्णात्पूर्णमुदच्यते । पूर्णस्य पूर्णमादाय पूर्णमेवावशिष्यते । ओं शान्तिः शान्तिः शान्तिः । 
६. ओं आप्यायन्तु ममाङ्गानि वाक् प्राणश्चक्षुःश्रोत्रमथो बलमिन्द्रियाणि च सर्वाणि । सर्वं ब्रह्मौपनिषदम् । माहं ब्रह्म निराकुर्याम् । मा मा ब्रह्म निराकरोदनिराकरणमस्त्वनिराकरणं मे अस्तु । तदात्मनि निरते य उपनिषत्सु धर्मास्ते मयि सन्तु ते मयि सन्तु । ओं शान्तिः शान्तिः शान्तिः । 
७. ओं वाड् मे मनसि प्रतिष्ठिता । मनो मे वाचि प्रतिष्ठितम् । आविरावीर्म एधि । वेदस्य म आणीस्थः । श्रुतं मे मा प्रहासीः । अनेनाधीतेन । अहोरात्रान् सन्दधामि । ऋतं वदिष्यामि । सत्यं वदिष्यामि । तन्मामवतु । तद्वक्तारमवतु । अवतु माम् । अवतु वक्तारम् । अवतु वक्तारम् । ओं शान्तिः शान्तिः शान्तिः । 
८. ओं भद्रं नोऽपि वातय मनः । ओं शान्तिः शान्तिः शान्तिः । 
९. ओं भद्रं कर्णेभिः श्रृणुयाम देवाः । भद्रं पश्येमाक्षभिर्यजत्राः । स्थिरैरङ्गैस्तुष्टुवाँसस्तनूभिः । व्यशेण देवहितं यदायुः । स्वस्ति न इन्द्रो वृद्धश्रवाः । स्वस्ति नः पूषा विश्ववेदाः । स्वस्ति नस्ताक्ष्यों अरिष्टनेमिः । स्वस्ति नो बृहस्पतिर्दधातु । ओं शान्तिः शान्तिः शान्तिः ।
१०. ओं यो ब्रह्माणं विदधाति पूर्वं यो वै वेदांश्च प्रहिणोति तस्मै । तं ह देवमात्मबुद्धिप्रकाशं मुमुक्षुर्वैशरणमहं प्रपद्ये। ओं शान्तिः शान्तिः शान्तिः। 
इतः परं दक्षिणामूर्त्यष्टके अवशिष्टाः श्लोकाः पठनीयाः, प्रणामश्च कर्तव्यः । 
११. राहुग्रस्तदिवाकरेन्दुसदृशो मायासमाच्छादनात् 
सन्मात्रः करणोपसंहरणतो योऽभूत्सुषुप्तः पुमान् ।
प्रागस्वाप्समिति प्रबोधसमये यः प्रत्यभिज्ञायते 
तस्मै श्रीगुरुमूर्तये नम इदं श्रीदक्षिणामूर्तये ।।
१२. बाल्यादिष्वपि जाग्रदादिषु तथा सर्वास्ववस्थास्वपि 
व्यावृत्तास्वनुवर्तमानमहमित्यन्तः स्फुरन्तं सदा ।
स्वात्मानं प्रकटीकरोति भजतां यो मुद्रया भद्रया 
तस्मै श्रीगुरुमूर्तये नम इदं श्रीदक्षिणामूर्तये ।। 
१३. विश्वं पश्यति कार्यकारणतया स्वस्वामिसम्बन्धतः 
शिष्याचार्यतया तथैव पितृपुत्राद्यात्मना भेदतः ।
स्वप्ने जाग्रति वा य एष पुरुषो मायापरिभ्रामितः 
तस्मै श्रीगुरुमूर्तये नम इदं श्रीदक्षिणामूर्तये ।।  
१४. भूरम्भांस्यनलोऽनिलोऽम्बरमहर्नाथो हिमांशुः पुमान् 
इत्याभाति चराचरात्मकमिदं यस्यैव मूर्त्यष्टकम् ।
नान्यत्किञ्चन विद्यते विमृशतां यस्मात्परस्प्राद्विभोः 
तस्मै श्रीगुरुमूर्तये नम इदं श्रीदक्षिणामूर्तये ।। 
१५. सर्वात्मत्वमिति स्फुटीकृतमिदं यस्मादमुष्मिंस्तवे 
तेनास्य श्रवणात्तदर्थमननाद् ध्यानाच्च सङ्कीर्तनात् ।
सर्वात्मत्वमहाविभूतिसहितं स्यादीश्वरत्वं स्वतः 
सिध्येत् तत्पुनरष्टधा परिणतञ्चैश्वर्यमव्याहतम् ।। 
१६. वटविटपिसमीपे भूमिभागे निषण्णम् 
सकलमुनिजनानां ज्ञानदातारमारात् ।
त्रिभुवनगुरुमीशं दक्षिणामूर्तिदेवम् 
जननमरणदुःखच्छेददक्षं नमामि ।। 
१७. चित्रं वटतरोर्मूले वृद्धाश्शिष्या गुरुर्युवा । 
गुरोस्तु मौनं व्याख्यानं शिष्यास्तु छिन्नसंशयाः ।।
१८. अड्गुष्ठतर्जनीयोगमुद्राव्याजेन देहिनाम् ।
श्रुत्यर्थं ब्रह्मजीवैक्यं दर्शयन्नोऽवताच्छिवः ।। 
१९. ओं नमः प्रणवार्थाय शुद्वज्ञानैकमूर्तये ।
निर्मलाय प्रशान्ताय दक्षिणामूर्तये नमः ।।
२०. गुरवे सर्वलोकानां भिषजे भवरोगिणाम् । 
निधये सर्वविद्यानां दक्षिणामूर्तये नमः ।। 
२१. चिद्धनाय महेशाय वटमूलनिवासिने । 
सच्चिदानन्दरूपाय दक्षिणामूर्तये नमः ।। 
C
महाप्रदोषदिने परं यथापूर्वं भाष्यपाठश्रवणं कृत्वा दशशान्तिमन्त्रपाठानन्तरं (B/1-10) श्रीशङ्करभगवत्पादपूजां कृत्वा श्वेतसर्षपेण मधुना च मिश्रं दधि दूर्वातृणानि च निवेदनं कृत्वा एते मङ्गलपाठश्लोकाः वारत्रयं पठनीयाः - 
१. अशुमानि निराचष्टे तनोति शुभसन्ततिम् ।
स्मृतिमात्रेण यत्पुंसां ब्रह्म तन्मङ्गलं परम् ।। (त्रिः)
२. अतिकल्याणरूपत्वात् नित्यकल्याणसंश्रयात् ।
स्मर्तॄणां वरदत्वाच्च ब्रह्म तन्मङ्गलं परम् ।। (त्रिः)
३. ओंकारश्चाथशब्दश्च द्वावेतौ ब्रह्मणः पुरा।
कण्ठं भित्वा विनिर्यातौ तस्मान्माङ्गलिकावुभौ ।। (त्रिः)
ओं अथ ओं अथ ओं अथ ।
अतः परं "राहुग्रस्त" इत्यारब्धाः "दक्षिणामूर्तये नम"
इत्यन्ताः (११ - २१) श्लोकाः सप्रणामं पठनीयाः । 
इति शान्तिपाठसम्प्रदायः । 

उपनिषत्प्रस्थानम्  
वैदिकसाहित्ये उपनिषदां स्थितिरन्यादृशी । औपनिषदानां महर्षीणां कार्यात्मकविश्वब्रह्माण्डस्य अभिन्नत्वे अखण्डत्वे च महान् विश्वासः दरीदृश्यते । औपनिषदा मुनयस्सुखदुःखेभ्य उदासीना दृश्यन्ते । सान्तस्यानन्तेन सम्बन्धः औपनिषदानां महर्षीणां रहस्यात्मकवाणीष्वेव प्रथमं अभिव्यक्तः । 
ब्राह्मणग्रन्थास्तु कर्मसु पुरुषं प्रेरयन्ति । उपनिषदस्तु ज्ञाने प्रेरयन्ति । जीवनात् जीवनसम्बद्धविचारे उपनिषदामैदम्पर्यं दृश्यते । वेदकालिका भारतीया ऐहिकस्यैश्वर्यस्य साधने यथा बद्धपरिकरा दृश्यन्ते, यथा च ब्राह्मणकालिका भारतीयास्स्वर्गादिपरलोकेप्सवश्व दृश्यन्ते न तथोपनिषत्कालिका भारतीयाः । परन्तु ते साधका ऐहिकैश्वर्यात् परलोकेच्छायाश्चोदासीनाः मुमुक्षवश्च दृश्यन्ते । गतिशीलैः प्राकृतिकवस्तुभिरेव जनिमतां न सम्बन्धः, परन्तु अनिर्वचनीयेन केनचित् स्थिरतत्वेनैव तेषां सम्बन्ध इत्युपनिषत्कालवर्तिनां भारतीयानां सिद्धान्तः । अतएवोपनिषदां सिद्धान्ताः विचारप्रधाना वर्तन्ते । 
उपनिषदः गम्भीरानर्थान् काव्यसुलभया गद्यपद्यात्मकशैल्या प्रतिपादयन्ति । परम् उपनिषदः नैककर्तृकाः । यत एकस्यामेवोपनिषदि शिक्षकभेदो दृश्यते । 
उपनिषदां संख्याः - 
सवोंपनिषदां मध्ये सारमष्टोत्तरं शतम् ।
सकृच्छ्रवणमात्रेण सर्वाघौघविकृन्तनम् ।। (१ - ४४ ) इति 
अष्टोत्तरशतोपनिषदां सारभूतत्वं प्रतिपादयन्त्या मुक्तिकोपनिषदा अष्टोत्तरशताधिकानां उपनिषदां सत्वे प्रमाणमप्यावेदितं भवति। परन्तु न तास्सर्वा उपनिषद अद्यावधि प्रकाशिता उपलब्धाः वा सन्ति । काश्चनोपनिषद अडयारपुस्तकालयात् प्रकाशतां नीतः। तास्वष्टोत्तरशतोपनिषत्सु दशोपनिषद ऋग्वेदीयाः, एकोनविंशत्युपनिषदश्शुक्लयजुर्वेदीयाः, द्वात्रिंशदुपनिषदः कृष्णयजुर्वेदीयाः, षोडशोपनिषदस्सामवेदीयाः, एकत्रिंशदुपनिषद अथर्ववेदीया इति ज्ञायते । 
अष्टोत्तरशतोपनिषत्स्वपि विषयप्रतिपादनदृष्ट्या त्रयोदशोपनिषद एव प्राचीनतमा इति ज्ञायन्ते । ऐतरेय - कौषीतक्युपनिषद ऋग्वेदीयाः, छान्दोग्याकेनोपनिषदस्मामवेदीयाः, तैत्तरीय नारायण - कठ - श्वेताश्वतर - मैत्रायण्युपनिषदः कृप्णयजुर्वेदीयाः, ईशावास्योपनिषच्छुक्लयजुर्वेदीया बृहदारण्यकोपनिषदपि शुक्लयजुर्वेदीया, मुण्डकमाण्डूक्यप्रश्नोपनिषद अथर्ववेदीया इति च प्रसिद्धाः । एतास्वपि दशोपनिषदामेव शङ्करभगवत्पादैर्भाष्यं कृतमिति ता एवात्र प्रधानतमास्स्वीक्रियन्ते। नृसिम्होत्तरतापिनीकौषीतकी- श्वेताश्वतरोपनिषदामपि व्याख्याः दृश्यन्ते । अतः अद्वैताचार्यैः व्याख्याताः अथवा व्याख्यातत्वेन प्रसिद्धाः सर्वा अप्युपनिषदः निर्देशार्हा इति सामान्यं नियमं मनसि कृत्वा शङ्करैरव्याख्याता अपि नारायणाश्रमि - शङ्करानन्दादिभिर्व्याख्याता अद्वैतमतप्रतिपादकास्सर्वा अप्युपनिषदः निर्दिष्टाः ।
उपनिषदां पौर्वापर्यम् - 
सर्वासूपनिषत्सु का वा उपनिषत् प्राचीना ? का वा नवीना ? इति निश्चेतुं न शक्यते । यतोऽस्मिन् विषये मतभेदास्सप्रमाणाः दृश्यन्ते ।  
१. प्रोफेसर - डायसनस्तु गद्यशैल्यां दृश्यमाना उपनिषद एव प्राचीना इति "फिलासफि आफ द उपनिषत्स् " इति ग्रन्थे अभिप्रैति । 
२. प्रो - रामचन्द्ररानडे तु "ए कंस्ट्रक्टिव सर्वे आफ़ उपनिषदिक फिलासफ़ि " नामके ग्रन्थे उपनिषदां पौर्वापर्यमेवमभिप्रैति - बृहदारण्यक - छान्दोग्यईश - केन - ऐतरेय - तैत्तरीय - कौषीतकी - कठ-मुण्डक - श्वेताश्वतर - प्रश्न - मैत्रायणि - माण्डूक्या इति उपनिषदां उत्तरोत्तरकालोत्पन्नतां प्रतिपादयन्ति। तस्य मतेन उपनिषदः पञ्चभिर्वर्गैः परिगणिताः - प्रथमवर्गे बृहदारण्यकछान्दोग्ये, द्वितीयवर्गे ईशकेनोपनिषदौ, तृतीयवर्गे ऐतरेयतैत्तरीयकौषीतक्यः, चतुर्थवर्गे कठमुण्डकश्वेताश्वतराः, पञ्चमरवर्गे प्रश्नमैत्रीमाण्ड्क्यानि चेति । एषु प्रथमवर्गः प्राचीनतमः, अन्तिमश्चार्वाचीन इति । 
३. बेलवलकर महाशयस्तु - एकस्यामेवोपनिषदि भिन्नकालिकानां रचनाविशेषाणां दर्शनात् , एकस्या एवोपनिषदः केचन भागाः प्राचीनाः, केचनार्वाचीना भवन्तीति वर्णयति । 
४. सर् - राधाकृष्णमहोदयास्तु वेदकालादारभ्य क्रिस्तोः पूर्वं (B.C) षष्ठशतकात्पूर्वावधिकः काल उपनिषदां काल इति निश्चिन्वन्ति। 
प्राचीनतमासु उपनिषत्सु दार्शनिकविचारधारायाः प्राधान्यं, अनन्तरभाविनीषु धार्मिकविचारधारायाः भक्तिधारायाश्च प्रवेशस्सन्दृश्यते । 
उपनिषत्सु शाण्डिल्य - दध्यौच - सनत्कुमार - आरुणि - याज्ञवल्क्य -  उद्दालकरैक्व - प्रतर्दन - अजातशत्रु - जनक - पिप्पलाद् - वरुण - गार्ही - मैत्रेयी - नचिकेतिःप्रभृतीनां बहूनां आचार्याणां महर्षीणाञ्च नाम दृश्यते । केचिदेतेषु सपत्नीकारसापत्याश्च दृश्यन्ते । श्वेतकेतोः पुत्रः आरुणिः, वरुणस्य पुत्रः भृगुः, याज्ञक्ल्क्यः द्विभार्य इत्यादि । औपनिषदास्सिद्धान्तास्सर्वेऽपि पतिपत्नींंसवादशैल्यां पितृपुत्रसंवादशैल्याञ्च प्रतिपादिता इति तु विशेषत अवधेयार्हः विषयः । 
उपनिषच्छब्दनिर्वचनम् - 
कठोपनिषदां प्रस्तावनाभाष्याप्रामाण्यात् , मुण्डकोपनिषदां प्रस्तावनाभाष्यप्रामाण्यात् , केनोपनिषदि ( ४- ३२) "उपनिषदमब्रूत " इति वाक्यस्य भाष्यप्रामाण्यात् , छान्दोग्योपनिषत्स्थाष्टमाध्यायाष्टमखण्ड चतुर्थखण्डिकास्थभाष्यप्रामाण्यात् बृहदारण्यकोपनिषत्प्रस्तावनाभाष्यप्रामाण्याच्च संसारबीजविनाशिनी या विद्या सा उपनिषच्छब्देन मुख्यया वृत्या बोध्यत इति निश्चीयते । तादृशविद्याप्रतिपादकत्वात् लक्षणया ईशावास्यादयो ग्रन्था अप्युपनिषच्छब्देन व्यवह्रियन्ते । 
उपनिषदां सिद्धान्ताः -
प्रायः मुख्यासूपनिषत्सु ब्रह्मस्वरूपं, तत्प्रतिपत्तय उपासनाः, काश्चित् स्वतन्त्राः, काश्चित्कर्माङ्गत्वेन, आख्यायिकासहिताः प्रतिपादिताः । ब्रह्मविद्याप्राप्तेस्साधनानि ब्रह्मविद्याजिज्ञासूनां आवश्यकगुणविशेषाः, कर्ममार्गस्य जटिलत्वम् , ज्ञानमार्गस्य सुगमत्वम् , अनासक्तकर्मपरत्वं वैराग्यञ्चेत्येवमादीनि आत्मज्ञानाप्तेस्साधनानि, न तु तर्कविचारः युक्तिवादा वा इत्यादि प्रतिपादितम्। 
उपनिषत्सु जाग्रत्तत्वानुशीलनपराः नैकविधाः कथाश्श्रूयन्ते । औपनिषदाः मननशीला मुनयः जगतः मूलतत्वानुसन्धाने बद्धश्रद्धाः दृश्यन्ते । प्रापञ्चिकानां नैकविधानां वस्तूनां विभिन्नतायां एकत्वसम्पादकं तत्वं किं स्यात् ? तादृशतत्वलाभोपायः क ? इति शङ्काकुलाः पञ्चभूतानुशीलनमार्गेण आत्मतत्वविचारपराः भूत्वा स्वस्मिन्नेव तादृशं तत्त्वं प्रत्यक्षीचक्रुः । एतादृशस्वानुभवशीलानां तेषां क्रान्तदर्शिन्याः दृष्टेः आन्तरबाह्यजगतोर्न कोऽपि भेदः विषयीबभूव। छान्दोग्योपनिषदीयया इन्द्रविरोचनकथया ( 8 - 7 - 12 ), आरुणिश्वेतकेतुसंवादरूपया न्यग्रोधकथया (6-12) च " तत्त्वमसि "  " अयमात्मा ब्रह्म " इत्येष एव सिद्धान्तः प्रतिपादितः । एवं सप्रपञ्च- निष्प्रपञ्च ब्रह्मस्वरूपर्वर्णना, मनस्तत्वविवेचना, कर्म -  सन्यास - मोक्षसिद्धिान्ताः, भारतीयविविधदर्शनमूलभूतास्सर्वेऽपि सिद्धान्ताः प्रतिपादिताः । 
एवञ्च भारतीयाध्यात्मिकविचारधाराया उपजीव्यत्वात् उपनिषदः प्रस्थानत्रये प्रथमगणनामर्हन्ति । यद्यप्युपनिषद अंसख्यास्तथापि या उपनिषद अद्वैतसिद्धान्तमूलभूताः, याश्च गौडपादशङ्करभगवत्पादादिभिरद्वैताचार्यैः व्याख्यातास्त एवात्र सव्याख्योपव्याख्या वर्णमालाक्रमेण निर्दिश्यन्ते ।। 
१. अथर्वशिखोपनिषत् - 
अस्यामुपनिषदि सकलवेदमूलभूतस्य प्रणवस्य स्वरूपं, तदीयमात्राणां देवतादयः, प्रणववाचकानां ओङ्कारतारकादिपदानां व्युत्पत्तिः, तद्ध्यानध्यातृघध्येयस्वरूपं च इत्येतत् सर्वं निरूप्यते । इयमुपनिषद् आनन्दाश्रममुद्रणालये मुद्रिता । अस्या व्याख्याः - 
नारायणाश्रमिकृता - अथर्वशिखोपनिषद्दीपिका 
अथर्वशिखोपनिषदां व्याख्यात्मकोऽयं ग्रन्थः आनन्दाश्रममुद्रणालये मुद्रितः । अस्य कर्ता श्रीनाथपौत्र रत्नाकरभट्टपुत्रः आनन्दात्मशिष्यः नारायणाश्रम इति ज्ञायते । अनेन विरचितायां " माणडूक्योपनिषद्दीपिकायां " अमुद्रितायां सरस्वतीमहालयपुस्तकालयस्थायां (1556 D.C.T.S.M.L) अानन्दगिरिर्निर्दिष्टः । आनन्दाश्रममुद्रितायां जाबालोपनिषद्दीपिकायां आनन्दात्मा अध्यात्मगुरुरिति निर्दिष्टः। आनन्दात्मा तु शङ्करानन्दस्यापि गुरुरिति नारायणाश्रमस्य कालः त्रयोदशचतुर्दशशतकम् (1275 - 1350 A.D) इति निश्चीयते ।। 
शङ्करानन्दविरचिता - अथर्वशिखोपनिषद्दीपिका 
आनन्दात्मनः विद्यातीर्थस्य च शिष्योऽयं शङ्करानन्दः विद्यारण्यस्य गुरुः त्रयोदशशतकारम्भकालवासी (1275 - 1350 A.D.) इति ज्ञायते । अमुद्रितोऽयं ग्रन्थः सरस्वतीमहालयसूच्यां (1427 TSML) अडयारपुस्तकालये बरोडापुस्तकालये च दृश्यते । शङ्करानन्दमधिकृत्याधिकं सूत्रवृत्तिप्रकरणे प्रकरणग्रन्थप्रकरणे अद्वैताचार्यप्रतिपादनावसरे च प्रतिपाद्यते ।। 
शङ्कराचार्यकृतं - अथर्वशिखोपनिषद्भाष्यम् (१) 
ग्रन्थोऽयं शङ्कराचार्यकृतत्वेन अडयारपुस्तकालयस्थे (30 B 22 ग्र 6 AL) आदर्शपुस्तके दृश्यते । अमुद्रितश्चायं ग्रन्थः । अस्य कर्ता न प्रसिद्धश्शङ्कराचायों भवितुमर्हति । तथा प्रसिद्धेरभावात् । अन्तरङ्गपरीक्षायान्तु कृतायां सर्वथा न शङ्कराचार्यकृतोऽयं ग्रन्थ इत्येव प्रतीयते । यतः - यासामुपनिषदां शङ्कराचार्यैर्व्याख्या कृता तासां न व्याख्या कृता तासां व्याख्याप्रस्तावे नारायणाश्रमिणा " शङ्करोक्त्युपजीविना " इत्युच्यते । यासां न व्याख्या कृता तासां व्याख्याप्रस्तावे " श्रुतिमात्रोपजीविना " इत्येव निर्दिश्यते । दृश्यते चात्र श्रुतिमात्रोपजीविना इति । तस्मादपि कारणात् नेयं व्याख्या शङ्कराचार्यकृता इत्येव प्रतिभाति ।। 
उपनिषद्ब्रह्मेन्द्रकृतं - विवरणम् अडयार पुस्तकालये मुद्रितम् । 
२. अथर्वशिर उपनिषदः - 
अस्यामुपनिषदि देवानां को भवानिति रुद्रं प्रति प्रश्नः, तदुत्तरेण तस्य सर्वात्मकत्वज्ञानेन तन्नतिपूर्वकं बहुधा तस्य स्तुतिः, तत्प्रतिपादकत्वेन प्रसिद्धानां " ओङ्कार - प्रणव - सर्वव्याप्यनन्तादिपदानां निर्वचनं, इत्यादयो विषयाः प्रतिपादिताः । ग्रन्थोऽयमानन्दाश्रममुद्रणालये मुद्रितः । अस्या व्याख्याः - 
नारायणाश्रमकृता - अथर्वशिर उपनिषद्दीपिका 
श्रुतिमात्रोपजीविना नारायणाश्रमिणा रचितेयं दीपिका आनन्दाश्रमुद्रणालये मुद्रिता । अस्य कालादि पूर्ववत् । 
शङ्करानन्दकृता - अथर्वशिर उपनिषद्दीपिका   
ग्रन्थोऽयं आनन्दाश्रममुद्रणालये मुद्रितः । शङ्करानन्दस्य कालादि पूर्ववत् । 
शङ्कराचार्यकृतं - अथर्वशिर उपनिषद् भाष्यम् ?
अमुद्रितोऽयं ग्रन्थः अडयारपुस्तकालये (30 B 22 ग्र 28 AL) लभ्यते । उपनिषद्ब्रह्मेन्द्रकृतं विवरणं अडयार पुस्तकालये मुद्रितम् । 
३. अमृतनादोपनिषत् - 
अस्यामुपनिषदि शास्त्राभ्यासस्य ब्रह्मज्ञानफलकत्वमुपवर्ण्य प्रत्याहारध्यानप्रणायामधारणातर्कसमाध्यभिधेयाङ्गषट्ककस्य योगस्य तत्तदङ्गलक्षणकथनपूर्वकं अभ्यसनप्रकारं फलञ्च निरूप्य प्राणादिवायूतां स्थानवर्णादिकमभिधीयते । मुद्रिता चेयमुपनिषदानन्दाश्रममुद्रणालये । अस्या व्याख्याः - 
(क) शङ्करानन्दविरचिता - अमृतनादोपनिषद्दीपिका 
नारायणविरचिता, उपनिषद्ब्रह्मेन्द्रविरचिताश्च व्याख्यास्सन्ति । मुद्रिता आनन्दाश्रममुद्रणालये अडयार पुस्तकालये च । 
४. अमृतबिन्दूपनिषत् -
अस्यामुपनिषदि मनश्शुद्धिप्रशंसापूर्वकं ब्रह्मज्ञानावाप्तये साधनमुपवर्ण्य मुक्तिस्वरूपं निरूप्य ब्रह्मज्ञानं प्रशस्यते । मुद्रिताचेयमुपनिषदानन्दाश्रममुद्रणालये । अस्या व्याख्याः - 
(क) नारायणाश्रमि विरचिता - अमृतविन्दूपनिषद्दीपिका 
श्रुतिमात्रोपजीविना नारायणाश्रमिणा विराचितेयं दीपिका आनन्दाश्रममुद्रणालये मुद्रिता । अस्य कालादि पूर्ववत् ।। 
(ख) शङ्करानन्दविरचिता - अमृतबिन्दूपनिषद्दीपिका 
व्याख्याचेयं मुद्रिताऽनन्दाश्रममुद्रणालये । कालादि पूर्ववत् । 
(ग) सदाशिवेन्द्रसरस्वतीकृता - अमृतविन्दूपनिषद्दीपिका 
अमुद्रितेयं व्याख्या मद्रासराजकीयप्राचीनहस्तलिखितपुस्तकालये (R 1492 M.G. O. M. L) लभ्यते । अस्य कर्ता कामकोटिपीठाधीशस्य महादेवेन्द्रसरस्वत्याः प्राचार्यः सदाशिवेन्द्रसरस्वती षोडशशतकीयः (1550 - 1650 A.D.) इति ज्ञायते । अनेेन आत्मानात्मविवेकोऽपि रचितः ।। उपनिषद्ब्रह्मयोगिकृतं विवरणमपि मुद्रितम् । 
५. आत्मप्रबोधोपनिषत् - 
अत्र प्रणवस्वरूपं प्रशस्य तत्सहिताष्टाक्षरेण महामन्त्रेण ब्रह्मानुसन्धानं कुर्वत उत्तमलोकाद्यवाप्त्या परमानन्दानुभवप्रकारोऽभिधीयते । ऋक्शाखीयायां अस्यामुपनिषदि आत्माद्वैत - जीवन्मुक्ततादि - प्रतिपादिकाः कारिकाः विशेषत उल्लेखार्हाः । मुद्रिता चेयमुपनिषदानन्दाश्रममुद्रणालये । अस्या व्याख्या :-
(क) नारायणाश्रमिकृता - आत्मप्रबोधोपनिषद्दीपिका 
श्रुतिमात्रोपजीविना नारायणाश्रमिणा रचितेयं व्याख्यानन्दाश्रममुद्रणालये मुद्रिता गद्यभागानामेव विद्यते । 
(ख) शङ्करानन्दविरचिता - आत्मप्रबोधदीपिका 
अमुद्रितोऽयं ग्रन्थः श्रृङ्गगिरिसूच्यां (10. C) दृश्यते ।। उपनिषद्ब्रह्मेन्द्रव्याख्या च मुद्रिता ।
६. आरुणिकोपनिषत् - 
आरुणिप्रजापतिप्रश्नप्रतिवचनरूपायामस्यामुपनिषदि सर्वसङ्गपरित्यागपूर्वकसन्यासाश्रमग्रहणप्रकारमुपवर्ण्य सन्यासिनां धर्मांश्चाभिधायान्ते परमपदावाप्तिरिति प्रतिपादितम् । मुद्रिता चेयमुपनिषदानन्दाश्रममुद्रणालये । अस्य व्याख्याः -
(क) नारायणाश्रमिकृता - आरुणेकोपनिषद्दीपिका 
" नारायणेन रचिता शङ्करानन्दपाठत " इति दीपिकायामस्यां दर्शनात् शङ्करानन्ददीपिकाया अनन्तरं रचितेयं दीपिकेति ज्ञायते । मुद्रिता चेयमानन्दाश्रममुद्रणालये । 
(ख) शङ्करानन्दरचिता - आरुणिकदीपिका 
मुद्रिताचानन्दाश्रमे । उपनिषद्ब्रह्मेन्द्रव्याख्या च मुद्रिता । 
७. ईशावास्योपनिषत् - 
अस्यामुपनिषदि चिदचित्स्वरूपस्य जगतः परमात्माधीनस्वरूपस्थित्यादिमत्वं, आदेहपातं यथाशक्ति ब्रह्मविद्याङ्गभूतकर्मयोगस्यानुष्ठेयत्वं, अविदुषो निन्दनं, परमात्मनः विचित्रानन्तशक्तिमत्वं, ब्रह्मात्मकजगदनुसन्धानस्य फलं, ईशेशितव्यवेदिनः ज्ञानयोगाद्युपदेशः केवलकर्मयोगावलम्बिनां विनिन्दनं भगवद्भक्तिनिष्ठस्यावश्यानुसन्धेयोपदेश इत्यादिकमुपवर्णितं दृश्यते । यजुर्वेदीयेयमुपनिषत् आनन्दाश्रममुद्रणालये (ASS 5) मुद्रिता । अस्या उपनिषदः रचनाकालः क्रिस्तोः पूर्वं सप्तमशतकम् (700 BC) इति विमर्शकाः। अस्याः व्याख्याः, उपव्याख्याश्च- 
(क) शङ्कराचार्यकृतं - ईशावास्योपनिषद्भाष्यम् 
मुद्रितञ्चेदं भाष्यमानन्दाश्रममुद्रणालये (ASS 5)। शङ्कराचार्यकालादि सविस्तरं प्रकरणग्रन्थप्रस्तावे अद्वैताचार्यप्रस्तावे च प्रतिपाद्यते । 
(A) आनन्दगिरिकृता - ईशावास्यभाष्यव्याख्या
व्याख्यायामस्यां "तत्वालोकः " निर्दिष्टः । भास्करमतं खण्डितञ्च । मुद्रितश्चायं ग्रन्थ आनन्दाश्रममुद्रणालये (ASS 5) ।
व्याख्याया अस्याः कर्ता आनन्दज्ञानापराभिध आनन्दगिरिः । सन्यासस्वीकारात्पूर्वं जनार्दन इत्यस्यैव नाम । गुजरातप्रान्तजोऽयं द्वारकास्थ शङ्करपीठाधीश आसीदिति प्रसिद्धिः। आन्ध्रदेशज इति साम्प्रदायिकाः । अनुभूतिस्वरूपाचार्यशुद्धानन्दयोशिशष्योऽयं अखण्डानन्दस्य प्रज्ञानानन्दस्य च गुरुः, कलिङ्गदेशाधिपतेर्नृसिम्हदेवस्य सामयिकस्त्रयोदशशतकीय (1260 - 1320 A.D) इति ज्ञायते । अदसीया अन्ये ग्रन्थाः प्रकरणप्रस्तावे अद्वैताचार्यप्रस्तावे च प्रतिपाद्यन्ते ।। 
(B) शिवानन्दयतिकृतं - ईशावास्यभाष्यटिप्पणम् 
अमुद्रितोऽयं पूर्णग्रन्थः मद्रासराजकीयहस्तलिखितपुस्तकालये (R 3882 M.G.O.M.L) लभ्यते ।
अस्य कर्ता रामनाथविदुष आचार्यस्सप्तदशाष्टादशशतकमध्यवर्ती (1650 - 1750 A.D) शिवानन्दयतिरिति ज्ञायते । अनेनरचित आनन्ददीपाख्य. प्रकरणग्रन्थः अन्यत्र निरूपितः ।। 
(ख) अनन्ताचार्यकृता - वेदार्थदीपिका 
समग्रवेदभागस्य व्याख्यात्मकोऽयं ग्रन्थः । प्रकृतिप्रत्ययविवेचनपूर्वकं सप्रमाणं प्रक्रियां निरूपयन्नयं ग्रन्थ आनन्दाश्रममुद्रणालये मुद्रितः । 
अस्य कर्त काण्वशाखीयः नागदेवभट्टपुत्रः, समग्रवेदभागव्याख्याता अनन्ताचार्यः। अनेन विधानपारिजाताख्यः ग्रन्थः (1625 A.D) काले रचितः । 
(ग) आनन्दभट्टकृतम् - ईशावास्यभाष्यम् ।
ग्रन्थोऽयमानन्दाश्रममुद्रणालये मुद्रितः । ग्रन्थेऽस्मिन् शङ्करानन्दः निर्दिष्टः। अस्य कर्ता जातवेदभट्टजाह्नव्योः पुत्रः वासुदेवपुरी - आत्मावासपूज्यपादशिष्य आनन्दभट्टश्शङ्करानन्दादर्वाचीन इति परं ज्ञायते । 
(घ) उपनिषद्ब्रह्मेन्द्रकृतम् - ईशावास्यविवरणम् 
भावेन वाक्यविन्यासेन च शाङ्करं भाष्यं पूर्णतयानुसरदिदं विवरणं अडयारपुस्तकालये मुद्रितम् ।
अस्य कर्ता प्रथमवासुदेवेन्द्रप्रशिष्यः द्वितीयवासुदेवेन्द्रशिष्यः रामचन्द्रेन्द्रसतीर्थ्यः, कृष्णानन्दगुरुः, अष्टादशशतकापरार्घकालवासी (1765 - 1850 A.D) उपनिषब्रह्मेन्द्रापरनामा रामचन्द्रेन्द्र इति ज्ञायते । 
(ङ) उवटाचार्यकृतम् - ईशावास्यभाष्यम् 
ईशावास्योपनिषदां विवरणात्मकः माध्यन्दिनशाखान्तर्गतस्य समग्रवेदभागस्य व्याख्यात्मकश्चायं ग्रन्थ आनन्दाश्रप्तमुद्रणालये मुद्रितः । अस्य कर्तां आनन्दपुरवास्तव्यस्य वज्रटभदृस्य सूनुरवन्तीपुरवासी भोजराजसामयिकः उवटाचार्य इति शुक्लयजुर्वेदसंहिताभाष्यप्रमाणाज्ज्ञायते । यद्ययं भोजराजः धारानगराधीशः सरस्वतीकण्ठाभरणशृङ्गारप्रकाशकारस्यात् तर्हि तस्य शासनकालः (1010 - 1062 A.D) इति यफिग्राफिका इण्डिकापत्रिकायाः प्रथमभाग (230 page) प्रमाणात् ज्ञायते । तस्मादुवटाचायोंऽपि एकादशशतकीय इति निर्णेतु शक्यते ।। 
(च) गोपालानन्दकृता - ईशावास्यटीका 

