\chapter{Introduction}
The growth of Literature on Advaita Vedanta can be treated in relation to the following periods - Ancient, Mediaeval, the late mediaeval and the present one. Of these the former extends upto the times of Sri Sankara and his contemporaries. In the works of these preceptors the basis for the doctrinal differences within the fold of Advaita such as the Vivarana view and the Bhamati view is found. During this period many independent works on Advaita were composed. The mediaeval period represents the growth of independent works chiefly concerned with a critical analysis of the Nyaya - Vaisesika school. 

According to Advaita, Brahman is the sole reality. Everything, else is false. The entire world appears in Brahman like silver in a piece of shell. Brahman is attributeless and so it can not be primarily signified. The Upanisad-s, therefore, state that Brahman transcends both speech and mind. Attributes like omniscience, etc., too are adventitious in Brahman. Jiva in its essential nature is identical with Brahman. Yet, owing to avidya it appears as if different from Brahman. Everyone has the cognition 'I' and it is due to the limiting adjunct mind. In the state of deep sleep, however, mind provisionally merges in avidya and hence there is no cognition of 'I' in that state. In the same manner, in the state of liberation too there is the absence of the cognition 'I' in view of the absence of mind then. Thus the cognition 'I' is only adventitious. The 'I' is the empirical jiva and in its essential nature it is one and not many. the essential nature of jiva is pure consciousness; it is all - pervasive; it is designated by the term Atman. It is three - fold as Isvara, jiva, and saksi. Ajnana or nescience is the primal cause of the entire world. Atman associated with ajnana is termed Isvara. The latter is designated as Visnu Brahma and Siva owing to the preponderance of the strands of sattva rajas, and tamas respectively. Atman is of the nature of knowledge. it is not an agent. It becomes a knower owing to its association with mind. The world becomes an abject of knowledge only through avidya. Three levels of reality are admitted, absolute, empirical, and apparent. Of these, the apparently real object like shell - silver is removed by the knowledge of the empirically real object - the shell. Pot, etc., are empirically real. And, these are annihilated by the immediate knowledgae of Brahman or atman which is absolutely real. For Brahman there is no sublation at any one of the three divisions of time - past, present, and future. Hence it is absolutely real. Pot, cloth, etc., are real only provisionally. Really they are non-real like the objects of the dream state. The objects of the dream state do not exist in the state of waking. Hence they are non-real. In the same way, pot, etc., do not exist from the stand - point of absolute. Hence athough they appear to exist, they are not real. They are annihilated by the direct knowledge of Brahman which is absolutely real.

The order of creation of the fundamental elements mentioned in one set of Upanisad-s is different from the order mentioned in another set of Upanisad-s. Some texts speak that creation proceeds in the order of space, etc., while certain other texts state that it proceeds in the order of fire, etc. Yet another set of Upanisadic texts states that everything is Brahman. All the texts are valid and hence the apparent contradiction among them must be resolved. It comes to this that Brahman alone is real and the world of objects being non-real, the Upanisadic texts do not have creation as their final import. Moreover, if the world were absolutely real, then the Upanisadic text which states that nothing exists apart from Brahman would become non-significant. The world is real from the stand-point of the ignorant. From the stand-point of the realized soul, it is non-real. The Upanisad not only speaks of the non-reality of the world; it speaks of its non-reality too. 

The Primal cause of the entire world is beginningless nescience. it consists of the three stands of sattva, rajas, and tamas. When the dissolution is about to end, the supreme self, that is, Brahman, associated with avidya and aided by merits and demerits of the individual souls contemplates thus 'Let me become many.' Then proceeds cration in the order of space, etc. Then from the quintuplicated elements spring forth physical body, etc. And, Atman which is pure consciousness gets reflected in the physical body and it is designated as jiva. The latter is not atomic in size; it is pervasive, as there is the experience of happiness and misery present throughout the body. The jiva-hood endures till liberation is attained. the jiva, like the one who has forgotten the golden ornament round one` s neck, experiences transmigration by having lost sight of its identity with Brahman. The three factors namely, nescience, subtle body, and gross body constitute the limiting adjunct; the latter is always proximate to jiva. The limiting adjunct is removed by the knowledge of the Atman, that is, the knowledge of the identity of the true nature of the individual soul and God. This is similar to the removal of ignorance in the case of one who has lost sight of his true nature of being a tenth man by the knowledge arising from the sentence 'you are the tenth man.' Never indeed is it noticed that erroneous knowledge gets itself removed without the true knowledge of the substratum. Thus when avidya is annihilated the so-called jiva becomes liberated. in his case the fire in the form of knowledge removes the accumulated and the future merits and demerits. The fructified merits and demerits alone operate at this stage. The jiva experience the fruits of fructified merits and demerits without any feeling of identity with the body, sense organs, and mind. He is termed jivanmukta. When the fructified merits and demerits are exhausted, the body falls off and he remains as Brahman; and, this is known as videhamukti. The latter alone is the superme form of liberation. Herein, the so-called jiva attains what is known as absolute identity with the supreme self. In other words, it remains non-different from Brahman. The jiva doed not remain as a servant of God as in some other systems. Although jiva is always identical with Brahman, yet owing to the influence of ajnana which is positive in nature, is considers itself to be an experient of happiness, and misery. It is designated as 'bound'. When ajnana is removed the jiva is designated as 'released'.

One has to perform one` s allotted duties as anoffering to God. There results purity of heart. Then one resorts to a preceptor who is devoted to Brahman, and who initiates him into the major-texts of the upanisad-s. He lives the life of an ascetic, gives up the performance of Karma, pursues vedantic study, reflection, and meditation, attains the knowledge of Brahman and is liberated. In order to attain liberation, knowledge is the sole means. Performance of Karma and meditative worship on the conditioned Brahman give rise to purity of heart and concentration of mind which are essential for the rise of the knowledge of self and hence they are the indirect menas to liberation. Combinations of jnana and Karma is not accepted as the means to liberation. Ascertainment of what is real and what is non-real, detachment towards enjoyment of objects here and in a hereafter, acquisition of the qualities of control of mind, control of external senses, etc., and intense desire for liberation-these constitute what is known as the four-fold aid. Then follows enquiry into the nature of Brahman.

On the epistemological side, advaita admits what is known as anirvacaniyakhyati to explain the nature of error. Perception, inference, comparison, verbal testimon, presumption, and non-apprehension are the six proofs. In regard to Brahman which is extra-empirical, verbal testimony is the primal source. Perception, etc. are pramana-s only in regard to empirically real objects. The well-known verse that summarizes the teachings of Advaita is : "Brahman is the reality; the world is non-real; and jiva is not different from Brahman".
The doctrines of Advaita have been virulently attacked by the Nyayavaisesika school which advocated the reality of difference and by the purva-mimamsa school represented by Salikanatha and others. Duing the 13th century Sri Madhvacarya the chief expounder of the Dualisitic school became one of the opponents of Advaita. In order to defend Advaita against the attacks of the above-mentioned schools, works like the Tattvapradipika, Nyaya-makaranda Nyaya-dipavali,. Prancadasi, Khandana khanda khadhya were written. Their chief aim has been to defend Advaita purely on rational grounds. The style of the text however differs. The Tattvapradipika. Nyaya-makaranda and other texts do not involve much of the technicalities of the navya-nyaya. But the text Advaita Siddhi and its commentary famously known as Brahmanandiya abound in logical subtleties to refute the doctrine of difference advaocated by the dualistic school. 

The advocates of the dualistic school raises the powerful objection that if the world were admitted to illusory then it would be an absolute nothing. Just as shell-silver is an absolute nothing, in the same way, the world too would be an absolute nothing. If this were the case, there would be the impossibility of the conduct of worldly activity and also the function of pramana-s. Moreover the knowledge of reality admitted to be derived from scripture would also be illusory. It is because the entire world of objects is admitted to be illusory by the Advaitins. The scripture too is included in the world of objects and hence it is also illusory. It comes to this, that the knowledge derived from the scripture too is illusory. 

The above objection has been carefully analysed and ultimately rejeceted by the Preceptors of Advaita. What exactly is meant by the term asat ? Is it an absolute nothing? Or, non-existence for a given period of time ? It cannot be the former because the world which is hiven in perception, which is adapted to practical needs of life cannot be an absolute nothing. That alone is an absolute nothing which does not come within the range of one` s perception. The world comes within the range of one` s perception and hence it cannot be an absolute nothing. 

Nor can the world be treated as real. Never indeed perceptuality or adaptability to practical needs of life could serve as the criterion of reality. In that case dream objects would become ral as they are perceived, Thus. the world is not an absolute nothing like the horn of a hare; nor is it real like the self. That alone is real which does not leave out its essential nature in the three divisions of time past, present and future. The world does not come within the sphere of the above definition and hence it is not real. 

Mithyatva or illusoriness is sadviviktatva - that which is distinct from reality; or, it is sadasadvilaksanatva - that is, that which is different from being both real and an absolute nothing. In the texts like the Nyayaratna -dipavali, Tattvapradipila, the Tarkasangraha of Anandagiri, the Pramanamala and other texts of Ananda Bodha, the definitions of mithyatva have been set forth, analysed and established on Sound basis. 

In the Brahamanandiya otherwise known as Laghucandrika five definitions of mithyatva are identified as one mentioned in the Pancapadika, the second and the third advocated in the vivarana the fourth ser forth in the Tettvapradipika and the fifth one referred to in the works of Ananda Bodha.

It has already been said that the philosophy of Advaita is based upon the three texts of the Upanisad-s, the Bhagavad - Gita, and the Brahma-Sutra. During the post-Sankara period doctrinal differences within the fold of Advaita developed to suit the needs of the aspirants of different levels of intellect. These doctrinal differences constitute glory to the system of Advaita. In the language of Suresvara any theory within the fold of Advaita is to be taken as valid if it facilitates the easy understanding of the concepts of Advaita ultimately leading to the realization of the non-dual self. Of these different schools within the fold of Advaita, the school of Bhamati and the school of Vivarana are important. 

The Bhamati school maintains that 1) Karma-s are useful for giving rise to the desire to know the self; 2) the realization of Brahman arises through the instrument of mind; 3) there is no injunction in the vedic text 'Atman should be realized'; for that purpose it should be heard, reflected and meditated upon; 4) meditation is the principle factor and vedantic study and reflection are its subsidiaru factors; 5) jiva is the consciousness that is limited by ajnana and Isvara is the consciousness that transcends the limiting adjunct; 6) the locus and content of ajnana is different; 7) the primal nescience is manifold; 8) it is only Brahman that is conditioned by the vrtti that is the content of the direct knowledge of Brahman; 9) the first factor in the fourfold aif in the discrimination between what is real and non-real; 10) the injuctiove text 'One` s own recension of veda must be studied' has for its fruit the knowledge of the meaing of the veda; 11) the world-creation is explained by adapting the theory of triplication; 12) the Omniscience of Brahman is derived from the essential nature of Brahman; 13) mind is a sense organ; and, 14) avidya is located in jiva. 

The Vivarana school on the other hand maintains that 1) Karma is responsible for the rise of the knowledge of Self; 2) the direct knowledge of Brahman arises from major texts of the Upanisad-s; 3) in the text Atman Should be realized, etc.,there is restrictive injuction; 4) vedantic study is the principal factor and reflection and meditation are the subsidiary features; 5) jiva is the reflected image of Brahman in avidya, mind etc.; 6) the locus and content of avidya is the same; 7) the primal nescience is one only; 8) the content of the direct knowledge of Brahman is pure Brahman; 9)the first factor in the fourfold aid is the discrimination between what is eternal and non-eternal; 10) the injunctive text 'One` s own recension of the veda must be studied' has for its fruit the learning of the veda by rote; 11) the world creation is explained on the basis of quintuplication theory; 12) the omniscience of Brahman is based on the modes of avidya;  13) mind is not a sense organ; and, 14) avidya is located in pure consciousness.

It is only pure consciousness which is Brahman when delimited by avidya, attains to the state of jiva. For each and every jiva, there is a distinct limiting adjunct-mind. Hence there is the usage that jiva is limited, ignorant, etc. This is the Bhamati view. According to the vivarana view it is only.

In regard to the nature of jiva, there are three theories known as avaccheda-vada, abhaasa-vada and pratibimba-vada. The first one is advocated by Vacaspati Misra, the second by Suresvera and the third by Sarvajnatman and others in his line of thinking.

Avaccheda means only immanence. That which is immanent is known as avacchinna or conditioned. just as the space immanent in water is stated to be conditioned by water, in the same way pure consciousness which serves as the locus of ajnana is jiva. If jiva is admitted to be consciousness associated with or conditioned by or reflected in avidya which is one, then jiva according to this view is one only. This theory is therefore known as eka-jivavada. The distinct experience of happiness, misery, etc., is due to the multiplicity of mind, the limiting adjunct. If jiva is admitted to be pure consciousness associated with mind, then multiplicity of mind answers for the plurality of souls. Suresvara, Padmapada, Prakasatman, Sarvajnatman and Anubhutisvarypa advocate the plurality of jiva-s. Ananda Bodha in his Nyaya-makaranda examines the view-points of his predecessors, approves specifically the view-point of the Brahma-siddhi and the Ista-siddhi, Shows that the eka-jivavada has the authority of scriptures and proves its soundness on logical grounds. 

In order to explain the rise of the world characterized by duality from Brahman which is non-dual and attributeless, the preceptors of advaita, following the teachings of the Upanisad-s, introduce the principle of maya. Sri Sankara in his commentary on the Brahma-sutra (i, iv, iii) uses the words maya, avidya, ajnana, avyakta, etc., as synonyms. Advaitins of the post-Sankara Isvara and jiva etc. Vacaspatimisra Advocates avacchedavada to explain the nature of jiva and Isvara and also maintains difference between the locus and content of avidya. Prakasatman, the author of the Vivarana advocates the pratibimbavada to explain the nature of jiva and isvara and maintains the identity of the locus and content of avidya. Other preceptors maintained the distinction between maya and avidya. The primal factor which is termed prakrti is termed maya when the pure sattva guna is predominant in it. It is termed avidya when the impure sattvaguna is predominant in it. And the consciousness conditioned by maya is termed Isvara. And that which is conditioned by avidya is jiva. 

There cannot be any activity in Brahman if it is free from relation to avidya. It is avidya or maya that is referred to by the term avyakta; it is the seed of the universe. It is located in Brahman. Just as the power to burn is not separable from fire in the same way the power of maya is not separable from Brahman. As Vidyaranya in his Pancadasi states : Just as the power to transform into pot exists only in the wet clay forming part of earth in the same way the power to transform into the world, that is, maya exist in a part as it were of Brahman. This maya consisting of the three strands of sattava, rajas and tamas is opposed to the knowledge of Brahman. It is positive in nature in the sense that it is not absence of knowledge. Since maya is sublated by the direct knowledge of Brahman it is not real. Since it is given in perception in the form 'I am ignorant' it is not an absolute nothing. An Absolute nothing will never be presented in any cognition. It cannot be real and an absolute nothing at once. For, it is a discrepant notion. Hence avidya or maya is said to be indeterminable either as real or absolute nothing. The very nature of avidya as Suresvara puts it, lies in its not coming within the range of any pramana.

This maya must be admitted to be the transformative material cause of the world. The world is indeterminable either as real or an absolute nothing. It is an effect ; it must, there-fore, have a cause. And, the cause must belong to the same order as that of the effect. The effect is indeterminable. Hence, the cause also must be indeterminable and it is maya or avidya. Sri Sankara in his Vivekacudamani states, 'this power belonging to the supreme self is termed avyakta. It is beginningless; it consists of the three strands of sattva, rajas, and tamas. It is to be inferred from the nature of the effect, namely, the world. 

This maya consists pf a two-fold power of veiling and revealing. By the former power, it  conceals the absolute aspect of Brahman ans by the latter it makes Brahman appera as God, soul and the world. In the work Drgdrsyaviveka attributed to Vidyaranya this distinction is clearly made out. Therein, it is stated that the power of revealing known as viksepa-sakti brings into existence the entire world. The power of veilling known as avarana sakti conceals the distinction between the self and mind etc., and Brahman and the world of objects. In the samksepa sariraka it is stated : avidya owing to the strength of having pure consciousness as its locus and content, comes to have a veiling  and transfiguring faculty. It veils the ever-luminous Brahman and then projects it illusorily in the form of embodied souls, God and the phenomenal world. 

Illusory projection presupposes concealment of the true nature of the substratum. Just as a juggler first conceals the right perception of the onlookers by charm etc., and the manifests illusory objects in the same way, maya too first conceals the true nature of Brahman and then projects it illusorily as God, souls and the world. It might be asked as to how maya which is limited could conceal Brahman which is all pervasive. This is answered by making a reference to the analogy of cloud which though limited conceals the orb of the sun which is more pervasive than that. In the pratyaktattvacintamani it is said, that just as blue colour is ascribed to the sky,snake to the rope, water to the barren land lit by the sun` s rays, in the same way the entire world is superimposed upon Brahman. 

According to Advaita, six factors are admitted to be beginningless. 1. the soul, 2. God, 3. pure consciousness, 4. the distinction between soul and God, 5. Avidya, and, 6. the association of avidya with pure consciousness. 

The concept of maya would be intelligible only if we refer to its locus and content. It is only the pure - conciousness that is the locus and content of avidya and in this aspect it is viewed as the mahat-tattva and from the latter there proceeds the world creation.

The Neo-Advaitins are of the view that the powers of ajnana are jnanasakti and kriyasakti. jnana-sakti is only sattva guna that is not overpowered by rajoguna and tamoguna. The kriyasakti is two-fold as avarana-sakti and viksepasakti. Of these, the former is the tamoguna not overpowered by rajoguna and sattavaguna. The latter is rajoguna not subdued by tamoguna and sattvaguna. The pure consciousness conditioned by ajana associated with the viksepasakti is the material cause of the world. According to this view ajnana associated with the avarana sakti is termed avidya and when it is associated with the viksepa sakti it is termed maya. This word is significant as maya is that by which the world is directly presented to us as real. It termed tamas,, avidya, avyakta, etc. It is termed maya because it is responsible for the manifestation of Brahman as something else. It is termed tamas because it conceals the true nature of Brahman. tamas because it conceals the true nature of Brahman. It is termed avidya because it is annihilated by the direct knowledge of Brahman. It is termed moha because it is the oause of delusion. It is termed asat as it is different from being real. Since it is not clearly manifested like pot, etc., it is termed avyakta. It is not independent like the supreme self. 

In respect of the creation of the world Brahman is the material and the efficient cause of the world only through maya or avidya. The sruti text 'The supreme self assumes mainfold forms through maya which possesses innumerable powers' confirms the above view. In the Bhagavad-Gita, the Lord states, 'Although I am the lord of all beings and not subject to birth yet depending upon maya located in me I appear to be born.' The proves that maya is related to Brahman. In the pancadasi Vidyaranya a follower of the vivarana school brings about a synthesis between some of the views of the vivarana and those of the Bhamati. He states that maya is the primal power in which the pure sattva guna is predominant by not being overpowered by rajo - guna and tamo -guna. Avidya is the primal power wherein the impure sattva -guna is predominant by being overpowered by the rajpguna and tamoguna. This view, vidyaranya sets forth in his vivaranaprameya-sangraha too.

According to the vivarana school, conciousness is threefold as Brahman, jiva and Isvara. Vidyaranya adds one more namely immutable reality, that is, kutastha - caitanya. 

In ragard to the nature of the witness-self some of the preceptors who preceded vidyaranya maintained the view that consciousness which transcend both jiva and Isvara, is the witness-self. vidyaranya however argues that it is the kutastha that is, the witness-self. Kutastha is different from jiva, the latter is being witnessed by the kutastha. vidyaranya explains this on the analogy of a lamp in a theatre. The lamp illumines three objects. the owner of the show, the members of the audience, and the danseuse, and the lamp continues to throw light even when the above three are absent after the show is over. In the same way, the kutastha which is consciousness conditioned by the gross and the subtle body and which is the substratum of jiva illumines three factors. 

1. The jiva who is comparable to the owner of the show.\\
2. Teh objects of the world which are comparable to the members of the audience, \\
3. The mind which is comparable to the danseuse. \\
The kutastha manifests the jiva and its activities, the objects of the world and the mind with all its variations in the state of waking and in that of dream. In the state of deep sleep when the mind provisionally merges in avidya, when jiva is not determinately perceived as 'I' and when the objects of the world are not manifested, the kutastha shines manifesting only avidya. Vidyaranya is of the view that the reflected image of the kutastha in mind is jiva; and, the kutastha which remains as prototype consciousness is the saksi and jiva and saksi are different. Vidyaranya` s contribution to Advaita lies in his synthesis of both the Vivarana and Bhramati views.

Till the end of the 16th century the advaitic preceptors were chiefly concerned with elucidating the view - points of the Vivarana and the Bhamati and also proving the theory of anirvcaniya-khyati and of the impartite sense. After this period preceptors of Advaita had for their main target of attack, the writers on Dvaita, who in their works such as Nyayamra, Nyayamrta-tarangini, Nyaya-Bhaskara and the like vehemently refuted the doctrines of Advaita. Advaitasiddhi of Madhusudana Sarasvati, the Guru-candrika and the Laghucandrika of Brahmananda and the work of Vitthalesa deserve special mention in the history of Advaita literature. Many a writer during the present century wrote refuting the charges levelled against Advaita by the Visistadvaitins in their texts like the Sri Bhasya and satadusani.

The doctrines of Advaita are confirmed not only by the three text of the Upanisads, the Bhagavad-Gita and the Brahmasutra but by texts like pancadasi, and Yoga-vasista and those which critically examine the objections raised by the other schools of thought against Advaita and thereby defend Advaita. 

\section {THE AIM OF THE PRESENT WORK}
In languages other than Sanskrit, such as English and HIndi and other modern languages, there are encyclopaedic works concerning almost every subject providing bibliographical information regarding the number of works available on a particular subject, its earliest beginnings, development, the preceptors responsible for its growth, etc. At a time when foreign countries have made tremendous progress in the fields of science and technology, our country is just opening her eyes like a child as it were. If our country is just opening her eyes like a child as it were. If our country has earned any name and fame among the comity of nations it is chiefly due to her contribution to philosophical thinking through the system of Advaita. The present attempt is to give an authentic and systematic history of the Advaita literature from the earliest beginnings. It is the normal practice among writers of literary history to write accounts of works in a language, in that particular language only. This is the reason why I too have taken keen interest in writing a history of Advaita works in Sanskrit language itself. I must at this point record my grateful thanks to the former Professor of Sanskrit, University of Madras, the late Dr. V Rahavan, Who enthused me first to undertake this work. 

\section{THE METHOD ADOPTED TO INTRODUCE THE SUBJECT MATTER IN THIS WORK}
A history of the Advaita literture should consist not only of the texts on the triple path, the so-called "Prasthanatraya" namely, the Upanisads, the Brahmasutra and the Bhagavadgita but also of all allied works. In fact, apart from the commentaries on the three Prasthanas, the Advaita literature has three more 'Prasthanas' as it were, namely, the ''Vada''-tracts or disquisitions, expositions on "anubhava" (the experience of the oneness of the self), and works couched in literary style. It will therefore be in the fitness of things to call this Advaita system, a fusion of "six" Prasthanas. For convenience, the present literary history is arranged in two parts. part I divided into six chapters in view of the vastness and importance of the Advaita texts that have appeared down the centuries. 

Chapter One is mainly concerned with the Upanisads. It takes into account only those Upanisads that are Advaitic in character, and present an account of the topics dealt with in them, the commentaries extant on them, the commentaries written by teachers of Advaita and their sub-commentaries. Discussion on the date of the Upanisads being outside the purview of the present analysis, the dates given by renowned scholars have been adopted here. 

Chapter Two deals with the Bhagavadgita and its allied literature. In this chapter, an account of the so-called "gita" s found in several puranas that are favourable to Advaita is given. A detailed account of the commentaries on these "gitas" and also on the well-known Bhagavadgita (which forms part of the Mahabharata) also finds a place in this chapter. 
Chapter Three is devoted to third Prasthana, namely, the Brahmasutra. This chapter gives accounts of the Brahmasutra, Sankara` s commentary thereon, texts following the Vivarana school and the Bhamati school, independent treatises, minor commentaries and the like.
Chapter Four deals mainly with independent treaties on Advaita, including their commentaries, which are about 800 in number. Some of the authors of these treatises are also found to have written on all the three Prasthanas. In some cases, a single author is seen to have composed several treatises. These treatises are many in number. So as to avoid repetition we have given in this chapter, only, the names of independent treatises and their commentaries in the order in which their basic texts appear. The topics dealt with in each of these works are to be found in the chapter dealing with writers of the Advaita texts. 

Chapter Five is concerned with works whose titles alone are known to us and which are known only through quotations by other writers. There are about 40 works belonging to these categories. This chapter also provides exact references to these quotations. 

Chapter Six is concerned with anonymous texts on Advaita Vedanta. There are about 500 such anonymous works both in manuscript and print. Of these again, some have the same title and are distributed over different libraries of the country. all the works mentioned in the Descriptive Catalogues of the Government Oriental Manuscripts Library, Madras. Adyar Library and Research centre, Madras, and the Sarasvati Mahal Library, Tanjore, are consulted for the present work. The identification of texts having the same name has been done in some cases; but wherever such identification proved impossible, we had to leave the matter as it is. Thus for instancem Anandabodha in his Nyayamakaranda (p. 170), while discussing the capacity of words in apprehending the relation with an already existing entity, states : "Details of this view may be understood from the Nyayadipika." From the context the Nyayadipila appears to be an independent Advaitic treatise. But there are many works bearing the title "Nyayadipika". in the Dvaita, Purvamimamsa, Jaina and Nyaya schools, which are not, of course, in favour of Advaita. There is a work called Nyayadipika in the manuscripts Libraries at Madras and Trivandrum; but it has turned out to be only a commentary on the sabdanirnaya, but not an independent work on Advaita school. Further. it is known as "Sabdadipika" also. Likewise, what could be done in the case of works deposited in the libraries at palces like Nasik, Ujjain and the Punjab, but which do not have any Descriptive Catalogue or index? Doubts in regard to such works do indeed persist. Hence a separate chapter had to be devoted for works whose authors are not known or identified. Thus in six chapters we have tried to cover works on Advaita which are basic and those which are in the form of commentaries. Already it has been pointed out that some texts are published and some, unpublished. We have also mentioned here the places where these unpublished works could be traced. Excepting for the well-known basic texts and their commentaries, all other works, printed or not, are briefly analysed in this Volume. 

As for the commentaries of the basic texts, whether they are published or not, they are mentioned in the section dealing with the authors, and also while discussing the basic texts. For the benefit of the readers and cursory viewers, an index of the basic texts and their commentaries is also included here. Thus in six chapters of Part I on the whole, 1,076 works of both known and unknown authorship, independent works, basic texts and their commentaries are mentioned. 

Of these, the number of printed texts is 402, Even among the printed texts, only those printed in Grantha and Devanagari characters are covered in the present survey. The total number of commentaries included in this book in 466. 

Part II concerns itself with authors of texts on Advaita. This part also it divided into five chapters. Chapter One deals with the pre-Sankara preceptors of Advaita, whose exact date, place and other biographical details could not be ascertained. Chapter Two attempts a chronological account of some well-known post-Sankara writers who can be identified as preceptors of Advaita. On the basis of available textual data, we have provided such details as the names of these writers, their titles reflection their station in life, the period in which they flourished, the names of their preceptors, grand-preceptors, classmates disciples and grand-disciples, places from which they hailed, names of their contemporaries, royal patrons and the like. This is followed by brief account of important anecdotes or events, of these lives of the preceptors, which are handed down from tradition, and are as such, popular in scholarly circles. There after are stated their settled opinions and specific concepts as given in the accounts of well-known critics of East and West. This is followed by works composed by these writers, along with a brief outline of the topics dealt with therein. We have also provided several pedigress (vamsa-vrksas) or  genealogical tables showing the line of the preceptors, their family descendants, the works handed down as basic texts and their commentaries. the works handed down as basic texts and their commentaries. As for writers who are known to be samnyasins, having the same name and same preceptors, they are identified as one and the same author, but not as several. But those who have identical names but different preceptors, grand-preceptors and so on, are treated as different but not identical. Such authors are indicated in the text by Roman figures (I,II, etc.,) after their names, to mark out their mutual difference.

In the present survey which is devoted to a literary history of Advaita Strictly speaking, only texts bearing on Advaita Should find a place. But, if the works of some non-Advaita writers are also found to yield valuable information in arranging our biography, such works are also consulted and mentioned here. But they are not included in the table of contents. 

In Chapter Three are given the life-accounts and works of those writers who, though well-known authorities in systems other than Advaita, have written a few works in Advaita also. This chapter also contains names of those writers whose dates and other details are either not known, or little knownm, and those who are not known as ascetics (smanyasins).

In Chapter Four are dealt with the works of the post-Sankara authors whose names alone are known to us, but not dates and other details. In Chapter Five is given the list of authors whose names are not known, but who are known through their own references as the disciples or preceptors of others. 

A list of abbreviations is given at the beginning of this volume, and at the end are appended general indexes of the names of works and authors. Letters in bold print used in these lists bring out the difference between texts and authors, as also the distinction among the work of the same title, and authors  of the same name. The general indexes furnish such details as: the other names by which an author is known, his work on other schools of thought, the village/town part of the country from which he hailed, his family, lineage, etc. list of technical termsof the Advaita system, list of meditational exercises (vidyas) list of different theories of knowledge, etc. There is a saying in Sanskrit : "the question of slipping does arise only when one starts walking". ['gacchatah khalu skhalanam']. Errors of commission and omission that have crept into the present volume due to inadvertance or in the process of printing have been rectified and indicated in the Errata. At the end is given a big stemma codicum [vamsa-vrksa] indicating the preceptor-disciple tradition obtaining in the Advaita system. It must be pointed out that only authors of texts on Advaita, their teachers and disciples alone are included in this pedigree, but not the names of those who are known through the records and chronicles of the Advaitic maths as authors of works which are passed on an oral tradition, but not actually recorded on paper or palm-leaf. Also this "tree" does not contain the names of those who worte on Advaita in languages other than sanskrit. 

Some problems encountered in writing this work:
Naturally in a work of this nature and magnitude, there are hound to occur several gaps and loopholes. I have tried, to the best of my ablility, to include in this volume, works both unpublished, and published, of known or unknown authorship, whether they are commentaries on other basic texts, or have formed the basic texts for other sub-commentaries, etc. But, as Nilakantha Diksita has rightly observed, "how many works have been composed so far in this world! How many poets were actually born! How many works are lost to posterity, how many are known and available to us. How many are badly damaged beyond recovery." I do not know as to how many works I could actually lay my hands on for the present work, and as to how many are actually lost or escaped my notice. This is one great problem I had to face. The next one is concerning the identity of several works and authors, in terms of the name and title. This problem is to some extent, settled with the help of the several Descriptive Catalogues available in the Libraries at Madras, Adyar, Tanjore and Trivandrum. But in so far as the works in libraries in distant places in other parts of the country are concerned, there being no Descriptive Catalogues or indexes, nothing could be done because of their inaccessibility, even through correspondence with those in charge of such libraries. So our doubts regrading the identity of a few works are still unresolved. This is the main reason in devoting a separate chapter for works of unknown authorship in the present volume. 

\section{Acknowledgements:} 
It is now my pleasant duty to express to all the members of the sanskrit Department, University of Madras for having laid at my disposal all the Descriptive Catalogues, Indexes and other refernce material needed for my work. I ma grateful to Prof. K. Kunjunni Raja who gave me the benefit of his advice and opinion. I am also thankful to the Curator of the Government Oriental Library, Madras, and the Director of the Adyar Library and Research centre, Madras, for permitting me to consult the unpublished manuscripts in their collections. I am also thankful to the authorities of the Mr. Kuppuswami Sastri Research Institute, Mylapore, Madras, the Sanskrit College, Madras, and the Sanskrit Departments of several local colleges, who graciously allowed me to use their libraries, even relaxing their rules, purely prompted by their love of Sanskrit. 

Although there are many works awaiting publication, the authorities of the University of Madras have magnanimously undertaken my work for printing and publishing under their aegis, out of their regard for the subject-matter and in view of the pains taken by me in preparing this compendium. I am particularly beholden to them for the kind help and encouragement extended to me in this regard. My thanks are also due to the authorities, the University Grants Commission, New Delhi, for their financial assistance in bringing out this volume. I am also thankful to messrs. Rathnam Press, Madras, for their neat and pleasant get-up and printing. I would also like to convey my appreciation and blessings to chis. K. Swaminathan and Babu Rajendran, office assistants of the Sanskrit Department, University of Madras, who have been very helpful in matters connected with the printing of this book, and I also thank Dr. N Veezhinathan Dr. M Narasimhachary for translating the Sanskrit Introduction in to English and Dr. C. S. Sundaram for assistiong in publication. 

I shall feel amply rewarded in my labours if this volume meets the approval of the world of scholars in general, and lovers of Sanskrit literary history in particular. Perhaps I cannot make a better appeal to scholars than in the words of Kumarila Bhatta the great mimamsaka, who, several centuries ago, declared:

"Let Scholars therefore, listen to this (work of mine) with ears in the form of their minds properly and gracefully attuned Good men in fact, will approve, without any taint of jealousy, the words of merit and wisdom."

I shall now, in all humility, offer this work at the lotus feet of the renowned world-teacher, the great Adi Sankara. ]

Madras - 600 005.\\
Siddharthi Varsa\\
Kartikasukla Saptami\\
26-11-1979\\
R THANGASWAMI SARMA

