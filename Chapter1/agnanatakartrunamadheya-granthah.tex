\chapter{।। विभिन्नपुस्तरकालयस्थाः मुद्रिता अमुद्रिताश्च अज्ञातकर्तृनामधेया अद्वैतसिद्धान्त प्रधाना ग्रन्थाः ।।}
1.अखण्डात्मप्रकाशिका - 
ग्रन्थोऽयं मैसूरकूरघसूच्यां अज्ञातकर्तृनामा निर्दिष्टः । परन्तु मद्रपुरीराजकीयहस्तलिखित प्राचीनपुस्तकालये ( R. 3891 MGOML) गोपालानन्दसरस्वतीकृत इति दृश्यते । किमयं ग्रन्थः स एव? उतान्य इति न निश्चेंतु पार्यते ।।
2. अखण्डत्वनिरूपणम् - 
	ग्रन्थोऽयं काकिनाडा तेलुगु अकाडमिसूच्यां दृश्यते ।
3. अखण्डविषयः - ग्रन्थोऽयं आर्फट सूच्यां दृश्यते ।।
4. अखण्डार्थः - सव्याख्योऽयं ग्रन्थः बरोडापुस्तकालये लभ्यते ।। 
5. अखण्डार्थत्वनिरूपणम् - ग्रन्थोऽयं आनन्दाश्रमसूच्यां दृश्यते ।।
6. अखण्डार्थत्ववादः - ग्रन्थोऽयमुपनिषद्ब्रह्मेन्द्रसूच्यां दृश्यते ।।
7. अखण्डात्मदीपिका - ग्रन्थोऽयमार्फट-सूच्यां दृश्यते ।।
8. अखण्डैकरसवाक्यार्थः -
	अज्ञातकर्तृकोऽयं ग्रन्थ अद्वैतसिद्धान्तं निर्विशेषवादांश्च वर्णयन् मधुसूदनसरस्वतीकृतान् अद्वैतसिद्धिसिद्धान्तान् सङ्गृह्णाति । अत एव मधुसूदनसरस्वत्या अनन्तरभावी ग्रन्थोऽयमिति निश्चीयते । ग्रन्थोऽयमडयार पुस्तकालये अमुद्रित उपलभ्यते ।। 
9. अज्ञानस्वरूपम् - अमुद्रितोऽयं ग्रन्थः मद्रासराजकीयपुस्तकालये लभ्यते ।
10. अद्वैतचिन्तामणिटीका -
	ग्रन्थोयमुज्जैन् सूच्यां दृश्यते । अद्वैतचिन्तामणिनामानस्त्रयः ग्रन्था अत्र ग्रन्थे पूर्वमुपवर्णिताः। तेषु कुमारभवस्वामिना कृतः कश्चन, श्रीदेवकृत अपरः। रङ्गोजिभदृकृत अन्यः। तेषु कस्येयं व्याख्येति निर्णयं कर्तुं न शक्यते । 
11. अद्वैतचूडामणिः - 
	अस्य कर्ता चिद्धनानन्दगुरुशिष्य इत्येव ज्ञायते । ग्रन्थेऽस्मिन् बालयोगी शिवानन्दः पूर्णानन्दः, पुरुषोत्तमयोगी च नमस्कृताः । ग्रन्थकारोऽयं वाञ्छ्यनाथं प्रणमति। तस्मात् चोलदेशजस्सम्भाव्यते । प्रकरणग्रन्थोऽयमडयारपुस्तकालय उपलभ्यते ।। 
12. अद्वैततत्वरत्नम् - ग्रन्थोऽयं R. A. शास्त्रिसूच्यां दृश्यते । 
13. अद्वैतदीपः - ग्रन्थोऽयं कामकोटिसूच्यां दृश्यते ।
14. अद्वैतदीपिका - 
	न्यायमतखण्डनपरोऽयं ग्रन्थः मद्रासराजकीय हस्तलिखितपुस्तकालय उपलभ्यते । अनन्तकृष्णशास्त्रि-कामाक्षी-नृसिम्हाश्रमि-रामेश्वरभट्टकृताभ्य अद्वैतदीपिकाभ्यः भिद्यते च । 
15. अद्वैतनिर्णयः - 
	ग्रन्थोनयमज्ञातकर्तृनामा भारतपुरसूच्यां निर्दिष्टः। मद्रासराजकीयपुस्तकालये तु अच्युतमुनिकृत इति दृश्यते। नव्यन्यायशैल्यां न्यायमतखण्डनपर इति तु द्वयोरपि समानम् ।।
16. अद्वैतपञ्चरन्त व्याख्या -
	अमुद्रितोऽयं ग्रन्थः तिरुवनन्तपुरराजगृह पुस्तकालयसूच्यां दृश्यते । 
17. अद्वैतपञ्चरत्न व्याख्या दीघितिः -
	अज्ञातकर्तृनामायं ग्रन्थ मद्रासराजकीय पुस्तकालये लभ्यते । 
18. अद्वैतप्रकरणम् -
	श्रुतिप्रामाण्यात् ब्रह्मणो विज्ञानानन्दमयत्वं प्रसाधयन्नयं ग्रन्थः न्यायवैशेषिकादिप्रतिपादितं आत्मनो गुणवत्वसिद्धान्तं खण्डयन् इच्छादयो गुणाः बुद्धिनिष्ठः, न त्वात्मनिष्ठा इति प्रतिपादयति । अमुद्रितोऽयमनादिरनन्तः एकविंशत्युत्तरशतपद्यात्मकश्च ग्रन्थः अडयार पुस्तकालय उपलभ्यते ।। 
19. अद्वैतप्रकरणम् सव्याख्यम् -
	ग्रन्थोऽयं विश्वभारतीशात्निनिकेतन पुस्तकालये लभ्यते ।। 
20. अद्वैतब्रह्मसुधा - 
	ग्रन्थोयं योगीन्द्रशान्ताश्रमशिष्यकृत इति ज्ञायते । बरोडापुस्तकालये तु अज्ञातकर्तृनामा वरींवर्ति । 
21. अद्वैतबोधप्रकरणम् -
	ग्रन्थोऽयं मद्रासराजकीयहस्तलिखितपुस्तकालये लभ्यते । 
22. अद्वैतबोधामृतम् -
	अमुद्रितोऽयं ग्रन्थ मद्रासराजकीयपुस्तकालये (R. 1393 B. MGOML) तिरुवनन्तपुर पुस्तकालये च लभ्यते ।। 
23. अद्वैत ब्रह्मसिद्धिविनियोगसंग्रहः -
	ग्रन्थोऽयमार्फट सूच्यां दृश्यते । आनन्दपूर्णकृत इति रायलआसियाटिक बंगाल सूच्यां दृश्यते । कोऽयमानन्दपूर्ण इति न ज्ञायते ।।
24. अद्वैतमकरन्दसारः - 
	लक्ष्मीधरकृताद्वैतमकरन्दसारात्मकोऽयं ग्रन्थः कल्कत्ताआसियाटिक सोसाइटिसूच्यां दृश्यते ।। 
25. अद्वैतमकरन्दसङ्गग्रहः - ग्रन्थोऽयमार्फट सूच्यां दृश्यते ।।
26. अद्वैतमतसारः - ग्रन्थोऽयं "रैसू" सूच्यां दृश्यते ।।
27. अद्वैतमुक्ताकलापः - ग्रन्थोऽयं विद्याविलासमुद्रणालये मद्रासनगरे मुद्रितः ।। 
28. अद्वैतमञ्जरी - सूत्रवृत्तिरूपोऽयं ग्रन्थ बम्बईनगरे मुद्रितः।
29. अद्वैतमञ्जरी व्याख्या - ग्रन्थोऽयं सरस्वतीमहालये लभ्यते ।
30. अद्वैतरत्नकोशः - सूत्रवृत्तिरूपोऽयं ग्रन्थ मैसूरसूच्यां दृश्यते । 
31. अद्वैतरत्नदीपिका - अभेदरत्नदीपिकापरनामायं ग्रन्थः टैलर्सूच्यां दृश्यते ।
32. अद्वैतरत्नदीपिकाव्याख्या - अभेदरत्नदीपिकाव्याख्यापर नामायं ग्रन्थः मद्रासराजकीयहस्तलिखितपुस्तकालये लभ्यते । 
33. अद्वैतरत्नप्रकाशः - 
	ग्रन्थोऽयं ग्रन्थपुरसूच्यां दृश्यते । अनेन समाननामा कश्चन ग्रन्थ अमरेश्वर शास्त्रिकृतः पूर्वमस्माभिः प्रतिपादितः। किं स एवायमन्यो वा इति निर्णेतुं न पार्यते । 
34. अद्वैतरत्नप्रकाशिका -
	शतश्लोकैः पूर्णोऽयं ग्रन्थ अद्वैतप्रक्रिया आनुष्ठुभैः पद्मैः प्रतिपादयति । प्रकारणग्रन्थोऽयं मद्रपुरीजकीयहस्तलिखितपुस्तकालये (R. 5050 MGOML) मैसूरपुस्तकालये च लभ्यते ।। 
35. अद्वैतरसमञ्जरी व्याख्या - ग्रन्थोऽयं सरस्वतीमहालये (7144 DC TSML) लभ्यते । 
36. अद्वैततत्वविवेकः - ग्रन्थोऽयं बरोडापुस्तकालये लभ्यते । 
37. अद्वैतविजयः - ग्रन्थोऽयं बरोडापुस्तकालये (7994 BRd.) लभ्यते । 
38. अद्वैतवेदान्तप्रकरणम् - ग्रन्थोऽयं मैसूरसूच्यां दृश्यते ।
39. अद्वैतवेदान्तविषयः - ग्रन्थोऽयं मद्रासराजकीय पुस्तकालये (D 14 946 MGOML) लभ्यः। 
40. अद्वैतवेदान्तरहस्यकारिकावलिः - ग्रन्थोऽयमपि मद्रासपुस्तकालये (R. 929 B. MGOML) लभ्यते । 
41. अद्वैतवेदान्तसंक्षेपः - ग्रन्थोऽयं मद्रास पुस्तकालये (R. 59 C. MGOML) लभ्यते । 
42. अद्वैतशतकम् - 
	प्रकरणग्रन्थोऽयं अज्ञातकर्तृकः अनन्तशयनपुस्तकालये लभ्यते । पूर्वमस्माभिः गङ्गाधरकृतः कश्चनाद्वैतशतकनामा ग्रन्थः निर्दिष्टः ।
43. अद्वैतश्रुतिभेदनिरासः -
	ग्रन्थोऽयं मद्रासराजकीयपुस्तकालये (R. 2921 C, MGOML) लभ्यते ।
44. अद्वैतसारः -
	ग्रन्थोऽयं कुम्भघोणस्थशङ्करमठपुस्तकालये मुद्रित उपलभ्यते । 
45. अद्वैतसाम्राध्यम् - 
	ग्रन्थोऽयमानन्दाश्रमसूच्यां अज्ञातकर्तृकः निर्दिष्टः । पूर्वमस्माभिः कृष्णानन्दसरस्वती कृतः कश्चन एतत्समाननामा ग्रन्थः निर्दिष्टः । 
46. अद्वैतसिद्धाञ्जनम् -
	वादप्रधानोऽयं ग्रन्थः रामानुजीयसिद्धान्तान् श्रुतप्रकाशिकाकारमनूद्य खण्डयति अद्वैतसिद्धान्तं स्थापयति च । ग्रन्थोऽयं मद्रासराजकीयपुस्तकालये (R. 2291 MGOML) लभ्यते । 
47. अद्वैतवेदान्तसारः - ग्रन्थोऽयं "रैस " सूच्यां दृश्यते । 
48. अद्वैतसिद्धान्तप्रकाशः - ग्रन्थोऽयं मैसूर सूच्यां दृश्यते । 
49. अद्वैतसिद्धान्तसूत्रमुक्तावलिः (N. 44 - 8 C MGOML) लभ्यते । 
50. अद्वैतसिद्धान्तविजयः - (R. 165 B MGOML) लभ्यते ।
51. अद्वैतसिद्धान्तविजय व्याख्या - (R 165 B. MGOML) लभ्यते। 
52. अद्वैतसिद्धान्तसंक्षेपः - 
	एते ग्रन्था मद्रासराजकीयपुस्तकालये अमुद्रिता उपलभ्यन्ते ।
53. अद्वैतसिद्धिव्याख्या - (23. C. 4. AL) अडयारपुस्तकालये दृश्यते । 
54. अद्वैतसिद्धिपत्रम् -
	अद्वैतसिद्धिरत्नमित्यपरनामायं ग्रन्थः अद्वैतसिद्धिगतमिथ्यात्वानां दलप्रयोजनोपवर्णन पूर्वकं परिष्कारं कुर्वन् अद्वैतसिद्धिक्रोडपत्रमिति नाम्ने अर्हते । अमुद्रितोऽयं अडयार पुस्तकालये लभ्यते ।। 
55. अद्वैतसिद्धिव्याख्या - सारचन्द्रिका - असम्पूर्णोऽयं ग्रन्थः पञ्जाबसूच्यां दृश्यते । 
56. अद्वैतसिध्युपन्यासः - ग्रन्थोऽयं राजेन्द्रलालासूच्यां दृश्यते । 
57. अद्वैतसुधा - सूत्रवृत्तिरूपोऽयमज्ञातकर्तृनामा ग्रन्थ अडयारसंक्षिप्तसूच्यां दृश्यते । 
58. अद्वैतसुधानिधिः - ग्रन्थोऽयमुपनिषद्ब्रह्मेन्द्र सूच्यां दृश्यते । 
59 अद्वैतसूक्तभाष्यम् - "नासदासीदि"ति सूक्तस्याद्वैतपरमिदं भाष्यं R.A शास्त्रिसूच्यां दृश्यते । 
60. अद्वैतानुभवप्रकाशः - अमुद्रितोऽयं ग्रन्थः सरस्वतीभवनपुस्तकसूच्यां दृश्यते । 
61. अद्वैतानुभूतिः - उज्जैन-बरोडा सूच्यां दृश्यते । 
62. अद्वैतोपनिषत् - ग्रन्थोऽयं मध्यप्रान्तीय बरार्ग्रन्थसूच्यां दृश्यते । 
63. अधिकरणसंख्याश्लोकाः - 
	ग्रन्थोऽयं अज्ञातकर्तृनामा अद्वैतपराणि अधिकरणानि वर्णयन् वर्णकविवरणञ्च प्रतिपादयन् कल्कत्तायां (C.S.S.I) मुद्रितः । परन्तु कृष्णानुभूतिकृत इति पूर्वमस्माभिरुपपादितः ।
64. अध्यात्मप्रकरणम् - ग्रन्थोऽयं बरोडासूच्यां दृश्यते ।
65. अध्यात्ममालिका - बरोडासूच्यां दृश्यते । 
66. अध्यारोपप्रकरणम् - अमुद्रितोऽयं ग्रन्थः मद्रासराजकीयहस्तलिखितपुस्तकालये बरोडापुस्तकालये अडयारपुस्तकालये च लभ्यते । 
67. अनिर्वचनीयख्यातिसिद्धिः - सादिरनन्तोऽयं ग्रन्थः सरस्वतीमहालये लभ्यते ।। 
68. अनुभवपञ्चकम् - ग्रन्थोऽयममुद्रित अडयार पुस्तकालये लभ्यते ।
69. अनुभवपञ्चविंशतिः - अज्ञातकर्तृकोऽयं ग्रन्थस्सरस्वतीमहालयवर्णनात्मक ग्रन्थसूच्यास्त्रयोदशे भागे (7751 DC TSML Vol. XIII) मुद्रितः ।।०
70. अनुभववेदान्तविषयः (R 203 C MGOML)
71. अनुभवोल्लासः -अडयारपुस्तकालयाल्लभ्योऽयं ग्रन्थ अद्वैतात्मानुभवं 612 पद्यैः विवृणोति ।। 
72. अनुभूतिप्रकाशिका - ग्रन्थोऽयं बरोडापुस्तकालये लभ्यते ।
73. अनुभूतिरत्नमालिका -
	अडयारपुस्तकालयस्थेऽमुद्रितेऽस्मिन् ग्रन्थे विद्यमानानि 58 पद्यानि बोधार्याग्रन्थे दृश्यन्ते। तस्मात् विषयप्राधान्यं मनसि कृत्वा केनापि निर्दिष्ट स्यादिति ज्ञायते । 	
74. अपरोक्षानुभूतिप्रदीपिका - मध्यप्रान्तीयबरार्ग्रन्थसूच्यां दृश्यते। विद्यारण्यीयाद्भिद्यते वा? 
75. अपवादप्रकरणम् - अमुद्रितोयं ग्रन्थ आत्मव्यतिरेकेण प्रपञ्चासत्तां प्रतिपादयति । 
76. अभिव्यक्ता - गोविन्दानन्देनाथवा रामानन्देन कृतायाः भाष्यरत्नप्रभायाः व्याख्यात्मकोऽयं ग्रन्थ अडयारपुस्तकालयेऽमुद्रित उपलभ्यते । 
77. अमनस्कम् - स्वयम्बोध इत्यपरनामायं ग्रन्थः बाम्बेयुनिवर्सिटिहस्त लिखितपुस्तकालये, भारतकार्यालयपुस्तकालये लन्दननगरस्थे च लभ्यते ।। 
78. अर्थप्रकाशिका - सुरेश्वर पञ्चीकरणवार्तिकव्याख्यात्मकोऽयं ग्रन्थः बरोडा पुस्तकालये लभ्यते ।। 
79. अवधूताष्टकम् - ग्रन्थोऽयं बरोडापुस्तकालये लभ्यते ।। 
80. अवस्थात्रयोल्लासः - अमुद्रितोऽयं ग्रन्थः मद्रासराजकीयहस्तलिखितपुस्तकालये लभ्यते ।। 
81. अऴिमुक्तनिरुक्तिसारः - अमुद्रितोऽयं ग्रन्थ रायल आसियाटिक सोसाइटि बम्बई पुस्तकालये लभ्यते ।। 
82. अष्टादशश्लोकगीतामहावाक्यमन्त्रः - ग्रन्थोऽयं सरस्वतीमहालये लभ्यते ।।
83. असङ्गात्वविवरणम् - ग्रन्थोऽयं मद्रासराजकीयहस्तलिखितपुस्तकालये लभ्यते ।। 
84. अहमर्थविवेकः - स्कन्दशङ्करसंवादरूपेऽस्मिन् ग्रन्थे अहमिदंशब्दानामर्थ अद्वैतसिद्धान्तानुसारमुपवर्णितः । सादिरनन्तोऽयं ग्रन्थः सरस्वतीमहालये लभ्यते ।। 
85. आ आत्मचिन्तनम् - ग्रन्थोऽयं नासिक-तिरुपति पुस्तकालयर्योलभ्यः ।  
86. आत्मतत्वम् - दशभिः पद्यैः पूर्णोऽयं ग्रन्थस्सरस्वतीमहालयवर्णनात्मकग्रन्थसूच्याः त्रयोदशे भागे (7762 DC T S M L Vol XIII) मुद्रितः । 
87. आत्मतत्वविचारः - ग्रन्थोऽयं बरोडासूच्यां दृश्यते । 
88. आत्मविवेकः - ग्रन्थोऽयं लन्दननगरपुस्तकालये लभ्यते । 
89. आत्मतत्वविवेकः - अमुद्रितोऽयं ग्रन्थ मद्रासराजकीयहस्तलिखितपुस्तकालये ( R. 47d. MGOML) लभ्यते । 
90. आत्मबोधः - शाङ्करादात्मबोधाद्भिन्नोऽयं प्रकरणग्रन्थः बन्धमोक्षविद्याऽविद्या जाग्रत्स्वप्नसुषुप्त्यादीन्, पञ्च कोशान् जीवक्षेत्रज्ञ- साक्षिकूटस्थ- अन्तयभिप्रत्यगात्मपरमात्म-मायाशब्दानां अर्थं च वर्णयति । ग्रन्थोऽयं अडयारपुस्तकालये लभ्यते । 
91. आमबोघप्रकरणदीपिका - अमुद्रितोऽयं ग्रन्थस्मरस्वतीमहालये विश्वभारती शान्तिनिकेतनसूच्यां च दृश्यते । विश्वेश्वरपण्डितकृतायाः भिद्यते वा इति संशयः ।। 
92. आत्मबोधविवृतिः - पञ्चाबसूच्यां दृश्यतेऽयं ग्रन्थः ।। 
93. आत्मषट्कदीपिका । अमुद्रितोऽयं कल्कत्तासंस्कृतकलाशालापुस्तकसूच्यां दृश्यते । 
94. आत्मानात्मविवेकः। अज्ञातकर्तृनामायं ग्रन्थस्सर्वोपनिषत्सारसङ्ग्रहापराभिधः शङ्कराचार्यकृतात् आत्मानात्मविवेकात् शैल्या पद्यसंख्यया च भिन्न अडयारपुस्तकालये सरस्वतीमहालये च लभ्यते । पूर्वमस्माभिः विश्वेश्वर पण्डित-स्वयम्प्रकाशयति - सदाशिवब्रह्मेन्द्र-शङ्कराचार्यकृता आत्मानात्मविवेकग्रन्था निर्दिष्टाः, तेषु शाङ्कराद्धिन्न इति तु निश्चयः । 
95. आत्मानात्मविवेकव्याख्या - वेदान्तचूर्णिका । ग्रन्थोऽयं तिरुवनन्तपुरसूच्यां दृश्यते । 
96. आत्मानात्मपञ्चकोशविवेकः - सादिरनन्तश्चायं ग्रन्थस्सरस्वतीमहालयसूच्यां दृश्यते । 
97. आनन्दः - पञ्चीकरणविवरणम् पञ्चीकरणविवरणनामायं ग्रन्थः तिरुपति वेङ्कटेश्वरपुस्तकालये लभ्यते। 
98. आनन्दप्रकरणव्याख्या - ग्रन्थोऽयं सरस्वतीमहालये लभ्यते । 
99. आनन्दब्रह्मलक्षणम् - ग्रन्थोऽयं सरस्वतीमहालये लभ्यते । ग्रन्थेऽस्मिन् ग्रन्थकर्तुर्नाम न निर्दिष्टम् । परन्तु ग्रन्थकर्तायं महाराष्ट्रदेशीयः परमानन्दस्वरूपशिष्य इति परं ज्ञायते । 
100. आनन्दानुभवः - ग्रन्थोऽयं (R. 47 C. MGOML) लभ्यते ।
101. आमोदः - सूत्रवृत्तिरूपोऽयं ग्रन्थः मैसूरपुस्तकालये लभ्यते । 
102. आम्नायव्याख्यानम् - ग्रन्थोऽयं बरोडासूच्यां दृश्यते । 
103. उन्मत्तपलापः - असुद्रितोऽयं ग्रन्थः लन्दननगरे लभ्यते (IOL) ।
104. उपदेशशङ्कनिरासः - ग्रन्थोऽयं मैसूर सूच्यां दृश्यते । 
105. उपदेशसाहस्री व्याख्या - ग्रन्थोऽयं मद्रासपुस्तकालये लभ्यते । 
106. उपाधिमण्डनम् - ग्रन्थोऽयं मद्रास पुस्तकालये लभ्यते ।
107. ऊर्मिषट्कम् - ग्रन्थोऽयं सरस्वतीमहालयवर्णनात्मकग्रन्थसूच्यास्त्रयोदशेभागे (7766 DC TSML Vol XIII) मुद्रितः।
108. ऋजुविवरणव्याख्या - अपूर्णोऽयं ग्रन्थस्तिरुवनन्तपुरग्रन्थसूच्यां (T.C.D.) दृश्यते ।। 
109. ऋभुगीता - ऋभुनिदाधसंवादरूपेऽस्मिन् ग्रन्थे पौराणिकसरण्या अद्वैतसिद्धान्ताः प्रदर्श्यन्ते। मुद्रितश्चायं ग्रन्थ आष्टेकरकम्पेनि पूनानगरे। 
110. कठोपनिषद्भाष्यटीका - अमुद्रितोऽयं ग्रन्थः कलकत्तासंस्कृतकलाशाला पुस्तकालये लभ्यते । 
111. कुर्तकखण्डनम् - द्वैतसिद्धान्तखण्डनपरोऽयं ग्रन्थ अपूर्णः मद्रासराजकीयपुस्तकालये (D. 4573 MGOML) लभ्यते । 
112. कुतर्कनिरासः - अपूर्णोऽयं ग्रन्थः सरस्वतीमहालयपुस्तकालये (7690 TSML) लभ्यते । 
113. केनोपनिषद्भाष्यटीका -
	केनोपनिषद्भाष्यटिप्पणी । ग्रन्थाविमौ सरस्वतीमहालयपुस्तकालये वाराणसीस्थसरस्सतीभवनपुस्तकालये च लभ्यते । 
114. कैवल्यरत्नम् - ग्रन्थोऽयं मद्रासपुस्तकालये (R. 193 G. MGOML) लभ्यते ।
115. कैवल्यसौधनिःश्रेणिः - अमुद्रितोऽयं ग्रन्थस्तिरुविडमरुदूर देवालयपुस्तकालये लभ्यते । 
116. कैवल्यस्थानम् - ग्रन्थोऽयं मद्रासराजकीयपुस्तकालये लभ्यते। 
117. खण्डनखण्डखाद्यटीका - अद्वैतबोधामृतम् । प्रथमपरिछेदान्तोऽयं ग्रन्थः मद्रासराजकीयपुस्तकालये तिरुवनन्तपुरपुस्तकालये न लभ्यते । 
118. खण्डनखण्डनव्याख्या - अमुद्रितोऽयं ग्रन्थ मद्रासराजकीयपुस्तकालये लभ्यते । अडयारपुस्तकालये च लभ्यते । 
119. गणेशदर्शनम् - मद्रासपुस्तकालये लभ्यते । 
120. गुरुप्रसादः - अज्ञातकर्तृकोऽयं ग्रन्थ अमुद्रिते सरस्वतीमहालयस्थे (7697 TSML) बोघप्रक्रियाग्रन्थे निर्दिष्टः। अस्य प्राप्तिस्थानादि न ज्ञायते । 
121. गुरुशिष्यकथनम् - हरिहरसंवादरूपेऽस्मिन् ग्रन्थे आत्मनः साक्षात्कार एव मुक्तिरिति प्रतिपाद्यते । ग्रन्थोऽयं सरस्वतीमहालये लभ्यते । 
122. गौडपादीयविवेकः - माण्डूक्यकारिकाव्याख्यात्मकोऽयं ग्रन्थः मद्रासराजकीयहस्तलिखितपुस्तकालये (R. 3882 d. MGOML) लभ्यते । 
123. चतुर्विधमहावाक्यानुभवः - ग्रन्थोऽयमडयारपुस्तकालये लभ्यते ।
124. चतुस्सूत्रीव्याख्या - ग्रन्थोऽयं कलकत्ता संकृतकलाशालापुस्तकालये प्राप्यते । 
125. चिन्द्रका - ब्रह्मसूत्राणां वृत्तिरूपोऽयं अज्ञातकर्तृकः पूर्णप्रायश्च ग्रन्थः अडयारपुस्तकालयस्थः। एतत्समाननामा ग्रन्थः लन्दननगरपुस्तकालयस्थः भवदेवमिश्रकृत इति ज्ञायते । 
126. चिदद्वैतकल्पवल्ली - ग्रन्थोऽयं मैसूरसूच्यां दृश्यते ।
127. चिदानन्दद्वादशकम् - ग्रन्थोऽयं मद्रासपुस्तकालये लभ्यते । 
128. चेतनाचेतनप्रकरणम् - ग्रन्थोऽयं आत्मानात्मविवेकवत् आत्मनः प्रपञ्चातीतत्वं, प्रपञ्चस्य चेतनाचेतनात्मकत्वञ्च उपपादयन् प्रकरणग्रन्थतामर्हति। अडयारपुस्तकालये लभ्यते च ।
129. छान्दोग्यप्रकाशिका - ग्रन्थोऽयं हलषसूच्यां दृश्यते । 
130. छान्दोग्योपनिषल्लघुव्याख्या - ग्रन्थोऽयं मद्रासराजकीयपुस्तकालये लभ्यते । 
131. छान्दोग्योपनिषद्भाष्यटिप्पणम् - ग्रन्थस्यास्य कर्ता न ज्ञायते । परन्तु नरेन्द्रपुरीयतीश इति प्रथमपद्ये दृश्यते । अनुभूतिस्वरूपाचार्यशिष्यस्सारस्वतप्रक्रियाव्याख्याता नरेन्द्रनगरीति प्रसिद्धः कश्चनास्माभिरनुभूतिस्वरूपाचार्यप्रकरणे निर्दिष्टः। किमयं स एवोतान्य इति संशयः। अमुद्रितोऽयं ग्रन्थः मद्रासराजकीयपुस्तकालये (R. 3690 M MGOML) लभ्यते । 
132. जगदुत्पत्तिप्रकरणम् - ग्रन्थोऽयं मद्रासपुस्तकालये लभ्यते ।
133. जगन्मिथ्यात्वोपदेशः - पद्यबद्धोऽयमपूर्णग्रन्थ अडयारपुस्तकालये लभ्यते । 
134. जिज्ञासाधिकरणविचारः - (R. 1803 E. MGOML) लभ्यते । 
135. जीवन्मुक्तसञ्चारः - (D. 4578 MGOML) लभ्यते । 
136. जीवन्मुक्तिविचारः - (R. 2851 MGOML) लभ्यते । 
137. जीवब्रह्माभेदः - प्रकरणग्रन्थोऽयं सरस्वतीमहालये लभ्यते । 
138. जीवब्रह्मैक्यबोधिनी - ग्रन्थोऽयमुज्जैन् सूच्यां दृश्यते । 
139. जीवेश्वरप्रकरणम् - जीवेश्वरयोर्लक्षण निरूपयन्नयं ग्रन्थः अन्तःकरणं अन्तःकरणाधिष्ठानचैतन्यं तत्प्रतिविम्बचैतन्यमित्येतत्त्रयं सम्भूय जीव इति प्रतिपादयति । अमुद्रितोऽयं ग्रन्थ अडयारपुस्तकालये लभ्यते । 
140. जीवेश्वरसन्धानक्रमः - ग्रन्थोऽयं मद्रासपुस्तकालये लभ्यते । 
141. ज्ञानतिलकः - ग्रन्थोऽयं बरोडासूच्यां दृश्यते । 
142. ज्ञानपञ्चाशिका - मध्यप्रान्तबरार्ग्रन्थसूच्यां दश्यते । 
143. ज्ञानप्रदीपिका - प्रपञ्चस्य ब्रह्मात्मकत्वं वर्णयन्नयं प्रकरणग्रन्थः अडयारपुस्तकालये लभ्यते ।।
144. ज्ञानप्रबोधमञ्जरी - जगतोमिथ्यात्वमुपवर्णयन्नयं ग्रन्थः मद्रासराजकीयहस्तलिखितपुस्तकालये (R 1419 GMGOML) अडयारपुस्तकालये बरोडा पुस्तकालये च लभ्यते ।। 
145. ज्ञानमार्गबोधिनी -
146. ज्ञानमुद्रानाटकम् - एतौ ग्रन्थौ अडयारपुस्तकालये लभ्यते ।।
147. ज्ञानवासिष्ठव्याख्या - ग्रन्थोऽयं (R 4422 MGOML) लभ्यते । 
148. ज्ञानसंन्यासः - ग्रन्थोऽयं  बरोडा सूच्यां दृश्यते । 
149. ज्ञानाङ्कुशः - सव्याख्योऽयं ग्रन्थः । मनोनिग्रहोपायप्रदर्शनपरोऽयं ग्रन्थः मनोमतङ्गजे ज्ञानाङ्कुशवशं नीते सन्मार्गप्रापिते च साधनचतुष्टयसम्पन्नाधिकारिणः सद्गुरूपसत्तितः ज्ञानोत्पत्तौ जीवन्मुक्तिरिति प्रतिपादयति । ग्रन्थोऽयं शङ्कराचार्यकृतत्वेन मुद्रितः । नीलकण्ठतीर्थकृतोऽन्यो ग्रन्थः सव्याख्यः अडयारपुस्तकालये लभ्यते । अज्ञातकर्तृकोऽयं प्रकृतः ग्रन्थः सरस्वतीमहालये नासिकसूच्यां च दृश्यते ।। 
150. ज्ञानानुष्ठानप्रकरणम् - अपूर्णोऽयं ग्रन्थः मद्रासराजकीयहस्तलिखितपुस्तकालये लभ्यते ।। 
151. ज्ञानोपदेशसारः - ग्रन्थोनयं पञ्चाबसूच्यां दृश्यते । 
152. णत्वबाधानिरसनम् - वेङ्कटेश्वरपुस्तकालये लभ्यते ।। 
153. तत्वचन्द्रिका - पञ्चीकरणव्याख्या - पञ्चीकरणविवरणव्याख्यात्मकोऽयं ग्रन्थः मद्रासराजकीयपुस्तकालये (R 3637 B MGOML) विद्यते। शाङ्करपञ्जीकरणस्य आनन्दगिरिणा विवरणं कृतम् । तस्य व्याख्यात्मकोऽयं ग्रन्थः। एतत्समाननामा ग्रन्थः रामतीर्थकृतः पूर्वमुपवर्णितः । अनन्तशयनपुस्तकालये च लभ्यते ।। 
154. तत्वदीपिकाव्याख्या "श्रुतिसारसमुद्धरणव्याख्यायाः व्याख्या"। ग्रन्थोऽयं मद्रासपुस्तकालये लभ्यते ।।
155. तत्वप्रबोधिनी - ग्रन्थोऽयं मद्रासपुस्तकालये लभ्यते ।। 
156. तत्वबिन्दव्याख्या - कुमरिलभट्टीयस्फोटवादखण्डनपरस्य वाचस्पतिमिश्रकृतस्य तत्वबिन्दोः व्याख्येयमद्वैतपराऽमुद्रिता च मद्रासराजकीयहस्तलिखित पुस्तकालये (R 3800 MGOML) लभ्यते ।। 
157. तत्त्वम्पदविवरणम् (R 193 H. MGOML) 
158. तत्वम्पदविवेकः। इमौ ग्रन्थौ मद्रासपुस्तकालये लभ्येते ।। 
159. तत्वमसिदशकम् - अद्वैतसिद्धान्तप्रतिपदिशा तत्वमसिवाक्यार्थं पद्यैः प्रतिपादयन् प्रतिचतुर्थपाद "तत्वमसि तत्वमसि तत्वमसि तत्वम् " इति महावाक्यसुपदिशति। अमुद्रितोऽयं ग्रन्थ अडयारपुस्तकालये लभ्यते ।  
160. तत्वमसिशतकम् -ग्रन्थोऽयमडयारपुस्तकालये लभ्यते। रामचन्द्रेन्द्रकृतान् पूर्वमुपवर्णिताद्भिद्यते च । 
161. तत्वमसिप्रकरणविवरणम् - तिरुवनन्तपुरसूच्यां दृश्यते ।
162. तत्वमस्यादिवाक्यार्थविरोधनिरासः - ग्रन्थोऽयं मद्रासराजकीयपुस्तकालये (R 2921 B. MGOML) लभ्यते । 
163. तत्वविवेकोपन्यासः - ग्रन्थोऽयं मैसूर सूच्यां लभ्यते । 
164. तत्वविवेकव्याख्या - बरोडापुस्तकालये लभ्यतेऽयं ग्रन्थः।
165. तत्वविवेचनम् - ग्रन्थोऽयं विद्यारण्यपुरसूच्यां दृश्यते । 
166. तत्वविषयकम् - ग्रन्थोऽयं (D 4593 MGOML) लभ्यते ।
167. तत्वसारः - ग्रन्थोऽयं मद्रासपुस्तकालये (R. 4067 C. MGOML) लभ्यते । 
168. तत्वसारायणम् - ग्रन्थोऽयं वासिष्ठरामायणान्तर्गत इति वदन्ति । परन्तु न तत्रोपलभ्यते । ग्रन्थेऽस्मिन् ब्रह्मा दक्षिणामूर्ति प्राप्य ब्रह्मसूत्राथोंपदेशाय प्रार्थयते दक्षिणामूर्तिः व्यासायोपदिष्टनि सूत्राणि व्याख्यातुमारभते । परिच्छेदद्वयपूर्णोऽयं ग्रन्थः । अस्य व्याख्या अधिकरणकञ्जुकाख्या अप्पय्यदीक्षितकृतेति प्रसिद्धिः । मुद्रितश्चायं ग्रन्थः कारवेटनगरे तेलगु लिप्याम् ।
169. तत्वसङ्ग्रहः - सादिरनन्तोऽयं ग्रन्थः अडयारपुस्तकालये लभ्यते । 
170. तत्वार्थप्रकरण्म् - ग्रन्थोऽयं (7702 TSML) लभ्यते ।
171. तत्त्वोपतिलयक्रमः - ग्रन्थोऽयं तिरुपतिसूच्यां दृश्यते ।
172.तत्वोपदेशः - मैसूरसूच्यां मद्रासराजकीयपुस्ककालये (R 3046 MGOML) अमन्तशयनसूच्याञ्च दृश्यते। 
173. तैत्तरीयोपनिषद्व्याख्या लघुदीपिका - ग्रन्थोऽयं सरस्वतीमहालये मद्रासराजकीयपुस्तकालये लन्दननगरपुस्तकालये च लभ्यते ।
174. तैत्तरीयसारव्याख्या - तिरुपति सूच्यां दृश्यते । 
175. दक्षिणामूर्तिविलासः - ग्रथोऽयं अडयारुपुस्तकालये लभ्यते । 
176. दशश्लोकी - दशभिःश्लोकैः अद्वैतसिद्धान् प्रतिपादयन्नयं ग्रन्थः शाङ्कराद्दशश्लोकीग्रन्थाद्भिद्यते । अमुद्रितोऽयं ग्रन्थः मद्रासराजकीयहस्तलिखित पुस्तकालये (R 5050 MGOML) लभ्यते । 
177. दीपिका शाब्दनिर्णयव्याख्यात्मकोऽयं ग्रन्थः मद्रासराजकीयपुस्तकालये (R. 2986 MGOML) लभ्यते । 
178. द्वादशमहावाक्यम् । 
179. दुरितमुखभञ्जनम् । 
180. देहचतुष्टयव्यवस्थालक्षणम् - एते ग्रन्थाः बरोडापुस्तकापये लभ्यन्ते । 
181. द्वैतखण्डनम् - मायासत्वादिशब्दार्थानां विवेचकोऽयं द्वैतखण्डनपरः ग्रन्थः अमुद्रित अडयारपुस्तकालये लभ्यते । 
182. द्वैतमिथ्यात्वनियः - (R. 1803 E. MGOML) 
183. द्वैतनिरासः - एतौ ग्रन्थौ मद्रासपुस्तकालये लभ्येते । 
184. ध्यानसारः सव्याख्यः - अनन्तशयने लभ्यते । 
185. निगमागमत्रिशतीनामस्तोत्रम् - सङ्कलानात्मकोऽयं ग्रन्थः मद्रासराजकीयपुस्तकालये लभ्यते । 
186. निगमार्थदीपिका - ग्रन्थोऽयमडयारपुस्तकालये लभ्यते । 
187. निर्गुणाराधानक्रमः - ग्रन्थोऽयं मैसूरसूच्यां दृश्यते ।
188. निजानन्दानुभूतिप्रकरणम् - अडयारपुस्तकालये लभ्यते । 
189. नित्योपासना - ग्रन्थोऽयं अनन्तशयनपुस्तकालये लभ्यते । 
190. निर्वेदाष्टकविवरणम् - ग्रन्थोऽयं वेङ्कटेश्वरालयपुस्तकसूच्यां अमुद्रित उपलभ्यते । 
191. नृसिम्होत्तर तापिनी कारिका - तिरुवनन्तपुरपुस्तकालये लभ्यते ।
192. नृसिम्हपूर्वतापिनीभाष्यम् - मैसूरसूच्यां दृश्यते । 
193. नृसिम्हस्तुतिः सव्याख्या - अमुद्रितोऽयं ग्रन्थः मद्रासराजकीयपुस्तकालये लभ्यते । व्याख्याकार अभिनवस्वयम्प्रकाशानन्दः। 
194. नैष्कर्म्यसिद्धिसम्बन्धोक्तिः - ग्रन्थोऽयममुद्रितस्तिरुवनन्तपुरपुस्तकालये लभ्यते । वेदाध्यक्षपूज्यपादशिष्येण वेदानन्देन कृतः स्यादित्यूह्यते । 
195. न्यायदीपावलीव्याख्या - तिरुवनन्तपुरसूच्यां दृश्यते । नरेन्द्रपुरीकृत इति मद्रासराजकीयपुस्तकालयात् (D. 15306) ज्ञायते । 
196. न्यायप्रमाणमञ्जरी - सटीका। ग्रन्थोऽयं (IOL) लभ्यते। 
197. न्यायमकरन्दसङ्ग्रहः - ग्रन्थोऽयं बरोडापुस्तकालये लभ्यते । 
198. पदशक्तिबोधः - सिम्हाद्रिनाथोपसेविना अज्ञातनामधेयेन ग्रन्थकारेण सर्वेषां साधुशब्दानां अद्वैतब्रह्मण्येव शक्तिरिति प्रतिपाद्यते । ग्रन्थोऽयं मद्रासराजकीयपुस्तकालये (R. 3284 MGOML) लभ्यते । 
199. परब्रह्मोपनिषदः। (D 593 MGOML) लभ्यते । 
200. परमतभञ्जनम् - अद्वैतेतरमतखण्डनपरोऽयं ग्रन्थ अपूर्णस्सरस्वतीमहालये (7643 TSML) लभ्यते ।। 
201. परमसिद्धान्तसारः - ग्रन्थोनयं स्वयम्प्रकाशशिष्यकृतः। 
202. परमाहसोपनिषदः - (D 601 MGOML) लभ्यते । 
203. परमात्मनिरूपणम् - अमुद्रितोऽयं ग्रन्थ अडयारपुस्तकालये लभ्यते ।। 
204. परमानन्ददीपिका - ग्रन्थोऽयं (4920 d. BRD) लभ्यते ।।
205. परमार्थसारसङ्ग्रहः - अमुद्रितोऽयं ग्रन्थः मद्रासराजकीयपुस्तकालये लभ्यते ।। 
206. परिव्राजकोपनिषदः - (R. 902 MGOML) लभ्यते ।।
207. पञ्चकोशविचारः- (R. 2111 B. MGOML) लभ्यते ।
208. पञ्चदशीसूक्तम् - ग्रन्थोऽयं मध्यप्रान्तीयबरार्ग्रन्थसूच्यां दृश्यते ।।
209. पञ्चप्रकरणम् - (R. 2599 H. MGOML) लभ्यते ।
210. पञ्चप्रकरणी - (R. 2460 MGOML) लभ्यते।
211. पञ्चपादिकाव्याख्या तत्वदीपिका - अमुद्रितोऽयं ग्रन्थः नरकेसरिशिष्यकृत इति ज्ञायते। ग्रन्थोऽयं पञ्चाबसूच्यां 668 दृश्यते ।।
212. पञ्चब्रह्मतत्वम् - (7732 TSML) लभ्यते ।। 
213. पञ्चभ्रमनिरूपणम् - वैराजवैश्वानर हिरण्यगर्भाख्यानां अवयवविश्लेषनिर्वचनपूर्वकं पञ्चानां भ्रमाणां स्वरूपं तेषां निवर्तनोपायञ्च प्रतिपादयन्नमुद्रितोऽयं ग्रन्थः मद्रासराजकीयहस्तलिखितपुस्तकालये (R. 4629 MGOML) लभ्यते ।। 
214. पञ्चभूतविकारः - (D. 4628 MGOML) लभ्यते । 
215. पञ्चरत्नम् पञ्चरत्नकला - (2776 - 78) मध्यप्रान्तीयबरार्ग्रन्थ सूच्यां दृश्यते । 
216. पञ्चरत्नदीधितिः-शाङ्करपञ्चरत्नव्याख्यात्मकोऽयं ग्रन्थः तिरुवनन्तपुरसूच्यां (274 TCD), मद्रासराजकीयपुस्तकसूच्याञ्च (D.4633 MGOML) लभ्यते। 
217. पञ्चरत्नव्याख्या - ग्रन्थोऽयमडयारपुस्तकालये (99 F. 44) लभ्यते । 
218. पञ्चरत्नमालिकाव्याख्या कल्पवल्ली - (R. 4632 MGOML) लभ्यते । 
219. पञ्चश्लोकी - ग्रन्थोऽयं बरोडासूच्यां (6669 g. BRD) लभ्यते ।
220. पञ्चश्लोकीव्याख्या -(6183 F. BRD) लभ्यते ।
221. पञ्चीकरणम् - ग्रन्थोऽयं सरस्वती महालयसूच्यां(7720 TSML) अस्ति।
222. पञ्चीकरणम् - पूर्वोक्ताद्भिन्नोऽयं वर्णनात्मकसरस्वतीमहालयसूच्यां (7737 DC. TSML. Vol. XIII) मुद्रितः।
223. पञ्चीकरणम् - पूर्वोक्तद्वाभ्यां भिन्नः शाङ्करपञ्चीकरणाच्च भिन्नोऽयं ग्रन्थः(7720 TSML) लभ्यते । 
224. पञ्चीकरणम् - अज्ञातकर्तृनामायं ग्रन्थः मद्रासराजकीयपुस्तकालये (R 1391 B. MGOML) लभ्यते । अभिनवसदाशिवब्रह्मेन्द्रकृतमिति (D. 4636 MGOML) ग्रन्थात् (908 DC. AL) ग्रन्थाच्च ज्ञायते।  
225. पञ्चीकरणम् - (2550 V.B.S.) विश्वभारतीसूच्यां दृश्यते ।
226. पञ्चीकरणतत्वनिर्णयः - पञ्चीकरणतत्वानि पञ्चनवतिधा विभज्य प्रकशयन्नयं ग्रन्थः (265 TCD) दृश्यते। 
227. पञ्चीकरणव्याख्या -(D. 14255 MGOML) लभ्यते ।
228. पञ्चीकरणवार्तिक व्याख्या - बरोडापुस्तकालयमद्रासराजकीयपुस्तकालययो (D. 4642 MGOML) र्लभ्यते । 
229. पञ्चीकरणविधिः-(4895 BRD) पुस्तकालये लभ्यते । 
230. पञ्चीकरणविवरणम् - आत्मानुसन्धानम् मद्रासपुस्तकालये लभ्यः । 
231. पञ्चीकरणविवरणव्याख्या - तत्वपञ्चिका (9827 D. BRD) लभ्यते । 
232. पञ्चीकरणमहावाक्यार्थः-(7700 TSML) पुस्तकालये लभ्यते ।
233. पञ्चीकरणमहावाक्यार्थः -(3833 BRD) पुस्तकालये लभ्यते ।
234. पञ्चीकरणविवेचना - (TSML, TMPL) पुस्तकालययोर्लभ्यते । 
235. पञ्चीकरणसङ्ग्रहगः-पञ्चीकरणतत्वनिर्णयापर नामायं ग्रन्थः (247 B.TCD) लभ्यते । 
236. पञ्चीकरणोपनिषदः-(D. 592 MGOML) लभ्यते । 
237. पञ्चीकृतम् - ग्रन्थोऽयं (DC TSML Vol XIII) मुद्रितः 
238. पञ्चीकृतव्याख्या -(DC TSML Vol XIII) लभ्यते ।
239. प्रचण्डराहूदयव्याख्या - अमुद्रितोऽयं ग्रन्थस्सरस्वतीमहालये लभ्यते । घनश्यामकृतस्य प्रचण्डराहूदयस्य व्याख्यात्मकोऽयं ग्रन्थः। प्रबुद्धचन्द्रोदयवदद्वैतवेदान्तपरः।
240. प्रणवार्थप्रकाशिका - अमुद्रितोऽयं अनन्तशयने लभ्यते । 
241. प्रत्यक्तत्वप्रकाशिका -(10321 A. BRD) पुस्तकालये लभ्यते ।
242. प्रत्यक्पूजानुसन्धानम् - (4920 F. BRD) पुस्तकसूच्यां दृश्यते । 
243. प्रबोधचन्द्रोदयव्याख्या - अमुद्रितोऽयं मद्रासूपुस्तकालये विद्यते । 
244. प्रबोधदीपिका-अध्यारोपापवादमुखेनाद्वैतसिद्धान्तान् प्रतिपादयन्नयं प्रकरनग्रन्थ अनन्तशयनपुस्तकालये	(309 T.C.D) मद्रासपुस्तकालये (R. 1849 A. MGOML) च लभ्यते । 
245. प्रबोधमञ्जरी - (2689 BRD)
246. प्रमाणतत्वम् - (R. 2251 MGOML)
247. प्रमाणव्याख्या - (311 TCD)
248. प्रश्नोत्तरमाला - (1725 BRD)
249. पाखण्डदलनम् - (2912 CCPB)
250. बहुविधमतखण्डनम् - (7634 TSML)
251. बोधप्रक्रिया - अपूर्णोऽयं ग्रन्थः 7697 सरस्वतीमहालयसूच्यां दृश्यते। अनेन ग्रन्थकृता गुरुप्रसादाख्याः ग्रन्थः कृत इति ग्रन्थादस्मात् ज्ञायते । 
252. बोधरत्नाकरः - (3358 CCPB)
253. बोधसारः - सरस्वतीमहालये सरस्वतीभवने च लभ्यते । 
254. बोधामृतम् - मैसूरपुस्तकालये लभ्यते ।
255. ब्रह्मचिन्तनिकाविवरणम् - उज्जैनसूच्यां दृश्यते । 
256. ब्रह्मज्ञाननिर्णंयः - मद्रासपुस्तकालयात्प्राप्यः। 
257. ब्रह्मज्ञानविचारः - (D 4646 MGOML)
258. ब्रह्मनिर्गुणत्ववादः - अडयार पुस्तकालयाल्लभ्यते । 
259. ब्रह्मनिरामयाष्टकम् - मुद्रितश्चायं ग्रन्थः (DC. TSML Vol. XIII)
260. ब्रह्मभावनादीपिका - जीवब्रह्मणोरभेदस्सभाधिना बोध्य इति प्रतिपादयन्नयं ग्रन्थ अमुद्रित अडयार पुस्तकालये (34. O 39. AL) लभ्यते ।
261. ब्रह्ममीमांसाशास्त्रङ्ग्रहः - ब्रह्मसूत्रशाङ्करभाष्यार्थसङ्ग्रहकोऽयं ग्रन्थ अमुद्रितः पूर्णश्च अडयार पुस्तकालये (8 F 53 AL) लभ्यते । ग्रन्थेऽस्मिन् शङ्कराचार्यकालः (845 सं 788 A.D) इति प्रतिपाद्यते T	वार्तिककारसुरेश्वरविश्वरूपयोरैक्यं च प्रतिपाद्यते । 
262. ब्रह्ममीमांसासूत्रव्याख्या - (R. 2205 MGOML)
263. ब्रह्मलक्षणा - बाम्बेविश्वविद्यालयसूच्यां दृश्यते । 
264. ब्रह्मविचाराधिकारनिरूपणम् - (28. R. 50 AL) 
265. ब्रह्मवित्कर्मविचारः-(12204 BRD) सूच्यां दृश्यते । 
266. ब्रह्मविद्या -(3371 CCPB) सूच्यां दृश्यते ।
267. ब्रह्मविद्यातरङ्गिणी -(34. N. 19. AL)	पुस्त कालये लभ्यते।
268. ब्रह्मविद्यारहस्यम् -(8 G. 30 AL) पुस्तकालये लभ्यते।।
269. ब्रह्मविद्यासारसंग्रहः - (8 G. 28-30 AL) पुस्ककालये सङ्कलनात्मकोऽयं ग्रन्थ उपलभ्यते।।
270. ब्रह्मविन्महिमा - (D. 4648 MGOML)
271. ब्रह्मसूत्रक्रमः - प्राप्तिस्थानादि न ज्ञायते । 
272. ब्रह्मसूत्रचन्द्रिका - मद्रासपुस्तकालये लभ्यते ।
273. ब्रह्मसूत्रवृत्तिः - ग्रन्थोऽयं अद्वैतमञ्जरीग्रन्थमालायां कुम्भघोणे मुद्रितः । तत्र शङ्करभगवत्पादशिष्यकृत इति निर्दिश्यते । सुरेश्वराचार्य इति केचित्साम्प्रदायिकाः । 
274. ब्रह्मसूत्रवृत्तिः - (40 C. 53 AL, TMPL)
275. ब्रह्मसूत्रवृत्तिः - (20 C. 75 AL, TMPL) गद्यमय अपूर्णश्च ।
276. ब्रह्मसूत्रवृत्तिः- (8-1-22 AL, TSML) अस्य कर्ता शङ्कराचार्य इति इति ग्रन्थाज्ज्ञायते । 
277. ब्रह्मसूत्रवृत्तिः - (20 M 42 AL, 29 B. 7 AL, 10 C.5 AL, R. 1394 A. MGOML, BRd IOL) पुस्तकालयेषु लभ्यते । 
278. ब्रह्मसूत्रघुवृत्तिः-(R. 1385 MGOML)
279. ब्रह्मसूत्रविवरणम् -(R. 1807 MGOML)
280. ब्रह्मसूत्रविवृतिः - अनन्तशनय अडयार पुस्तकालययोर्लभ्यते ।
281. ब्रह्मासूत्रव्याख्या-(7105 TSML, IOL, MGOML)
282. ब्रह्मसूत्रसङ्ग्रहतात्पर्यनिरूपणम् -(R 90 MGOML)
283. ब्रह्मसूत्रभाष्यव्याख्या-अपूर्णोऽयं ग्रन्थः (7104 TSML) लभ्यते।
284. ब्रह्मसूत्रभाष्यार्थसङ्ग्रहः - मैसूर पुस्तकालये लभ्यते । 
285. ब्रह्मसूत्रकारिका - तिरुपति सूच्यां 20-89 C. दृश्यते । 
286. ब्रह्मसूत्राधिकरणसूत्रानुक्रमणिका -(R. 3305 A. MGOML)
287. ब्रह्मानन्दः - वेङ्कटेश्वरसूच्यां 1998 B. दृश्यते । 
288. भगवद्गीता भाष्यटिप्पणी -(6840 BRd. MGOML)
289. भगवद्गीता भाष्यटिप्पणी - (TSML, AL, TCL)
290. भगवद्गीतामाला-(3503 CCPB)
291. भगवद्गीतासङ्गतिमालाव्याख्या - मद्रासपुस्तकालये लभ्यते । 
292. भगवद्गीताव्याख्या - रायल आसियाटिक सोसाईट्यां बम्बई नगरे लभ्यते । 
293. भगवद्गीतारहस्यार्थदर्पणम्- प्रत्यध्यायमद्वैतसिद्धान्तप्रदर्शनपरोऽयं ग्रन्थः समग्रगीताया अद्वैते तात्पर्यं वर्णयन् अमुद्रितः पूर्णश्च मद्रासपुस्तकालये (D. 4547 MGOML) लभ्यते ।
294. भगवद्गीतार्थसङ्ग्रहः - मैसूरपुस्तकालये लभ्यते । 
295. भारतीयमननम् - मैसूर पुस्तकालये लभ्यते । 
296. भावार्थदीपिका - अडयारपुस्तकालयाल्लभ्यते । 
297. भेदखण्डनम् -(D. 4595, R. 2851 B. MGOML)
298. भेदधिक्काररिप्पणी - (7514 TSML, Mysore) पुस्तकालययोरमुद्रित उपलभ्यते । 
299. भेदधिक्कारसत्कियोज्वला -(D. 4702 MGOML)
300. भेदधिक्कारोपन्यासः - अमुद्रितोऽयं ग्रन्थस्सरस्वतीमहालये लभ्यते (7515 TSML) अस्य कर्ता नृसिम्हाश्रमिणः भेदधिक्कारकर्तुश्शिष्य इति ज्ञायते। अनेन ग्रन्थ कर्त्रा सिद्धानन्दतीर्थः अनन्तकृष्णश्च स्मर्येते । 
301. भेदध्वान्तचण्डप्तार्ताण्डः - चण्डमारुतमित्यपि नामान्तरं श्रूयते । परन्तु चण्डमार्ताण्ड इति नामैव सार्थकं भवति । अमुद्रितेऽयं ग्रन्थ (R. 4209 B. MGOML) लभ्यते । 
302. भेदनिराकरणम् - अद्वैतसिद्धिग्रन्थस्यावान्तरभागात्मकोऽयमिति ज्ञायते । विषयप्रधान्यं मनसि कृत्वा पृथक्कृतः । अडयार पुस्तकालये (21 F. 39 AL) लभ्यते। 
303. मध्वमतध्वान्तदिवाकरः - अप्पय्यदीक्षितीयमध्वतन्त्रमुखमर्दनवत् अयमपि ग्रन्थ आनन्दतीर्थीय ब्रह्मसूत्रभाष्यं खण्डयति । प्रथमाध्याय प्रथमपादर्विशतिसूत्राणामेवोपलभ्यते। ज्ञानेन्द्रगुरुकृतः मध्वन्यक्कारः, परमेष्ठिगुरूकृतं वेदान्त भूषणञ्चात्र प्रमाणीक्रियेते। 
304. मध्वमतविध्वंसनम् - 
305. मध्वसिद्धान्त भञ्जनम् - इमौ ग्रन्थौ मद्रासपुस्तकालये लभ्येते । 
306. मनीषापञ्चकलघुविवरणम् - (R. 3132 A. MGOML)
307. मनीषापञ्चकव्याख्या - (24 C. 18. AL, D. 707 MGOML, TMPL) सरस्वतीमहालये च लभ्येयं व्याख्या । 
308. मनोलय प्रकरणम् - (7763 TSML DC. Vol. XIII) मुद्रितः।
309. महावक्यम् - 37148 तिरुपतिसूच्यां (4885 a BRD) लभ्यते । 
310. महावाक्यदर्पणम् - शाङ्करान्महावाक्यदर्पणादेतस्मिन् ग्रन्थे क्यचिदेव स्थले भेदो दृश्यते । (33. E. 13 AL) लभ्यते । 
311. महावाक्यदीक्षा - सरस्वतीमहालये लभ्यते ।
312. महावाक्यदीपिका - (9.F.84 AL. 411 A) तिरुपतिसूच्यां च लभ्यते । 
313. महावाक्यनिर्णयः - (1668 BRD)
314. महावाक्यप्रकरण् -(22. P. 4. AL)
315. महावाक्यविवरणम् - (7740 DC. TSML Vol. XIII) मुद्रितः । 
316. महावाक्यविवरणम् - (7739 TSML, 40 C. 51 AL)
317. महावाक्यविवरणव्याख्या - (7567 TSML)
318. महावाक्यव्याख्या - (4903 BRD)
319. महावाक्यानि सव्याख्यानि-(IOL) पुस्तकसूच्यां दृश्यते। 
320. महावाक्यार्थदीपिका - वाक्यार्थदीपिकापरनामायं ग्रन्थः बाम्बे विश्वविद्यलये (2086 DC. BUL) लभ्यते । 
321. महावाक्यार्थबोधप्रकारः - (7740 DC TSML Vol XIII) मुद्रितः । 
322. महावाक्यार्थविवरणम् - (D. 4714 MGOML)
323. महावाक्यविवेकः (7721 TSML)
324. महावाक्यविवेकबोधकम् - (615 TMPL) लभ्यते ।
325. महावाक्यार्थः - (2551 VBS 7241 D. BRD)
326. महावाक्यार्थदीपिका - (2229 V.B.G. 9883 BRD)
327. महावाक्योपदेशः- (R. 2599 G. MGOML)
328. महावाक्यार्थदीपकम् - (616 TMPL)
329. महावाक्यार्थदीपिका - (11255 (i) BRD)
330. महावाक्यार्थबोधप्रकारः - (7727 TSML)
331. महावाक्यार्थतत्वबोधिनी - (N. 38 -19 MGOML)
332. माण्डूक्योपनिषत्प्रकाशः सरस्वती भवनसूच्यां दृश्यते। 
333. माण्डूक्योपनिषद्व्याख्या - पदार्थदीपिका (D 17021 MGOML)
334. माण्डूक्योपनिषत्सारभूतव्याख्या - ग्रन्थस्यास्य प्राप्तिस्थानं न ज्ञायते। 
335. मिथ्यात्वविचारः- (1744 B.) तिरुपति सूच्यां दृश्यते । 
336. मुक्तचिन्तामणिः- (4169 C.C.P.B.)
337. मुक्तिविचारः-(BUL) लभ्यते । 
338. मुक्तिसोपानपद्धतिः - अन्तश्शत्रुणां कामक्रोधादीनां निग्रहादेव अद्वितीयाखण्डैकरसब्रह्मात्मकत्वसिद्धिरिति प्रतिपादयन्नयं ग्रन्थः कामक्रोधादि शब्दानां निर्वचनञ्च प्रतिपादयति । ग्रन्थोऽयं (7610 TSML) लभ्यते।
339. मोक्षोदयः - मद्रासपुस्तकालये लभ्यते । 
340. मङ्गलाभरपणम् - ईशाद्युपनिषदां व्याख्यात्मकोऽयं ग्रन्थः मैसूर पुस्तकालये लभ्यते । 
341. योगवासिष्ठसारः । ग्रन्थोऽयमज्ञातकर्तृनामा नासिकसूच्यां IV 19 दृश्यते । अस्य व्याख्या महीघरकृतापि वर्तते । परन्तु मूलग्रन्थकारः महीघर इति न निर्दिष्टम् । परन्तु लन्दननगरस्थपुस्तकालयसूच्यां बरोडासूच्याञ्च मूलकृदपि महीघर इत्येव निर्दिष्टम् । अस्माभिरद्वैताचार्यप्रस्तावे योगवासिष्ठप्रकरणे चोपपादितम् ।। 
342. योगार्णवः ज्ञानयोगार्णवः - (R. 3748 MGOML)
343. रामानुजशृङ्गभङ्गः - ग्रन्थोऽयं शङ्करमठपुस्तकालये अडयार पुस्तकालये च लभ्यते ।।
344. लघुवाक्यवृत्तिव्याख्या पुष्पाञ्जलिः - ग्रन्थोऽयं बाम्बे विश्वविद्यालयहस्तलिखितपुस्तकालये रायलासियाटिकपुस्तकालये बाम्बेनगरस्थे च लभ्यतेे ।। 
345. वाक्यवृत्तिव्याख्या लघुटीका - (20 E 22 AL)
346. वाक्यवृत्तिव्याख्या - (R. 3324d. MGOML)
347. वाक्यसुधाटीका - बरोडापुस्तकालये मद्रासराजकीयपुस्तकालये रायलासिमाटिक सोसाईटि कल्कत्तायाञ्च लभ्यते । 
348. वाक्यसुधाकरः - वाराणस्यां मुद्रिताद्भिन्नोऽयं ग्रन्थ अडयारपुस्तकालये लभ्यते ।। (7664 TSML)
349. वाक्यार्थचन्द्रिका - (N 35 - 14- MGOML)
350. वार्तिकसारः बृहदारण्यकवार्तिकसारः - (R. 3901 A. MGOML)
351. वासिष्ठशिक्षा - (D 957 MGOML)
352. वासिष्ठयोगकाण्डः IOL पुस्तकसूच्यां दृश्यते । 
353. वासुदेवमननसङ्ग्रहः - (20 E 49 AL) 
354. वासुदेवमननसङ्ग्रहसारः - (9 B 24 AL)
355. विज्ञानदीपिका - (7262 E BRd)
356. विवेकचूडामणि व्याख्या - 6205 तिरुपति सूच्यां दृश्यते । 
357. विवेकसारः । मद्रास पुस्तकालये अनन्तशयने च मुद्रितः लभ्यते । 
358. विवेकसिद्धिः - चित्तनैर्मल्यादेव ज्ञानसिद्धिरिति प्रतिपाद्य चित्तवैमल्योपायान् वदन्नयं ग्रन्थः एकोननवतिभिः पद्यैः पूर्ण अडयारपुस्तकालये लभ्यते । ( 9F 43 AL, 354 TCL)
359. विरक्तिरत्नावलिः - अद्वैतब्रह्मपाक्षात्कारोपयोगिनं बैराग्यभावं प्रापञ्चिकविप येषु अनासक्तिञ्च प्रतिपादयन्नयं ग्रन्थः अडयार पुस्तकालये (9 F 43 AL) लभ्यते । 
360. वेदान्तकारिकाः - (D. 14407 MGOML)
361. वेदान्ततत्वसङ्ग्रहः - ग्रन्थोऽयं विद्यारण्यपुरसूच्यां 97-98 दृश्यते । 
362. वेदान्ततत्वसारः - (4221 B.) तिरुपतिसूच्यां दृश्यते। 
363. वेदान्ततात्पर्यविवेकव्याख्या - बरोडासूच्यां दृश्यते । 
364. वेदान्तदर्शनम् आत्मोल्लासः - गुरुशिष्यसंवादरूपोऽयं ग्रन्थस्तत्वमसि वाक्यार्थनिर्णयपरः। ग्रन्थोऽयं तिरुवनन्तपुरसूच्यां (401 d. TCL) मद्रासराजकीयपुस्तकालये (R. 3132 C. MGOML) च लभ्यते । 
365. वेदान्तदीपिका - (R. 1567 MGOML)
366. वेदान्तप्रकरणम् - ( TSML, TCL AL) लभ्यते ।  
367. वेदान्तप्रकरणश्लोकसङ्ग्रहः - सङ्कलनात्मकोऽयं ग्रन्थः (361 A. TCD) दृश्यते । 
368. वेदान्तभूषणम् - ग्रन्थोऽयं प्रकरणप्रतिपाद्यान् विषयान् विवेचयति । ग्रन्थोऽयं विद्येन्द्रसरस्वतीकृते वेदान्ततत्वसाराख्ये ग्रन्थे निर्दिष्टः । तस्मादस्य कर्ता षोडशशतकात्पूर्वतन इति ज्ञायते । ग्रन्थेऽस्मिन् पञ्चनदक्षेत्रं निर्दिष्टम् । तस्मात् ग्रन्थकर्ता द्रविडदेशजस्स्यादित्यूह्यते । ग्रन्थोऽयममुद्रितः विद्यारण्यपुरसूच्यां 94, अडयारपुस्तकालये (8. i. 20 AL) च लभ्यते । 
369. वेदान्तभूषणव्याख्या - (12658 B.R.D.)
370. वेदान्तमननम् - (363 TCD)
371. वेदान्तमन्त्रविश्रामः - ग्रन्थोऽयं बरोडासूच्यां दृश्यते । 
372. वेदान्तमुक्तावलीटीका - कल्कत्तासंस्कृतकलाशालासूच्यां दृश्यते। 
373. वेदान्तरत्नम् - (359d. TCD)
374. वेदान्तवादार्थः - (1517 TSML)
375. वेदान्तविलासः - (7679 TSML)
376. वेदान्तविषयः - ग्रन्थोऽयं ब्रह्मणोऽन्यस्य सद्भावप्रतिषेधकस्य "अद्वितीयं ब्रह्म " इति वाक्यस्य अद्वितीयशब्दे कस्समासः? किं तत्पुरुषः ? उतबहुव्रीहिरिति परिशीलयति । सादिरनन्तोऽयं ग्रन्थः मद्रासराजकीयपुस्तकालये (D. 4743, D. 14807 MGOML) लभ्यते । 
377. वेदान्तविषयः सव्याख्यः - ग्रन्थोऽयं शृङ्गगिरिसूच्यां (33 C, D. 4744 R. 144 C, R. 2111 F, R. 845 C, MGOML) च दृश्यते । 
378. वेदान्तविषयकश्लोकसङ्ग्रहः - सङ्कलनात्मकोऽयं ग्रन्थः (7703 TSML D. 14411 MGOML, D. 1492 MGOML) लभ्यते । अस्यैव ग्रन्थस्य वेदान्तविषयकश्लोकानुक्रमणिकेति नामान्तरम् । 
379. वेदान्तव्यासोक्तसूत्रवृत्तिः - अमुद्रितोयं (IOL) लभ्यते। 
380. वेदान्तशास्त्रतत्वनिरूपणम् - (2618 V.B.S.)
381. वेदान्तसिद्धान्तमुष्टिः - (R. 3305 MGOML) पद्यबद्धोऽयं ग्रन्थः । 
382. वेदान्तसारः - (7583, 7344, 7349 TSML, 360  TCD)
383. वेदान्तसारटीका - बरोड पुस्तकालये लभ्यते । 
384. वेदान्तसारसङ्ग्रहः - (R. 110 D, R. 211 C MGOML)
385. वेदान्तसिद्धन्तः - (5371 CC PB)
386. वेदान्तसिद्धान्तचूडामणिः - (30 B. 10. AL.) 
387. वेदान्तसिद्धान्तप्रकाशः - (6058 (i) BRD.)
388. वेदान्तसिद्धान्तसूत्रतात्पर्यसङ्ग्रहः - (5562 a) तिरुपतिसूच्यां दृश्यते । 
389. वेदान्तसिद्धान्तसङ्ग्रहः - (AL. IOL)
390. वेदान्तसूत्रसिद्धान्तसङ्ग्रहः - (IOL)
391. वेदान्तसूत्रवृत्तिः - (MGOML)
392. वेदान्तसंज्ञाप्रकरणम् - अध्यारोप-सूक्ष्म-स्थूल-शाक्ति-निःश्रेयसादिशब्दानां अर्थः, षड्भावविकाराः, गुणत्रयम्, षडूर्मयः, ग्रन्थेऽस्मिन् विविच्यन्ते । ग्रन्थकारोऽयं वाराणसीवासीति परं ज्ञायते । ग्रन्थोऽयममुद्रित ( 7626 TSML, 9F. 22 AL, 9816 D. BRD. R. 1719 B. MGOML) लभ्यते । 
393. वेदान्तामृतम् - (5374 CC. PB.)
394. वेदान्तोपनिषत् - ग्रन्थोऽयं सुरेश्वराचार्यकृतवार्तिकार्थं सङ्गृहणाति । उपनिषच्छव्दोऽत्र ग्रन्थपरत्वेन गृहीत इति वेदान्तग्रन्थ इति च नामान्तरम् । बृहदारण्यकवार्तिकसारसङ्ग्रह इति च वक्तुं शक्यते । ग्रन्थोऽयं (7612 TSML 5375 CC. PB.) लभ्यते । 
395. वेदान्तोपन्यासः - (7613 TSML. AL.)
396. वैय्यसिकन्यायमालाधिकरणश्लोकानुक्रमणिका - (D. 14897 MGOML)
397. व्यामोहविद्रावणम् - (R 1825 MGOML)
398. व्यामतात्पर्यनिर्णयः - पूर्वमस्मामिरय्यण्णादीक्षितकृतः कश्चत एतत्पमामनामा ग्रन्थ निर्दिष्टः तस्मादेष भिद्यते। ग्रन्थोऽयं (19 G. 6 AL.) लभ्यते। 
399. व्याससूत्रपङ्गतिः - (7115 TSML 6975 TSML)
400. शतश्लोकीटीका - (2180 DC. BUL. IOL.)
401. शारीरकदर्पणम् - ग्रन्थोऽयं कल्कत्तासंस्कृतकलाशालापुस्तकालयसूच्यां दृश्यते । 
402. शाब्दनिर्णयदीपिका - तिरुवनन्तपुरसूच्यां दृश्यते । 
403. शारीरकन्यायमणिमाला - (R. 3830 MGOML)
404. शारीरकमीमांसाशास्त्रम् - सूत्रवृत्तिग्रन्थोऽयं (7120 TSML) लभ्यते । शार्दूलस्रग्घरावृत्तघटितैः पद्यैः मधुरानगरीक्षेत्रस्थभगवत् सुन्दरेशस्य स्तुतिशैल्यां निर्मितोऽयं ग्रन्थस्प्तरस्वतीमहालये लभ्यते । 
405. शारीरकमीमांसासारार्थः - सरस्वतीमहालयस्थोऽयं ग्रन्थः । 
406. शारीरकमीमांसासूत्रवॉत्तिः - (379 TED IOL, AL.) लभ्यते । 
407. शिष्यप्रबोधः - अडयारपुस्तकालये लभ्यते । 
408. शिष्यप्रश्नोपनिषत् - गुरुशिष्यसप्तप्रश्नीत्यपरनामायं ग्रन्थः (7668 TSML) लभ्यते 
409. शुद्धानन्दः - प्तव्याख्यः । ग्रन्थकारोऽयं वेङ्कटकृष्णारायसामयिकः गोदावरीतीरवासी, आन्ध्रदेशज इति परं ज्ञायते । ग्रन्थोऽयं (R. 2459 MGOML) लभ्यते । 
410. शाङ्करदशोपनिषत्प्रकाशः - 194 पञ्चाबसूच्यां दृश्यते । 
411. शङ्करपादरक्षाप्रयोगप्रत्याम्नायः - मणिमञ्जरीति द्वैतग्रन्थस्य खण्डनपरोऽयं ग्रन्थस्सप्तगोदावरीतीरग्रामवासिना केनापि लिखितस्स्यादिति ज्ञायते । ग्रन्थोऽयं (R. 1327 C. MGOML) लभ्यते । 
412. शङ्कराचार्यप्रादुर्भावनिर्णयः - नासिकसूच्यां दृश्यते ।
413. श्रुतिगीताव्याख्या - भावप्रकाशिका (R. 2729 MGOML)
414. श्रुतितात्पर्यनिर्णयः - औपनिषदानां वाक्यानां उपक्रमादिभिः षड्भिः प्रमाणैरद्वैतब्रह्मण्येव ऐदम्पर्यं साधयन्नयं ग्रन्थः अडयारपुस्तकालये (28-A 79 AL.) लभ्यते । 
415. श्रुतिसारसमुद्धरणटीका - मुद्रितपुस्तकाद्भिन्नोऽयं ग्रन्थः गौडपादाचार्यं द्रविडाचार्यं इष्टसिद्धिकारञ्च प्रमाणयति । ग्रन्थोऽयं (385 TCD.) पुस्तकसूच्यां दृश्यते । 
416. श्रुतिसारसमुद्धरणसम्बन्धोक्तिः - (R. 3679 MGOML)
417. श्रुतिसङ्ग्रहः - (6004 CC. PB.)
418. श्रुतिस्मृतिसङ्ग्रहः - (7699 TSML) सङ्कलनात्मकोऽयं ग्रन्थः । 
419. षोडशाध्यायी - (R. 3400 MGOML)
420. षोडशाध्यायटिप्पणी - (R. 3400 MGOML)
421. सकलसिद्धान्तसंग्रहः - (R. 3299 A. MGOML)
422. सच्चिदानन्दपदव्याख्या - बन्दरकारप्राच्यभाषापरामर्शालयपुस्तकालये लभ्यते। 
423. सच्चिदानन्दभुजङ्गम् - (6798 TSML)
424. सच्चिदानन्दस्वरूपविचारः - (D. 4752 MGOML)
425. सत्वनिरुक्तिः - (7986 BRD)
426. सद्वृत्तिप्रक्रिया - बरोडासूच्यां दृश्यते । 
427. सप्तश्लोकीप्रकरणम् - मुद्रितश्चायं (7760 DC. TSML Vol XIII)
428. सम्मिश्रपञ्चीकरणम् - (7747 TSML)
429. सर्वमतसङ्ग्रहः - ग्रन्थोऽयं अनन्तशयनग्रन्थमालायां (TSS 62) मुद्रितः । 
430. साक्ष्यसाक्षिविवेकः - चतुर्विशतितत्वोत्पत्तिप्रकारप्रदर्शनपूर्वकं त्वम्पदलक्ष्यार्थकूटस्थस्यैव साक्षिणः तत्पदलक्ष्यार्थेन परिपूर्णेन ब्रह्मणा सम्बन्धं बोधयन्नयं प्रकरणग्रन्थ अमुद्रितः (7678 TSML) पुस्तकालये लभ्यते । 
431. साधनचतुष्टयसम्पत्तिः - (AL)
432. साधनचतुष्टयपद्धतिः - (TSML, AL)
433. साधनपञ्चिका (BRD)
434. स्वात्मप्रकाशिका - (D 4774 MGOML)
435. स्वात्मबोधप्रकरणम् 99-100 विद्यारण्यपुरसूच्यां दृश्यते । 
436. स्वात्मसाक्षात्कारोपदेशः - शिवसुब्रह्मण्यसंवादरूपोऽयं ग्रन्थः अडयारपुस्तकालये (19. J. 2. AL) लभ्यते । 
437. स्वानुभवप्रकरणम् - शाङ्करात् स्वानुभावाद्भिन्नोऽयं ग्रन्थः (D. 4776 MGOML) लभ्यते । 
438. स्वानुभूतिः - (D. 4748 MGOML)
439. स्वानुभूतिप्रकाशिका - मुद्रितश्चायं (7770 DC. TSML Vol XIII)
440. सिद्धान्ततत्वविवेकः (284 BRD)
441. सिद्धान्ततत्वव्याख्या (BRD)
442. सिद्धान्तमञ्जरी - सव्याख्या । (1009 VBS)
443. सिद्धान्तबिन्दुसारः - ग्रन्थोऽयं कल्कत्तायां मुद्रितः । 
444. सिद्धान्तलेशङ्ग्रहव्याख्या - ग्रन्थेऽस्मित् अद्वैतानन्दः, विश्वेश्वरसरस्वती च निर्दिष्टौ । (D. 4768 MGOML)
445. सिद्धान्तसारः - अमुद्रितोऽयं पञ्जाबसूच्यां 990	बरोडापुस्तकालयेऽपि लक्ष्यते । अस्य व्याख्या सदाशिवकृतापि विद्यते । 
446. सिद्धान्तसारः - (3615 BRD.)
447. सिद्धान्तसारदीपकः - (6558 CC. PB.)
448. सिद्धान्तसिद्धान्तपद्धतिः - (AL)
449. सिद्धान्तसङ्ग्रहः - अमुद्रितोऽयं ग्रन्थः सृष्टेः पूर्वं परमात्मन एकस्यैव स्थितिः तस्मात् शक्तित्रयस्योत्पत्तिः । तादृशीनामेव शक्तीनांं मायाऽविद्यातामसी त्यादिनाम्ना व्यवहारः । मायाप्रतिबिम्बितः परमात्मा ईश्वरः । अविद्याप्रति बिम्बितः जीव इति प्रतिपादयति । ग्रन्थोऽयं (D. 4770 MGOML) लभ्यते । 
450. सिद्धस्वरूपविवरणम् - (TSML)
451. सूत्रभाष्यसारसङ्ग्रहः - ग्रन्थोऽयं हलषसूच्यां दृश्यते ।
452. सृष्टिक्रमः - (34. 0. 25. AL)
453. सृष्टिप्रकारः - (6650 CC PB)
454. सृष्टिप्रक्रिया - (19.H. 10. AL.)
455. सोपानपद्धति - (10383. B.R.D.)
456. सङ्ग्रहोक्तपञ्चीकरणम् - शृङ्गगिरिमठसूच्यां (7. B.) दृश्यते । 
457. संक्षिप्त वेदान्तशास्त्रप्रक्रिया - बरोडासूच्यां दृश्यते। 
458. संक्षेपशारीरकव्याख्या - तत्वबोधिनी । ग्रन्थोऽयं (387 TED. 573 DC TMPL.) सूच्यां दृश्यते । 
459. संक्षेपशारीरकटीका - कल्कत्ता संस्कृतकलाशालासूच्यां दृश्यते । 
460. संज्ञाप्रकरणम् - ब्रह्माहमस्मीति प्रत्यगभिन्नब्रह्मज्ञानान्मोक्ष इति प्रतिपादयन्नयं ग्रन्थः वेदान्तशास्त्रप्रसिद्धान् अध्यारोप-अपवाद-शक्ति-प्रपञ्च-अज्ञानादिशब्दान् विवृणोति। ग्रन्थोऽयं (2110 DC. BUL,  7997 DC. IOL.) लभ्यते । 
461. संन्यासस्य ज्ञानाङ्गत्वसमर्थनम् - (AL.)
462. संन्यासविचारः - (24 E 7 AL. )
463. सम्प्रदायनिरूपणम् - सिद्धान्ततत्वव्याख्या (1963 BRD.)
464. सम्मिश्रपञ्चीकरणम् - मुद्रितश्चायम् (7748 AC. TSML. Vol. XIII) 
465. स्नेहपूर्तिपरीक्षा - स्नेहपूर्तिरिति ग्रन्थः विशिष्टाद्वैतपरः । स च वाराणसी वासिना राममिश्रेण कृतः । वाराणस्यां मुद्रितश्व । तस्मिन् ग्रन्थे राममिश्रोत्प्रेक्षितानां अद्वैतमताक्षेपाणां स्नेहपूर्तिपरीक्षाख्येऽस्मिन् ग्रन्थे अद्वैत मतमण्डनपरे प्रतिक्षेपः परीक्षा च कृता । अद्वैतसाधकोऽयं ग्रन्थः पण्डित संस्कृतग्रन्थमालायां (P.S.S.3) मुद्रितः । 
466. स्वयम्बोधः - (2436 IOL)
467. स्वरूपम् - ग्रन्थोऽयं बरोडासूच्यां दृश्यते । 
468. हरिमीडेलघुदीपिका - मद्रासराजकीयपुस्तकालये लभ्यते । 
469. हस्तामलकश्लोकव्याख्या - मद्रासराजकीयहस्तलिखितप्राचीनपुस्तकालये (R. 3324 F. MGOML, D. 4780 MGOML,) पश्चाबसूच्यां 519 च लभ्यते । 
470. हंसगायत्री -  ग्रन्थोऽयं बरोडासूच्यां 	4862 लभ्यते। 
471. हंसपरमहंसनिर्णयः - ग्रन्थोऽयं मध्यप्रान्तीयवरार्ग्रन्थसूच्यां दृश्यते । 
परिशिष्टम् 
	अधोनिर्दिष्टाः केचन ग्रन्थाः जयपुरपोटीखानासूच्यां दृश्यन्ते। ते च ज्ञातकर्तृका अज्ञातकर्तृकाश्चोभयविधास्सन्ति । ते चात्र वर्णमालाक्रमेण सविशेषा निरूप्यन्ते - 
472. ईशावास्योपनिषद्भाष्यम् - बालकृष्णानन्दसरस्वती । 
473. उत्तरगीता टीका - गौडपादाचार्यंः ।
474. छान्दोग्यकारिकाः - कृष्णपण्डितः । 
475. ज्ञानविवरणम् - अज्ञातकर्तृनामायं ग्रन्थः । 
476. निम्बादित्यशङ्कराचार्यभाष्यसंयोजनम् - अज्ञातकर्तृकम् । 
477. पूर्णचन्द्रोदयः - नारायणेन्द्रसरस्वती । 
478. पञ्चदशीसारः - कृष्णरामचक्रवर्ती। 
479. ब्रह्मसूत्रतत्वार्थदीपिका - विद्याभूषणः । 
480. ब्रह्मसूत्रवृत्तिः - जयसिम्हः । 
481. ब्रह्मसूत्रवृत्तिप्रबोधिनी - शेषयादवः । 
482. ब्रह्मप्रकाशिका - वासुदेवसरस्वती। 
483. माण्डूक्योपनिषद्व्याख्या निगूढार्थदीपिका - नारायणः।
484. वेदान्तदीपिका - अज्ञातकर्तृका । 
485. शारीरकभाष्यसारः - बालकृष्णानन्दसरस्वती ।
486. शारीरकमीमांसोपयोगि श्रुतिवाक्यसंग्रहः - नृसिम्हाश्रमिशिष्यः । 
487. शारीरकमीमांसोपयोगि श्रुतिवाक्यसंग्रहव्याख्या - नृसिम्हाश्रमिशिष्यः । 
488. शारीरकार्थसंक्षेपः - महादेवः । 
489. शारीरकार्थसंक्षेपविवृतिः - महादेवः ।
490. सच्चिदानन्दानुभवदीपिका - वासुदेवब्रह्म। 
491. स्वानुभूतिनाटकम् - ग्रन्थोऽयमनन्तपण्डितेन कृतः । प्रबोधचन्द्रोदयवत् आध्यात्मिकवस्तुप्रधानमिदं नाटकम् । ग्रन्थोऽयं जयपु पोटीकानासूच्यां (Vol XXVII 2/12) दृश्यते । अपरे केचन ग्रन्थाः अद्वैतसिद्धान्तं पोषयन्तः समीक्षापद्धत्यां लिखिताः मुद्रिताश्च वर्तन्ते । ते चात्र वर्णमालाक्रमेण निरूप्यन्ते
492. द्रविडात्रेयदर्शनम् - श्रीरामशास्त्री । 
	ग्रन्थोऽयं शङ्करभगवत्पादेभ्यः प्राचीनानां द्रविडाचार्याणां ब्रह्मनन्दिनाञ्च मतवादादिकं विविच्य दर्शयति । ग्रन्थोऽयं मद्रासनगरे मुद्रितः । 
493. ब्रह्मसूत्रभाष्यनिर्णयः - चिद्धनानन्दपुरी । 
494. भगवद्गीता भारतीयदर्शनानि च - अनन्तकृष्णशास्त्री । 
495. सुरेश्वरहृदयम् -- गणेशशर्मा । 
