\chapter{उपनिषत्प्रस्थानम् }
वैदिकसाहित्ये उपनिषदां स्थितिरन्यादृशी । औपनिषदानां महर्षीणां कार्यात्मकविश्वब्रह्माण्डस्य अभिन्नत्वे अखण्डत्वे च महान् विश्वासः दरीदृश्यते । औपनिषदा मुनयस्सुखदुःखेभ्य उदासीना दृश्यन्ते । सान्तस्यानन्तेन सम्बन्धः औपनिषदानां महर्षीणां रहस्यात्मकवाणीष्वेव प्रथमं अभिव्यक्तः । 
ब्राह्मणग्रन्थास्तु कर्मसु पुरुषं प्रेरयन्ति । उपनिषदस्तु ज्ञाने प्रेरयन्ति । जीवनात् जीवनसम्बद्धविचारे उपनिषदामैदम्पर्यं दृश्यते । वेदकालिका भारतीया ऐहिकस्यैश्वर्यस्य साधने यथा बद्धपरिकरा दृश्यन्ते, यथा च ब्राह्मणकालिका भारतीयास्स्वर्गादिपरलोकेप्सवश्व दृश्यन्ते न तथोपनिषत्कालिका भारतीयाः । परन्तु ते साधका ऐहिकैश्वर्यात् परलोकेच्छायाश्चोदासीनाः मुमुक्षवश्च दृश्यन्ते । गतिशीलैः प्राकृतिकवस्तुभिरेव जनिमतां न सम्बन्धः, परन्तु अनिर्वचनीयेन केनचित् स्थिरतत्वेनैव तेषां सम्बन्ध इत्युपनिषत्कालवर्तिनां भारतीयानां सिद्धान्तः । अतएवोपनिषदां सिद्धान्ताः विचारप्रधाना वर्तन्ते । 
उपनिषदः गम्भीरानर्थान् काव्यसुलभया गद्यपद्यात्मकशैल्या प्रतिपादयन्ति । परम् उपनिषदः नैककर्तृकाः । यत एकस्यामेवोपनिषदि शिक्षकभेदो दृश्यते । 

उपनिषदां संख्याः - 
सवोंपनिषदां मध्ये सारमष्टोत्तरं शतम् ।
सकृच्छ्रवणमात्रेण सर्वाघौघविकृन्तनम् ।। (१ - ४४ ) इति 
अष्टोत्तरशतोपनिषदां सारभूतत्वं प्रतिपादयन्त्या मुक्तिकोपनिषदा अष्टोत्तरशताधिकानां उपनिषदां सत्वे प्रमाणमप्यावेदितं भवति। परन्तु न तास्सर्वा उपनिषद अद्यावधि प्रकाशिता उपलब्धाः वा सन्ति । काश्चनोपनिषद अडयारपुस्तकालयात् प्रकाशतां नीतः। तास्वष्टोत्तरशतोपनिषत्सु दशोपनिषद ऋग्वेदीयाः, एकोनविंशत्युपनिषदश्शुक्लयजुर्वेदीयाः, द्वात्रिंशदुपनिषदः कृष्णयजुर्वेदीयाः, षोडशोपनिषदस्सामवेदीयाः, एकत्रिंशदुपनिषद अथर्ववेदीया इति ज्ञायते । 

अष्टोत्तरशतोपनिषत्स्वपि विषयप्रतिपादनदृष्ट्या त्रयोदशोपनिषद एव प्राचीनतमा इति ज्ञायन्ते । ऐतरेय - कौषीतक्युपनिषद ऋग्वेदीयाः, छान्दोग्याकेनोपनिषदस्मामवेदीयाः, तैत्तरीय नारायण - कठ - श्वेताश्वतर - मैत्रायण्युपनिषदः कृप्णयजुर्वेदीयाः, ईशावास्योपनिषच्छुक्लयजुर्वेदीया बृहदारण्यकोपनिषदपि शुक्लयजुर्वेदीया, मुण्डकमाण्डूक्यप्रश्नोपनिषद अथर्ववेदीया इति च प्रसिद्धाः । एतास्वपि दशोपनिषदामेव शङ्करभगवत्पादैर्भाष्यं कृतमिति ता एवात्र प्रधानतमास्स्वीक्रियन्ते। नृसिम्होत्तरतापिनीकौषीतकी- श्वेताश्वतरोपनिषदामपि व्याख्याः दृश्यन्ते । अतः अद्वैताचार्यैः व्याख्याताः अथवा व्याख्यातत्वेन प्रसिद्धाः सर्वा अप्युपनिषदः निर्देशार्हा इति सामान्यं नियमं मनसि कृत्वा शङ्करैरव्याख्याता अपि नारायणाश्रमि - शङ्करानन्दादिभिर्व्याख्याता अद्वैतमतप्रतिपादकास्सर्वा अप्युपनिषदः निर्दिष्टाः ।

उपनिषदां पौर्वापर्यम् - 
सर्वासूपनिषत्सु का वा उपनिषत् प्राचीना ? का वा नवीना ? इति निश्चेतुं न शक्यते । यतोऽस्मिन् विषये मतभेदास्सप्रमाणाः दृश्यन्ते ।  

१. प्रोफेसर - डायसनस्तु गद्यशैल्यां दृश्यमाना उपनिषद एव प्राचीना इति "फिलासफि आफ द उपनिषत्स् " इति ग्रन्थे अभिप्रैति । 

२. प्रो - रामचन्द्ररानडे तु "ए कंस्ट्रक्टिव सर्वे आफ़ उपनिषदिक फिलासफ़ि " नामके ग्रन्थे उपनिषदां पौर्वापर्यमेवमभिप्रैति - बृहदारण्यक - छान्दोग्यईश - केन - ऐतरेय - तैत्तरीय - कौषीतकी - कठ-मुण्डक - श्वेताश्वतर - प्रश्न - मैत्रायणि - माण्डूक्या इति उपनिषदां उत्तरोत्तरकालोत्पन्नतां प्रतिपादयन्ति। तस्य मतेन उपनिषदः पञ्चभिर्वर्गैः परिगणिताः - प्रथमवर्गे बृहदारण्यकछान्दोग्ये, द्वितीयवर्गे ईशकेनोपनिषदौ, तृतीयवर्गे ऐतरेयतैत्तरीयकौषीतक्यः, चतुर्थवर्गे कठमुण्डकश्वेताश्वतराः, पञ्चमरवर्गे प्रश्नमैत्रीमाण्ड्क्यानि चेति । एषु प्रथमवर्गः प्राचीनतमः, अन्तिमश्चार्वाचीन इति । 
३. बेलवलकर महाशयस्तु - एकस्यामेवोपनिषदि भिन्नकालिकानां रचनाविशेषाणां दर्शनात् , एकस्या एवोपनिषदः केचन भागाः प्राचीनाः, केचनार्वाचीना भवन्तीति वर्णयति । 
४. सर् - राधाकृष्णमहोदयास्तु वेदकालादारभ्य क्रिस्तोः पूर्वं (B.C) षष्ठशतकात्पूर्वावधिकः काल उपनिषदां काल इति निश्चिन्वन्ति। 
प्राचीनतमासु उपनिषत्सु दार्शनिकविचारधारायाः प्राधान्यं, अनन्तरभाविनीषु धार्मिकविचारधारायाः भक्तिधारायाश्च प्रवेशस्सन्दृश्यते । 
उपनिषत्सु शाण्डिल्य - दध्यौच - सनत्कुमार - आरुणि - याज्ञवल्क्य -  उद्दालकरैक्व - प्रतर्दन - अजातशत्रु - जनक - पिप्पलाद् - वरुण - गार्ही - मैत्रेयी - नचिकेतिःप्रभृतीनां बहूनां आचार्याणां महर्षीणाञ्च नाम दृश्यते । केचिदेतेषु सपत्नीकारसापत्याश्च दृश्यन्ते । श्वेतकेतोः पुत्रः आरुणिः, वरुणस्य पुत्रः भृगुः, याज्ञक्ल्क्यः द्विभार्य इत्यादि । औपनिषदास्सिद्धान्तास्सर्वेऽपि पतिपत्नींंसवादशैल्यां पितृपुत्रसंवादशैल्याञ्च प्रतिपादिता इति तु विशेषत अवधेयार्हः विषयः । 
उपनिषच्छब्दनिर्वचनम् - 
कठोपनिषदां प्रस्तावनाभाष्याप्रामाण्यात् , मुण्डकोपनिषदां प्रस्तावनाभाष्यप्रामाण्यात् , केनोपनिषदि ( ४- ३२) "उपनिषदमब्रूत " इति वाक्यस्य भाष्यप्रामाण्यात् , छान्दोग्योपनिषत्स्थाष्टमाध्यायाष्टमखण्ड चतुर्थखण्डिकास्थभाष्यप्रामाण्यात् बृहदारण्यकोपनिषत्प्रस्तावनाभाष्यप्रामाण्याच्च संसारबीजविनाशिनी या विद्या सा उपनिषच्छब्देन मुख्यया वृत्या बोध्यत इति निश्चीयते । तादृशविद्याप्रतिपादकत्वात् लक्षणया ईशावास्यादयो ग्रन्था अप्युपनिषच्छब्देन व्यवह्रियन्ते । 
उपनिषदां सिद्धान्ताः -
प्रायः मुख्यासूपनिषत्सु ब्रह्मस्वरूपं, तत्प्रतिपत्तय उपासनाः, काश्चित् स्वतन्त्राः, काश्चित्कर्माङ्गत्वेन, आख्यायिकासहिताः प्रतिपादिताः । ब्रह्मविद्याप्राप्तेस्साधनानि ब्रह्मविद्याजिज्ञासूनां आवश्यकगुणविशेषाः, कर्ममार्गस्य जटिलत्वम् , ज्ञानमार्गस्य सुगमत्वम् , अनासक्तकर्मपरत्वं वैराग्यञ्चेत्येवमादीनि आत्मज्ञानाप्तेस्साधनानि, न तु तर्कविचारः युक्तिवादा वा इत्यादि प्रतिपादितम्। 
उपनिषत्सु जाग्रत्तत्वानुशीलनपराः नैकविधाः कथाश्श्रूयन्ते । औपनिषदाः मननशीला मुनयः जगतः मूलतत्वानुसन्धाने बद्धश्रद्धाः दृश्यन्ते । प्रापञ्चिकानां नैकविधानां वस्तूनां विभिन्नतायां एकत्वसम्पादकं तत्वं किं स्यात् ? तादृशतत्वलाभोपायः क ? इति शङ्काकुलाः पञ्चभूतानुशीलनमार्गेण आत्मतत्वविचारपराः भूत्वा स्वस्मिन्नेव तादृशं तत्त्वं प्रत्यक्षीचक्रुः । एतादृशस्वानुभवशीलानां तेषां क्रान्तदर्शिन्याः दृष्टेः आन्तरबाह्यजगतोर्न कोऽपि भेदः विषयीबभूव। छान्दोग्योपनिषदीयया इन्द्रविरोचनकथया ( 8 - 7 - 12 ), आरुणिश्वेतकेतुसंवादरूपया न्यग्रोधकथया (6-12) च " तत्त्वमसि "  " अयमात्मा ब्रह्म " इत्येष एव सिद्धान्तः प्रतिपादितः । एवं सप्रपञ्च- निष्प्रपञ्च ब्रह्मस्वरूपर्वर्णना, मनस्तत्वविवेचना, कर्म -  सन्यास - मोक्षसिद्धिान्ताः, भारतीयविविधदर्शनमूलभूतास्सर्वेऽपि सिद्धान्ताः प्रतिपादिताः । 
एवञ्च भारतीयाध्यात्मिकविचारधाराया उपजीव्यत्वात् उपनिषदः प्रस्थानत्रये प्रथमगणनामर्हन्ति । यद्यप्युपनिषद अंसख्यास्तथापि या उपनिषद अद्वैतसिद्धान्तमूलभूताः, याश्च गौडपादशङ्करभगवत्पादादिभिरद्वैताचार्यैः व्याख्यातास्त एवात्र सव्याख्योपव्याख्या वर्णमालाक्रमेण निर्दिश्यन्ते ।। 
१. अथर्वशिखोपनिषत् - 
अस्यामुपनिषदि सकलवेदमूलभूतस्य प्रणवस्य स्वरूपं, तदीयमात्राणां देवतादयः, प्रणववाचकानां ओङ्कारतारकादिपदानां व्युत्पत्तिः, तद्ध्यानध्यातृघध्येयस्वरूपं च इत्येतत् सर्वं निरूप्यते । इयमुपनिषद् आनन्दाश्रममुद्रणालये मुद्रिता । अस्या व्याख्याः - 
नारायणाश्रमिकृता - अथर्वशिखोपनिषद्दीपिका 
अथर्वशिखोपनिषदां व्याख्यात्मकोऽयं ग्रन्थः आनन्दाश्रममुद्रणालये मुद्रितः । अस्य कर्ता श्रीनाथपौत्र रत्नाकरभट्टपुत्रः आनन्दात्मशिष्यः नारायणाश्रम इति ज्ञायते । अनेन विरचितायां " माणडूक्योपनिषद्दीपिकायां " अमुद्रितायां सरस्वतीमहालयपुस्तकालयस्थायां (1556 D.C.T.S.M.L) अानन्दगिरिर्निर्दिष्टः । आनन्दाश्रममुद्रितायां जाबालोपनिषद्दीपिकायां आनन्दात्मा अध्यात्मगुरुरिति निर्दिष्टः। आनन्दात्मा तु शङ्करानन्दस्यापि गुरुरिति नारायणाश्रमस्य कालः त्रयोदशचतुर्दशशतकम् (1275 - 1350 A.D) इति निश्चीयते ।। 
शङ्करानन्दविरचिता - अथर्वशिखोपनिषद्दीपिका 
आनन्दात्मनः विद्यातीर्थस्य च शिष्योऽयं शङ्करानन्दः विद्यारण्यस्य गुरुः त्रयोदशशतकारम्भकालवासी (1275 - 1350 A.D.) इति ज्ञायते । अमुद्रितोऽयं ग्रन्थः सरस्वतीमहालयसूच्यां (1427 TSML) अडयारपुस्तकालये बरोडापुस्तकालये च दृश्यते । शङ्करानन्दमधिकृत्याधिकं सूत्रवृत्तिप्रकरणे प्रकरणग्रन्थप्रकरणे अद्वैताचार्यप्रतिपादनावसरे च प्रतिपाद्यते ।। 
शङ्कराचार्यकृतं - अथर्वशिखोपनिषद्भाष्यम् (१) 
ग्रन्थोऽयं शङ्कराचार्यकृतत्वेन अडयारपुस्तकालयस्थे (30 B 22 ग्र 6 AL) आदर्शपुस्तके दृश्यते । अमुद्रितश्चायं ग्रन्थः । अस्य कर्ता न प्रसिद्धश्शङ्कराचायों भवितुमर्हति । तथा प्रसिद्धेरभावात् । अन्तरङ्गपरीक्षायान्तु कृतायां सर्वथा न शङ्कराचार्यकृतोऽयं ग्रन्थ इत्येव प्रतीयते । यतः - यासामुपनिषदां शङ्कराचार्यैर्व्याख्या कृता तासां न व्याख्या कृता तासां व्याख्याप्रस्तावे नारायणाश्रमिणा " शङ्करोक्त्युपजीविना " इत्युच्यते । यासां न व्याख्या कृता तासां व्याख्याप्रस्तावे " श्रुतिमात्रोपजीविना " इत्येव निर्दिश्यते । दृश्यते चात्र श्रुतिमात्रोपजीविना इति । तस्मादपि कारणात् नेयं व्याख्या शङ्कराचार्यकृता इत्येव प्रतिभाति ।। 
उपनिषद्ब्रह्मेन्द्रकृतं - विवरणम् अडयार पुस्तकालये मुद्रितम् । 
२. अथर्वशिर उपनिषदः - 
अस्यामुपनिषदि देवानां को भवानिति रुद्रं प्रति प्रश्नः, तदुत्तरेण तस्य सर्वात्मकत्वज्ञानेन तन्नतिपूर्वकं बहुधा तस्य स्तुतिः, तत्प्रतिपादकत्वेन प्रसिद्धानां " ओङ्कार - प्रणव - सर्वव्याप्यनन्तादिपदानां निर्वचनं, इत्यादयो विषयाः प्रतिपादिताः । ग्रन्थोऽयमानन्दाश्रममुद्रणालये मुद्रितः । अस्या व्याख्याः - 
नारायणाश्रमकृता - अथर्वशिर उपनिषद्दीपिका 
श्रुतिमात्रोपजीविना नारायणाश्रमिणा रचितेयं दीपिका आनन्दाश्रमुद्रणालये मुद्रिता । अस्य कालादि पूर्ववत् । 
शङ्करानन्दकृता - अथर्वशिर उपनिषद्दीपिका   
ग्रन्थोऽयं आनन्दाश्रममुद्रणालये मुद्रितः । शङ्करानन्दस्य कालादि पूर्ववत् । 
शङ्कराचार्यकृतं - अथर्वशिर उपनिषद् भाष्यम् ?
अमुद्रितोऽयं ग्रन्थः अडयारपुस्तकालये (30 B 22 ग्र 28 AL) लभ्यते । उपनिषद्ब्रह्मेन्द्रकृतं विवरणं अडयार पुस्तकालये मुद्रितम् । 
३. अमृतनादोपनिषत् - 
अस्यामुपनिषदि शास्त्राभ्यासस्य ब्रह्मज्ञानफलकत्वमुपवर्ण्य प्रत्याहारध्यानप्रणायामधारणातर्कसमाध्यभिधेयाङ्गषट्ककस्य योगस्य तत्तदङ्गलक्षणकथनपूर्वकं अभ्यसनप्रकारं फलञ्च निरूप्य प्राणादिवायूतां स्थानवर्णादिकमभिधीयते । मुद्रिता चेयमुपनिषदानन्दाश्रममुद्रणालये । अस्या व्याख्याः - 
(क) शङ्करानन्दविरचिता - अमृतनादोपनिषद्दीपिका 
नारायणविरचिता, उपनिषद्ब्रह्मेन्द्रविरचिताश्च व्याख्यास्सन्ति । मुद्रिता आनन्दाश्रममुद्रणालये अडयार पुस्तकालये च । 
४. अमृतबिन्दूपनिषत् -
अस्यामुपनिषदि मनश्शुद्धिप्रशंसापूर्वकं ब्रह्मज्ञानावाप्तये साधनमुपवर्ण्य मुक्तिस्वरूपं निरूप्य ब्रह्मज्ञानं प्रशस्यते । मुद्रिताचेयमुपनिषदानन्दाश्रममुद्रणालये । अस्या व्याख्याः - 
(क) नारायणाश्रमि विरचिता - अमृतविन्दूपनिषद्दीपिका 
श्रुतिमात्रोपजीविना नारायणाश्रमिणा विराचितेयं दीपिका आनन्दाश्रममुद्रणालये मुद्रिता । अस्य कालादि पूर्ववत् ।। 
(ख) शङ्करानन्दविरचिता - अमृतबिन्दूपनिषद्दीपिका 
व्याख्याचेयं मुद्रिताऽनन्दाश्रममुद्रणालये । कालादि पूर्ववत् । 
(ग) सदाशिवेन्द्रसरस्वतीकृता - अमृतविन्दूपनिषद्दीपिका 
अमुद्रितेयं व्याख्या मद्रासराजकीयप्राचीनहस्तलिखितपुस्तकालये (R 1492 M.G. O. M. L) लभ्यते । अस्य कर्ता कामकोटिपीठाधीशस्य महादेवेन्द्रसरस्वत्याः प्राचार्यः सदाशिवेन्द्रसरस्वती षोडशशतकीयः (1550 - 1650 A.D.) इति ज्ञायते । अनेेन आत्मानात्मविवेकोऽपि रचितः ।। उपनिषद्ब्रह्मयोगिकृतं विवरणमपि मुद्रितम् । 
५. आत्मप्रबोधोपनिषत् - 
अत्र प्रणवस्वरूपं प्रशस्य तत्सहिताष्टाक्षरेण महामन्त्रेण ब्रह्मानुसन्धानं कुर्वत उत्तमलोकाद्यवाप्त्या परमानन्दानुभवप्रकारोऽभिधीयते । ऋक्शाखीयायां अस्यामुपनिषदि आत्माद्वैत - जीवन्मुक्ततादि - प्रतिपादिकाः कारिकाः विशेषत उल्लेखार्हाः । मुद्रिता चेयमुपनिषदानन्दाश्रममुद्रणालये । अस्या व्याख्या :-
(क) नारायणाश्रमिकृता - आत्मप्रबोधोपनिषद्दीपिका 
श्रुतिमात्रोपजीविना नारायणाश्रमिणा रचितेयं व्याख्यानन्दाश्रममुद्रणालये मुद्रिता गद्यभागानामेव विद्यते । 
(ख) शङ्करानन्दविरचिता - आत्मप्रबोधदीपिका 
अमुद्रितोऽयं ग्रन्थः श्रृङ्गगिरिसूच्यां (10. C) दृश्यते ।। उपनिषद्ब्रह्मेन्द्रव्याख्या च मुद्रिता ।
६. आरुणिकोपनिषत् - 
आरुणिप्रजापतिप्रश्नप्रतिवचनरूपायामस्यामुपनिषदि सर्वसङ्गपरित्यागपूर्वकसन्यासाश्रमग्रहणप्रकारमुपवर्ण्य सन्यासिनां धर्मांश्चाभिधायान्ते परमपदावाप्तिरिति प्रतिपादितम् । मुद्रिता चेयमुपनिषदानन्दाश्रममुद्रणालये । अस्य व्याख्याः -
(क) नारायणाश्रमिकृता - आरुणेकोपनिषद्दीपिका 
" नारायणेन रचिता शङ्करानन्दपाठत " इति दीपिकायामस्यां दर्शनात् शङ्करानन्ददीपिकाया अनन्तरं रचितेयं दीपिकेति ज्ञायते । मुद्रिता चेयमानन्दाश्रममुद्रणालये । 
(ख) शङ्करानन्दरचिता - आरुणिकदीपिका 
मुद्रिताचानन्दाश्रमे । उपनिषद्ब्रह्मेन्द्रव्याख्या च मुद्रिता । 
७. ईशावास्योपनिषत् - 
अस्यामुपनिषदि चिदचित्स्वरूपस्य जगतः परमात्माधीनस्वरूपस्थित्यादिमत्वं, आदेहपातं यथाशक्ति ब्रह्मविद्याङ्गभूतकर्मयोगस्यानुष्ठेयत्वं, अविदुषो निन्दनं, परमात्मनः विचित्रानन्तशक्तिमत्वं, ब्रह्मात्मकजगदनुसन्धानस्य फलं, ईशेशितव्यवेदिनः ज्ञानयोगाद्युपदेशः केवलकर्मयोगावलम्बिनां विनिन्दनं भगवद्भक्तिनिष्ठस्यावश्यानुसन्धेयोपदेश इत्यादिकमुपवर्णितं दृश्यते । यजुर्वेदीयेयमुपनिषत् आनन्दाश्रममुद्रणालये (ASS 5) मुद्रिता । अस्या उपनिषदः रचनाकालः क्रिस्तोः पूर्वं सप्तमशतकम् (700 BC) इति विमर्शकाः। अस्याः व्याख्याः, उपव्याख्याश्च- 
(क) शङ्कराचार्यकृतं - ईशावास्योपनिषद्भाष्यम् 
मुद्रितञ्चेदं भाष्यमानन्दाश्रममुद्रणालये (ASS 5)। शङ्कराचार्यकालादि सविस्तरं प्रकरणग्रन्थप्रस्तावे अद्वैताचार्यप्रस्तावे च प्रतिपाद्यते । 
(A) आनन्दगिरिकृता - ईशावास्यभाष्यव्याख्या
व्याख्यायामस्यां "तत्वालोकः " निर्दिष्टः । भास्करमतं खण्डितञ्च । मुद्रितश्चायं ग्रन्थ आनन्दाश्रममुद्रणालये (ASS 5) ।
व्याख्याया अस्याः कर्ता आनन्दज्ञानापराभिध आनन्दगिरिः । सन्यासस्वीकारात्पूर्वं जनार्दन इत्यस्यैव नाम । गुजरातप्रान्तजोऽयं द्वारकास्थ शङ्करपीठाधीश आसीदिति प्रसिद्धिः। आन्ध्रदेशज इति साम्प्रदायिकाः । अनुभूतिस्वरूपाचार्यशुद्धानन्दयोशिशष्योऽयं अखण्डानन्दस्य प्रज्ञानानन्दस्य च गुरुः, कलिङ्गदेशाधिपतेर्नृसिम्हदेवस्य सामयिकस्त्रयोदशशतकीय (1260 - 1320 A.D) इति ज्ञायते । अदसीया अन्ये ग्रन्थाः प्रकरणप्रस्तावे अद्वैताचार्यप्रस्तावे च प्रतिपाद्यन्ते ।। 
(B) शिवानन्दयतिकृतं - ईशावास्यभाष्यटिप्पणम् 
अमुद्रितोऽयं पूर्णग्रन्थः मद्रासराजकीयहस्तलिखितपुस्तकालये (R 3882 M.G.O.M.L) लभ्यते ।
अस्य कर्ता रामनाथविदुष आचार्यस्सप्तदशाष्टादशशतकमध्यवर्ती (1650 - 1750 A.D) शिवानन्दयतिरिति ज्ञायते । अनेनरचित आनन्ददीपाख्य. प्रकरणग्रन्थः अन्यत्र निरूपितः ।। 
(ख) अनन्ताचार्यकृता - वेदार्थदीपिका 
समग्रवेदभागस्य व्याख्यात्मकोऽयं ग्रन्थः । प्रकृतिप्रत्ययविवेचनपूर्वकं सप्रमाणं प्रक्रियां निरूपयन्नयं ग्रन्थ आनन्दाश्रममुद्रणालये मुद्रितः । 
अस्य कर्त काण्वशाखीयः नागदेवभट्टपुत्रः, समग्रवेदभागव्याख्याता अनन्ताचार्यः। अनेन विधानपारिजाताख्यः ग्रन्थः (1625 A.D) काले रचितः । 
(ग) आनन्दभट्टकृतम् - ईशावास्यभाष्यम् ।
ग्रन्थोऽयमानन्दाश्रममुद्रणालये मुद्रितः । ग्रन्थेऽस्मिन् शङ्करानन्दः निर्दिष्टः। अस्य कर्ता जातवेदभट्टजाह्नव्योः पुत्रः वासुदेवपुरी - आत्मावासपूज्यपादशिष्य आनन्दभट्टश्शङ्करानन्दादर्वाचीन इति परं ज्ञायते । 
(घ) उपनिषद्ब्रह्मेन्द्रकृतम् - ईशावास्यविवरणम् 
भावेन वाक्यविन्यासेन च शाङ्करं भाष्यं पूर्णतयानुसरदिदं विवरणं अडयारपुस्तकालये मुद्रितम् ।
अस्य कर्ता प्रथमवासुदेवेन्द्रप्रशिष्यः द्वितीयवासुदेवेन्द्रशिष्यः रामचन्द्रेन्द्रसतीर्थ्यः, कृष्णानन्दगुरुः, अष्टादशशतकापरार्घकालवासी (1765 - 1850 A.D) उपनिषब्रह्मेन्द्रापरनामा रामचन्द्रेन्द्र इति ज्ञायते । 
(ङ) उवटाचार्यकृतम् - ईशावास्यभाष्यम् 
ईशावास्योपनिषदां विवरणात्मकः माध्यन्दिनशाखान्तर्गतस्य समग्रवेदभागस्य व्याख्यात्मकश्चायं ग्रन्थ आनन्दाश्रप्तमुद्रणालये मुद्रितः । अस्य कर्तां आनन्दपुरवास्तव्यस्य वज्रटभदृस्य सूनुरवन्तीपुरवासी भोजराजसामयिकः उवटाचार्य इति शुक्लयजुर्वेदसंहिताभाष्यप्रमाणाज्ज्ञायते । यद्ययं भोजराजः धारानगराधीशः सरस्वतीकण्ठाभरणशृङ्गारप्रकाशकारस्यात् तर्हि तस्य शासनकालः (1010 - 1062 A.D) इति यफिग्राफिका इण्डिकापत्रिकायाः प्रथमभाग (230 page) प्रमाणात् ज्ञायते । तस्मादुवटाचायोंऽपि एकादशशतकीय इति निर्णेतु शक्यते ।। 
(च) गोपालानन्दकृता - ईशावास्यटीका 
अमुद्रितोऽयं ग्रन्थः बरोडापुस्तकसृच्यां (4527 DC BRD) दृश्यते । अस्य कर्ता सहजानन्दशिष्यः गोपालानन्दः माकिं सप्तदशएकोनर्विशतिशतकमध्यवर्तीति ज्ञायते ।। 
(छ) नरसिम्हभट्टकृता - ईशावास्यटीका 
अमुद्रितोऽयं ग्रन्थः मध्यप्रान्तीयबरार्ग्रन्थसूच्यां (481 CCPB) दृश्यते । अस्य कर्ता अद्वैतचन्द्रिकाकारः, नागेश्वररामभद्राश्रमयोश्शिष्यः रघुनाथभट्टपुत्रः किम्मिडि (खिमुण्डिः वंशजस्य जगन्नाथनृपतेस्सामयिकः मिथिलावासी अष्टादशशतकीयः नरसिम्हभट्ट इति ज्ञायते ।।
(ज) ब्रह्मानन्दसरस्वतीकृतम् - ईशावास्यरहस्यम् 
आनुष्ठुभेण छन्दसा घटितैः पद्यै रचितोऽयं ग्रन्थ आनन्दाश्रममुद्रणालये मुद्रितः। अस्य कर्ता ब्रह्मानन्दसरस्वती। यद्ययं गुरुचन्द्रिकाकारस्स्यात्तर्हि नारायणतीर्थंशिष्यस्सप्तदशशतकीय (1600 - 1700 A.D) इति निर्णेतु शक्यते ।।
(झ) भास्करानन्दसरस्वतीकृता - ईशावास्यव्याख्या 
शाङ्करभाष्यानुसारिणीयं व्याख्या भारतीजीवनमुद्रणालये वाराणस्यां मुद्रिता। अस्य कर्ता शाण्डिल्यगोत्रजः कान्यगुब्जब्राह्मणः, गोभिलसूत्री, सामवेदी, कौथुमशाखीयः, पूर्वाश्रमे वेङ्कटमिश्रनामा, मिश्रीलालमिश्रपुण्यमत्योः पुत्रः अनन्तरामपण्डितशिष्यः, पूर्णानन्दस्वामिनो लब्धदीक्षः, मणिराजचौबेदैहित्रः मैथिलोऽपि काशीवासी भास्करानन्दसरस्वती एकोनर्विशतिशतकीयः (1800 -1900 A.D) इति ज्ञायते ।
(ञ) रामचन्द्रपण्डितकृता - ईशावास्यरहस्यविवृतिः 
प्रतिमन्त्र अर्थसंग्राहकश्लोकेन साकं भाष्यार्थसंग्राहकोऽयं ग्रन्थ आनन्दाश्र मुद्रणालये मुद्रितः । अस्य कर्ता माध्यन्दिनशाखाध्यायी, आत्रेयगोत्रजः, श्रीसिद्धपुत्रः सिद्धेश्वरशिष्यः एकोनविंशतिशतकवासी (1817 A.D) रामचन्द्रपण्डित इति ज्ञायते ।। 
(ट) श्रीधरानन्दकृतः - ईशावास्यविवेकः 
अमुद्रितोऽयं ग्रन्थः अडयारपुस्तकालये (33 H.L आ 2 A.L.) लभ्यते । 
(ठ) सच्चिदानन्दाश्रमिकृता - ईशावास्यदीपिका 
अमुद्रितोऽयं ग्रन्थः बरोडापुस्तकसूच्यां (1969) दृश्यते । अस्याः कर्ता नृसिम्हाश्रमिशिष्यस्सच्चिदान्दाश्रमीति ज्ञायते । यद्ययं नृसिम्हाश्रमी अद्वैतदीपिकाकारस्स्यात्तर्हि षोडशशतकीयोऽयमिति निर्णेतुं शक्यते ।। 
(ड) शङ्करानन्दकृता - ईशावास्यदीपिका 
मुद्रितश्चायं ग्रन्थ आनन्दाश्रममुद्रणालये । एनमधिकृत्य विस्तरेण अद्वैताचार्यप्रकरणे पूर्वञ्चोपपादितम् । 
(ढ) सदानन्दकृतः - ईशावास्यचिन्तामणिः
अमुद्रितोऽयं ग्रन्थ उज्जैनूसूच्यां (1947 ) दृश्यते । अस्य कर्ता सदानन्द इति ज्ञायते । कोऽयं सदानन्दः? किं काश्मीरी सदानन्दः ? उत सदानन्दसरस्वती ? आहोस्वित् सदानन्दव्यासवर इत्यत्र न किमपि प्रमाणमुपलभामहे ।। 
%ईशावस्योपनिषद् 
ऐतरेयोपनिषत् - 
ऋग्ब्राह्मणान्तर्गतेयमुपनिषत् । अध्यायत्रिकरूपाया अस्या उपनिषदः खण्डषट्कं विद्यत इति कृत्वोपनिषदियम् आत्मषट्क मित्यपि प्रसिद्धा । अस्यामुपनिषदि त्रिष्वध्यायेषु सर्वशरीरस्य ब्रह्मणः जगत्सृष्ट्यादिवर्णनम्, बद्धस्य जीवात्मनो वैराग्योदयाय गर्भप्रवेशदुःखानुभवादिनिरूपणं, निश्श्रेयसावाप्त्यर्थं परमात्मोपासनञ्च प्रतिपाद्यते । अस्या रचनाकाल 600 - 500 B.C इति विमर्शकाः । मुद्रिता चेयमुपनिषदानन्दाश्रममुद्रणालये । अस्या व्याख्याः - 
१. शङ्कराचार्यकृतम् - ऐतरेयमाष्यम् 
मुद्रितश्चायं ग्रन्थ आनन्दाश्रममुद्रणालये वाणीविलासमुद्रणालये च । 
(क) अभिनवनारायणेन्द्रकृता - भाष्यव्याख्या 
अमुद्रितोऽयं ग्रन्थ मद्रासराजकीयहस्तलिखितपुस्तकालये (R. 1475 MGOML) अडयारपुस्तकालये च लभ्यते । 
अस्याः कर्ता कैवल्येन्द्रसरस्वतीप्रशिष्यः ज्ञानेन्द्रसरस्वतीशिष्य अग्निहोत्रभट्टसतीर्थ्यः परमशिवेन्द्रसरस्वतीगुरुः षोडशसप्तदशशतकमध्यवासी 1550 - 1650 A.D अभिनवनारायणेन्द्रसरस्वतीति विज्ञायते । अदसीया अन्येऽपि ग्रन्थास्तत्तत्प्रकरणे वर्णिताः ।।
(ख) आनन्दगिरिकृता - भाष्यटिप्पणी 
मुद्रितश्चांय ग्रन्थ आनन्दाश्रममुद्रणालये । ग्रन्थेऽस्मित् ( Page 28) विद्यारण्यकृता दीपिका निर्दिष्टा । विद्यारण्यश्चानन्दगिरेरर्वाचीन इति प्रसिद्धिः । 
तस्मात् न प्रसिद्धोऽयमानन्दगिरिरस्य कर्ता । परन्तु यः कोऽप्यन्य एव स्यात् । मुद्रितग्रन्थे शुद्धानन्दादिगुरुवन्दनापि न दृश्यते ।। 
(ग) ज्ञानामृतयतिकृता - भाष्यटिप्पणी 
अमुद्रितोऽयं ग्रन्थः मद्रासराजकीयहस्तलिखितपुस्तकालये (D. 332 MGOML) लभ्यते । ग्रन्थादस्मात् सायणाचार्यसामयिकोऽयमिति ज्ञायते । उत्तमामृत - आनन्दाख्ययोश्शिष्यः, सायणाचार्यसामयिकः नैष्कर्म्यसिद्धिव्याख्याविद्यासुरभिकर्ता चतुर्दशशतकीय (1350 A.D.) ज्ञानामृतयतिरिति ज्ञायते ।। नृसिम्हाश्रमकृता भाष्यव्याख्या चास्तीती श्रूयते । सीतानाथतर्कभूषणेन आधुनिकेन कृता 'शङ्करकृपा' नाम्नी व्याख्या च मुद्रिता ।  
(घ) उपनिषद्ब्रह्मेन्द्रकृतम् - भाष्यविवरणम् ।
अमुद्रितोऽयं ग्रन्थ अडयारपुस्तकालये (36 F. 14 ग्र 79 A.L) लभ्यते अस्य कर्ता उपनिषद्ब्रह्मेन्द्रः वासुदेवेन्द्रशिष्यः अष्टादशशतकीय इति प्रतिपादि तमन्यत्र ।। 
२. भास्करानन्दकृता - ऐतरेयदीपिका (व्याख्या)
मुद्रितश्चायं ग्रन्थः भारतीजीवनमुद्रणालये वाराणस्याम् । अस्याः कर्ता एकोनविंशतिशतकीयः भास्करानन्दः पूर्वमुपपादितः ।
३. विद्यातीर्थकृता - ऐतरेयदीपिका 
अमुद्रितोऽयं ग्रन्थः वेङ्कटेश्वरपुस्तकालये तत्सूच्याञ्च लक्ष्यते । अस्याः कर्ता रुद्रप्रश्नभाष्यकर्ता नृसिम्हतीर्थशिष्यः भारतीतीर्थविद्यारण्यगुरुः, कृष्णानन्दभारती - ब्रह्मानन्दभारती-विद्यारण्यानां प्रगुरुः, विद्याशङ्करश्शङ्करानन्द इत्यपरनामा त्रयोदशशतकीयः ( 1228 - 1333 A.D) विद्यातीर्थ इति ज्ञायते । विद्यारण्यकृताया दीपिकाया दीपिकाया इयं भिद्यते न वा इति न निश्चेतुं शक्यते । 
४. विद्यारण्यकृता - ऐतरेयदीपिका 
मुद्रितश्चायं ग्रन्थ आनन्दाश्रममुद्रणालये । अस्य कर्ता विद्यारण्यः श्रीकण्ठाचार्यशङ्करानन्दभारतीतीर्थानां शिष्यः, विद्यातीर्थप्रशिष्यः, रामकृष्ण - कृष्णानन्दभारतीब्रह्मानन्दभारतीगुरुः पञ्चदश्यादिग्रन्थप्रणेता चतुर्दशशतकीयः (1300 - 1400 A.D) इति ज्ञायते । ग्रन्थस्यास्य भाष्यमिति नाम सरस्वतीमहालयस्थे पुस्तके (1451 TSML) दृश्यते । 
५. शङ्करानन्दकृता - ऐतरेयदीपिका । ग्रन्थोऽयं शृङ्गगिरिसृच्यां दृश्यते ।। 
%एेतरेयोपनिषद् 
काठकोपनिषत -
नचिकेतोमृत्युंसवादरूपायामस्यामुपनिषदि प्रत्येकं वल्लित्रितययुते अध्यायद्वितये सर्वमेधाख्ये यज्ञे दत्तसर्वस्वदक्षिणस्य वास्रवसनाम्नो महर्षेः पुत्रेण नचिकेतोनाम्ना निजनिर्बन्धादेव पित्रा मृत्यवे प्रदत्तेन मृत्युलोकमधिगम्य त्रिस्रो रात्रीरुपोषितवता मृत्युप्रसादादेव निजपितुश्शान्तचित्तत्वं, नचिकेता इति स्वनामधेयोपेतस्याग्नेः ब्रह्मविद्याङ्गतया विज्ञानं सविस्तरं मृत्युनैवोपदिष्टं प्रत्यगात्मपरमात्मनोस्स्वरूपज्ञानं च इत्येतद्वरत्रितयं पुत्रपौत्रचिरायुस्सम्पत्यादिकं प्रत्याचक्षाणेन समधिगतमिति प्रतिपाद्यते । अस्य उपनिषदो रचनाकालः 500 - 400 BC इति विमर्शकः । मुद्रिता चेयमुपनिषदानन्दाश्रममुद्रणालये । अस्या व्याख्याः - 
1. शङ्कराचार्यकृतम् - कठोपनिषद्भाष्यम् 
सव्याख्योऽयं ग्रन्थः आनन्दाश्रममुद्रणालये मुद्रितः ।
(क) अच्युतकृष्णानन्दकृता - भाष्यटीका 
अमुद्रितेयं काठकशाङ्करभाष्यटीका महीशूरहस्तलिखितपुस्तकालये (1278 ग्र 22 प My GML) लभ्यते ।  
अस्याः कर्ता रामानन्दप्रशिष्यस्स्वयम्प्रकाशशिष्य अद्वैतानन्दसरस्वत्याश्च शिष्यः तैत्तरीयव्याख्यावनमालाकारः सप्तदशाष्टादशशतकीय (1650 - 1750 A.D,) अच्युतकृष्णानन्द इति ज्ञायते ।
(ख) आनन्दगिरिकृता - भाष्यटीका । मुद्रितश्चायं ग्रन्थ आनन्दाश्रममुद्रणालये । 
(ग) गोपालबालयतिकृतम् - कठवल्लीभाष्यविवरणम् । मुद्रिता चेयं व्याख्या आनन्दाश्रममुद्रणालये ।।
अस्याः कर्ता बालगोपालयतिरित्यपरनामा जगन्नाथाश्रमिशिष्यः, नृसिम्हाश्रमिसतीर्थ्यः, स्वयम्प्रकाशगुरुः, मनीषापञ्चकव्याख्यामधुमञ्जरीकारः षोडशशतकीयः 1500 - 1600 A.D. गोपालबालयतिरिति ज्ञायते ।।
(घ) शिवानन्दयतिकृता - कठभाष्यटिप्पणी 
अमुद्रितोऽयं ग्रन्थ मद्रासराजकीयहस्तलिखितपुस्तकालये ( R 3882 MGOML) लभ्यते । अस्य कर्ता रामनाथविदुष आचार्यः सप्तदश - अष्टादशशतकमध्यवर्ती 1650 - 1751 A.D. शिवानन्द इति पूर्वमुपपादितम् ।
(ङ) श्रीधरशास्रिपाठककृता - भाष्यव्याख्या - बालबोधिनी 
मुद्रिताचेयं व्याख्या पूनानगरे । अस्याः कर्ता त्र्यम्बकशास्त्रिपुत्रः डेक्कान संस्कृतकलाशालाप्रधानाध्यापकः विंशतिशतकीयः 1850 - 1920 A.D. श्रीधरशास्त्रीति ज्ञायते ।।
2. अभिनवनारायणेन्द्रकृता - कठोपनिषद्व्याख्या 
अमुद्रितोऽयं ग्रन्थ औधसूच्यां दृश्यते । अस्य कर्ता कैवल्येन्द्र ज्ञानेन्द्र योशिशष्य अग्निहोत्रभट्टसतीर्थ्यः षोडशशतकीयः 1550 - 1650 A.D. अभिनवनारायणेन्द्र इति ज्ञायते । 
3. उपनिषद्ब्रह्मेन्द्रकृता - कठवल्लीव्याख्या 
मुद्रिताचेय व्याख्या अडयारपुस्तकालये । अस्य कर्तारं उपनिषद्ब्रङ्मेन्द्रमधिकृत्य पूर्वमेवोपपादितम् ।। 
4. दिगम्बरानुचरकृता - कठवल्लीव्याख्या - अर्थप्रकाशिका 
मुद्रिता चेयंव्याख्या आनन्दाश्र मुदणालये । अस्याः कर्ता एकनाथसामयिकः महाराष्ट्रीयसाधुर्दिगम्बरानुचर इत्येव ज्ञायते । 
5. नारायणगजपतिराजकृता - कठोपनिषद्व्याख्या - द्विमतप्रकाशिका
अमुद्रितोऽयं ग्रन्थः शाङ्करभाष्यानुसारिणी व्याख्यां द्वयोर्मतयोरन्यतमाञ्च स्वीकरोति। ग्रन्थोऽयं बरोडासूच्यां (10058 BRD) दृश्यते । अस्य कर्ता नाराराववेङ्काम्बयोः पुत्रः जमदग्निगोत्रजः क्षत्रियः नारायणगजपतिराय इति परं ज्ञायते ।।
6. नारायणाश्रमिकृता - कठवल्लीदीपिका । मुद्रितश्चायं ग्रन्थ आनन्दाश्रममुद्रणालये ।  
7. भास्करानन्दसरस्वतीकृता - कठवल्लीव्याख्या 
मुद्रितश्चायं ग्रन्थः भारतीजीवनमुद्रणाले वाराणस्याम् । भास्करानन्दश्च पूर्वमुपपादितः ।।
8. शङ्करानन्दकृता - कठवल्लीदीपिका 
मुद्रितश्चायं ग्रन्थः आनन्दाश्रममुद्रणालये । शङ्करानन्दश्च उपपादितपूर्वः । 
%कठोपनिषद

केनोपनिषत् - 
खण्डचतुष्टयेन विभक्तायामस्यां केनोपनिषदि प्रथमखण्डे मनःप्रभृतिकरणप्रवृत्तेश्चेतनायत्तत्वेन तत्प्रवर्तकचेतनविशेषस्य परब्रह्मत्वमभिधीयते । द्वितीयखण्डे ब्रह्मणोऽपरिच्छिन्नत्वात् तस्य कार्त्स्न्येन विज्ञातुमशक्यत्वेन योग्यैकदेशज्ञानादेव इष्टावाप्तिरित्यभिधीयते । तृतीयखण्डे - परं ब्रह्म कार्त्स्न्येन ज्ञातुमशक्यमित्यस्मिन्नर्थे ब्रह्मानुप्रवेशलब्धासुरविजयानां स्वेषामेव जेतृत्वाभिमानवतां देवानां गर्वापनोदनमुखेन स्वमाहात्म्यं यक्षरूपधरेण ब्रह्मणा प्रकाशितमित्याख्यायिकाभिधीयते । चतुर्थखण्डे यक्षरूपधरे ब्रह्मणि देवानामुपदेशाय हैमवतों देवीमाकाशेऽवस्थाप्या अन्तर्हिते सति तया देव्या इन्द्रः, तस्माच्चान्येऽपि आघिदैवताध्यात्मात्मरूपां उपास्याब्रह्मणो रूपमजानन् । अस्या रचनाकाल (500 -400 B.C) इति विमर्शशीलाः । मुद्रिता चेयमुपनिषत् सव्याख्याऽनन्दाश्रममुद्रणालये ।। 
अस्या व्याख्या :-
 (1) शङ्कराचार्यकृतम् -
 1. पदभाष्यम् 
 2. वाक्यभाष्यम् 
 मुद्रितञ्चेदं भाष्यद्वयमानन्दाश्रममुद्रणालये । किमेतत् भाष्यद्वयमपि शङ्कराचार्यकृतम् ? उत नेति विमृश्यते - 
 भाष्यद्वयस्यापि एककर्तृकत्वे अर्थैक्यं अभिप्रायसाम्यञ्चापेक्ष्यते । तत्तु नात्रोपलभ्यते । वैलक्षण्यस्य दर्शनात् । वैलक्षण्यानि - 
(क) पदभाष्ये "नाहं मन्ये सुवेदेति" 2-10  मन्त्रे अहमिति पाठं गृहीत्वा व्याख्यातम् । वाक्याभाष्ये तु अवधारणार्थकं अह इति पाठं गृहीत्वा व्याख्यातम् । यद्युभयोर्भाष्ययोरेकः कर्ता स्यात्तर्हि मया पदभाष्येऽयं पाठो व्याख्यात इत्याद्युच्येत । न तथोक्तमिति कर्तुर्वैलक्षण्यं स्पष्टम् ।
(ख) पदभाष्ये 1-1 मन्त्रे " प्रेषित " इत्यस्यार्थः प्रेरितमिति वर्णितः। वाक्यभाष्ये तु तस्यैव " प्रेषितमिव " इति पूर्वविलक्षणोपमार्थ उपवर्णितः । 
(ग) पदभाष्ये " श्रोत्रस्य श्रोत्रम् " 1-2 मन्त्रे श्रोत्रमाद्याः प्रथमास्सन्तीति साधितम् । वाक्यभाष्ये तु तत्रैव प्रथमास्सन्तीति साधितम् । वाक्यभाष्ये तु तत्रैव प्रथमाद्वितीयेति विभक्तिद्वयं साध्यते ।
(घ) 1-3 मन्त्रे, 2-9 मन्त्रे, च अवतरणिकाभेदस्सन्दृश्यते । 1-4 मन्त्रे पदवाक्यभाष्ययोरक्षरार्थसाम्येऽपि अभिप्रायभेदो दृश्यते । "भूतेषु भूतेषु विचित्य" 2-13 मन्त्रे विचित्येत्यस्य पदवाक्यभाष्ययोर्भिन्नार्थता च दृश्यते ।   
एवमादिभिर्वैलक्षण्यदर्शनैः पदभाष्यवाक्यभाष्ययोर्निर्मातारौ भिन्नाविति प्रतीयते । एवं भिन्नकर्तृकत्वे सिद्धे कतरत् पूर्वम् किं पदभाष्यम् ? उत वाक्यभाष्यम् ? इति विमर्शः कर्तव्यः।
केनोपनिषदां पदभाष्यं भगवत्पूज्यपादानां आदिशङ्कराणां कृतिः। वाक्यभाष्यन्तु तदनुयायिनां श्रीशङ्करपीठमधिष्ठितानां केषाञ्चिदाचार्याणामित्यस्माकं मतिः। तत्रेमानि कारणानि - 
पदभाष्यापेक्षया वाक्यभाष्ये श्रुत्यन्तराणां प्रमाणार्थनिर्देशोऽल्पीयान् । यत्र तत्र पूर्वपक्षखण्डनेश्वरसिध्यादिर्लेखो भूयान् दृश्यते । आद्याचायों हि उन्मूलमप्रासङ्गिकं उपनिषद्भाष्ये न कदापि लिखति । उपनिषदां बहूनि वाक्यानि आद्याचार्याणां ग्रन्थे सम्भवन्ति। तस्मात् बाक्यभाष्यं आद्याचार्येतरकृतमिति भाति । 
"प्रतिबोधविदितम् " 2 - 22 इति मन्त्रे वाक्यभाष्ये प्रतिबोधमित्यस्यार्थत्रितयं वर्णितम् । तत्र द्वितीयव्याख्या सद्योमुक्तिं प्रत्याययति । सद्योमुक्तिश्च न शास्त्रीया । एतादृशमशास्त्रीयं मतमाद्याचार्याः कदापि न प्रतिपादयन्तीति वाक्यभाष्यमाद्याचार्ये तरकृतम् । पदभाष्ये यथाक्रमं शब्दशो विवरणम् । वाक्यभाष्ये तु सुगमशब्दविशिष्टो मन्त्रो न व्याख्यायते । आद्याचार्याणां सर्वत्रेयं परिपाठी यत् ते यथाक्रमं सर्वान् शब्दान्‌ निर्दिशन्ति, विशिष्टशब्दान् व्याख्यान्ति च ।
वाक्यभाष्ये चतुर्थखण्डे षष्ठमन्त्रे "सेवन्ते स्म" इति पदं निरर्थकं प्रयुक्तम् । एवं निरर्थकशब्दप्रयोग आद्याचार्याणां ग्रन्थे न सञ्जाघटीति इदं तत्कृत न सम्भवति । 
अत्र केचन साम्प्रदायिका इत्थं समादधते यत् - यथाऽन्यासामुपनिषदां मन्त्रान् विषयवाक्यत्वेन गृहीत्वा तेषां व्याख्यानं आचार्यैः सूत्रभाष्ये कृतं तथा केनोपनिषत्स्थमन्त्राणां न कृतम् । तत्रत्यानां मन्त्राणां विषयवाक्यत्वाभावात् । तेन पदभाष्यं कृत्वाऽपरितुष्यन्तो भगवत्पादा पुनर्वाक्यभाष्यमकुर्वन्निति । अतएव "पदशो व्याख्यायापि न तुतोष भगवान् भष्यकार" इत्यानन्दज्ञानोक्तिस्सङ्गच्छत इति च । परं तन्न विचारसहम् । यतो यथा केनोपनिषत्स्थमन्त्राः न विषयवाक्यत्वेन सूत्रभाष्ये व्याख्याता तथा ईशावास्यमन्त्रा अपि विषयवाक्यत्वेन ब्रह्मसूत्रे न व्याख्याताः । ततश्च ईशावास्योषनिषदि भगवत्पूज्यपादैः भाष्यान्तरं कृतं स्यात्, तत्तु नोपलभ्यते । तेन पूर्वोक्तं समाधानं स्थवीयः। किञ्चैतत्समाधानं पूवोंक्तवैलक्षण्यविवरणं न सहत इति तु अन्यदेतत् । तस्मात् पदभाष्यं प्राचीनं, वाक्यभाष्यं पश्चात्तनमिति सिध्यति । 
अयं वाक्यभाष्यकृत् शङ्करानन्दात्प्राक्तनः। शङ्करानन्दो विद्यारण्यगुरुः । तेन शङ्करानन्दकालः त्रयोदशशतकं भवितुमर्हति । शङ्करानन्देन च "व्याकरिष्ये पदाध्वना" इति तत्कृतदीपिकायां कथ्यते । आनन्दगिरिणा तु ऐतरेयोपनिषद् भाष्यव्याख्यायां विद्यारण्यीयैतरेयोपनिषद्दीपिकास्थवाक्यस्य निर्देशात् शङ्करानन्दादर्वाचीनेन भाव्यम् । तस्मात् भाष्यद्वये व्याख्याकरणं सुसङ्गतमेव । 'अपरितुष्य' न्निति अवतरणिकाप्रदानमपि तन्मत्यनुसारं न विरुध्यते च । 
तस्मात् - शाङ्करपीठाधिष्ठितेनान्येन केनचित् एकं भाष्यं केनोपनिषदि कृतम् । तस्य वाक्यभाष्यमिति सज्ञा कृता । परं कालवशात् अज्ञात्वा तत्कर्तारं भाष्याध्यायिभिरस्य वाक्यसंज्ञां दृष्ट्वा पूर्वभाष्यस्य पदसज्ञा कृता भवेत्  इति निश्चीयते ।।
भाष्यव्याख्याः - 
(क) अभिनवनारायणेन्द्रकृता - भाष्यटीका 
अमुद्रितोऽयं ग्रन्थ औंधसूच्यां XXI 26 दृश्यते । अभिनवनारायणेन्द्रः पूर्वमुपपादितः । 
(ख) आनन्दगिरिकृता - माष्यटिप्पणी । मुद्रितश्चायं ग्रन्थ आनन्दाश्रममुद्रणालये । उपपादितपूर्वश्चायमानन्दगिरिः ।
(ग) शिवानन्दयतिकृता - भाष्यटिप्पणी 
अमुद्रितोऽयं ग्रन्थः मद्रासराजकीयहस्तलिखितग्रन्थागारे D 392 लभ्यते । सरस्वतीमहालयसूच्याञ्च दृश्यते । अस्य कर्ता शिवानन्दयतिः पूर्वमुपपादितः ।। 
(घ) श्रीधरशास्त्रिकृता - भाष्यव्याख्या बालबोधिनी 
मुद्रितश्चायं ग्रन्थः पूनानगरे । अस्य कर्ता श्रीधरशात्री डेक्कानसंस्कृतकलाशालाप्रधानाध्यापकः त्र्यम्बकशास्त्रिपुत्रः विंशतिशतकीय इति पूर्वमुपपादितम् ।। आधुनिकेन सच्चिदानन्देन्द्रसरस्वत्या कृता काचन भाष्यव्याख्या मुद्रिता च विद्यते ।
(2) उननिषद्व्रह्मेन्द्रकृतम् - केनोपनिषद्विवरणम् । मुद्रितश्चायं ग्रन्थ अदृयार पुस्तकालये । अस्य कर्ता उपनिषद्बह्मेन्द्र पूर्वमुपपादितः ।
3.  कृष्णलीलाशुककृता - केनोपनिषद्व्याख्या, शङ्करहृदयङ्गमा मुद्रितश्चायं ग्रन्थः मद्रासराजकीयहस्तलिखितपुस्तकालयसंस्कृतग्रन्थमालायाम् । ग्रन्थेऽस्मिन् क्षीरस्वामी बुद्धाः शङ्कराचार्याश्च निर्दिष्टाः । 
अस्याःकर्ता कृष्णलीलाशुकः । अयमेव बिल्वमङ्गलाचार्य इति प्रसिद्धः। यद्येवमस्य पिता दामोदरः । माता नीलीनाम्नी । ईशानदेवशिष्योऽयं कृष्णलीलाशुकः केरलदेशवासी कोच्चिनूसाम्राज्यवर्तिनि तिरुच्चूरन्तर्गते तेक्कमठे लब्धवासः 
1220 - 1300 A.D. काले उवासेति प्राच्यभाषामहासम्मेलनपत्रिकायाः नवमे भागे दृश्यते ।
यद्ययं कृष्णलीलाशुकः सरस्वतीकण्ठाभरणव्याख्यापुरुषकार - कृष्णलीलामृतादिकर्ता स्यात् तर्हि अस्य कालः द्वादशत्रयोदशशतकोत्तरार्धपूर्वार्धावधिक इति निर्णेतुं शक्यते । यतः पुरुषकारे हेमचन्द्र उद्धृतः । हेमचन्द्रकालस्तु 1166 - 1220 A.D. । पुरुषकारस्य पंक्तयः देवराजकृतायां निघण्डुटीकायां उद्धृताः । देवराजकालस्तु 1293 - 1343 A.D. । एवञ्च तयोर्मध्यपाती  पुरुषकारकर्तायं कृष्णलीलाशुकः 1168 - 1293 A.D. काले आसीदिति निर्णेतुं शक्यते ।
सीतारामजयराम जोशी तु कृष्णलीलाशुकं एकादशशतकीयं 1100 A.D. स्वीये संक्षिप्तसाहित्येतिहासे वदति । 
श्रीचिह्ननामा कश्चन ग्रन्थः कृष्णलीलाशुकेन रचितः। तत्रायं श्लोकस्सन्दृश्यते । "श्रीपद्मपादमुनिवर्यविनेयवर्गश्रीभूषणं मुनिरसौ कविसार्वभौमः" इति । पद्यादस्मात् पद्मपादशिष्यः कृष्णलीलाशुक इति वक्तुं शक्यते । केनोपनिषद्व्याख्यायां (Page 10)  दृश्यमाना "ब्रह्म भूयं गते पूर्वे शंकरे कृत्स्नवेदिनि । पूर्वे च तादृशे" इति श्लोकीया "पूर्वे च तादृशे " इति पंक्तिरपि पद्मपादाचार्यं निर्दिशतीव लक्ष्यते । एवञ्च वेदान्ते पद्मपादाचार्यशिष्योऽयं कृष्णलीलाशुकः नवमशतकीय इति सिध्यति । अत एव "इण्डियन् हिस्टारिकल रेव्यू" नामकपत्रिकायां सप्तमे भागे (I.H.R VII Page 334) कृष्णलीलाशुकः नवमशतकीय इति प्रतिपादितम् । केचित्तु वङ्गदेशीयं वदन्ति। 
4.  नारायणाश्रमिकृता - केनोपनिषद्दीपिका 
मुद्रितश्चायं ग्रन्थ आनन्दाश्रममुद्रणालये । बालकृष्णानन्दसरस्वत्या कृता काचन व्याख्या अमुद्रिता लन्दननगरकार्यालयपुस्तकालये विद्यते ।
5. भास्करानन्दकृता - केनोपनिषद्व्याख्या । मुद्रितश्चायं ग्रन्थः भारतजीवनमुद्रणालये वाराणस्याम् । 
6. शङ्करानन्दकृता - केनोपनिषद्दीपिका । मुद्रितश्चायमपि ग्रन्थ आनन्दाश्रममुद्रणालये ।। 
% केनोपनिषद् 

कैवल्योपनिषत् - 
आश्वलायनपरमेष्ठिप्रश्नोत्तररूपायामस्यामुपनिषदि ब्रह्मज्ञानेनाविद्याविनाशः, सन्यासिनां प्रशंसना, सगुणतिर्गुणोपासनाक्रमश्च प्रतिपाद्यन्ते। मुद्रिता चेयमुपनिषदानन्दाश्रममुद्रणालये । अस्याः व्याख्याः - 
1. उपनिषद्ब्रह्मेन्द्रकृतम् - कैवल्योपनिषद्विवरणम् । मुद्रितोऽयं ग्रन्थ अडयार पुस्तकालये । 
2. नारायणाश्रमिकृता - कैवल्योपनिषद्दीपिका 
श्रुतिमात्रोपजीविना नारायणाश्रमिणा कृतेयं दीपिका आनन्दाश्रममुद्रणालये मुद्रिता । 
3. शङ्करानन्दकृता - कैवल्योपनिषद्दीपिका 
शङ्करानन्दकृतत्वेन मुद्रितेयं दीपिका आनन्दाश्रममुद्रणालये । परन्तु बरोडा-अडयार-नासिकादिपुस्तकालयस्थेषु आदर्शपुस्तकेषु विद्यारण्यकृता इति ग्रन्थान्तपुष्पिकाया ज्ञायते ।। अस्या तैलनाम्नी व्याख्या (Mad: uni RAS. 12 C.) विद्यते ।
4. सदाशिवब्रह्मेन्द्रसरस्वतीकृता - कैवल्योपनिषद्दीपिका 
अमुद्रितोऽयं ग्रन्थः मद्रासराजकीयहस्तलिखितपुस्तकालये लभ्यते ( R 1492d MGOML) ।
अस्याः कर्ता परमशिवेन्द्रसरस्वतीशिष्यः, ब्रह्मतत्त्वप्रकाशिका - आत्मविद्याविलासादिकारः, नल्लादीक्षितगुरुरष्टादशशतकीयः सदाशिवब्रह्मेन्द्रसरस्वतीति सूत्रवृत्तिप्रकरणे प्रकरणग्रन्थप्रस्तावे चोपपादितम् ।। 
कौषीतक्युपनिषत् - 
ऋक्शाखीयायामस्यामुपनिषदि फलसङ्गाद्यभिसन्धियुतानां धूमादिमार्गेण चन्द्रमसं प्राप्य पुनरावृत्तानां संसृतौ भ्रमणम् , विद्याकर्मविहीनानां कीटपतङ्गाद्यात्मना क्लेशानुभूतिः, गुर्वनुग्रहाधिगतविद्यानां ब्रह्मचारिगृहिप्रभृतीनां अर्चिरादिमार्गेण उत्तमां गर्ति प्रपन्नानां ब्रह्मणस्सारूप्यर्माधगतानां ब्रह्मालङ्कारपरिमण्डितानां परब्रह्मानुभवप्रकारः, इन्द्रप्रतर्दनाख्यायिकया स्वदेहभोगस्य प्रारब्धपरिक्षयैकपर्यवसायित्वकथनम् ,प्राणोपासनाभिधानम् , अजातशत्र्त्वाख्यायिकया आदित्य - चन्द्र - अग्निविद्युत् - स्तनयित्वादिषु उपासनासु विसंवादप्रदर्शनपूर्वकं परमात्मस्वसूपतद्विज्ञानादिप्रकाशनमित्येतत्सर्वं प्रतिपाद्यते । अस्या रचनाकालः 200 - 100 BC  इति विमर्शकाः। मुद्रिता चेयमुपनिषत् सव्याख्याऽनन्द्राश्रममुद्रणालये ।। अस्या व्याख्या :-
1. उपनिषद्ब्रह्मेन्द्रकृतम् - कौषीतकिविवरणम् । अडयारपुस्तकालये मुद्रितोऽयं ग्रन्थः। 
2. नागरनारायणकृता - ज्ञानमाला । रामेन्द्रसरस्वतीशिष्येण कृतेेयं व्याख्या अमुद्रिता बरोडापुस्तकालये तत्सूच्यां (3827 BRD) च दृश्यते । 
3. नारायणाश्रमिकृता - दीपिका 
4. शङ्करानन्दकृता - दीपिका । मुद्रिते चेमे दीपिके आनन्दाश्रममुद्रणालये ।।
क्षुरिकोपनिषत् - 
यजुश्शाखीयायामस्यामुपनिषदि स्वयम्मूपदिष्टंं योगाभ्यसनक्रममनुष्ठातुर्जातवैराग्यस्य योगिनः प्राणायामपूर्वकनाडीभेदनद्वारा उत्तमलोकावाप्तिरभिधीयते ।। मुद्रिता चेयमुपनिषदुपनिषदां समुच्चये आनन्दाश्रममुद्रणालये ।। अस्या व्याख्या :-
1. उपनिषद्ब्रह्मेन्द्रकृतम् - विवरणम् । अडयार पुस्तकालये मुद्रितम् ।
2. नारायणाश्रमिकृता - दीपिका । आनन्दाश्रममुद्रणालये मुद्रिता ।
3. शङ्करानन्दकृता - दीपिका । आनन्दाश्रममुद्रणालये मुद्रिता । 
गोपालतापिन्युपनिषत् - 
इयमुपनिषदाथर्वणी पूर्वॉत्तरभागद्वयेन विभक्ता दृश्यते । तत्र पूर्वभागे - मुनिब्राह्मणप्रश्नोत्तररूपेण मृत्योरपि भयजनकः परमो देवश्श्रीकृष्ण इति प्रतिपाद्य तदुपासनोपयोगिनस्संसारोत्तारकाः तद्देवताकाः केचिन्मन्त्रा आम्नाताः उत्तरभागे श्रीकृष्णस्य नित्यब्रह्मचारित्वं दुर्वाससो मुनेर्दूर्वाशित्वं प्रतिपाद्य दुर्वाससा गान्धर्वी गाम्न्याः उपदेशव्याजेन श्रीकृष्णस्य परब्रह्मत्ववर्णनपूर्वकं तन्माहात्म्यममभिधाय सप्तपुरीमध्यगतब्रह्मगोपालपुरीवर्णनपूर्वकं श्रीभगवन्माहात्म्यं  तदुपासनाप्रकारश्च प्रतिपाद्यन्ते । अस्या व्याख्याः - 
1. उपनिषद्ब्रहेन्द्रकृतम् - विवरणम् । अडयारपुस्तकालये मुद्रितम् । 
2. नारायणाश्रमिकृता - दीपिका (ASS 29) 
3. विश्वेश्वरकृता - दीपिका 
स्वयम्प्रकाशशिष्योऽयं विश्वेश्वर इति परं ज्ञायते । अमुद्रितोऽयं ग्रन्थस्सरस्वतीमहालये तत्सृच्यां (1474 DCTSML), बरोडापुस्तकालये तत्सूच्यां (234 DCBRD) च दृश्यते ।।
छान्दोग्योपनिषत् - 
सामशाखीयायामस्यां छान्दोग्योपनिषदि प्रथमप्रपाठके त्रयोपदशभिः खण्डैः उद्गीथस्यानेकधा उपासनानि. स्तोभाक्षरोपासनम् , तत्फलम् , सामादिज्ञानं तत्फलञ्च प्रतिपाद्यते । द्वितीयप्रपाठके चतुर्विंशतिखण्डैः - साम्नः पञ्चधा सप्तधा चोपासनम् , तत्फलञ्च कथ्यते । तृतीयप्रपाठके एकोनविंशतिखण्डैः आदित्योपासनम् , मधुविद्या, गायत्रीविद्या, शाण्डिल्यविद्या, तासां फलानि चाभिधीयन्ते । चतुर्थप्रपाठके सप्तदशखण्डैः विद्यादानग्रहणविचारः उपकोसलविचारः , ब्रह्मविदो गतिविचारश्च कृतः । पञ्चमप्रपाठके त्रयोविंशतिख्णडैः - प्राणस्य मुख्यत्वविचारः, पञ्चाग्निविद्या, सगुणब्रह्मविद्याफलं आख्यायिकामुखेन वैराग्योदयाय संसारगतिः, वैश्वानरविद्या चाभिधीयते । षष्ठप्रपाठके षो़डशभिः खण्डैराख्यायिकामुखेन ब्रह्मविद्या विचारिता । अनेकधा आत्मतत्वविचार उपदेशश्च कृतः। सप्तमप्रपाठके पञ्चदशभिः खण्डैः भूमविद्या प्रपञ्च्यते । अष्टमप्रपाठके पञ्चदशखण्डैर्दहरविद्या तत्फलं, प्रत्यगात्मविद्या तत्फलं च प्रकाश्यते । अस्या रचनाकालः 700 BC इति विमर्शकाः । सव्याख्येयमुपनिषन्मुद्रिता आनन्दाश्रममुद्रणालये । अस्या व्याख्या :- 
1. द्रविडाचार्यकृतम् - छान्दोग्यभाष्यम् 
2. ब्रह्मनन्दिकृतम् - छान्दोग्यवाक्यम् 
इमौ द्वावपि ग्रन्थौ नोपलभ्येते, तथापि तत्र विभिन्नग्रन्थेषु उद्धरणात् तयोर्ग्रन्थयोस्सत्ता ऊह्यते । एतावधिकृत्य विस्तरेण प्रतिपादितं अद्वैताचार्यप्रकरणे । 
3. शङ्कराचार्यकृतम् - छान्दोग्यभाष्यम् (ASS 14) 
(क) अभिनवनारायणेन्द्रकृता - भाष्यदीका । अमुद्रितोऽयं ग्रन्थ मद्रासराजकीयहस्तलिखितपुस्तकालये (R 1662 MG OML) अडयारपुस्तकालये च लभ्यते ।।
(ख) आनन्दगिरिकृता - भाष्यव्याख्या (ASS 14)
4. उपनिषद्ब्रह्मेन्द्रकृतम् - छान्दोग्यऋजुविवरणम् । अडयारपुस्तकालये मुद्रितोऽयं ग्रन्थः ।  
5.  त्यागराजशास्त्रिकृतः- सद्विद्याविलासः सव्याख्यः 
(राजुशास्त्रिकृत ) (रसानुभूतिरिति नामान्तरं व्याख्यायाः) 
छान्दोग्यषष्ठाध्यायसंग्रहात्मकः पद्यात्मकश्चायं ग्रन्थः भाष्यतद्व्याख्यानानुसारं विस्पष्टार्थप्रकाशकः चिदम्बरनगरे ग्रन्थाक्षरेषु मुद्रितः । 
अस्य कर्ता आधुनिकोऽयं प्रकाण्डविद्वान् न्यायेन्दुशेस्वरकर्ता मन्नार्गुडि राजुशास्रीति विख्यातः, त्यागराजपौत्रः. अप्पादीक्षितपुत्रः, भारद्वाजगोत्रजः, यज्ञस्वामिदीक्षितपिता, नारायणसरस्वती - रघुनाथशास्रि - गोपालशास्रिशिष्यः एकोनविंशतिशतकीयः (1815 - 1904 A.D.) त्यागराजदीक्षितः । अदसीया अन्ये ग्रन्था अन्यत्र प्रतिपादिताः ।। 
6. नित्यानन्दाश्रमिकृता -  मिताक्षरा 
मुद्रिता चेयं मिताक्षरानाम्नी छान्दोग्यव्याख्या बम्बर्इसंस्कृतमुद्रणालये । क्वचिदादर्शपुस्तकेषु बरो़डा - अडयारपुस्तकालयस्थेषु ग्रन्थस्यास्य "अर्थप्रकाशिका" इत्यपि नामान्तरं दृश्यते । 
अस्याः कर्ता पुरुषोत्तमाश्रमशिष्यः, नित्यानन्दाश्रमीति ज्ञायते । एनमधिकृत्य नाधिकं ज्ञातुं पार्यते । मुद्रितपुस्तकात् कालनिर्णययोग्यानि प्रमाणानि नेपलभ्यन्ते । परं तु उज्जयिनीपुस्तकालयस्थे 1373  संख्याके आदर्शपुस्तके 1746  संवत् इति दृश्यते । यदि स लेखककालस्स्यात् तर्हि सप्तदशशतकात् (1690 AD)  पूर्वतन इति सिध्यति । यदि वा स एव ग्रन्थकर्तुः कालस्स्यात्तर्हि सुवर्णे सौगन्ध्यायितम् । 
7.  शङ्करानन्दकृता - छान्दोग्यदीपिका । मुद्रिता चेयमानन्दाश्रममुद्रणालये ।। 
%छान्दोग्योपनिषद 
जाबालोपनिषत् - 
पैप्पलादिजाबालप्रश्नोत्तररूपायामस्यामुपनिषदि पशुपदार्थनिरूपणपूर्वकं पशुपतिशब्दार्थं प्रकाश्य विभूतिधारणात् ज्ञांन मोक्षसाधनं भवतीत्युक्त्वा तत्प्रसङ्गात् भस्ममहिमानमुपवर्ण्य भस्मधारणप्रकारञ्च प्रकाश्य भस्मधारणप्राशस्त्यं च प्रतिपादितम् । सामशाखीयेयमुपनिषदानन्दाश्रममुद्रणालये मुद्रिता । अस्या व्याख्या :- 
1. नारायणाश्रमिकृता - जाबालोपनिषद्दीपिका 
व्याख्यायामस्यां नारायणाश्रमस्य गुरुः "आनन्दात्मा" इति निर्दिष्टः । "आनन्दात्मानमध्यात्मगुरुं देवं नतोऽस्म्यहम् " Page 275 इति दृश्यते । 
श्रुतिमात्रोपजीविना नारायणाश्रमिणा रचितेयं दीपिका आनन्दाश्रममुद्रणालये मुद्रिता । अनेन बृहदारण्यकस्यापि दीपिका कृता इति जाबालोपनिषद्दीपिकाया Page 275  ज्ञायते । नारायणाश्रमी श्रीनाथपौत्रः भट्टरत्नाकरपुत्रः आनन्दगिरेरर्वाचीन आनन्दात्मशिष्य इति पूर्वंमपि प्रतिपादितम् । 
2. वल्लभेन्द्रसरस्वतीकृता - जाबालोपनिषद्व्याख्या मोक्षलक्ष्मीविलासः ।
असुद्रितोऽयं ग्रन्थः अडयारपुस्तकालये तत्सूच्यां (38 E 38 दे 152 AL) बरोडासूच्यां (1701 BRD) लन्दननगरस्थभारतकार्यालयपुस्तकालये तत्सूच्याञ्च ( 2433 DCIOL) दृश्यते । अस्य कर्ता वासुदेवेन्द्रशिष्यः काशीवासी वल्लभेन्द्रसरस्वती (1801 AD) कालात्प्राचीन इति ज्ञायते । 
3. अमुद्रितोऽयं ग्रन्थस्सरस्वतीमहालयसूच्यां (1484 DCTSML) दृश्यते । आनन्दाश्रमे मुद्रिता च ।  
4. श्रीनिवासशास्त्रिकृतम् - जाबालोपनिषद्भाष्यम्  
जाबालोपनिषद्दीपिकापरनामायं ग्रन्थः ब्रह्मविद्यामुद्रणालये चिदम्बरक्षेत्रे ग्रन्थलिप्यां मुद्रितः । अस्य कर्ता राजुशास्त्र्यपरनामकस्य त्यागराजशास्त्रिणः शिष्यः रामयज्वसीताम्बयोः पुत्रः कौण्डिन्यगोत्रजः द्रविडदेशवासी चोलदेशीयः एकोनविंशतिशतकवासी (1894 AD) श्रीनिवासशास्त्रीति ज्ञायते ।। 
तैत्तरीयोपनिषत् - 
तैत्तरीयोपनिषद्यस्यां संहित्याख्यायां शिक्षावल्यां अङ्गभूतोपासनोपदेशपूर्वकं परविद्या प्रस्तूयते । आनन्दवल्यां परमतत्वहितपुरुषार्थप्रतिपादनं दृश्यते । वारुण्याख्यायां भृगुवल्यां तपोवधूतकल्मषमनसो ब्रह्मप्रतिपत्तिरित्यभिधीयते । अस्याः रचनाकाल (600 - 500 BC) इति विमर्शकाः । आनन्दाश्रममुद्रणालये मुद्रिताच । अस्या व्याख्या :- 
1. शङ्कराचार्यकृतम् - तैत्तरीयभाष्यम् (ASS 12)
क. अच्युतकृष्णानन्दकृता - भाष्यव्याख्या वनमाला 
वनमालाख्या तैत्तरीयशाङ्करभाष्यव्याख्याऽनन्दाश्रमे वाणीविलासमुद्रणालये च मुद्रिता । अस्याः कर्ता अद्वैतानन्दस्वयम्प्रकाशसरस्वतीशिष्यः रामानन्दप्रशिष्यः सिद्धान्तलेशसंग्रहव्याख्याकृष्णालङ्कारकर्ता सप्तदशशतकापरार्धवासी (1650 - 1750 AD) अच्युतकृष्णानन्दतीर्थः ।
(ख) अभिनवनारायणेन्द्रकृता - भाष्यटिप्पणी । अमुद्रितोऽयं ग्रन्थः महीशरपुस्तकालये लभ्यते । 
(ग) आनन्दगिरिकृता - भाष्यटिप्पणी (ASS 12) ज्ञानामृतयतिकृतं व्याख्यानं आनन्दगिरिव्याख्यायाः अस्तीति ज्ञायते ।
(घ) सुरेश्वराचार्यकृतम् - तैत्तरीयभाष्यवार्तिकम् 
शाङ्करतैत्तरीयकभाष्यस्य पद्यमयी वार्तिकाख्या व्याख्या आनन्दाश्रममुद्रणालये (ASS 13) वाराणस्याञ्च मुद्रिता । अस्याःकर्ता शङ्कराचार्यशिष्येषु अन्यतमः पद्मपादादिसतीर्थ्यः अष्टमशतकवासी (800 AD) सुरेश्वराचार्यः । 
पूर्वाश्रमे शोणानदीतिरवासी पञ्चगौडान्तर्गतः कुमरिलभट्टजामाता पूर्वकाण्डप्रवर्तकः मण्डनमिश्र इति ख्यातः विश्वरूप एव सन्यासस्वीकारादनन्तरं सुरेश्वर इति प्रसिद्ध इति साम्प्रदायिकाः वदन्ति । 
"जागोप् महाशयेन नैष्कर्म्यसिद्धिभूमिकायां मण्डनमिश्रसुरेश्वरविश्वरूपाणामैक्यं स्वीक्रियते । सप्तदशशतकीयेन बालकृष्णानन्दसरस्वत्या कृते शारीरकमीमांसाभाष्यवार्तिके मण्डनमिश्रसुरेश्वरविश्वरूपाणामैक्यमेवोपवर्णितम् । विद्यारण्यै र्विववरणप्रमेयसंग्रहे बृहदारण्यकभाष्यवार्तिकादुद्धरणं दत्तम् । तत्रापि विश्वरूपशब्देन सुरेश्वर एव निर्दिष्टः ।।"
दासगुप्तस्तु सुरेश्वरविश्वरूपावभिन्नौ मण्डनमिश्रस्तु भिन्न इति (HIP Vol .II) वदति । हिरियण्णामहाशयस्तु (J. R. A. S. 1924)  जर्नल आफ रायल आसियारिक सोसाइटि पत्रिकायां सुरेश्वरः मण्डनादन्य इति प्रतिपादयति । 
कुप्पुस्वामिशास्रिणस्तु ब्रह्मसिद्धिभूमिकायां सुरेश्वरब्रह्मसिद्धिकारयोस्सिद्धान्तगतभेदमुपवर्ण्य ब्रह्मसिद्धिकारः सुरेश्वरादन्य इति प्रतिपादयन्ति। 
संक्षेपशारीरककर्ता सर्वज्ञात्मा सुरेश्वरशिष्य इति तु प्रसिद्धिः । सुरेश्वरश्च देवेश्वरशब्देन सर्वज्ञात्मना निर्दिष्ट इति तु पण्डितपरम्परागता वार्ता । 
श्रीकण्ठशास्त्री (I.H.Q. val. XIV) एवं (J.O.R. 1937) पत्रिकायां नायं सर्वज्ञात्मगुरुस्सुरेश्वर इति प्रतिवादयति । देवेश्वरस्त्वन्य इति च प्रतिपादयति । चिन्तामणिमहोदयेन च अङ्गीक्रियते । निरूपितञ्चैतदस्माभिर्विस्तरेण सर्वज्ञात्मप्रसङ्गे । 
श्रीकण्ठशास्त्रिणा प्रदर्शितायां शृङ्गगिरिगुरुपरम्परायां नित्यबोधघनाभिधः नित्यबोधाचार्य एव सुरेश्वरशिष्य इति निर्दिश्यते । नित्यबोधाचार्यकालश्च 773 - 848  इति च निर्दिश्यते । तस्मात् सर्वज्ञात्मन एव नित्यबोधाचार्य इति नामान्तरस्वीकारोऽपि सुष्ठु लगति । 
कुप्पुस्वामिशास्त्रिसिद्धान्तानुसारं, श्रीकण्ठशास्त्रिप्रदर्शितशृङ्गगिरिपरम्परानुसारञ्च सुरेश्वरकालः सप्तमाष्टमशतकमिति (620 - 777 A.D) इति सिध्यति । 
(A)  आनन्दगिरिकृता - तैत्तरीयवार्तिकव्याख्या । ( ASS. 13) 
(B)  लिङ्गनसोमयाजिकृतम् - कल्याणविवरणम् 
शाङ्करभाष्यार्थप्रकाशकः वार्तिकभावं सुलभं बोधयन् अयं ग्रन्थः शारदामुद्रणालये भटनविल्लीनगरे मुद्रितः । 
अस्य कर्ता श्रीरमणराज्यलक्ष्म्योः पुत्रः कल्याणानन्दभारतीशिष्यः आत्रेयगोत्रजः गुण्टूराख्यान्ध्रदेशवासी पञ्चदशीव्याख्याता विंशतिशतकीयः  (1900 - 1950 A.D.) लिङ्गनसोमयाजीति ज्ञायते ।। 
(C) विश्वानुभवकृता - तैत्तरीयभाष्यवार्तिकसङ्गतिः 
अमुद्रितोऽयं ग्रन्थः मद्रासराजकीयहस्तलिखितपुस्तकालये (R 2929 M.G. O. M. L.) लभ्यते । अस्य कर्ता प्रत्यग्बोधभगवतशिशष्यः विश्वनुभव इति परं ज्ञायते ।।
2. अद्वैतानन्दतीर्थकृता - तैत्तरीयदीपिका 
"तैत्तरीयोपनिषत्तात्पर्यदीपिका" नाम्ना प्रसिद्धोऽयं ग्रन्थः ब्रह्मविद्यामुद्रणालये मुद्रितः । अस्य कर्ता दक्षिणदेशवासी सदानन्दतीर्थशिष्यः, पूर्वश्रमे माध वसूरि-महालक्ष्म्योः पुत्रः हारीतगोत्रजः, आधुनिकोऽयं अद्वैतानन्द इति ज्ञायते । अदसीया ब्रह्मसूत्रवृत्तिरपि प्रतिपादिता ।। 
3. उपनिषद्ब्रह्मेन्द्रकृतम् - तैत्तरीयविवरणम् ।  (A. L. S.)
4. तारकब्रह्माश्रमिकृताः - तैत्तरीयोपनिषत्सारसंग्रहः
अमुद्रितोऽयं ग्रन्थ अडयार पुस्तकालये लभ्यते । अस्य रचयिता दक्षिणदेशीयः रामचन्द्राश्रमिशिष्यः, कल्पतरुपरिमलसंग्रहकर्ता अष्टादशशतकीयः (1700 - 1800 A.D.) तारकब्रह्माश्रमीति ज्ञायते ।। 
5. नारायणाश्रमिकृता - दीपिका । (ASS. 12)  
6. बालकृष्णानन्दसरस्वतीकृतम् - तैत्तरीयविवरणम् 
अमुद्रितोऽयं ग्रन्थः मद्रासराजकीयहस्तलिखितपुस्तकालये (R 383 M.G.O.M.L.) लभ्यते । अस्य कर्ता पद्यमयशारीरकमीमांसाभाष्यवार्तिककर्ता काञ्चीमण्डलान्तर्गतवेदपुरीवासी अभिनवद्रविडाचार्यबिरुदभूषितः, श्रीधरानन्दसरस्वत्याः प्राप्तदीक्षः, गौडब्रह्मानन्दसरस्वत्याश्शिष्यः, महादेवकैलासेशगुरुः सप्तदशशतकीयः (1600 - 1700 A.D.) बालकृष्णानन्दसरस्वतीति ज्ञायते ।
7. भास्करानन्दकृता - तैत्तरीयकव्याख्या । मुद्रितश्चायं ग्रन्थः भारतीजीवनमुद्रणालये वाराणस्याम् ।।
8.  विज्ञानात्मकृता - तैत्तरीयकविवृतिः 
अमुद्रितोऽयं ग्रन्थः मद्रासराजकीयपुस्तकालये (R. 3208 M.G.O.M.L) लभ्यते । अस्य कर्ता परमानन्दमस्करीत्यपरनामा ज्ञानोत्तमस्य शिष्यः चित्सुखाचार्यसतीर्थ्यः द्वादशशतकवासी (1100 - 1200 A.D.) विज्ञानात्मा इति ज्ञायते ।।
9. विद्यारण्यकृता - लघुदीपिका । अमुद्रितोऽयं ग्रन्थः मद्रासरजकीयपुस्तकालये (R. 1968 M.G.O.M.L) बरोडासूच्याञ्च लभ्यते । अस्य कर्ता विवरणप्रमेयसंग्रहकारः पञ्चदशीनिर्माता विद्यारण्यः ।।
(क) कृष्णानन्दश्रीरामशिष्यकृता - लघुदीपिकाटीका 
विद्यारण्यकृतलघुदीपिकासारसंग्रहकारी ग्रन्थोऽयं अमुद्रितः सरस्वतीमहालये (1494 T.S.M.L) मद्रासराजकीयपुस्तकालये (D. 515 M.G.O.M.L) च लभ्यते । 
10. वेङ्कटनाथकृता - तैत्तरीयकटीका । ग्रन्थोऽयं वेङ्कटनाथकृतायां ब्रह्मानन्दगिरिनाम्न्यां भगवद्गीताव्याख्यायां (461 V.V.P. Edn.) ग्रन्थकारेणैव निर्दिष्टंः । 
11. शङ्करानन्दकृता - दीपिका । (ASS 12)
12. सीतारामकृतम् - तैत्तरीयकव्याख्यानम् आगमामृतम् 
अमुद्रितोऽयं ग्रन्थः मद्रासराजकीयहस्तलिखितपुस्तकालये (D.514 M.G.O.M.L) लभ्यते । ग्रन्थोऽयं शिक्षावल्याः परं लभ्यते । अस्य कर्ता उत्तरमायूर क्षेत्रवासी कौण्डिन्यगोत्रजः वीरमाम्बागर्भजः अच्चन्नासूरिपुत्रः शिष्यश्च सप्तदशशतकीयस्तीतारामशास्त्रीति ज्ञायते ।। 
13. अज्ञातकर्तृका - तैत्तरीयकव्याख्या । ग्रन्थोऽयं मद्रासराजकीयहस्तलिखितपुस्तकालये (D 508 M.G.O.M.L) लभ्यते ।। 

%तैत्तरीयोपनिषद् 

नारायणोपनिषत् -
अस्यामुपनिषदि भगवतो नारायणस्य सङ्कल्पात् प्राणेन्द्रियभूतानामुत्पत्तिः, तस्मादेव ब्रह्मरुद्रेन्द्रादीनां स्वस्वकार्येषु प्रवृत्यादिकञ्च ऋग्वेदशिरोधीतमभिहितम् । तथा यजुर्वेदशिरोधीतं नारायणस्य ब्रह्मशिवाद्यन्तर्यामित्वजगद्वयाप्त्यादिकम् , सामवेदशरोधीतं नारायणाष्टक्षरमन्त्रस्वरूपतदध्ययनफलादिकं, अथर्वशिरोधीतं प्रणवोपासनाभ्यां दिव्यलोकगमनादिकं प्रतिपादितम् । यद्यपीयमुपनिषद्विशिष्टाद्वैतसिद्धान्तमृलमिति प्रथा वर्तते तथापि अद्वैतिभिरियमुपनिषदद्वैतसिद्धान्तप्रदर्शिनी व्याख्याता ।। 
1. उपनिषद्ब्रह्मेन्द्रकृतम् - विवरणम् (ALS)
2. नारायणाश्रमिकृता - दीपिका । ग्रन्थो बरोडासूच्यां दृश्यते (11529 BRD)
3. माधवाचार्यकृतम् - भाष्यम् । सरस्वतीमहालये (1504 D.C.T.S.M.L) दृश्यते ।
4. विज्ञानात्मकृतम् - विवरणम् । सरस्वतीमहालये तत्सूच्यां (1505 D.C.T.S.M.L) दृश्यते ।
5. शङ्करानन्दकृता - दीपिका । सरस्वतीमहालये तत्सूच्यां(1508 D.C.T.S.M.L) बरोडासूच्याञ्च (98191 BRD) दृश्यते ।।
नृसिम्हतापनीयोपनिषत् - 
पूर्वोत्तरभागद्वयेन विभक्तायामस्यामुपनिषदि पूर्वभागे श्रीनृसिम्हविद्यायाः साख्यायिकाप्रदर्शनमवताणम् , सामसम्बन्धित्वेन पृथिव्यादिलोकानामुपासनम् , साङ्गसामोपासनम् , सामाङ्गदेवतानामग्न्यादिदेवतानामुपासनम् , आरूढयोगस्य नरकेसरिणः उपासनाकारविशेषादिनिरूपणपूर्वकं सामचतुष्टयोद्धारप्रकाशनम् , साख्यायिकमधिकारिविशेषणान्तरकथनम् , प्रणवचतुर्मात्राव्यूहोपासनम् , हृदयाद्यङ्गपञ्चकोपन्याससहितम् , पादाक्षरसंख्यापूर्वककृत्स्नमूलमन्त्राक्षरसंख्यानिरूपणम् , मूलमन्त्रगतपदोद्धारपूर्वकं तदर्थप्रकाशनम् , मूलमन्त्रजपोपयोगिशक्तिबीजादिनिरूपणम् , मूलमन्त्राङ्गमन्त्राणामुपदेशः, प्रणवसावित्रीमहालक्ष्मीनृसिम्हगायत्रीरूपाणां अङ्गमन्त्राणां व्याख्यानपूर्वकं द्वात्रिंशन्नृसिम्हव्यूहस्तुतिमन्त्राणां प्रदर्शनम् , नृसिम्ह विद्याफलञ्च प्रतिपादितम् । 
उत्तरस्मिन् भागे प्रणवमहिमानुवर्णनपूर्वकं प्रणवोपासनाप्रकारं प्रकाश्यतन्मुखेन ब्रह्मणस्स्वरूपादिकं विस्तरेणोपपादितं दृश्यते ।। मुद्रिता चेयमुपनिषदानन्दाश्रममुद्रणालये अडयार पुस्तकालये च । 
1. उपनिषद्ब्रह्मेन्द्रकृतम् - विवरणम् (ALS)
2. गौडपादमुनिकृतम् - नृसिम्हतापिनीभाष्यम् 
अमुद्रितोऽयं ग्रन्थः मद्रासराजकीयहस्तलिखितपुस्तकालये (D. 581 D. 582 MGOML) उज्जैन् सूच्यां नासिकसूच्यां च दृश्यते । किमयं माण्डूक्याकारिकाकार उत अन्य इति न ज्ञायते ।
गौडपादमधिकृप्य माण्डूक्योपनिषत्प्रस्तावे उपपादयिष्यते ।। 
3. नारायणाश्रमिकृता - दीपिका । अमुद्रितोऽयं ग्रन्थः बरोडासूच्यां (11490 BRD) लभ्यते ।
4. विद्यारण्यकृता - दीपिका । उत्तरतापिन्याः परं विद्यारण्यकृता दीपिका अमुद्रिता मद्रासराजकीयपुस्तकालये (R. 3615 MGOML) बरोडापुस्तकालये च लभ्यते । विद्यारण्यः उपपादितचरः ।। 
5. शङ्कराचार्यकृतम् - नृसिम्हतापिनीभाष्यम् 
ग्रन्थोऽयं मद्रासराजकीयपुस्तकालये (D. 581-583 MGOML) सरस्वती महालये (1509 DC. TSML) बरोडासूच्यां 269 च लभ्यते । शङ्कराचार्येण नृसिम्हतापनीयस्य व्याख्या कृता इत्यत्र माध्वाचार्येण अथवा अभिनवकालिदासेन कृतं शङ्करदिग्विजयकाव्यमपि प्रमाणं भवति । ग्रन्थोऽयं आनन्दश्रामे (ASS 30) मुद्रितः ।
6. शङ्करानन्दकृताच दीपिका विद्यते । 
प्रश्नोपनिषत् - 
अस्यामुपनिषदि सृष्टिप्रकारः, प्राणस्य प्राधान्यम् , प्राणविद्योपासनम् , परविद्याविचारः, प्रणवोपासनम् , षोडशकलपुरुषविचारश्व विविच्य प्रतिपादिताः । अर्थर्ववेदीया इयमुपनिषत् । अस्या रचनाकाल (500 - 400 B.C.) इति विमर्शकाः । मुद्रिता चानन्दाश्रममुद्रणालये । 
1. शङ्कराचार्यकृतम् - प्रश्नोपनिषद्भाष्यम् (ASS 8) वाणीविलासमुद्रणालये आनन्दाश्रमे च मुद्रितम् । 
(क) अभिनवनारायणसरस्वतीकृता - भाष्यव्याख्या 
अमुद्रितेयं व्याख्या मद्रासराजकीयहस्तलिखितपुस्तकालये (D. 621 MGOML)  अडयारपुस्तकालये बरोडासूच्याञ्च (6944 DCBRD) दृश्यते । 
(ख) आनन्दगिरिकृतम् - भाष्यटिप्पणम् ।
यद्यपि आनन्दगिरिकृता इति व्याख्येयमानन्दाश्रमे मुद्रिता तथापि आनन्दगिरेरर्वाचीनाया विद्यारण्यकृतदीपिकायास्तत्रोल्लोखात् नेयं प्रसिद्धानन्दगिरिकृता भवितुमर्हति । शुद्धानन्दादिगुरुस्मरणमपि न तत्र दृश्यते । मुद्रितग्रन्थस्य अन्तिमपुष्पिकायां विद्यमानः " कैवल्येन्द्रशिष्यज्ञानेन्द्रगुरुचरणसेविनारायणेन्द्रसरस्वतीविरचितं प्रश्नोपनिषद्भाष्यविवरणं समाप्तम् " इति पाठभेदोऽपि आनन्दगिरिकृतत्वे संशयमुत्पादयति ।। 
(ग) शिवानन्दयतिकृतम् - भाष्यटिप्पणम् । अमुद्रितोऽयं ग्रन्थः मद्रासराजकीयहस्तलिखितपुस्तकालये (D.389) अडयारपुस्तकालये च लभ्यते ।।
(घ) अज्ञातकर्तृकम् - भाष्यटिप्पणम् । अमुद्रितोऽयं ग्रन्थः मद्रासपुस्तकालये (D. 620 MGOML) लभ्यते । 
2. उपनिषद्ब्रह्मेन्द्रकृतम् - प्रश्नोपनिषद्विवरणम् । (ALS)
3. नारायणाश्रमिकृता - दीपिका । (ASS 8)
4. शङ्करानन्दकृता - दीपिका । (ASS 8)

%प्रश्नोपनिषद् 

बृहदारण्यकोपनिषत् - 
शुक्लयजुश्शाखान्तर्गतायां षडध्यायीपरिमितायां अस्यां उपनिषदि षड्भिर्ब्राह्मणैर्विभक्ते प्रथमेऽध्याये अश्वमेधीयाश्वस्य ब्रह्मदृष्ट्यौपयिकविश्वरूपत्वोत्कीर्तनम् , उद्गातरि मुख्यप्राणदृष्टिकथनम् , नारायणतदुपासनतत्प्राप्तीनां परमतत्वहितपुरुषार्थत्वकथनम् , मनुष्यादिसृष्टिकथनम् कृत्स्नब्रह्मोपासनाभिधानम् , सर्वविज्ञानप्रश्नोत्तराभिधानम् , आत्मलोकोपासनकथनम् , पांक्तयज्ञादिकथनम् , अविद्याकार्योपसंहार इत्येतत्सर्वं दृश्यते । 
ब्राह्मणषट्कपरिमिते द्वितीयेऽध्याये कार्यकरणात्मकमूतस्वरूपनिर्धारणं मूर्तामूर्तब्रह्मस्वरूपाभिधानम् , "आत्मैवेदं सर्व" मिति प्रतिज्ञातार्थनिर्वहणम् , विद्याप्रवर्तकाचार्यवंशकथनम् इतीमे विषयाः प्रतिपाद्यन्ते । 
नवभिर्ब्राह्मणैर्विभक्ते तृतीयेऽध्याये याज्ञवल्क्यमहिमानुवर्णनपूर्वकस्तेन सह कहोलगार्ग्युद्दालकशाकल्यादिसंवादः परिदृश्यते ।
ब्राह्मणषट्कपरिमिते चतुर्थाध्याये जनकं प्रति याज्ञवल्क्येन जीवस्वरूपदेहान्तराप्तिक्रमतदुपेयपरमफलतत्साधनादिकमभिहितं दृश्यते । 
पञ्चदशभिर्ब्रह्मणैर्विभक्ते पञ्चमाध्याये प्रणवोपासना सर्वोपास्त्यङ्गभूतशमदमादयः, ब्रह्मोपासनाङ्गं हृदयोपासनम् , विद्युदादिषु ब्रह्मदृष्ट्योपासनानि, ब्रह्मविदां गतिः, तपसःउपासनम् , अन्नप्राणयोर्ब्रह्मत्वेनोपासनम् , उक्थजुस्सामदृष्ट्या प्राणोपासनानि, गायत्र्युपहितब्रह्मोपासनम् , आतिवाहिकादित्याग्न्योः प्रार्थनामन्त्राश्च अभिधीयन्ते ।
ब्राह्मणैः पञ्चभिरुपेते षष्ठेऽध्याये वाक्च्क्षुःश्रोत्रमनोरेत आत्मकप्राणोपासनानि, वागादीनामुत्क्रमणप्रवेशक्रमः, प्रवहणनाम्ना राज्ञा गौतमाय पञ्चाग्निविद्योपदेशः, पञ्चाग्निविदां तदविदाञ्च गतिः, कर्मानुष्ठातुः सति दोषे प्रायश्चित्तानां कामनाविशेषेषु कर्तव्यानाञ्च विधिः ओदनपाककालजातकर्मनामकरणानि पुत्रवत्वप्रशंसा, एतद्विद्याप्रवचनवंशश्चाभिधीयन्ते ।
शुक्लयजुश्शाखीयशतपथब्राह्मणान्तर्गतमिदं उपनिषत्काण्डं बृहत्वात् आरण्यपाठ्यत्वात् ब्रह्मविद्याप्रतिपादकत्वाच्च बृहदारण्यकोपनिषदिति व्यपदिश्यते । अध्यायष्टकपरिमितमप्येतत् प्रथमद्वितीययोर्ब्रह्माप्रतिपादकत्वात् तृतीयस्याश्वमेधविषयकत्वेऽपि ब्रह्मदृष्टिविधिरूपतया ब्रह्मात्मकत्वप्ततिपादनपरतया च ब्रह्मसम्बन्धित्वेन तृतीयाध्यायप्रभृति अध्यायषट्कपरिमितमेव उपनिषदिति व्यवह्नियते ।
अस्या रचनाकालः (600 B.C)  इति विमर्शकाः । मुद्रिता चेयनुपनिषदानन्दाश्रममुद्रणालयेऽन्यत्र च । एतत्सम्बद्धाः व्याख्यानुव्याख्यारूपाः ग्रन्थाः - 
1. शङ्कराचार्यकृतम् - बृहदारण्यकभाष्यम् (ASS 16)
(क) आनन्दगिरिकृता - भाष्याव्याख्या (ASS 16) व्याख्यायामस्यां भर्तृप्रपञ्चकृता बृहदारण्यकव्याख्या निर्दिष्टा या चाद्याविधि न लभ्यते ।। 
(ख) महादेवेन्द्रसरस्वतीकृता - भाष्यतात्पर्यदीपिका ।
द्वितीयाध्यायचतुर्थब्राह्मणसारात्मकोऽयं ग्रन्थस्सरस्वतीमहालये ( 1539 - 1540 DC. TSML) लभ्यते । अस्य कर्ता गोपालबालयोगिप्रशिष्यः स्वयम्प्रकाशानन्दसरस्वत्याः सतीर्थ्यः सुदर्शनेन्द्रेत्यपरनामा महादेवसरस्वती षोडशसप्तदशशतकवासी (1600 - 1700 A.D.) इति ज्ञायते । एनमधिकृत्याधिकं प्रकरणग्रन्थनिरूपणवसरे प्रतिपादितम् ।
(ग) शिवानन्दयतिकृतम् - भाष्यटिप्पणम् । अमुद्रितोऽयं ग्रन्थः मद्रासराजक्रीयपुस्तकालये (R. 3882 MGOML) लभ्यते । 
(घ) सुरेश्वराचार्यकृतम् - बृहदारण्यकभाष्यवार्तिकम् । मुद्रितश्चायं ग्रन्थ आनन्दाश्रममुद्रणालये (ASS 16) सव्याख्याः । 
(A) आनन्दगिरिकृता - भाष्यवार्तिकव्याख्या । (ASS 16)  शास्त्रप्रकाशिकानामायं ग्रन्थः आनन्दाश्रमे (ASS 16) मुदितः । सुरेश्वराश्चोपपादितपूर्वः । 
(B) आनन्दपूर्णविद्यासागरकृता - भाष्यवार्तिकव्याख्या। 
न्यायकल्पलतिका नाम्नीयं व्याख्या अमुद्रिता मद्रासराजकीयपुस्तकालये (R. 5283 MGOML) बरोडासूच्याञ्च (8938 B.C.BRD) दृश्यते । अस्य कर्ता खण्डनखण्डखाद्यव्याख्याता श्वेतगिर्यभयानन्दशिष्यः, पुरुषोत्तमानन्दगुरुः विद्यासागरापरनामा आनन्दपूर्णश्चतुर्दशशतकीय इति ज्ञायते । तिरुपति केन्द्रीय विद्यापीठेन ग्रन्थस्यास्य भागद्वयं प्रकाशितम् । एनमधिकृत्याधिकमन्यत्र निरूपितम् ।। 
(C) ज्ञानोत्तममिश्रकृता - भाष्यवार्तिकव्याख्या ।
ग्रन्थोऽयं दशमशतकीयेन ज्ञानोत्तममिश्रेण कृत इति दासगुप्तमहाशयेन (HIP. Vol. II) निर्दिश्यते । परन्तु ग्रन्थस्य प्राप्तिस्थानं न लभ्यते । 
(D) नृसिम्हप्रज्ञकृतम् - भाष्यवार्तिकविवरणम् ।
न्यायतत्वविवरणमित्यपरनामायं ग्रन्थ अमुद्रितः मद्रासराजकीयहस्तलिखितपुस्तकालये (R. 3327 MGOML) अडयारपुस्तकालये अनन्तशयनपुस्तकालये शृङ्गगिरिमठसूच्यां वेङ्कटेश्वरपुस्तकालयसूच्याञ्च दृश्यते । अस्य कर्ता कुलशेखरपुरीशिष्यः नृसिम्हप्रज्ञ इति परं ज्ञायते ।।
(E) विश्वानुभवकृता - भाष्यवार्तिकसम्बन्धोक्तिः । 
अनादिरनन्तश्चायं ग्रन्थस्तृतीयाध्यायचतुर्थाध्याययोः परं अमुद्रितः मद्रासराजकीयपुस्तकालये (R. 4435 MGOML) अडयारपुस्तकालये च लभ्यते । अस्य कर्ता प्रत्यग्बोधशिष्यः विश्वानुभव इति परं ज्ञायते ।।
(F) विद्यारण्यकृतः - वार्तिकसारः । मुद्रितश्चायं ग्रन्थः वाराणसीग्रन्थमालायाम् । अस्य कर्ता प्रसिद्धः विद्यारण्यः पञ्चदश्यादिकर्ता ।
1.  अज्ञातकर्तृका - वार्तिकसारव्याख्या । अमुद्रितोऽयं ग्रन्थः तिरुवनन्तपुरसूच्यां (397 TCD) दृश्यते ।
2. महेश्वरतीर्थकृतः - वार्तिकसारसंग्रहः ।
ग्रन्थोऽयं वारणसीसंस्कृतग्रन्थमालायां मुद्रितः। अस्य कर्ता चिदानन्दघनस्य विद्यारण्यस्य च शिष्यः महेश्वरतीर्थः । स्वग्रन्थे आनन्दगिरिं निर्दिशन्नयं ग्रन्थकारः विद्यारण्यशिष्यः चतुर्दशशतकीय (1350 - 1450 A.D) इति निश्चेतुं अर्हते ।। 
3.  अज्ञातकर्तृका - वेदान्तोपनिषत् । वार्तिकसारात्मकोऽयं ग्रन्थः (7612 TSML) सरस्वतीमहालये लभ्यते ।
4. उपनिषद्ब्रह्मेन्द्रकृतम् - बृहदारण्यकविवरणम् ।। (ALS)
5.  द्रविडाचार्यकृतम् - बृहदारण्यकभाष्यम् । 
भाष्यमिदं नोपलभ्यते । तथापि शङ्करात् पूर्वतनेन द्रविडाचार्येण कृतमिति भाष्यादिषु उद्धृतत्वात् ज्ञायते । एतच्च द्रविडाचार्यप्रकरणे प्रतिपादयिष्यते । 
6. नारायणाश्रमिकृता - दीपिका । यद्यपीयं दीपिका नोपलभ्यते तथापि जाबालोपनिषद्दीपिकायाः 275 तमे पुटे ग्रन्थकृतैव निर्दिष्टा ।। 
7. नित्यानन्दाश्रमकृता - मिताक्षरा । (ASS 31) 
8. वासुदेवब्रह्मकृता - प्रकाशिका । अमुद्रितोऽयं बरोडासूच्यां दृश्यते । अस्य कर्ता अनिरूद्धपुत्रः हृषीकेशशिष्यः वासुदेवब्रह्म एकोनविंशतिशतकीय (1800 - 1909 A.D.) इति ज्ञायते । 
9. विद्यारण्यकृता - बृहदारण्यकव्याख्या । ग्रन्थोऽयममुद्रितः नासिकसूच्यां दृश्यते ।
10. शङ्करानन्दकृता - दीपिका। ग्रन्थोऽयं सरस्वतीमहालयसूच्यां दृश्यते ।। 

% बृहदारण्यकोपनिषद् 

ब्रह्मविद्योपनिषत् -
अस्यामुपनिषदि प्रणवस्वरूपविवरणं कृतम् । प्रणवोपासनाप्रकारा अनेकविधा विद्याश्च प्रदर्शिताः । तासु हंसविद्याया मुख्यत्वं प्रदर्शितम् । निष्कलसकल भावविवेचनं ज्ञानमाहात्म्यप्रदर्शनञ्चावधेयार्हविषयाः । मुद्रिता चेयमुपनिषदानन्दाश्रममुद्रणालये (ASS 29) । 
1. उपनिषद्ब्रह्मेन्द्रकृतम् - विवरणम् । (ALS)
2. नारायणाश्रमिकृता - दीपिका । (ASS 29)
3. शङ्करानन्दकृता - दीपिका । (ASS 29)
ब्रह्मोपनिषत् - 
अस्यामुपनिषदि आत्माभिव्यक्तिस्थानानि ब्रह्मस्वरूपं, ब्रह्मोपासनाक्रमः, तेषां फलञ्च प्रतिपादितम् । याजुषीयमुपनिषदानन्दाश्रमे मुद्रिता  (ASS 29) । 
1. उपनिषद्ब्रह्मेन्द्रकृतम् - विवरणम् । (ALS)
2. नारायणाश्रमिकृता - दीपिका । (ASS 29)
3. शङ्करानन्दकृता - दीपिका । (ASS 29)
महोपनिषत् - 
अध्यायषट्कपरिच्छिन्नायामस्यामुपनिषदि सृष्ट्यादौ ब्रह्मा निर्दिश्य तस्मात् सृष्टिक्रमः, पञ्चविंशतितत्वोत्पत्तिः, नारायणात् रुद्रादीनामुत्पत्तिः, नारायणस्य सर्वात्मकत्वप्रतिपादनं, शुकजनकसंवादमुखेन ब्रह्मविद्याप्रशंसापूर्वकं शुकस्येत्तमां गतिमभिधाय, निदाघऋभुसंवादेन चिदचित्स्वरूपं प्रकाश्य जीवन्मुक्तत्वस्वरूपप्रकाशनपूर्वकं वैराग्यं प्रशस्य ब्रह्मज्ञानतत्प्राप्तिप्रकारं च निरूप्य एतत्पाठकस्य फलञ्चाभिहितम् ।
1. उपनिषद्ब्रह्मेन्द्रकृतम् - विवरणम् । (ALS)
2. नारायणाश्रमिकृता - दीपिका । (ASS 29)
3. शङ्करानन्दकृता - दीपिका । (ASS 29)
माण्डूक्योपनिषत् - 
नवधाथर्वण इति व्याकरणमहाभाष्यवाक्यात् अथर्वणवेदः नवशाखात्मकः । तस्यान्यतमा शाखा माण्ड्क्यशाखा । दुर्दैवपरिपाकात् तस्यास्संहिताभागः, ब्राह्मणभागो वा साम्प्रतं नोपलभ्यते । केवलं द्वादशमन्त्रात्मकं उपनिषन्मात्रं उपलभ्यते । तच्च माण्डूक्योपनिषदिति व्यवह्नियते । 
मण्डूको नाम मन्त्रद्रष्टा कश्चित् ऋषिः । मण्डूकस्य गोत्रापत्यं माण्डूकः । स एव माण्डूक्यः । माण्डूक्यस्य ऋषिपरत्वे प्रमाणन्तु बृहदारण्यकवंशब्राह्मणम् । तत्र हि 6-5-2 " जायन्तीपुत्रो माण्डूकायनिपुत्रात् , माण्डूकायनिपुत्रो माण्डूकीपुत्रात् , माण्डूकीपुत्रो शाण्डिलीपुत्रात् " इति दृश्यते । तेन प्रोक्ता उपनिषत् माण्डूक्योपनिषत् । अस्या उपनिषदः प्रशंसा मुक्तिकोपनिषदि 1-26, 29 मन्त्रैः कृता । अस्या उपनिषदः द्वादश मन्त्राः । एषु सप्त मन्त्राः नृसिम्हतापिन्युपनिषदि रामतापिन्युपनिषदि च दृश्यन्ते । तथा च तयोरूपनिषदोरर्वाचीनत्वं ज्ञायते । अत एव नृसिम्हतापनीयोपनिषदोऽपि गौडपादेन व्याख्या कृता इत्युक्तिरम्शतस्सत्यतामुपैति । 
अस्या उपनिषदः प्रथमे मन्त्रे चित्तशुद्धये उपास्यस्य प्रणवस्य सर्वात्मकता प्रदर्शिता । द्वितीये मन्त्रे ब्रह्मणश्शुद्धचिद्रूपस्य सर्वात्मकता वर्णिता । सा च ब्रह्मणस्सर्वात्मकता दुग्धदधिवन्न परिणामवामाश्रित्य, निरवयवे कूटस्थब्रह्मणि तदभावात् , किन्त रज्जुसर्पवत् विवर्तवादमाश्रित्य । अन्ते आत्मनश्चतुष्पात्वं प्रतिज्ञातम् । ततश्चतुर्भिः 3,4,5,6 मन्त्रैः "अध्यारोपापवादाभ्यां निष्प्रपञ्च आत्मा प्रपञ्च्यते " इति रीत्या विश्ववैश्वानरयोस्तैजसहिरण्यगर्भयोः प्राज्ञाव्याकृतयोर्व्यष्टिसमष्टिरूपयोस्तुरीये चिद्रूपे आरोपः कृतः । सप्तमेन नान्तःप्रज्ञमिति मन्त्रेण विश्वादीनामपवादेन तुरीयोऽवशेषितः । 
ये तु विशुद्धचित्ता उत्तमाधिकारणस्तेषां कृते नान्तःप्रज्ञमित्यादि मुख्योपदेशः। ये तु मध्यमाधिकारिणस्तेषां चित्तशुद्धये क्रममुक्तये च प्रणवमात्राणां अकार - उकार - मकार - तुरीयाणां आत्मनो ब्रह्मणः पादैः विश्वतैजसप्राज्ञतुरीयैः व्यष्टिभिः समष्टिभिश्च वैश्वानरहिरण्यगर्भईश्वरतुरीयैरभेदभावना  8,9,10,11 मन्त्रैरूपदिष्टा । द्वादशेन मन्त्रेण प्रणवोपासनाफलमुक्तम् ।
उपनिषदां औपनिषदानाञ्च ऋषीणामनादिः काल इति सम्प्रदायः। परन्तु पाश्चात्यानां विमर्शकानां मत माण्डूक्योपनिषदां कालः (200 - 100 B.C) इति प्रतिपादयति। मुद्रिता चेयमुपनिषदानन्दाश्रममुद्रणालयेऽन्यत्र च । 
अस्या व्याख्या एतत्सम्बद्धाः कृतयश्च - 
1.  गौडपादाचार्यकृता - माण्डूक्यकारिका ।  (ASS 10 ) 
गौडपादकारिकाभिधेऽस्मिन् माण्डूक्योपनिषदां व्याख्यात्मके पद्यबद्धे ग्रन्थे चत्वारि प्रकरणानि सन्ति । तत्र प्रथमे भागे आगमाख्ये एकोनत्रिंशत् द्वितीये वैतथ्याख्ये अष्टत्रिंशत् , तृतीये अद्वैताख्ये अष्टाचत्वारिंशत् , चतुर्थे अलातशान्तिप्रकरणे शतमिति संहत्या 215 कारिकास्सन्ति । मुद्रितश्चायं ग्रन्थ आनन्दाश्रममुद्रणालये (ASS 10) 
आगमप्रकरणम् - इदमागमप्रकरणं माण्डूक्योपनिषदां भावार्थरूपम् । आगममूलकत्वादन्वर्थं नाम । प्रकरणेऽस्मिन् अकार-उकारणकारैः प्रतिपादितेभ्यः वैश्वानर-हिरण्यर्गभ-ईश्वरेभ्यः जाग्रत् - स्वप्न- सुषुप्ति - अवस्थाभ्यश्च भिन्नं तदनुगतं साक्षिरूपञ्च परमात्मतत्वं तुरीय इति नाम्ना वर्णितम् । 
वैतथ्यप्रकरणम् - 
द्वितीयेऽस्मिन् वैतथ्याख्ये प्रकरणे दृश्यप्रपञ्चस्य मायामयत्वं, मिथ्यात्वञ्च सयुक्तिकं साधितम् । आत्मा एक एव नित्यः, तस्मिन् विविधकल्पनावशात् प्रपञ्चस्योत्पत्तिरिव विकल्पो भवति । अस्य मूलकारणं माया । मायाकल्पितजगतः गन्धर्वनगरवत् असत्यत्वमिति प्रतिपाद्य " न निरोधो न चोत्पत्ति " रित्यादिना अखण्डचिद्धनानन्दात्मतत्वादन्यस्य असत्वं साधितम् ।
अद्वैतप्रकरणम् -
तृतीयेऽस्मिन् प्रकरणे अनेकाभिस्सुदृढाभिर्युक्तिभिः अद्वैततत्वं साधितम् । आत्मनि सुखदुःखभावना नितरामसङ्गता यथा बालाः धूलिधूमादिंससर्गेण आकाशं मलिनमामनन्ति वस्तुतः यथा चाकाशो मालिन्यशून्यः तथैवात्मनोऽपि सुखित्वदुःखित्वकथनं बालबुद्धिविलासतुल्यमिति प्रतिपादितम् । असङ्गोह्यात्मा । माया हि द्वैतकल्पनायाः कारणम् । अमृतस्य मर्त्यत्वं मर्त्यस्यामृतत्वञ्चासङ्गतम् । अत अमृतस्यात्मनो यद्युत्पत्तिस्स्वीक्रियते तर्हि मर्त्यत्वधर्म आपद्येत इति आत्मनः उत्पत्तिः - जातिः नास्तीति प्रतिपादितम् । अयमेव गौडपादाचार्याणामजातिवादः सोऽयं वादः 
स्वप्नमाये यथा दृष्टे गन्धर्वनगरं यथा ।
तथा विश्वमिदं दृष्टं वेदान्तेषु विचक्षणैः ।। 2-31 
न निरोधो न चोत्पत्तिर्न बद्धो न च साधकः । 
न मुमुक्षुर्नवै मुक्त इत्येषा परमार्थता ।। 2 - 32 
प्रपञ्चो यदि विद्येत निवर्तेत न संशयः । 
मायामात्रमिदं सर्वमद्वैतं परमार्थतः ।। 1-17
न किञ्चिज्जायते जीवसम्भवोऽस्य न विद्यते ।
एतत्तदुत्तमं सत्यं यत्र किञ्चिन्न जायते ।। 3-4. 
इत्यादिषु तत्र तत्र प्रतिपादितः ।
अयमजातिवादः गौडपादात्प्राचीनस्य बौद्धाचार्यस्य दिङ्नागस्य माध्यमिकवृत्तौ, एवं दिङ्नागात् प्राचीनेषु पालीभाषाप्रणीतबौद्धग्रन्थेषु दर्शनात् ततो गृहीत इति विधुशेखरभट्टाचार्याः प्रवदन्ति। पालीभाषाया अपि प्राचीनेषु उपनिषद्ग्रन्थेषु " अजायमानो बहुधा व्याजायत " इत्यादिदर्शनात् , " नासतो विद्यते भाव " इति गीतायाञ्च दर्शनात् पूर्वोक्तोक्त्यनुपपत्तौ भारतीयंसस्कृतिभक्ताः प्रमाणम् । 
अलातशान्तिप्रकरणम् - 
चतुर्थऽलातशान्त्याख्ये प्रकरणे " अलाते भ्रामिते सति यथा गोलाकार प्रतीतिर्जायते परन्तु सा गोलाकृतिः भ्रमणजन्यैव, न वस्तुतः, एवं जगदादि मायाकल्पितमेव । मनसः व्यापारादेवोत्पत्तिस्तस्य मनसो निरोधे च स नास्त्येवेति। यथा च भ्रमणादिक्रियाशान्तौ गोलाकारकालातप्रतीतिशान्तिः, एवं मनस अमनीभावात् जगतश्शान्तिः। जगदुत्पत्तिलयौ प्रतीत्यप्रतीती उभेऽपि भ्रान्तिजनिते परमार्थतः परमात्मतत्वं पारमार्थिकमिति प्रतिपादितम् ।"
अद्वैतवेदान्तस्य प्राणभूता अनिर्वचनीयख्यातिरपि प्रकरणेऽस्मिन्नेव -
विपर्यासात् यथा जाग्रत् अचिन्त्यात् भूतवत् सृजेत् । 
तथा स्वप्ने विवपर्यासात् धर्मान् तत्रैव पश्वति ।। 4 - 41 
न निर्गतास्ते विज्ञानात् द्रव्यत्वाभावयोगतः ।
कार्यकारणताभावात् यतोऽचिन्त्यास्सदैव ते ।। 4-52. 
उभे ह्यन्योन्यं दृश्येते किं तदस्तीति नोच्यते । 
लक्षणाशून्यमुभयं तन्मतेनैव गृह्यते ।।  4 -67.
इत्यादिभिः कारिकाभिः प्रदर्शिता । 
प्रकरणस्यास्य भाषा "विज्ञप्ति" रित्यादिपारिभाषिकशब्दैः पूर्णा । ते च शब्दाः बुद्धग्रन्थेषु दृश्यन्त इति बुद्धमतमेव गौडपादः वेदान्तापदेशेन प्रतिपादयतीति, बुद्धप्रभावः गौडपादे दृश्यत इति, पच्छन्नबौद्धा अद्वैतिन इति व केचिद्वदन्ति। 
परन्तु शब्दसाम्यं नात्र प्रमाणमकिञ्चित्करञ्च । यतः - अध्यात्मशास्त्रस्य पारिभाषिकशब्दाः न केवलं बौद्धानां स्वम् । परन्तु ते सर्वदर्शनसामान्याः । तेषां प्रयोगे यथा गौडपादस्य तथा बौद्धानाम् यथा बौद्धानां तथा गौडपादस्यापीति समानस्वातन्त्र्यादिसत्वात् तादृशी युक्तिरसङ्गतैव । 
अस्याः कारिकायाः कर्ता गौडपादाचार्यः शुकमुनीन्द्रशिष्य इति वेदान्तसम्प्रदायप्रतिपादकात् "नारायणं पद्मभुवं वसिष्ठ " मितित्यादिश्लोकात् जायते । एवं नृसिम्हतापनीयोपनिषदां गौडपादकृते व्याख्याने ( D. 581, 582 MGOML)  दृश्यते । वेताश्वतरोपनिषदां शाङ्करे भाष्ये तथा च शुकशिष्यो गौडपादाचार्यः (Page 30 ASS Edn 17) दृश्यते । लक्ष्मणशास्त्रिकृते गुरुवंशकाव्येऽपि (Page 12) दृश्यते ।। 
गौडपादस्य स्थानं नाद्यापि निश्चितम् । विषयेऽस्मिन् विचाराःभिन्ना एव दृश्यन्त । परन्तु माण्डूक्यकारिकाशाङ्करभाष्यव्याख्याने आनन्दगिरीये अलातशान्तिप्रकरणस्थस्य " तं वन्दे द्विपां वरम् " इति पद्यांशस्य व्याख्याने एवं दृश्यते -
"आचार्यो हि पुरा बदरिकाश्रमे नरनारायणाधिष्ठिते नारायणं भगवन्त मभिप्रेत्य तपो महदतप्यत " (Page 157 ASS Edn) इति । एवञ्च बदरिकाश्रम एव गौडपादस्य स्थानमिति ज्ञायते । 
सप्तशतशतकीयेन बालकृष्णानन्दसरस्वत्या स्वीये शारीरकमीमांसाभाष्यवार्तिके - 
"गौडचरणाः कुरुक्षेत्रगताः हीरारावतीनदीतीरभवगौडजातिश्रेष्ठा देशविशेषभवनाम्नैव प्रसिद्धाः द्वापरयुगमारभ्यैव समाधिनिष्ठत्वेन आधुनिकैर्जनैरपरिज्ञातविशेषाभिधानाः सामान्यनाम्नैव लोकविख्याता " इति प्रतिपादितम् (ASI Page 6)। एवञ्च गौडपादः कुरुक्षेत्रवासी गौडजात्युत्पन्न इति गौडपादीयनामान्तरापरिज्ञाने च कारणं सूचितम् , केचित्तु गौडपादं बंगालदेशानां उत्तररणागवर्तिगौडदेशोद्भवमामनन्ति । 
भारतीयाद्वैतवेदान्तपरम्पराप्रामाण्यात् गौडपादश्शुकशिष्यः, शङ्करश्च गौडपादशिष्यः गौडपादेनानुगृहीतश्चेति शङ्करात्पूर्वतनो वा शङ्करकालपर्यन्तजीवीति वा निश्चीयते । यदि वयं कुप्पुस्वामिशास्त्रिणां सिद्धान्तमनुसृत्य ( 632 - 664 A.D.) कालवर्तिनं शङ्कराचार्यमभ्युपगच्छामस्तर्हि गौडपादकाल (520 - 620 A.D.) इति, इच्छामात्रशरीरत्यागिनां गौडपादानां कालश्शङ्कराचार्यानुग्रहपर्यन्तमिति वा स्वीकर्तव्यम् । 
विधुशेखरभट्टाचार्यास्तु - आगमशास्त्रस्य भूमिकायां द्वितीयशतकादारब्धानां चतुर्थशतकपर्यन्तानां बौद्धापण्डितानां ग्रन्थस्य गौडपादकारिकायाश्च शब्दसाम्यदर्सनात् गौडपादस्तदर्वाग्भव इति गौडपादः पञ्चमशतकीय ( 500 A.D.) इति प्रतिपादयन्ति । 
गौडपादस्यातिप्राचीनत्वेन प्रसिद्धेः, गौडपादकारिकास्था एव शब्दाः द्वितीयशतकादारब्धचतुर्थशातकान्तकालवर्तिभिः बैद्धपण्डितैस्स्वीकृताः , बौद्धैरेव गौडपादोऽनुसृतः, न तु गौडपादेन बौद्धपण्डिता इति च सम्प्रदायविदां सिद्धान्तः । गौडपादकारिकाया व्याख्या :-
(क) शङ्कराचार्य कृतम् - माण्डूक्यकारिकाभाष्यम् (ASS 10) 
1.  अनुभूतिस्वरूपाचार्यकृतम् - गौडपादीयभाष्यटिप्पणम् ।
अमुद्रितोऽयं ग्रन्थः मद्रासराजकीयपुस्तकालये (R. 2911 MGOML)  अडयारपुस्तकालये च लभ्यते । अस्य कर्तारं अनुभूतिस्वरूपाचार्यं अधिकृत्य प्रकटार्थविवरणस्थले उपपादयिष्यते ।
2. आनन्दगिरिकृता - माण्डूक्यकारिकाभाष्यव्याख्या । मुद्रितश्चायं ग्रन्थस्समूल आनन्दाश्रममुद्रणालये (ASS 10) । शुद्घानन्दकृता शाङ्कर कारिकाभाष्यस्य व्याख्या चास्तीति श्रूयते । 
(ख) स्वयम्प्रकाशयोगिकृता - मिताक्षरा (BSS 48)  
गौडपादीयकारिकाव्याख्यात्मकोऽयं ग्रन्थः वाराणसीसंस्कृतग्रन्थमालायां मुद्रितः । अस्य कर्ता स्वयम्प्रकाशयोगी यदि तत्वानुसन्धानव्याख्याता स्यात्तर्हि तस्य कालः (1700 - 1900 A.D.) समयान्तर्गतो भवति। नान्यदत्र प्रमाणं दृश्यते ।। 
 (ग) स्वामियतिकृता - मिताक्षरा । 
 माण्डूक्यकारिकाव्याख्यात्मकोऽयं ग्रन्थः चौखाम्बामुद्रणालये मुद्रितः । अस्य कर्ता स्वामियतिर्विरूपाक्षपुत्रः धारणकोटिकुलोत्पन्नः ब्रह्मानन्दशिष्यः रसरामशैलशशिगे शकवत्सरे 1736 शक (1813 A.D.) स्वग्रन्थं चकारेति ज्ञायते ।।
 (घ) उपनिषद्ब्रह्मेन्द्रकृता - कारिकाव्याख्या । अमुद्रितोऽयं ग्रन्थ अडयारपुस्तकालये लभ्यते ।।
 (ङ) अज्ञातकर्तृकः - गौडपादीयविवेकः । अमुद्रितोऽयं ग्रन्थः मद्रासराजकीयपुस्तकालये ( R. 3882 MGOML ) लभ्यते । माण्डूक्योपनिषद्भाष्यादि -
 2. शङ्कराचार्यकृतम् - माण्डूक्योपनिषद्भाष्यम् ।
 (क) आनन्दगिरिकृता - माण्डूक्यभाष्यव्याख्या । उभावपीमौ ग्रन्थौ आनन्दाश्रमे मुद्रितो । (ASS 10) 
 (ख) मधुरानाथशुक्लकृता - माण्डूक्यभाष्यव्याख्या । ग्रन्थोऽयं दासगुप्तमहाशयेन (H.I.P Vol. II Page 78) निर्दिष्टः । 
 (ग) अज्ञातकर्तृका - पदार्थविवृतिः । माण्डूक्यभाष्यव्याख्यात्मकोऽयममुद्रितः ग्रन्थः मद्रासराजकीयहस्तलिखितपुस्तकालये (D. 17021 MGOML) लभ्यते । 
 (घ) राघवानन्दकृतः - माण्डूक्यभाष्यार्थसंग्रहः ।
 ग्रन्थेऽयं दासगुप्तमहाशयेन (H.I.P. Vol. II Page 78)  निर्दिष्टः । अस्य कर्ता विश्वेश्वरसरस्वतीप्रशिष्यः अद्वियभगवत्पादशिष्यः राघवानन्द इति ज्ञायते । 
 3. उपनिषद्ब्रह्मेन्द्रकृतम् - माण्डूक्यविवरणम् (ALS) 
 4. नारायणाश्रमिकृता - दीपिका । अमुद्रितोऽयं ग्रन्थ अडयारपुस्तकालये सरस्वतीमहालये (1556 TSML) च लभ्यते ।  
 5. भास्करानन्दकृता - माण्डूक्यव्याख्या । 
 6. शङ्करानन्दकृता - दीपिका । (1552 TSML) मद्रासराजकीयहस्तलिखितपुस्तकालये (D. 707 MGOML) बरोडापुस्तकालये च लभ्यते ।। 
 
 % माण्डूक्योपनिषद् 
 
मुण्डकोपनिषत् -
अथर्वशाखान्तर्गतायां प्रत्येकं खण्डद्वितययुतमुण्डकाख्यप्रकरणत्रितयविभक्तायामस्यां उपनिषदि ब्रह्मविद्योपदेशपरम्पराप्रदर्शनपूर्वकं ब्रह्मज्ञानसाधनभूतपरापरविद्यास्वरूपादिकमभिधाय परविद्यावेद्याब्रह्मणः स्वरूपं प्रकाश्य तस्मादेव जगतस्संग्रहेण प्रादुर्भावमुपवर्ण्य मुमुक्ष्वमुमुक्षुकर्तव्यानि च संगृह्य ब्रह्मणो जगत्सृष्टिं विस्तरतः, प्रतिपाद्य प्रणवाख्याशरेण ब्रह्मात्मकलक्ष्यभेदनप्रकारमुक्त्वा प्राप्यस्वरूपतत्प्राप्तिप्रकारतत्साधनादिकं ब्रह्मविदाऽङ्गीरसा शौनकायोपदिष्टं दृश्यते ।। अस्या रचनाकालः 500-400 BC इति विमर्शकाः । मुद्रिता चेयमुपनिषदानन्दाश्रममुद्रणालये ।
अस्या व्याख्यादि - 
1. शङ्कराचार्यकृतम् - मुण्डकोपनिषद्भाष्यम् । (ASS 9) 
(क अभिनवनारायणेन्द्रकृता - मुण्डकभाष्यटीका । अमुद्रितोऽयं ग्रन्थ "औध " सूच्यां (XXI 26)  दृश्यते । 
(ख) आनन्दगिरिकृता - मुण्डकभाष्यव्याख्या । (ASS 9)
(ग) शिवानन्दयतिकृता - मुण्डकभाष्यटिप्पणी । 
अमुद्रितोऽयं ग्रन्थः सरस्वतीमहालयसूच्यां (1563 DCTSML) मद्रासराजकीयपुस्तकालये (D. 722 MGOML) च लभ्यते । 
2. उपनिषद्ब्रह्मेन्द्रकृतम् - मुण्डकोपनिषद्विवरणम् ( ALS)
3. नारायणाश्रणिकृता - मुण्डकोपनिषद्दीपिका । (ASS 9) 
4. भास्करानन्दकृता - मुण्डकोपनिषद्दीपिका ।
5. शङ्करानन्दकृता - मुण्डकोपनिषद्दीपिका । अमुद्रितोऽयं ग्रन्थस्सरस्वती महालयसूच्यां (1561 DCTSML) लभ्यते ।

% मुण्डकोपनिषद् 

रामतापिन्युपनिषत् -
पूर्वोत्तरतापिनीभेदेन द्विधा विभक्तेयमुपनिषत् । अत्र पूर्वतापिन्यां रामशब्दार्थः परं ब्रह्म इत्युक्त्वा तदुपासनोपयोगि मन्त्रमभिधाय रामख्याबीजात् जगदुत्पत्तिमभिवर्ण्य रामायणाख्यायिकाश्रवणफलान्युक्त्वा रामयन्त्रोद्धारकमश्च प्रतिपादितः । उत्तरतापिन्यां तारकशब्दनिर्वचनपूर्वकं तारकमन्त्रस्योपदेष्टारं स्थानमभिधाय तद्देवतानमस्कारमन्त्रकथनपूर्वकं तारकमन्त्रस्य महात्म्यं फलञ्च निरूपितम् । मुद्रिता चेयमुपनिषदानन्दाश्रममुद्रणालये । 
1. उपनिषद्ब्रह्मेन्द्रकृतम् - रामतापिनीविवरणम् । (ALS) 
2. नागेश्वरसूरिकृता - रामतापिनीवृत्तिः । अमुद्रिता चेयं मद्रासराजकीयपुस्तकालये (D. 763 MGOML) लभ्यते । 
3. नारायणाश्रमिकृता - रामतापिनीदीपिका । (ASS 29)
4. भट्ट मुद्गलसूरिकृता - रामोत्तरतापिनीव्याख्या । अमुद्रितेयं व्याख्या (D. 726 MGOML) लभ्यते । 
5. रामयतिकृता - पदयोजनिका  
अमुद्रितोऽयं ग्रन्थः (D. 764 MGOML) मद्रासराजकीयपुस्तकालये, अडयारपुस्तकालये, बरोडासूच्याचञ्च दृश्यते । गोविन्दानन्द शिष्योऽयं रामयतिरिति परं ज्ञायते । 
6. विश्वेश्वरकृता - रामतापिनीपूर्वखण्डव्याख्या । 
अमुद्रितोऽयं ग्रन्थः (D 762. MGOML) अडयारपुस्तकालये बरोडापुस्तकालये च लभ्यते । स्वयम्प्रकाशशिष्योऽयं विश्वेश्वरः ।।
7. सुरेश्वराश्रमिकृता - रामचन्द्रज्योत्स्ना । 
रामोत्तरतापिन्याः परं व्याख्यात्मकोऽयं ग्रन्थ अमुद्रितः बरोडापुकालये (246 DC BRD) दृश्यते । अस्य कर्ता रघुरामतीर्थशान्ताश्रमिशिष्य इति परंज् ज्ञायते । 
8. अज्ञातकर्तृका - रामतापिनीव्याख्या । ग्रन्थोऽयं सरस्वतीमहालयसृच्यां (1564 DC TSML) दृश्यते ।। 
श्वेताश्वतरोपनिषत् - 
अस्यामुपनिषदि क्षराक्षरस्वरूपकथनपूर्वंकं योगक्रममुपपाद्य योगगम्यस्य परमात्मनस्स्वरूपादिकं तस्य पुरुषसूक्तप्रतिपाद्यात्वमोक्षप्रदत्वादिकं चोपवर्ण्य, प्रकृतिस्वरूपं विविच्य परमात्मनो गायत्रीप्रतिपाद्यत्वमुक्त्वा क्षराक्षरविद्याफलभेदादिकं प्रकाश्य परमात्मस्वरूपतदुपासनातन्महिमादिकं प्रपञ्च्यते । श्वेताश्वतरविद्वदुपदिष्टत्वादियं श्वेताश्वतरोपनिषदिति कथ्यते । अस्या रचनाकालः (200 - 100 BC)  इति विमर्शकाः । मुद्रिता चेयमानन्दाश्रममुद्रणालये । (ASS 17) 
1. शङ्कराचार्यकृतम् - श्वेताश्वतरभाष्यम् । (ASS 17)
शङ्कराचार्यकृतत्वेनेदं भाष्यं मुद्रितमानन्दाश्रममुद्रणालये । सरस्वतीमहालयस्थादर्शग्रन्थान्ते (1565 TSML) च शङ्कराचार्यकृतमिति दृश्यते । परन्तु शङ्कराचार्यैः श्वेताश्वतराणां भाष्यं न कृतमिति ज्ञायते । अथवा नेदं भाष्यं शङ्कराचार्यकृतमिति निश्चीयते । अत्रेमानि कारणानि । 
(1) विद्यारण्यप्रणीतशङ्करविजये षष्ठसर्गे आचार्यशङ्करकृतोपनिषद्भाष्यादिनामनिर्देशे श्वेताश्वतरोपनिषद्भाष्यनामानिर्देशः । तत्रहि - 
"करतलकलिताद्वयात्मतत्वं क्षपितदुरन्तचिरन्तनप्रमोहम् ।
उपचितमुदितोदितैर्गुणौघैः उपनिषदामयमुज्जहार भाष्यम् ।।"
अत्र धनपतिसूरिकृतडिण्डिमाख्यटीकायां उपनिषदामित्येतत्पदव्याख्याने " ईशादिबृहदारण्यकान्तानां दशानामेवोपनिषदां संग्रहः कृतः। यदि श्वेताश्वतरोपनिषदामपि भाष्यं शङ्कराचार्यैः प्रणीतं स्यात्तर्हि डिण्डिमकारेणास्या उपनिषदोऽप्युल्लेखः कृतो भवेत् । "
(2) यासामुपनिषदां भाष्यं शङ्कराचार्येण कृतं तासाम् , यासामुपनिषदां शङ्कराचार्येण भाष्यं न कृतं तासामपि दीपिकानाम्नी व्याख्या नारायणाश्रमिणा रचिता आनन्दाश्रममुद्रणालये मुद्रिता च । तासु शङ्कराचार्यकृतत्वेन निश्चितभाष्याणानुपनिषदां व्याख्यावसरे नारायणाश्रमिणा " शङ्करोक्त्युपजीविना " इत्युच्यते । तासुदीपिकासु दृश्यते वाक्यतोऽपि साम्यं भाष्येण । शङ्कराव्याख्यातानामुपनिषदां व्याख्यावसरे "श्रुतिमात्रोपजीविना " इत्येव दीपीकायां निर्दिश्यते । श्वेताश्वतरदीपिकापि श्रुतिमात्रोपजीविना नारायणाश्रमिणैव कृता न तु शङ्करोक्त्युपजीविना । तस्मात् श्वेताश्वतरोपनिषदां शाङ्करं भाष्यं नास्तीत्येव ज्ञायते । 
(3) नारायणविरचिते दीपिकाख्ये व्याख्याने षष्ठेऽध्याये - " यदा चर्मवदाकाशं " इत्यादिविंशतितमर्क्व्याख्यानावसरे " अयमर्थ आचार्यसम्मतः । चर्मवदाकाशवेष्टनासम्भववत् अविदुषो मोक्षासम्भववत् अविदुषो मोक्षासम्भवश्रुतेरिति सर्वधरर्मान् परित्यज्येति श्लोके शाङ्करगीताभाष्ये उक्तत्वादिति " विद्यते । यदि श्वेताश्वतरोपनिषदः भाष्यं कृतं स्यात्तर्हि गीताभाष्ये उक्तत्वादिति हेतुरसङ्गतः । एतद्भाष्ये एव आचार्यैरुक्तत्वादित्येव वक्तुं शक्याम् । 
(4) आचार्यशङ्करप्रणीतदशोपनिषद्भाष्येषु दृश्यमाणाः पदलालित्य गाम्भीर्य सरलतादयो गुणाः अस्या भाष्ये न दृश्यते । तस्मात् शङ्करपीठाधिष्ठितैः अन्यैर्वा शङ्कराचार्याणां नाम्ना केनापि विदुषा वा कृतं स्यादित्येव निर्णीयते । शाङ्करभाष्याणां सर्वेषामपि आनन्दगिरिणा व्याख्या कृता श्रूयते । परन्तु श्वेताश्वतरोपनिषद्भाष्यस्य आनन्दगिरिकृता व्याख्या नाद्यापि लक्ष्यते च ।
2. उपनिषद्ब्रह्नेन्द्रकृतम् - श्वेताश्वतरविवरण् । (ALS) 
3. नारायणाश्रमिकृता - दीपिका । (ASS 17)
4. विज्ञानभगवत्कृतम् - श्वेताश्वतरविवरणम् । (ASS 17) 
मुद्रितञ्चेदं विवरणमानन्दाश्रममुद्रणालये (ASS 17)। अस्य कर्ता विज्ञानात्मा ज्ञानोत्तमषिष्यः, चित्सुखाचार्यसतीर्थ्यः, द्वादशशतकवासी (1100 - 1200 A.D.) इति प्रकृतग्रन्थपरामर्शात् श्रीकण्ठशास्त्रिकृतोपन्यासात् (IHQ Vol XIV) च ज्ञायते । 
5. शङ्करानन्दकृता - दीपिका । (ASS 17)
6. अज्ञातकर्तृका - दीपिका । सरस्वतीमहालयपुस्तकालये (1567 DCTSML) दृश्यते ।। 
हंसोपनिषत् - 
अस्यामुपनिषदि सर्वव्यापिनस्त्रिमात्रस्य हंसाख्यपरमात्मनः ध्यानक्रमं निरूप्य तस्य नादे विलयप्रकारमुक्त्वा ऋषिच्छन्दोदेवतान्यासपूवर्कं तन्मन्त्रस्वरूपादिकं संगृह्य तस्य दशविधनादस्वरूपतायां फलविशेषाश्च संगृहीताः । मुद्रिता चेयमानन्दाश्रम मुद्रणालये ।।
1. उपनिषद्ब्रह्मेन्द्रकृतम् - विवरणम् । (ALS)
2. नारायणाश्रमिकृता - दीपिका । अमुद्रितेयं दीपिका बरोडासूच्यां ( 11529 BRD) दृश्यते । 
3. शङ्करानन्दकृता - दीपिका । (ASS 27) आनन्दाश्रममुद्रणालये मुद्रिता । शङ्करानन्दः पूर्वमुपपादितः ।। 
