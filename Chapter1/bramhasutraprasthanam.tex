\chapter{ब्रह्मसूत्रप्रस्थानम् }
सर्वकारणकारणं एकमेवाद्वितीयं सत्यज्ञानानन्दस्वरूपं अनन्तं असङ्गं नित्यं ब्रह्म, तज्ज्ञानान्मोक्ष इत्यादिकं सर्वं अलौकिकविषयान्तर्गतम् । एतादृशेऽलौकिके विषये अनाद्यपौरुषेयश्रुतिरेव प्रमाणम् । श्रुतिश्चात्यन्तं दुरवबोधा, यतः श्रुतिवाक्यं संक्षिप्तम् । सङ्गतिप्रदर्शकवाक्यैः प्रायशो विरहितञ्च । भाषा चातिपुरातनी । लोकाश्चोत्तरोत्तरं मन्दबुद्धयः । एवञ्च श्रुतिप्रदर्शितो मार्गः मीमांसकादिविविधदर्शनकारैः कलुषीकृत आसीत् । " उपनिषद्भागोऽयं न श्रुतिशिरोभूतः । किन्तु जीवात्मनोऽविनाशित्वप्रतिपादनेन स्वर्गादिपारलौकिके सुखे विश्वासोत्पादनद्वारा यज्ञादिकर्मकाण्डस्यैवोपोद्बलकस्तदङ्गभूतश्चेति लोकानां मतिरभवत् । एतादृशं लोकप्रत्ययं निराकरिष्णुः भगवान् बादरायणः ब्रह्मसूत्राणि प्रणिनाय । ब्रह्मसूत्राणि च श्रुतिमीमांसामुखेनैव तत्वोपदेशाय प्रवृत्तानि । श्रुत्यर्थनिर्णयार्थं यो लोकवेदसाधारणो नियमः मीमांसकैराविष्कृत अवधारितश्च तेनैव नियमेन श्रुर्त्यथ अद्वैतमते निरूप्यते । " स एव वेदान्तसिद्धान्तः । सोऽयं सिद्धान्त एव ब्रह्मसूत्रैः प्रतिपाद्यते च । अत एव " वेदान्तवाक्यकुसुमग्रथनार्थत्वात् सूत्राणा " मिति 1-1-2 सूत्रभाष्य उपवर्णितम् । प्रायेण वेदान्तमतविरोधिनस्सर्वेऽपि वेदान्तमतखण्डनायाद्वैतमतमेव खण्डितवन्त इति " व्यासतात्पर्यनिर्णये " अय्यण्णादीक्षितेन प्रदर्शितम् । एतेनेदं स्पष्टं ज्ञायते यत् - " प्राचीना ऋषयः वेदान्तशब्देन अद्वैतमेव बुबुधिरे " इति । शाण्डिल्यभक्तिसूत्रेऽपि  " आत्मैकपरां बादरायण " (30) इति सूत्रेण बादरायणमतस्याद्वैत एव ऐदम्पर्यमुक्तं ज्ञायते । बौद्ध-जैना अपि अद्वैतमेव खण्डयन्ति। तस्मात् बादरायणव्यासनिर्मितानि ब्राह्मसूत्राणि अद्वैतमतमूलभूतानि अद्वैतवेदान्ते तृतीयप्रस्थानं सूत्रप्रस्थानमिति विशिष्यते । तस्माद् प्रकरणेऽस्मिन् ब्रह्मसूत्राणि तेषां वृत्यादयः ग्रन्थाः, भाष्यं, भाष्यव्याख्या इत्यादयस्सर्वेऽपि ग्रन्थाः यथाक्रमं निरूप्यन्ते । 
ब्रह्मसूत्राणि - 
" वेदान्तसूत्राणि " " शारीरकसूत्राणि " " ब्रह्ममीमांसासूत्राणि " " उत्तरमीमांसासूत्राणि " इति नाम्ना प्रसिद्धानि ब्रह्मसूत्राणि व्यासरचितानीति प्रसिद्धिः । सूत्राण्येतानि निखिलोपनिषदामर्थसङ्ग्राहकानि । महाभारतादिपुराणानां कर्ता व्यास एवैतेषामपि सूत्राणां कर्ता इति सम्प्रदायविदो वदन्ति। पाणिनेरपि बादरायणव्यासोऽयं प्राक्तन इति " पाराशर्यशिलालिभ्यां भिक्षुनटसूत्रयो " रिति सूत्रात् ज्ञायते । पाराशर्य - बादरायण - कृष्णद्वैपायन - द्वैपायन - सत्यवतीसुतादि नाम व्यासस्यवेति महाभारतादिग्रन्थाद् ज्ञायते । ब्रह्मसूत्राणां रचनाकालः द्वापरयुगान्ते कलेरादौ वा प्रायः पञ्चसहस्रवर्षेभ्यः पूर्वमिति सम्प्रदायसिद्धान्तः । "ऋषिभिर्बहुधा गीतं छन्दोभिर्विविधैः पृथक् । ब्रह्मसूत्रपदैश्चैव हेतुमद्भिर्विनिश्चितैः ।। (13-3) इति भगवद्गीतायां दर्शनात् सूत्राणामेषां प्राचीनतमत्वं सुसिद्धमेव ।"
गीतायां निर्दिष्टानि ब्रह्मसूत्राणि किं बादरायणकृतानि? उतान्यानि? इति कर्माकरमहाशयेन "अनालस आफ भण्डार कार" पत्रिकायास्तृतीये भागे (A.B.I Val. III Page 73) विमृष्टम् । विमर्शकैस्तु गीतायां निर्दिष्टानि नैतानि सूत्राणि, तानि नष्टानि, एतानि तु अन्यान्येवेति कथ्यते । ब्रह्मसूत्राणां कालनिर्णये न वयं शक्तः। परं तु हिरियण्णामहाशयाः क्रैस्तवीयचतुर्थतकमिति (400 A.D.) वदन्ति। क्रिस्तोः पूर्वं षष्ठशतकात्प्रागिति तु प्रसिद्धिः । 
समन्वय - अविरोध - साधन - फलाख्यैश्चतुर्भिरध्यायैः पूर्णोऽयं ग्रन्थः । प्रत्यध्यायं चत्वारः पादाः। प्रतिपादं बहूनि अधिकरणानि । प्रत्यधिकरणं भिन्नविषयः। शाङ्करमतेन सूत्राणां समष्टिसंख्या 555, अधिकरणानि 191, मुद्रितश्चायं ग्रन्थः निर्णयसागरादिषु बहुषु मुद्रणालयेषु । अस्य व्याख्या भाष्यं वृत्यादि च -  
I शङ्कराचार्यकृतम् - ब्रह्मसूत्रभाष्यम् । 
शारीरकमीमांसाभाष्यमित्यपरनामायं ग्रन्थः निर्णयसागर मुद्रणालयेऽन्यत्र च मुद्रितः। उमामहेश्वरेण तत्वचन्द्रिकायां शङ्करभागवत्पादैः स्वीये सूत्रभाष्ये एकोनशतं ब्रह्मसूत्रव्याख्याः खण्डिता इत्युक्तम् । भाष्यमिदं स्वीयया रचनाशैल्या गृढतमानां अर्थानां सरससरलभाषया प्रतिपादयत् प्रसन्नगम्भीरमिति प्रसिद्धम् । भाष्येऽस्मिन् वृत्तिकाराः भेदवादिनो नैय्यायिकवैशेषिकसांख्याश्च खण्डिताः । बुद्धमतञ्च खण्डितम् । भाष्येऽस्मिन् ब्रह्मसूत्राणि  191 विभिन्नविषयकेषु अधिकरणेषु विभज्य चतुर्भिरध्यायैरूपवर्णितानि । 
अस्य कर्ता विद्याघिराजपौत्रः, शिवगुरुपुत्रः, आत्रेयगोत्रजः, गौडपादप्रशिष्यः, गोविन्दभगवत्पादशिष्यः शङ्करभगवत्पाद इति प्रसिद्धिः। राजशेखरराजेन पालिते केरलदेशान्तर्गते कालटिक्षेत्रे लब्धजन्मान एते शङ्कराचार्याः पद्मपाद-सुरेश्वर-हस्तामलक-तोटकाचार्याणां गुरव अष्टमशतकापरार्धार्धादारब्धे नवमशतकारम्भान्ते च (788 - 820 A.D.) काले आसन्निति ज्ञायते । द्विसहस्रेभ्यः वर्षेभ्यः पूर्वमिति साम्प्रदायिकाः। श्री कुप्पुस्वामिशास्रिणस्तु स्वीयब्रह्मसिद्धिभूमिकायां शङ्कराचार्यकालं  (623 - 664 A.D.) इति प्रतिपादयन्ति। बालकृष्णपिल्लै महाशयस्तु  "इण्डियन आण्टिक्वैरि " पत्रिकायाः पञ्चाशत्तमे भागे (I.A. Vol. 50 Page 136) शङ्कराचार्याः दशमशतकादिमे भागे आसन्निति वर्णयति । शङ्कराचार्यकालमधिकृत्य विभिन्नास्सिद्धान्ताः शङ्कराचार्यकृतप्रकरणग्रन्थप्रस्तावे प्रतिपादिताः ।। 
(क) सूत्रभाष्यव्याख्या - शारीरकन्यायनिर्णयः । 
सूत्रभाष्यव्याख्यात्मकोऽयं शारीरकन्यायनिर्णयाख्यः ग्रन्थः निर्णयसागरमुद्रणालयेऽन्यत्र च मुद्रितः । अस्य कर्ता आनन्दज्ञानापराभिध आनन्दगिरिः । अस्यैव सन्यासस्वीकारात्पूर्वं जनार्दन इति नाम । गुजरातप्रान्तजोऽयं द्वारकास्थशाङ्करपीठाधीश आसीत् । आन्ध्रदेशजोऽयमिति साम्प्रदायिकाः। अनुभूतिस्वरूपाचार्यशुद्धानन्दयोश्शिष्योऽयं आनन्दगिरिः तत्बदीपनकारस्य अखण्डानन्दसरस्वत्याः, तत्वालोकव्याख्यातुः तत्वप्रकाशिकाकर्तुश्च प्रज्ञानानन्दस्य गुरुः, कलिङ्गदेशाधिपतेर्नृसिम्हदेवस्य सामयिकस्त्रयोदशशतकीयः (1260 - 1320 A.D.) इति ज्ञायते । आनन्दगिरिणा रचितास्सर्वे ग्रन्था अन्यत्रोपवर्णिताः ।।
(ख) सूत्रभाष्यव्याख्या - ब्रह्मविद्याभरणम् । 
सूत्रभाष्यव्याख्यात्मकोऽयं ग्रन्थः समग्रचतुरध्यायीभामतीनिगूढभावप्रकाशनपर अद्वैतमतस्याभरणमेव । ग्रन्थोऽयं अद्वैतमञ्जरीग्रन्थमालायां कुम्भघोणे मुद्रितः ।
अस्य कर्ता चिद्विलास इत्यपरनामा अद्वैतानन्दबोधेन्द्रः। अस्यैव पूर्वाश्रमे सीतापतिरिति नाम । प्रेमनाथपार्वतीपुत्रोऽयं द्रविडदेशस्थपञ्चनदक्षेत्रवासीति ज्ञायते । भूमानन्दापरनामभिश्चन्द्रशेखरसरस्वतीभिर्दीक्षितोऽयं रामानन्दसरस्वत्या लब्धविद्य इति च ज्ञायते । (1168 - 1201 A.D.) द्वादशतमे शतके अद्वैतानन्दबोध आसीदिति । साम्प्रदायिकाः । परन्तु सप्तदशशतकस्यादिमे भागे आसीदिति विमर्शकाः । अनेन " शान्तिविवरण " मिति ग्रन्थोऽपि कृतः ।। 
(ग) सूत्रभाष्यव्याख्या - प्रकटार्थविवरणम् । 
पञ्चपादिकाविवरणस्य गूढार्थत्वेन व्याख्यापेक्षतां प्रकटार्थविवरणस्य तदनपेक्षताञ्च प्रकटीकुर्वन्नयं ग्रन्थः " वाचस्पतिस्त मण्डनपृष्ठसेवीत्यादिना" भामतीखण्डनपर उदयनसुन्दरपाण्ड्याचार्यब्रह्मप्रकाशिकाः प्रमाणयति । ग्रन्थोऽयं मद्रासविश्वविद्यालयसंस्कृतग्रन्थमालायां मुद्रितः । 
यद्यपि मुद्रितपुस्तके प्रकटार्थकारस्य नाम न निर्दिष्टम् । नापि कल्पतर्वादौ प्रकटार्थविवरणखण्डनावसरे नाम निर्दिश्यते, नैव च दासगुप्तमहाशयेन प्रकटार्थविवरणस्य द्वादशशतकापरार्घकालिकत्वं वदता ग्रन्थकर्तुर्नाम निर्दिश्यते, तथापि मद्रासविश्वविद्यालयसंस्कृतविभागभूतपूर्वाध्यक्षेण राघवमहाशयेन अनुभूतिस्वरूपाचार्यः प्रकटार्थविवरणकार इति भण्डारकाररजतजयन्तीस्मारकपत्रिकायां सिद्धान्तितम् । 
दक्षिणदेशजोऽपि न्यायवेदान्ताध्ययनाय गुजरातप्रान्तीयकठियावारद्वारकादिक्षेत्रवासी अनुभूतिस्वरूपाचार्योऽयं आनन्दगिरेः, तत्त्वालोकव्याख्यातुः प्राज्ञानानन्दस्य सारस्वतप्रकियाव्याख्यातुर्नरेन्द्रनगर्याश्च गुरुः, खण्डनखण्डखाद्यव्याख्याने शिष्यहितैषिण्याख्ये श्रीहर्षं प्रणमन्नयं श्रीहर्षशिष्य इति च ज्ञायते । अदसीया अन्ये ग्रन्था व्याख्यारूपा बहवस्सन्ति । ते तत्तत्प्रकरणे अद्वैताचार्यप्रकरणे च प्रतिपादिताः ।।
(घ) सूत्रभाष्यव्याख्या - भाष्यभावप्रकाशिका ।
सूत्रभाष्यव्याख्यारूपोऽयं ग्रन्थ अध्यासभाष्यान्त अमुद्रितः मद्रासराजकीय हस्तलिखितपुस्तकालये (R 3020, R. 5140 MGOML) अडयार - महीशूर हस्तलिखितपुस्तकालययोश्च लभ्यते ।
अस्य कर्ता चित्सुखाचार्यः । चित्सुखाचार्योऽयं नित्यबोधघनज्ञानघनापरनाम्नः बोधघनस्य प्रशिष्यः, न्यायसुधाज्ञानसिद्धिकारस्य ज्ञानोत्तमस्य शिष्यः, श्वेताश्वतरोपनिषद्दीपिकाकर्तुर्विज्ञानात्मनस्सतीर्थ्यः, सुखप्रकाशाचार्यस्य, आनन्दगिरिगुरोश्शुद्धानन्दस्य च गुरुरिति ज्ञायते । सुखप्रकाशशिष्य अमलानन्दः शुद्धानन्दशिष्य आनन्दगिरिश्च चित्सुखाचार्य प्रशिष्याविति चार्थात् सिध्यति । त्रयोदशशतकापरार्धकालिकेन वेदान्तदेशिकेन निर्दिश्यमानोऽयं चित्सुखाचार्यः द्वादशशतकीय इति (1120 1220 A.D.) निश्चेतुं शक्यते । अनेनान्येऽपि व्याख्याग्रन्थाः तत्त्वप्रदीपिकाख्याः प्रकरणग्रन्थाश्च विरचिताः । ते चान्यत्रोपवर्णिताः ।। 
(ङ) सूत्रभाष्यव्याख्या - भाष्यदीपिका - टीकायोजना ।
ग्रन्थोऽयं गोविन्दानन्देन अथवा रामानन्देन कृतायां रत्नप्रभायां निर्णयसागरमुद्रितायां पञ्चमं पुटे (Page 5. NSP Edn.) " आश्रमचरणास्तु दीकायोजनायामेवमाहः " इति निर्दिष्टः । दासगुप्तमहाशयेन खीये भारतीयदर्शनसाहित्येतिहासे (HIP Vol. II Page 103) ग्रन्थस्यास्य भाष्यदीपिका इति नाम निर्दिश्यते । 
अस्य कर्ता नृसिम्हाश्रमिणां विद्यागुरुर्जगन्नाथाश्रमः । दक्षिणदेशीयोऽयं पञ्चदशशतकापरार्धादारब्धे काले (1475 - 1573 A.D.) आसीदिति ज्ञायते ।।
(च) सूत्रभाष्यव्याख्या - विद्याश्रीः । 
अनादिरनन्तोऽयं सूत्रभाष्यव्याख्यात्मकः ग्रन्थ अमुद्रितः मद्रासराजकीयहस्तलिखितपुस्तकालये (R. 3783 MGOML) लभ्यते । अस्य कर्ता ज्ञानधनशिष्यो ज्ञानोत्तमभट्टारक इति ग्रन्थादस्मात् ज्ञायते। कोऽयं ज्ञानोत्तमः? किं नैष्कर्म्यसिद्धिव्याख्याता मङ्गलग्रामवासी चोलदेशीयः ज्ञानोत्तमः? आहोस्वित् गौडेश्वराचार्यापराभिधः गौडदेशवासी न्यायसुधाकारः चित्सुखाचार्यगुरुर्ज्ञानोत्तम इति विशिष्य निर्णेतुं न शक्यते । दासगुप्तेन तु ज्ञानोत्तममिश्र इति नाम निर्दिश्यते । मिश्रान्त नाम निर्दिष्टं नैष्कर्म्यसिद्धिव्याख्यातुरेव । अमुद्रितग्रन्थे तु भट्टारकान्तनाम दृश्यते । भट्टारकान्तनामश्रवणात् आनन्दबोधभट्टारकादिवत् अयमपि द्वादशशतकीयो भवितुमर्हतीति ज्ञायते । सर्वथा द्वादशशतकात्पूर्ववतीर्ति निश्चयः ।।
(छ) सूत्रभाष्यटिप्पणी - प्रदीपः 
शारीरकभाष्यटिष्पणीत्यपरनामायं ग्रन्थः कल्कत्तासंस्कृतग्रन्थमालायां मुद्रितः । अस्य कर्ता महामहोपाध्यायबिरुदभूषितः पालक्काडन्तर्गतनूरणिग्रामवासी सुब्रह्मण्योपाध्यायपुत्र अनन्तकृष्णशास्त्री विंशतिशतकीयः (1886 - 1965 A.D.) न केवलं जन्मना शतभूषणीकारः (नूरणिकारः) परन्तु ग्रन्थकरणेनापि शतभूषणीकारः । अदसीया अन्ये ग्रन्थास्समुचितस्थानेषु प्रतिपादिताः ।। 
(ज) सूत्रभाष्यार्थप्रदीपिका - 
ग्रन्थोऽयं चौखाम्बामुद्रणालये मुद्रितः । अस्य कर्ता गोविन्दानन्दगिरिरिति परं ज्ञायते ।। बेल्लङ्कोण्डा - रामरायकविकृतो विमर्शाख्यः सूत्रभाष्यार्थसंग्राहक ग्रन्थो विद्यते । 
(झ) सूत्रभाष्यटीका - सुबोधिनी ।
चतुस्सूत्र्यन्तोऽयं ग्रन्थ अद्वैतसभापत्रिकायां मुद्रितः । अस्य कर्ता शिवरामतीर्थशिष्य रामचन्द्रतीर्थ इति ज्ञायते ।
(ञ) सूत्रभाष्यव्याख्या - सुबोधिनी ।
अमुद्रितोऽयं भाष्यव्याख्यात्मकः ग्रन्थः मद्रासराजकीयहस्तलिखितपुस्तकालये (R. 2201 MGOML) अडयारपुस्कालये च लभ्यते । अस्य कर्ता पञ्चीकरणवार्तिकव्याख्यायाः विवरणदीपिकायाः ( 7306 TSML) कर्ता शिवरामानन्दतीर्थशिष्यः शिवनारायणानन्दतीर्थ इति ज्ञायते । यद्यस्य गुरुश्शिवरामान्द अद्वैतसिद्धिव्याख्याकारेण ब्रह्मानन्दसरस्वत्या निर्दिष्टात् शिवरामाख्यवर्णिन रत्नप्रभाकर्तुः प्राचार्यात् शिवरामान्नान्यस्तर्हि सुबोधिनीकारोऽयं शिवनारायणानन्दतीर्थः गोपालसरस्वत्यास्सतीर्थ्यष्षोडशशतकादिमे भागे (1550 A.D.) आसीदिति ज्ञायते । परन्त्वस्मिन् प्रबलतरप्रमाणान्वेषणेऽपि ग्रन्थात् तन्न किमपि लभ्यते ।। 
(ट) सूत्रभाष्यवार्तिकम् - शारीरकमीमांसाभाष्यवार्तिकम् ।
गद्यमयोऽयं वार्तिकग्रन्थः भाष्यस्य व्याख्यानरूपः । वार्तिकशब्देन व्यवहारेऽपि भाष्ये दुरुक्तचिन्तनं भाष्यखण्डनं वा न दृश्यते । मुद्रितश्चायं ग्रन्थ आशुतोषमुखर्जीग्रन्थमालायां कलकत्तानगरे । अस्य कर्ता नारायणतीर्थापराभिधः नारायणसरस्वती । गोविन्दानन्दसरस्वत्याश्शिष्यः, लघुचन्द्रिकाकारस्य गौडब्रह्मानन्दसरस्वत्याः गुरू रत्नप्रभाकारस्य रामानन्दसरस्वत्यास्सतीर्थ्यस्सप्तदशशतकीयः (1600 - 1700 A.D.) इति ज्ञायते । अनेनान्येऽपि अद्वैतामृतकन्दादिग्रन्थाः प्रणीता ये च प्रकरणग्रन्थप्रस्तावे निरूपिताः ।। 
(ठ) सूत्रभाष्यवार्तिकम् - शारीरकमीमांसाभाष्यवार्तिकम् । 
पद्यमयोऽयं भाष्यवार्तिकग्रन्थस्सर्वत्र सुरेश्वरवार्तिकं प्रमाणयन् , भामतीकाराणां अवच्छेदवादं, ईश्वरभावापत्तिरेव मुक्तिरिति, अनेकजीववादे सर्वमुक्तिरिति विवरणकारमतञ्च, अज्ञाननिवृत्तिरेव जीवानां मुक्तिरिति प्रकटार्थकारमतञ्च प्रतिपादयति । सिद्धान्तलेशसंग्रहादिषु प्रतिपादितान् प्रक्रियाविशेषान् भाष्यारोपणप्रयत्नं कुर्वन्नयं ग्रन्थ आशुतोषग्रन्थमालायां कलकत्तानगरे मुद्रितः । 
अस्य कर्ता काञ्चीमण्डलान्तर्गतवेदपुरीवासी अभिनवद्राविडाचार्यविरुदभूषितः श्रीधारानन्दसरस्वत्याः प्राप्तदीक्षः गौडब्रह्मानन्दस्य शिष्यः महादेवकैलासेशगुरुः सप्तदशशतकीय (1600 - 1700 A.D.) इति निश्चीयते । स्वग्रन्थे " गौणमिथ्य़ात्मनोऽसत्वे " इत्यादि पद्यं शङ्कराचार्यनिर्दिष्टं द्रविडाचार्यकर्तृकमिति प्रतिपादयन्नयं ग्रन्थकारः बालकृष्णानन्दसरस्ती - श्वेतगिरिप्रारब्धां श्रीधरान्ताञ्च अद्वैताचार्यपरम्परां निर्दिशति । अस्य व्याख्यापि मूलकृतैव कृता मुद्रिता च ।  
(ड) सूत्रभष्यसंग्रहः - शारीरकभाष्यसंग्रहः ।
अमुद्रितोऽयं शाङ्करभाष्यार्थसंग्राहकः ग्रन्थ उज्जैनसूच्यां दृश्यते । अस्य कर्ता भास्करशर्माभिध आधुनिक विंशतिशतकीय इति ज्ञायते । 
(ढ) सूत्रभाष्यसिद्धान्तसंग्रहः- ब्रह्मसूत्रसिद्धान्तसंग्रहः । 
शाङ्करभाष्यस्य प्रत्यधिकरणसारार्थं प्रतिपादयति । ग्रन्थकृता प्रथमद्रितीयतृतीयाध्यायानां यथाक्रमं विराटू-विश्व-बीजतुर्याख्या समाख्याता । ग्रन्थोऽयं अडयारग्रन्थालयपुस्तिकामालायां मुद्रितः । अस्य कर्ता रामचन्द्रेन्द्र इत्यपरनामा उपनिषद्ब्रह्मेन्द्रः। प्रथमावासुदेवेन्द्रप्रशिष्यः द्वितीयवासुदेवेन्द्रशिष्यः रामचन्द्रेन्द्रसतीर्थ्यः कृष्णानन्दगुरुश्चायं उपनिषद्ब्रह्नेन्द्रः अष्टादशशतकापरार्धारब्ध एकोनविम्शतिशतकपूर्वान्ते च काले (1760 - 1850 A.D.) आसीत् । अदसीया अन्ये ग्रन्था अन्यत्र प्रतिपादिताः ।।
(ण) सूत्रभाष्यसिद्धान्तसंग्रहः -
ग्रन्थोऽयं चौखाम्बामुद्रणालये मुद्रितः । अस्य कर्ता विश्वेश्वरानन्दशिष्यः ब्रह्मानन्द इति परं ज्ञायते ।।
(त) ब्रह्मसूत्रभाष्यार्थरत्नमाला - 
पद्यमयोऽयं भाष्यार्थसंग्राहकः ग्रन्थ आनन्दाश्रममुद्रणालये मुद्रितः । अस्य कर्ता सुब्रह्मण्यसूरिः मिण्डिसोमयाजिवंशजः शोणाद्रिसूरिणः वेङ्कटपण्डितस्य च शिष्यस्सर्वेश्वरसूरिणः पुत्रश्चेति परं ज्ञायते ।। 
(थ) सूत्रभाष्यगाम्भीर्यनिर्णयः - 
अनुभाष्यगाम्भीर्यमित्यपरनामायं ग्रन्थः सूत्रभाष्यसारभूतश्शाङ्करभाष्योपरि कथितानां दूषणानां खण्डनपरः । भागवतादिपुराणेभ्य अद्वैतमतस्य श्रैष्ठ्यत्वप्रतिपादानाय बह्व्यः युक्त्यः प्रतिपादिताश्च । ग्रन्थोऽयमानन्दाश्रममुद्रणालये मद्रासनगरे मुद्रितश्च । अस्य कर्ता महामहोपाध्यायबिरुदभूषितः अश्वत्थनारायणशास्त्रिपौत्रः रामशङ्करशास्त्रिपुत्रः चोलदेशान्तर्गतशाहजग्रामवासी (तिरुविशनल्लूर) श्रीवत्सगोत्रोत्पन्नः शिवरामशास्त्रिशिष्यः रामसुब्रह्मण्यशास्त्री चन्द्रिकाचार्यभिक्षुसामयिक एकोनविंशतिशतकीय (1850 - 1920 A.D.) इति ज्ञायते । अदसीया अन्येऽपि बहवः ग्रन्था अन्यत्रोपवर्णिताः ।। 
(i) सूत्रभाष्यगाम्भीर्यार्थनिर्णयखण्डनम् - 
रामसुब्रह्मण्यशास्त्रिकृतस्य भाष्यगाम्भीर्यार्थनिर्णयाख्यग्रन्थस्य खण्डनपरः तत्तद्दर्शितपद्धत्यवहेलनात्मक अद्वैतग्रन्थः वाणीविलासमुद्रणालये मुद्रितः । अस्य कर्ता स्वामिनाथशास्त्रिपौत्रः नृसिम्हशास्त्रिपुत्रः शाण्डिल्यगोत्रजः, बन्दरपुरवासी सच्चिदानन्दसरस्वतीशिष्यः गौरीनाथशास्त्री एकोनविंशतिशतकीयः (1850 - 1920 A.D.) इति ज्ञायते ।।
(ii) सूत्रभाष्यगाम्भीर्यार्थनिर्णयमण्डनम् - 
गौरीनाथशास्त्रिकृतस्य भाष्यगाम्भीर्यनिर्णयखण्डनग्रन्थस्य खण्डनात्मकः रामसुब्रह्मण्यशास्त्रिकृतभाष्यगाम्भीर्यनिर्णयग्रन्थस्य मण्डनात्मकोऽयं ग्रन्थः ब्रह्मवादिनीमुद्रणालये मद्रासनगरे मुद्रितः। अस्य कर्ता रामसुब्रह्मण्यशास्त्रिशिष्यः कृष्णशास्त्रिनारायणशास्त्रिसतीर्थ्यः वेङ्कटराघवशास्त्री एकोनविंशतिशतकापरार्धादारब्धे काले उवासेति (1850 - 1920 A.D.) ज्ञायते ।। 
(द) ब्रह्मसूत्रभाष्यव्याख्या - शारीरकन्यायमणिमाला - 
गन्थोऽयं दासगुप्तमहाशयेन (HIP. Vol II Page 82) उद्धृतः । अस्य कर्ता पञ्चपादिकाविवरणकारस्य प्रकाशात्मनः गुरुर्द्वादशशतकीय (1200 A.D)  अनन्यानुभव इति ज्ञायते । (1) गोविन्दविष्णुकृता शाङ्करभाष्यटिप्पणी च मुद्रिता (Bombay, 1867). (2) अनन्तानन्दगिरिकृतः सारसंग्रहनामा भाष्यार्थसंग्राहकः ग्रन्थः मुद्रितः (Benaras, 1900). 

 % ब्रह्मसूत्राणि 
 
 (ध) सूत्रभाष्यव्याख्या - पञ्चपादिका - 
 शाङ्करभाष्यस्य व्याख्यासु बह्वीषु विषयगाम्भीर्यदृष्ट्या अद्वैतसिद्धान्तप्रतिपादनदृष्ट्या महत्त्वेन कालेन च प्राथमिकीयं व्याख्या अद्वैतवेदान्तशास्त्रे विशिष्टप्रस्थानस्य मूलम् । एतस्य ग्रन्थस्य व्याख्योपव्यामुखेन बहवः ग्रन्थास्समायाताः । ग्रन्थोऽयं चतुस्सूत्र्यन्त एवोपलभ्यते । नवभिर्वर्णकैः पूणोंऽयं ग्रन्थः विजयनगरसंस्कृतग्रन्थमालायां निर्णयसागरमुद्रणालये वाणीविलासमुद्रणालये कल्कत्तासंस्कृतग्रन्थमालायां च मुद्रितः । ग्रन्थस्यास्य नाम्नः पर्यालोचनायां पञ्चभिः पादैर्भाव्यम् । परन्तु प्रथमेऽध्यये प्रथमे पादे चतुस्सूत्री एव सर्वत्रोपलभ्यते । ततोऽप्यधिकं पद्मपादेन लिखितः परन्तु नष्ट इति साम्प्रदायिका वदन्तिः । 
अस्य कर्ता पद्मपादाचार्यः शङ्कराचार्यशिष्येष्वन्यतमः । आश्रमस्वीकारात् पूर्वं सनन्दनापरनामा विमलनामकस्य विप्रस्य पुत्रः । नद्या उत्तरतीरवर्ती अयं गुरुणाऽहूतः सत्वरमागमनाय नदीजले पद्मेषु च पादौ निक्षिष्याजगाम । तादृशीं गुरुभक्तिं दृष्ट्वा शङ्करभगवान् तस्य पद्मपादे इति नाम व्यतानीदिति कथापि प्रसिद्धा। शङ्करभगवच्छिष्योऽयं (800 A.D.) अष्टमशतके आसीदिति निश्चयः कुप्पुस्वामिशास्त्रिणस्तु ब्रह्मसिद्धिभूमिकायां (625 - 705 A.D.) इति वदन्ति। पद्मपादाचार्यमधिकृत्य विस्तरत अन्यत्रोपवर्णितम् ।
(i) पञ्चपादिकाव्याख्या - कण्ठीरवः -
ग्रन्थोऽयं पञ्जाबसूच्यां 666 दृश्यते । अस्य कर्ता विज्ञानवासोयतिः । विज्ञातवरयतिरितिनामान्तरमपि श्रूयते । मद्रासराजकीयहस्तलिखितपुस्तकालये च ( R. 5387 MGOML) लभ्यते ।। 
(ii) पञ्चपादिकाव्याख्या - पदयोजनिका - 
अमुद्रितोऽयमपूर्णग्रन्थः महीशूरपुस्तकालये ( आ. 237) विद्यते । अस्य कर्ता अद्वैतवेदान्तपरिभाषाकारः नृसिम्हाश्रमिप्रशिष्यः वेङ्कटनाथशिष्यः वेङ्कटनाथपुत्रः रामकृष्णदीक्षितस्य पिता त्रिवेदीनारायणदीक्षितस्य भ्राता कौण्डिन्यगोत्रजः ऋग्वेदाध्यायी चोलदेशीयकण्डरमाणिक्कग्रामवासी सप्तदशशतकीयः (1500 - 1650 A.D.) धर्मराजाध्वरीति ज्ञायते ।। 
(iii) पञ्चपादिकाव्याक्या - प्रबोधपरिशोधनी -
अमुद्रितोऽयं पञ्चापदिकाव्याख्यात्मकः ग्रन्थः मद्रासराजकीयहस्तलिखित ग्रन्थालये (R. 3225 MGOML) लभ्यते ।। 
अस्य कर्ता नृसिम्हस्वरूपशिष्य आत्मस्वरूपः । अनेन द्वादशशतकीयस्या नन्दानुभवीयस्य पदार्थतत्वनिर्णयस्य व्याख्या (R. 4219 MGOML) कृता । प्रबोधपरिशोधिन्यां विवरणकारः आचार्यसुन्दरपाण्डयः, गौडाचार्यः, प्रभाकरः, भाट्टाश्च निर्दिष्टः। (Page 31 - 37) पर्यन्तं अनिर्वचनीयख्यातिस्सम्यङ्निरूपिता । न तत्र खण्डनखण्डखाद्यादिः ग्रन्थ उल्लिखितः । पदार्थतत्वनिर्णयटीकायां उदयनः न्यायसारः, भूषणम् , ब्रह्मसिद्धिः, प्रमाणमाला, इष्टसिद्धिश्च प्रमाणीकृताः प्रमाणमालाकारस्तु (1050 -1150 A.D.) इति द्वादशशतकीयो भवति । तस्मादस्मादर्वाचीन इति परं ज्ञायते । चित्सुखाचार्यः आनन्दगिरिर्वा नैव निर्दिष्टौ। तस्मात्  तयोः पूर्वतनः, अनुभूतिस्वरूपादिवत् स्वरूपान्तनामा चायं द्वादशशतकादारभ्य त्रयोदशशतकपूर्वार्धावधिके काले उवासेति परं ज्ञायते।।
(iv)  पञ्चपादिकाव्याख्या - वक्तव्यप्रकाशिका -
पञ्चपादिकाव्याख्यात्मकोऽयं ग्रन्थः शृङ्गगिरिसूच्यां (56 A) दृश्यते । अस्य कर्ता तत्वशुद्धिव्याख्याता चित्सुखाचार्यविज्ञानात्मगुरोः न्यायसुधादिग्रन्थप्रणेतुः ज्ञानोत्तमस्य शिष्य उत्तमज्ञयतिः । अस्य कालः (910 - 953 A.D.) इति शृङ्गगिरिगुरुपरम्परासूच्यां दृश्यते । श्रीकण्ठशास्त्रिणा तु स्वीये प्रबन्धे (IHQ Vol. XIV) ज्ञानघनकाल एव प्रकाशात्मकाल इति प्रतिपाद्यते । ज्ञानघनश्च ज्ञानोत्तमगुरुरिति ज्ञानघनप्रशिष्यस्य ज्ञानोत्तमस्य शिष्यस्य उत्तमज्ञयतेः कालः दशमशतकान्तादारब्धः एकादशशतकपूर्वार्धावधिकः (958 - 1038 A.D.) इति सिध्यति । मतान्तरे (1100 - 1200 A.D.) इति च । अनेन कृता तत्वशुद्धिव्याख्या प्रकरणग्रन्थप्रस्तावे लिखितेति नेह प्रतन्यते । 
(v) पञ्चपादिकाव्याख्या - वेदान्तरत्नकोशः -
पञ्चपादिकाव्याख्यात्मकोऽयं गन्थ अमुद्रित अपूर्णश्च मदरासराजकीयहस्तलिखितपुस्तकालये (R. 2626 MGOML,) तञ्जपुर सरस्वतीमहालये, महीशूर राजकीयपुस्तकालये च लभ्यते । अस्य कर्ता अद्वैतदीपिकातत्वविवेकादिकर्ता जगन्नाथाश्रमि - गीर्वाणेन्द्रसरस्वतीशिष्यः, धर्मराजाध्वरिगुरोर्वेङ्कटनाथस्य, भट्टोजिदीक्षितकनीयसः भ्रातू रङ्गोजीभट्टस्य च गुरुः, काञ्चीमण्डलान्तर्गतपुरुषोत्तमपुरवासी सच्चिदानन्दशास्त्रीत्यपरनामा नृसिम्हाश्रमीति ज्ञायते । अद्वैतदीपिकानेन "वेदवियद्रसेन्दुगुणिते" (1603 - 1547 A.D.) काले रचितेति कालोऽस्य (1500 - 1600 A.D.) इति निश्चीयते । अदसीयाः प्रकरणग्रन्था अन्यत्रोपपादिताः ।। 
(vi) पञ्चपादिकाव्याख्या -
अमुद्रितोऽयं ग्रन्थः मद्रासराजकीयहस्तलिखितपुस्तकालये (R. 4336 MGOML) लभ्यते । अस्य कर्ता ज्ञानोत्तमशिष्यः, ज्ञानघनप्रशिष्यः चित्सुखाचार्यसतीर्थ्यः परमानन्दमस्करीत्यपरनामा विज्ञानात्मा इति ज्ञायते । अस्य कालः द्वादशशतकम् (1100 - 1200 A.D.)। शृङ्गगिरिसूच्यांतु ज्ञानोत्तमकालः दशमशतकमिति ज्ञायते । " तात्पर्यद्योतनिका " इति नामेति केचित् ।। 
(vii) पञ्चपादिकाव्याख्या -
अमुद्रितोऽयं ग्रन्थः लन्दननगरस्थ भारतकार्यालयपुस्तकालये ( 2261 D. C. IOL London Vol. JV) दृश्यते । ग्रन्थेऽस्मिन् " प्रकाशात्मयतिष्टीकां विवृतिं कृतवान् पराम् । आनन्दपूर्णमुनिना भावस्तस्याः प्रकाश्यते" ।। इति दर्शनात् पञ्चपादिकाविवरणानुसारी स्वतन्त्रव्याख्यात्मकः ग्रन्थ इति ज्ञायते । अस्य कर्ता खण्डनखण्डखाद्यव्याख्यात विद्यासागर इति प्रसिद्धः, श्वेतगिर्यभयानन्दयोः शिष्यः पुरुषोत्तमानन्दसरस्तीगुरुरानन्दपूर्ण इति ज्ञायते । अनेन कृतायां बृहदारण्यकोपनिषद्व्याख्यायां न्यायकल्पलतिकायां प्रतिलिपिकालः (1499 सं.1443 A.D.) इति (R. 5283) ज्ञायते । तस्मात्तत्कालात् पूर्वतन इति तु निश्चयः। राघवमहोदयास्तु स्वीये प्रबन्धे ( AOR Madras Vol IV Part I) कामदेवभूपालकालिकोऽयं विद्यासागर (1350 A.D.) काल आसीदिति प्रतिपादयन्ति ।।
(viii) पञ्चपादिकाव्याख्या - तत्वप्रदीपिका - 
अज्ञातकर्तृनामधेयोऽयं ग्रन्थः पञ्चाबसूच्यां 668 लभ्यते । 
(ix) पञ्चपादिकाव्याख्या - पञ्चपादिकाविवरणम् - 
वेदान्तशास्त्रे प्रस्थानद्वयस्य प्रचारे प्रकाशात्मनः पञ्चपादिकाविवरणम् , भामती च कारणम् । अनयोः प्रस्थानयोः भेदस्य कारणं एकेश्वरवादाविद्याश्रयादिवादयोरङ्ीकारानङ्गीकार एवेति च प्रसिद्धम् । पञ्चपादिकाव्याख्यात्मकोऽयं ग्रन्थः विजयनगरसंस्कृतग्रन्थमालायां कलकत्तासंस्कृतग्रन्थमालायाञ्च मुद्रितः ।
अस्य कर्ता स्वयम्प्रकाशानुभवापरनामा अनन्यानुभवशिष्यः प्रकाशात्मा तत्वशुद्धिकारस्य ज्ञानघनस्य सामयिकः दशमशतकीय (1000 A.D.) इति निश्चीयते । अस्य गुरुणा अनन्यानुभवेन "आत्मतत्व" मिति ग्रन्थः कृतः यश्च ज्ञानघनेन तत्वशुद्धौ उद्धृत इति श्रीकण्ठशास्त्री (IHQ Vol. XIV) प्रतिपादयति। अनन्यानुभवेन शारीरकन्यायमणिमालाख्यः ग्रन्थः कृता इति दासगुप्तः (HIP Vol. II Page 82) प्रतिपादयति । प्रकाशात्मा त्रयोदशशतकीय इति, श्रीकण्ठशास्त्री दशमैकादशशतकान्तरालवर्तीति च विभिन्नाः मतभेदाः दृश्यन्ते । दासगुप्तस्तु (1200 - 1300 A.D.) इति । श्रीकण्ठशास्त्री तु (950 - 1050 A.D.) इति ।। । 
1. पञ्चपादिकाविवरणोज्जीविनी - 
अमुद्रितोऽयं ग्रन्थः मद्रासराजकीयहस्तलिखितपुस्तकालये (R. 592) लभ्यते । वादिराजाख्येन द्वैतमतावलम्बिना पञ्चपादिकाविवरणव्याख्या- " विवरणविविरण " - निर्मणव्याजेन विवरणोपरि उद्भावितानां दूषणानां खण्डनपरोऽयं ग्रन्थः । अस्य कर्ता काश्यपगोत्रजः बह्वृचशाखाध्यायी, चर्कूरिवंश्यः, यज्ञे श्वरपुत्रः, कोण्डुभट्टोपाध्यायगङ्गाम्बिकयोः पुत्रः तिरुमलैदीक्षितकनीयान् भ्राता, यज्ञेश्वरकृष्णाश्रमयोः शिष्यः यज्ञेश्वरदीक्षित इति एतद्गन्थपरिशीलनात् ज्ञायते । " नृसिम्हाश्रमियोगीन्द्र ग्रन्थ शाणनिकषातः । क्षुद्रग्रन्थानुपेक्ष्याहं करोमि विमलां धियम् ।" इति पञ्चपादिकाविवरणोजीविन्यां  दर्शनात् नृसिम्हाश्रमादर्वाचीन इति (1600 - 1700 A.D.) ज्ञायते । दक्षिणदेशवासी शाहेन्द्रकालिकोऽयं ईश्वरगीताया अपि व्याख्याता शास्त्रदीपिकाव्याख्याप्रभामण्डलकारश्चेति निश्चीयते ।। 
2. पञ्चपादिकाविवरणव्याख्या - ऋजुविवरणम् 
पञ्चपादिकाविवरणव्याख्यात्मकोऽयं ग्रन्थ कल्कत्तानगरे मुद्रितः । अमुद्रितादर्शग्रन्थास्तु मद्रासराजकीयहस्तलिखितपुस्तकालये (R. 2957 MGOML) अडयारपुस्तकालये तिरुवनन्तपुरं पुस्तकालये च लभ्यते । अस्य कर्ता त्रिणेत्रपौत्रः जनार्दनपुत्रः हरिहरबुक्करायपालितस्य विजयनगरवासिनः तर्कभाषाव्याख्याकारस्य चतुर्दशशतकान्तवासिनः चेन्नुभट्टस्य (चिन्नभट्टस्य) पिता स्वामीन्द्रपूर्णशिष्यः सर्वशास्त्रविशारदः चतुर्दशशतकपूर्वार्धवासी (1280 - 1350 A.D.) सर्वज्ञविष्णुभट्टोपाध्याय इति ज्ञायते । 
आनन्दगिरेरेव सन्यास्वीकारात्पूर्वं जनार्दन इति नामेति प्रसिद्धम् । विष्णुभट्टोपाध्यायपितुर्जनार्दनस्य जनार्दनापरनामकानन्दगिरेश्च यदि ऐक्यं तर्हि आनन्दगिरिपुत्रः विष्णुभट्टोपाध्याय इति सिध्यतीति तर्कसंग्रहभूमिकायां त्रिपाठीमहाशयः । एवञ्चायं गुजरातप्रान्तज इति सिध्यति । आन्ध्रदेशाभिजन इति तु साम्प्रदायिकाः ।
विष्णुभट्टशिष्येण सर्वदर्शनसंग्रहकर्त्रा माधवाचार्येण शाङ्करदर्शनप्रतिपाद नावसरे ऋजुविवरणं निर्दिष्टम् । एवञ्चायं सायणमाधवगुरुरिति "इण्डियन् अण्टिक्वैरि " पत्रिकायां (1916, Page 2) प्रतिपादितम् ।
श्रीकण्ठशास्त्रिणस्तु भारतीयैतिहासिकत्रैमासिकपत्रिकाया चतुर्दशतमे भागे (IHQ. Vol. XIV) एवं वर्णयन्ति। 
विद्याशङ्करश्शार्ङ्गपाणिपुत्रः बिल्वारण्यजश्च । अस्य आश्रमस्वीकारात्पूर्वं सर्वज्ञविष्णुरिति नाम । स च  कामकोटिपीठाधीशात्  चन्द्रशेखरसरस्वत्याः प्राप्ताश्रमः कामकोटिपीठाधीशश्चासीत् । अत्र प्रमाणम् - 
"विल्वारण्यज शार्ङ्गपाणितनयः सर्वज्ञविष्णुश्श्रयन् 
सन्यासं गुरुचन्द्रशेखरमुनेरास्थाय पीठं गुरोः ।
योगेशस्य च चक्रराजवसतेः देव्याश्च सक्तोऽर्चने 
श्रीमन्माधवबुक्कभूपतियति (भारतियति ) प्रेष्ठैः- र्महिष्ठैर्वृतः ।। "
इति कामकोटिपरम्परा । एवञ्च शार्ङ्गपाणि - जनार्दनौ यद्यभिन्नौ तर्हि ऋजुविवरणकारस्सर्वज्ञविष्णुभट्टः विद्याशङ्करान्न भिन्न इति ।। 
विद्यातीर्थस्यैव विद्याशङ्कर इति नामान्तरमपीति प्रसिद्धिः। एवञ्च विद्यातीर्थविद्याशङ्करः सर्वज्ञविष्णुभट्टविद्याशङ्कराद्भिन्नो वा उत्ताभिन्न इत्यस्माकं संशयउदेति ।।
क. ऋृजुविवरणव्याख्या - त्रय्यन्तभावदीपिका -
अमुद्रितोऽयं ऋजुविवरणव्याख्यात्मको ग्रन्थः मद्रासराजकीयहस्तलिखितपुस्तकालये (R. 2956 MGOML) लभ्यते । अस्य कर्ता वैय्यासिकन्यायमालादृग्दृश्यविवेककारस्य भारतीतीर्थस्य शिष्यः रामानन्दतीर्थः चतुर्दशशतकीयः (1300 - 1400 A.D.) इति ज्ञायते।। 
ख. अज्ञातकर्तृनामधेया - ऋजुविवरणव्याख्या । ग्रन्थोऽयं अनन्तशयनपुस्तकालये लभ्यते ।। 
3. पञ्चपादिकाविवरणव्याख्या - टीकारत्नम् -
अमुद्रितोऽयं पञ्चपादिकाविवरणव्याख्याग्रन्थः मद्रासराजकीयहस्तलिखितपुस्तकालये (R. 3406 MGOML) लभ्यते । अस्य कर्ता विद्यासागर इति प्रसिद्धः श्वेतगिर्दभयानन्दयोश्शिष्यः पुरुषोत्तमसरस्वतीगुरुः चतुर्दशशतकीयः (1350 A.D.) आनन्दपूर्ण इति ज्ञायते।। 
4. पञ्चपादिकाविवरणव्याख्या - तत्वदीपनम् - 
पञ्चपादिकाविवरणव्याख्यात्मकोऽयं ग्रन्थश्चतुस्सूत्र्यन्तः वाराणस्यां, कल्कत्तायां विजयनगरसंस्कृतग्रन्थमालायाञ्च मुद्रितः । अस्य कर्ता अखण्डानुभूतिआनन्दगिर्योश्शिष्यः विष्णुभट्टोपाध्यायसतीर्थ्यः त्रयोदशचतुर्दशशतकमध्यवासी (1250 - 1350 A. D.) अखण्डानन्दमुनिरिति ज्ञायते। अनेन आनन्दगिरिर्बोधपृथ्वीधरशब्देन निर्दिष्टः। महामहोपाध्याय अनन्तकृष्णशास्त्रिणः स्वसम्पादगितो ब्रह्मसूत्रभाष्यग्रन्थे (CSSI) तत्वदीपनकारस्य अखण्डानन्दमुनेः भामतीव्याख्या ऋजुुप्रकाशिकाकर्तुः अखण्डानन्दसरस्वत्याश्च ऐक्यं सम्भावयन्ति। ऋजुप्रकाशिकाकर्तुः गुरोस्स्वयम्प्रकाशाख्यता गौणीति आनन्दगिरिरेव नमस्कृत इति च प्रतिपादयन्ति। 
परन्तु इम्मिडिजगदेकरायकालिकात् षोडशसप्तदशशतकवर्तिनः ऋजुप्रकाशिकाकारात् अखण्डानन्दसरस्वत्या तत्वदीपनकारोऽयं अखण्डानन्दमुनिर्भिन्न एवेति प्रतीयते । केशवमिक्षकृतायाः तर्कभाषायाः व्याख्या गोवर्धनेन कृता । गोवर्धनकृतायाः व्याख्यायाः व्याख्या ऋजुप्रकाशिकाकारेण अखण्डानन्देन रचिता । गोवर्धनकालस्तु (1560 A.D.) इति प्रसिद्धम् । तस्माच्च ऋृजुप्रकाशिकाकारात् तत्वदीपनकारः भिन्न इति प्रतीयते । ऋजुप्रकाशिकाकारस्तत्वानुसन्धानकारस्य महादेवसरस्वत्यास्सतीर्थ्यः । महादेवसरस्वती च षोडशशतकापरार्धादारब्घे समये आसीत् । तस्माच्च तत्वदीपनकारः ऋजुप्रकाशिकाकारात् भिन्न इति प्रतीयते । रङ्गनाथापरनाम्ना ऋजुप्रकाशिकाकारात् अखण्डानन्दसरस्वत्या नृसिम्हाश्रमीयस्य अद्वैतरत्नकोशस्य व्याख्या रत्नकोशप्रकाशिका (भावप्रकाशिका) नाम्नी लिखिता या च  मैसूरपुस्तकालये लभ्यते । नृसिम्हाश्रमिणस्तु कालः  1547 A.D. इति प्रसिद्धम् । तस्मादपि कारणात् तत्वदीपनकारः ऋजुप्रकाशिकाकारात् भिन्न इति प्रतिभाति ।। 
(A) पञ्चपादिकाविवरणतत्वदीपनसारः 
अमुद्रितोऽयं विवरणतत्वदीपनसंग्राहकः ग्रन्थः बरोडासूच्यां (1955) दृश्यते । अस्य कर्ता राघवानन्दयोगिशिष्यः केरलवर्मसामयिकः विनायक इति ज्ञायते । यद्ययं विनायकगुरू राघवानन्दः परमार्थसारव्याख्याता स्यात् तर्हि विनायकस्यापि कालः (1600 - 1700 A. D.) सप्तदशशतकमिति निर्णेतुं शक्यते । नान्यदत्र प्रमाणं लभ्यते ।। 
5. पञ्चपादिकाविवरणव्याख्या - भावद्योतनिका 
" तात्पर्यदीपिका " इत्यपरनामायं ग्रन्थ अमुद्रितः मद्रासराजकीयहस्तलिखितपुस्तकालये (R. 4305) लभ्यते । तत्रैव मुद्रितश्च । अस्य कर्ता ज्ञानघनप्रशिष्यः न्यायसुधाकारस्य ज्ञानोत्तमस्य शिष्यः विज्ञानात्मनस्सतीर्थ्यः सुखप्रकाशशुद्धानन्दयोर्गुरुः अमलानन्दानन्दगिर्योः प्राचार्यः द्वादशशतकीयः चित्सुखाचार्य इति ज्ञायते । 
6. पञ्चपादिकाविवरणभावप्रकाशिका
प्रथमवर्णकमात्रोऽयं ग्रन्थः पञ्जाबसूच्यां (900) दृश्यते । अस्य कर्ता परिव्राजकाचार्य इति निर्दिष्टः । नृसिम्हाश्रमिकृता भावप्रकाशिका स्यादिति संशयः ।। 
7. पञ्चपादिकाविवरणव्याख्या - भावप्रकाशिका 
मुद्रितोऽयं ग्रन्थः मद्रासराजकीयहस्तलिखितपुस्तकालये । अस्य कर्ता अद्वैतदीपिकातत्वविवेकादिग्रन्थकर्ता जगन्नाथाश्रमगीर्वाणेन्द्रसरस्वतीशिष्यः नारायणाश्रमिवेङ्कटनाथरङ्गोजिभट्टगुरुः काञ्चीमण्डलान्तर्गतपुरुषोत्तमपुरवासी सच्चिदानन्दशास्त्रीत्यपरनामा नृसिम्हाश्रमः षोडशशतकीय  (1500 - 1600 A.D.) इति ज्ञायते ।।
8. पञ्चपादिकाविवरणव्याख्या - विवरणदर्पणम्  
अपूर्ण अमुद्रितश्चायं ग्रन्थः सरस्वतीमहालये (7064 TSML) लभ्यते। अस्य कर्ता अद्वैतविद्यामुकुरकर्ता भारद्वाजगोत्रोत्पन्नः, काञ्चीपुरान्तर्गतअडयप्पलग्रामवासी प्रसिद्धतराप्पय्यदीक्षितपिता वक्षस्थलाचार्यापरनामकाचार्यदीक्षितस्य तोतारम्ब्याश्च पुत्रःबोम्मराजसभापण्डितः षोडशशतकपूर्वार्धान्तकालवासी (1500 - 1550 A.D.) रङ्गराजाध्वरीति ज्ञायते ।। 
9. पञ्चपादिकाविवरणोपन्यासः - 
पञ्चपादिकाविवरणतात्पर्यसङ्ग्राहकोऽयं ग्रन्थः चौखाम्बामुद्रणालये मुद्रितः । गद्यात्मकविवरणेन साकं अर्थसङ्गाहकश्लोका अपि विद्यन्ते । अस्य कर्ता शिवरामगोपालानन्दसरस्वत्योः प्रशिष्यः गोविन्दानन्दशिष्यः, स्वयम्प्रकाशानन्दसरस्वत्याश्च शिष्यः रत्नप्रभाकारः षोडशशतकापरार्धवासी रामानन्दसरस्वतीति ज्ञायते ।।
10. "विवरणप्रमेयसङ्ग्रहः -"
पञ्चपादिकाविवरणगतानर्थात् सङ्गृह्णन्नयं ग्रन्थः विजयनगरसंस्कृतग्रन्थमालायां मुद्रितः । सिद्धान्तलेशसंग्रहे " विवरणोपन्यासे भारतीतीर्थवचनम् " (Page 68) इति दर्शनात् अस्यैव विवरणोपन्यास इति नामान्तरं स्यादित्यूह्यते । अस्य विवरणप्रमेयसंग्रहस्य कर्ता भारद्वाजगोत्रजः, माधवाचार्यापरनामा, सङ्गमराजमहामन्त्रिणः मायणस्य श्रीमत्याश्च पुत्रः सायणभोगनाथयोः सिङ्गलायाश्च भ्राता, अद्वैतमकरन्दकारस्य लक्ष्मीधरस्य मातुलः, विजयनगराधीशबुक्कणक्षमापतिसामयिकः विद्यातीर्थभारतीतीर्थ-श्रीकण्ठाचार्य-शङ्करानन्दानां शिष्यः, विद्यातीर्थ - नृसिम्हतीर्थप्रशिष्यः, कृष्णानन्दभारती - ब्रह्मानन्दभारती - रामकृष्णानां गुरुः, अमलानन्दसामयिकः त्रयोदयचतुर्दशशतकमध्यवासी (1296 - 1386 A. D.) विद्यारण्य इति ज्ञायते ।। 
11. पञ्चपादिकाविवरणसंग्रहः - अद्वैतभूषणम् 
विवरणप्रस्थानानुसारी ग्रन्थोऽयं पञ्चपादिकाविवरणगतानर्थान् सङ्गृह्णाति । विवरणप्रमेयसंग्रह इत्यपि क्वचिदादर्शपुस्तकेष्वस्य नाम दृश्यते । मुद्रितश्चायं ग्रन्थ ब्रह्मविद्यापत्रिकायाम् । अस्य कर्ता गीर्वाणेन्द्रसरस्वतीशिष्यः विश्वाधिकसरस्वतीशिष्यः नृसिम्हाश्रमिसतीर्थ्यः चोलदेशीयतञ्जाऊरसमीपस्थगोविन्दपुरवासी नामामृतरसायनात्मबोधव्याख्याकारः बोधेन्द्रसरस्वती पञ्चदशशतकापरार्धान्तारब्धरावलवासी (1450 - 1550 A.D.) इति ज्ञायते । साम्प्रदायिकास्तु बोधेन्द्रगुरुर्गीर्वाणेन्द्रः नीलकण्ठदीक्षितगुरुर्गीर्वाणेन्द्र एवेति वदन्ति। एवञ्चास्य काल (1650 - 1750 A.D.) इति सिध्यति ।।
(A) अद्वैतभूषणव्याख्या - आनन्ददीपिका ।
बोधेन्द्रसरस्वतीकृतस्य पञ्चपादिकाविवरणसङ्ग्राहकस्य अद्वैतभूषणस्य व्याख्यात्मकोऽयं ग्नन्थः आनन्ददीपिकानामा अमुद्रितः मैसूरपुस्तकालये लभ्यते । अस्य कर्ता रामचन्द्रेन्द्रापरनामानां उपनिष्द्ब्रह्मेन्द्राणां शिष्येष्वन्यतमः, कृष्णानन्दसतीर्थ्यः वासुदेवप्रशिष्यः एकोनविंशतिशतकीय (1800 - 1900 A.D.) वासुदेवेन्द्र इति ज्ञायते ।।
12. पञ्चपादिकाविवरणव्याख्या 
ग्रन्थोऽयं दासगुप्तमहाशयेन स्वीयभारतीयदर्शनसाहित्यचरिते (HIP Vol. II Page 52) निर्दिष्टः । अस्य कर्ता संक्षेपशारीरकव्याख्यान्वयार्थप्रकाशिकाकारः कृष्णतीर्थशिष्यः, सिद्धान्ततत्वकर्तुरनन्तदेवप्रथमस्य संक्षेपशारीरकव्याख्यासुबोधिनीकारस्य पुरषोत्तममिश्रस्य च गुरुः नृसिम्हाश्रमिसामयिक रामतीर्थ इति ज्ञायते । अनेन कृतायां पञ्चीकरणविवरणव्याख्यायां " जगन्नाथाश्रमाद्या ये गुरवो मे कृपालव" इति नृसिम्हाश्रमिगुरुर्जगन्नाथाश्रमोऽपि स्वगुरुत्वेन निर्दिष्टः। अनेेन मानसोल्लासव्याख्या (1630 सं 1574 A.D.) काले कृत इति रायल आसियाटिकसोसाइटि बम्बर्इस्थात् आदर्शग्रन्थात् (120 R.A.S Bombay) ज्ञायते । तस्मादयं रामतीर्थः षोडशसप्तदशशतकान्तरालवर्तीति (1570 - 1670 A. D.) निश्चीयते ।। 
13. पञ्चपादिकाविवरणव्याख्या - 
ग्रन्थोऽयमपि दासगुप्तमहाशयेन भारतीयदर्शनेतिहासे (Hip Vol II Page 103) निर्दिष्टः । अस्य कर्ता श्रीकृष्णाख्यः यश्चाधुनिक इति निश्चीयते । 
14. पञ्चपादिकाविवरणव्याख्या - 
ग्रन्थोऽयं कुत्रास्तीति न ज्ञायते । परन्तु अस्य कर्ता नरेन्द्रपुरीति कामकोटिसूच्या अवगम्यते । 
15. विवरणतात्पर्यम् - अज्ञातकर्तृनामायं ग्रन्थ अडयारपुस्तकालये (470 AL) लभ्यते ।। 

% पञ्चपादिकाविवरणप्रस्थानम् 

(न) सूत्रभाष्यव्याख्या - भामती 
शङ्करभाष्यव्याख्यात्मकोऽयं अद्वैतवेदान्तशास्त्रे विशिष्टप्रस्थानस्य भामतीप्रस्थानाख्यस्य मूलं महान् ग्रन्थः सव्याख्यः निर्णयसागरमुद्रणालये कल्कत्तायां अन्यत्र बहुत्र स्थलेषु मुद्रितः । अस्य कर्ता ब्रह्मतत्वसमीक्षा - न्यायवार्तिकटीकादिग्रन्थकर्ता षड्दर्शिनीटीकाकृत् मण्डनमिश्रमतानुसारी तत्वचिन्तामणिप्रकाश - न्यायसूत्रोद्धारादिकर्तुर्वाचस्पतेर्भिन्न अाचार्यवाचस्पतिमिश्रः। सोऽयं वाचस्पतिमिश्रः मिश्रान्तनामसम्बन्धात् मुरारिमिश्रपार्थसारथिमिश्रादिवत् मैथिल इति ज्ञायते । केचित्तु नेपालप्रान्ते भामा नामकः कश्चन ग्रामः । नमघिवसन् वाचस्पतिमिश्रः तद्देशप्रसिध्यै तद्देशस्मरणाय वा भामतीग्रन्थं चकारेति वदन्ति । सम्प्रदायसमागता तु कथा - 
वाचस्पतिमिश्रस्य भामतीनाम्न्या श्रीमत्या सह विवाह आसीत् । विवाहकाले काचन पण्डितपरिषत्सञ्जाता । वेदान्तशास्त्रे वादस्समजनि । वेदान्तेतरविदुषां वादः युक्तयश्च प्रबला अद्वैतवादिनां युक्तय दुर्बलाश्चासन् । ततः प्रभृति अद्वैतवेदान्तयुक्तीः सबलास्सप्रमाणाश्च कर्तुं यत्नं कुर्वतः ग्रन्थस्यास्य रचयितुर्वाचस्पतिमिश्रस्यगतं नवीनं वय। ग्रन्थनिर्माणे मग्नस्यास्यापरे वयसि काचन नातिवृद्धा नष्टुप्राययौवना स्त्री समायाता । ताञ्च स्वपत्नी अक्षतयोनिं असञ्जातसन्ततिं स्वं ग्रन्थनिर्माणे मग्नञ्च ज्ञात्वा स्वरचितग्रन्थस्य भामतीति स्वपत्न्याः नाम चकारेति । अपरे तु वाचस्पतिमिश्रस्य कन्या भामतीनाम्नी यस्यास्स्मृत्यै स्वग्रन्थस्य तन्नाम चकारेति वदन्ति। परन्तु ग्रन्थात् एतादृशवादानां किमपि प्रमाणं नास्ति । 
तात्पर्यटीकायां वाचस्पतिमिश्रैः "त्रिलोचनगुरून्नीतमार्गानुगमनोन्मुखैः " इति कथनात् वाचस्पतिमिश्रगुरोर्नाम त्रिलोचनमिश्र इति ज्ञायते । केचित्तु "मार्ताण्डतिलकस्वामिमहागणपतीन्वयम्  " इति भामत्यां मङ्गलाचरणात् तिलकस्वामी अस्याचार्य इति वदन्ति। परन्तु अमलानन्देन समस्तशब्दोऽयं मार्ताण्डतिलकस्वामिशब्दः देवतापर इति व्याख्यातम् । एवञ्च तिलकस्वामी नास्याचार्य इति भवति । उदयनाचार्येण न्यायवार्तिकतात्पर्यपरिशुध्यां वाचस्पतिमिश्रः त्रिलोचनशिष्यइत्येव प्रतिपाद्यते । दासगुप्तमहाशयस्तु विद्यातरुरपि अस्य गुरुरिति (HIP Vol II 107) प्रतिपादयति । 
वाचस्पतिमिश्रेण "न्यायसूचीनिूबन्धः" वस्वङ्कवसुवत्सरेषु निर्मितः । यदिवयं वस्वङ्कवसुवत्सरान् 898 शकाब्दान् स्वीकुर्मस्तर्हि अस्य काल 898 - 976 A.D. इति दशमशतकमस्य काल इति सिध्यति । यदि विक्रमाब्दान् स्वीकुर्मस्तर्हि अस्यकालः 898 सं, 842 A.D. इति सिध्यति । सर्वथा नवमशतकवर्तीति तु निश्चयः ।। 
(न) I. भामतीव्याख्या - कल्पतरुः 
भामती व्याख्यात्मकोऽयं ग्रन्थः यत्र तत्र प्रकटार्थकारं खण्डयन् भामतीपक्षमक्षुण्णं साधयति । मुद्रितश्चायं ग्रन्थः वाणीविलासमुद्रणालये निर्णियसागर मुद्राणालये च । अस्य कर्ता दीक्षाग्रहणे आनन्दात्मप्रशिष्यः अनुभवानन्दशिष्यः विद्यायां चित्सुखाचार्यप्रशिष्यः सुखप्रकाशशिष्यश्चायं व्यासाश्रम इत्यपरनामा अमलानन्दः त्रयोदशशतकापरकालिकः 1247 - 1347 A.D. इति निश्चीयते । देवगिरिराजः कृष्णः तद्भ्राता महादेवश्च ग्रन्थेऽस्मिन् निर्दिष्टौ । 
1. कल्पतरुव्याख्या - परिमलः 
कल्पतरुव्याख्यात्मकोऽयं ग्रन्थः वाणीविलासमुद्रणालये मुद्रितः । अस्यकर्ता वक्षस्थलाचार्यापरनामकस्य आचार्यदीक्षितस्य पौत्रः, विवरणदर्पणकारस्य रङ्गराजाध्वरिणः पुत्रः, पितुरेव प्राप्तविद्यः, सच्चिदानन्दशास्त्रीत्यपरनामकात् नृसिम्हाश्रणिणः प्राप्ताद्वैतमतपक्षपातः काञ्चीमण्डलान्तर्गताडयप्पलग्रामवासी, आचार्यदीक्षितस्य भ्राता रत्नखेटश्रीनिवासदीक्षितजामाता, मङ्गलनायिकायाः भर्ता, नीलकण्ठउमामहेश्वर - चन्द्रावतंसानां पिता, चतुरधिकशतग्रन्थप्रणेतृत्वेन प्रसिद्धः षोडशशतकवासी (1520 - 1593 A.D.) अप्पय्यदीक्षितः । 
(A) परिमलसंग्रहः 
अप्पय्यदीक्षितकृतं परिमलं संगृह्णात्ययं ग्रन्थः। अमुद्रितोऽयं ग्रन्थः प्रथमाध्यायतृतीयपादान्तं मद्रासराजकीयहस्तालिखितपुस्तकालये (R. 2811) लभ्यते । अस्य कर्ता दक्षिणदेशीयः उपनिषद्बह्मेन्द्रपरम्परागतः रामचन्द्राश्रमिशिष्यः अष्टादशशतकीयः 1700 - 1800 तारकब्रह्माश्रमीति ज्ञायते। अनेनोपनिषदामपि सारस्सङ्गृहीतः ।
2.  कल्पतरुव्याख्या - आभोगः 
अप्पय्यदीक्षितीयपरिमलार्थानुसारिणीयं कल्पतरुव्याख्या आभोगनाम्नी मद्रासराजकीयहस्तलिखितपुस्तकालयग्रन्थमालायां मुद्रिता । अस्य कर्ता कोण्डुभट्टरमाम्बयोः पुत्रः महीधरवंशजः नारायणेन्द्रसरस्वतीशिष्यः सप्तदशशतकापरार्धवासी (1650 - 1750 A.D.) लक्ष्मीनृसिम्हः । यद्ययं कोण्डुभट्टः रङ्गोजिभट्टस्य पुत्र भट्टोजिदीक्षितस्य भ्रात्रीयस्स्यात्तर्हि लक्ष्मीनृसिम्हः रङ्गोजिभट्टपौत्र इति भवेत् । तिरुविशनल्लर रामसुब्रह्मण्यशास्त्रिकृते " न्यायेन्दुशेखर दोषयोगघटन " ग्रन्थे तु लक्ष्मीनृसिम्हः कोट्टयूर्ग्रामवासीति निर्दिश्यते । एवञ्चायं दाक्षिणात्यो भवति । अनेन व्याकरणशास्त्रेऽपि सिद्धान्तकौमुदीव्याख्या विलासनामा ग्रन्थो विरचितः । 
(A) आभोगटिप्पणी 
आभोगार्थं सङ्गृह्णन् आभोगार्थं मन्दाधिकारिबोधार्थं विशदयन्नयं ग्रन्थः मद्रासराजकीयहस्तलिखितग्रन्थमालायां मुद्रितः । अस्याः कर्ता शास्त्ररत्नाकरबिरुदभूषितः द्रविडात्रेयदर्शन - चतुर्मतसामरस्यादिग्रन्थप्रणेता अद्वैतवेदान्तनिष्णातः मद्रपुरीसंस्कृतकलाशालावेदान्तप्राध्यापकः विंशतिशतकीयः पोलकग्रामाभिजनः श्रीरामशास्त्रीति ज्ञायते । 
3. कल्पतरुव्याख्या - कल्पतरुमन्दारमञ्जरी 
कल्पतरुव्याख्यात्मकः परिमलसंग्राहकश्चायं ग्रन्थ अमुद्रितः लन्दननगरस्थभारतकार्यालयपुस्तकालये (2249 IOL) लभ्यते । अस्य कर्ता पदवाक्यप्रमाणाभिज्ञः रामचन्द्रतत्सदाख्यस्य पुत्रः वैद्यनाथपायुगुण्डे इत्यपरनामा भट्टवैद्यनाथः । अनेन शास्त्रगदीपिकायाः व्याख्या प्रभानाम्नी (1767 1711 A.D.) काले कृता इति हालसूच्यां (Page 174 XV) दृश्यते। वेदान्तकल्पतरुमन्दारमञ्जर्याः प्रतिलेखनावसरः (1778 सं 1722 A.D.) इति दृश्यते । तस्मादस्य कालः (1650 1750 A.D.) भवितुमर्हति ।  
4. कल्पतरुव्याख्या - अज्ञातकर्तृनामधेया । इयं व्याख्या हरप्रसादशास्त्रिसंस्कृतहस्तलिखितग्रन्थसूच्यां निर्दिश्यते ।। 
(न) II. भामतीव्याख्या - ऋजुप्रकाशिका 
भामतीव्याख्यारूपाया ऋजुप्रकाशिकायाश्शैली अतिसरला बालसुबोधाऽर्थपूर्णा च । प्रतिपदव्याख्यारूपोऽयं ग्रन्थः कल्कत्तासंस्कृतग्रन्थमालायां मुद्रितः । ग्रन्थेऽस्मिन् अद्वैतरत्नकोशप्रकाशिका निर्दिष्टा । अस्य कर्ता सन्यासस्वीकारात्पूर्वं रङ्गनाथनामा अद्वैतरत्नकोशव्याख्यातुः भेदधिक्कारविवृतिकर्तुश्च कालहस्तीशयज्वनः (1550 - 1620 A.D.) पुत्रः यज्ञाम्बायाः गर्भजः नलगन्तुवंशजः स्वयम्प्रकाशानन्दसरस्वत्याश्शिष्य अखण्डानन्दसरस्वतीति ज्ञायते । अस्याश्रयदाता सामयिकश्च इम्मिडिजगदेकरायः (जगदेकरायद्वितीयः) । अस्य राज्यशासनकालः क्रैस्तवीयषोडशशतकात् सप्तदशशतकमिति S. कृष्णस्वाम्यय्यङ्गार्कृतविजयनगरेतिहासग्रन्थात्  ' यफिग्राफिका ' कर्नाटिका आफ मैसूर पार्ट I (Page 27-28) भूमिकायाश्च ज्ञायते । तस्मादयं षोडशसप्तदशशतकवासी तत्त्वदीपनकाराद्भिन्न इति च निश्चीयते । 
III.  भामतीव्याख्या - भामतीतिलकम् 
भामतीव्याख्यात्मकोऽयं ग्रन्थ अमुद्रितः मद्रासराजकीयहस्तलिखितपुस्तकालये (R. 4190 MGOML) मैसूर - बरोडासरस्वतीमहालयादिषूपलभ्यते । अस्य कर्ता कोटिकलाग्रामवासी नागमाम्बात्रिविक्रमाचार्ययोः पुत्रः अनन्तार्यंप्रज्ञानारण्ययोश्शिष्य अल्लालसूरिः । अनेन स्वग्रन्थे व्यासाश्रमशब्देन अमलानन्दः निर्दिष्टः । चित्सुखाचार्यश्च प्रमाणीकृतः । " अक्षुण्णायामरण्यान्यां भामत्यां वर्त्म यो व्यधात् । सुगमं प्रणमासस्तं व्यासाश्रममुनीश्वरम् " इति पद्ये व्यासाश्रमशब्द अमलानन्दमेव बोधयति । भामतीव्याख्यानेषु तस्यैव व्याख्यानस्य प्राथम्यात् । केचित्तु व्यासाश्रमामलानन्दौ भिन्नावित्यभ्युपगच्छन्ति । तेषां मतेन सुगम इति काचन भामतीव्याख्या आसीद्या च नाममात्रप्रसिद्धेदानीमिति वक्तव्यमापतति । 
यदि व्यासाश्रमामलानन्दावभिन्नौ तर्हि अल्लालसमयः (1300 A.D.) कालादर्वाचीन अ पय्यदीक्षितनामाग्रहणात् अप्पय्यदीक्षितात् प्राचीन इति सिध्यति । बरोडाहस्तलिखितपुस्तकालयस्थे भामतीतिलकादर्शपुस्तके तस्य प्रतिलिपिकालः (1335 A.D.) इति दृश्यते । तस्मात् कालात्प्राचीनः चित्सुखाचार्यकालात् (1120 -1220 A.D.) अर्वाचीन इति तु निश्चयः । केचित्तु (1600 - 1750 A.D.)  कालमध्यवर्ती अल्लालसूरिरिति वदन्ति ।। शशतकापरार्धादारब्धे अष्टादशशतकपूर्वार्धान्ते काले वसन् अच्युतकृष्णानन्दतीर्थ इति ज्ञायते ।।
V. भामतीविवरणम् - 
ग्रन्थोऽयं वाणीविलासमुद्रणालये मुद्रितः । अस्य कर्ता रामस्वामिशर्मणः पुत्रः पालकाडन्तर्गत काविशेरिग्रामाभिजनः अङ्गाडिपुरसुब्रह्मण्यशास्त्रिणः प्राप्तन्यायशास्त्रः व्याकरणे छन्दश्शास्त्रे च निष्णातः (1879 - 1947 A.D.) कालवासी सुब्रह्मण्यशास्त्रीति ज्ञायते। 
VI. भामतीटीकाप्रकाशः - भामतीविकासश्च ।
चतुस्तूत्रीभामतीटीकाप्रकाशकाविमौ ग्रन्थौ चौखाम्बामुद्रणालये मुद्रितौ। अनयोः कर्ता लक्षमीनथझा इति ज्ञायते। आधुनिकोयं (1952 A.D.) वाराणसीवासी ।
VII. भामतीयुक्तार्थसंग्रहः -
अज्ञातकर्तुनामायं ग्रन्थः " हलषरिपोर्ट आक सान्स्कृट् मानस्क्रिप्ट्स " ग्रन्थे दृश्यते ।  
VIII. भामतीविलासः -
ग्रन्थोऽयं दासगुप्तमहाशयेन स्वीयदर्शनसाहित्यचरिते (HIP Vol II Page 108) निर्दिष्टः। अज्ञातकर्तृनामायं ग्रन्थः कुत्रत्य इत्यादि न ज्ञायते । श्रीरङ्गनाथकृता काचन भामतीव्याख्यापि अस्तीति Rice महाशयो निर्दिशति । 

%भामतीप्रस्थानम् 

(प) सूत्रभाष्यव्याख्या - रत्नप्रभा - 
शाङ्करभाष्यव्याख्यात्मकोऽयं ग्रन्थः निर्णयसागरमुद्रणालये चौखम्बामुद्रणालये च मुद्रितः । ग्रन्थेऽस्मिन् (Page 5) जगन्नाथाश्रमिकृता सूत्रभाष्यव्यख्या " भाष्यदीपिका " निर्दिष्टा । अस्य कर्ता द्रवि़डदेशवासी श्वेतगिरिअभयानन्द (सत्यानन्द) - आनन्दपूर्ण - पुरुषोत्तम - शिवरामानन्द - गोपालानन्द - सरस्वतीपरम्परागतः गोविन्दानन्दस्य स्वयम्प्रकाशानन्दस्य च शिष्यः ब्रह्मविद्याभरणकारस्याद्वैतानन्दस्य गुरुप्षोडशशतकापरार्घवासी (1550-1650 A.D.) रामानन्दसरस्वतीति ज्ञायते । मुद्रितपुस्तके तु गोविन्दानन्दकृतत्वेनैव निर्दिष्टम् । परन्तु " गोविन्दानन्दवाणीचरणकमलगो निर्वृतोऽहं यथालिः" इति दर्शनात् गोविन्दानन्दशिष्य एवास्य कर्ता भवितुमर्हति। गुरुकृतत्वेन निर्देशस्तु आदरातिशयेनेति निर्णीयते । अत एव रत्नप्रभाया रामानन्दीयमिति व्यवहारोऽपि सङ्गच्छते । केचित्तु साम्प्रदायिकाः गोविन्दानन्दकृत इत्येव वदन्ति। रामानन्देव विवरणोपन्यासादय अन्येऽपि ग्रन्था विरचिताः । 
I. रत्नप्रभाभागव्याख्या - रत्नप्रभाभागदीपिका -
" जिज्ञासासूत्रमारभ्य प्रागानन्दमयोक्तितः। भाष्यरत्नप्रभाव्याख्यां व्याकुर्वेभक्तितः कृतिम् ।। " इति ग्रन्थकृतैव प्रतिज्ञातत्वेनामुद्रितोऽयं ग्रन्थ ईक्षत्यधिकरणान्त एव अडयारपुस्तकालये मद्रासराजकीयहस्तलिखितपुस्तकालये (R. 2782) लभ्यते । अस्य कर्ता स्वयम्प्रकाशाद्वैतानन्दयोश्शिष्यः रामानन्दप्रशिष्यः दाक्षिणात्यः भामतीव्याख्याभावदीपिका - वनमालादिकारः सप्तदशशतकापरार्धवासी (1650 - 1750 A.D.) अच्युतकृष्णानन्दतीर्थ इति ज्ञायते । 
II. रत्नप्रभाव्याख्या - पूर्णानन्दीया -
चतुस्सूत्र्यन्तोऽयं ग्रन्थः चौखाम्बामुद्रणालये मुद्रितः। अस्य कर्ता ब्रह्मविद्याभरणकारस्य अद्वैतानन्दस्य शिष्यः रामानन्दस्य प्रशिष्यः, अच्युतकृष्णानन्दसतीर्थ्यः, सप्तदशशतकीयः (1650 - 1750 A.D.) पूर्णानन्दसरस्वतीति ज्ञायते । अनेन समन्वयसूत्रे " अस्मद्गुरुभिः ब्रह्मविद्याभारणे " इति ब्रह्मविद्याभरणं निर्दिश्यते । 
III.  रत्नप्रभाव्याख्या -
अमुद्रितोऽयं ग्रन्थः मैसूरपुस्तकालये लभ्यते । अस्य कर्ता ब्रह्मविद्याभरणकारस्य अद्वैतानन्दस्य ज्ञानानन्दस्य च शिष्यः प्रकाशानन्दः नानादीक्षितगुरुः रत्नप्रभाकर्तू रामानन्दस्य सामयिकश्च षोडशशतकापरार्घकालवासी (1550 - 1650 A.D.) ति निश्चीयते । 
IV. रत्नप्रभाव्याख्या - अभिव्यक्ता । अज्ञातकर्तृनामायममुद्रितग्रन्थः अडयारपुस्तकालये लभ्यते ।। 

%ब्रह्मसूत्राणि
अन्याश्व काश्चन विश्वेदकृता ब्रह्मसूत्रभाष्यव्याख्या, रघुनाथभट्टाचार्यकृता सिद्धान्तार्णवब्रह्मसूत्रभाष्यव्याख्याद्या विद्यन्त इति ज्ञायते । 
ब्रह्मसूत्रवृत्तिग्रन्थाः 

1. अद्वैतकामधेनुः 
शाङ्करभाष्यानुसारी ब्रह्मसूत्रवृत्तिरूपोऽयं ग्रन्थः नागरीलिप्याममुद्रितस्सरस्वतीमहालये (7526 TSML) लभ्यते । तेल्लुगुलिष्यां मुद्रितश्च । अस्य कर्ता तत्त्वचन्द्रिका - विरोधबरूधिनीग्रन्थकारः, वेल्लालकुललब्धजन्मा, अक्कय्य इत्यपरनाम्नः अक्षयशास्त्रिणश्शिष्यः वेङ्कटरायपुत्रः सभारञ्जनकारकविकुञ्जरगुरुः, तप्तमुद्राविद्रावणकारस्य भास्करदीक्षितस्यपिता चोलदेशीयः मोक्षकुण्ड (मुडिकोण्डान् ) ग्रामवासी अभिनवकालिदास इत्यपरनामा उमामहेश्वरः । एतेन तत्त्वचन्द्रिकायां " मध्वविध्वंसन - मध्वन्यक्काराभ्यां मध्वमतस्त खण्डितत्वेन श्रीकण्ठरामानुजमते एवात्र खण्डयेते इति प्रतिपादितम् । एवं रत्नतूलिका - तप्तमुद्राविद्रावणादिकर्त्रा उमामहेश्वर पुत्रेण भास्करदीक्षितेन आत्मनः नृसिम्हाश्रमि - कृष्णानन्दसरस्वत्योश्शिष्यत्वं प्रतिपादितम् । एवं रत्नतूलिका - तप्तमुद्राविद्रावणादिकर्त्रा उमामहेश्वरपुत्रेण भास्करदीक्षितेन आत्मनः नृसिम्हाश्रमि - कृष्णानन्दसरस्वत्योश्शिष्यत्वं प्रतिपादितम् । एवञ्चाप्पय्यदीक्षितस्य अन्तिमसामयिकोऽयमुमामहेश्वर इति सिध्यति । T. R. चिन्तामणिमहोदयास्तु स्वसम्पादितसाहित्यरत्नाकरभूमिकायां भास्करदीक्षितं रघुनाथनायकसामयिकं वर्णयन्ति। रघुनाथनायकशासनकालस्तु (1600 - 1650 A.D.) इति P.P. शास्त्रिणः । एवञ्च भास्करदीक्षितपिता उमामहेश्वरः (1550-1650 A.D.) काले आसीदिति निर्णेतुं शक्यते ।।"
2. अद्वैतप्रकाशः -
नवभिः भागैः पूर्णोऽयं ब्रह्मसूत्रवृत्यात्मक अमुद्रितः ग्रन्थः मद्रासराजकीयहस्तलिखितपुस्तकालये R. 4208 लभ्यते । नारायणीया वृत्तिरित्यपि नामान्तरम् । अस्य कर्ता नारायणप्रिययतिरित्यपरनामा गोविन्दाश्रमशिष्यः दुर्गाप्रसादयतिः स्वग्रन्थे रामतीर्थं प्रणमति । तस्मादयं उत्तरदेशवासी सप्तदशशतकवीसी (1700 A.D.) इति परं ज्ञायते ।। 
3. अद्वैतमञ्जरी -
पञ्चचत्वारिंशद्भिस्स्तबकैः पूर्णोऽयं सूत्रवृत्तिरूपस्स्वतन्त्रग्रन्थः नासिकसूच्यां 27-2 लभ्यते । अस्य प्रणेता परमानन्दयोगिशिष्य इत्येव ज्ञायते ।। अस्य प्रणेतुर्नाम ज्ञानेन्द्रस्वामीति स्यात् इति (A BORI XXI P.145) ज्ञायते। मुद्रितश्चायम् (Bombay 1914) ।
4. अद्वैतरत्नाकरः -
ब्रह्मसूत्रवृत्तिग्रन्थोऽयं मैसूरराजकीयपुस्तकालये लभ्यते । अस्य कर्ता शारीरकमीमांसाभाष्यगद्यमयवार्तिककर्ता रत्नप्रभाकर्तृत्वेन प्रसिद्धस्य गोविन्दानन्दसरस्वत्याश्शिष्यः गौडब्रह्मानन्दसरस्वत्या गुरुः, नारायणतीर्थापराभिधः नारायणसरस्वती सप्तदशशतकीय (1600 - 1700 A. D.)इति ज्ञायते।। 
5. अद्वैतसुधा - 
उत्तरमीमांसासारार्थसुधा - उत्तरमीमांसासारार्थसुधानिधि - वेदान्तार्थसारसंग्रह - वेदान्तसारचिन्तामणि - वेदान्तकौस्तुभादिनाम्ना प्रसिद्धोऽयं ब्रह्मसूत्रवृत्तिग्रन्थ अमुद्रित अडयारपुस्तकालये (33. L. 25. AL,) मद्रासराजकीयहस्तलिखितप्राचीनपुस्तकालये, तिरुपतिवेङ्कटेश्वरालयपुस्तकालये, तिरुवनन्तपुरराजकीयपुस्तकालये, विश्वभारतीशान्तिनिकेतनसूच्यां च लक्ष्यते । अस्य कर्ता कौण्डिन्यगोत्रज उत्तरमायूरदक्षिणास्योपसेवी चोलदेशीयस्सीतारामशास्त्रीति ज्ञायते । अस्य कालः षोडशशतकापरार्घ इति (1560 - 1750 A.D.) दासगुप्तः (HIP Vol. II Page 82)
6. अद्वैतसूत्रार्थपद्धतिः -
अमुद्रितोऽयं सूत्रवृत्तिग्रन्थः मद्रासराजकीयहस्तलिखितपुस्तकालये (R. 5727 MGOML) लभ्यते । अस्य कर्ता षङ्भाषासरसकविरिति विश्रुतः गुहपुरवासी मतत्रयेऽपि ग्रन्थप्रणेता कृष्णावधूतपण्डितः । 
7. अद्वैतामृतम् -
सूत्रवृत्तिरूपोऽयं ग्रन्थः प्रथमपरिच्छेदान्तं भारतकार्यालयपुस्तकालये लन्दननगरे (2405 IOL.) रायल आषियाटिक सोसाइटि बेङ्गालनगरे च लभ्यते । अस्य कर्ता देवेन्द्रसरस्वतीशिष्यः वाणीगीरित्यपरनामा ब्रह्मेन्द्रसरस्वतीति ज्ञायते । यद्यस्य गुरुर्देवेन्द्रसरस्वती " स्वानुभूतिप्रकाश " कर्ता नृसिम्हाश्रमिसापयिकस्स्यात् तर्हि तच्छिष्यस्य ब्रह्मेन्द्रसरस्वत्या अपि कालष्षोडशशतकमिति (1500 - 1600 A.D.) निर्णेतुं शक्यते । 
8. अधिकरणकौमुदी - 
ग्रन्थोऽयं कुत्रास्ति? किं मुद्रितः ? इत्यादिकं न ज्ञायते । परन्तु कामकोटिसूच्यां दृश्यते । अस्य कर्ता रामकृष्ण इति परं ज्ञायते । पूर्वमीमांसाग्रान्थ इति प्रतिभाति । 
9. अधिकरणचतुष्टयी - 
आनन्दमयाधिकरण - यथाश्रयभावाधिकरण - ऐहिकाधिकरण - लिङ्गभूयस्त्वाधिकरणानां विष्यविचारात्मकोऽयं ग्रन्थः बालमनोरमामुद्रणालये मद्रासनगरे मुद्रितः । अस्य कर्ता महामहोपाध्यायबिरुदभूषितः कृष्णतटाक (करुङ्गुलम् ) ग्रामवासी श्रीशालिवाटिनगरा(तिरुनेलवेली) भिजनः हरिहरशास्त्रिशिष्यः, मद्रपुरीसंस्कृतकलाशालाप्रधानाध्यापकः अपरे वयसि प्राप्तसन्यासाश्रमः र्विशतिशतकीय (1870 - 1937 A.D.) कृष्णशास्त्रीति ज्ञायते । 
10. अधिकरणमञ्जरी -
 " रामग्रहेन्दुसंख्याता (193) न्यायाश्शारीरकाश्रया " इत्यादिना अधिकरणसंख्याः भाष्यानुसारं अधिकरणार्थञ्च पद्यबद्धोऽयं सङ्गृह्णाति । ग्रन्थोऽयं प्राच्यभाषाशोधनपत्रिकायाः पञ्चमे भागे (J.O.R. Madras Vol. V) मुद्रितः । अस्य कर्ता ज्ञानघनप्रशिष्यः, न्यायसुधाकारस्य ज्ञानोत्तमस्य शिष्यः, विज्ञानात्मसतीर्थ्यः, सुखप्रकाशशुद्धानन्दयोर्गुरुः, अमलानन्दानन्दगिर्योः प्राचार्यः द्वादशज्ञातकीयश्चित्सुखाचार्य इति ज्ञायते । 
11. अधिकरणरत्नमाला - 
सूत्रवृत्तिग्रन्थोऽयं मद्रासराजकीयहस्तलिखितपुस्तकालये (R. 2902 MGOML) अमुद्रित उपलभ्यते। अस्य कर्ता चित्सुखाचार्यशिष्यः, आनन्दात्मशिष्यः अमलानन्दगुरुः, श्रीकण्ठशास्त्रिणां प्रबन्धप्रमाणात् (IHQ XIV) आनन्दगिरेरपि गुरुः, त्रयोदशशतकीयः (1200 - 1300 A.D.) सुखप्रकाशः। अनेन तत्त्वदीपिकाया व्याख्या भावद्योतनिका कृता या त्वन्यत्र निरूपयिष्यते । अस्य भारतीतीर्थकृता व्याख्यापि काचन तिरुवनन्तपुरे लभ्यते ।। 
12. अधिकरणसङ्गतिः - 
भाष्यवर्णितानां अधिकरणानां सङ्गतिं प्रदर्शयन्नयं ग्रन्थस्स्वतन्त्रः, नतु व्याख्यारूपः । प्रकाशितश्चायं ग्रन्थः जर्नल आफ ओरियण्टलपत्रिकायास्सप्तमे भागे (JOR Madras Vol VII)। अस्य कर्ता ज्ञानघनप्रशिष्यः न्यायसुधाकारस्य ज्ञानोत्तमस्य शिष्यः विज्ञानात्मसतीर्थ्यः सुखप्रकाशशुद्धानन्दयोर्गुरुः, अमलानन्दानन्दगिर्योः प्राचार्यः द्वादशशतकीयश्चित्सुखाचार्य इति ज्ञायते।। 
13. अधिकरणार्थसंग्रहः । ग्रन्थोऽयं कामकोटिकोशस्थाने मद्रासनगरे मुद्रितः । अस्य कर्ता श्रीसाधनयोगीति परं ज्ञायते । 
14. अर्थप्रकाशिका -
वृत्तिग्रन्थोऽयं ब्रह्मसूत्रार्थप्रकाशिकापरनामा पञ्चाबविश्वविद्यालयग्रन्थसूच्यां (231) अमुद्रित उपलभ्यते । अस्य कर्ता नागरवंशजः जयरामपण्डितः। अनेनरचितस्य महावाक्यादर्शस्य प्रतिलेखनकालः (1773 1717 A.D.) इति बरोडास्थमहावाक्यादर्शग्रन्थात् (12424 DCBRD) ज्ञायते। एवञ्चास्य कालस्सप्तदशशतकापरार्धदारब्धस्स्यादिति (1650 - 1750 A.D.) निश्चीयते ।। 
15. अर्थदीपिका - 
ब्रह्मसूक्षदीपिकापरनामायं ग्रन्थ अमुद्रितः भाष्यतद्व्याख्यादिसारसंग्रहरूपः मद्रासराजकीयहस्तलिखितपुस्तकालये (R. 3481 MGOML) लभ्यते। अस्य कर्ता शिवरामगौरीपुत्रः वेङ्कटपण्डित इति परं ज्ञायते ।  
अर्थसंग्रहनाम्नी काचन ब्रह्मसूत्रवृत्तिः ब्रह्मानन्दतीर्थकृता मुद्रता (I.H.Q. 13, 1937. Suppliment ।
16. ऋजुव्याख्या । अमुद्रितोऽयं ग्रन्थः पञ्चाबविश्वविद्यालयसूच्यां (722) दृश्यते । अस्य कर्ता विज्ञानभिक्षुरिति परं ज्ञायते। 
17. चन्द्रिका - 
अमुद्रितोऽयं सूत्रवृत्तिग्रन्थः लन्दननगरस्थभारतकार्यालयपुस्तकालये (2270 DC IOL) चोपलभ्यते । अस्य रचयिता महामहोपाध्यायबिरुदभूषितः सन्मिश्रभवदेवापरनामा मिथिलावासिनः कृष्णदेवसन्मिश्रस्य पुत्रः " पाटना " नगरवासी, शाहजहाँख्यमोगलसम्राजस्सामयिकः, सठक्कुरभवदेवशिष्यस्सप्तदशशतकवासी (1600 - 1650 A.D.) भवदेवमिश्र इति ज्ञायते । अनेनायं ग्रन्थः (1571 1649 A.D.) काले निर्मितः । 
तत्त्वप्रबोधिनी - 
अद्वैतसिद्धान्तावलम्बिनी इयं वृत्तिः गिरीन्द्रनाथवेदान्तरत्नविरचिता (1922 A.D.) मुद्रिता विद्यते। 
18. तात्पर्यबोधिनी - 
शङ्करसिद्धान्तानुसारिणी मुलभबोधा चेयं सूत्रवृत्तिः आनन्दाश्रममुद्रणालये मुद्रिता । अस्य प्रणेत्रा शङ्करानन्देन अन्यापि काचन " दीपिका " नाम्नी वृत्तिरारचिता । तस्या अयं भिद्यते । अस्य कर्ता आनन्दात्मशिष्यः विद्यारण्यगुरुः कावेरीतिरवासीति एतत्कृतग्रन्थदर्शनात् ज्ञायते । पञ्चदश्यादिविद्यारण्यग्रन्थाश्चात्रप्रमाणानि । अयं आनन्दात्मविद्यातीर्थयोश्शिष्यः, विद्यारण्यगुरुः, चोलदेशीयमध्यार्जुनग्रामवासी (तिरुविडैमरुदूरः), पूर्वाश्रमे वाञ्छेशवेङ्कटसुब्बाम्बयोः पुत्र इति सम्प्रदायागता जनश्रुतिः । सूर्यनारायणशास्त्री तु विद्यातीर्थस्यैव विद्याशङ्करतीर्थ इति शङ्करानन्द इति च नामान्तरमिति वदति । दासगुप्तस्तु विद्यारण्यामलानन्दयोर्गुरुश्शङ्करानन्द इति वदति । शृङ्गगिरिनठीयपरम्परायान्तु विद्यारण्यः विद्यातीर्थशिष्य इति दृश्यते । यदि विद्यातीर्थ एव शङ्करानन्दस्स्यात् तर्हि नृसिम्हतीर्थस्यैव आनन्दात्मा इति नाम भाव्यम् । अथवा आनन्दात्मा विद्यागुरुर्नृसिम्हतीर्थः दीक्षागुरुरिति वक्तव्यं भवति । " आनन्दात्मयतीश्वरं तमनिशं वन्दे गुरुणां गुरु" मिति अमलानन्दकथनात् आनन्दात्मप्रशिष्य अमलानन्द इति सिध्यति । " यथार्थानु भवानन्दपदगीतं गुरुं नुम " इति दर्शनात् अनुभवानन्दोऽपि अमलानन्दगुरुरिति सिध्यति । किमयममलानन्द अनुभवानन्दशिष्यः? उत शङ्करानन्दशिष्य इति स्थिरीभवति संशयः । " सुखप्रकाशयतिनं तं " "नौमि विद्यागुरुं " इति दर्शनात् अमलानन्दस्य विद्यागुरुस्सुखप्रकाश इति तु निश्चितम् । सर्वथा शङ्करानन्दसामयिक अमलानन्द इति तु निश्चयः । शङ्करानन्दशिष्यः सदानन्दप्रथमः । सदानन्दप्रथमस्य शिष्य अद्वैतानन्दसरस्वती । अद्वैतानन्दसरस्वत्याश्शिष्यस्सदानन्दद्वितीय इति " अनालस आफ ओरियण्टल रिसर्च " पत्रिकायाः षष्ठे भागे (AOR Madras Vol VI Part I) प्रतिपादितम् । सर्वथा त्रयोदशशतकादारव्घे काले (1275 - 1350 A.D.) शङ्करानन्द आसीदिति निश्चीयते । 
तात्पर्यविवरणम् । शाङ्करभाष्यानुसारी सूत्रवृत्तिग्रन्थोऽयं भैरवतिलककृतः (1800 A.D.) वाराणस्यां मुद्रितः 1917 ।
19. त्र्यम्बकवृत्तिः (भाष्यभानुप्रभा) -
भाष्यभानुप्रभापरनामायं शाङ्करभाष्यार्थसंग्राहकः क्रोडपत्रात्मक अमुद्रित अपूर्णश्चायं ग्रन्थ अडयारपुस्तकालये (41. C. 90 AL) शृङ्गगिरिमठपुस्तकालये च लभ्यते । अस्य कर्ता त्र्यम्बकशास्त्री त्र्यम्वकभट्टनाम्ना प्रसिद्धः, भारद्वाजगोत्रोत्पन्नः, बाबाजीयज्वनः पौत्रः गङ्गाधराध्वरिकृष्णाम्विकयोः पुत्रः, नरसिम्हयज्वनः कनीयान् भ्राता परमार्थसारव्याख्यातू राघवानन्दस्य लधुचन्द्रिकाकारस्य ब्रह्मानन्दसरस्वत्याश्च शिष्यः, शाहजीशरभोजीशासनकाले चोलदेशवासी पश्चात् महीशूरपूनादिनगरवासी सप्तदशशतकापरार्धारब्धकालवासी (1650 - 1700 A.D.) ति ज्ञायते। अस्य शिष्यस्स्थाणुशास्त्रीति प्रसिद्धः " घृतशौचदीपिका " कर्ता इति मद्रासराजकीयहस्तलिखितपुस्तकालयस्थात् (R. 3854) पुस्तकात् ज्ञायते । 
20. दीपिका - 
सूत्रवृत्तिरूपोऽयं ग्रन्थ आनन्दाश्रममुद्रणालये मुद्रितः । ग्रन्थोऽयं तात्पर्यबोधिन्याश्शङ्करानन्दकृतायाः भिद्यते च । अस्य कर्ता आनन्दात्मशिष्यः विद्यारण्यगुरुरमलानन्दसामयिकस्सदानन्दगुरुस्त्रयोदशशतकापरार्धवासी (1275 - 1350 A.D.)शङ्करानन्द इति ज्ञायते ।
21. निर्मलकृष्णवृत्तिः -
निर्मलकृष्णभाष्यमित्यपरनामायं ग्रन्थः लन्दननगरस्थभारतकार्यालयपुस्तकालयसूच्यां दृश्यते। अस्य कर्ता निर्मलकृष्ण इति परं ज्ञायते । 
22. न्यायरक्षामणिः -
शारीरकन्यायरक्षामणिरिति प्रसिद्धोऽयं वृत्तिग्रन्थः भाष्यं भामतीञ्चानुसरति। ग्रन्थेऽस्मिन् प्रत्यधिकरणं पूर्वपक्षसिद्धान्तौ मीमांसान्यायानुसारं सविस्तरमुपवर्णितौ । प्रतिसूत्रं दलप्रयोजनोपवर्णनमस्य ग्रन्थस्य महान् विशेषः । प्रथमाध्यायान्तमुपलभ्यमानः भाष्यस्य क्रोडपत्ररूपोऽय ग्रन्थः वाणीविलासमुद्रणालये मुद्रितः । अस्य कर्ता वक्षस्थलाचार्यापरनामकस्य आचार्यदीक्षितस्य पौत्रः विवरणदर्पणकारस्य रङ्गराजाघ्वरिणः पुत्रः, पितुरेव प्राप्तविद्यः, नृसिम्हाश्रमिप्रेरणया प्राप्ताद्वैतमतपक्षपाती काञ्चीमण्डलान्तर्गत अडयप्पलग्रामवासी आचार्यदीक्षितस्य भ्राता, रत्नखेटश्रीनिवासदीक्षितजामाता मङ्गलनायिकाभर्ता नीलकण्ठ - उमामहेश्वर-चन्द्रावतंसानां पिता चतुुरधिकशतग्रन्थप्रणेतृत्वेन प्रसिद्धः षोडशशतकवासी (1520 - 1593 A.D.) अप्पप्यदीक्षित इति ज्ञायते । 
23. न्यायसग्रहः - 
अमुद्रितोऽयं ग्रन्थः बम्बई विशविद्यालयहस्तलिखितपुस्तकालयसूच्यां (644 DC. BUL.) दृश्यते । अस्य कर्ता छान्दोग्योपनिषदां व्याख्यायाः मिताक्षरायाः कर्ता नित्यानन्दाश्रमीति ज्ञायते । 
24. पुरुषार्थसुधामिधिः - 
अस्यैव ग्रन्थस्य सर्ववेदान्तश्रुतिसारसंग्रहः, ब्रह्मसूत्रविषयवाक्यवृत्तिः, वैय्यासिकब्रह्ममीमांसासारसंग्रहः, वैय्यासिकब्रह्ममीमांसासूत्रभाष्यसारसंग्रह इति नामान्तराणि दृश्यन्ते। भाष्यानुसारी सूत्रवृत्तिरूपोऽयं ग्रन्थ अमुद्रितः मद्रासराजकीयहस्तलिखितपुस्तकालये (R. 2471 MGOML) सरस्वतीमहालये अडयार पुस्तकालये, तिरुवनन्तपुरपुस्तकालये बरोडापुस्तकालये, लन्दननगरस्थभारतकार्यालयपुस्तकालये शङ्कराचार्यमठपुस्तकालये च लभ्यते । 
" वासुदेवेन्द्रचरणाम्बुरुहद्वयम् । प्रणम्य व्याससूत्राणां विवृतिः क्रियतेऽधुना ।" इति पुरुषार्थसुधानिधौ ज्ञानेन्द्रसरस्वत्या वासुदेवेन्द्रः नमस्कृतः । अद्वैतरत्नकोशपूरणीकर्त्रा अग्निहोत्रभट्टेन " ज्ञानेन्द्रापरनामानं दक्षिणामूर्तिदैवतम् । वासुदेवादिसच्छिष्य " इत्यादिना ज्ञानेन्द्रवासुदेवेन्द्रौ	 नमस्कृतौ । एवञ्चाग्निहोत्रभट्टः वासुदेवेन्द्रज्ञानेन्द्रयोश्शिष्य इति सिध्यति । अग्निनहोत्रभट्टेन स्वकृते मैसूरपुस्तकालयमुद्रिते अद्वैतरत्नकोशपूरणीग्रन्थे (Page 49) "अस्मत्परमगुरुचरणमतमेव सम्यक् " इति, जिज्ञासाशब्दाथविचारे (Page 72)" परमगुरुचरणास्तु जिज्ञासापदं तन्त्रेणोपात्तमित्याहु" रित्यादिना ज्ञानेन्द्रसरस्वतीग्रन्थ अनूद्यते । एवञ्चास्य पुरुषार्थसुधानिधेः कर्ता वासुदेवेन्द्रशिष्यः वासुदेवेन्द्रगुरुः अग्निहोत्रभट्टप्राचार्यः गुरुश्च ज्ञानेन्द्रसरस्वती षोडशशतकापरार्धारब्धकालवासीति (1550 - 1650 A.D.) ज्ञायते । 
25. ब्रह्मतत्वप्रकाशिका - 
शाङ्करभाष्यार्थानुसारिणीयं संक्षिप्ता वृत्तिस्सूत्रार्थावबोधसुलभा भवति। मुद्रिता चेयं वृत्तिर्वाणीविलासमुद्रणालये । अस्य कर्ता मधुरानगरवासिनस्सोमनाथावधानिनः पार्वत्याश्च पुत्रः, पूर्वाश्रमे शिवरामकृष्णनामा दहरविद्याप्रकाशिकाकर्तुः परमशिवेन्द्रस्य शिष्यः, अद्वैतरसमञ्जरीकारस्य नल्लादीक्षितस्य गुरुः, तिरुविशनल्लूरग्रामवासिन अय्यावालिति प्रसिद्धस्य श्रीधरवेङ्कटेसशास्त्रिणः, महाभाष्या-गोपालकृष्णशास्त्रिणश्च सतीर्थ्यः, शरभोजिप्रथमस्य सामयिकः अष्टादशशतकीयः (1700 - 1800) सदाशिवब्रह्मेन्द्रसरस्वती। अस्य समाधिस्त्रिशिरपुर्यन्तर्गतनेरूर्गामे अद्याप्यस्तीति ज्ञायते। कृष्णाचार्यकृते संस्कृतसाहित्येतिहासे सदाशिवब्रह्मेन्द्रकालप्षोडशशतकमिति दृश्यते ।।
26. ब्रह्ममीमांसासंग्रहः - अज्ञातकर्तृनामायं ग्रन्थः अडयारपुस्तकालये लभ्यते ।। 
27. ब्रह्मसूत्रकुतूहलम् - 
ग्रन्थस्यास्यावतरणिकायां अद्वैतसिद्धान्ताः विशदीकृताः । " अथातो ब्रह्मजिज्ञासा "  इति सूत्रादारभ्य ' ज्योतिश्चरणाभिधानात् ' इत्यन्तानां चतुर्विशति सूत्राणां अद्वैतमतपोषिणी काचित् वृत्तिरियम् । मुद्रितश्चायं ग्रन्थः राजराजेश्वरीमुद्रणालये वाराणसीनगरे । अस्य कर्ता सच्चिदानन्दाश्रमवासुदेवेन्द्रयोश्शिष्यः वाराणसीवासी कृष्णानन्दसरस्वती रामदुर्गाख्यविप्रप्रेरणया वेदेन्दुवसुभूमिते 1814  श 1890 A.D. काले शास्त्राकूतप्रकाशग्रन्थं चकारेति ज्ञायते । तस्मादयं एकोनविंशतिशतकवासी (1825 - 1900 A.D.) इति निश्चीयते ।।
28. ब्रह्मसूत्रतात्पर्यदीपिका -
ग्रन्थोऽयं तेलुगुलिप्यां मुद्रितः । अस्य व्याख्या तात्पर्यविमर्शिनीनाम्नी गुरुमूर्तिकृतापि मुद्रिता । अस्य कर्ता दक्षिणदेशवासी श्रीरङ्गचिदम्बरादिपुण्यक्षेत्रवासी सदानन्दतीर्थशिष्य अद्वैतानन्दतीर्थः । अस्य पूर्वाश्रमे पितुर्नाम माधवसूरिरिति । मातुर्नाम महालक्ष्मीरिति । हरीतगोत्रजोऽयम् । ब्रह्मसूत्रतात्पर्यदीपिकाव्याख्यातुर्गुरुमूर्तिशास्त्रिणः कालः (1850 - 1901 A.D.) इति ज्ञायते। तस्मात् मूलग्रन्थकर्तुः अद्वैतानन्दस्यापि काल एकोनर्विशतिशतकमिति परं ज्ञायते। 
(A) ब्रह्मसूत्रतात्पर्यदीपिकाव्याख्या - तात्पर्यविमर्शिनी -
अद्वैतानन्दतीर्थकृतायाः ब्रह्मसूत्रदीपिकायाः व्याख्यात्मकोऽयं ग्रन्थः तेलुगुलिप्यां मुद्रितः। अस्य कर्ता गुरुस्वामिशास्त्री एकोनविंशत्तिशतकापरार्धारब्धकालवासी (1850 1910 A.D.) ति ज्ञायते । 
29. ब्रह्मसूत्रतात्पर्यप्रकाशः - 
सदानन्दसरस्वतीकृतोऽयं ग्रन्थः । ग्रन्थस्यास्य प्राप्तिस्थानादि न ज्ञायते । परन्तु कामकोटिमठस्थाद्वैतग्रन्थकोशे दृश्यते। 
30. ब्रह्मसूत्रतात्पर्यविवरणम् - 
शाङ्करभाष्यभावनावदिदं विवरणं पण्डितनूतनग्रन्थमालायां (पंडिट न्यूसीरीजं वनारस) मुद्रितम् । अस्य तिलकोपनामा भैरवशर्मा वेदाक्षिवसुचन्द्रेऽब्दे (1824 श. 1788 A.D.) स्वग्रन्थंं कृतवानिति ज्ञायते । तस्मादस्य काल अष्टादशशतकापरार्धागदारब्ध (1750 - 1850 A.D.) इति निश्चीयते ।। 
31. ब्रह्मसूत्रभाष्यान्वयार्थसूत्रवृत्तिः - 
अमुद्रितोऽयं ग्रन्थ रामभद्रकृत इति पञ्चाबविश्वविद्यालयसूच्यां दृश्यते । 
32. ब्रह्मसूत्रविषयवाक्यविवरणम् -
अमुद्रितोऽयं ग्रन्थः महीशूरराजकीयपुस्तकालये दृश्यते । अस्य कर्ता रामचन्द्रेन्द्रापरनाम्नाम् उपनिषद्ब्रह्मेन्द्राणां शिष्येष्वन्यतमः कृष्णानन्दसतीर्थ्यः, वासुदेवेन्द्रप्रशिष्यः, एकोनविंशतिशकीय ( (1800 - 1900) बोधेन्द्रसरस्वतीकृताद्वैतभूषणव्याख्या - आनन्ददीपिकाकारः वासुदेवेन्द्र इति ज्ञायते । 
33. ब्रह्मसूत्रविषयवाक्यवृत्तिः -
अमुद्रितोऽयं वृत्तिग्रन्थः मद्रासराजकीयहस्तलिखितपुस्तकालये (R. 2471 MGOML) लभ्यते । अस्य कर्ता विवरणोपन्यासरत्नप्रभाकारश्शिवरामानन्दगोपालानन्दसरस्वत्योः प्रशिष्यः गोविन्दानन्दप्रकाशानन्दयोश्शिष्यः षोडशशतकापरार्धादारब्धकालवासी (1550 - 1650 A.D.) रामानन्दसरस्वतीति ज्ञायते । 
34. ब्रह्मसूत्रव्याख्या -
अमुद्रितोऽयं ग्रन्थः जयपुरसूच्यां दृश्यते । अस्य कर्ता जयपुरराज्यस्थापकः जयसिम्ह इति ज्ञायते ।। 
35. ब्रह्मसूत्रविवरणम् - 
अमुद्रितोऽयं ग्रन्थः कुत्र लभ्यत इति न ज्ञायते । परन्तु कामकोटिसूच्यां दृश्यते । अस्य कर्ता परमानन्दघन इति परं ज्ञायते ।।
36. ब्रह्मसूत्रतत्वार्थविलासः -
ब्रह्मसूत्रसद्वृत्तिरित्यपरनामायं ग्रन्थ अमुद्रितः मद्रासराजकीयहस्तलिखितपुस्तकालये (R. 1808 MGOML) लभ्यते । ब्रह्मसूत्रविवरणमिति कश्चन ग्रन्थ एतद्ग्रन्थकर्तृकृतत्वेन न्यायरक्षामणिभाष्योक्तिविरोधग्रन्थे निर्दिष्टः । किं स एवायं ग्रन्थ उतान्य इति तु न ज्ञायते । ब्रह्मसूत्रविवरणग्रन्थस्यानुपलम्भात् । अस्य कर्ता महामहोपाध्यायबिरुदभूषितः अश्वत्थनारायणशास्त्रिपौत्रः रामशङ्करशास्त्रिपुत्रः चोलदेशान्तर्गतशाहजग्रामवासी (तिरुविशनल्लूर) श्रीवत्सगोत्रोत्पन्नः, शिवरामशास्त्रिशिष्यः चन्द्रिकाचार्यभिक्षुसामयिकः एकोनर्विशतिशतकीयः (1850 - 1920 A.D.) रामसुब्रह्मण्यशास्त्रीति ज्ञायते ।। 
37. ब्रह्मसूत्ररत्नावली -
शाङ्करभाष्यार्थमानुष्ठुभैः पद्यैः प्रतिपादयन्नयं ग्रन्थ आनन्दाश्रममुद्रणालये मुद्रितः। अस्य कर्ता सुब्रह्मण्यशास्त्री आधुनिक इति ज्ञायते।।
38. ब्रह्मसूत्रसारसङ्ग्रहः -
ब्रह्ममीमांसासूत्रसारसंग्रहापरनामायं ग्रन्थ अमुद्रितः मद्रासराजकीयहस्तलिखितपुस्तकालये (R. 399 C. MGOML) लभ्यते । अस्य कर्ता वासुदेवेन्द्रशिष्यस्सप्तदशशतकीयः 1700 A.D. प्रज्ञातापरनामा प्रज्ञानेन्द्रमुनिरिति ज्ञायते । स्थानादिनिरूपणप्रमाणं नोपलभामहे ।। 
39. ब्रह्मसूत्रसारार्थः -
ग्रन्थोऽयं श्लोकसहस्रात्मकः ब्रह्मसूत्राणां सारभूतान् अर्थान् प्रतिपादयति । ग्रन्थोऽयं गौडपादेन कृत इति वाराणस्यां जगन्नाथपण्डितगृहे आसीदिति च वैशेषिकदर्शनसम्पादको विन्ध्येश्वरीप्रसादद्विवेदी निर्दिशति । परमन्यत्र न दृश्यते । 
40. ब्रह्मसूत्रसिद्धान्तमुक्तावलिः -
ग्रन्थोऽयं चौखाम्बामुद्रणालये मुद्रितः । अस्य कर्ता वनमालिमिश्रनामा एकोनविंशतिशतकीय इति ज्ञायते ।
41. ब्रह्मसूत्राधिकरणन्यायानुक्रमणिका - 
अमुद्रितोऽयं ग्रन्थः मद्रासराजकीयहस्तलिखितपुस्तकालये (R. 3305 (C) MGOML) लभ्यते । अस्य कर्ता आनन्दानुभूतिशिष्यः, विबुधेन्द्रयोगीत्यपरनामा केरलदेशजः राजराजरविवर्मसामयिकः नृसिम्हाश्रमिगुरुगीर्वाणेन्द्रसरस्वतीभक्तः पञ्चदशशतकीयः कृष्णानुभूतियतिरिति ज्ञायते ।। 		
42. ब्रह्मसूत्रानुगुण्यसिद्धिः -
ग्रन्थोऽयं भाष्यभावसंग्रहात्मकः सूत्राणां सक्षिप्तभावमावेदयति । मुद्रितश्चायं ग्रन्थः गोपालविलासमुद्रणालये कुम्भघोणनगरे । अस्य कर्ता महामहोपाध्यायबिरुदभूषितः कृष्णतटाकग्रामवासी श्रीशालिपुराभिजनः हरिहरशास्त्रिशिष्य अद्वैतसभापण्डितः मद्रपुरीसंस्कृतकलाशालाप्रधानाध्यापकः स्वीकृतसन्यासाश्रमः विंशतिशतकीयः (1870 - 1939 A.D.) कृष्णशास्त्रीति ज्ञायते । 
43. ब्रह्मसूत्रार्थचिन्तामणिः सव्याख्यः -
सव्याख्योऽयं गन्थ अमुद्रितः कुम्भघोणशङ्करमठे मैसूरपुस्तकालये मद्रासराजकीयहस्तलिखितपुस्तकालये च लभ्यते । अस्य कर्ता चोलदेशवासी शाहजग्रामाभिजनः महिषशतकव्याख्यातुर्वाञ्छेश्वरस्य प्रपितामहः नरसिम्हस्य पितामहः, महादेवपिता तञ्जपुरशासकस्य प्रतापसिम्हस्य राजसभापण्डित अष्टादशशतकापरार्धारब्धकालवासी (1780 A.D.) वाञ्छेश्वर इति ज्ञायते। 
44. ब्रह्मसूत्रोपन्यासः -
वैय्यासिकसूत्रोपन्यासापरनामायं ग्रन्थः महीशूरपुस्तकालये लभ्यते । अस्य कर्ता भारतीतीर्थशिष्यश्चतुर्दशशतकीय (400 A.D.) परमेश्वरभारतीति ज्ञायते । अनेन निजतत्वामृतरसाख्यः कश्चन प्रकरणग्रन्थोऽपि कृतः ।
45. ब्रह्मसूत्रार्याद्विशतिका -
" ब्रह्मसूत्रार्याद्विशतिका" " शारीरकसूत्रार्याद्विशतिका" परनामायं ग्रन्थः द्विशतैः पद्यैः पूर्ण अडयारपुस्तकालये निर्णयसागर मुद्रणालये च मुद्रितः । अस्य कर्ता रामचन्द्रेन्द्र इत्यपरनामा उपनिषद्ब्रह्मेन्द्रः। प्रथमवासुदेवेन्द्रस्य प्रशिष्यः द्वितीयवासुदेवेन्द्रस्य शिष्य, रामचन्द्रेन्द्रसतीर्थ्यः, कृष्णानन्दगुरुश्चायं उपनिषद्ब्रह्मेन्द्रः अष्टादशशतकापरार्घारब्धे एकोनविंशतिशतकपूर्वान्ते काले (1750 - 1820 A.D.) आसीदिति ज्ञायते । कुत्रचिदयमेव ग्रन्थ आत्मारामकृत इति निर्दिश्यते च । 
46. ब्रह्मानन्दप्रकाशिका -
सूत्रवृत्तिरूपोऽयं ग्रन्थ अमुद्रितः नासिकपुस्तकालयसूच्यां दृश्यते । अस्य कर्ता वाराणसीवासी अच्युताश्रमशिष्यः अद्वैताश्रमप्रशिष्यः जगज्जीवनविद्वान् अष्टादशशतकापरार्धकालवासी (1775 - 1850 A.D.) इति ज्ञायते। अयमेव वेदान्तसाररत्नावलीकारश्च ।। 
47. ब्रह्मामृतवर्षिणी -
ब्रह्मसूत्रवृत्तिरूपोऽयं ग्रन्थस्सूत्राणां सङ्गतिं संशयं पूर्वपक्षसिद्धान्तान् , विवरणभामतीकृतांश्च मतभेदान् प्रदर्शयन् शाङ्करभाष्यमनुसरति । मुद्रितश्चायं ग्रन्थः वाराणसीग्रन्थमालायां आनन्दाश्रममुद्रणालये च । ग्रन्थोऽयं तेलुगुलिप्यामपि मुद्रितः । तेलुगुलिप्यां मुद्रितस्य ग्रन्थस्य कर्तुर्नाम धर्मभट्ट इति दृश्यते । चौखाम्बामुद्रिते तु मुकुन्दगोविन्दशिष्य अस्य कर्ता इति दृश्यते । आनन्दाश्रममुद्रितग्रन्थे तु रामकिङ्गरधर्म अस्य कर्ता इति दृश्यते । अडयार पुस्तकालयस्थादर्शपुस्तकेऽपि (10 E 41  AL) रामकिङ्करधर्म अस्य कर्ता इत्येव दृश्यते । बरोडानासिक - पुस्तरकालयसूचीष्वपि रामकिङ्करधर्म एवास्य कर्तेति दृश्यते । मद्रासराजकीयहस्तलिखितपुस्तकालयपुस्तके तु (D. 4689 MGOML) धर्मभट्ट इति दृश्यते । ग्रन्थेषु भेदोऽपि नैव दृश्यते । यद्यस्य कर्ता धर्मभट्टस्स्यात्तर्हिधर्मभट्टः तिरुमलाचार्यसूनुः मुकुन्दाश्रमस्य मन्त्रवादिरामचन्द्रार्यस्य च शिष्यः अन्नम्भट्टादिवत् सप्तदशशतकापरार्धकालिकः (1650 - 1750 A.D.) इति निश्चीयते । अन्नम्भट्टपितुर्धर्मभट्टपितुश्च नामसादृश्यमपि अवधेयार्हो विषयः।। यद्यस्य कर्ता रामानन्दसरस्वती स्यात्तर्हि रामानन्दसरस्वती मुकुन्दगोविन्दाचार्यशिष्य एकोनविं शतिशतकपूर्वार्धात्प्राक्तन इति निश्चीयते । यतः बम्बई युनिवर्सिटीहस्तलिखित पुस्तकालयस्थे आदर्शपुस्तके (1877 सं 1821 A.D.) इति प्रतिलेखनावसरः परिलक्ष्यते । तस्मात्तत्प्राक्तन इति निश्चयः । रामकिङ्करधर्मस्यैव रामानन्द इत्यपि स्यान्नाम । अथवा रामानन्दस्यैव राककिङ्करधर्म इति स्यान्नाम । उभयोरपि आचार्यनामसादृश्यमपि अवधेयार्हः विषयः ।। यद्यस्य कर्ता रामकिङ्करस्स्यात् स च रामानन्द इति व्यवहाराद्भिन्न स्स्यात्तर्हि रामकिङ्करोऽयं हरिनाथरामनाथयोः प्रशिष्य मुकुन्दगोविन्दशिष्यः षोडशसप्तदशशतकीयः (1600 - 1700 A.D.) अद्वैततत्त्वसुधाकार इति न्यूकाटलागस काटलागरप्रमाणात् ज्ञायते ।। 
48. ब्रह्मामृतवर्षिणी -
पदयोजनापरनामायं ग्रन्थः अमुद्रितः लन्दननगरस्थभारतकार्यालयपुस्तकालयसूच्यां (2268 DC. IOL) दृश्यते। धर्मभट्टरामानन्दरामकिङ्करकृतायाः ब्रह्मामृतवर्षिण्याः भिद्यतेऽयं ग्रन्थः। अस्य कर्ता स्वयम्प्रकाशशिष्यस्सदाशिवानन्दसरस्वतीति ज्ञायते। सप्तदशशतकीयोऽयं सदाशिवानन्द इति निर्णयः । 
49. मिताक्षरा - 
भामतीकल्पतरुवैय्यासिकन्यायमालादिप्रदर्शितदिशा ब्रह्मसूत्राणां वृत्तिरूपोऽयं ग्रन्थ आनुपूर्व्या क्वचित् वाक्यविन्यासेन क्वचित् कल्पतरुवैय्यासिकन्यायमालादीननुसरति। मुद्रितश्चायं ग्रन्थः मद्रासराजकीयहस्तलिखितपुस्तकालयग्रन्थमालायाम् । अस्य कर्ता तर्कसंग्रहतर्कसंग्रहदीपिका अष्टाध्यायीव्याख्यामिताक्षरा कैयटकृतमहाभाष्यप्रदीपव्याख्या -'उद्योतन' कर्ता पदवाक्यप्रमाणपारावारपारीणः आन्ध्रदेशजः कृष्णानदीतीरोपन्तचित्तूरग्रामवासी अद्वैतविद्याचार्यराघव सोमयाजिकुलोत्पन्नः, तिरुमलाचार्यसूनुः कौशिकगोत्रजः, सिद्धान्तकौमुदीव्याख्या सिद्धान्तरत्नाकरग्रन्थप्रणेतू रामकृष्णाभट्टस्य, सर्वदेवस्य च कनीयान् भ्राता, व्याकरणाचार्यस्य शेषकृष्णपुत्रस्य शेषविश्वेशस्य शिष्यः, वेदन्ते ब्रह्मानन्दसरस्वत्याश्शिष्यः अन्नम्भट्ट इति एतदीयग्रन्थपरिशीलनात् ज्ञायते। अयं काश्यां शास्त्राण्यधीती उवासेति विश्वेश्वरध्यानादवगम्यते । अन्नम्भट्टप्रशंसनपरा "काशीगमनमात्रेण नान्नम्भट्टायते द्विजः" इति किंवदन्त च प्रसिद्धा । अस्य शिष्यः वेदाद्रिसूरिरिति प्रसिद्धः वेदान्तपरिभाषाव्याख्यातत्वबोधिनीकारः । न्याये व्याकरणे मीमांसायाञ्चादसीयाः ग्रन्था प्रसिद्धाः । सप्तदशशतकीयोऽयं (1600 - 1700) अन्नम्भट्टः । नृसिम्हाश्रमिकृतस्य तत्वविवेकदीपनस्यापि व्याख्यानेन कृता ।। 
50. ब्रह्मसूत्ररत्नप्रकाशिका -
ग्रन्थोऽयं कलकत्तायां मुद्रिते ब्रह्मसूत्रव्याख्याने (Page 844) निर्दिष्टः । अस्य कर्ता अखण्डानुभूतिरिति च नाम निर्दष्टम् । ग्रन्थोऽयं कुत्र लभ्यत इति न ज्ञायते। 
51. लघुवार्तिकम् -
वेदान्तसूत्रलघुवार्तिकनामायं ग्रन्थ आनुष्ठुभेण छन्दसा विरचितः कुत्रचिदेकेन पादेन एकैकस्याधिकरणस्य कुत्रचिदेकेन श्लोकेन अधिकरणार्थस्य च संग्राहकः चौखाम्बामुद्रणालये मुद्रितः । अस्य व्याख्यापि मूलकृतैव कृता न्यायसुधाख्या । लघुवाक्यसुधा इत्यपि व्यवहारः । अस्य कर्ता उत्तमश्लोक इति शुद्धानन्दशिष्यस्स इति च ग्रन्थात् ज्ञायते । शुद्धानन्दस्य शिष्य आनन्दगिरिरिति च प्रसिद्धम् । यदि स एवायं शुद्धानन्दस्स्यात्तर्हि उत्तमश्लोक आनन्दगिरिसामयिकस्सतीर्थ्यश्चेति त्रयोदशशथकापरार्धादारब्धकालबासीति (1250 0- 1350 A.D.) निश्चेतुं शक्यते । श्रीकण्ठशास्त्री तु " लक्ष्मीधरगुरोर्नाम अनन्तानन्दगिरिरिति, स आनन्दगिरिरिति व्यवह्नियते । लक्ष्मीधरशिष्येषु शुद्धानन्दः कश्चन आसीत् । तस्य च स्वयस्प्रकाशशिशष्यः ।" इति भारतीयत्रैमासिकैतिहासिकपत्रिकायाश्चतुर्दशतमे भागे (IHQ Vol. XIV) प्रतिपादयति । यद्येवं शुद्धानन्दशिष्यौ स्वयम्प्रकाशउत्तप्तश्लोकौ इति स्वीक्रियते तर्हि चतुर्दशशकीयोऽयमिति (1400 A.D.) सिध्यति । वाराणसीवासी चायमुत्तमश्लोकः । 
(A) लघुवार्तिकव्याख्या - लघुन्यायसुधा -
लघुवार्तिकस्य व्याख्यात्मकोऽयं ग्रन्थः वाराणसीसंस्कृतग्रन्थमालायां मुद्रितः । अस्य कर्ता मूलग्रन्थकार उत्तमश्लोकश्शुद्धानन्दशिष्यस्त्रयोदशशतकीय इति प्रपञ्चितं पुरस्तात् । 
52. विद्वज्जनमनोहरी -
अमुद्रितोऽयं ग्रन्थः लन्दननगरस्थभारतकार्यालयपुस्तकालये (2267 DC Vol. IV IOL) बन्दरगार ओरियण्टलरिसर्च इन्सृिटयूट पुस्तकालये, कलकत्तासंस्कृतकलाशालापुस्तकालये च लभ्यते । अस्य कर्ता आनन्दाश्रमशिष्यः रङ्गनाथ इति ज्ञायते। " विद्यातीर्थकृतैश्शलोकैर्नृसिम्हाश्रमिसूक्तिभिः । सन्दृव्ध्वा व्याससूत्राणां वृ्त्तिर्भाष्यानुसारिणी ।" इति दर्शनात्तयोरर्वाचीन इति सिध्यति । आनन्दाश्रमोऽयमस्य गुरुर्यदि आनन्दरससागरकर्ता आनन्दाश्रमस्स्यात् तर्हि अस्य कालष्षोडशशतकादर्वाचीनः (`1600 - 1700 A.D.) इति निश्चेतुं शक्यते । भट्टोजिदीक्षितस्य भ्रातुः अद्वैतचिन्तामणिकर्तू रङ्गोजिभट्टस्यापि गुरोर्नाम आनन्दश्रमइत्येव । तस्मान् रङ्गोजिभट्ट एव रङ्गनाथो वेति संशय उदेति ।। 
53. विद्वन्मुखभूषणम् सव्याख्यम् -
शाङ्करभाष्यानुसारिणी अधिकरणार्थप्रदर्शिनीयं ब्रह्मसूत्रवृत्तिस्सव्याख्याऽमुद्रिता प्रथमाध्यायान्तं मद्रासराजकीयहस्तलिखितपुस्तकालये (R. 2325 MGOML) लभ्यते । अस्या व्याख्यापि मूलकृतैव कृता । अस्य कर्ता प्रयागकुलोत्पन्नः नरसिह्मार्यपुत्रः वेङ्कटाद्रिभट्टारक इत्यपरनामा वेङ्कटाद्रिसूरिरिति परं ज्ञायते । 
54. वेदान्तकौस्तुभः -
अमुद्रितोऽयं सूत्रवृत्तिग्रन्थः मद्रासरजकीयहस्तलिखितपुस्त कालये (R 4143 MGOML) लभ्यते । अस्य कर्ता कौण्डिन्यगोत्रजश्वोलदेशीयस्सीतारामः षोडशशतकापरार्धकालवासीति ज्ञायते। 
55. वेदान्तदीपिका - 
विषयविदग्धा इत्यपरनामायं ग्रन्थ अमुद्रितस्तिरुवनन्तपुरपुस्तकालये (356 TCD Vol III) दृश्यते । पद्यवद्धोऽयं ग्रन्थः भाष्यार्थसंग्राहकः चौखाम्बामुद्रणालये च मुद्रितः। अस्य कर्ता सभानाथसर्वक्रतुरित्यपरनामा अग्निहोत्रभट्टपितामहः नारायणशास्त्रिगणपत्यम्बयोः पुत्रः, नारायणसुब्रह्मण्य च पिता सुन्दरेश इत्यपरनामा चोलदेशीयः चोक्कनाथदीक्षितः। अस्य जामाता रामभद्रमखी । सप्तदशशतकीयोऽयं (1600 - 1700 A.D.) नल्लादीक्षितधर्मराजाध्वरीन्द्रसामयिकश्चेति ज्ञायते । वेङ्कटेश्वरदीक्षितशिष्योऽयं प्रसिद्धवैय्याकरणोऽपीति ज्ञायते। 
56. वेदान्तनयभूषणम् -
भामतीप्रस्थानानुयायी भामतीब्रह्मविद्याभरणार्थानां संग्राहकस्सूत्रवृत्तिरूपोऽय ग्रन्थ अमुद्रितः रायलआसियाटिकसोसाइटि कल्कत्तानगरे, शङ्करमठपुस्तकालये च लभ्यते । अस्य कर्ता विवरणोपन्यासकारस्य रामानन्दसरस्वत्याः प्रशिष्यः ब्रह्म विद्याभरणकारस्य अद्वैतानन्दस्य शिष्यः रत्नप्रभाभागव्याख्यातुरच्युतकृष्णानन्दस्य मानमालाकारस्य रामानन्दस्य च गुरुस्स्वयम्प्रकाशानन्दस्सप्तदशशतकीयः (1700 - 1800 A.D.) इति निश्चीयते । 
57. वेदान्तनवमालिका -
लघुवृत्तिरित्यपरनामायं ग्रन्थ  "ओरियण्टलपब्लिषिङ् हाउस " मद्रासनगरे मुद्रितः । अस्य कर्ता चोलदेशीयः तिरुविशनल्लूरग्रामाभिजनरामसुब्रह्मण्यशास्त्रिणश्शिष्य एकोनविंशतिशतकीय (1850 - 1910 A.D.) नीलमेघशास्त्रीति ज्ञायते ।
58. वेदान्तन्यायरत्नावलिः -
ब्रह्माद्वैतप्रकाशिकापरनामायं ग्रन्थ पुरुषोत्तमतीर्थकृत इति परं ज्ञायते । एनमधिकृत्य नान्यत्किमपि श्रूयते । ग्रन्थोऽयं शङ्करमठसूच्यां दृश्यते। 
59. वेदान्तभाष्यप्रतिपाद्यानि - 
शाङ्करभाष्यानुसार्ययं ग्रन्थ अमुद्रित उज्जैनसूच्यां दृश्यते । अस्यैव वेदान्तभाष्यप्रदीपोद्योतनमित्यपि नामेति केचिद्वदन्ति । अस्य कर्ता महाराष्ट्रविप्रकुलोत्पन्नश्शिवभट्टसतीदेव्योः पुत्रः, भट्टोजिदीक्षितपौत्रस्य हरिदीक्षितस्य शिष्यः वैद्यनाथपायुगुण्डे इत्याख्यस्य बालशर्मणः गुरुः, शृङ्गिवेरपुराधीशात् रामतो लब्धजीवकः अपरे वयसि प्राप्तसंन्यासाक्षमः प्रसिद्धः वैय्याकरणः सप्तदशशतकापरार्धारब्धकालवासी (1674 - 1754 A.D.) 
60. वेदान्तशोधनम् - अमुद्रितोऽयं वृत्तिग्रन्थ उज्जैतसूच्यां दृश्यते । अस्य कर्ता विदृलबुधभण्डारक इति परं ज्ञायते।
61. वेदान्तसूत्रभाष्यम् - ग्रन्थोऽयं विश्वनाथसिम्हदेवेन कृत इतिपरं ज्ञायते ।
62. वेदान्तसूत्रमुक्तावलिः - सूत्रवृत्तिरूपोऽयं ग्रन्थ आनन्दाश्रममुद्रणालये मुद्रितः। ग्रन्थेऽस्मिन् निर्णयदर्पणाख्यः ग्रन्थः निर्दिष्टः। अस्य कर्ता गद्यमयशारीरकमीमांसाभाष्यवार्तिककर्तुर्नारायतीर्थस्य परमानन्दतीर्थस्य च शिष्यः शिवरामानन्दगोपालानन्दगोविन्दानन्दसरस्वतीनां प्रशिष्यः बालकृष्णानन्दसरस्वत्याश्शास्त्रारम्भसमर्थनकारस्य त्र्यम्वकभट्टस्य वेदान्तसारव्याख्यातू रामचन्द्रानन्दसरस्वत्याश्च गुरुः गौडदेशीयस्सप्तदशशतकवासी (1600 - 1700 A.D.) गुरुचन्द्रिकाकारः ब्रह्मानन्दसरस्वतीति ज्ञायते । 
63. वेदान्तसूत्रार्थचन्द्रिका - ग्रन्थोऽयं केशवदेवकृत अमुद्रित अडयारपुस्तकालये लभ्यते । 
64. वैय्यासिकन्यायमाला -
व्यासनिर्मितेषु ब्रह्मसूत्रेषु शङ्कराभिमतानां अधिकरणानां वर्णनपरोऽयं ग्रन्थः प्रत्यधिकरणं श्लोकद्वयमिति सरणिमनुगच्छति । अस्या व्याख्यापि न्यायमालाविस्तराख्या मूलकृतैव कृता । अधिकरणरत्नमाला इत्यपि नामान्तरं दृश्यते । मुद्रितश्चायं ग्रन्थ आनन्दाश्रममुद्रणालये । अस्य कर्ता विद्यातीर्थशिष्यः, विद्यारण्य - ब्रह्मानन्दभारती-कृष्णानन्दभारतीनां गुरुस्रयोदशशतकापरार्धकालवासी (1280 - 1350 A.D.) भारतीतीर्थ इति ज्ञायते । 
(A) वैयासिकन्पायभालाव्याख्या - वैय्यासिकन्यायमालाविस्तरः 
ग्रन्थोऽयं आनन्दाश्रममुद्रणालये मुद्रितः । अस्य कर्ता मूलग्रन्थकार भार तीतीर्थ इति ज्ञायते । 
65. वैय्यासिकसूत्रोपन्यासः -
शाङ्करभाष्य - तट्टीकादिसारसंग्रहभूतोऽयं ग्रन्थ अमुद्रितः मद्रासराजकीयपुस्तकालये (D. 4693 MGOML) तिरुवनन्दपुरपुस्तकालये अडयारपुस्तकालये सरस्वतीमहालये मैसूरपुस्तकालये च लभ्यते । व्याससूत्रदीपिका वेदान्तसूत्रोपन्यास इति नामान्तराण्यस्यैव ग्रन्थस्य । अस्य कर्ता विद्याशङ्करप्रशिष्यः आत्माभारतीशिष्यः रामेश्वरभारतीति ज्ञायते । विद्याशङ्करस्यैव विद्यातीर्थ इति शृङ्गगिरिपीठारोहणात्पूर्वं प्रसिद्धिरासीत् । पीठारोहणादनन्तरं "शङ्करानन्दः" विद्याशङ्कर इति च प्रसिद्धिः श्रयते । यदि स एवायं विद्याशङ्करस्स्यात्तर्हि विद्याशङ्करपीठारोहणकालः (1280 A.D.) इति शृङ्गगिरिगुरुपरम्परया ज्ञायते । एवाञ्चायं विद्याशङ्करप्रशिष्यः रामेश्वरभारती चतुर्दशशतकावसाने (1320 - 1400 A.D.) स्यादिति निर्णेतुमर्हते । 
66. शारीरकचतुस्सूत्रीविचारः -
ग्रन्थोऽयं नाराशरापेट गुण्टूरनगरे मुद्रितः । अस्य कर्ता बेल्लङ्कोण्ड - रामराय आधुनिक इति अद्वैताचार्यप्रकरणे प्रतिपादयिष्यते । 
67. शारीरकन्यायनिर्णयः - 
"शारीरकमीमांसाभाष्यनिर्णयः" "भाष्यन्यायनिर्णयः" "भाष्यन्यायसंग्रह" इत्याद्यपरनामायं ग्रन्थः मद्रासविश्वविद्यालयसंस्कृतग्रन्थमालायां मुद्रितः । अस्य कर्ता स्वयम्प्रकाशानुभवापरनामक - अनन्यानुभवशिष्यः तत्वशुद्धिकारस्य ज्ञानघनस्य सामयिकः दशमशतकीयः (1000 A.D.) प्रकाशात्मा इति ज्ञायते । अधिकं एनमधिकृत्य अन्यत्र निरूपितम् ।
68. शारीरकमीमांसाशास्त्रसंग्रहः - 
जीवब्रह्मणोरैक्यप्रतिपादकोऽयं वृत्तिग्रन्थ अमुद्रितः मद्रासराजकीयहस्तलिखितपुस्तकालये (R. 2905) अडयारपुस्तकालये तिरुवनन्तपुर पुस्तकालये विश्वभारतीशान्तिनिकेतनपुस्तकालये च लभ्यते । अस्य कर्ता आनन्दानुभूतिशिष्यः विबुधेन्द्रयोगीत्यपरनामा कृष्णानुभूतियतिरिति ज्ञायते । केरलदेशजोऽयं कृष्णानुभूतियतिस्स्वसामयिकौ राजराजरविवर्मनामानौ स्वग्रन्थे निर्दिशति । केरलीयः वासुदेवकविनामा प्रसिद्धः रविवर्मणस्सभायामासीत् । यस्य च कालः पञ्चदशशतकमध्यभाग (1450 A.D.) इति कृष्णाचार्यरचितसंस्कृतसाहित्यचरिते 252 पुटे दृश्यते । यदि कृष्णानुभूतिनिर्दिष्ट एव रविवर्मा अयं स्यात्तर्हि अस्यापि कृष्णानुभूतियतेः कालः पञ्चदशशतकमिति (1450 - 1550 A.D.) निर्णेतुं शक्यते । मद्रासराजकीय स्तलिखितपुस्तकालयस्थे (R. 3305) ग्रन्थान्तरे विद्यमानं " गीर्वाणेन्द्रसरस्वत्याः पादाब्जं हृदि बिभ्रतः " इति पदं नृसिम्हाश्रमिगुरुं गीर्वाणेन्द्रं स्मारयति । 
69. शारीरकमीमांसासूत्रसिद्धान्तकौमुदी -
पद्यमयीयं वृत्तिः अमुद्रिता मद्रासराजकीयहस्तलिखितपुस्तरकालये (R. 3734 MGOML) लभ्यते। अस्यैव ग्रन्थस्य " तात्पर्यप्रकाशिका" "तात्पर्यार्थप्रकाशिका" "भाष्यार्थन्यायमाला" इति नामान्तराणि दृश्यते। ग्रन्थेऽस्मिन् विद्यारण्यः वैय्यासिकन्यायमाला चोद्धतौ । अस्य कर्ता सुब्रह्मण्याग्निचिन्मखीन्द्र इति परं ज्ञायते । अस्य कालादिनिर्णये प्रमाणं नोपलम्यते । चतुर्दशशतकादर्वाचीन इति तु निश्चयः ।
70. शारीरकरहस्यार्थप्रकाशिका -
शारीरकरहस्यार्थवस्तुतत्वप्रकाशिकापरनामायं अध्यायचतुष्टयैः पूर्णः शाङ्करभाष्यसारभूत अमुद्रितः वृत्तिग्रन्थः रायल आसियाटिक सोसाइटि बाम्बे पुस्तकालये बन्दरकार ओरियण्टलरिसर्च पुस्तकालये च लभ्यते । अस्य कर्ता उपदेशसाहस्रीव्याख्याता कृष्णतीर्थशिष्य अनन्तदेवप्रथमस्य पुरुषोत्तममिश्रस्य च गुरुर्नृसिम्हाश्रमिसामयिकः षोडशशतकीयः (1570 - 1670 A.D.) रामतीर्थ इति ज्ञायते। अस्यैव ग्रन्थस्य शारीरकशास्त्रसंग्रह इति नामान्तरं श्रूयते । 
71. शारीरकार्थसंक्षेपः -
ग्रन्थोऽयममुद्रित अडयारपुस्तकालये लभ्यते । अस्य कर्ता राघवार्य इति परं ज्ञायते । 
72. शास्त्रदर्पणम् -
अधिकरणरचनासु भामतीमतमनुसरन् भाष्यभामत्युक्तार्थसंग्राहकोऽयं वृत्तिग्रन्थः वाणीविलासमुद्रणालये मुद्रितः । अस्य कर्ता दीक्षाग्रहणे आनन्दात्म - अनुभवानन्दशिष्यः विद्यायां पुस्वप्रकाशशिष्यश्चित्सुखाचार्यप्रशिष्यः व्यासाश्रमापरनामा त्रयोदशशथकापरार्धकालिकः(1247 - 1347 A.D.) कल्पतरुकार अमलानन्द इति ज्ञायते । 
73. शङ्करपादभूषणम् -
व्याससूत्राणां प्रथमाध्यायस्य प्रथमः पादः, द्वितीयाध्यायस्य आद्यःपादश्च व्याख्यातः । प्रतिसूत्र शाङ्करमतं प्रदर्श्य आनन्दतीर्थीयद्वैतमतमनूद्य तदुद्भावितानि दूषणानि दूरीकृतानि शतशः कणशश्चैव खण्डितानि । नव्यतर्कशैल्या सम्पूर्णः वेदान्तार्थप्रतिपादक अद्वैतरत्नरक्षाकरण्डकोऽयं ग्रन्थ आनन्दाश्रमे मुद्रितः । अस्य कर्ता वासिष्ठगोत्रोत्पन्नः बहवृचशाखाध्यायी आश्वलायनसृत्री रामचन्द्रसूरिपुत्रः रामशास्त्रिपिता महाराष्ट्रदेशान्तर्गत वरहदृदेशीयः मीमांसान्यायपञ्चास्याराघवाचार्यशिष्यः एकोनविंशतिशतकीयः (1850 A.D.) रघुनाथसूरिः। अस्य गुरु राघवाचार्यः न मध्वमतानुयायी वैय्याकरणः नापि गजेन्द्रगड्करकुलोत्पन्न सतारापुरवासी, परन्त्वन्य इति बोध्यम् । ग्रन्थकारोऽयं रघुनाथसूरिः पुण्य(पूना)पत्तने राजकीयाङ्गलअधिकारि प्रार्थनावशात् न्यायाधीशपदमलञ्चकार। अस्य पुत्र रामशास्त्री अपिभोर राजधान्यां राजाश्रितः नीतिवक्ता वकीलः आसीत् । ग्रन्थकारोऽयं अपरे वयसि नीरातीरसमीपस्थ "भोर" राजधान्यन्तर्गतशासितुः अधीश्वरात् पण्डितनानासाहेबपन्तसचिवात् प्राप्ताश्रयः ग्रन्थञ्चकारेति ज्ञायते। अनेन भगवद्गीताया अपि शङ्करपदभूषणमिति व्याख्या कृता । 
74. शङ्कराशङ्करभाष्यविमर्शः -
ग्रन्थोऽयं तेलुगुलिप्यां मुद्रितः। अस्य कर्ता आन्ध्रदेशज एकोनविंशतिशतकीय (1875 A.D.) रामशास्त्रिसुब्रह्मण्यशास्त्रिशिष्यः बेल्लङ्गोण्डरामराय कविरिति ज्ञायते । 
75. समन्वयसूत्रवृत्तिः -
ब्रह्मसूत्रसमन्वयापरनामायं ग्रन्थः "हल्षरिपोर्ट आफ सान्स्कृट मानस्कृष्ट्" निर्दिश्यते । अस्य कर्ता अनूपनारायणतर्कशिरोमणिरिति परं ज्ञायते । 
76. सिद्धान्तार्णवशाङ्करब्रह्मसूत्रभाष्यव्याख्या - ग्रन्थोऽयं काञ्चीकामकोटिसूच्यां दृश्यते । अस्य कर्ता रघुनाथभट्टाचार्य इति परकं ज्ञायते । 
77. सुबोधिनी -
शारीरकसूत्रसारार्थचन्द्रिकापरनामायं ग्रन्थ अमुद्रितः लन्दननगरस्थभारतकार्यालयपुस्तकालये, उज्जयिन्यां च लभ्यते । अस्य कर्ता अग्निहोत्रिवीरेश्वरसूरिपौत्रः सदाशिवभट्टपुत्रः, महाङकरित्युपनामायं गङ्गाधरभगवद्भक्तकिङ्करः वत्सर्षिगोत्रज अष्टादशशतकीयः (1780 - 1880 A.D.) इति ज्ञायते । अस्य नाम्ना स्वाराज्यसिद्धि-वेदान्तसिद्धान्तसूक्तिमञ्जरीकारस्य गङ्गघरेन्द्रसरस्वत्याश्शिष्य इति ज्ञायते । 
78. सूत्रभाष्यसारसंग्रहः -
अमुद्रितोऽयं ग्रन्थ अडयारपुस्तकालये (27 D. 27 AL) लभ्यते । अस्य कर्ता वासुदेवेन्द्रशिष्यः कृष्णानन्द - उपनिषद्ब्रह्मेन्द्र - रामचन्द्रेन्द्राणां सतीर्थ्यंश्चन्द्रिकाचार्यगुरुः अष्टादशशतकीय (1800 - 1900 A.D.) रामब्रह्मेन्द्रसरस्वतीति ज्ञायते । 
79. सूत्रसंक्षिप्तवृत्तिः - 
अमुद्रितोऽयं ग्रन्थ पञ्जाबसूच्यां (718) दृश्यते । अस्य कर्ता रामधनाख्यः अनेन (1753 A.D.) काले ग्रन्थोऽयं कृत इति ज्ञायते । एतद्रीत्या ब्रह्मसूत्राणि शाङ्करभाष्यवत् (555) संख्यया संख्यातानि । 
80. सूत्रार्थामृतलहरी - सव्याख्या 
अमुद्रितोऽयं ग्रन्थः मद्रासराजकीयहस्तलिखितपुस्तकालये लभ्यते । अस्य कर्ता कृष्णावधूतपण्डित इति ज्ञायते । अस्य व्याख्यापि मूलकारेणैव कृता वरीवर्ति । 
81. सूत्रवृत्तिः - अमुद्रिताचेयं वृत्तिः मैसूरसूच्यां दृश्यते । अस्य कर्ता कौण्डिन्य इति ज्ञायते कौण्डिन्यवृत्तिरित्यपि नामान्तरम् ।
82. सूत्रवृत्तिः -
ग्रन्थोऽयं दासगुप्तमहाशयेन (H.I.P. Vol. II Page 81) निर्दिष्टः । शाङ्करभाष्यानुसारिण्या वृत्या अस्याः कर्ता वैद्यनाथ इति नाम निर्दिश्यते । कोऽय वैद्यनाथ परिभाषासंग्रहकार वैय्याकरणस्स्यात्तर्हि अस्य मातुलः रामभद्रदीक्षितः, वाधूलगोत्रजः, रामभद्रदीक्षितशिष्यः, कण्डरमाणिक्कग्रामवासी चोलदेशीयस्सप्तदशशतकापरार्धवासी (1650 - 1700 A.D.) इति निश्चीयते ।  
83. सूत्रवृत्तिः - ग्रन्थोऽयं मैसूरपुस्तकालये लभ्यते । अस्य कर्ता वेङ्कामात्यापरनामा चिन्मयमुनिरिति परं ज्ञायते । 
84. सूत्रवृत्तिः -  
अमुद्रितोऽयं ग्रन्थः बरोडापुस्तकालये (12717 BRD) लन्दननगरस्थभारतकार्यालयसूच्याञ्च दृश्यते । अस्य कर्ता रामगोविन्दशिष्यः वासुदेवतीर्थशिष्यः ब्रह्मानन्दसरस्वतीगुरुः सप्तदशशत्तकवासी (1600 - 1700 A.D.) नारायणतीर्थ इति ज्ञायते । 
85. सूत्रवृत्तिः - ग्रन्थोऽयं भारतकार्यालयपुस्तकालये लन्दननगरे लक्ष्यते । अस्य कर्ता प्रकाशात्मा इत्यपि ज्ञायते ।
86. सूत्रवृत्तिः - 
ग्रन्थोऽयं आनन्दाश्रमुद्रणालये मुद्रितः । अस्य कर्ता लक्ष्मीनरहरिदीक्षितपुत्रः रामरायगुरुः हरिदीक्षितः । अनेन (1658 श 1736 A.D.) काले ग्रन्थोऽयमारचित इति अष्टादशशतकीय इति ज्ञायते । 
87. सूत्रवृत्तिः -
ग्रन्थोऽयं अद्वैतमञ्जरी ग्रन्थमालायां कुम्भधोणनगरे मुद्रितः अस्य कर्ता शङ्कराचार्यशिष्यः, सुरेश्वर इति च केषाञ्चिन्मतम् । 
88. सूत्रवृत्तिः -
ग्रन्थोऽयं दासगुप्तमहाशयेन (H.I.P. Vol. II Page 81) निर्दिष्टः। अस्य कर्ता देवराप्रभट्ट इति च प्रतिपादितम् । 
89. सूत्रवृत्तिः - 
ब्रह्ममीमांसात्रिशत्यपरनामायं ग्रन्थ अज्ञातकर्तृकः सूत्रार्थसंग्राहकः चौखाम्बामुद्रणालये मुद्रितः । 
90. सूत्रवृत्तिः -
ग्रन्थोऽयममुद्रितः पञ्चाबसूच्यां 719 दृश्यते । अस्य कर्ता शुकभगवत्पाद इति च निर्दिश्यते । कोऽयं शुकभगवत्पाद इति निर्णये विशिष्टं प्रमाण किमपि न दृश्यते । 
91. सूत्रवृत्तिः -
यामुनाचार्येण सिद्धित्रये अनुपादेयत्वेन प्रदर्श्यमानोऽयं ग्रन्थः भर्तृहरिकृतः अद्वैतमतपोषकस्स्यादित्यूह्यते । ग्रन्थोऽयं यद्यपि नोपलभ्यते तथापि शब्दाद्वैतवादी भर्तृहरिरिति प्रसिद्धेर्ब्रह्मसूत्राणां वृत्तिरचनायास्सम्भवः। भर्तृहरेः कालष्षष्ठशतकात् पञ्चमशतकाद्वा प्राचीन इति ज्ञायते । अधिकं भर्तृहरिप्रकरणे निरूपितम् । 
एवं सूत्रप्रस्थाने - गणपतिशास्त्रिकृत अद्वैतमतावलम्बी "अथ शब्दार्थविचारः", कौण्डिन्यकृता रामभद्रविद्वत्कृता, नृसिम्हाश्रमिशिष्यरामाश्रमिकृताश्च सूत्रवृत्तयः भारतकार्यालयस्थापुस्तकालये लन्दननगरस्थे दृश्यन्ते । एवं नारायणकृता न्यायनिर्णयसंग्रहनाम्नी वृत्तिः, अज्ञातकर्तृनामानौ "अद्वैतरत्नकोश" - 'आमोद'नामानौ द्वौ वृत्तिग्रन्थौ च मैसूरसूच्यां दृश्येते । शान्तिनिकेतनपुस्तकालये कृष्णे न्द्रानुभूतिकृता ब्रह्मसूत्रभाष्यव्याख्या काचन विद्यते । "काटलागस काटलागर" मिति ग्रन्थे ज्ञानेन्द्रयतिकृता "दीपिका" नागेशभट्टकृतः "सूत्रेन्दुशेखरः" रामानन्दतीर्थकृतं "वेदान्तसूत्ररत्नं" आनन्दपूर्णकृता "सामान्यसूत्रवृत्तिश्च" दृश्यन्ते । एतेषां ग्रन्थानां प्राप्तिस्थानादिकं न ज्ञातुं पार्यते ।  
