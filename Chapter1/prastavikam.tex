\chapter{प्रास्ताविकम्}

\begin{center}
 \textbf{।। ओं नमो ब्रह्मादिभ्यो ब्रह्मविद्यासंप्रदायकर्तृभ्यो वंशऋषिभ्यो नमो \\ महद्भ्यो नमो गुरुभ्यः ।।}
\end{center}

\begin{center}
\begin{enumerate}
\item
नमः श्रुतिशिरःपद्मषण्डमार्तण्डमूर्तये ।\\
 वादरायणसंज्ञाय मुनये शमवेश्मने ।।
\item
 ब्रह्मसृत्रकृते तस्मै वेदव्यासाय वेधसे ।\\
 ज्ञानशक्त्यवताराय नमो भगवतो हरेः ।।
\item
 शङ्करं शङ्कराचार्यं केशवं वादरायणम् ।\\
 सृत्रभाष्यकृतौ वन्दे भगवन्तौ पुनः पुनः ।।
 \end{enumerate}
\end{center} 

 (४) श्रुतिस्मृतिपुराणानामालयं करुणालयम् ।\\
 नमामि भगवत्पादं शङ्करं लोकशङ्करम् ।। 
 
 (५) वेदान्ताम्भोगभीरा नयमकरकुला ब्रह्मविद्याव्जषण्डा \\
 पाषण्डोत्तुङ्गवृक्षप्रमथननिपुणा मानवीचीतरङ्गा ।\\
 यस्यास्योत्था सरस्वत्यखिलभवभयध्वंसिनी शङ्करस्य \\
 गङ्गा शम्भोः कपर्दादिव निखिलगुरोर्नौमि तत्पादपद्मम् ।। 
 
 (६) वेदान्तार्थं गभीरं ह्यतिसुगमतया बोधयामीति विष्णु-\\
 र्व्यासात्माऽसूत्रयत तद् दुरधिगमममूद् वादिदुर्बुद्घिभेदात् ।\\
 भिन्दन् दुर्बुद्धिभेदं य इह करुणयाऽभाष्ययद् भाष्यमेतत् \\
 तं वन्दे सर्ववन्द्यं त्रिजगति भगवत्पादसंज्ञं महेशम् ।। 
 
 (७) नारायणं पद्मभुवं वसिष्ठं शक्तिं च तत्पुत्रपराशरं च । \\
 व्यासं शुकं गौडपदं महान्तं गोविन्दयोगीन्द्रमथास्य शिष्यम् ।\\
श्रीशङ्कराचार्यमथास्य पद्मपादञ्च हस्तामलकञ्च शिष्यम् ।\\
तं तोटकं वार्तिककारमन्यान् अस्मद्गुरून् सन्ततप्रानतोऽस्मि ।। 

(८) सदाशिवसमारम्भां शङ्कराचार्यमध्यमाम् ।
अस्मदाचार्यपर्यन्तां वन्दे गुरुपदम्पराम् ।।


\section{किञ्चित् प्रास्ताविकम् }
संशोधनाय ग्रन्थरचनाय च स्वीकृतोऽयं विषयः - अद्वैतवेदान्तसाहित्येतिहासकोशः - ( A Bibliographical History of Advaita Vedanta literature ) इति । तत्सम्पादने मयानुमूतस्य कृतस्य च परिश्रमस्य फलरूपोऽय विषयः ग्रन्थरूपेण भवतां सन्निधिमागच्छति । 

सर्वदर्शनोत्तमस्य अद्वैतदर्शनस्य विकासकालः प्राचीन मध्यम - आधुनिकभेदेन त्रेधा विभक्तुं शक्यते । तत्र प्राचीनो भागः दशमशतकान्तः, मध्यमो भागः षोडशशतकान्तः, आधुनिकभागस्तु अद्यावधिक इतः परश्च। प्राचीने काले शाङ्करभाष्यस्य व्याख्यानभूतानां ग्रान्थानाम् , भामतीप्रस्थान्स्य विवरणप्रस्थानस्य च वीजावापः, संख्येयानां स्वतन्त्रग्रन्थानाञ्च आविर्भावो दृश्यते । न्यायशैलीनिब्द्धाः न्यायवैशेषिकादिमतसिद्धभेदवादखण्डनपरा अद्वैतग्रन्थास्स्वतन्त्रभूताः मध्यम एव काले प्रादुरभृवन् ।

\section{अद्वैतसिद्धान्तः}
शाङ्करे अद्वैतमते ब्रह्मैव सत्यम् । अन्यत्सर्वं मिथ्या । जगदादि सर्वं शुक्तौ रजतमिव भासते । ब्रह्म निर्गुणम् , अत एव शब्दैरप्रतिपाद्यम् । अत एव वाड्मनसयोरगोचरमिति श्रुतौ कथ्यते । सर्वज्ञत्वादयो गुणा अपि ब्रह्मणि औपाधिका एव । जीवो ब्रह्मरूपोऽपि अज्ञानात् भिन्न इव भाति। मनोबुद्ध्याद्युपाधिभिस्सर्वोऽपि जनः "अह" मिति प्रत्येति । अत एव सुषुप्तौ मनोबुद्ध्यादिलयेन अहमाकारप्रतीत्यभावः। एवं मोक्षावस्थायामपि उपाधिलयात् अहमाकारप्रतीत्यभावः। एवञ्चाहमिति प्रतीतिरौपाधिकी । अयं प्रातीतिको जीव एक एव न तु नाना । अन्तःकरणभेदात् सुखदुःखानुभवभेदः। चैतन्यं सर्वत्र्यापकं स एव आत्मा इत्यभिधीयते । अयमात्मा ईश्वर - जीव - साक्षीति भेदेन त्रिविधः। सर्वजगन्मूलकारणम्  अज्ञानम् , तस्मिन् चैतन्यान्तर्गते सति स आत्मा ईशअवर इत्यभिधीयते । स एव ईश्वरस्सत्त्वरजस्तमोभिर्विप्णु ब्रह्म - शङ्कराख्याः लभते । आत्मा ज्ञानस्वरूपो न तु ज्ञातृस्वरूपः । तस्य ज्ञातृत्वमहङ्काराद्युपाधिभिः । प्रपञ्चस्यापि ज्ञेयत्वमज्ञानादेव । त्रिविधं सत्यत्वं - प्रातिभासिकं व्यावहारिकं पारमार्थिकञ्च। तत्र प्रातिभासिकस्य शुक्तिरजतादेः व्यावहारिकसत्यरजतेन बाधः। व्यावहारिकसत्यत्वं घटपटादीनाम् , तच्च पारमार्थिकसत्येन ब्रह्मदर्शनेन बाध्यते । ब्रह्मणस्तु कस्यामप्यवस्थायां न बाधः । अत एव तत् परमार्थसत् । धटपटदीनां व्यवहारदशायां सत्यत्वम् । वस्तुतस्तु ते स्वाप्निकपदार्थवत् मिथ्याभूताः । यथा स्वप्नगताः पदार्थाः जाग्रद्दशायां न सन्ति अतो मिथ्या, तथा पारमार्थिकदशायां अभावात् एते व्यवहारदशायां विद्यनाना अपुि घटपटादयो मिथ्या । पारमार्थिकात्मज्ञानेन वाधसम्भवात् ।

कासुचित् श्रुतिषु सृष्टिराकाशक्रमेण, कासुचित् श्रुतिषु तेजाआदिक्रमेण, कुत्रचित् सर्वं ब्रह्मैव नान्यदित्युक्तम् । सर्वासामपि श्रुतानां अवाधितप्रामाण्यात् तत्समन्वयोऽवश्यं कर्तव्यः । तथा च इदमेव सिध्यति यत् ब्रह्मैव सत्यमिति पारमार्थिकदृष्ट्या उक्तम् । आकाशादिकमसृष्टिस्तु व्यावहारिकी । अनेनापि प्रमाणेन जगतो व्यावहारिकत्वम्, न परमार्थसत्यत्वम् । यदि जगत् परमार्थसत् स्यात् तर्हि अन्यत् किमपि नास्तीति प्रतिपादिनी श्रुतिरनर्थिका स्यात् । एवञ्च सत्यानृते मिथुनीकृत्य नैसर्गिकोऽयं लोकव्यवहार इति मन्तव्यम् । अत एव अज्ञानिदृष्ट्या तत्सत्यत्वं भासते । ज्ञानिदृष्ट्या च " तस्य पिता अपिता भवति " इत्युक्तरीत्या तस्य मिथ्यात्वम् । न केवलं सर्वस्य जगतो मिथ्यात्वं श्रुत्या वोध्यते, अपि तु स्वस्यापि मिथ्यात्वं श्रुतिर्निरभिमानितया ब्रूते ।

एतस्य सर्वस्य जगतः मूलस्वरूपमनाद्यविद्या । सा च त्रिगुणात्मिका । प्रलयकालसमाप्तिवेलायां अविद्यया जीवकृतकर्मभिश्च " तदैक्षत बहुस्यां प्रजायेय " इति रीत्या परमात्मा संकल्पयति, तत आकाशादिक्रमेण सृष्ट्युद्गमः । ततश्च पञ्चीकृतमहाभूतेभ्यः शरीरादीनि प्रादुर्मवन्ति । एवं उत्पन्नशरीरे प्रविष्टं चैतन्यं जीव इत्यभिधीयते । स च  न अणुः, किन्तु व्यापकः, सर्वस्मिन् शरीरे सुखदुःखानुभवात् । स च जीवः मुक्तिपर्यन्तं स्थायी, जन्मान्तरीयकर्मजसुखदुःखसम्बन्धात् । जीवः विस्मृतकण्ठस्थचामीकरः पुरुष इव आत्मविस्मृतेः, अज्ञानात् सुखदुःखानुभवभाक् । अज्ञानं, लिङ्गशरीरम्, स्थूलशरीरञ्चेति उपाधिस्तस्य सदा प्रत्यासन्नः। अयमुपाधिः " दशमस्त्वमसीति " ज्ञानेन विस्मृतात्मस्वरूपस्य अज्ञानमिव, आत्मज्ञानेन जीवात्मपरमात्मैक्यज्ञानापरपर्यायेण नश्यति, सत्यज्ञानेन विना मिथ्याभूतदर्शनस्यानिवृतेः । एवमविद्यानाशे जीवो मुक्तो भवति । तस्य ज्ञानाग्निः प्रारब्धेतराणि सञ्चितक्रियमाणानि कर्माणि नाशयति । दशायामस्यां प्रारब्धकर्माणि भुञ्जानः स विगतशरीराद्यभिमानो जीवन्मुक्त इत्यभिवीयते । प्रारब्धकर्मावसाने देहपाते स विदेहमुक्तो भवति। इयमेव परमा मुक्तिः अस्यां जीवः परमात्मसायुज्यं नाम स्वरूपमधिगच्छति । न च मतान्तरवत् अल्पेवापि सेवकादिरूपेण भिन्नस्तिष्ठति।। यद्यपि जीवस्सदा आत्मस्वरूप एव तथापि भावरूपेणाज्ञानेन आत्मानं सुखदुःखभाजं मनुते तदैव वद्ध इत्यभिधीयते, तादृशाज्ञाननिवृत्तौ स एव मुक्त इत्युच्यते ।

वेदविहितकर्मभिश्चित्तशुद्धिः ततो ब्रह्मनिष्ठं गुरुं प्रतिशरणगमनम्, तेन तत्त्वमसीतिज्ञानोपदेशः, तच्छ्रवणमनननिदिध्यासनैः सन्यासाश्रममधिवसन् कर्माणि परित्यजन् स सुस्थिरज्ञानो मोक्षं लभते । मोक्षाप्तौ ज्ञानमेव साक्षात् साधनम् । कर्मोपासने तु चित्तशुद्धिचित्तैकाग्र्यप्रापकत्वात् परम्परितसाधने । ज्ञानकर्मसमुच्चयस्तु नेष्टः । नित्यानित्यवस्तुविवेकः, इहामुत्रार्थभोगविरागः शमदमादिसाधनसम्पत् , मुमुक्षुत्वञ्चेति साधनचतुष्टसम्पत्यनन्तरं ब्रह्मजिज्ञासा । अस्मिन् मते अनि र्वचनीयख्यातिः। प्रत्यक्षानुनानोपमानशाब्दार्थापत्यनुपलब्धि - आख्यानि षट् प्रमाणानि । अलौकिकेऽर्थे आत्मनि शाब्दमेव मुख्यं प्रमाणम् । अन्यदनुमानादि तदवष्टम्भेन प्रमाणम् । मतस्यास्य संग्राहकः श्लोकः " ब्रह्म सत्यं जगन्मिथ्या जीवो ब्रह्मैव नापरः " इति । 

\section{अद्वैतसिद्धान्तविकासः}
 एतादृशस्य श्रुतिसिद्धस्य प्राचीनतमस्य अज्ञविज्ञोपकारकस्य सकलदर्शन शिरोऽलङ्काररत्नभूतस्य अद्वैतवेदान्तदर्शनस्य विशेषतः प्रबलप्रतिवादिनः भेदवादिनः न्यायवैशेषिकादयः, शालिकनाथप्रभृतयः पूर्वमीप्रांसकाः आसन् । त्रयोदशशतका दनन्तरं द्वैतमतप्रवर्तका आनन्दतीर्था अपि प्रतिवादिषु अन्यतमा अभूवन् । एतादृशैस्तर्काभिमानिभिर्वैशेषिकादिभिर्युकत्याभासैः कलुषितस्य आनन्दतीर्थादिभिर्निन्दितस्य अद्वैतसिद्धान्तस्य रक्षणार्थं तर्केणैव समाधानार्थं च प्रवृत्तेषु ग्रन्थेषु चित्सुखीयन्यायमकरन्द न्यायदीपावली - अद्वैतसिद्धि - पञ्चदशीत्यादयः खण्डनखण्डस्वाद्यमित्यादयश्च विशिऽय उल्लेवार्हाः । इतरमतखण्डनपरग्रन्थेषु शैलीद्रयं दृश्यते । तत्र प्राचीना शैली तु युक्तिसहिता अर्थगाम्भीर्यवती श्रुतिमधुरशब्दविन्यासयुक्तः सम्भाषणसमा दृश्यते । नव्यनैय्यायिकैः गङ्गेशोपाध्यायप्रभृतिभिः नवीनतया न्यायशास्त्रपरिवर्तने कृते भेदवादखण्डनार्थं प्रवृत्ताः मधुसूदनसरस्वतीप्रभृतयः ब्रह्मानन्दसरस्वत्यन्ता आचार्याः " यक्षानुरूपो बलि " रिति न्यायेन परिष्कारप्रधानान्येव  वाक्यान्यारचय्य इतरमतखण्डने प्रवृत्ताः। सेयं नवीना शैली। चित्सुख - आनन्दबोधादीनां शैली तु प्राचीना । नैतच्छल्यां काठिन्यं दृश्यते । नापि विषयप्रतिपादने युक्तिकथने शब्दविन्यासे वा मन्दता दृश्यते । चाटुकारीणि दृढयुक्तिकानि न्यायोपबंहितानि च वाक्यानि तेषां वाग्मितां सिद्धान्तविमर्शक्षमताञ्च प्रतिपादयन्ति । तत्र मेदवादिनामयमाक्षेपः प्रवलतरः - यत् - जगतो मिथ्यात्वे मायिकत्वे वा तस्य असत्वप्रसङ्गः । यथा च शुक्तिरजतं असत् भवति तथा जगदपि असत् भवेत् । तथा सति सर्वप्रमाणव्यवहारस्य असम्भवस्त्यात् । किञ्चैवं सति वैदिकत्वज्ञानस्यापि मिथ्यात्वापत्तिः। सर्वदृश्यान्तर्गतस्य वेदस्य मिथ्यात्वेन तदुक्तस्य तत्त्वज्ञानस्य अर्थादेव मिथ्यात्वं स्यात् इति । सोऽयमाक्षेपः विचार्य विकल्प्य दूरीकृत आचार्यैः। असत्वमित्यस्य कोऽर्थः ? किं अत्यन्तासत्वम् ? उत किञ्चित्कालासत्वम् ? । नाद्यः, प्रत्यक्षं दृश्यमानस्य अर्थक्रियाकारिणो जगतः शशश्रृङ्गवत् अत्यन्तासत्वानुपपत्तेः । तद्धि अत्यन्तासत् यत् कदापि केनापि नोपलब्घम् , न तु जगत् तथा, सर्वैरपि प्रत्यक्षमुपलभ्यमानत्वात् । तस्मात् जगत् न असत् । नापि सत्वेन स्वीकर्तुं शक्यते । नहि प्रत्यक्षेण दृश्यमानत्वं अर्थक्रियाकारित्वं वा अत्यन्तसत्वस्य प्रयोजकम्, तथा सति स्वाप्नादार्थानां प्रत्यक्षं अनुमूयमानत्वेन, मनोरथानां अर्थक्रियाकारित्वेन च व्यभिचारात् । तस्मात् न शशश्रृङ्गदिवत् अत्यन्तासत् । नापि आत्मवत् अत्यन्तसत् , यद्रूपेण यत् निश्वितं तद्रूपं न व्यमिचरति तत् सत्यम् इति प्रतिपादितं सत्यत्वं यस्य भवति तदेव सत्यमिति वक्तव्यम् । प्रतिक्षणपरिणामिनः सततवञ्चलस्वभावस्य नियतपरिवर्तनशीलस्य अस्य जगतः तद्गतपदार्थानां वा निरुक्तसत्यत्वलक्षणानाक्रान्तत्वात् । किन्तु सद्विविक्तत्वम् , अथवा सदसद्वि लक्षणत्वं वा मिथ्यात्वमिति वक्तव्यम् । इदमेव मिथ्यात्वलक्षणं
 परिष्कृतं न्यायरत्नदीपवली तत्वप्रदीपिका - तर्कसंग्रह (आनन्दगिरीय) प्रमाणमालादिषु ग्रन्थेषु । लघुचन्द्रिकायाञ्च -
आद्यं स्यात् पञ्चपाद्युक्तं ततो विवरणोदिते । 
चित्सुखीयं चतुर्थं तु अन्त्यमानन्दबोधजम् ।। इति । 
अद्वैतसिद्धान्तस्यास्य उपनिषद् गीता - सूत्ररूपाणि श्रुति - स्मृति - युक्तिताम्ना व्यवहृतानि प्रस्थानानि महाप्रस्थ नानि प्रस्थानत्रयमिति च प्रसिद्धानि । तथा भामतीप्रस्थानं विवरणप्रस्थानमिति च अवान्तरप्रस्थानं सिद्धान्तप्रतिपादनभेदमूलकं प्रसिद्धं विद्यते, एतत् प्रस्थानद्वयमपि - 
यया यया भवेत् पुंपां व्युत्पत्तिः प्रत्यगात्मनि ।
सा सैव प्रक्रियेह स्यात् साध्वी सा चानवस्थिता ।।
इति सुरेश्वरोक्त्या मान्यतां प्राप्नोत्येव । तयोस्सिद्धान्तभेदास्तु एवं दृश्यन्ते - 

भामतीकारः कर्मणां विविदिषार्थत्वं वदति, विवरणकारस्तु कर्मणां विद्यार्थत्वं वदति। भामतीकारः ब्रह्मसाक्षात्कारः मनःकरणकः, न तु शब्दाकरणक इति वदति। विवरणकारस्तु वेदान्तवाक्यारूपशब्दकरणक इति वदति । भामतीकारः श्रोतव्यो मन्तव्य इत्य़ादिवाक्ये विधिर्नास्तीति वदति, विधिरस्तीति विवरणकारः । भामतीकारः निदिध्यासंन अङ्गि, श्रवणमनने अङ्गमूते इति वदति । श्रवणं अङ्गि मनननिदिध्यासने अङ्गभूते इति विवरणकारः । जीवेश्वरविषये भामतीकार अवच्छेइवादी, विवरणकारः प्रतिविम्बवादी भवति । भामतीकार अज्ञानाश्रयस्य अज्ञानविषयस्य च भेदं स्वीकरोति, विवरणकार अज्ञानस्य आश्रयविषयभेदो नास्तीति वदति । भामतीकारः प्रतिजीवं मूलाविद्या नाना इति वदति । विवरणकारः मूलाविद्या एकैवेति वदति । भामतीकार अखण्डाकारवृत्तेरूपहितं ब्रह्म विषय इति वदति। विवरणकार अखण्डाकारवृत्तेः शुद्धं ब्रह्म विषय इति वदति। भामतीकारस्य मते साधनचतुष्टये सत्यासत्यवस्तुविवेकः प्रथमसाधनम् , विवरणकारस्य मते नित्यानित्यवस्तुविवेकः प्रथमसाधनम् । स्वाध्यायोऽध्येतव्य इति विधिरर्थावबोधफलक इति भामती, अक्षरग्रहणफलक इति विवरणम् । भामती त्रिवृत्करणम् , विवरणं पञ्चीकरणं च स्वीकरोति । भामतीकारः ब्रह्मणः सर्वज्ञत्वं स्वरूपचैतन्येनेति वदति, विवरणकारः ब्रह्मणः सर्वज्ञत्वं मायावृत्तिभिरिति । भामतीकारस्य मते मनस इन्द्रियत्वमस्ति, विवरणकारस्य मते मनस इन्द्रियत्वं नास्ति । भामतीमतेे अविद्या जीवाश्रिता भवति, विवरणमते अविद्या ईश्वाराश्रिता भवति ।

	अविद्यया संसक्तं अविद्यारूपिणा अपाधिना सहितञ्च ब्रह्मणः विशुद्धं चैतन्यमेव जीव इत्युच्यते । प्रतिजीवं एकं अन्तःकरणं उपाधिर्भवति । अतश्च जीवः परिच्छिन्न अल्पज्ञ इति व्यवहारः ।
विवरणमते अन्तःकरणतत्संस्कारावच्छिन्नाज्ञानप्रतिविम्बितं चैतन्यं जीव इति, अन्तःकरणे ब्रह्मप्रतिबिम्बमेव जीव इति भवति। जीवस्वरूपविषये अवच्छेदवादः, आभासवादः, प्रतिविम्ववादश्चेति पक्षाःसन्ति - 
वाचस्पतेरवच्छिन्न आभासो वार्तिकस्य च ।
संक्षेपशारीरककृतः प्रतिविम्बं तथेष्यते ।। इति ।। 
तत्र अवच्छेदो नाम अन्तःप्रवेशः । तेन युक्त अवच्छिन्नः । यथा वा जले अन्तःप्रविष्टं आकाशं जलावच्छिन्नमित्युच्यते एवं अज्ञानाश्रयीभूतं शुद्धचैतन्यं जीव इत्युच्यते । येषां मते अविद्यया संयुक्तं चैतन्यं जीवः, स च अवच्छिन्नो वा उपहितो वा प्रतिबिम्बितो वा तेषां मते जीवावस्थायां जीवस्य एकत्वम् । अविद्याया एकत्वात् । अयमेव एकजीववादः । सुखदुःखादिवैचित्र्यन्तु उपाधिभेदात् भवति । अयमेकजीववादः भामतीकाराणां मते । बहुप्रकारया अविद्यया अविद्याकार्यबुद्ध्या वा संयुक्तं चैतन्यं जीवः, स च जीवः अविच्छिन्नो वा, उपहितः, प्रतिविम्बितो वा, भवति इति ये वदन्ति तेषां मते जीवनानात्वम् । वार्तिककारसुरेश्वरः पञ्चपादिकाविवरणकारप्रकाशात्मा संक्षेपशारीरककारसर्वज्ञात्मा प्रकटार्थविवरणकारश्व बहुजीववादिनः । न्यायमकरन्दकारानन्दबोधश्च सर्वेषामाचार्याणां सिद्धान्तं सविमर्शं निरूप्य, विषेषतः ब्रह्मसिद्धि - इष्टसिद्धिसिद्धान्तं मण्डयन् एकजीववादे श्रुतिप्रामाण्यं प्रदर्श्य युक्तियुक्ततां साधयति । 

%शाङ्करभाष्यम् 
 
एवं निर्विशेषात र्न्लिक्षणाच्च स्वयग्प्रकाशात् ब्रह्मणः सविशेषस्य सलक्षणस्य श्च जगतः कथमुत्पत्तिः? एकस्मात् अद्वैतात् ब्रह्मणः नानात्मकस्य जगतः कथं सृष्टिः ? इत्येतेषां प्रश्नानां यथावत्
समाधनं मायास्वरूपवर्णनद्वारा आचार्यैरूपवर्णितम् । शङ्करभगवत्पादैः माया - अविद्या अज्ञान - अव्यक्तदि शब्दानां समानार्थकत्वं सूत्रभाष्ये ( १ -४ -३) उपवर्णितम् । शङ्करभगवत्पादादर्वाचीनेषु आचार्येषु माया - अविद्ययोस्तारतम्यविषये जीवेश्वरस्वरूपविषये अविद्याया आश्रयत्वे मायाया आश्रयत्वे च सिद्धान्तभेदा उद्भाविताः । भावतीकारः जीवेश्वरविषये अवच्छेदवादं अज्ञानाश्रयस्य अज्ञानविषयस्य च भेदं स्वीकुर्वन्ति । विवरणकाराश्च जीवेश्वरविषये प्रतिविम्ववादं अज्ञानाश्रयत्वविषयत्वयोर्भेदाभावं च स्वीकुर्वन्ति । भामतीमते अज्ञानविषयीकृतं चैतन्यं ईश्वरः । अज्ञानाश्रयीभूतं चैतन्यं जीव इति । विवरणप्रस्थाने ईश्वरविषये आभासवादं, जीवविषये प्रतिबिम्बवादं च स्वीकृत्य अज्ञानोपहितं विम्वचैतन्यं ईश्वरः, अन्तःकरणतत्संस्कारावच्छिन्नाज्ञानप्रतिबिम्बितं चैतन्यं जीव इति स्वीक्रियते । एवं माया - अविद्याशब्दयोस्सूक्ष्ममर्थभेदं वर्णयन्ति। जीवत्व - ईश्वरत्वप्रापकोपाधिस्तु भावरूपं त्रिगुणात्मकं सदसद्भ्यां अनिर्वचनीयं अनादि अज्ञानम् । तच्चाज्ञानं माया - अविद्याभेदेन द्विविधम् । शुद्धसत्वप्रधान मायापदवाच्यम् । मलिनसत्वप्रधानं अविद्यापदवाच्यम् । मायोपहिंत चैतन्यं ईश्वरः । अविद्योपहिंत चैतन्यं जीवः ।

मायारहिते परमेश्वरे प्रवृत्तिर्न भवति । मायाशक्तिरहितः परमेश्वरः जगत्सर्जने अशक्तो भवति। मायाशक्तिरेव अविद्यात्मिका बीजशक्तिरव्यक्तनाम्ना व्यवह्रियते । मायेयं परमेराश्रया भवति। 
अग्नेरपृयक्मूता दाहिका शक्तिरिव माया ब्रह्मण अपृथक्मूता शक्तिः। मायेयं ब्रह्मण एकदेशवर्तिनी न तु कृत्स्नवर्तिनी । 

न कृत्स्नवृत्तिः सा शक्तिस्तस्य किन्त्वेकदेशभाक् । \\
घटशक्तिर्यथा भूमौ स्निग्धमृद्येव वर्तते ।। \\
पादोऽस्य विश्वा भूतानि त्रिपादस्ति स्वयम्प्रभः ।\\
इत्येकदेशवृत्तित्वं मायाया वदति श्रुतिः ।। (पञ्चदशी)\\
 
"पादोऽस्य विश्वा भूतानि, त्रिपादस्यामृतं दिवि " इति श्रुतिश्व प्रतिपादयति । 
सत्वरजस्तमोगुणात्मिकेयं माया ज्ञानविरोधिनी भावरूपः पदार्थः। माया न सती । नापि असती । सदसदुभयविलक्षणतया शास्रे सा अनिर्वचनीयेति कथ्यते । ब्रह्मज्ञानेन बाधसम्भवात् माया न सती । त्रिकालावाधितत्वं हि सत्वम् । यदि माया सती स्यात् तर्हि तस्या अवाधितत्वमेव स्यात् । ब्रह्मज्ञानेन तु  माया बाध्यते। तस्मात् सा न सतीति वक्तुं शक्यते । परन्तु तस्याः प्रतीतिरनुमूयते । तस्मात् असतीति च न वक्तुं शक्यते । यतोऽसत् वस्तु न प्रतीयेत। एवञ्च मायायां उभयविरुद्धयोर्बधितत्वप्रतीतिविषयत्वरूपयोः गुणयोरनुभवगम्यत्वात् माया अनिर्वचनीयेति निश्चीयते । मायाभिधाया अविद्यात्वं प्रमाणासहिप्णुत्वमेव । तर्कादिवलेन माया न ज्ञातुं पार्यते । यथा च अन्धकारस्य साहाय्येन अन्धकारस्य प्रतीतिस्तथैव तर्कवलेन मायायाः प्रतीतिः। उक्तञ्चेदं बृहदारण्यकवार्तिके -
अविद्याया अविद्यात्वे इदमेव तु लक्षणम् ।
यत् प्रमाणासहिण्णुत्वं अन्यथा वस्तु सा भवेत् ।। इति । 
सूर्योदयकाले यथा च अन्धकारस्य नाशो दृश्यते तथा ज्ञानोदयकाले मायायाः प्रतीतिर्नश्यति - 
सेयं भ्रान्तिर्निरालम्वा सर्वन्यायविरोधिनी ।
सहते न विचारं सा तमो यद्वद् दिवाकरम् ।। 
इत्युक्तम्  नैष्कर्म्यसिद्धौ । एवञ्च प्रमाणासहिण्णुरूपिणी माया जगत उत्पत्तौ कारणमिति स्वीकर्तव्यम् । सा अव्यक्ता शक्तिः कार्यानुमेया इति विवेकचूडामणौ - 
अव्यक्तनाम्नी परमेशशक्तिरनाद्यविद्या त्रिगुणात्मिका या ।
कार्यानुमेया सुघियैव माया यया जगत्सर्वमिदं प्रसूयते ।। इति । 
मायायाः शक्तिरावरणविक्षेपभेदेन द्विधा भिन्ना भवति । आभ्यामेव शक्तिभ्यां ब्रह्मणः वास्तविकं सद्रूपमाच्छाद्यते । ब्रह्मणि असत असत्यस्य च जगतः प्रतीतिरारोप्यते ।
शक्तिद्वयं हि मायाया विक्षेपावृतिरूपकम् ।
विक्षेपशक्तिर्लिड्गादि ब्रह्माण्डान्तं जगत् सृजेत् ।। 
अन्तर्दृग्दृश्ययोर्भेदं बहिश्च ब्रह्मसर्गयोः ।
आवृणोत्यपरा शक्तिस्सा संसारस्य कारणम् ।। इति दृगदृश्यविवेके । 
एवं -
" आच्छाद्य विक्षिपति संरफुरदात्मरूपम् 
जीवेश्वरत्वजगदाकृतिभिर्मृषैव ।
अज्ञानमावरणविभ्रमशक्तियोगात् 
आत्मत्वमात्रविषयाश्रयताबलेन ।। इति । "
संक्षेपशारीरके च उक्तम् । अधिष्ठानस्य सत्यत्वापलापादनन्तरमेव अधिष्ठाने नृतनधर्मारोपो भवेत् । यथा च ऐन्द्रजालिकस्य इन्द्रजालविद्या द्रष्ट़़ृणां नेत्रेषु वास्तविकीं दर्शनक्षम्तां आच्छाद्य भ्रान्तेरुत्पादनादेव सफला इति लोके दृश्यते, एवमेव मायाया आवरणरूपा शक्तिः ब्रह्मणश्शुद्धस्वरूपमाच्छादयति । यथा च अग्निः स्वविरोधिनि जले साक्षात् प्रवेप्टु शक्तोऽप सूक्ष्मरूपेण पात्रादिद्वारा जले प्रविश्य तदीयं शैत्यं अपहनुत्य तत्र स्वीयं उष्णत्वं प्रदर्शयति तथेयं मायां सूक्ष्मतरेण स्वकीयमू रूपेण ब्रह्मणि प्रविश्य तदीयं निर्विषयं निराश्रयञ्च स्वरूमपह्नुत्य स्वीयं साश्रयत्वसविशेषत्वरूपं तत्र प्रदर्शयति । एवं लघुर्मेघः दर्शकाणां नेत्रं आच्छादयन् विस्तुतस्यादित्यमण्डलस्य यथा आच्छादकरो भवति तथा परिच्छिन्नमज्ञानं अपरिच्छिन्नस्य असंसारिण आत्मन आवारकं भवति । आवरणशक्तिरेव द्रष्टृदृश्ययोरन्तर्भेदम् , बहिर्ब्रह्मजगतोश्च भेदं उत्पादयति । यथा च रज्जौ सर्पभ्रान्तिस्सर्पज्ञानमुद् भावयति तथा मायापि अज्ञानावृते आत्मनि आकाशादिजगतिः ज्ञानमुद्भावयति । उक्तञ्च - 
मायाशक्तिर्निखिलकलनां साम्प्रतं वा विरुद्धाम् 
स्वाधिष्ठाने चितिफलयुता दर्शयत्याविमोक्षम् । 
नैल्यं व्योम्नि स्रजि विषधरो वार्यथा रश्मिपूगे 
तद्वन् मिथ्यात्मनि जगदिदं कल्पितं स्वप्नवच्च ।। 
इति  प्रत्यक्तत्वचिन्तामणौ । इयं माया अविद्यापदवाच्य़ा अज्ञानाख्या कथं जाता ? केन कारणेनास्याः ब्रह्मणा सम्बन्धो जात इति न चोदनीयम् । अविद्यायाः तत्सम्बन्धस्य च अनादित्वाङ्गीकारात् - 
" जीव ईशो विशुद्धा चित् तथा जीवेशयोर्भिदा ।
अविद्या तच्चितोर्योगः षडस्माकमनादयः ।। "
इति। मायासम्बन्धादेव प्रपञ्चोत्पत्तिः। मायेयं साश्रया सविषया च भवति । मायायां विषय ईश्वरः । आश्रयो जीवः । एतदेव साश्रयत्वं सविषयत्वञ्च ज्ञातरूपे ब्रह्मणि तदीयत्वं प्रदर्शयति । साश्रयत्वेन सविषयत्वेन च भासमानं यत् ज्ञानं तदेव महत्तत्वम् । महत्तत्वादेव प्रपञ्चस्योत्पत्तिः । 
माया (अविद्या)

नवीनास्तु अद्वैतिनः अज्ञानस्य शक्तिः - ज्ञानशक्तिः । क्रियाशक्तिस्तु आवरणविक्षेपभेदेन द्विविधा भवति । रजस्सत्वाभ्यामनभिमूतं तम आवरणशक्तिः । सा च अत्र घटो नास्तीति प्रपञ्चत्र्यवहारहेतुः। तमस्सत्वाभ्यामनभिमूतं रजो विक्षेपशक्तिः । सा च आकाशादिप्रपञ्चोत्पतिहेतुः । विक्षेपशक्तिपता अज्ञानेन उपहितस्यैव ईश्वरस्य जगत उपादानकारणता । अत्र "यथोर्णनाभिस्सृजते गृह्णते च" इति श्रुतिः प्रमाणम् । अत्र मते आवरणशक्तिपधानं अज्ञानमविद्या । विक्षेपशक्तिप्रधानमज्ञानं माया इत्युच्यते । मीयते अपरोक्षवत् प्रदर्श्यते अनयेति माया । स्वीयशक्तिवलात् प्रपञ्चमिमं प्रत्यक्षवत् सत्यमिव च प्रदर्शयतीयमित्यस्याः मायाख्या अन्वर्था भवति । इयमेव माया तमः, अविद्या, अव्यक्तमित्यादिशब्देन व्यवह्रियते - इत्यत्र 
अन्यथा भानहेतुत्वात् इयं मायेति कीर्तिता ।
आत्मतत्वतिरस्करात् तम इत्युच्यते बुधैः ।
विद्यानाश्यत्वतोऽविद्या मोहस्तत्कारणत्वतः ।
सद्वैलक्षण्यदृष्ट्रयायं असदित्युच्यते बुधैः ।। 
कार्यवत् व्यक्तताभावात् अव्यक्तमिति गीयते ।
एषा माहेश्वरीशक्तिर्न स्वकतन्त्रा परात्मवत् ।। 
इत्यादिवृद्धवचनानि प्रमाणानि । एवं जगतः सृष्टौ ईश्वरस्य अभिन्ननिमितोपादानत्वमपि मायाद्वारेणैवेत्यत्र - इन्द्रो मायाभिः पुरुरूप ईयते । सर्वं खलु इदं ब्रह्म तज्जलानिति शान्त उपासीत यतो वा इमानि भूतानि इत्यादीनि श्रुतिवाक्यानि, 
अजोऽपि सन्नव्ययात्मा भूतानामीश्वरोऽपि सन् ।
प्रकृतिं स्वां अधिष्ठाय सम्भवाम्यात्ममायया ।। 
निरुपमनिर्गुणेऽप्यखण्डे मयि चिति सर्वविकल्पनादिशून्ये । 
घटयति जगदीशजीवभेदान् अघटितघटनापटीयसी माया ।।
इत्यादिवचतानि च प्रमाणानि । एवञ्च तत्वप्रतिभासप्रतिवन्धेन अतत्वप्रतिभासहेतुः आवरणविक्षेपशक्तिद्वयवृत्ती अविद्या सर्वप्रपञ्चपकृतिरिति प्रतिपादयन्ति । प्रतिपादितञ्च विद्यारण्यस्वामिभिः पञ्चदश्याम् । विद्यारण्यस्वामी तु विवरणप्रस्थानानुयायी सन्नपि विवरणप्रस्थानानुकृलानां सिद्धन्तानां मार्गप्रदर्शकः भामतीविवरणप्रस्यानयोस्समन्वयकारी च विराजते । रजस्तमोऽनभिभूता शूद्धसत्वप्रधाना माया, रजस्तमोऽभिभूता मलिनसत्वप्रधाता अविद्या इति प्रतिपादयन् मायाप्रतिविम्बितं चैतन्यं सर्वज्ञत्वादिगुणविशिष्टं ईश्वर इति, अविद्याप्रतिबिम्बितं जीव इति - 
सत्वशुध्यविशुद्धिभ्यां मायाविद्ये च ते मते । 
मायाबिम्बो वशीकृत्य तां स्यात् सर्वज्ञ ईश्वरः ।
मायाख्यायाः कामघेनोर्वत्सौ जीवेश्वरावुभौ । 
यथेच्छं पिबतां द्वैतं अद्वैतं परमार्थतः ।। इत्यादिना पञ्चदश्यां विवरणप्रमेयसंग्रहे च प्रतिपादयति ।
मायाविषये प्राचीना :-

मायाविषये नवीना  :-

एवं विवरणमते ब्रह्मचैतन्यं ईश्वरचैतन्यं जीवचैतन्यञ्चेति चैतन्यत्रयं स्वीकृतम् । परन्तु कूटस्थचैतन्यं, ब्रह्मचैतन्यं, जीवेश्वरचैतन्यद्वयञ्चेति चतुर्विधं चैतायं स्वीकुर्वन् विद्यारण्यः भामतीविवरणप्रस्थानयोस्समन्वयकारिणं आत्मानं परिचाययति - 
कूटस्थो ब्रह्म जीवेशौ इत्येवं चिच्चतुर्विधा । 
घटाकाशमहाकाशौ जलाकाशाभ्रखे यथा ।। इति । 
साक्षिस्वरूपविषये विद्यारण्यात् प्राचीना वेदान्तिनः जीवेश्वराभ्यां भिन्नं शुद्धं चैतन्यात्मानं साक्षिणं वदन्ति । विद्यारण्यस्तु कूटस्थचैतन्यमेव साक्षीति प्रतिपादयति । यथा नृत्यशालास्थदीपः साक्षी भवति तथा कूटस्थचैतन्यमेव साक्षीति - 
नृत्यशालास्थितो दीपः प्रभुं सभ्यांश्च नर्तकीम् । 
दीपयेदविशेषेण तदभावेऽपि दीप्यते ।।
अहंकारः प्रभुः, सभ्याः विषया नर्तकी मतिः ।
तालादिधारीण्यक्षाणि दीपः साक्ष्यवभासकः ।। इति 
इमानि तत्वानि पञ्चदश्यादिषु ग्रन्थेषु वर्णितानि दृश्यन्ते ।
एवं दशम्शतकादनन्तरमुत्पन्नेषु षोडशशतकान्तेषु ग्रन्थेषु न्यायमकरन्दन्यायदीपावली - वेदान्तकौस्तुभाद्यादिषु ग्रन्थेषु भामतीविवरणप्रस्थानानां विकासः, अनिर्वचनीयख्यातिसाधनम् , अखण्डार्थत्वनिरूपणमित्यादिविषयश्च विकासं प्राप्तो दृश्यते । विशेषतश्च अद्वैतसिद्धान्तखण्डनपरस्य द्वैतसिद्धान्तमण्डनपरस्य व्यासतीर्थकृतन्यायामृतस्य तद्व्याख्यायाः न्यायामृततरङ्गिण्याः न्यायभास्करस्य च खण्डनाय प्रवृत्तानां अद्वैतसिद्धि - गुरु - सघुचन्द्रिका - विट्ठलेशीयादीनां ग्रन्थानाम् , एवं आधुनिके काले विशिष्टाद्वैतखण्डनपराणां शतदूषणीखण्डनपराणां शतभूषण्यादिग्रन्थानाञ्च आविर्भाव अद्वैतसिद्धान्तस्य विकासोन्मुखतामापादयति । 
यद्यपि अद्वैतवेदान्तस्य विकासः प्रस्थानत्रय्येति व्यपदिश्यते तथापि तस्य विकासः मद्दृष्ट्या न केवलं प्रस्थानत्रय्या परन्तु प्रस्थानत्रयीबहिर्मूतैः खण्डनमण्डनपरैर्वादप्रधानैश्च ग्रन्थैः, अद्वैतानन्दानुभाविभिराचार्यैरुपरचितैरनुभवप्रधानभा वाविष्करणात्मकैः - आत्मविद्याविलास - पञ्चदशयादिप्रकरणग्रन्थैः, अद्वैतवेदान्तसिद्धान्तप्रतिपादकैः ब्रह्मनैर्गुण्यवाद - विद्वन्मोदतरङ्गिणीप्रभृतिभिः काव्यैः योगवासिष्ठादिभिर्महाकाव्यैश्च सुतरामद्वैतवेदान्तसाहित्यं विकसितमिति तु नापरोक्षम् । 

\section{एतद्ग्रन्थप्रयोजनम्}
संस्कृतेतरासु विशेषत आङ्गिलहिन्दीप्रभृतिषु प्रसृततरासु आधुनिकासु भाषासु प्रतिविषयमेतादृशाः ग्रन्थास्समुपलभ्यन्ते येषु तत्तद्विषयविशेषसम्बद्धाः कति ग्रन्थाः विद्यन्ते ? तत्तद्विषयस्योत्पत्तिः कुतः ? तत्तद्विषयस्य वृद्धिर्विकासश्च कथम् ? तत्तद्विषयाभिवृद्धौ दत्तचित्ता आचार्याः के के ? इति विशिष्टेतिहासो दरीदृश्यते । भौतिकविज्ञानाभिवृद्धौ वृद्धसमे नितरां विज्ञानवादिनि भारतेतरदेशे भौतिकविज्ञानविकासे स्तनन्धयशिशुकल्पस्य भाग्तवर्षस्य या कीर्तिः, या श्रद्धा,  यश्च गौरवगरिमा तस्य मुख्यं कारणं भारतीयानामद्वैतवेदान्तशास्रमिति तु निस्संशयं विदुषाम् । एतादृशस्याद्वैतवेदान्तसाहित्यस्येतिहासलेखने ममायं विशेष अभिनिवेशस्समुदपद्यत । संस्कृतेतरभाषासु प्रायशस्तत्तद्भाषायामेव तत्तद्विषयविशेषस्य इतिहासः दृश्यते । तस्मान्ममापि संस्कृतभाषायामेव संशोधनपूर्वकेतिहासलेखने महानादरस्समभूत् । तादृश भिनिवेशपूर्तये मद्रासविश्वविद्यालयसंस्कृतविभागभूतपूर्वप्राध्यक्षाः पूज्यतमाः यशश्शरीरमापन्नाः Dr. V. राघवमहोदयाः मां प्रोत्साहितवन्त इति तेषामधमणोंऽस्मि । 

\section{ग्रन्थस्यास्य विषयप्रतिपादनसरणिः}
अद्वैतवेदान्तसाहित्यं न केवलं प्रस्थानत्रय्य परमन्यैरपि ग्रन्थैरिति पूर्वमुपवर्णितम् । तस्मात् अद्वैतवेदान्तसाहित्यस्येतिहासः उपनिषत् - गीता - सूत्रतद्भाष्य - वादप्रधान - अनुभवप्रधान - काव्यात्मकशैलीप्रधानैः ग्रन्थैर्विकसित इति अद्वैतवेदान्तस्य प्रस्थानषट्कमिति कथनमपि अविरुद्धमन्वर्थञ्च भवेत् । अत एवायं ग्रन्थः पूर्वोत्तरभागद्वयेन विभक्तः। तत्र पूर्वभागे ग्रन्थप्राधान्यं मनसिकत्य षट् परिच्छेदाः परिकल्पिताः । 

तत्र प्रथमे उपनिषत्प्रस्थानप्रधाने परिच्छेदे अद्वैतपरा उपनिषदः ग्रन्थप्रतिपाद्यवर्णनपूर्वकं तासां भाप्याणि, अद्वैताचार्यकृताः व्याख्याः, उपव्याख्याश्च निरूपिताः । उपनिषदां कालः विचारनिर्णयव्राह्य इति मत्या विमर्शकवरेण्यानांं सिद्धान्ता एव तत्र तत्र निरूपिताः । 
द्वितीये गीताप्रस्थानप्रधाने परिच्छेदे निखिलपुराणान्तर्गता अद्वैतमताविरोधिन्यः गीताः भगवगद्गीताश्च प्रतिपाद्यविशिष्टा सभाष्य़व्याख्योपव्याख्या निरूपिताः । 

तृतीये सृत्रतद्भाष्यप्रस्थानप्रधाने परिच्छेदे ब्रह्मसृत्राणि तेषां भाष्यम् , पञ्चपादिकाविवरणप्रस्थानानुयायिनः ग्रन्थाः, भामतीप्रस्थानानुयांयिनः ग्रन्थाः भाष्यस्य स्वतन्त्रव्याख्यात्मकाः ग्रन्थाः भाष्यानुसारिण्यः वृत्तयश्च सव्याख्या निरूपिताः । 
चतुर्थः परिच्छेदः प्रकारणग्रन्थप्रधानः । परिच्छेदेऽस्मिन् प्रकरणग्रन्थाः सत्र्याख्याः मार्कि अष्टाशतेभ्य अन्यूना निर्दिष्टः। प्रकारणग्रन्थरचयितारः प्रस्यानत्रयेऽपि ग्रन्थकारा दृश्यन्ते एवं बहूनां प्रकारणग्रन्थानां रचयिता दृश्यते । प्रकरणग्रन्थाश्च असंख्येया वर्तन्ते । तस्मात् पोतरुक्त्यादिदोषपरिहाराय परिच्छेदेऽस्मिन् सव्य ख्यानां प्रकरणग्रन्थानां नामनिर्देशमात्रं मातृकाक्रमेण निर्दिष्टम् । तत्तद्ग्रन्थ प्रतिपादितास्तु विषयाः तत्तद्ग्रन्थकर्तृविचारावसरे (अद्वैतग्रन्थकारपरिच्छेदे ) प्रतिपादिताः । 
पञ्चमे परिच्छेदे उद्धारमात्रज्ञाताः नाममात्रप्रसिद्धाश्च ग्रन्थाः निरूपिताः । ते च ग्रन्थाः कुत्रोद्धुता इत्यादिभिर्विवरणैस्साकं चत्वारिंशदन्यूताः प्रतिपादिताः । 

षष्ठः परिच्छेद अज्ञातकर्तृकाद्वैतग्रन्थप्रधानः । परिच्छेदेऽस्मिन् मुद्रितामुद्रितभेदेन अज्ञातकर्तृकाः पञ्चशतं ग्रन्थाः वर्णिताः । तेष्वपि समाननामानः विभिन्नग्रन्था अमुद्रिता विभिन्नहस्तलिखितपुस्तकालयेषु लभ्यन्ते । ये च मद्रासराजकीय हस्तलिखितपुस्तकालये अडयारपुस्तकालये तञ्जपुरपुस्तकालये च लभ्यन्ते, येषाञ्च वर्णनात्मकविस्तृतग्रन्थसूची विद्यते ते चाधीतास्सन्दृष्टाश्च । तेषाञ्च ग्रन्थानां मिन्नत्वाभिन्नत्वे निश्चिते च । तथापि अनिवार्यं काठिन्यं यत्र तत्र मौनीभाव एव स्वीकर्तव्यः- उदाहरणार्थम् - आनन्दबोधेन न्यायमकरन्दे ( p. No . 170) पदानां सिद्धे सङ्गतिग्रहस्थापनावसरे "विस्तरतस्तु न्यायदीपिकायां अवगन्तव्यम् " इत्युक्तम् । एष ग्रन्थः प्रकरणवशात् स्वतन्त्रो वेदान्तग्रन्थ इत्येव निर्णतुं शक्यते । परं न्यायदीपिकाख्याः बहवो ग्रन्थाः द्वैतपूर्वमीमांसाजैनन्यायशास्त्रेषु विद्यन्ते । न ते अद्वैतसाधकाः। मद्रपुरी तिरुवनन्तपुर हस्तलिखितपुस्तकालययोर्विद्यमानः न्यायदीपिकाख्यः ग्रन्थस्तु शाब्दनिर्णयव्याख्यारूप एव न तु स्वतन्त्रग्रन्थः, तस्य शाब्ददीपिका इत्येव नाम दृश्यते । एवमेव नासिकउज्जैनपञ्चावादिपुस्तकालयस्थेषु वर्णनात्मकविस्तृतसृचीरहितेषु पुस्तकेषु किमस्माभिः कर्तव्यम् । तेषां ग्रन्थानां विषये संशयस्सुस्थ एव । तस्मादज्ञातकर्तृकग्रन्थप्रधानोऽयं परिच्छेदस्सन्दृव्धः । 

एवं प्रस्थानषट्केन विकासितस्यास्य वेदान्तसाहित्यस्य ग्रन्थाः मूलव्याख्यानमेदभिन्नाः परिच्छेदषट्केन प्रतिपादिताः । ग्रन्थाः मुद्रिता उतामुद्रिता इति च निर्दिष्टाः । अमुद्रितग्रन्थानां प्राप्तिस्थानानि च निरूपितानि । प्रसिद्धतमेभ्यः मूलव्याख्याग्रन्थेभ्य ऋते प्रायस्सर्वेषामपि मुद्रितामुद्रितग्रन्थानां विषयाश्च सङ्ग्रहेण प्रदर्शिताः । व्याख्येयग्रन्थानां यावन्त्यः व्याख्या उभयविधास्सन्ति तास्सर्वा अपि व्याख्येयग्रन्थप्रस्तावे अन्यत्र व्याख्यानग्रन्थकर्तृप्रस्तावे च निर्दिष्टाः । पठितृणां प्रेक्षितृणाञ्च सौलभ्याय अनुक्रमणिकापि व्याख्येयव्याख्यान ग्रन्थानुसारं समारचिता। 

एवं षट्सु परिच्छेदेषु मूलव्याख्यानभेदभिन्ना ज्ञाताज्ञातकर्तृकभेदभिन्नाश्च ग्रन्थाः षट्सप्तत्यधिकसहस्रसंख्यापरिमिताः (1076) ग्रन्थेऽस्मिन् निर्दिष्टाः । ग्रन्थेप्वेतेषु मुद्रित्ग्रन्थाः द्वयधिकचतुश्शतसंख्याका (402) निर्दिश्यन्ते स्म अमुद्रितेष्वपि ग्रन्थलिपिदेवनागरीलिप्योर्मुद्रिता एव निर्देष्चटुं पारिताः । व्याख्याग्रन्थाः षट्षष्ठ्यधिकचतुश्शताधिकग्रन्थाः (466) निर्दिष्टाः । 

द्वितीयो भागः 

भागेऽस्मिन् द्वितीये अद्वैतवेदान्तग्रन्थकर्तार उपवर्णिताः । भागोऽयमपि पञ्चभिः परिच्छेदैः परिच्छिन्नः । तत्र प्रथमे परिच्छेदे शङ्करभगवत्पादेभ्यः प्राचीना अनिर्णीतकालादिविषयविशेषाः अद्वैताचार्या वर्ण्यन्ते स्म । द्वितीये परिच्छेद्दे शङ्करादर्वाचीनाः प्रसिद्धतमाः अद्वैताचार्यशब्दव्यपदेश्याः ग्रन्थकारास्तिथिक्रमानुसारं निर्दिष्टाः । प्रत्यद्वैताचार्यं आश्रमभेदेन नामान्तराणि, कालः, आचार्य - प्राचार्यसतीर्थ्याः, शिष्य - प्रशिष्याः, वासमूमिः, सामायेकाः, पोषक - पोष्या राजान इत्यादिकं तत्तत्प्रणीतग्रन्थप्रामाण्यादुपवर्णितम् । पश्चात् सदुपदेशात्मकगुरुशिष्यपरम्परारूपसम्प्रदायसमागता पण्डितमण्डलीमात्रप्रसिद्धा कथा अथवा किंवदन्ती च प्रतिपादिता । अनन्तरं संशोधनधुरीणानां प्राच्यप्रतीच्यभाषाविदुषां सिद्धान्ताश्च न्यरूपिषत । तत्तदाचार्यप्रणीता अद्वैतग्रन्थाः प्रतिपादिताः । ग्रन्थप्रतिपाद्यानां रूपरेखा च समाकृष्टा । तत्र तत्र आवश्यकस्थलेषु गुरुपरम्परावृक्षुः, वंशवृक्षः व्याख्यानव्याख्येयग्रन्थपरम्परावृक्षा इति बहवो वृक्षास्समारोपिताः। ग्रन्थकारा ये सन्यासिनस्समानाचार्याश्च ते अभिन्ना इति निर्णीयन्ते स्म । ये तु समानाभिख्याः भिन्नप्राचार्याश्च त अभिन्ना इति न निर्णीताः परन्तु भिन्ना इति प्रथमद्वितीयसंख्याभिर्विशेषिताः । 

यद्यपि ग्रन्थेऽस्मिन् ग्रन्थकर्तॄणां इतिहासलेखने तत्तत्प्रणीत - अद्वैतग्रन्थनिर्देशेनैवालम् , तथापि यदि तेषां इतिहासलेखने तत्तत्प्रणीतेभ्य अद्वैतेतरग्रन्थेभ्यश्च प्रमाणं लभ्यते तदा तत्तद्ग्रन्थनिर्देशोॄ ऽप्यावश्यकस्मम्पन्नः । तस्मात्तेऽपि ग्रन्था अत्र निर्दिष्टाः । परन्तु न ते अनुक्रमणिकायां स्थानमर्हन्ति। 

तृतीये परिच्छेदे ये च ज्ञातकालादिविषयविशेषाः, ये तु न प्रसिद्धतमाः, ये च न सन्यासिनः, दर्शनान्तरे प्रतिष्ठिता अपि अद्वैतेऽपि कतिपयग्रन्थकर्तारस्तेषां जीवनवृत्तान्तः ग्रन्थाश्च ग्रन्थप्रतिपाद्यसहिताः वर्णमालाक्रमेण निरूपिताः । 
चतुर्थे परिच्छेदे ज्ञातनामान अज्ञातकालादिविषयविशेषाश्शङ्करादर्वक्तना ग्रन्थकारास्तेषां ग्रान्थश्च निरूपिताः। पञ्चमे परिच्छेदे अज्ञातनामानः परन्तु तत्तद्गुरुशिष्यत्वेन मुद्रितामुद्रितग्रन्थेषु आत्मानं निर्दिशन्तो ग्रन्थकाराः तेषां ग्रन्थाश्च निर्दिष्टाः । 
ग्रन्थस्यादौ सड्क्षेपसङ्केतबोधिनी निर्दिष्टा । सग्प्रदायसमागतश्शान्तिपाठकमः, मङ्गलपाठक्रमसहितस्संयाजितः यश्च अद्वैतवेदान्तसाहित्यस्य अनिवार्य आवश्यकः कल्याणकारी च भाग इति ममाभिनिवेशः । ग्रन्थस्यान्ते अद्वैतग्रन्थग्रन्थकर्तृ सामान्यानुक्रमणिकाश्च संयोजिताः तत्र स्थूलाक्षराङ्कितपत्रसंख्याः ग्रन्थानां, ग्रन्थकतॄणां, समानाख्यानां ग्रन्थानां ग्रन्थकर्तॄणाञ्च भिन्नतां वृत्तान्तञ्च बोधयन्ति । सामान्यानुक्रमणिकायां ग्रन्थकर्तुर्विभिन्नानि तथा अन्यानि नामानि, अद्वैतवेदान्तबहिर्भूताः ग्रन्थाः, देशपुरप्रामादयः गोत्रवंशादयः, अद्वैतवेदान्तसम्बद्धाः पारिभाषिकशब्दाः, विद्या, विभिन्ना वादाश्च स्थानं लभन्ते स्म। गच्छतस्स्खलनमिति न्यायात् , मुद्रणालयानवधानाद्वा उत्पन्नाः केचन मुद्रणदेषाः शोधनेन योजनेन दूरीकृतास्सन्ति । क्वचित् क्वचित्  संस्कृत्सुलभीकरणघोषगौरवदानाय केवलं सन्धिविच्छेदस्स्वीकृतः । द्वित्रिस्थलेषु षष्टिश्ब्दस्थाने षष्ठिरिति प्रयोगः, सिंहशब्दस्थाने सिम्ह इति प्रयोगश्च स्वीकृतः । उद्धृतवाक्यानुक्रमणिका च संयोजिता । अन्ते च महानद्वैताचार्यगुरुशिप्यपरम्परावृक्षस्संरोपितः, यस्मिन् अद्वैतग्रन्थकर्तारः गुरवः, शिष्याश्च केवल स्थानमलभन्त, न मठाम्नायादिप्रसिद्धाः, नापि अद्वेंतसम्प्रदायागता अपि अकृतग्रन्थाः । नापि च संरकृतेतरभाषासु ग्रन्थप्रणेतारो वा । 

एतद्ग्रन्थरचनायामनिवार्यं कठिन्यम् ।
एवममुद्रित मुद्रित उद्रधृत - ज्ञातकर्तृक - अज्ञातरकर्तृ-व्याख्यान - व्याख्येयभेदेन भिन्नास्सर्वेऽपि ग्रन्थाः प्रबन्धेऽस्मिन् यथामति यथाशक्ति च व्यलिख्यन्त तथापि 
"कति कृतयः कति कवयः कति लुप्ताः कति चरन्ति कति शिथिलाः । "
इत्यभियुक्तं क्तिमनुसृत्य कति ग्रन्थाः मद्दृष्टिगोचरतामानीता ? कति विलुप्ता इति न ज्ञायन्ते । इदमेकं काठिन्यम् । अपरमेकं काठिन्यमिदम् यत् समाननामानस्समानकर्तृकाः विभिन्नकर्तृकाश्च बहवो ग्रन्थाः विभिन्नहहस्तलिखितपुस्तकालयेषु लक्ष्यन्ते । ये च मद्रासगजकीयअडयार - तञ्जपुर - तिरुवनन्तपुरपुस्तकालयेषु दृश्यन्ते, येषाञ्च वर्णनात्मकविस्तृतसूच्यो विद्यन्ते, तेषां ग्रन्थानां भिन्नात्वाभिन्नत्वे निर्णीते । परन्तु दूरदेशस्थानां वर्णनात्मकसूचीरहितानां पत्रव्यवहारेणापि अवेद्यानां ग्रन्थानां विषये किमस्माभिः कर्तव्यम् ? । तेषां विषये संशयस्सुस्थ एव । अत एवाज्ञातकर्तृकपरिच्छेदस्संदृब्धः । 
कृतज्ञताविष्करणम् 
एतादृशमदीयज्ञानवृद्धिकरस्य कार्यस्य सफलतायै नूतनबृात्सूचीसम्पादनाय मद्रासविश्वविद्यालयसंस्कृतविभागस्थानां निखिलविदेशस्वदेशस्थ स्तलिखितपुस्तकालय वर्णनात्मकविस्तृतसृचीनां प्रदानेन बहूपकृतवतां मद्रासविश्वविद्यालयसंस्कृतविभागस्थानां समेषाम् , अनर्घाभिप्रायोपदेशादिप्रदानेन उपकृतवतां प्राध्यापकवर्याणां Dr. K.K. Raja महोदयानाञ्च अधमर्णतमः कृतज्ञश्चास्मि ।
मदीयं प्रास्ताविकं आङ्गिलभाषायां संक्षिप्य अनूद्य उपकृतवते Dr. N. Veezinathan M.A. Ph.D , \& Vedanta Siromani महाशयाय कृतज्ञतां निवेदये । 
अप्तुदितग्रन्थावलोकनाय हस्तलिखितग्रन्थप्रदानेन उपकारिभ्यः मद्रामराजकीयहस्तलिखितपुस्तकालयाध्यक्षेभ्यः, अडयारपुस्तकालयाधियेभ्यश्च मदीयां कृतज्ञतां विनिवेदये । 
मुद्रितग्रन्थावलोकनाय नियमानपि शिथिलीकृत्य असंख्यग्रन्थप्रदानेन प्रोत्साहितवतां M.M. कुप्पुस्वामिशास्रियुस्तकालयाधिकारिणाम् , मद्रपुरीसंस्कृतकलाशालाधिकारिणाम् , प्रान्तीयकलाशाला (Presidency College ) सस्कृतविभागाध्याक्षाणाञ्च घन्यवादवादी कृतज्ञश्चास्मि । 
बहुषु ग्रन्थेषु मुदापणाय सत्स्वपि विषयविशेषम् , एतादृशग्रन्थनिर्माणपरिश्रमम् , ग्रन्थयोग्यताञ्व परिशील्य मुद्रापणाय प्रकाशनाय च प्रयतितवते उपकृतवते च गुणग्राहिणे मद्रासविश्वविद्यालयाय विश्वविद्यालयानुदानायोगाय (University Grant Commission) च, एवं सुचारुरूपेण मुद्रितवते रत्नमुद्रणलयाय (Rathnam Press) मदीयां कृतज्ञतां विनिवेदये । मुद्रापणकार्यस्यास्य सफलतायै बहूयकृतवद्भ्यां संस्कृतविभागकार्यालयकार्यत्र्यापृताभ्यां चिरञ्जीविस्वामिनाथ - बाबू राजेन्द्मभ्यां सन्तु श्रेयांसि भूयांसीत्याशासे । 
अनेन मदीयेन परिश्रमेण विद्वासस्संस्कृताभिज्ञाश्च नूनं किञ्चिदपि प्रयोजनं प्राप्स्यन्तीति विश्वस्य - 
"तद्विद्वांसोऽनुगृह्णन्तु चित्तश्रोत्रैः प्रसादिभिः । 
सन्तः प्रणयिवाक्यानि गृह्णन्ति ह्यनसूयवः ।। इति "
कुमरिलभट्टवाक्येन सम्प्रार्थ्य ग्रन्थमिमं आचार्य - जगद्गुरुशङ्करभगवत्पादकमलयोस्समर्पये - 

सिद्धार्थिनामसंवत्सरम् 
कार्तिक - शुक्ल सप्तमी
षड्विंशतितमो दिवसः 
26 -11 - 1979 
मद्रास-5

इत्थम् 
विदुषां विधेयः 
R THANGASWAMI SARMA

