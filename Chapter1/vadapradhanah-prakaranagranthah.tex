\chapter{खण्डनमण्डनपरा वादप्रधानाश्च प्रकरणग्रन्थाः }
706. अद्वैततरणिः 							नटेशार्यः
707. अद्वैतदीपिका							अनन्तकृष्णशास्त्री
708. अद्वैतब्रह्मसिद्धिः 						सदानन्दकाष्मीरकः
709. अद्वैतब्रह्मसिद्धिविनियोग सङ्ग्रहः		आनन्दपूर्णमुनिः 
		अमुद्रितोऽयं ग्रन्थ राय आसियाटिक सोसाइटि कल्कत्तानगरस्थ पुस्तकालये लभ्यते । कोऽयमानन्दपूर्णः ? ब्रह्मसिद्धिव्याख्याकारः भावशुद्धिकोरो न भवितुमर्हति । सदानन्दकाष्मीरकादतिप्राचीनत्वात् । तस्मादन्योऽयमानन्दपूर्ण इत्येव निणींयते । अथवा ब्रह्मसिद्धेरेवाद्वैतब्रह्मासिद्धिरिति नामान्तरमिति स्वीकृत्य तस्याः व्याख्याया एव "विनियोगसङ्ग्रह" इति नामान्तरमिति निर्देशस्सम्भाव्यते । परन्तु अस्य निश्चयः न कर्तुं पार्यते । समग्रग्रन्थस्य द्रष्टुमशक्तेः । 
710. अद्वैतमार्ताण्डः							अनन्तकृष्णशास्त्री
711. अद्वैतरत्नरक्षणम्						मधुसूदनसरस्वती
712. अद्वैतसिद्धान्तदीपिका 					भवानीशङ्करानन्दः
713. अद्वैतसिद्धिः							मधुसूदनसरस्वती
		अद्वैतसिद्धिचन्द्रिका - धर्मदत्तबच्चा झा Ref. P. 205 of History of Navya - Nyaya - in Mithila
714. अद्वैतसिद्धिव्याख्या गुरुचन्दिका 		ब्रह्मानन्दसरस्वती
715. अद्वैतसिद्धिव्याख्या लघुचन्द्रिका		ब्रह्मानन्दसरस्वती
716. अद्वैतसिद्धिसाधकः						पुरुषोत्तमः
717. अद्वैतसिद्धि अद्वैतचन्द्रिका				बलभद्रः
718. अद्वैतसिद्धिसिद्धान्तसारः सव्याख्यः	सदानन्दव्यासः
719. अभेदरत्नम् 							मल्लनाराध्यः
720. अवैदिकमततिरस्कारः					अच्युतशर्मामोडकः
721. इष्टसिद्धिः								विमुक्तात्मा
722. इष्टसिद्धिविवरणम्						अनुभूतिस्वरूपः
723. इष्टसिद्धिविवरणम् 						आनन्दानुभवः
724. इष्टसिद्धिविवरणम्						ज्ञानोत्तमः
725. उपाधिखण्डनम् 						पुरुषोत्तमसरस्वती
726. खण्डनखण्डखाद्यम् 					श्रीहर्षः
727.	खण्डनखण्डखाद्यव्याख्याः आनन्दवर्धिनी शङ्करमिश्रः
728. खण्डनखण्डखाद्य व्याख्या 			चित्सुखाचार्यः
729. खण्डनखण्डखाद्यटीका				पद्मनाभपण्डितः
730. खण्डनमण्डनम्						भवनाथः
731. खण्डनमण्डनम् 						परमानन्दः 
732. खण्डनमण्डनम् 						वरदपण्डितः
		ग्रन्थोऽयं अनन्तशयनपुस्तकालये, अडयारपुस्तकालये, भारतकार्यालयपुस्तकालये लन्दननगरे च लभ्यते । 
733. खण्डनमण्डनव्याख्या
		अज्ञातकर्तृकेयं व्याख्या अडयारपुस्तकालये मद्रस राजकीयहस्तलिखित पुस्तकालये च लभ्यते । 
734. खण्डनमहातर्कः						चरित्रसिम्हः
735. खण्डनखण्डनम् 						प्रगल्भमिश्रः
736. खण्डनकुठारः							गोकुलनाथः
737. खण्डनोद्धारः							वाचस्पतिः
738. खण्डनदीधितिः						रघुनाथशिरोमणिः
739. खण्डनप्रकाशः							वर्घमानः
740. खण्डनखाद्य विद्याभरणी				विद्याभरणः
741. खण्डनखण्डखाद्य विद्यासागरी			आनन्दपूर्णः
742. खण्डनखाद्य शिष्यहितैषिणी			अनुभूतिस्वरूपः
743. खण्डनखाद्य श्रीदर्पणम् 				शुभङ्करमिश्रः
744. खण्डनखण्डखाद्य अद्वैतबोधामृतम् 	अज्ञातम् 
745. खण्डनखण्डखाद्यव्याख्या शारदा		शङ्करचैतन्यः 
746. खण्डनखण्डस्वाद्यसारः खण्डनरत्नमालिका	सूर्यनारायणः
747. खण्डनपरिशिष्टम् 						ताराचरणशर्मा
748. तत्वकौस्तुभम् 						भट्टोजिदीक्षितः
749. तत्वचन्द्रिका							उमामहेश्वरः
750. तत्वविवेकः							नृसिम्हाश्रमी
751. तत्वविवेकव्याख्या वाक्यमाला			भट्टोजिदीक्षितः
752. तत्वविवेकदीपनम् अद्वैतरत्नकोशः	नृसिम्हाश्रमी
753. तत्वविवेकदीपन भावप्रकाशिका		अखण्डानन्दः
754. तत्वविवेकदीपनव्याख्या 				अन्नम्भट्टः 
		अमुद्रितोऽयं ग्रन्थःमद्रासरासराजकीयपुस्तकालये लभ्यते । 
755. तत्वविवेकदीपन भावप्रकाशिका		कालहस्तीशः
756. तत्वविवेकदीपन कोशरत्नप्रकाशः 	अनुभवानन्दः
757. तत्वविवेकदीपनभावार्थप्रकाशिका		शाश्वतानन्दतीर्थः
758. तत्वविवेकदीपन तत्वविवेचनी अद्वैतरत्नकोशपूरणी अग्निहोत्रभट्टः
759. तत्वविवेकदीपनव्याख्या अद्वैतरत्नकोशपालनी रामाध्वरी
760. तत्वसंख्यान खण्डतम् 				त्र्यम्बकभट्टः
761. तप्तचक्राङ्कविध्वंसनम् 					गरुडाचलयज्वा
		अमुद्रितोऽयं ग्रन्थः लन्दननगर भारतकार्यालयपुस्तकालये लभ्यते ।
762. तप्तमुद्राविद्रावणम् 						भास्करदीक्षितः
763. दशकोटिः 								अप्पय्यदीक्षितः
		ग्रन्थोऽयमडयारपुस्तकालये लभ्यते । 
764.	दृग्दृश्यसम्बन्धानुपपत्तिप्रकाशः		ऋम्बकभट्टः
765. दृश्यविषयताखण्डनम् 				अच्युतशर्मा
766. ध्वान्तानुबन्धधिक्कारः				रामेश्वरभट्टः
767. नवकोटिः								रामाशास्त्री
768. न्यायचन्द्रिका							आनन्दपूर्णः
769. न्यायचन्द्रिकाव्याख्या					स्वरूपानन्दः
770. न्यायदीपावलिः						आनन्दबोधः
771. न्यायदीपावलीव्याख्या तात्पर्यटीका	सुखप्रकाशः
772. न्यायदीपावली चन्द्रिका				अनुभूतिस्वरूपः
773. न्यायदीपावली न्यायविवेकः			अमृतानन्दः
774. न्यायदीपावली वेदान्तविवेकः			आनन्दगिरिः
775. न्यायभास्कर खण्डनम्				रामसुब्रह्मण्यशास्त्री
		ग्रन्थोऽयं मद्रपुरीसंस्कृतकलाशालाप्रधानाध्यापकेन म. म. अनन्तकृष्णशास्त्रिगुरूणा पञ्चापगेश्शास्त्रिणा कृतः। ग्रन्थोऽयं "ब्रह्मानन्दीयभावप्रकाशिका" नाम्ना पण्डितराज सुब्रह्मण्यशास्त्रिभिः (अण्णामलै विश्वविद्यालय) गोश्रीमहाराजरामवर्मणा च परिष्कृतः। प्रकाशिता च गोश्रीमहाराजेन । ग्रन्थोऽयं रामसुब्रह्मण्यशास्त्रिकृतात् भिन्नो वा नवेवि न ज्ञायते । 
776. न्यायमकरन्दः							आनन्दबोधः 
777. न्यायमकरन्दटीका						चित्सुखः
778. न्यायमकरन्द विवेचनी				सुखप्रकाशः
779. न्यायरत्नदीपावलिः					आनन्दानुभवः	
780. न्यायरक्षामणि भाष्योक्तिविरोधः		रामसुब्रह्मण्यशास्त्री
781. न्यायेन्दुशेखरः							त्यागराजशास्त्री
782. न्यायेन्दुशेखरः							रामसुब्रह्मण्यशास्त्री
783. न्यायेन्दुशेखरदोषयोग घटनम्			हरिहरशास्त्री
784. पदार्थतत्वनिर्णयः						आनन्दानुभवः
785. पदार्थतत्वनिर्णयटीका					आत्मस्वरूपः
786. पदार्थतत्वनिर्णय तत्वविवेकः			आनन्दगिरिः
787. परिहार खण्डनम्						रूद्रभट्टशर्मा
		वेदान्तदेशिककृतस्य "विरोधपरिहारा" ख्यग्रन्थस्य खण्डनपरोऽयं ग्रन्थः विद्याविलासमुद्रणालये वाराणस्यां मुद्रितः। 
788. ब्रह्मनैर्गुण्यवादः							विदूलशास्त्री
789. ब्रह्मसिद्धिः								मण्डनमिश्रः
790. ब्रह्मसिद्धिव्याख्या						चित्सुखाचार्यः
791. ब्रह्मसिद्धि समीक्षाफक्किका 			शङ्खपाणिः
792. ब्रह्मसिद्धि भावशुद्धिः					आनन्दपूर्णः
		अस्यैव ग्रन्थस्य टीकारत्नमित्यपि नामान्तरम् । 
793. भेदखण्डनम्							रामेन्द्रशिष्यः 
794. भेदखण्डनम् 							नृसिम्हाश्रमः
795. भेदधिक्कार विवृतिः					कालहस्तीशः
796. भेदधिक्कारसत्क्रिया 					नारायणाश्रमः 
797. भेदधिक्कारसत्क्रियोज्वला			रामभद्रानन्दः
798. भेदधिक्कारन्यक्काराङ्कुशम्			वेङ्कटनाथभट्टः
		ग्रन्थोऽयं मैसूर पुस्तकालये लभ्यते ।
799. भेदबिभीषिका							अभेदोपाध्यायः
		ग्रन्थोऽयं भारतकार्यालये लन्दननगरे अमुद्रित उपलभ्यते । 
800. भ्रमभञ्जिनी								मल्लादिरामकृष्णः 
		ग्रन्थोऽयं वाणीसुद्रणालये बेजवाडानगरे मुद्रितः । 
801. मध्वचन्द्रिकाखण्डनम् 					रामसुब्रह्मण्यशास्त्री
802. मध्वतन्त्रमुखमर्दनम्					अप्पय्यदीक्षितः
803. मध्वतन्त्रमुखप्तर्दनव्याख्या व्याध्वविध्वंसनम्	अप्पय्यदीक्षितः
804. मध्वन्यक्कारः							ज्ञानेन्द्रगुरुः
805. मध्वमतकथनम्						एकोजीराजः
806. मध्वमतचपेटिका						रामकृष्णः
807. मध्वमतचपेटिकाव्याख्या प्रदीपः		रामकृष्णः
808. मध्वमतविध्वंसः						भट्टोजिदीक्षितः 
809. मध्वसिद्धान्तभञ्चिनी					आनन्दाश्रमशिष्यः
810. माध्वमुखभङ्ः							सूर्यनारायणाशुक्लः
811. मिथ्यात्वनिरुक्तिरहस्यम् 				गोकुलनाथः
812. मिथ्यात्वानुमानम् 						रामाशास्त्री
813. रामानुजमतखण्डनम् 					एकोजीराजः 
814. लघुचन्द्रिकाव्याख्या विदृलेशीया		विट्टलेशः
815. वादनक्षत्रमालिका						अप्यय्यदीक्षितः
816. वादावलिः								रत्नखेटश्रीनिवासः
817. विरोधवरूधिनी							उमामहेश्वरः
818. विशिष्टाद्वैतदूषणसारसङ्ग्रह			ब्रह्मदेवपण्डितः 
819. विशिष्टाद्वैत भञ्जनम्						रामकृष्णः
820. वेदान्तरक्षामणिः 						अनन्तकृष्णशास्त्री
821. व्यासतात्पर्यनिर्णयः					अय्यण्णादीक्षितः
		शतदूषणीखण्डनम् 
		विशिष्टाद्वैतग्रन्थस्य शतदूषण्याख्यस्य खण्डनपरोऽयमद्वैतग्रन्थः 1885 A. D. वत्सरेषु आफर्ट महाशयेन सम्पादिते दक्षिणदेशीय प्राइवेटलाइबरीस मानस्कृष्ट सूच्यां No. 5417 एवं  8960 इति निर्दिष्टः । ग्रन्थोऽयं महाराजपुर शिवरामकृष्णशास्त्री गृहे No. 5417, एवं तिरुवयार जम्बनाथशास्त्रिगृहे 8960 इति ग्रन्थदर्शनस्थलमपि निर्दिष्टम् ।। 
822. शतभूषणी								अनन्तकृष्णशास्त्री
823. शतभूषण्यनुबन्धौ						अनन्तकृष्णशास्त्री
824. श्रीभाष्यदूषणम्						स्वामिशास्त्री
825. श्रुतिमतानुमानोपपत्तिः				त्र्यम्बकभट्टः
826.	श्रुतिमतोद्योतः							त्र्यम्बकभट्टः
827. श्रुतिमतोद्योतटिप्पणी					कामाक्षी
828. श्रुतिरत्नप्रकाशः						त्र्यम्बकभ्ट्टः
829. श्रुतिरत्नप्रकाशटिप्पणी				कामाक्षी
830. सप्तविधानुपपत्तिभङ्गः					गोविन्दानन्दः 
831. सिद्धान्तरत्नमाला						श्रीवत्सलाञ्जनः
832. सिद्धान्तसिद्धाञ्जनम् 					कृष्णानन्दसरस्वती
833. सिद्धान्तसिद्धाञ्जनव्याख्या	रत्नतूलिका भास्करदीक्षितः 
