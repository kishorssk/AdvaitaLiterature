\chapter{अद्वैतग्रन्थकाराः}
प्रथमे प्रकरणे प्रसिद्धतमानां अद्वैताचार्यपदव्यपदेश्यानां अद्वैतग्रन्थकर्तृणां तिथिक्रमानुसारी निर्देशः कृतः । प्रकरणेऽस्मिन् द्वितीये प्रसिद्धाः अद्वैतग्रन्थ प्रणेतारः तेषां ग्रन्थाश्च वर्णमालाक्रमेण निरप्यन्ते ।

१. अखण्डधामा (1250-1350 A.D.)
अस्याखण्डधाम्नाः गुरुरखण्डानुभूतिः । पञ्चपादिकाविवरणव्याख्याता तत्वदीपनकार अखण्डानन्दसरस्वती च अखण्डानुमूत्यानन्दगिर्योः शिष्यः । यदि स एव तत्वदीपनकारगुरुरखण्डानुभूतिरस्य गुरुस्तर्हि आनन्दगिरिसामयिकस्त्रयोदशचतुर्दशशतकयोरपरपूर्वभागे आसीदिति सिध्यति । अनेन स्वग्रन्थे श्रुतयः परं प्रमाणत्वेनोल्लिखिता नान्ये प्रबन्धकाः प्रबन्धा वा ।
१. उपदेशसाहस्रीव्याख्या - गूढार्थदीपिका ।
शाङ्करोपदेशसाहस्रीगद्यभागव्याख्यात्मकोऽयं ग्रन्थः नातिविस्तृतस्सुगमश्चामुद्रितः मद्रासपुस्तकालये R. 2793 MGOML लभ्यते ।

२. अग्निचित्पुरुषोत्तमः (1600 A.D.)
संक्षेपशारीरकव्याख्यातुष्षोडशशतकीयस्य रामतीर्थस्य शिष्योऽयं पुरुषोत्तमदीक्षित इति ग्रन्थाज् ज्ञायते । अनेन सर्वज्ञात्मकृतसंक्षेपशारीरकस्य सुबोधिनीनाम्नी अन्वर्थसंज्ञा व्याख्या कृता या च आनन्दाश्रममुद्रणालये (A. S. S. 83) मुद्रिता ।

३. अग्निहोत्रभट्टः (1600-1700 A.D.)
दाक्षिणात्योऽयं अग्निहोत्रभट्टः शब्दकौमुदीव्याख्यातुर्द्वादशाहेज्यस्य पुत्रः । चोक्कनाथशास्त्रिणः पौत्रः । रामान्वयप्रसूतस्यास्याग्निहोत्रभट्टस्य माता भवानीनाम्नी । मीमांसान्यायवेदान्तेषु निष्णातोऽयं ज्ञानेन्द्रसरस्वतीवासुदेवेन्द्रसरस्वतीगण्डनृसिम्हसृरि-कालहस्तीशयज्वनां शिष्यः । ज्ञानेन्द्रगण्डनृसिम्हावुभावपि वासुदेवेन्द्रशिष्यौ अग्निहोत्रभट्टस्य गुरू चेति ज्ञायते । कालहस्तीशयज्वा तु अग्निहोत्रमट्टस्य भावुकः गुरुश्च । तेषु ``वासुदेवाभिद्यं सौम्यं पचञ्जलिमहं भजे" इति दर्शनात् व्याकरणशास्त्रे गुरुर्वासुदेवेन्द्रः, वेदान्ते ज्ञानेन्द्रसरस्वती, न्याये गण्डनृसिम्हः, मीमांसायां कालहस्तीशयज्वा इत्यादि ऊहितुं अस्त्यवकाशः ।
सरस्वतीमहालयस्थादर्शग्रन्थान्तपुष्पिकायाः (6095 DC TSML) अग्निहोत्रभट्टकालस्प्तप्तदशशतकार्धावधिकष्षोडशशतकापरार्धादिकश्चेति ज्ञायते ।
१. अद्वैतरत्नकोशपूरणी (तत्वविवेचनी) ।
नृसिम्हाश्रमिकृतस्य तत्वविवेकदीपनापरनाम्नः तत्वविवेकव्याख्यात्मकस्य अद्वैतरत्नकोशस्य व्याख्याभूतोऽयं अद्वैतरत्नकोशपूरण्याख्यः ग्रन्थः मैसूरराजकीयविश्वविद्यालयसंस्कृतग्रन्थमालायां (O.R.I.S.S. 96 Mysore) मुद्रितः ।
तर्कोज्जीविनीनामा केशवमिश्रकृतायाः तर्कभाषाया व्याख्यायाः तत्वबोधिन्याख्यायाः व्याख्याभूतः ग्रन्थः कृतः । गङ्गेशोपाध्यायकृतस्य तत्वचिन्तामणे र्व्याख्यायाः पक्षधरमिश्रकृतायाः आलोकाख्यायाः व्याख्याः ``तत्वचिन्तामण्यालोकस्फूर्तिरिति कृता । सिद्धान्तकौमुदीव्याख्या सुमनोरमाख्या अनेन कृता उतनेति संशयः । प्रामाण्यवादाख्यः कश्चन ग्रन्थः (D. 4272 MGOML) अदसीय लभ्यते ।"

४. अच्युतरायमोडकः (1775-1839 A.D.)
नासिक (पञ्चवटी) क्षेत्रवासी अच्युतशर्मायं नारायणसाठे - अन्नपूर्णयोः पुत्रः । नारायणसाठे इत्याख्यस्यास्य पितुस्सन्यासाश्रमस्वीकारादनन्तरं अद्वैतसच्चिदानन्देन्द्रसरस्वतीति नाम । अद्वैतसच्चिदानन्दस्य शिष्यष्षष्ठिनारायण इति प्रसिद्धः अच्युतशर्मण अद्वैतदर्शने गुरुः । शिवभक्त्युपपदेशकः शैवदर्शनगुरुर्महादेवबु आख्यः । जनस्थानक्षेत्रवासी रघुनाथभट्टश्चास्य गुरुः । पितुरपि प्राप्तविद्योऽयम् । क्रैस्तवीयाष्टादशशतकापरार्धादारभ्य एकोनर्विशतिशतकपूर्वार्धान्तं चतुष्षष्ठितमास्सम । अस्य जीवनकाल इति महावाक्यार्थमञ्जरी शङ्करदिग्विजयव्याख्याया अद्वैतराज्यलक्ष्म्याख्यायाश्चावगम्यते ।
१. अद्वैतजलजातम् २. अद्वैतराज्यलक्ष्मीः ३. अद्वैतविद्याविनोदः (317 BRD, 12 Nasir)  ४. अवैदिकमततिरस्कारः ५. अद्वैतामृतमञ्जरी ६. जीवन्मुक्तिविवेकव्याख्या (पूर्णानन्देन्दुकौमुदी (ASS 20)) 7. पञ्चदशीव्याख्या (ASS 20) 8. बोधैक्यसिद्धिः (आमबोधव्याख्या) ९. महावाक्यार्थमञ्जरी (II. D. 14. A. L.) एते नव ग्रन्था अपि अद्वैतसिद्धान्तसम्बद्धाः । मद्रपुरीविश्वविद्यालये महावाक्यार्थमञ्जरी मुद्रिता (A. O. R. Vol 24 Part II)
अपरे च - अच्युतशतकम् , (नीतिशतपत्रम्) अकल्पितचिदम्बरीस्तोत्रम्, ईशदेशिकविवेचनमञ्जरी, कृष्णलीलामृतकाव्यम्, गीतासीतापतिः, गोदालहरी, (N. S. P.), दृश्यविषयताखण्डनम् (12378 B. R. D.) निरञ्जनमञ्जरी, प्रारब्धध्वान्तसंहृतिः, भागीरथीचम्पूः, भामिनीविलासव्याख्या, मतोपन्यासः, (न स्वतन्त्रः ग्रन्थः साहित्यसारस्य चतुर्थः अध्यायः) रामगीताचन्द्रिका, विष्णुपादलक्षणम्, वेदान्तामृतचिद्रत्नचषकम् (322 B. R. D.), साहित्यसारः . (N. S. P.), सौभाग्यकल्पद्रुमः सौन्दर्यलहरीव्याख्या, सदाचारः सव्याख्यः (10845 B. R. D.), हेरम्बचरणामृतलहरी च ग्रन्थाः कृताः । अधोनिर्दिष्टाः ग्रन्था एतत्कृता उत नेति न निश्चेतुं पार्यन्ते-स्मरहरविजयभागः, कृष्णशतकम्, नीतिशतकम्, रतिनीतिमुकुलम्, विशुद्धमाधवनाटकम्, मृत्युञ्जयचम्पूः, दुःखक्षयेन्दूदयः, द्वैतव्यक्तिक्षयः, मणिमयादर्शः, मुक्तिरमालङ्क्रिया, ईशकेशादिपादान्तस्तुतिः, चिच्चिन्तामणिचिन्तनम्, कारुण्यलहरी, अवयवोक्तिप्रत्युक्तिमञ्जरी, सम्यमसत्कृती, जगद्विजयः, हिरण्यकेशीयाह्निकम्, भूभृदुद्वाहः, शिवस्तवनमञ्जरी, प्रियव्रतचरितचन्द्रिका, शिवस्तुतिमुक्ताभरणम्, सिद्धान्तरत्नसिद्धान्तः, रेवापीयूषलहरी, हरि भक्तिस्प्तरामृतसिन्धुसारः, अमरप्रथमकाण्डटीका, अर्थद्वयात्मिका अमरुशतकटीका, गोवर्धनसप्तशतीटीका, सोपानपञ्चकव्याख्या, सदाशिवोक्तार्थटीका, स्वप्नमन्त्रत्रयीटीका, शृङ्गारकलिकाटीका, स्वकृतबोधैक्यसिद्धिटीका सप्तशतीटीका च । वेदान्त संग्रहनामा ग्रन्थः मद्रपुरीपुस्तकालयेऽमुद्रित उपलभ्यते ।
एषु महावाक्यार्थमञ्जरी पञ्चाशीतिभिरानुष्ठुभैः पद्यैः पूर्णः तत्वमस्यादि महावाक्यानामर्थं प्रसाधयति मद्रपुर्थां विश्वविद्यालये मुद्रिता च ।

५. अद्वयानन्दः (1500-1600 A.D.)
वेदान्तसारकर्तुस्सदानन्दस्य गुरुः शङ्करानन्दप्रशिष्यः प्रथमसदानन्दस्य शिष्योऽयमद्वयानन्दः पञ्चदशषोडशशकीय इति ज्ञायते । अदसीयः वेदान्तसंग्रहाख्यः प्रकरणग्रन्थ अमुद्रित अडयारपुस्तकालये लभ्यते ।

६. अनन्तदेवः (1600 A.D.)
महाराष्ट्रजोऽयमनन्तदेवः देवोपनामककुलप्रसूतः गोदावरीतीरग्रामाभिजनः, एकनाथपौत्रः, आपदेवपुत्रः, द्वितीयानन्तदेवप्रपितामहः, द्वितीयापदेवपितामहः, भट्टनारायण-रामतीर्थयोश्शिष्य इति ``श्रुतं यच्छ्रीरामतीर्थेभ्यः सम्प्रदायसमागतम्" इति एतदीये सिद्धान्ततत्वे, एतदीयपौत्रकृतस्मृतिकौस्तुभे च दर्शनाज्जायते । पूर्वोत्तरमीमांसापण्डितोऽयं मीमांसान्यायप्रकाशिकाकारस्यापदेवस्य पिता गुरुश्च ।
केचित्तु - न्यायप्रकाशोपान्तिमश्लोके ``गोविन्दगुरुपादयो" रितिदर्शनात् भ्रान्ताः आपदेवः गोविन्दशिष्य इति वदन्ति । अथवा अनन्तदेवस्यैव गोविन्ददेव इति नामान्तरं स्यादिति वदन्ति ।
%%% Chrat
१. सिद्धान्ततत्वम् - (P. S. 22)
अयं ग्रन्थः श्रवणादिसाधनपरिपाकसमुत्पन्नतत्वज्ञानस्य अज्ञानतत्कार्यबाधेसति सत्यज्ञानानन्दाखण्डाद्वितीयं वस्तुमात्रमवशिष्यत इति, विक्षेपशक्तिसम्बद्धं चैतन्यं ईश्वरः, आवरणशक्तिसम्बद्धं चैतन्यं जीव इति जीवेश्वरपरिष्कृर्ति कुर्वन्नयं प्रकरणग्रन्थकोटिमारोहति । ग्रन्थोऽयं पण्डितग्रन्थमालायां वाराणस्यां मुद्रितः । अस्य व्याख्या सम्प्रदायनिरूपणाख्यापि अनेनैव कृता ।
२. सिद्धान्ततत्वव्याख्या - सम्प्रदायनिरूपणम् तत्त्वप्रक्रियेतिनामान्तरम् (7547 TSML) मनोरञ्जननाटकम्,  भगवन्नामकौमुदीव्याख्या - ``प्रकाशः" (328 TCD) च अनेन कृतः ।

७. अनन्तभट्टः (1654 A.D.)
यदुभट्टापरनाम्नः दादूभट्टस्य पुत्रोऽयं अनन्तभट्टः बीकानेराधीशेन अनूपसिम्हेन प्रोत्साहितः ग्रन्थमेनञ्चकारेति ज्ञायते । अस्य पुत्रः वैद्यनाथभट्टनामा । १. अद्वैतरत्नाकरः - ग्रन्थोऽयं वेङ्कटेश्वरमुद्रणालये मुद्रितः । अस्य व्याख्या अमरदासवर्मणा कृता रत्नप्रभानाम्नी मुद्रिता च । शास्त्रमालाविवृतिरपि अनेन कृतेति ज्ञायते ।

८. अनुभवानन्दः (1600-1700 A.D.)
अनुभवानन्दोऽयं दक्षिणदेशवासी प्रसिद्धस्य सिद्धान्तसिद्धाञ्जनकर्तुः कृष्णानन्दसरस्वत्याश्शिष्येषु अन्यतम इति अमुद्रिते एतदीये कोशरत्नप्रकाशाख्ये ग्रन्थे ``कृष्णानन्दयतीश्वरं गुरुवरम्" इति, ``गुरुचोदितः कोशरत्नप्रकाशाख्यां व्याख्यां कुर्वे यथामति" इति च दर्शनाज् ज्ञायते । यद्यपि अनेन स्वग्रन्थारम्भे महेश्वरानन्द, शङ्करानन्द, कैवल्यतीर्थ शुद्धानन्दयति पूर्णानन्दाद्याः वहवो नमस्कृतास्तथापि ग्रन्थस्यादिमध्यान्तपुष्पिकाभ्यः कृष्णानन्दसरस्वत्येवास्स्य गुरुरिति निश्चीयते । सिद्धान्तचन्द्रकाव्याख्याता रामानन्दयतिः कृष्णानन्दसरस्वतीकृतानुष्ठानपद्धतिव्याख्याता अप्पाध्वरी, रत्नतृलिकाकारः भास्करदीक्षितश्चास्य सतीर्थ्या इति निश्चीयते ।
अस्य सतीर्थ्येन भास्तरदीक्षितेन शाहजीप्रथमः तञ्जपुरमण्डलाधिराजः स्वपोषक इति रत्नतूलिकायां निर्दिश्यते । शाहजीप्रथमकालः (1684-1711 A. D.) इति इतिहासविदः । एवञ्च गुरोश्शिष्याणां सतीर्थ्यानाञ्च कालस्सप्तदशशतकमिति ज्ञायते ।
१. कोशरत्नप्रकाशः - (7502 TSML)
अमुद्रितोऽयं ग्रन्थः नृसिम्हाश्रमिकृतस्वीयतत्वविवेकव्याख्याभूतस्य तत्व विवेकदीपनापरनाम्न अद्वैतरत्नकोशाख्यग्रन्थस्य व्याख्यात्मकः सरस्वतीमहालये लभ्यते ।
२. प्रभामण्डलम् - (6932 TSML) शास्त्रदीपिकाव्याख्यात्मकोऽयं ग्रन्थः यज्ञनारायणविरचिताद्भिन्न अमुद्रितश्च सरस्वतीमहालये लभ्यते ।

९. अन्नम्भट्टः (1600-1700 A.D.)
आन्ध्रदेशजोऽयं कृष्णानन्दीतीरोपान्तग्रामवासी (चितूर) अद्वैतविद्याचार्यराघवसोमयाजिकुलोत्पन्नः तिरुमलाचार्यसूनुः कौशिकागोत्रजः, सिद्धान्तकौमुदीव्याख्यासिद्धान्तरत्नाकरकर्तू रामकृष्णभट्टस्य सर्वदेवस्य च कनीयान् भ्राता, व्याकरणाचार्यंशेषकृष्णपुत्रस्य शेषवीरेश्वरापराभिधस्य शेषविश्वेश्वरस्य शिष्यः, वेदान्ते ब्रह्मानन्दसरस्वत्याः शिष्यश्चेति एतत्कृतग्रन्थाज् ज्ञायते । अयं काश्यां शास्त्राण्यधीती उवासेति एतत्कृतविश्वेश्वरध्यानात् अवगम्यते । ``काशीगमनमात्रेण नान्नम्भट्टायते द्विज" इत्यन्नम्भट्टप्रशंसनपरा किंवदन्ती च प्रसिद्धा । अस्य शिष्यः वेदाद्रिसूरिरिति प्रसिद्धः वेदान्तपरिभाषाच्याख्यातत्वबोधिनीकारः । न्याये व्याकरणे च एतदीयाः ग्रन्था उपलभ्यन्ते ।
१. मिताक्षरा-
ब्रह्मसूत्रवृत्तिग्रन्थोऽयं भामतीकल्पतरुवैय्यासिकन्यायमालाप्रदर्शितवर्त्मना ब्रह्मसूत्रार्थं वर्णयन् वाक्यविन्यासेन कल्पतरुं न्यायमालामनुसरति । ग्रन्थश्चायं मद्रासराजकीयहस्तलिखितपुस्तकालयमालायां (MGOMLS 19) मुद्रितः ।
तर्कसङ्ग्रहः, तर्कसङ्ग्रहदीपिका, तर्कभाषा, तत्वबोधिनीटीका, जयदेवकृतत्वचिन्तामणिव्याख्या सिद्धाञ्जनम्, महाभाष्यप्रदीपव्याख्या राणकोज्जीविनी, तत्वचिन्तमणिदीधितिव्याख्या सुबुद्धिमनोहरा, तन्त्रवार्तिकव्याख्या सुबोधिनी, स्वरविवेकः, तत्वविवेकदीपनव्याख्याश्च कृता इति ज्ञायते ।

१०. अभिनवनारायणेन्द्रसरस्वती (1600-1700 A.D.)
अयमभिनवनारायणः ज्ञानेन्द्रसरस्वतीशिष्यः, कैवल्येन्द्रसरस्वतीप्रशिष्यः सदाशिवब्रह्मेन्द्रगुरोः परमशिवेन्द्रस्य गुरुः, अग्निहोत्रभट्टस्य सतीर्थ्यश्चेति ज्ञायते । प्रसिद्धनृसिम्हाश्रमिणां शिष्यस्य नारायणाश्रमिण अपरे वयसि सामयिकोऽसाविति कारणादेवास्य अभिनवनारायणत्वं सार्थकं भवति । अनेन स्वीये छान्दोग्यभाष्यव्याख्याने विद्याप्रकाशाख्यः ग्रन्थः बहुवारं प्रमाणीकृतः ।
%%% Chart
१. ऐतरेयोपनिषद्भाष्यटीका - (R. 1475 MGOML)
२. कठोपनिषद्भाष्यटीका - (XXI old h)
३. केनोपनिषद्भाष्यटीका - (XXI 26 Oudh)
४. छान्दोग्योपनिषद् भाष्यटीका (R. 1662 MGOML)
५. पञ्चीकरणभावप्रकाशिका (R. 1492 B. MGOML)
६. प्रश्नोपनिषद्भाष्यटीका (D. 621 MGOML)
७. मुण्डकोपनिषद्भाष्यटीका (XXI 26 Oudh)
८. वार्तिकाभरणम् (B. S. S.)
सुरेश्वराचार्यकृतपञ्चीकरणवार्तिकव्याख्यात्मकोऽयं ग्रन्थः वाराणस्यां मुद्रितः । अनेन पञ्चरत्नव्याख्या कल्पवल्लीनाम्नी कृतेति वदन्ति ।

११. अभिनवसदाशिवब्रह्मेन्द्रः (1800-1900 A.D.)
कर्माकर्मविवेकतत्वम्पदार्थल्क्ष्यैकशतकादिग्रन्थप्रणेतुर्वासुदेवेन्द्रशिष्यरामचन्द्रेन्द्रस्य शिष्योऽयं अभिनवसदाशिवः आत्मानं स्वीयाभिनवत्वविशेषणेनैव प्रसिद्धसदाशिवब्रह्मेन्द्रस्य पश्चाद्भवं दक्षिणदेशजमावेदयति ।
१. पञ्चीकरणम् (D. 4572 MGOML)

१२. अमरदासः (1800-1900 A.D.)
अनेन वेदान्तपरिभाषा-व्याख्यामणिप्रभायाः विषयपरिच्छेदे (Page 298) स्वस्य दीक्षागुरुः श्रीचन्द्र इति निर्दिश्यते । ब्रह्मविज्ञानोऽस्य विद्यागुरुः । नानकवादनात् गुरुसिक्खमतप्रवर्तकनानकसम्प्रदायानुगतोऽपि अद्वैतसम्प्रदाये महानादर अनेन प्रदर्श्यते । अमरदासोऽयं एकोनविंशतिशतकीय इति ग्रन्थेभ्यः निर्णीयते ।
१. वेदान्तपरिभाषा-शिखामणि-व्याख्या-मणिप्रभा ग्रन्थोऽयं वेङ्कटेश्वर स्ट्रीममुद्रणालये मुद्रितः ।
२. ईश-ऐतरेय-कठ-केन-तैत्तरीय-प्रश्न-माण्डूक्य-मुण्डकोपनिषदां व्याख्या मणिप्रभा अनेन कृता । अतएवायं मणिप्रभाकार इत्यपि प्रसिद्धः ।
उपनिषदां व्याख्या मणिप्रभा चौखाम्बामुद्रणालये मुद्रिता ।

१३. अमरानन्दः (1225-1300 A.D.)
कर्णाटकदेशाभिजनस्यापि काशीवासिनोऽस्यामरानन्दस्य पूर्वाश्रमे पितुर्नाम कुमारेश्वर इति ज्ञायते । कर्णाटकदेशशासितुः होयलावंशजस्य त्रयोदशशतकीयस्य नरसिम्हपुत्रसोमेश्वरस्य सामयिकोऽयममरानन्दः अमरानन्दप्रशिष्यस्य जगदाराध्येत्यपरनामकानुपमसुखशिष्यस्य विश्वनाथापराख्यनिरूपमबोधस्य शिष्य इति ग्रन्थेषु दर्शनात् अस्य परात्परगुरुः अमरानन्दप्रथमः, परमगुरुर्जगदाराध्येत्यपरनामा अनुपमसुखः, विश्वनाथापरनामा निरुपमबोधः गुरुरिति निश्चीयते ।
१. स्वात्मयोगप्रदीपः - (R. 3428 MGOML)
आत्माद्वैतप्रतिपादकोऽयं ग्रन्थः जीवेश्वरस्वरूपं वर्णयन् तत्त्वम्पदविचारं सपरिकरं प्रतिपादयति । गौडपादाचार्यं भट्टाचार्यञ्च प्रमाणयति । अस्य व्याख्यापि मूलकृतैव स्वात्मयोगप्रदीपप्रबोधिनी नाम्नी कृता अनुद्रिता च वर्तते ।
२.स्वात्मयोगप्रदीपव्याख्या - प्रबोधिनी (R. 3428 MGOML) विष्णुपुराणव्याख्या विष्णुवल्लभानाम्न्यपि अदसीयः ग्रन्थः ।

१४. अमृतानन्दमुनिः (1400 A.D.)
``अपरं दक्षिणामूर्तिं तमानन्दगिरिं भजे" इति आनन्दशैलाङ्घ्रिसरोजभृङ्गमाराद्भजे यादवशक्रशेलमिति आनन्दगिरियादवेन्द्रगिर्योर्नमस्कृतिश्रवणात् अमृतानन्दोऽयं आन्दगिर-यादवेन्द्रगिर्योश्शिष्य इति ज्ञायते । यादवेन्द्रगिरिस्तु आनन्दगिरेरपि शिष्य इति ज्ञायते ।
१. न्यायदीपावलीव्याख्या - न्यायविवेकः
आनन्दबोधाचार्यकृतन्यायदीपावलीव्याख्यात्मकोऽयं ग्रन्थः प्रथमानुमानपर्यन्तं मुद्रितश्चौखाम्बामुद्रणालये । अमुद्रितश्चापूर्णः ग्रन्थः सरस्वतीमहालये लभ्यते ।

१५. अल्लालसूरिः (1300-1400 A.D.)
कोटिकलाग्रामवासी नागमाम्बात्रिविक्रमाचार्ययोः पुत्र अनन्तार्यप्रज्ञानारण्ययोश्शिष्योऽयमल्लालसूरिस्स्वग्रन्थे कल्पतरुकारं अमलानन्दं व्यासाश्रमशब्देन निर्दिशन् चित्सुखाचार्यं प्रमाणयति । अनेन स्वग्रन्थारम्भे वाचस्पतिमिश्रः व्यासाश्रमः प्रज्ञानारण्यश्च सनामग्रहणं निर्दिष्टाः ।
``अक्षुण्णायामरण्यान्यां भामत्यां वर्त्म यो व्यधात् ।
सुगमं प्रणमामस्तं व्यासाश्रममुनीश्वरम् ।।"
इति व्यासाश्रमः निर्दिष्टः । व्यासाश्रम एवामलानन्द इति केचित् । केचित्तु तर्योर्मेदमामनन्ति । तेषां मतेन सुगम इति भामतीव्याख्या काचनासीद्या नाममात्रेण प्रसिद्धेदानीमिति वक्तव्यमापतति ।
यदि व्यासाश्रमामलानन्दावभिन्नौ (632 TMPL Mss.) तर्हि अल्लालकाल (1300 A. D.) कालादर्वाग्भवः । स्वग्रन्थे परिमलकारस्याप्पय्यदीक्षितास्यानुल्लेखात् अप्पय्यदीक्षितात्प्राचीन इति निश्चीयते । बरोडापुस्तकालयस्थे (13768 B. R. D.) भामतीतिलकादर्शपुस्तकेतु तस्य प्रतिलिपिकाल (1335 A. D.) इति दृश्यते । तस्मात् (1335 A. D.) कालात्प्राक्तन इत्यत्र तु न संशयः । यदि व्यासाश्रमामलानन्दौ भिन्नौ तर्हि अमलानन्दव्यासाश्रमात् सुगमकारः व्यासाश्रमभिन्न इति बरोडास्थप्रतिलिपिकालस्यायमर्थ आपतति यत् कल्पतरुकारादपि प्राचीन इति । केचित्तु अल्लालसूरिः (1600-1700 A. D.) काले आसीदिति वदन्ति । अन्ये तु (1756 A.D.) कालादर्वाग्भव इति वदन्ति । तेषां मतेन बरोडास्थ प्रतिलिपिकाल विरुध्यते ।
१. भामतीतिलकम् (R. 4190 MGOML)
भामतीव्याख्यात्मकोऽयं ग्रन्थ अमुद्रित अपूर्णश्च बहुत्रोपलभ्यते ।

१६. आत्मस्वरूपः (1500-1250 A.D.)
अप्रकाशिते एतत्कृते प्रबोधपरिशोधिन्याख्ये ग्रन्थे ``श्रीनृसिम्हस्वरूपस्य शिष्येणेयं मयेरिता" इति दर्शनात् नृसिम्हस्वरूपशिष्योऽयमात्मस्वरूपः अनुमूतिस्वरूपादिवत्स्वरूपान्तनामा द्वादशशतकादारभ्य त्रयोदशशतकपूर्वार्धावधिककालिकस्स्यादिति ज्ञायते । अत्रेमानि कारणानि -
आत्मस्वरूपरचितौ द्वौ ग्रन्थावुलभ्येते । प्रबोधपरिशोधिनीनामा पञ्चपादिका व्याख्यात्मकः कश्चन ग्रन्थः । द्वादशशतकीयेन आनन्दनुभवेन कृतस्य पदार्थतत्वनिर्णयस्य व्याख्यात्मक अपरो ग्रन्थः । तयोः प्रबोधपरिशोधिन्यां पञ्चपादिकाविवरणाचार्यः, आचार्यसुन्दरपाण्ड्यः, गौडाचार्यः प्रभाकराः भाट्टाश्च निर्दिष्टाः । 31-37 पत्रपर्यन्तमनिर्वचनीयख्यातिस्सम्यङिनरूपिता । न तत्र खण्डनादिग्रन्थो वा ग्रन्थकर्ता वोल्लेखार्हों नोल्लिखितः । पदार्थतत्वनिर्णयटीकायां भाट्टः किरणावली, उदयनः, न्यायसारः, भूषणं, ब्रह्मसिद्धिः, लीलावतीकारः, प्रमाणमाला, प्रशस्तपादभाष्यम्, कुसुमाञ्जलिः, इष्टसिद्धिकारश्चोद्धृताः । प्रमाणमालाकारः आनन्दबोध 11-12 शतकीयः । दशमशतकीय इष्टमिद्धिकारः । आनन्दगिरिश्चित्सुखो वा न र्निदिष्टः । तस्मात्तयोः पूर्वतनः आनन्दबोधादर्वाक्तन इति निश्चीयते । नान्यदत्र प्रमाणमुपलभ्यते ।
१. पञ्चपादिकाव्याख्या - प्रबोधपरिशोधिनी । (R. 3225 MGOML) अमुद्रित उपलभ्यते ।
२. पदार्थतत्वनिर्णयटीका - (R. 4219 MGOML) मध्यलुप्तोऽयं ग्रन्थ अमुद्रितः ।

१७. आदिवेङ्कटयोगी (1700-1800 A.D.)
अयमादिवेङ्कटयोगी भारद्वाजगोत्रजः गन्नेपुडीग्रामाभिजनः कोण्डयपुत्रः सुब्बय्यामात्यपुत्रः रामचन्द्रसरस्वतीप्रशिष्यः स्वयम्प्रकाशसरस्वत्याश्शिष्यः आन्ध्रदेशज इति ज्ञायते । ``अङ्गीकृता हि मत्कृतिरखिलज्ञैर्बालकृष्णयतिवर्यैः । शाहजियाग्रहारस्थितविद्वन्मुख्यसेवितपदाब्जैः ।" इति वदन्नयं ग्रन्थकृत् सिद्धान्तसिद्धाञ्जनकाराणां शाहजग्रामवासिनां बालकृष्णसरस्वतीनां सामयिक इति ज्ञायते ।
१. व्रह्मविन्निधिः - (29 G. 29 AL. R. 4362 MGOML)
अमुद्रितोऽयं पूर्णः ग्रन्थः त्रयस्त्रिंशद्भिः प्रकरणः परिमितः जगन्मिथ्यात्व, गुरुशिष्यलक्षणम्, महावाक्यार्थविचारम् , इन्द्रियजयोपायम् , ध्यानस्वरूपम् , जीवस्वरूपम् ,  सुषुप्तिमृत्योंर्मेदम्, मुक्तिस्वरूपम्, जीवन्मुक्तिविदेहमुक्तिनिरूपणम् , अद्वैतब्रह्मस्वरूपम् , ब्रह्माविचाराधिकारिनिरूपणञ्च कुर्वन् समग्राद्वैतवेदान्तप्रकरणग्रन्थ प्रतिपादितान् विषयान् साकल्येनैकत्र प्रतिपादयति । ग्रन्थोऽयं अडयार - मद्रासराजकीयहस्तलिखितपुस्तकालययोर्लभ्यते ।

१८.आनन्दबोधेन्द्रसरस्वती (1780-1850 A. D.)
सर्वज्ञसरस्वती - रामचन्द्रसरस्वत्योः प्रशिष्यः स्वाराज्यसिद्धिकारस्य गङ्गाधरेन्द्रसरस्वत्याश्शिष्य इति ज्ञायते । अनेन ऋतुरसतुरगमहीशक विकारि शुभवत्सरस्य शिशिरर्तोः इति स्वग्रन्थनिर्माणकाल (1766-श 1i842 A.D.) निर्दिश्यते ।
१. योगवासिष्ठव्याख्या-तात्पर्यप्रकाशः (N. S. P.) योगवासिष्ठव्याख्यत्मकोऽयं ग्रन्थः निर्णयसागरमुद्रणालये मुद्रितः ।

१९. आनन्दरायमखी (1684-1728 A.D.)
गङ्गाधराध्वरिणः पौत्रः, नृसिम्हरायपुत्रः त्र्यम्बकरायज्येष्ठभ्राता चायं आनन्दरायः शाहजीशरभोजीसचिवश्चेति सप्तदशाष्टादशशतकीयः इति ज्ञायते । अस्य कृतिर्वेदान्ततत्वप्रधाना विद्यापरिणयाख्या । मुद्रिताचेयं निर्णयसागरे ।

२०. आनन्दस्वरूपभट्टारकः (1300 A.D.)
अयमानन्दस्वरूपभट्टारक आनन्दात्मशिष्यः । आनन्दात्मा शङ्करानन्दस्यापि गुरुः । एवञ्च शङ्करानन्दसतीर्थ्योऽयं त्रयोदशशतकीय इति निश्चयः ।
१. वाक्यदीपिका (R. 3324 C. MGOML)
वेदोत्तमभट्टारकरचिताया बृहद्वाक्यवृत्याः व्याख्यारूपोऽयं ग्रन्थ अमुद्रितः मद्रपुरीपुस्तकालये लभ्यते ।

२१. आनन्दाश्रमः (1650-1750 A.D.)
सच्चिदानन्दचिद्धनानन्दाश्रमवरशिष्यपरमहंसपरिव्राजकाचार्यआनन्दाश्रमेत्यादिग्रन्थात् सच्चिदानन्दचिद्धनानन्दशिष्योऽयं आनन्दाश्रमः महाराष्ट्रदेशज इति ज्ञायते । अस्याचार्यस्य विश्वेश्वर इति नामान्तरमपि स्यादिति ग्रन्थतो ज्ञायते । एतद्विरचिते आनन्दरससागराख्ये ग्रन्थे शङ्कराचार्यः, मानसोल्लासः, दशश्लोकी, योगवासिष्ठञ्च प्रमाणत्वेन वर्णितानि ।
कालनिर्णये प्रबलतरप्रमाणं नोपलभ्यते । परन्त्वस्य शिष्येण कृतायां मध्वसिद्धान्तभञ्जिन्यां भट्टोजिदीक्षितः निर्दिष्टः । यद्ययं भट्टोजिः प्रसिद्धस्स्यात् तर्हि तस्य सप्तदशशतकीयत्वेन तदर्वाग्भवोऽयमानन्दाश्रम इति निर्णेतुं शक्यते ।
१. आनन्दरससागरः (R. 7543. R. 5749 MGOML)
त्रिपञ्चाशद्भिः प्रकरणैः पूर्णोऽयं प्रकरणग्रन्थः अखण्डार्थत्वं जगन्मिथ्यात्वं आत्मानात्मविचारञ्च कुर्वन् मद्रासराजकीयपुस्तकालये लभ्यते ।

२२. आपदेवः (1600-1700 A.D.)
दाक्षिणात्योऽयं आपदेवः महाराष्ट्रदेशीयः देवोपनामककुलप्रसूतः गोदावरीतीरग्रामाभिजनः एकनाथनप्ता आपदेवप्रथमस्य पौत्रः, सिद्धान्ततत्वकारस्य अनन्तदेवस्य पुत्रः स्मृतिकौस्तुभकारस्य अनन्तदेवस्य पितामहः स्वपितुरेवाधीतदर्शन इति ज्ञायते ।
केचित्तु मीमांसान्यायप्रकाशोपान्त्यश्लोके ``गोविन्दगुरुपादयो" रिति दर्शनात् भ्रान्ताः गोविन्दशिष्य आपदेव इति अनन्तदेवस्यैव गोविन्द इति नामान्तरमिति वा वदन्ति ।
१. वेदान्तसारव्याख्या - बालबोधिनी (V. V. S.)
ग्रन्थोऽयं सदानन्दकृतवेदान्तसारव्याख्यात्मकः । ग्रन्थस्यास्य तत्वदीपिका इत्यपि नामान्तरं दृश्यते । ग्रन्थोऽयं वाणिविलासमुद्रणालाये श्रीरङ्गनगरे मुद्रितः । मीमांसान्यायप्रकाशोऽपि अनेन कृतः ।

२३. ईश्वरतीर्थः (1098-1146 A.D.)
शृङ्गगिरिशङ्करपीठपरम्परागतोऽयमीश्वरतीर्थः नृसिम्हगिरिशिष्यः, नृसिम्हतीर्थगुरुश्चेति शृङ्गगिरिगुरुपरम्परासूच्याः ज्ञायते । भारतीयैतिहासिकत्रैमासिकपत्रिकायां (IHQ Vol. XIV) प्रतिपादितश्च ।
१. शतश्लोकी - (5539 C.C.P.B.) वैराग्यप्रकरणापराभिधानोऽयं ग्रन्थ अमुद्रितः मध्यप्रान्तीयबरार्ग्रन्थसूच्यां लभ्यते ।

२४. उत्तमश्लोकः (1250-1350 A.D.)
उत्तमश्लोकोऽयं शुद्धानन्दशिष्य इति ग्रन्थाज् ज्ञायते । आनन्दगिरिस्तु शुद्धानन्दस्यापि शिष्यः । यद्येवं तर्हि आनन्दगिरिसामयिकस्सतीर्थ्यश्चेति सिध्यति । श्रीकण्ठशास्त्री तु आनन्दगिरिकालं 1114-1228 A. D. इति (IHQ Vol. XIV) वदति । लक्ष्मीधरगुरोरनन्तान्दगिरिरिति नाम । लक्ष्मीधरस्य शिष्येषु शुद्धानन्दः कश्चन आसीत् । यस्य च शिष्यस्स्वयन्प्रकाश इति प्रतिपादयति । यद्येवं शुद्धानन्दशिप्यौ स्वयम्प्रकाशोत्तमश्लोकौ द्वावपि सामयिकौ सतीर्थ्यौ चेति (1400 A.D.) कालीनाविति सिध्यति । ग्रन्थेषु विश्वनाथस्थ विशालनयनानाथस्य च नमस्कृतिप्रकाशनात् वाराणसीवासीति ज्ञायते ।
१. लघुवार्तिकम् (B. S. S. 205)
वेदान्तसूत्रलघुवार्तिकापरनामायं ग्रन्थः आनुष्ठुभेण छन्दसा रचितः, कुत्र चिदेकेन पादेन एकैकस्याधिकरणस्य सङ्ग्राहकः मुद्रितश्च ।
२. न्यायसुधा (B. S. S. 205)
लघुवार्तिकव्याख्यात्मकोऽयं ग्नर्थः लघुवाक्यसुधा लघुन्यायसुधा इत्यपि व्यवह्नियते ।

२५. उत्तमज्ञयतिः (1100-1200 A.D.)
``यन्नामश्रवणाद्भीता वादिनो मोहिता भृशम् ।
तस्मै ज्ञानोत्तमाख्याय जगन्मोहभिदे नमः ।।"
इति तत्वशुद्धिव्याख्यायां दर्शनात् उत्तमज्ञस्यास्य गुरुर्ज्ञानोत्तम इति ज्ञायते । ज्ञानोत्तमोऽयं चोलदेशीयात् मङ्गलग्रामवासिनः ज्ञानोत्तमाद्भिन्नः न्यायसुधादिग्रन्थ प्रणेता विज्ञानात्मचित्सुखज्ञानगिरिगुरुरिति च परिशीलनाज् ज्ञायते । ज्ञानघन शिष्यस्य ज्ञानोत्तमस्य काल (1100-1200 A.D.) एवाञ्चास्यापि कालस्य एव ।
श्रीकण्ठशास्त्रिणां मतेन (IHQ Vol. XIV) ज्ञानघनः प्रकाशात्मनः सामयिक इति ज्ञानघनप्रशिष्यस्योत्तमज्ञस्य कालः (958-1038 A.D.) इति सिध्यति ।
%%% Chart
१. तत्वशुद्धिव्याख्या - (754 C, O. L. 291 TCD) ज्ञानघनविरचितस्य तत्वशुद्धिग्रन्थस्य व्याख्यात्मकोऽयं पूर्णग्रन्थः तिरुवन्तपुरपुस्तकाये लभ्यते ।
२. पञ्चपादिकाव्याख्या - वक्तव्यप्रकाशिका (56. A. Sringeri Math) अमुद्रितोऽयं ग्रन्थः शृङ्गगिरिमठपुस्तकालये लभ्यते ।

२६. उपेन्द्रदत्तः (1859-1900 A.D.)
वाराणसीक्षेत्रवास्ययं भास्करान्दापराभिधानस्यानन्तराममिश्रस्य शिष्यश्चन्द्रमणिपाण्डेयपुत्र इति ज्ञायते ।
१. पञ्चीकरणवार्तिकपाठः । ग्रन्थोऽयं वाराणसीसरस्वतीभवनग्रन्थालये मुद्रितः ।

२७. उमामहेश्वरः (1550-1650 A.D.)
``गुरून् प्रज्ञाजितगुरून्नमाम्यक्कय्यशास्त्रिण" इति ``उमामहेश्वराख्येन वेल्लालकुलजन्मना" इति तत्वचन्द्रिकायां दर्शनात् अक्कय्येत्यपरनाम्न अक्षयसूरेः शिष्य विल्लालकुलप्रसूतः वेङ्कटरायपुत्रः सभारञ्जनकारकविकुञ्जरगुरुः तप्तमुद्राविद्रावणकारस्य भास्करदीक्षितस्य पिता, चोलदेशीयः मोक्षकुण्ड (मुडिकोण्डान्) ग्रामवासी अभिनवकालिदासापरनामा चेति ज्ञायते । एतेन तत्वचन्द्रिकाया भूमिकायां मध्वविध्वंसन-मध्वन्यक्काराभ्यां मध्वमतस्य खण्डितत्वेन श्रीकण्ठमतरामानुजमत एवात्र खण्ड्येते इति प्रतिज्ञातम् । एवं रत्नतूलिकातप्तमुद्राविद्रावणादिकर्त्रा उमामहेश्वरपुत्रेण भास्करदीक्षितेन आत्मनः नृसिम्हाश्रमि कृष्णानन्दसरस्वत्योश्शिष्यत्वं प्रतिपादितम् । एवञ्चाप्पय्यदीक्षितात् उमामहेश्वरः किञ्चिदर्वाचीनो अन्तिमसामयिको वा भवितुमर्हति । T. R. चिन्तामणिमहाशयस्तु स्वसम्पादिते साहित्यरत्नाकरकाव्यभुमिकायां भास्करदीक्षितं रघुनाथसामयिकं वर्णयति । रघुनाथनायकस्य शासनकाल (1600-1650 A.D.) इति P. P. शास्त्रिणः । एवञ्च भास्करदीक्षितपिता उमामहेश्वरः (1550-1650 A. D.) कालवासीति निर्णेतुं शक्यते ।
१. अद्वैतकामधेनुः (7526 T. S. M. L.) सूत्रवृत्तिरूपोऽयं ग्रन्थः परिच्छेदद्वयात्मकस्तेलुगुलिप्यां मुद्रितः ।
२. तत्वचन्द्रिका (R. 5156 MGOML)
निर्गुणब्रह्ममीमांसापरनामायं ग्रन्थः द्वादशभिरुल्लासैः पूर्णः श्रीकण्ठमतं रामानुजमतञ्च खण्डयति । अमुद्रितोऽयं ग्रन्थः अडयारजपुरपोटीखानापुस्तकालयेषु लभ्यते ।
३. विरोधवरूथिनी (R. 4750 MGOML)
सप्तविंशतिभिः परिच्छेदैः पूर्णोऽयं ग्रन्थः रामानुजमतं खण्डयति । ग्रन्थोऽयममुद्रितः अडयार पुस्तकालये च लभ्यते ।।
४. वेदान्तसिद्धान्तसारः (R. 1403 MGOML) । पाणिनीयवादनक्षत्रमाला, भागवतचम्पूश्चानेन कृतौ ।।

२८. उवटाचार्यः (1010-1062 A.D.)
आनन्दपुरवास्तव्यवज्रटाख्यस्य सूनुना ।
उवटेन कृतं भाष्यं पदवाक्यैस्सुनिश्चितैः ।।
ऋष्यार्दीश्च नमस्कृत्य अवन्त्यामुवटेवसन् ।
मन्त्राणां कृतवान् भाष्यं महीं भोजे प्रशासति ।
इति निर्णयसागरमुद्रितायां शुक्लयजुर्वेदसंहितायां दर्शनात् उवटाचार्योऽयं वज्रटपुत्रः आनन्दपुरवासी भोजगजसामयिक इति निर्णीयते । भोजराजश्च यदि धारानगराधीशः सरस्वतीकण्ठाभरणशृङ्गारप्रकाशकारस्स्यात्तर्हि तस्य शासनकाल (1010-1062 A.D.) इति (यपिय्राफिका इण्डिकाया I page 230) प्रामाण्यात् शृङ्गारप्रकाशप्रामाण्याच्चाध्यवसीयते ।।
१. ईशावास्योपनिषद्विवरणम् (A.S.S. 5) । नवमशतकीयोऽयमिति वृद्धत्रय्यां गुरुपादहालदारः ।।

२९.
एकोजीराजः (1700-1750 A.D.)
चोलदेशीयतञ्जपुरराज्यशासकः प्रसिद्धशिवाजीमहाराजपौत्रः तुकोजीकुमाराम्बयोः पुत्र अष्टादशशतकीयः शिवजीराजस्य त्रयः पुत्रा आसन् । शाहजी शरभोजीतुकोजीनामानः । शाहजी राज्यशासनकालः (1684-1717 A.D.) शरभोजीशासनकाल (1712-1728 A.D.) तुकोजीशासनकाल (1728-1735 A.D.) एकोजीशासनकाल (1735-1736 A.D.) इति ज्ञायते । एकोजीराजस्यास्य गुरुर्महादेवपण्डिताख्य राजसभापाण्डितः ।।
१. परब्रह्मतत्वनिरूपणम् (7655 TSML)
शिवपार्वतीसंपादपद्धत्यां रचितोऽयं ग्रन्थः सर्वकारणकारणं व्यापकं ब्रह्मस्वरूपं साधयति । शिवविष्ण्वभेदप्रतिप्रादकोऽयं ग्रन्थः महादेवपण्डितकृतप्रपञ्चसारान्तर्गतः । आदर्शपुस्तकेषु त्रिषु 7655 संख्याके ग्रन्थे महादेवविरचितप्रपञ्चसाराख्यराजरञ्जनपुराण इत्येव दृश्यते । एतत्परिशीलनेन एकोजीकृतत्वेन निर्दिष्टास्सर्वेग्रन्थाः महादेवपण्डितकृता राजरञ्जनाय राजकृता इति निर्दिष्टास्स्युरिति संशय उदेति ।
२. रामानुजमतखण्डनम् - (7659 TSML) प्रपञ्चसारान्तर्गतोऽयं ग्रन्थः कृष्णनारदसंवादशैल्या रामानुजमतदोषान् प्रतिपादयति ।
३. मध्वमतकथनम् - (7960 TSML) मध्वमतस्य दुष्टाचारतां विशदयन्नयं ग्रन्थः प्रपञ्चसारान्तर्गत एव ।

३०. कामाक्षी (1850-1920 A.D.)
भारतीयप्राक्तनविद्यायाः स्त्रीषु विरलताया हेतौ अस्मिन् आधुनिके काले कमलमिव प्रादुर्भूता इयं कामाक्षी कावेरीनदीतीरवर्तिनि मायूरक्षेत्रे चोलदेशे उवास । आन्ध्रदेशजेयं विप्रकुलोत्पन्ना अस्याः पितामहेन प्रपितामहेन वा दुभाषी रामस्वामिशर्मणा नवीन एक अग्रहार निर्मातः व्राह्मणेभ्यश्च दत्तः । इदानीमपि स उग्रहार ``दुभाषग्रहारः" इत्येव प्रसिद्ध आस्ते । अस्याः पिता रामस्वाम्यार्यः । भ्राता सुब्रह्मण्यार्यः ।
कामक्ष्याः भर्ता रामलिङ्गार्यः । स च B. A. उपाधिधारी पाश्चात्यभाषाप्रवीणः । स चाकालिकदैवगत्या (1871 A.D.) काले पञ्चत्वं गतः । भर्तुस्स्वः र्गारोहणकाले कामाक्ष्या वय एकोनर्विशतिः । तदा प्रभृति मातुस्समीपस्था कामाक्षी न्यायशास्त्राण्यद्वैतग्रन्थांश्च सम्यगधीतवती । अपरे वयसि स्वभ्रात्रधीना आसीत् । (1871 A.D.) काले मृतभर्तृकायाः अस्या वय एकोनविंशतिरित्युक्ते अस्य जननकाल (1852 A. D.) इति भवति ।
१. अद्वैतदीपिका - मधुसूदनसरस्वतीकृतां अद्वैतसिद्धिं तत्प्रतिपादितान् मिथ्यात्वपरिष्करांश्च सङ्गृह्णात्ययं ग्रन्थः । ग्रन्थोऽयं श्रीविद्यामुदणालये मुद्रितः ।
२. श्रुतिरत्नप्रकाशटिप्पणी - त्र्यम्बकभट्टकृतस्य श्रुतिरत्नप्रकाशस्य व्याख्या । मुद्रितोऽयं श्रीविद्यामुद्रणालये ।
३. श्रुतिमतोद्योतटिप्पणी - त्र्यम्बकभट्टकृतश्रुतिमतोद्योतव्याख्यात्मकोऽयं ग्रन्थश्रीविद्याविलासमुद्रणालये कुम्भधोणे मुद्रितः ।
न्यायबोधिनीनीलकण्ठीयविषयाणां सङ्ग्राहकोऽयमन्यः विषयमालाख्यश्च ग्रन्थस्समारचितः ।।

३१. कालहस्तीशयज्वा (1550-1620 A.D.)
``प्रणम्य दक्षिणामूर्ति रघुनाथाश्रमान् गुरून्" इति वदन्नयं कालहस्तीशयज्वा रघुनाथाश्रमशिष्य इति ज्ञायते । अस्य प्रपितामहस्सोमनाथः । पितामहः यौवनभारती । पिता मल्लिकार्जुनः । अस्य पत्नी यज्ञाम्बानाम्नी । अस्य द्वौ पुत्रौ-रङ्गनाथः बालकविश्च । बालकविना रामवर्मविलासाख्यः ग्रन्तः (R. 3673 MGOML) प्रणीतः । रङ्गनाथस्तु आश्रमस्वीकारादनन्तरं अखण्डानन्दसरस्वतीति प्रसिद्धः स्वयम्प्रकाशशिष्य ऋजुपकाशिकाकारः । इति श्रीकण्ठशास्त्री (HIQ. Vol. XIV) प्रतिपादयति । नलगन्तुवंशजोऽयं कालहस्तीशः काञ्चीपुरवासी कामाक्षीभक्तः कामाक्षीदास इति प्रसिद्धश्च । प्रसिद्धाप्पय्यदीक्षितादनेन शिवदीक्षा स्वीकृता च ।
%%% Chart
१. भावप्रकाशिका (9. 1. 13. AL)
नृसिंहाश्रमीयाद्वैतरत्नकोशापरनामकस्य तत्वविवेकव्याख्यातत्वदीपनस्य व्याख्यात्मकोऽयं ग्रन्थ अडयार पुस्तकालये लभ्यते ।।
२. भेदधिक्कारविवृतिः (R. 2187 MGOML)
नृसिंहाश्रमीयभेदधिक्कारव्याख्यात्मकोऽयं ग्रन्थ अमुद्रितः मद्रपुरीराजकीयहस्तलिखित ग्रन्थालये लभ्यते ।। वसुचरितचम्पूरप्यदसीया कृतिः ।।

३२. काशीनायशास्त्री (1800 A.D.)
अनेन वेदान्तपरिभाषाख्यः कश्चन प्रकरणग्रन्थः धर्मराजाध्वरिकृतवेदान्तपरिभाषासारगर्भित रचितः । ग्रन्थोऽयं दासगुप्तमहाशयेन (HIP Vol. II Page 54) निर्दिष्टः ।

३३. कुमारभवस्वामी (1300 A. D.)
अयं कुमारभवस्वामी मध्वध्वंसन-अद्वैतकौस्तुभ-भावनापुरुषोत्तमनाटकादिकर्तू रत्नखेटश्रीनिवासदीक्षितस्य पञ्चमः कुलपूरुषः । एतद्विरचितः ग्रन्थः अद्वैतचिन्तामणिः । ग्रन्थोऽयं रुक्मिणीकल्याणव्याख्यायां बालयज्ञदेवेश्वरकृतायां निर्दिष्टः । अम्बास्तवव्याख्यायाञ्च निर्दिश्यते । एवं मद्रासविश्वविद्यालयप्रकाशिते NCC ग्रन्थेऽपि निर्दिष्टः ।

३४. कृष्णः (1800 A.D.)
ब्रह्मानन्दापराभिधानकृष्णानन्दयतेश्शिष्योऽयं कृष्णपण्डितः दक्षिणदेशीय स्तमिलभाषायास्सहित्ये च निष्णातश्चोलदेशवासीति च ज्ञायते । एतदीयः ग्रन्थः कैवल्यदीपिकाख्याः
 । स च तमिलभाषायां ताण्डवरायस्वामिना विरचितस्य कैवल्यनवनीताख्यस्य संस्कृतीकरणरूपः पद्यबद्धः ग्रन्थः ।
ताण्डवरायश्चाष्टादशशतकीयश्चोलदेशीयः नन्निलग्रामवासी नारायणसरस्वतीशिष्यः । केचित्तु ताण्डवरायस्वामिनं तिरुवानैक्कावल-गोपालशास्त्रिणः प्राप्तविद्या, नारायणगुरुोः प्राप्तदीक्षं, परमशिवेन्द्रशिष्यं सदाशिवब्रह्मेन्द्रसतीर्थ्यञ्च वदन्ति । सर्वथापि ताण्डवरायस्वामिन अर्वाचीनोऽयं कैवल्यदीपिकाकारः कृष्णपण्डितः अष्टादशशतकीय इति निर्णयः । अज्ञातनामधेयेनास्य शिष्येण वाक्यसुधायाः व्याख्या कृता ।
१. कैवल्यदीपिका सव्याख्या - (10. E. 18. AL)
भागद्वयपरिमितोऽयं ग्रन्थः । प्रथमे तत्वप्रदीपिकाक्ये अष्टोत्तरशतैः पद्यैः परिमिते अद्वैतमततत्वानि प्रतिपादितानि । द्वितीये संशयच्छेदाख्ये षट्सप्तत्यधिकशतैः पद्यैः औपनिषदप्रमाणपूर्वकं अद्वैतवस्तुन्यारोपितास्संशयाश्छिद्यन्ते । अस्यव्याख्यापि प्रभाख्या मूलकृतैव कृता । अमुद्रितोऽयमडयारपुस्तकालये प्राप्यते । 
२. प्रत्यक्त्वस्वप्रकाशवादः - (23. F. 44 ग्र 22. AL)
३. भगवद्गीताभावप्रकाशः - (1300 CCPB ?)
४. अद्वैतसुधाबिन्दुः - (7999 BRD)
५. छान्दोग्योपनिषत्कारिकाः-
ग्रन्थोऽयं जयपुरपोटीखाना सूच्यां दृश्यते । परन्तु कैवल्यदीपिकाया ऋते अन्ये ग्रन्थाः किमनेनैव कृता उतान्येनति निर्णेतुं न पार्यन्ते ।

३५. कृष्णकान्तविद्यावागीशः (1800-1900 A.D.)
रामकृष्णभट्टाचार्यपौत्रेण कालीचरणन्यायालङ्कारतारिणीदेव्योः पुत्रेण रामनारायणतर्कवागीशशिष्येण नव्यनैय्यायिकेन अनेन कृता दशश्लोकीसिद्धान्तबिन्दुन्यायरत्नावल्याः व्याख्या प्रदीपिकाख्या कृतेति विमर्शका वदन्ति । परन्तु स ग्रन्थः न्यायग्रन्थ इति भाति । राजेन्द्रलालसूच्यामेव दृश्यतेऽयं ग्रन्थः । नास्यविवरणमुपलभ्यते ।

३६. कृष्णगिरिः (1800-1900 A.D.)
ग्रन्थान्तपुष्पिकाया उय कृष्णगिरिः कैलासाचालशिष्य इति ज्ञायते । वाराणसीवास्ययं रणोद्दीपाख्यनृपसामयिकस्तत्पोषितश्च । बाणेन्दुरन्ध्रेन्दुमिते वत्सरे अनेन स्वग्रन्थः कृतः । एवञ्च ग्रन्थनिर्माणकालः (1915 सं. 1858 A. D.) इति ज्ञायते ।
१. मोक्षसिद्धिः ।
ग्रन्थोऽयं मुन्नालालमुद्रणालये वाराणस्यां मुद्रितः । गद्यमयोऽयं ग्रन्थः शङ्कराचार्यं प्रमाणयन् कर्मोपासनाज्ञानानां क्रमेणानुष्ठानात् अद्वयात्मकब्रह्मज्ञानावाप्तिरिति प्रतिपादयन् प्रकरणग्रन्थतामर्हति ।

३७. कृष्णनाथन्यायपञ्चाननः (1892 A. D.)
अर्जुनमिश्रवंशजः केशवचन्द्रपुत्रश्चायं नवद्वीपवासी मीमांसार्थसंग्रह-सांख्यतत्त्वकौमुदी - अभिज्ञानशाकुन्तलादीनां व्याख्याता एकोनविंशतिशतकीय इति ज्ञायते । अनेन कृता वेदान्तपरिभाषाव्याख्या आशुबोधिन्याख्या कल्कत्तानगरे मुद्रिता ।

३८. कृष्णमिश्रः (1060-1100 A. D.)
आध्यात्मिकनाटककर्ता राजस्थानान्तर्गत ``जाजभुक्ति" वासी चायं कृष्णमिश्रः स्वग्रन्थे कीर्तिवर्माणं प्रस्तुवन् तत्सामयिकत्वं प्रतिपादयति । कीर्तिवर्मा च (1060-1100 A. D.) काले शशासेति ग्रन्थस्थभूमिकाया ज्ञायते । संस्कृतसाहित्ये आध्यात्मिकवस्तूनि नाटकवस्तूकृत्य नाटकप्रणयनजं श्रेयः प्रथमत अस्यैवेति ज्ञायते ।
१. प्रबोधचन्द्रोदयम् ।
ग्रन्थोऽयं निर्णयसागरमुद्रणालये मुद्रितः । ग्रन्थस्यास्य बहूनि व्याख्यानानि विद्यान्ते यानि चान्यत्र प्रतिपादितानि ।

३९. कृष्णलीलाशुकः (1168-1293 A. D.)
अयमेव बिल्वमङ्गलाचार्य इति प्रसिद्धः । यद्येवं पितास्य दामोदरः । माता नीली । ईशानदेवशिष्योऽयं कृष्णलीलाशुकः केरलदेशवसी गोश्रीसाम्राज्यान्तर्गते तिरुच्चूर्समीपवर्तिनि ``तेक्कमठे" लब्धवासः (1220-1300 A. D.) काल उवासेति प्राच्यभाषामहासभ्मेलनस्य नवमाघिवेशपत्रिकायां (प्रसीडिङ्स आफनैन्त ओरियण्टल कान्फ्रेंस) दृश्यते ।
यद्ययं सरस्वतीकण्ठाभरणव्याख्यापुरुषकार-कृष्णलीलामृतादिकर्ता कृष्णलीलाशुकस्स्यात्तर्हि कालोऽस्य द्वादशत्रयोदशशतकोत्तरार्घपूर्वार्धावधिक इति निर्णेतुं शक्यते । यतः पुरुषकारे हेमचन्द्रः निर्दिष्टः । हेमचन्द्रश्च (1166-1220 A. D.) काल आसीत् । पुरुषकारस्य पंक्तयश्च देवराजकृतायां निघण्डुटीकायामुद्धृताः । तस्मात्तर्योर्मध्यपातीति निश्चीयते । देवराजकालश्च (1293-1343 A. D.) सीताराम जयरामजोशी तु कृष्णलीलाशुकं एकादशशतकीयं (1100 A. D.) स्वीये संक्षिप्तसंस्कृतसाहित्येतिहासे प्रतिपादयति ।
श्रीचिह्नकाव्यनामा कश्चन ग्रन्थः कृष्णलीलाशुकेन कृतः । तत्रायं श्लोकः - ``श्रीपद्मपादमुनिवर्यविनेयवर्गश्रीभूषणां मुनिरसौ कविसार्वभौमः ।" इति । तेक्कमठस्य स्थापना पद्भपादाचार्येण कृता स्यात् । एवमेव पद्मपादशिष्योऽयं कृष्णलीलाशुक इति वक्तुं शक्यते । केनोपनिषद्वयाख्यायाञ्च दृश्यमाणा Page 10 ``ब्रह्मभूयं गते पूर्वे शङ्करे कृत्स्नवेदिनि पूर्वे च तादृशे" इत्यादिश्लोकीया ``पूर्वे च तादृश्ये" इति पंक्तिरपि पद्मपादाचार्यं निर्दिशति । एवं च वेदान्ते पद्भपादाचार्यशिष्योऽयं कृष्णलीलाशुकः नवमशतकीय इति वर्णितम् । कृष्णलीलाशुकं केचित् वङ्गदेशजं वदन्ति ।
१. केनोपनिषद्व्याख्या - शङ्करहृदयङ्गमा (MGOMLS)
कृष्णकर्णामृतम्, कृष्णलीलाचरितम्, कृष्णलीलाकौमुदी, गोविन्दस्तोत्रम्, गौविन्दैकविंशतिः, बालकृष्णक्रीडाकाव्यम्, पुरुषकारः, बिल्वमङ्गलास्तोत्रम्, श्रीचिहकाव्यञ्च अनेेन रचितानि ।।

४०. कृष्णशास्त्री (1650-1700)
भावाज्ञानप्रकाशिकाकर्तू रङ्गनाथसूरेश्शिष्यस्य शिवरामस्य पिता अयं कृष्णशास्त्रीति ज्ञायते । अनेनाद्वैतविद्याविजयाख्याः ग्रन्थः कृत इति ज्ञायते । स च ग्रन्थः नोपलभ्यत्ते । परन्तु तत्पुत्रकृतायां अमुद्रितायां भावाज्ञानप्रकाशिकायां निर्दिष्टः ।।
१. अद्वैतविद्याविजयः Q.

४१. कृष्णशास्त्री (1870-1939 A.D.)
महामहोपाध्याय इति राजकीयबिरुदभूषितोऽयं कृष्णशास्त्री करुङ्गुलं कृष्णशास्त्रीति प्रसिद्धः । श्रीशालिपुर (तिन्नेवली) समीपस्थे कृष्णतटाकाख्ये (करुङ्गुलम्) ग्रामे लब्धजन्मान एते हरिहरशास्त्रिणां शिष्याः । अद्वैतसभा पण्डितोऽयं मद्रपुरीसंस्कृतकलाशालाप्रधानाध्यापकपदवी बहुवत्सरपर्यन्तं अलञ्चकार । अपरे वयसि प्राप्तसन्यास विदेहमुक्तो जातः ।।
१. अधिकरणचतुष्टयी ।
आनन्दमयाधिकरण-यथाश्रयभावाधिकरण - ऐहिकाधिकरणलिङ्गभूतस्त्वाधिकरणानां विषयविचारप्रधानोऽयं ग्रन्थः बालमनोरमामुद्रणालये मद्रासनगरे मुद्रितः ।
२. ब्रह्मसूत्रनुगुण्यसिद्धिः । ग्रन्थोऽयं गोपालविलासमुद्रणालये कुम्भधोणे मुद्रितः ।
३. परिमलः (A. M. S. S. 25)
गङ्गाधरेन्द्रसरस्वतीकृतस्वाराज्यसिद्धिव्याख्यात्मकोऽयं ग्रन्थः आर्यमत संवर्द्धिनीग्रन्थमालायां मुद्रितः ।

४२. कृष्णानन्दभारती (1400 A. D.)
``श्रीगुरुं भारतीतीर्थं विद्यारण्यमुनीश्वरम्" इत्यादिना भारतीतीर्थविद्यारण्यौ नमन्नयं कृष्णानन्दभारती भारतीतीर्थविद्यारण्यशिष्यः शृङ्गगिरिमठपरम्परागतः दक्षिणदेशीयश्चतुर्दशशतकीय इति सिध्यति ।
१. महावाक्यार्थदर्पणम् । (54. A. 41. A.L.)
गुरुशिष्यप्रणल्यां औपनिषदमहावाक्यानां अद्वैतब्रह्मावबोधकत्वं प्रदर्शयन्नयं प्रकरणग्रन्थः पूर्ण अमुद्रितश्च अडयारपुस्तकालये प्राप्यते ।

४३. कृष्णानन्दसरस्वती (1825-1900 A. D.)
``नमोऽस्तु गुरवे तस्मै यत्र तत्र निवासिने । सच्चिदान्दपादाय चरणाय मुहुर्मम ।" इति ब्रह्मसूत्रकुतूहले सच्चिदानन्दाश्रमिणं, ``यस्योपदेशमाहात्म्यात् जडा अपि विनिर्गताः । संसारबन्धात् तं वन्दे वासुदेवेन्द्रयोगिनम् ।।" इति वासुदेवेेन्द्रञ्च नमस्यन्नयं बालकृष्णानन्दापरनामा कृष्णानन्दसरस्वती विद्यायां सच्चिदानन्दाश्रमशिष्यः, आश्रमस्वीकारे वादेवेन्द्रशिष्यश्चेति ज्ञायते । वाराणसीवास्ययं स्वनिर्मितेषु ग्रन्थेषु शङ्कराचार्यं विशेषतः प्रणमति ।।
अनेन रामदुर्गाख्यविप्रप्रेरणया वेदेन्दुवसुभूमिते शालिवाहनशके 1814-1890 A. D. शास्त्राकूतप्रकाशः ब्रह्मसूत्रकुतूहलञ्च रचिते इति ज्ञायते । एवञ्चास्य काल एकोनविशतितमशतकमिति निश्चयः । अस्य शिष्यः हरिकृष्णशर्माख्याः ।
१. अद्वैतपञ्चरत्नव्याख्या-किरणावली (R. 1613 (b) MGOML)
२. ब्रह्मसूत्रकुतूहलम् ।
ग्रन्थस्यास्यावतरणिकायां अद्वैतसिद्धान्ताः विशदीकृताः । अथातो ब्रह्मजिज्ञासा त आरभ्य ज्योतिश्चरणाभिधानात् - इत्यन्तानां चतुर्विशतिसूत्राणां काचित्स्वतन्त्रा वृत्तिरद्वैतमतपोषिण्यारचिता । ग्रन्थोऽयं राजराजेश्वरीमुद्रणालये काश्यां मुद्रितः ।।
३. शास्त्राकूतप्रकाशः 
ग्रन्थोऽयं द्वैतिवादनिरसनपूर्वकं अद्वैतवादं राद्धान्तयति । ग्रन्थेऽस्मिन् दृश्यप्रपञ्चस्य मिथ्यात्वमुपपाद्य एकजीववादानेकजीववादयोर्निरवकाशत्वमुपवर्ण्य सजातीयविजातीयस्वगतभेदशून्यत्वमेवात्मत्वमिति सिद्धान्तितम् । आह्निकत्रयपरिपरिमितोऽयं लघुग्रन्थः जगदीश्वरप्रेस बम्बई नगरे मुद्रितः ।।
४. तिमिरोद्धाटनम् ।
लघुग्रन्थेऽस्मिन् अद्वैतात्मस्वरूपवर्णनेन साकं अवतारपदार्थः मुक्तिशब्दार्थविचारश्च प्रसङ्गवशादुपवर्णिताविति विशेषः । ग्रन्थोऽयं नषनलप्रेस राजकोट नगरे मुद्रितः ।।
५. भगवद्गीतैकदेशपरामर्शः ।
गीताया भेदवादे तात्पर्यं निरस्याद्वैतब्रह्मवादे तात्पर्यं साधितम् । मुद्रितश्चया गर्वर्नमण्टप्रेस गोण्टारपुरनगरे ।।

४४. कृष्णानन्दसरस्वती (1900 A. D.)
कैवल्यानन्दकृष्णानन्दयोश्शिष्योऽयं वाराणसीवासी एकोनविंशतिशतकीय इति ज्ञायते । १. अद्वैतसाम्राज्यम् N.S.P. 2. अज्ञानतिमिरदीपकः IOL 3. कैवल्यगाथा Kalpathi Press Bombay ४. गीतासारोद्धारः ; ५. खानुमूतिप्रकाशः (विलासः) (9976 BRd.) ६. ब्रह्मगीताव्याख्या - ``चित्प्रकाशिनी" (136 नासिकसूच्यां) ७. अध्यात्मभागवतव्याख्या ``चित्प्रकाशिनी" । नासिकूसूच्यां (108) अदसीयाः ग्रन्था दृश्यन्ते ।।

४५. कृष्णानुभूतियतिः (1500-1600 A. D.)
आनन्दानुभूतिं नमस्कुर्वाणोऽयं विबुधेन्द्रापरनामा आनन्दानुभूतिशिष्यः । केरलदेशजोऽयं स्वसामयिकौ केरलशासकौ राजराजरविवर्मनामानौ शारीरकमीमांसासूत्रसङ्ग्रहे निर्दिशति । केरलीयः वासुदेवकविनामा प्रसिद्धः रविवर्मणस्सभायामासीद्यस्य च कालः पञ्चदशशतकमध्यभाग इति कृष्णाचार्यसंस्कृतसाहित्यचरिते 252 पुटे दृश्यते । यदि कृष्णानुभूतिनिर्दिष्ट एवायं रविवर्मास्यात्तर्हि अस्यापि काल (1550 A. D.) इति वक्तुं शक्नुमः । मद्रासराजकीयहस्तलिखितपुस्तकालयस्थ शारीरकशास्त्रसंग्रहे विद्यमानमिदं - ``गीर्वाणेन्द्रसरस्वत्याः पादाब्ज हृदि विभ्रतः" इति पद्यञ्च नृसिम्हाश्रमिगुरुं गीर्वाणेन्द्रं स्मारयतीव ।
१. ब्रह्मसूत्राधिकरणन्यायानुक्रमणिका (R. 3305 B. MGOML)
२. शारीरकमीमांसाशास्त्रसंग्रहः (R. 2905 MGOML)
ब्रह्मसूत्रवृत्तिरूपोऽयं जीवब्रह्माभेदप्रतिपादकः ग्रन्थः अडयारपुस्तकालये अनन्तशयनपुस्तकालये विश्वभारतीशान्तिनिकेतनपुस्तकालये चामुद्रित उपलभ्यते ।
३. अधिकरणसंख्याश्लोकाः । (C. S. S. 1)

४६. केशवशास्त्री (1800-1850 A. D.)
स्वग्रन्थे आत्मानमानन्दाश्रमप्रशिष्यं लक्ष्मणपन्तशर्मणः राजारामशास्त्रिवाल शास्त्रिणोश्च शिष्यं प्रतिपादयन्नयं केशवशास्त्री एकोनविंशतिशतकीयः । तत्र लक्ष्मणपन्तशर्म आनन्दाश्रमशिष्यश्च ।
१. आत्मसोपानम् । ग्रन्थोऽयं 479 आनुष्ठुभैः पद्यैः घटितः प्रकरणग्रन्थः । गुरुशिष्यप्रणाल्या प्रवृत्तोऽयं ग्रन्थः आत्मनः नित्यशुद्धबुद्धमुक्तत्वं, ज्ञेयविषयाणां मिथ्यात्वं जीवन्मुक्तत्वोपपत्तिञ्च प्रतिपादयति । मुद्रितश्चायं वाराणसी पण्डितसंस्कृतग्रन्थमालायाम् (Vol. IV Pandit Series Vol. IV)

४७. कैवल्येन्द्रः (1550-1650 A. D.)
अनेन वेदान्तभूषणनामा ग्रन्थः कृतः । ग्रन्थोऽयं एतच्छिष्येण विद्येन्द्रसरस्वत्या स्वकृते वेदान्ततत्वसाराख्ये सरस्वतीमहालयस्थे (7575 TSML) ग्रन्थे निर्दिष्टः । (368 DCAL Vol. VI) नीलकण्ठवाजपेयीयायां सुखबोधिनीव्याख्या याञ्च निर्दिष्टः ।

४८. गणपतिशास्त्री (1850-1920 A. D.)
चोलदेशीयमन्नार्गुडिसमीपस्थपाङूगानाङुग्रामवासी राजुशास्त्र्यपरनामकत्यागराजशास्त्रिणः, पषवानेरीस्वामिनश्च प्राप्तन्यायवेन्दान्तव्याकरणशास्त्रः आशुकविर्गणपतिशास्त्री स्वीयसप्तदशमे वयस्येव कटाक्षशतककविरिति प्रसिद्धः । कोनेरिराजपुरवासिना साम्बशिवार्येण अद्वैतग्रन्थप्रकाशनाख्यं मुख्यं महत् प्रयोजनं मनसि कृत्वा कुम्भघोणे स्थापितस्य अद्वैतमञ्जरीग्रन्थमालां समारब्धवतः श्रीविद्यामुद्रणालयस्य ग्रन्थप्रकाशनाय नियुक्तेषु पण्डितेषु गणपतिशास्त्र्यप्यन्यतमः । द्वारकापीठाधिष्ठातृणा वेदान्तशास्त्रे समाधेयत्वेन (1905 A. D.) काले प्रकाशितानां सप्त प्रश्नानां उत्तरदानेन गणपतिशास्त्री वेदान्तकेसरीति बिरुदेन भूषितश्च ।
१. अथशब्दविचारः २. ईशावास्यविवृतिः ३. कटाक्षशतकम् ४. केनोपनिषद्विवृतिः ५. गुरुराजसप्ततिः ६. जीवविजयचम्पूः ७. दुर्गाशतकम्, ७. दुर्गाशतकम्. ८. ध्रुवचरितम् ९. नैर्गुण्यसिद्धिः १०. पार्थप्रहारः ११. वैदिकाभरणव्याख्या - मुकुरः १२. शारीरकमीमांसारहस्यम्, १३. श्रवणविधिवाक्यार्थश्चानेन कृताः ग्रन्थाः । केचित् अमुद्रिताः, केचन अद्वैतसमास्वर्णमहोत्सवपत्रिकायां प्रकाशिताः । वैदिकामरणव्याख्या तु अण्णामलैविश्वविद्यालये मुद्रिता ।

४९. गणेशशर्मा (1906 A. D.)
दक्षिणदेशजोऽयं दक्षिणार्काडजिल्लान्तर्गतदीक्षामङ्गलाग्रहारवासी मद्रपुरीसंस्कृतकलाशालाप्रधानाध्यापक्योः कृष्णतटाक - कृष्णशास्त्रि रामचन्द्रदीक्षितयोश्शिष्यः र्विशतिशतकीयोऽयं गणेशशर्मा ।
१. सुरेश्वरहृदयम् । ग्रन्थेऽस्मिन् शबलब्रह्मणः जगत्कारणत्वं, साक्षिशब्दार्थः भावाविद्या, अज्ञानशब्दनिर्वचनम्, भावरूपाज्ञाने सुरेश्वरसम्मतिः, सुषुप्त्यवस्थायां भावाविद्यासद्भावे भाष्यवार्तिकतात्पर्यनिर्णयः, वार्तिकमतेन दृष्टिसृष्टिवादः प्रतिपादितः । ग्रन्थोऽयं मद्रासलाजर्नलमुद्रणालये मुद्रितः ।।

५०. गीर्वाणेन्द्रसरस्वती (1600-1700 A. D.)
अमरेन्द्रसरस्वतीप्रशिष्यः विश्वेश्वरसरस्वतीशिष्यश्चायं गीर्वाणेन्द्रसरस्वती नृसिम्हाश्रमिबोधेन्द्रयत्योः गुरुश्चेति ज्ञायते । रघुनाथाश्रमिसामयिकः जगन्नाथाश्रमिसामयिकश्व ।।
%%% Chart
1. प्रपञ्चसारसङ्ग्रह (17637 TSML)
शाङ्करप्रपञ्चसारसङ्ग्रहात्मकोऽयं ग्रन्थः अमुद्रितस्सरस्वतीमहालये लभ्यते ।।

५१. गुरुमूर्तिशास्त्री (1850-1910 A. D.)
अनेनाद्वैतानन्दतीर्थकृताया ब्रह्मसूत्रतात्पर्यदीपिकायाः व्याख्यातात्पर्य विमर्शिनी रचिता । ग्रन्थोऽयं तेलुगु लिप्यां मुद्रितः । एकोनविंशतिशतकीयोऽयमिति ज्ञायते ।।

५२.
गुरुस्वामिशास्त्री (1911 A. D.)
चोलदेशीयकुम्भघोणसमीपस्थे वरहूर्ग्रामे लब्धजन्मायं गुरुस्वामिशास्त्री साहित्यवेदान्तपण्डितः मद्रपुरीसंस्कृतकलाशालाया प्राप्तविद्यः वैद्यनाथधर्माम्बिकयो पुत्रः मद्रपुरीसंस्कृतकलाशालाध्यापकेभ्यः बालसुब्रह्मण्यशास्त्रि वैद्यानाथशास्त्रि-रामचन्द्रदीक्षितेभ्यः प्राप्तविद्यः विंशतिशतकीय इति ज्ञायते ।
१. शारीकव्याख्याप्रस्थानानि ।।
विमर्शकसरण्या शाङ्करभाष्योपरि प्रवृत्तानां पद्मपादमण्डन-सुरेश्वर-विमुक्तात्म-प्रकटार्थकार-ज्ञानधनपाद-नृसिम्हभट्टोपाध्यायकृतानां व्याख्यानानां आशयभेदा उपवर्णिताः । ग्रन्थोऽयं बालमनोरमामुद्रणालये मुद्रितः ।
५३. गोपालः (1850 A. D.)
अस्य पिता मुद्रगलभट्टः । औत्तरोऽयं गोपालः एकोनविंशतिशतकीय इति ज्ञायते ।
१. विवेकामृतम् - (10. B. 6. AL) द्वाभ्यां प्रकरणाभ्यां विभक्तोऽयं प्रकरणग्रन्थः देहविचारं, तात्पर्यशोधनम्, मुक्तिस्वरूपविचारम्, ब्रह्मणस्सर्वात्मकत्वं तस्याद्वयत्वञ्च वर्णयति । ग्रन्थोऽयमडयारपुस्तकालये लभ्यते ।

५४. गोपालबालयतिः (1500-1600 A. D.)
``श्रीमद्यतीन्द्रमानम्य जगन्नाथश्रमं गुरुम्" इति एतदीये ग्रन्थे दर्शनात् बालगोपालापरनामायं गोपालवालयतिः नृसिम्हाश्रमिसामयिकः नृसिम्हाश्रमिसतीर्थ्यश्चेति षोडशशतकीयः । अस्य शिष्यस्स्वयम्प्रकाशाख्यश्शाङ्करैकश्लोकव्याख्यामधुमञ्जरीकारः ।
१. काठोपनिवच्छाङ्करभाष्यव्याख्या - भाष्यविवरणम् । (ASS. 7) ग्रन्थोऽयं आनन्दाश्रमे मुद्रितः ।
२. मधुमञ्जरी - (D. 4706 MGOML) शाङ्करमनीषापञ्चक व्याख्यात्मकोऽयंग्रन्थ अपूर्ण अमुद्रितश्च मद्रासराजकीयपुस्तकालये लभ्यते । 

५५. गोपालानन्दसरस्वती (माकिं 1700-1900 A. D.)
योगानन्दशिष्योऽयं गोपालनन्दसरस्वती वाराणसीवासीति ज्ञायते । गोपालानन्दमेनं दासगुप्तमहाशयस्सप्तदश - एकोनर्विशतिशतकमध्यवर्तिनं प्रतिपादयति (HIP. Vol. II Page 57)
1. अखण्डात्मप्रकाशिका - (R. 3891 MGOML)
निष्कामकर्मणश्चित्तशुद्धिः, चित्तशुद्धेस्संसारयाथात्म्यावबोधः तस्माद्वैराग्योत्पत्तिः, ततो मुमुक्षुत्वम्, तस्मादुपायोपेयेषणापूर्वकसर्वकर्मसन्यासः इति निरूप्य महावाक्यार्थपरिज्ञानं मननं योगाभ्यसाच्चित्तस्य प्रत्यक्प्रवणता, तस्माच्च सर्वाविद्योच्छेदः, तस्माच्च पूर्णाद्वितीयसच्चिदानन्दात्मना अवस्थितिरिति प्रतिपादितम् । प्रकरणग्रन्थोऽयं पूर्ण अमुद्रितश्च मद्रासराजकीयपुस्तकालये मैसूरपुस्तकालये चोपलभ्यते ।
२, ईशावास्योपनिषट्टीका (4527 B. R. D.)
३. वेदान्तामृतम् (4913 B. R. D.)

५६. गोविन्दभट्टः (1752 A. D.)
आहिताग्निरयं गोविन्दभट्टः विश्वनाथभट्टपुत्रः उत्तरभारतीय इति ज्ञायते । अनेनात्मार्कबोधाख्ये ग्रन्थे ग्रन्थनिर्माणकालः 1676 शकवत्सराणीति प्रतिपाद्यते । एवञ्चास्यकाल अष्टादशशतकमिति ज्ञायते ।
१. आत्मार्कबोधः (B. D. 285 R. A. S. Bomboy)
विंशत्यधिकशतपद्यैः षड्भिरध्यायैश्च पूर्णोऽयं ग्रन्थः प्रकरणग्रन्थकोटिमारोइति । अमुद्रितोऽयं रायलआसियाटिकसोसाइटि बाम्बेनगरस्थपुस्तकालये लभ्यते ।
२. वेदान्ततात्पर्य निवेदनम् (908 D. C. Panjab)

५७. गोविन्दानन्दसरस्वती (1885 A. D.)'
माधवानन्दसरस्वतीशिष्योऽयं गोविन्दानन्दसरस्वती वाराणसीवासीति ज्ञायते । अनेन ब्रह्मसुधाकारिकाख्यः ग्रन्थः कृतः । स च निर्णयसागरमुद्रणालये मुद्रितः ।

५८. गोविन्देन्द्रयतिः (1800-1900 A.D.)
नारायणेन्द्रशिष्योऽयमाधुनिकः । अनेन रचिते तत्वानुभवाख्येऽमुद्रिते (R. 47 B. MGOML) पुस्तके स्वानुभवपद्धत्यां जीवन्मुक्तस्थितिः वर्णिता । अन्योऽयं ``असङ्गात्मप्रकाशिकाख्यः" ग्रन्थः विश्वभारतीशान्तिनिकेतनपुस्तकालयेऽमुद्रितः (3035 V. B. S.) लभ्यते ।

५९. गौरीनाथशास्त्री (1850-1920 A.D.)
वन्दरपुरवासी स्वामिनाथशास्त्रिपौत्रः, नृसिम्हशास्त्रिपुत्रः शाण्डिल्यगोत्रजोऽयं सच्चिदानन्दसरस्वतीशिष्य इति ज्ञायते ।।
१. शाङ्करभाष्यगाम्भीर्यनिर्णयखण्डनम् । ग्रन्थोऽयं चोलदेशीयेन रामसुव्रह्मण्यशास्त्रिणा कृतस्य शाङ्करभाष्यनिर्णयाख्यग्रन्थस्य खण्डनात्मक अद्वैतग्रन्थः । ग्रन्थोऽयं वाणीविलासमुद्रणालये मुद्रितः ।।

६०. गङ्गाधरः (1200 A.D.)
अस्य पिता मनोरथः । अयं (1137 A. D.) काले आसीदिति ``यफि़ग्राफि़का इंडिकापत्रिकायाः द्वितीयभागे 333 पुटे दृश्यते । अनेन अद्वैतशतकमितिग्रन्थः कृत इति च दृश्यते ।।"

६१. गङ्गाधरभगवद्भक्ताकिङ्करः (1750-1850 A. D.)
अग्निहोत्रिवीरेश्वरसूरिपौत्रः सदशिवभट्टपुत्रः महाडकरइत्युपनामायं गङ्गाधरभगवद्भक्तविङ्करः वत्सर्षिगोत्रजः अष्टादशशतकीय इति निश्चीयते ।।
१. सुबोधिनी-शारीरकसूत्रसारार्थचन्द्रिकानामायं ग्रन्थः सूत्रवृत्तिरूप अमुद्रितः लन्दननगरस्थभारतकार्यालयपुस्तकालये उज्जयिन्याञ्च लभ्यते ।।

६२. गङ्गाधरेन्द्रसरस्वती (1780-1880 A. D.)
दक्षिणदेशजोऽयं गङ्गाधरेन्द्रसरस्वती सर्वज्ञसरस्वतीप्रशिष्य रामचन्द्रसरस्वतीशिष्यः अष्टदशशतकोत्तरार्धादारभ्य एकोनविंशतिशतकमध्यभागावधिके काले (1780-1880 A. D.) आसीदिति निश्चीयते । एतदीये स्वाराज्यसिद्धिग्रन्थे ``वस्वब्धिमुन्यवनिमानशक" इति ग्रन्थनिर्माणकाल (1728 श= 1826 A.D) इति दृश्यते ।।
१. वेदान्तसिद्धान्तसूक्तिमञ्जरी (C.S.S. 4) । अप्पय्यदीक्षितकृतसिद्धान्तलेशसारात्मकोऽयं ग्रन्थस्समन्वयाविरोधसाधनफलाख्यैश्चतुर्भिः परिच्छेदैः परिच्छिन्नः कल्कत्तासंस्कृतमालायां मुद्रितः । अस्य व्याख्यापि ``प्रकाशाख्या" अनेन कृता ।
२. सिद्धान्तचन्द्रिकाव्याख्या - ``उद्गारः" ।। रामब्रह्मेन्द्रापरनाम्ना रामभद्रानन्देन विरचितस्य सिद्धान्तचन्द्रिकाग्रन्थस्य व्याख्यात्मकोऽयं ग्रन्थः गोपालनारायण मुद्रणालये मुम्बय्यां मुद्रितः ।
३. स्वाराज्यसिद्धिः । पद्यात्मकोऽयं ग्रन्थः द्वैतखण्डनपूर्वकं अध्यारोपापवादकैवल्याख्यैस्त्रिभिः प्रकरणैः निरावरणस्वप्रकाशपरमात्मस्वरूपं प्रकाशयति । ग्रन्थोऽयं आर्यमतसंवर्द्धिनीमुद्रणालये (A. M. S. S. 25) मद्रासनगरे मुद्रितः । अस्य व्याख्या मूलकृत्कृता कैवल्यद्रुमाख्या, कृष्णशास्त्रिकृता परिमलाख्या च वर्तेते ।
४. निर्वाणाष्टकव्याख्या (264 TCL), 5. प्रणवकल्पप्रकाशः,
६. शुकाष्टकाव्याख्या (214 Nasik) च अनेन कृताः ग्रन्थाः ।।

६३. गङ्गापुरीभट्टारकः (1100-1200 A. D.)
न्यायरत्नदीपावली-इष्टसिद्धिविवरणकार आनन्दानुभव एवायमिति आनन्दानुभवमधिकृत्य विचारावसरे प्रतिपादितम् । गङ्गापुरीभट्टारककृतत्वेन त्रिपेदीमहाशयेन वर्णितः पदार्थतत्वनिर्णयाख्यः ग्रन्थः मद्रासराजकीयहस्तलिखितपुस्तकालये आनन्दानुभवोपज्ञ इति दृश्यते । आनन्दानुभवस्यैव सन्यासस्वीकारात्पूर्वं गङ्गापुरीभट्टारक इति स्यान्नाम । दासगुप्तस्तु (950-1050 A. D.) अस्य कालं वर्णयति ।
१. पदार्थतत्वनिर्णयः (R. 2981 MGOML) अस्य व्याख्या आनन्दगिरिकृता चास्ति ।।

६४. घनश्यामः (1706-1786 A.D.)
महाराष्ट्रजातीयः चौण्डाजीपण्डित - आर्यक - सर्वज्ञकवि - वश्यवाचसू - कण्ठीरव - सरस्वतीति उपनामभिः प्रसिद्धोऽयं घनश्यामः तञ्जावूरनगरवासी तुक्कोजीक्षितीशस्य (1729-35 A.D.) मन्त्री चासीत् । काशी - महादेवयोः पुत्रः चौण्डबालाजीपौत्रः मौनभार्गववशंजः ईशपण्डितस्य कनिष्ठभ्राता शाकम्भर्याश्च भ्राता सुन्दरीकमलयोः पतिः चन्द्रशेखर-गौवर्धनयोः पिता संस्कृते प्राकृते प्रान्तीय भाषायाञ्च बहूनां ग्रन्थानां कर्ता चेति (J. O. R. Vol III & IV) ज्ञायते ।
१. अद्वैतबोधः । ग्रन्थोऽयं विद्धशालभञ्जिकोपोद्धाते घनश्यामपत्न्या निर्दिष्टः । (TSML 4678)
२. प्रचण्डराहूदयः । ग्रन्थोऽयं बेलगामनगरे (1960) मुद्रितः ।

६५. चन्द्रशेखरभारतिः (2000 A.D.)
एते शृङ्गगिरिशङ्कराचार्यपीठाधिष्ठिताः चतुस्त्रिंशत्तमा आचार्याः नितान्तं विद्वांसः ज्ञानानिष्ठाश्चासन् । विवेकचूडामणिव्याख्या अदसीयः ग्रन्थः । व्याख्येयं तत्र तत्र सूत्रभाष्याद्युक्तानर्थान् विशदय्य मूलग्रन्थं सुलभयति । मुद्रितश्चायं ग्रन्थः ।

६६. चन्द्रिकाचार्यभिक्षुः (1800-1900 A. D.)
दक्षिणदेशवासी कृष्णानन्द-रामब्रह्मेन्द्र सरस्वत्योश्शिष्योऽयं चन्द्रिकाचार्यभिक्षुः गुरुणा रामब्रह्मेन्द्रेण ``रामब्रह्मेन्द्र" इति दत्तनाम मन्नार्गुडिराजुशास्त्र्यपराभिधत्यागराजमखिप्रोत्साहित अद्वैतसिद्धान्तगुरुचन्द्रिकां चकारेति एकोनविंशतिशतकीय इति च निश्चीयते ।
१. अद्वैतसिद्धान्तगुरुचन्द्रिका - 
ग्रन्थोऽयं भागद्वयपरिमितः प्राचीनान् निलीनविलीनान् अद्वैतसिद्धान्तान् श्रुतियुक्तिभ्यां च गवेषयति । ग्रन्थोऽयं ओरियण्टलमुद्रणालये मद्रासनगरे मुद्रितः । अस्य व्याख्या मूलकृतैव अमृतरसझरीनाम्नी कृता ।

६७. चिद्धनानन्दः (1895-1945 A.D.)
वाराणसीवासी रामकृष्णपरमहंससम्प्रदायगतोऽयं चिद्धनान्दः लक्ष्मणशास्त्रिद्राविडशिष्यः विंशतिशतकीय इति निर्णीयते । अच्युतानन्दोऽस्य दीक्षागुरुः ।
%%% Chart
१. ब्रह्मसूत्रभाष्यनिर्णयः ।
ग्रन्थोऽयं ब्रह्मसूत्रोपरि प्रवृत्तानां शाङ्कररामानुजभास्करमध्वनिम्बार्कादिविबिधभाष्यस्य गुणदोषविवेचनां कुर्वन् शाङ्करभाष्यस्यैव बादरायणव्याससम्मतत्व प्रदर्शयति । समालोचनात्मकोऽयं ग्रन्थः बोधायनादिवृत्तिसद्भावे संशयमुत्पादयति । ग्रन्थोऽयं काशीस्थरामकृष्णसेवाश्रममुद्रणालये मुद्रितः ।

६८. चिरञ्जीविभट्टाचार्यः (1675-1750 A.D.)
गौडदेशजोऽयं काश्यपगोत्रजः चिरञ्जीविभट्टः रामदेवचिरञ्जीव-वामदेवचिरञ्जीव इत्यपरनामा सामुद्रिकाचार्यपौत्र इति ज्ञायते । स्वकृतविद्वन्मोदतरङ्गिण्यां 
``द्वैताद्वैतमतादिनिर्णयविधिप्रोद्बुद्धबुद्धिश्रुतः
भट्टाचार्यशतावधान इति यो गौडोद्भवोऽभूत् कविः ।
विद्वन्मोदतरङ्गिणी ननु चिरञ्जीवेन तज्जन्मना"
इत्यादिवर्णनात् शतावधानभट्टाचार्यापराभिधानराघवेन्द्रस्य पुत्र इति निर्णीयते । स्वकृतकाव्यविलासे रघुदेवचरणध्यानात् रघुदेवन्यायालङ्कारशिष्य इति ज्ञायते ।
१. विद्वन्मोदतरङ्गिणी - 
ग्रन्थोऽयं श्रव्यकाव्यशैल्यां प्रणीतः विविधदर्शनसारसङ्ग्रहात्मकः शास्त्रप्रसिद्धैर्युक्तिजार्लैर्व्यावहारिकैश्च युक्तिजालैः वैष्णवशाक्तवैशेषिकनैय्यायिक अलङ्कारिकसिद्धान्तान् तदीयानाशयभेदांश्चोपवर्णयन् नास्तिकैस्साकं ततद्दार्शनिकानां वादप्रतिवादं सयुक्तिक प्रकाशयन् नास्तिकवादखण्डनाय वेदान्तशास्त्रस्य न्यायशास्त्रापेक्षित्ववर्णनपूर्वकं आत्मन अद्वितीयत्वं 
``ब्रह्मण्येव समुत्पद्य जगत्तत्रैव लीयते ।
बुद्बुदा इव तोयेषु शुक्तयो रजतेष्विव" ।
इत्यादिना प्रतिपादयति । प्रसङ्गवशात् हरिहराद्वैतभावनां प्रकाशयति । ग्रन्थोऽयं कल्कत्ताग्रन्थमालायां वेङ्कटेश्वरमुद्रणालये बम्बईनगर्यां काव्यप्रकाशमुद्रणालये वाराणस्यां च मुद्रितः ।
काव्यविलासः, शृङ्गारतटिनी, वृत्तरत्नावली, माधवचम्पूरिति ग्रन्थाश्च रचिताः ।

६९. चोक्कनाथदीक्षितः (1600-1650 A.D.)
सभानाथ सर्वक्रत्वपराभिधानोऽयं अग्निहोत्रभट्टपितामहः, चोलदेशीयः नारायणशास्त्रि - गणपत्यम्बयोः पुत्रः, नारायणसुब्रह्मण्यवैद्यनाथयोः, द्वादशाहयाजि बालपतञ्जल्यपरनाम्नः बालचन्द्रदीक्षितस्य च पिता, अस्यैव सुन्दरेशापरनामा च ज्ञायते । अस्य जामाता रामभद्रमखी । रामभद्रमखिनश्श्वशुरः गुरुश्चायं चोक्कनाथदीक्षितस्सप्तदशशतकीयः नल्लादीक्षितधर्मराजाध्वरीन्द्रसामयिकश्चेति निश्चीयते । वेङ्कटेशदीक्षितशिष्यश्चायं प्रसिद्धवैय्याकरणोऽपीति ज्ञायते । अस्यैव सञ्चारिमहाभाष्यमिति नाम प्रसिद्धम् ।
शाब्ददीपिका, घातुरत्नावली, भाष्यरत्नावलीति व्याकरणग्रन्थाश्चानेन कृताः । शाब्ददीपिकायाः शब्दकौमुदी नाम प्रसिद्धम् । अत्रायं वंशपरम्परावृक्षः-
%%% Chart
१. वेदान्तदीपिका (356 T. C. D. Vol. III)
विषयविदग्धापरनामायं ग्रन्थ अमुद्रितस्तिरुवनन्तपुरपुस्तकालये लभ्यते । शाङ्करभाष्यगतार्थसंग्राहकोऽयं सूत्रवृत्तिरूपः ग्रन्थः चौखाम्बामुद्रणालये मुद्रितश्च ।।


 ``" ``" ``" ``" ``" ``" ``" ``" ``" ``" ``" ``" ``" ``" ``" ``" ``" ``" ``" ``" ``" ``" ``" ``" ``" ``" ``" ``" ``" ``" ``" ``" ``" ``" ``" ``" ``" ``" ``" ``" ``" ``" ``" ``" ``" ``" ``" ``" ``" ``" ``" ``" ``" ``" ``" ``" ``" ``" ``" ``"

ऽ  ?
``" ``" ``" ``" ``" ``" ``" ``" ``" ``" ``" ``" ``" ``" ``" ``" ``" ``" ``" ``" ``" ``"
`' `' `' `' `' `' `' `' `' `' 

ऽ  ।   ॥ ?
