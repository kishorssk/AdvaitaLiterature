\chapter{अद्वैतग्रन्थकाराः}
प्रथमे प्रकरणे प्रसिद्धतमानां अद्वैताचार्यपदव्यपदेश्यानां अद्वैतग्रन्थकर्तृणां तिथिक्रमानुसारी निर्देशः कृतः । प्रकरणेऽस्मिन् द्वितीये प्रसिद्धाः अद्वैतग्रन्थ प्रणेतारः तेषां ग्रन्थाश्च वर्णमालाक्रमेण निरप्यन्ते ।

१. अखण्डधामा (1250-1350 A.D.)
अस्याखण्डधाम्नाः गुरुरखण्डानुभूतिः । पञ्चपादिकाविवरणव्याख्याता तत्वदीपनकार अखण्डानन्दसरस्वती च अखण्डानुमूत्यानन्दगिर्योः शिष्यः । यदि स एव तत्वदीपनकारगुरुरखण्डानुभूतिरस्य गुरुस्तर्हि आनन्दगिरिसामयिकस्त्रयोदशचतुर्दशशतकयोरपरपूर्वभागे आसीदिति सिध्यति । अनेन स्वग्रन्थे श्रुतयः परं प्रमाणत्वेनोल्लिखिता नान्ये प्रबन्धकाः प्रबन्धा वा ।
१. उपदेशसाहस्रीव्याख्या - गूढार्थदीपिका ।
शाङ्करोपदेशसाहस्रीगद्यभागव्याख्यात्मकोऽयं ग्रन्थः नातिविस्तृतस्सुगमश्चामुद्रितः मद्रासपुस्तकालये R. 2793 MGOML लभ्यते ।

२. अग्निचित्पुरुषोत्तमः (1600 A.D.)
संक्षेपशारीरकव्याख्यातुष्षोडशशतकीयस्य रामतीर्थस्य शिष्योऽयं पुरुषोत्तमदीक्षित इति ग्रन्थाज् ज्ञायते । अनेन सर्वज्ञात्मकृतसंक्षेपशारीरकस्य सुबोधिनीनाम्नी अन्वर्थसंज्ञा व्याख्या कृता या च आनन्दाश्रममुद्रणालये (A. S. S. 83) मुद्रिता ।

३. अग्निहोत्रभट्टः (1600-1700 A.D.)
दाक्षिणात्योऽयं अग्निहोत्रभट्टः शब्दकौमुदीव्याख्यातुर्द्वादशाहेज्यस्य पुत्रः । चोक्कनाथशास्त्रिणः पौत्रः । रामान्वयप्रसूतस्यास्याग्निहोत्रभट्टस्य माता भवानीनाम्नी । मीमांसान्यायवेदान्तेषु निष्णातोऽयं ज्ञानेन्द्रसरस्वतीवासुदेवेन्द्रसरस्वतीगण्डनृसिम्हसृरि-कालहस्तीशयज्वनां शिष्यः । ज्ञानेन्द्रगण्डनृसिम्हावुभावपि वासुदेवेन्द्रशिष्यौ अग्निहोत्रभट्टस्य गुरू चेति ज्ञायते । कालहस्तीशयज्वा तु अग्निहोत्रमट्टस्य भावुकः गुरुश्च । तेषु ``वासुदेवाभिद्यं सौम्यं पचञ्जलिमहं भजे" इति दर्शनात् व्याकरणशास्त्रे गुरुर्वासुदेवेन्द्रः, वेदान्ते ज्ञानेन्द्रसरस्वती, न्याये गण्डनृसिम्हः, मीमांसायां कालहस्तीशयज्वा इत्यादि ऊहितुं अस्त्यवकाशः ।
सरस्वतीमहालयस्थादर्शग्रन्थान्तपुष्पिकायाः (6095 DC TSML) अग्निहोत्रभट्टकालस्प्तप्तदशशतकार्धावधिकष्षोडशशतकापरार्धादिकश्चेति ज्ञायते ।
१. अद्वैतरत्नकोशपूरणी (तत्वविवेचनी) ।
नृसिम्हाश्रमिकृतस्य तत्वविवेकदीपनापरनाम्नः तत्वविवेकव्याख्यात्मकस्य अद्वैतरत्नकोशस्य व्याख्याभूतोऽयं अद्वैतरत्नकोशपूरण्याख्यः ग्रन्थः मैसूरराजकीयविश्वविद्यालयसंस्कृतग्रन्थमालायां (O.R.I.S.S. 96 Mysore) मुद्रितः ।
तर्कोज्जीविनीनामा केशवमिश्रकृतायाः तर्कभाषाया व्याख्यायाः तत्वबोधिन्याख्यायाः व्याख्याभूतः ग्रन्थः कृतः । गङ्गेशोपाध्यायकृतस्य तत्वचिन्तामणे र्व्याख्यायाः पक्षधरमिश्रकृतायाः आलोकाख्यायाः व्याख्याः ``तत्वचिन्तामण्यालोकस्फूर्तिरिति कृता । सिद्धान्तकौमुदीव्याख्या सुमनोरमाख्या अनेन कृता उतनेति संशयः । प्रामाण्यवादाख्यः कश्चन ग्रन्थः (D. 4272 MGOML) अदसीय लभ्यते ।"

४. अच्युतरायमोडकः (1775-1839 A.D.)
नासिक (पञ्चवटी) क्षेत्रवासी अच्युतशर्मायं नारायणसाठे - अन्नपूर्णयोः पुत्रः । नारायणसाठे इत्याख्यस्यास्य पितुस्सन्यासाश्रमस्वीकारादनन्तरं अद्वैतसच्चिदानन्देन्द्रसरस्वतीति नाम । अद्वैतसच्चिदानन्दस्य शिष्यष्षष्ठिनारायण इति प्रसिद्धः अच्युतशर्मण अद्वैतदर्शने गुरुः । शिवभक्त्युपपदेशकः शैवदर्शनगुरुर्महादेवबु आख्यः । जनस्थानक्षेत्रवासी रघुनाथभट्टश्चास्य गुरुः । पितुरपि प्राप्तविद्योऽयम् । क्रैस्तवीयाष्टादशशतकापरार्धादारभ्य एकोनर्विशतिशतकपूर्वार्धान्तं चतुष्षष्ठितमास्सम । अस्य जीवनकाल इति महावाक्यार्थमञ्जरी शङ्करदिग्विजयव्याख्याया अद्वैतराज्यलक्ष्म्याख्यायाश्चावगम्यते ।
१. अद्वैतजलजातम् २. अद्वैतराज्यलक्ष्मीः ३. अद्वैतविद्याविनोदः (317 BRD, 12 Nasir)  ४. अवैदिकमततिरस्कारः ५. अद्वैतामृतमञ्जरी ६. जीवन्मुक्तिविवेकव्याख्या (पूर्णानन्देन्दुकौमुदी (ASS 20)) 7. पञ्चदशीव्याख्या (ASS 20) 8. बोधैक्यसिद्धिः (आमबोधव्याख्या) ९. महावाक्यार्थमञ्जरी (II. D. 14. A. L.) एते नव ग्रन्था अपि अद्वैतसिद्धान्तसम्बद्धाः । मद्रपुरीविश्वविद्यालये महावाक्यार्थमञ्जरी मुद्रिता (A. O. R. Vol 24 Part II)
अपरे च - अच्युतशतकम् , (नीतिशतपत्रम्) अकल्पितचिदम्बरीस्तोत्रम्, ईशदेशिकविवेचनमञ्जरी, कृष्णलीलामृतकाव्यम्, गीतासीतापतिः, गोदालहरी, (N. S. P.), दृश्यविषयताखण्डनम् (12378 B. R. D.) निरञ्जनमञ्जरी, प्रारब्धध्वान्तसंहृतिः, भागीरथीचम्पूः, भामिनीविलासव्याख्या, मतोपन्यासः, (न स्वतन्त्रः ग्रन्थः साहित्यसारस्य चतुर्थः अध्यायः) रामगीताचन्द्रिका, विष्णुपादलक्षणम्, वेदान्तामृतचिद्रत्नचषकम् (322 B. R. D.), साहित्यसारः . (N. S. P.), सौभाग्यकल्पद्रुमः सौन्दर्यलहरीव्याख्या, सदाचारः सव्याख्यः (10845 B. R. D.), हेरम्बचरणामृतलहरी च ग्रन्थाः कृताः । अधोनिर्दिष्टाः ग्रन्था एतत्कृता उत नेति न निश्चेतुं पार्यन्ते-स्मरहरविजयभागः, कृष्णशतकम्, नीतिशतकम्, रतिनीतिमुकुलम्, विशुद्धमाधवनाटकम्, मृत्युञ्जयचम्पूः, दुःखक्षयेन्दूदयः, द्वैतव्यक्तिक्षयः, मणिमयादर्शः, मुक्तिरमालङ्क्रिया, ईशकेशादिपादान्तस्तुतिः, चिच्चिन्तामणिचिन्तनम्, कारुण्यलहरी, अवयवोक्तिप्रत्युक्तिमञ्जरी, सम्यमसत्कृती, जगद्विजयः, हिरण्यकेशीयाह्निकम्, भूभृदुद्वाहः, शिवस्तवनमञ्जरी, प्रियव्रतचरितचन्द्रिका, शिवस्तुतिमुक्ताभरणम्, सिद्धान्तरत्नसिद्धान्तः, रेवापीयूषलहरी, हरि भक्तिस्प्तरामृतसिन्धुसारः, अमरप्रथमकाण्डटीका, अर्थद्वयात्मिका अमरुशतकटीका, गोवर्धनसप्तशतीटीका, सोपानपञ्चकव्याख्या, सदाशिवोक्तार्थटीका, स्वप्नमन्त्रत्रयीटीका, शृङ्गारकलिकाटीका, स्वकृतबोधैक्यसिद्धिटीका सप्तशतीटीका च । वेदान्त संग्रहनामा ग्रन्थः मद्रपुरीपुस्तकालयेऽमुद्रित उपलभ्यते ।
एषु महावाक्यार्थमञ्जरी पञ्चाशीतिभिरानुष्ठुभैः पद्यैः पूर्णः तत्वमस्यादि महावाक्यानामर्थं प्रसाधयति मद्रपुर्थां विश्वविद्यालये मुद्रिता च ।

५. अद्वयानन्दः (1500-1600 A.D.)
वेदान्तसारकर्तुस्सदानन्दस्य गुरुः शङ्करानन्दप्रशिष्यः प्रथमसदानन्दस्य शिष्योऽयमद्वयानन्दः पञ्चदशषोडशशकीय इति ज्ञायते । अदसीयः वेदान्तसंग्रहाख्यः प्रकरणग्रन्थ अमुद्रित अडयारपुस्तकालये लभ्यते ।

६. अनन्तदेवः (1600 A.D.)
महाराष्ट्रजोऽयमनन्तदेवः देवोपनामककुलप्रसूतः गोदावरीतीरग्रामाभिजनः, एकनाथपौत्रः, आपदेवपुत्रः, द्वितीयानन्तदेवप्रपितामहः, द्वितीयापदेवपितामहः, भट्टनारायण-रामतीर्थयोश्शिष्य इति ``श्रुतं यच्छ्रीरामतीर्थेभ्यः सम्प्रदायसमागतम्" इति एतदीये सिद्धान्ततत्वे, एतदीयपौत्रकृतस्मृतिकौस्तुभे च दर्शनाज्जायते । पूर्वोत्तरमीमांसापण्डितोऽयं मीमांसान्यायप्रकाशिकाकारस्यापदेवस्य पिता गुरुश्च ।
केचित्तु - न्यायप्रकाशोपान्तिमश्लोके ``गोविन्दगुरुपादयो" रितिदर्शनात् भ्रान्ताः आपदेवः गोविन्दशिष्य इति वदन्ति । अथवा अनन्तदेवस्यैव गोविन्ददेव इति नामान्तरं स्यादिति वदन्ति ।
%%% Chrat
१. सिद्धान्ततत्वम् - (P. S. 22)
अयं ग्रन्थः श्रवणादिसाधनपरिपाकसमुत्पन्नतत्वज्ञानस्य अज्ञानतत्कार्यबाधेसति सत्यज्ञानानन्दाखण्डाद्वितीयं वस्तुमात्रमवशिष्यत इति, विक्षेपशक्तिसम्बद्धं चैतन्यं ईश्वरः, आवरणशक्तिसम्बद्धं चैतन्यं जीव इति जीवेश्वरपरिष्कृर्ति कुर्वन्नयं प्रकरणग्रन्थकोटिमारोहति । ग्रन्थोऽयं पण्डितग्रन्थमालायां वाराणस्यां मुद्रितः । अस्य व्याख्या सम्प्रदायनिरूपणाख्यापि अनेनैव कृता ।
२. सिद्धान्ततत्वव्याख्या - सम्प्रदायनिरूपणम् तत्त्वप्रक्रियेतिनामान्तरम् (7547 TSML) मनोरञ्जननाटकम्,  भगवन्नामकौमुदीव्याख्या - ``प्रकाशः" (328 TCD) च अनेन कृतः ।

७. अनन्तभट्टः (1654 A.D.)
यदुभट्टापरनाम्नः दादूभट्टस्य पुत्रोऽयं अनन्तभट्टः बीकानेराधीशेन अनूपसिम्हेन प्रोत्साहितः ग्रन्थमेनञ्चकारेति ज्ञायते । अस्य पुत्रः वैद्यनाथभट्टनामा । १. अद्वैतरत्नाकरः - ग्रन्थोऽयं वेङ्कटेश्वरमुद्रणालये मुद्रितः । अस्य व्याख्या अमरदासवर्मणा कृता रत्नप्रभानाम्नी मुद्रिता च । शास्त्रमालाविवृतिरपि अनेन कृतेति ज्ञायते ।

८. अनुभवानन्दः (1600-1700 A.D.)
अनुभवानन्दोऽयं दक्षिणदेशवासी प्रसिद्धस्य सिद्धान्तसिद्धाञ्जनकर्तुः कृष्णानन्दसरस्वत्याश्शिष्येषु अन्यतम इति अमुद्रिते एतदीये कोशरत्नप्रकाशाख्ये ग्रन्थे ``कृष्णानन्दयतीश्वरं गुरुवरम्" इति, ``गुरुचोदितः कोशरत्नप्रकाशाख्यां व्याख्यां कुर्वे यथामति" इति च दर्शनाज् ज्ञायते । यद्यपि अनेन स्वग्रन्थारम्भे महेश्वरानन्द, शङ्करानन्द, कैवल्यतीर्थ शुद्धानन्दयति पूर्णानन्दाद्याः वहवो नमस्कृतास्तथापि ग्रन्थस्यादिमध्यान्तपुष्पिकाभ्यः कृष्णानन्दसरस्वत्येवास्स्य गुरुरिति निश्चीयते । सिद्धान्तचन्द्रकाव्याख्याता रामानन्दयतिः कृष्णानन्दसरस्वतीकृतानुष्ठानपद्धतिव्याख्याता अप्पाध्वरी, रत्नतृलिकाकारः भास्करदीक्षितश्चास्य सतीर्थ्या इति निश्चीयते ।
अस्य सतीर्थ्येन भास्तरदीक्षितेन शाहजीप्रथमः तञ्जपुरमण्डलाधिराजः स्वपोषक इति रत्नतूलिकायां निर्दिश्यते । शाहजीप्रथमकालः (1684-1711 A. D.) इति इतिहासविदः । एवञ्च गुरोश्शिष्याणां सतीर्थ्यानाञ्च कालस्सप्तदशशतकमिति ज्ञायते ।
१. कोशरत्नप्रकाशः - (7502 TSML)
अमुद्रितोऽयं ग्रन्थः नृसिम्हाश्रमिकृतस्वीयतत्वविवेकव्याख्याभूतस्य तत्व विवेकदीपनापरनाम्न अद्वैतरत्नकोशाख्यग्रन्थस्य व्याख्यात्मकः सरस्वतीमहालये लभ्यते ।
२. प्रभामण्डलम् - (6932 TSML) शास्त्रदीपिकाव्याख्यात्मकोऽयं ग्रन्थः यज्ञनारायणविरचिताद्भिन्न अमुद्रितश्च सरस्वतीमहालये लभ्यते ।

९. अन्नम्भट्टः (1600-1700 A.D.)
आन्ध्रदेशजोऽयं कृष्णानन्दीतीरोपान्तग्रामवासी (चितूर) अद्वैतविद्याचार्यराघवसोमयाजिकुलोत्पन्नः तिरुमलाचार्यसूनुः कौशिकागोत्रजः, सिद्धान्तकौमुदीव्याख्यासिद्धान्तरत्नाकरकर्तू रामकृष्णभट्टस्य सर्वदेवस्य च कनीयान् भ्राता, व्याकरणाचार्यंशेषकृष्णपुत्रस्य शेषवीरेश्वरापराभिधस्य शेषविश्वेश्वरस्य शिष्यः, वेदान्ते ब्रह्मानन्दसरस्वत्याः शिष्यश्चेति एतत्कृतग्रन्थाज् ज्ञायते । अयं काश्यां शास्त्राण्यधीती उवासेति एतत्कृतविश्वेश्वरध्यानात् अवगम्यते । ``काशीगमनमात्रेण नान्नम्भट्टायते द्विज" इत्यन्नम्भट्टप्रशंसनपरा किंवदन्ती च प्रसिद्धा । अस्य शिष्यः वेदाद्रिसूरिरिति प्रसिद्धः वेदान्तपरिभाषाच्याख्यातत्वबोधिनीकारः । न्याये व्याकरणे च एतदीयाः ग्रन्था उपलभ्यन्ते ।
१. मिताक्षरा-
ब्रह्मसूत्रवृत्तिग्रन्थोऽयं भामतीकल्पतरुवैय्यासिकन्यायमालाप्रदर्शितवर्त्मना ब्रह्मसूत्रार्थं वर्णयन् वाक्यविन्यासेन कल्पतरुं न्यायमालामनुसरति । ग्रन्थश्चायं मद्रासराजकीयहस्तलिखितपुस्तकालयमालायां (MGOMLS 19) मुद्रितः ।
तर्कसङ्ग्रहः, तर्कसङ्ग्रहदीपिका, तर्कभाषा, तत्वबोधिनीटीका, जयदेवकृतत्वचिन्तामणिव्याख्या सिद्धाञ्जनम्, महाभाष्यप्रदीपव्याख्या राणकोज्जीविनी, तत्वचिन्तमणिदीधितिव्याख्या सुबुद्धिमनोहरा, तन्त्रवार्तिकव्याख्या सुबोधिनी, स्वरविवेकः, तत्वविवेकदीपनव्याख्याश्च कृता इति ज्ञायते ।

१०. अभिनवनारायणेन्द्रसरस्वती (1600-1700 A.D.)
अयमभिनवनारायणः ज्ञानेन्द्रसरस्वतीशिष्यः, कैवल्येन्द्रसरस्वतीप्रशिष्यः सदाशिवब्रह्मेन्द्रगुरोः परमशिवेन्द्रस्य गुरुः, अग्निहोत्रभट्टस्य सतीर्थ्यश्चेति ज्ञायते । प्रसिद्धनृसिम्हाश्रमिणां शिष्यस्य नारायणाश्रमिण अपरे वयसि सामयिकोऽसाविति कारणादेवास्य अभिनवनारायणत्वं सार्थकं भवति । अनेन स्वीये छान्दोग्यभाष्यव्याख्याने विद्याप्रकाशाख्यः ग्रन्थः बहुवारं प्रमाणीकृतः ।
%%% Chart
१. ऐतरेयोपनिषद्भाष्यटीका - (R. 1475 MGOML)
२. कठोपनिषद्भाष्यटीका - (XXI old h)
३. केनोपनिषद्भाष्यटीका - (XXI 26 Oudh)
४. छान्दोग्योपनिषद् भाष्यटीका (R. 1662 MGOML)
५. पञ्चीकरणभावप्रकाशिका (R. 1492 B. MGOML)
६. प्रश्नोपनिषद्भाष्यटीका (D. 621 MGOML)
७. मुण्डकोपनिषद्भाष्यटीका (XXI 26 Oudh)
८. वार्तिकाभरणम् (B. S. S.)
सुरेश्वराचार्यकृतपञ्चीकरणवार्तिकव्याख्यात्मकोऽयं ग्रन्थः वाराणस्यां मुद्रितः । अनेन पञ्चरत्नव्याख्या कल्पवल्लीनाम्नी कृतेति वदन्ति ।

११. अभिनवसदाशिवब्रह्मेन्द्रः (1800-1900 A.D.)
कर्माकर्मविवेकतत्वम्पदार्थल्क्ष्यैकशतकादिग्रन्थप्रणेतुर्वासुदेवेन्द्रशिष्यरामचन्द्रेन्द्रस्य शिष्योऽयं अभिनवसदाशिवः आत्मानं स्वीयाभिनवत्वविशेषणेनैव प्रसिद्धसदाशिवब्रह्मेन्द्रस्य पश्चाद्भवं दक्षिणदेशजमावेदयति ।
१. पञ्चीकरणम् (D. 4572 MGOML)

१२. अमरदासः (1800-1900 A.D.)
अनेन वेदान्तपरिभाषा-व्याख्यामणिप्रभायाः विषयपरिच्छेदे (Page 298) स्वस्य दीक्षागुरुः श्रीचन्द्र इति निर्दिश्यते । ब्रह्मविज्ञानोऽस्य विद्यागुरुः । नानकवादनात् गुरुसिक्खमतप्रवर्तकनानकसम्प्रदायानुगतोऽपि अद्वैतसम्प्रदाये महानादर अनेन प्रदर्श्यते । अमरदासोऽयं एकोनविंशतिशतकीय इति ग्रन्थेभ्यः निर्णीयते ।
१. वेदान्तपरिभाषा-शिखामणि-व्याख्या-मणिप्रभा ग्रन्थोऽयं वेङ्कटेश्वर स्ट्रीममुद्रणालये मुद्रितः ।
२. ईश-ऐतरेय-कठ-केन-तैत्तरीय-प्रश्न-माण्डूक्य-मुण्डकोपनिषदां व्याख्या मणिप्रभा अनेन कृता । अतएवायं मणिप्रभाकार इत्यपि प्रसिद्धः ।
उपनिषदां व्याख्या मणिप्रभा चौखाम्बामुद्रणालये मुद्रिता ।

१३. अमरानन्दः (1225-1300 A.D.)
कर्णाटकदेशाभिजनस्यापि काशीवासिनोऽस्यामरानन्दस्य पूर्वाश्रमे पितुर्नाम कुमारेश्वर इति ज्ञायते । कर्णाटकदेशशासितुः होयलावंशजस्य त्रयोदशशतकीयस्य नरसिम्हपुत्रसोमेश्वरस्य सामयिकोऽयममरानन्दः अमरानन्दप्रशिष्यस्य जगदाराध्येत्यपरनामकानुपमसुखशिष्यस्य विश्वनाथापराख्यनिरूपमबोधस्य शिष्य इति ग्रन्थेषु दर्शनात् अस्य परात्परगुरुः अमरानन्दप्रथमः, परमगुरुर्जगदाराध्येत्यपरनामा अनुपमसुखः, विश्वनाथापरनामा निरुपमबोधः गुरुरिति निश्चीयते ।
१. स्वात्मयोगप्रदीपः - (R. 3428 MGOML)
आत्माद्वैतप्रतिपादकोऽयं ग्रन्थः जीवेश्वरस्वरूपं वर्णयन् तत्त्वम्पदविचारं सपरिकरं प्रतिपादयति । गौडपादाचार्यं भट्टाचार्यञ्च प्रमाणयति । अस्य व्याख्यापि मूलकृतैव स्वात्मयोगप्रदीपप्रबोधिनी नाम्नी कृता अनुद्रिता च वर्तते ।
२.स्वात्मयोगप्रदीपव्याख्या - प्रबोधिनी (R. 3428 MGOML) विष्णुपुराणव्याख्या विष्णुवल्लभानाम्न्यपि अदसीयः ग्रन्थः ।

१४. अमृतानन्दमुनिः (1400 A.D.)
``अपरं दक्षिणामूर्तिं तमानन्दगिरिं भजे" इति आनन्दशैलाङ्घ्रिसरोजभृङ्गमाराद्भजे यादवशक्रशेलमिति आनन्दगिरियादवेन्द्रगिर्योर्नमस्कृतिश्रवणात् अमृतानन्दोऽयं आन्दगिर-यादवेन्द्रगिर्योश्शिष्य इति ज्ञायते । यादवेन्द्रगिरिस्तु आनन्दगिरेरपि शिष्य इति ज्ञायते ।
१. न्यायदीपावलीव्याख्या - न्यायविवेकः
आनन्दबोधाचार्यकृतन्यायदीपावलीव्याख्यात्मकोऽयं ग्रन्थः प्रथमानुमानपर्यन्तं मुद्रितश्चौखाम्बामुद्रणालये । अमुद्रितश्चापूर्णः ग्रन्थः सरस्वतीमहालये लभ्यते ।

१५. अल्लालसूरिः (1300-1400 A.D.)
कोटिकलाग्रामवासी नागमाम्बात्रिविक्रमाचार्ययोः पुत्र अनन्तार्यप्रज्ञानारण्ययोश्शिष्योऽयमल्लालसूरिस्स्वग्रन्थे कल्पतरुकारं अमलानन्दं व्यासाश्रमशब्देन निर्दिशन् चित्सुखाचार्यं प्रमाणयति । अनेन स्वग्रन्थारम्भे वाचस्पतिमिश्रः व्यासाश्रमः प्रज्ञानारण्यश्च सनामग्रहणं निर्दिष्टाः ।
``अक्षुण्णायामरण्यान्यां भामत्यां वर्त्म यो व्यधात् ।
सुगमं प्रणमामस्तं व्यासाश्रममुनीश्वरम् ।।"
इति व्यासाश्रमः निर्दिष्टः । व्यासाश्रम एवामलानन्द इति केचित् । केचित्तु तर्योर्मेदमामनन्ति । तेषां मतेन सुगम इति भामतीव्याख्या काचनासीद्या नाममात्रेण प्रसिद्धेदानीमिति वक्तव्यमापतति ।
यदि व्यासाश्रमामलानन्दावभिन्नौ (632 TMPL Mss.) तर्हि अल्लालकाल (1300 A. D.) कालादर्वाग्भवः । स्वग्रन्थे परिमलकारस्याप्पय्यदीक्षितास्यानुल्लेखात् अप्पय्यदीक्षितात्प्राचीन इति निश्चीयते । बरोडापुस्तकालयस्थे (13768 B. R. D.) भामतीतिलकादर्शपुस्तकेतु तस्य प्रतिलिपिकाल (1335 A. D.) इति दृश्यते । तस्मात् (1335 A. D.) कालात्प्राक्तन इत्यत्र तु न संशयः । यदि व्यासाश्रमामलानन्दौ भिन्नौ तर्हि अमलानन्दव्यासाश्रमात् सुगमकारः व्यासाश्रमभिन्न इति बरोडास्थप्रतिलिपिकालस्यायमर्थ आपतति यत् कल्पतरुकारादपि प्राचीन इति । केचित्तु अल्लालसूरिः (1600-1700 A. D.) काले आसीदिति वदन्ति । अन्ये तु (1756 A.D.) कालादर्वाग्भव इति वदन्ति । तेषां मतेन बरोडास्थ प्रतिलिपिकाल विरुध्यते ।
१. भामतीतिलकम् (R. 4190 MGOML)
भामतीव्याख्यात्मकोऽयं ग्रन्थ अमुद्रित अपूर्णश्च बहुत्रोपलभ्यते ।

१६. आत्मस्वरूपः (1500-1250 A.D.)
अप्रकाशिते एतत्कृते प्रबोधपरिशोधिन्याख्ये ग्रन्थे ``श्रीनृसिम्हस्वरूपस्य शिष्येणेयं मयेरिता" इति दर्शनात् नृसिम्हस्वरूपशिष्योऽयमात्मस्वरूपः अनुमूतिस्वरूपादिवत्स्वरूपान्तनामा द्वादशशतकादारभ्य त्रयोदशशतकपूर्वार्धावधिककालिकस्स्यादिति ज्ञायते । अत्रेमानि कारणानि -
आत्मस्वरूपरचितौ द्वौ ग्रन्थावुलभ्येते । प्रबोधपरिशोधिनीनामा पञ्चपादिका व्याख्यात्मकः कश्चन ग्रन्थः । द्वादशशतकीयेन आनन्दनुभवेन कृतस्य पदार्थतत्वनिर्णयस्य व्याख्यात्मक अपरो ग्रन्थः । तयोः प्रबोधपरिशोधिन्यां पञ्चपादिकाविवरणाचार्यः, आचार्यसुन्दरपाण्ड्यः, गौडाचार्यः प्रभाकराः भाट्टाश्च निर्दिष्टाः । 31-37 पत्रपर्यन्तमनिर्वचनीयख्यातिस्सम्यङिनरूपिता । न तत्र खण्डनादिग्रन्थो वा ग्रन्थकर्ता वोल्लेखार्हों नोल्लिखितः । पदार्थतत्वनिर्णयटीकायां भाट्टः किरणावली, उदयनः, न्यायसारः, भूषणं, ब्रह्मसिद्धिः, लीलावतीकारः, प्रमाणमाला, प्रशस्तपादभाष्यम्, कुसुमाञ्जलिः, इष्टसिद्धिकारश्चोद्धृताः । प्रमाणमालाकारः आनन्दबोध 11-12 शतकीयः । दशमशतकीय इष्टमिद्धिकारः । आनन्दगिरिश्चित्सुखो वा न र्निदिष्टः । तस्मात्तयोः पूर्वतनः आनन्दबोधादर्वाक्तन इति निश्चीयते । नान्यदत्र प्रमाणमुपलभ्यते ।
१. पञ्चपादिकाव्याख्या - प्रबोधपरिशोधिनी । (R. 3225 MGOML) अमुद्रित उपलभ्यते ।
२. पदार्थतत्वनिर्णयटीका - (R. 4219 MGOML) मध्यलुप्तोऽयं ग्रन्थ अमुद्रितः ।

१७. आदिवेङ्कटयोगी (1700-1800 A.D.)
अयमादिवेङ्कटयोगी भारद्वाजगोत्रजः गन्नेपुडीग्रामाभिजनः कोण्डयपुत्रः सुब्बय्यामात्यपुत्रः रामचन्द्रसरस्वतीप्रशिष्यः स्वयम्प्रकाशसरस्वत्याश्शिष्यः आन्ध्रदेशज इति ज्ञायते । ``अङ्गीकृता हि मत्कृतिरखिलज्ञैर्बालकृष्णयतिवर्यैः । शाहजियाग्रहारस्थितविद्वन्मुख्यसेवितपदाब्जैः ।" इति वदन्नयं ग्रन्थकृत् सिद्धान्तसिद्धाञ्जनकाराणां शाहजग्रामवासिनां बालकृष्णसरस्वतीनां सामयिक इति ज्ञायते ।
१. व्रह्मविन्निधिः - (29 G. 29 AL. R. 4362 MGOML)
अमुद्रितोऽयं पूर्णः ग्रन्थः त्रयस्त्रिंशद्भिः प्रकरणः परिमितः जगन्मिथ्यात्व, गुरुशिष्यलक्षणम्, महावाक्यार्थविचारम् , इन्द्रियजयोपायम् , ध्यानस्वरूपम् , जीवस्वरूपम् ,  सुषुप्तिमृत्योंर्मेदम्, मुक्तिस्वरूपम्, जीवन्मुक्तिविदेहमुक्तिनिरूपणम् , अद्वैतब्रह्मस्वरूपम् , ब्रह्माविचाराधिकारिनिरूपणञ्च कुर्वन् समग्राद्वैतवेदान्तप्रकरणग्रन्थ प्रतिपादितान् विषयान् साकल्येनैकत्र प्रतिपादयति । ग्रन्थोऽयं अडयार - मद्रासराजकीयहस्तलिखितपुस्तकालययोर्लभ्यते ।

१८.आनन्दबोधेन्द्रसरस्वती (1780-1850 A. D.)
सर्वज्ञसरस्वती - रामचन्द्रसरस्वत्योः प्रशिष्यः स्वाराज्यसिद्धिकारस्य गङ्गाधरेन्द्रसरस्वत्याश्शिष्य इति ज्ञायते । अनेन ऋतुरसतुरगमहीशक विकारि शुभवत्सरस्य शिशिरर्तोः इति स्वग्रन्थनिर्माणकाल (1766-श 1i842 A.D.) निर्दिश्यते ।
१. योगवासिष्ठव्याख्या-तात्पर्यप्रकाशः (N. S. P.) योगवासिष्ठव्याख्यत्मकोऽयं ग्रन्थः निर्णयसागरमुद्रणालये मुद्रितः ।

१९. आनन्दरायमखी (1684-1728 A.D.)
गङ्गाधराध्वरिणः पौत्रः, नृसिम्हरायपुत्रः त्र्यम्बकरायज्येष्ठभ्राता चायं आनन्दरायः शाहजीशरभोजीसचिवश्चेति सप्तदशाष्टादशशतकीयः इति ज्ञायते । अस्य कृतिर्वेदान्ततत्वप्रधाना विद्यापरिणयाख्या । मुद्रिताचेयं निर्णयसागरे ।

२०. आनन्दस्वरूपभट्टारकः (1300 A.D.)
अयमानन्दस्वरूपभट्टारक आनन्दात्मशिष्यः । आनन्दात्मा शङ्करानन्दस्यापि गुरुः । एवञ्च शङ्करानन्दसतीर्थ्योऽयं त्रयोदशशतकीय इति निश्चयः ।
१. वाक्यदीपिका (R. 3324 C. MGOML)
वेदोत्तमभट्टारकरचिताया बृहद्वाक्यवृत्याः व्याख्यारूपोऽयं ग्रन्थ अमुद्रितः मद्रपुरीपुस्तकालये लभ्यते ।

२१. आनन्दाश्रमः (1650-1750 A.D.)
सच्चिदानन्दचिद्धनानन्दाश्रमवरशिष्यपरमहंसपरिव्राजकाचार्यआनन्दाश्रमेत्यादिग्रन्थात् सच्चिदानन्दचिद्धनानन्दशिष्योऽयं आनन्दाश्रमः महाराष्ट्रदेशज इति ज्ञायते । अस्याचार्यस्य विश्वेश्वर इति नामान्तरमपि स्यादिति ग्रन्थतो ज्ञायते । एतद्विरचिते आनन्दरससागराख्ये ग्रन्थे शङ्कराचार्यः, मानसोल्लासः, दशश्लोकी, योगवासिष्ठञ्च प्रमाणत्वेन वर्णितानि ।
कालनिर्णये प्रबलतरप्रमाणं नोपलभ्यते । परन्त्वस्य शिष्येण कृतायां मध्वसिद्धान्तभञ्जिन्यां भट्टोजिदीक्षितः निर्दिष्टः । यद्ययं भट्टोजिः प्रसिद्धस्स्यात् तर्हि तस्य सप्तदशशतकीयत्वेन तदर्वाग्भवोऽयमानन्दाश्रम इति निर्णेतुं शक्यते ।
१. आनन्दरससागरः (R. 7543. R. 5749 MGOML)
त्रिपञ्चाशद्भिः प्रकरणैः पूर्णोऽयं प्रकरणग्रन्थः अखण्डार्थत्वं जगन्मिथ्यात्वं आत्मानात्मविचारञ्च कुर्वन् मद्रासराजकीयपुस्तकालये लभ्यते ।

२२. आपदेवः (1600-1700 A.D.)
दाक्षिणात्योऽयं आपदेवः महाराष्ट्रदेशीयः देवोपनामककुलप्रसूतः गोदावरीतीरग्रामाभिजनः एकनाथनप्ता आपदेवप्रथमस्य पौत्रः, सिद्धान्ततत्वकारस्य अनन्तदेवस्य पुत्रः स्मृतिकौस्तुभकारस्य अनन्तदेवस्य पितामहः स्वपितुरेवाधीतदर्शन इति ज्ञायते ।
केचित्तु मीमांसान्यायप्रकाशोपान्त्यश्लोके ``गोविन्दगुरुपादयो" रिति दर्शनात् भ्रान्ताः गोविन्दशिष्य आपदेव इति अनन्तदेवस्यैव गोविन्द इति नामान्तरमिति वा वदन्ति ।
१. वेदान्तसारव्याख्या - बालबोधिनी (V. V. S.)
ग्रन्थोऽयं सदानन्दकृतवेदान्तसारव्याख्यात्मकः । ग्रन्थस्यास्य तत्वदीपिका इत्यपि नामान्तरं दृश्यते । ग्रन्थोऽयं वाणिविलासमुद्रणालाये श्रीरङ्गनगरे मुद्रितः । मीमांसान्यायप्रकाशोऽपि अनेन कृतः ।

२३. ईश्वरतीर्थः (1098-1146 A.D.)
शृङ्गगिरिशङ्करपीठपरम्परागतोऽयमीश्वरतीर्थः नृसिम्हगिरिशिष्यः, नृसिम्हतीर्थगुरुश्चेति शृङ्गगिरिगुरुपरम्परासूच्याः ज्ञायते । भारतीयैतिहासिकत्रैमासिकपत्रिकायां (IHQ Vol. XIV) प्रतिपादितश्च ।
१. शतश्लोकी - (5539 C.C.P.B.) वैराग्यप्रकरणापराभिधानोऽयं ग्रन्थ अमुद्रितः मध्यप्रान्तीयबरार्ग्रन्थसूच्यां लभ्यते ।

२४. उत्तमश्लोकः (1250-1350 A.D.)
उत्तमश्लोकोऽयं शुद्धानन्दशिष्य इति ग्रन्थाज् ज्ञायते । आनन्दगिरिस्तु शुद्धानन्दस्यापि शिष्यः । यद्येवं तर्हि आनन्दगिरिसामयिकस्सतीर्थ्यश्चेति सिध्यति । श्रीकण्ठशास्त्री तु आनन्दगिरिकालं 1114-1228 A. D. इति (IHQ Vol. XIV) वदति । लक्ष्मीधरगुरोरनन्तान्दगिरिरिति नाम । लक्ष्मीधरस्य शिष्येषु शुद्धानन्दः कश्चन आसीत् । यस्य च शिष्यस्स्वयन्प्रकाश इति प्रतिपादयति । यद्येवं शुद्धानन्दशिप्यौ स्वयम्प्रकाशोत्तमश्लोकौ द्वावपि सामयिकौ सतीर्थ्यौ चेति (1400 A.D.) कालीनाविति सिध्यति । ग्रन्थेषु विश्वनाथस्थ विशालनयनानाथस्य च नमस्कृतिप्रकाशनात् वाराणसीवासीति ज्ञायते ।
१. लघुवार्तिकम् (B. S. S. 205)
वेदान्तसूत्रलघुवार्तिकापरनामायं ग्रन्थः आनुष्ठुभेण छन्दसा रचितः, कुत्र चिदेकेन पादेन एकैकस्याधिकरणस्य सङ्ग्राहकः मुद्रितश्च ।
२. न्यायसुधा (B. S. S. 205)
लघुवार्तिकव्याख्यात्मकोऽयं ग्नर्थः लघुवाक्यसुधा लघुन्यायसुधा इत्यपि व्यवह्नियते ।

२५. उत्तमज्ञयतिः (1100-1200 A.D.)
``यन्नामश्रवणाद्भीता वादिनो मोहिता भृशम् ।
तस्मै ज्ञानोत्तमाख्याय जगन्मोहभिदे नमः ।।"
इति तत्वशुद्धिव्याख्यायां दर्शनात् उत्तमज्ञस्यास्य गुरुर्ज्ञानोत्तम इति ज्ञायते । ज्ञानोत्तमोऽयं चोलदेशीयात् मङ्गलग्रामवासिनः ज्ञानोत्तमाद्भिन्नः न्यायसुधादिग्रन्थ प्रणेता विज्ञानात्मचित्सुखज्ञानगिरिगुरुरिति च परिशीलनाज् ज्ञायते । ज्ञानघन शिष्यस्य ज्ञानोत्तमस्य काल (1100-1200 A.D.) एवाञ्चास्यापि कालस्य एव ।
श्रीकण्ठशास्त्रिणां मतेन (IHQ Vol. XIV) ज्ञानघनः प्रकाशात्मनः सामयिक इति ज्ञानघनप्रशिष्यस्योत्तमज्ञस्य कालः (958-1038 A.D.) इति सिध्यति ।
%%% Chart
१. तत्वशुद्धिव्याख्या - (754 C, O. L. 291 TCD) ज्ञानघनविरचितस्य तत्वशुद्धिग्रन्थस्य व्याख्यात्मकोऽयं पूर्णग्रन्थः तिरुवन्तपुरपुस्तकाये लभ्यते ।
२. पञ्चपादिकाव्याख्या - वक्तव्यप्रकाशिका (56. A. Sringeri Math) अमुद्रितोऽयं ग्रन्थः शृङ्गगिरिमठपुस्तकालये लभ्यते ।

२६. उपेन्द्रदत्तः (1859-1900 A.D.)
वाराणसीक्षेत्रवास्ययं भास्करान्दापराभिधानस्यानन्तराममिश्रस्य शिष्यश्चन्द्रमणिपाण्डेयपुत्र इति ज्ञायते ।
१. पञ्चीकरणवार्तिकपाठः । ग्रन्थोऽयं वाराणसीसरस्वतीभवनग्रन्थालये मुद्रितः ।

२७. उमामहेश्वरः (1550-1650 A.D.)
``गुरून् प्रज्ञाजितगुरून्नमाम्यक्कय्यशास्त्रिण" इति ``उमामहेश्वराख्येन वेल्लालकुलजन्मना" इति तत्वचन्द्रिकायां दर्शनात् अक्कय्येत्यपरनाम्न अक्षयसूरेः शिष्य विल्लालकुलप्रसूतः वेङ्कटरायपुत्रः सभारञ्जनकारकविकुञ्जरगुरुः तप्तमुद्राविद्रावणकारस्य भास्करदीक्षितस्य पिता, चोलदेशीयः मोक्षकुण्ड (मुडिकोण्डान्) ग्रामवासी अभिनवकालिदासापरनामा चेति ज्ञायते । एतेन तत्वचन्द्रिकाया भूमिकायां मध्वविध्वंसन-मध्वन्यक्काराभ्यां मध्वमतस्य खण्डितत्वेन श्रीकण्ठमतरामानुजमत एवात्र खण्ड्येते इति प्रतिज्ञातम् । एवं रत्नतूलिकातप्तमुद्राविद्रावणादिकर्त्रा उमामहेश्वरपुत्रेण भास्करदीक्षितेन आत्मनः नृसिम्हाश्रमि कृष्णानन्दसरस्वत्योश्शिष्यत्वं प्रतिपादितम् । एवञ्चाप्पय्यदीक्षितात् उमामहेश्वरः किञ्चिदर्वाचीनो अन्तिमसामयिको वा भवितुमर्हति । T. R. चिन्तामणिमहाशयस्तु स्वसम्पादिते साहित्यरत्नाकरकाव्यभुमिकायां भास्करदीक्षितं रघुनाथसामयिकं वर्णयति । रघुनाथनायकस्य शासनकाल (1600-1650 A.D.) इति P. P. शास्त्रिणः । एवञ्च भास्करदीक्षितपिता उमामहेश्वरः (1550-1650 A. D.) कालवासीति निर्णेतुं शक्यते ।
१. अद्वैतकामधेनुः (7526 T. S. M. L.) सूत्रवृत्तिरूपोऽयं ग्रन्थः परिच्छेदद्वयात्मकस्तेलुगुलिप्यां मुद्रितः ।
२. तत्वचन्द्रिका (R. 5156 MGOML)
निर्गुणब्रह्ममीमांसापरनामायं ग्रन्थः द्वादशभिरुल्लासैः पूर्णः श्रीकण्ठमतं रामानुजमतञ्च खण्डयति । अमुद्रितोऽयं ग्रन्थः अडयारजपुरपोटीखानापुस्तकालयेषु लभ्यते ।
३. विरोधवरूथिनी (R. 4750 MGOML)
सप्तविंशतिभिः परिच्छेदैः पूर्णोऽयं ग्रन्थः रामानुजमतं खण्डयति । ग्रन्थोऽयममुद्रितः अडयार पुस्तकालये च लभ्यते ।।
४. वेदान्तसिद्धान्तसारः (R. 1403 MGOML) । पाणिनीयवादनक्षत्रमाला, भागवतचम्पूश्चानेन कृतौ ।।

२८. उवटाचार्यः (1010-1062 A.D.)
आनन्दपुरवास्तव्यवज्रटाख्यस्य सूनुना ।
उवटेन कृतं भाष्यं पदवाक्यैस्सुनिश्चितैः ।।
ऋष्यार्दीश्च नमस्कृत्य अवन्त्यामुवटेवसन् ।
मन्त्राणां कृतवान् भाष्यं महीं भोजे प्रशासति ।
इति निर्णयसागरमुद्रितायां शुक्लयजुर्वेदसंहितायां दर्शनात् उवटाचार्योऽयं वज्रटपुत्रः आनन्दपुरवासी भोजगजसामयिक इति निर्णीयते । भोजराजश्च यदि धारानगराधीशः सरस्वतीकण्ठाभरणशृङ्गारप्रकाशकारस्स्यात्तर्हि तस्य शासनकाल (1010-1062 A.D.) इति (यपिय्राफिका इण्डिकाया I page 230) प्रामाण्यात् शृङ्गारप्रकाशप्रामाण्याच्चाध्यवसीयते ।।
१. ईशावास्योपनिषद्विवरणम् (A.S.S. 5) । नवमशतकीयोऽयमिति वृद्धत्रय्यां गुरुपादहालदारः ।।

२९.
एकोजीराजः (1700-1750 A.D.)
चोलदेशीयतञ्जपुरराज्यशासकः प्रसिद्धशिवाजीमहाराजपौत्रः तुकोजीकुमाराम्बयोः पुत्र अष्टादशशतकीयः शिवजीराजस्य त्रयः पुत्रा आसन् । शाहजी शरभोजीतुकोजीनामानः । शाहजी राज्यशासनकालः (1684-1717 A.D.) शरभोजीशासनकाल (1712-1728 A.D.) तुकोजीशासनकाल (1728-1735 A.D.) एकोजीशासनकाल (1735-1736 A.D.) इति ज्ञायते । एकोजीराजस्यास्य गुरुर्महादेवपण्डिताख्य राजसभापाण्डितः ।।
१. परब्रह्मतत्वनिरूपणम् (7655 TSML)
शिवपार्वतीसंपादपद्धत्यां रचितोऽयं ग्रन्थः सर्वकारणकारणं व्यापकं ब्रह्मस्वरूपं साधयति । शिवविष्ण्वभेदप्रतिप्रादकोऽयं ग्रन्थः महादेवपण्डितकृतप्रपञ्चसारान्तर्गतः । आदर्शपुस्तकेषु त्रिषु 7655 संख्याके ग्रन्थे महादेवविरचितप्रपञ्चसाराख्यराजरञ्जनपुराण इत्येव दृश्यते । एतत्परिशीलनेन एकोजीकृतत्वेन निर्दिष्टास्सर्वेग्रन्थाः महादेवपण्डितकृता राजरञ्जनाय राजकृता इति निर्दिष्टास्स्युरिति संशय उदेति ।
२. रामानुजमतखण्डनम् - (7659 TSML) प्रपञ्चसारान्तर्गतोऽयं ग्रन्थः कृष्णनारदसंवादशैल्या रामानुजमतदोषान् प्रतिपादयति ।
३. मध्वमतकथनम् - (7960 TSML) मध्वमतस्य दुष्टाचारतां विशदयन्नयं ग्रन्थः प्रपञ्चसारान्तर्गत एव ।

३०. कामाक्षी (1850-1920 A.D.)
भारतीयप्राक्तनविद्यायाः स्त्रीषु विरलताया हेतौ अस्मिन् आधुनिके काले कमलमिव प्रादुर्भूता इयं कामाक्षी कावेरीनदीतीरवर्तिनि मायूरक्षेत्रे चोलदेशे उवास । आन्ध्रदेशजेयं विप्रकुलोत्पन्ना अस्याः पितामहेन प्रपितामहेन वा दुभाषी रामस्वामिशर्मणा नवीन एक अग्रहार निर्मातः व्राह्मणेभ्यश्च दत्तः । इदानीमपि स उग्रहार ``दुभाषग्रहारः" इत्येव प्रसिद्ध आस्ते । अस्याः पिता रामस्वाम्यार्यः । भ्राता सुब्रह्मण्यार्यः ।
कामक्ष्याः भर्ता रामलिङ्गार्यः । स च B. A. उपाधिधारी पाश्चात्यभाषाप्रवीणः । स चाकालिकदैवगत्या (1871 A.D.) काले पञ्चत्वं गतः । भर्तुस्स्वः र्गारोहणकाले कामाक्ष्या वय एकोनर्विशतिः । तदा प्रभृति मातुस्समीपस्था कामाक्षी न्यायशास्त्राण्यद्वैतग्रन्थांश्च सम्यगधीतवती । अपरे वयसि स्वभ्रात्रधीना आसीत् । (1871 A.D.) काले मृतभर्तृकायाः अस्या वय एकोनविंशतिरित्युक्ते अस्य जननकाल (1852 A. D.) इति भवति ।
१. अद्वैतदीपिका - मधुसूदनसरस्वतीकृतां अद्वैतसिद्धिं तत्प्रतिपादितान् मिथ्यात्वपरिष्करांश्च सङ्गृह्णात्ययं ग्रन्थः । ग्रन्थोऽयं श्रीविद्यामुदणालये मुद्रितः ।
२. श्रुतिरत्नप्रकाशटिप्पणी - त्र्यम्बकभट्टकृतस्य श्रुतिरत्नप्रकाशस्य व्याख्या । मुद्रितोऽयं श्रीविद्यामुद्रणालये ।
३. श्रुतिमतोद्योतटिप्पणी - त्र्यम्बकभट्टकृतश्रुतिमतोद्योतव्याख्यात्मकोऽयं ग्रन्थश्रीविद्याविलासमुद्रणालये कुम्भधोणे मुद्रितः ।
न्यायबोधिनीनीलकण्ठीयविषयाणां सङ्ग्राहकोऽयमन्यः विषयमालाख्यश्च ग्रन्थस्समारचितः ।।

३१. कालहस्तीशयज्वा (1550-1620 A.D.)
``प्रणम्य दक्षिणामूर्ति रघुनाथाश्रमान् गुरून्" इति वदन्नयं कालहस्तीशयज्वा रघुनाथाश्रमशिष्य इति ज्ञायते । अस्य प्रपितामहस्सोमनाथः । पितामहः यौवनभारती । पिता मल्लिकार्जुनः । अस्य पत्नी यज्ञाम्बानाम्नी । अस्य द्वौ पुत्रौ-रङ्गनाथः बालकविश्च । बालकविना रामवर्मविलासाख्यः ग्रन्तः (R. 3673 MGOML) प्रणीतः । रङ्गनाथस्तु आश्रमस्वीकारादनन्तरं अखण्डानन्दसरस्वतीति प्रसिद्धः स्वयम्प्रकाशशिष्य ऋजुपकाशिकाकारः । इति श्रीकण्ठशास्त्री (HIQ. Vol. XIV) प्रतिपादयति । नलगन्तुवंशजोऽयं कालहस्तीशः काञ्चीपुरवासी कामाक्षीभक्तः कामाक्षीदास इति प्रसिद्धश्च । प्रसिद्धाप्पय्यदीक्षितादनेन शिवदीक्षा स्वीकृता च ।
%%% Chart
१. भावप्रकाशिका (9. 1. 13. AL)
नृसिंहाश्रमीयाद्वैतरत्नकोशापरनामकस्य तत्वविवेकव्याख्यातत्वदीपनस्य व्याख्यात्मकोऽयं ग्रन्थ अडयार पुस्तकालये लभ्यते ।।
२. भेदधिक्कारविवृतिः (R. 2187 MGOML)
नृसिंहाश्रमीयभेदधिक्कारव्याख्यात्मकोऽयं ग्रन्थ अमुद्रितः मद्रपुरीराजकीयहस्तलिखित ग्रन्थालये लभ्यते ।। वसुचरितचम्पूरप्यदसीया कृतिः ।।

३२. काशीनायशास्त्री (1800 A.D.)
अनेन वेदान्तपरिभाषाख्यः कश्चन प्रकरणग्रन्थः धर्मराजाध्वरिकृतवेदान्तपरिभाषासारगर्भित रचितः । ग्रन्थोऽयं दासगुप्तमहाशयेन (HIP Vol. II Page 54) निर्दिष्टः ।

३३. कुमारभवस्वामी (1300 A. D.)
अयं कुमारभवस्वामी मध्वध्वंसन-अद्वैतकौस्तुभ-भावनापुरुषोत्तमनाटकादिकर्तू रत्नखेटश्रीनिवासदीक्षितस्य पञ्चमः कुलपूरुषः । एतद्विरचितः ग्रन्थः अद्वैतचिन्तामणिः । ग्रन्थोऽयं रुक्मिणीकल्याणव्याख्यायां बालयज्ञदेवेश्वरकृतायां निर्दिष्टः । अम्बास्तवव्याख्यायाञ्च निर्दिश्यते । एवं मद्रासविश्वविद्यालयप्रकाशिते NCC ग्रन्थेऽपि निर्दिष्टः ।

३४. कृष्णः (1800 A.D.)
ब्रह्मानन्दापराभिधानकृष्णानन्दयतेश्शिष्योऽयं कृष्णपण्डितः दक्षिणदेशीय स्तमिलभाषायास्सहित्ये च निष्णातश्चोलदेशवासीति च ज्ञायते । एतदीयः ग्रन्थः कैवल्यदीपिकाख्याः
 । स च तमिलभाषायां ताण्डवरायस्वामिना विरचितस्य कैवल्यनवनीताख्यस्य संस्कृतीकरणरूपः पद्यबद्धः ग्रन्थः ।
ताण्डवरायश्चाष्टादशशतकीयश्चोलदेशीयः नन्निलग्रामवासी नारायणसरस्वतीशिष्यः । केचित्तु ताण्डवरायस्वामिनं तिरुवानैक्कावल-गोपालशास्त्रिणः प्राप्तविद्या, नारायणगुरुोः प्राप्तदीक्षं, परमशिवेन्द्रशिष्यं सदाशिवब्रह्मेन्द्रसतीर्थ्यञ्च वदन्ति । सर्वथापि ताण्डवरायस्वामिन अर्वाचीनोऽयं कैवल्यदीपिकाकारः कृष्णपण्डितः अष्टादशशतकीय इति निर्णयः । अज्ञातनामधेयेनास्य शिष्येण वाक्यसुधायाः व्याख्या कृता ।
१. कैवल्यदीपिका सव्याख्या - (10. E. 18. AL)
भागद्वयपरिमितोऽयं ग्रन्थः । प्रथमे तत्वप्रदीपिकाक्ये अष्टोत्तरशतैः पद्यैः परिमिते अद्वैतमततत्वानि प्रतिपादितानि । द्वितीये संशयच्छेदाख्ये षट्सप्तत्यधिकशतैः पद्यैः औपनिषदप्रमाणपूर्वकं अद्वैतवस्तुन्यारोपितास्संशयाश्छिद्यन्ते । अस्यव्याख्यापि प्रभाख्या मूलकृतैव कृता । अमुद्रितोऽयमडयारपुस्तकालये प्राप्यते । 
२. प्रत्यक्त्वस्वप्रकाशवादः - (23. F. 44 ग्र 22. AL)
३. भगवद्गीताभावप्रकाशः - (1300 CCPB ?)
४. अद्वैतसुधाबिन्दुः - (7999 BRD)
५. छान्दोग्योपनिषत्कारिकाः-
ग्रन्थोऽयं जयपुरपोटीखाना सूच्यां दृश्यते । परन्तु कैवल्यदीपिकाया ऋते अन्ये ग्रन्थाः किमनेनैव कृता उतान्येनति निर्णेतुं न पार्यन्ते ।

३५. कृष्णकान्तविद्यावागीशः (1800-1900 A.D.)
रामकृष्णभट्टाचार्यपौत्रेण कालीचरणन्यायालङ्कारतारिणीदेव्योः पुत्रेण रामनारायणतर्कवागीशशिष्येण नव्यनैय्यायिकेन अनेन कृता दशश्लोकीसिद्धान्तबिन्दुन्यायरत्नावल्याः व्याख्या प्रदीपिकाख्या कृतेति विमर्शका वदन्ति । परन्तु स ग्रन्थः न्यायग्रन्थ इति भाति । राजेन्द्रलालसूच्यामेव दृश्यतेऽयं ग्रन्थः । नास्यविवरणमुपलभ्यते ।

३६. कृष्णगिरिः (1800-1900 A.D.)
ग्रन्थान्तपुष्पिकाया उय कृष्णगिरिः कैलासाचालशिष्य इति ज्ञायते । वाराणसीवास्ययं रणोद्दीपाख्यनृपसामयिकस्तत्पोषितश्च । बाणेन्दुरन्ध्रेन्दुमिते वत्सरे अनेन स्वग्रन्थः कृतः । एवञ्च ग्रन्थनिर्माणकालः (1915 सं. 1858 A. D.) इति ज्ञायते ।
१. मोक्षसिद्धिः ।
ग्रन्थोऽयं मुन्नालालमुद्रणालये वाराणस्यां मुद्रितः । गद्यमयोऽयं ग्रन्थः शङ्कराचार्यं प्रमाणयन् कर्मोपासनाज्ञानानां क्रमेणानुष्ठानात् अद्वयात्मकब्रह्मज्ञानावाप्तिरिति प्रतिपादयन् प्रकरणग्रन्थतामर्हति ।

३७. कृष्णनाथन्यायपञ्चाननः (1892 A. D.)
अर्जुनमिश्रवंशजः केशवचन्द्रपुत्रश्चायं नवद्वीपवासी मीमांसार्थसंग्रह-सांख्यतत्त्वकौमुदी - अभिज्ञानशाकुन्तलादीनां व्याख्याता एकोनविंशतिशतकीय इति ज्ञायते । अनेन कृता वेदान्तपरिभाषाव्याख्या आशुबोधिन्याख्या कल्कत्तानगरे मुद्रिता ।

३८. कृष्णमिश्रः (1060-1100 A. D.)
आध्यात्मिकनाटककर्ता राजस्थानान्तर्गत ``जाजभुक्ति" वासी चायं कृष्णमिश्रः स्वग्रन्थे कीर्तिवर्माणं प्रस्तुवन् तत्सामयिकत्वं प्रतिपादयति । कीर्तिवर्मा च (1060-1100 A. D.) काले शशासेति ग्रन्थस्थभूमिकाया ज्ञायते । संस्कृतसाहित्ये आध्यात्मिकवस्तूनि नाटकवस्तूकृत्य नाटकप्रणयनजं श्रेयः प्रथमत अस्यैवेति ज्ञायते ।
१. प्रबोधचन्द्रोदयम् ।
ग्रन्थोऽयं निर्णयसागरमुद्रणालये मुद्रितः । ग्रन्थस्यास्य बहूनि व्याख्यानानि विद्यान्ते यानि चान्यत्र प्रतिपादितानि ।

३९. कृष्णलीलाशुकः (1168-1293 A. D.)
अयमेव बिल्वमङ्गलाचार्य इति प्रसिद्धः । यद्येवं पितास्य दामोदरः । माता नीली । ईशानदेवशिष्योऽयं कृष्णलीलाशुकः केरलदेशवसी गोश्रीसाम्राज्यान्तर्गते तिरुच्चूर्समीपवर्तिनि ``तेक्कमठे" लब्धवासः (1220-1300 A. D.) काल उवासेति प्राच्यभाषामहासभ्मेलनस्य नवमाघिवेशपत्रिकायां (प्रसीडिङ्स आफनैन्त ओरियण्टल कान्फ्रेंस) दृश्यते ।
यद्ययं सरस्वतीकण्ठाभरणव्याख्यापुरुषकार-कृष्णलीलामृतादिकर्ता कृष्णलीलाशुकस्स्यात्तर्हि कालोऽस्य द्वादशत्रयोदशशतकोत्तरार्घपूर्वार्धावधिक इति निर्णेतुं शक्यते । यतः पुरुषकारे हेमचन्द्रः निर्दिष्टः । हेमचन्द्रश्च (1166-1220 A. D.) काल आसीत् । पुरुषकारस्य पंक्तयश्च देवराजकृतायां निघण्डुटीकायामुद्धृताः । तस्मात्तर्योर्मध्यपातीति निश्चीयते । देवराजकालश्च (1293-1343 A. D.) सीताराम जयरामजोशी तु कृष्णलीलाशुकं एकादशशतकीयं (1100 A. D.) स्वीये संक्षिप्तसंस्कृतसाहित्येतिहासे प्रतिपादयति ।
श्रीचिह्नकाव्यनामा कश्चन ग्रन्थः कृष्णलीलाशुकेन कृतः । तत्रायं श्लोकः - ``श्रीपद्मपादमुनिवर्यविनेयवर्गश्रीभूषणां मुनिरसौ कविसार्वभौमः ।" इति । तेक्कमठस्य स्थापना पद्भपादाचार्येण कृता स्यात् । एवमेव पद्मपादशिष्योऽयं कृष्णलीलाशुक इति वक्तुं शक्यते । केनोपनिषद्वयाख्यायाञ्च दृश्यमाणा Page 10 ``ब्रह्मभूयं गते पूर्वे शङ्करे कृत्स्नवेदिनि पूर्वे च तादृशे" इत्यादिश्लोकीया ``पूर्वे च तादृश्ये" इति पंक्तिरपि पद्मपादाचार्यं निर्दिशति । एवं च वेदान्ते पद्भपादाचार्यशिष्योऽयं कृष्णलीलाशुकः नवमशतकीय इति वर्णितम् । कृष्णलीलाशुकं केचित् वङ्गदेशजं वदन्ति ।
१. केनोपनिषद्व्याख्या - शङ्करहृदयङ्गमा (MGOMLS)
कृष्णकर्णामृतम्, कृष्णलीलाचरितम्, कृष्णलीलाकौमुदी, गोविन्दस्तोत्रम्, गौविन्दैकविंशतिः, बालकृष्णक्रीडाकाव्यम्, पुरुषकारः, बिल्वमङ्गलास्तोत्रम्, श्रीचिहकाव्यञ्च अनेेन रचितानि ।।

४०. कृष्णशास्त्री (1650-1700)
भावाज्ञानप्रकाशिकाकर्तू रङ्गनाथसूरेश्शिष्यस्य शिवरामस्य पिता अयं कृष्णशास्त्रीति ज्ञायते । अनेनाद्वैतविद्याविजयाख्याः ग्रन्थः कृत इति ज्ञायते । स च ग्रन्थः नोपलभ्यत्ते । परन्तु तत्पुत्रकृतायां अमुद्रितायां भावाज्ञानप्रकाशिकायां निर्दिष्टः ।।
१. अद्वैतविद्याविजयः Q.

४१. कृष्णशास्त्री (1870-1939 A.D.)
महामहोपाध्याय इति राजकीयबिरुदभूषितोऽयं कृष्णशास्त्री करुङ्गुलं कृष्णशास्त्रीति प्रसिद्धः । श्रीशालिपुर (तिन्नेवली) समीपस्थे कृष्णतटाकाख्ये (करुङ्गुलम्) ग्रामे लब्धजन्मान एते हरिहरशास्त्रिणां शिष्याः । अद्वैतसभा पण्डितोऽयं मद्रपुरीसंस्कृतकलाशालाप्रधानाध्यापकपदवी बहुवत्सरपर्यन्तं अलञ्चकार । अपरे वयसि प्राप्तसन्यास विदेहमुक्तो जातः ।।
१. अधिकरणचतुष्टयी ।
आनन्दमयाधिकरण-यथाश्रयभावाधिकरण - ऐहिकाधिकरणलिङ्गभूतस्त्वाधिकरणानां विषयविचारप्रधानोऽयं ग्रन्थः बालमनोरमामुद्रणालये मद्रासनगरे मुद्रितः ।
२. ब्रह्मसूत्रनुगुण्यसिद्धिः । ग्रन्थोऽयं गोपालविलासमुद्रणालये कुम्भधोणे मुद्रितः ।
३. परिमलः (A. M. S. S. 25)
गङ्गाधरेन्द्रसरस्वतीकृतस्वाराज्यसिद्धिव्याख्यात्मकोऽयं ग्रन्थः आर्यमत संवर्द्धिनीग्रन्थमालायां मुद्रितः ।

४२. कृष्णानन्दभारती (1400 A. D.)
``श्रीगुरुं भारतीतीर्थं विद्यारण्यमुनीश्वरम्" इत्यादिना भारतीतीर्थविद्यारण्यौ नमन्नयं कृष्णानन्दभारती भारतीतीर्थविद्यारण्यशिष्यः शृङ्गगिरिमठपरम्परागतः दक्षिणदेशीयश्चतुर्दशशतकीय इति सिध्यति ।
१. महावाक्यार्थदर्पणम् । (54. A. 41. A.L.)
गुरुशिष्यप्रणल्यां औपनिषदमहावाक्यानां अद्वैतब्रह्मावबोधकत्वं प्रदर्शयन्नयं प्रकरणग्रन्थः पूर्ण अमुद्रितश्च अडयारपुस्तकालये प्राप्यते ।

४३. कृष्णानन्दसरस्वती (1825-1900 A. D.)
``नमोऽस्तु गुरवे तस्मै यत्र तत्र निवासिने । सच्चिदान्दपादाय चरणाय मुहुर्मम ।" इति ब्रह्मसूत्रकुतूहले सच्चिदानन्दाश्रमिणं, ``यस्योपदेशमाहात्म्यात् जडा अपि विनिर्गताः । संसारबन्धात् तं वन्दे वासुदेवेन्द्रयोगिनम् ।।" इति वासुदेवेेन्द्रञ्च नमस्यन्नयं बालकृष्णानन्दापरनामा कृष्णानन्दसरस्वती विद्यायां सच्चिदानन्दाश्रमशिष्यः, आश्रमस्वीकारे वादेवेन्द्रशिष्यश्चेति ज्ञायते । वाराणसीवास्ययं स्वनिर्मितेषु ग्रन्थेषु शङ्कराचार्यं विशेषतः प्रणमति ।।
अनेन रामदुर्गाख्यविप्रप्रेरणया वेदेन्दुवसुभूमिते शालिवाहनशके 1814-1890 A. D. शास्त्राकूतप्रकाशः ब्रह्मसूत्रकुतूहलञ्च रचिते इति ज्ञायते । एवञ्चास्य काल एकोनविशतितमशतकमिति निश्चयः । अस्य शिष्यः हरिकृष्णशर्माख्याः ।
१. अद्वैतपञ्चरत्नव्याख्या-किरणावली (R. 1613 (b) MGOML)
२. ब्रह्मसूत्रकुतूहलम् ।
ग्रन्थस्यास्यावतरणिकायां अद्वैतसिद्धान्ताः विशदीकृताः । अथातो ब्रह्मजिज्ञासा त आरभ्य ज्योतिश्चरणाभिधानात् - इत्यन्तानां चतुर्विशतिसूत्राणां काचित्स्वतन्त्रा वृत्तिरद्वैतमतपोषिण्यारचिता । ग्रन्थोऽयं राजराजेश्वरीमुद्रणालये काश्यां मुद्रितः ।।
३. शास्त्राकूतप्रकाशः 
ग्रन्थोऽयं द्वैतिवादनिरसनपूर्वकं अद्वैतवादं राद्धान्तयति । ग्रन्थेऽस्मिन् दृश्यप्रपञ्चस्य मिथ्यात्वमुपपाद्य एकजीववादानेकजीववादयोर्निरवकाशत्वमुपवर्ण्य सजातीयविजातीयस्वगतभेदशून्यत्वमेवात्मत्वमिति सिद्धान्तितम् । आह्निकत्रयपरिपरिमितोऽयं लघुग्रन्थः जगदीश्वरप्रेस बम्बई नगरे मुद्रितः ।।
४. तिमिरोद्धाटनम् ।
लघुग्रन्थेऽस्मिन् अद्वैतात्मस्वरूपवर्णनेन साकं अवतारपदार्थः मुक्तिशब्दार्थविचारश्च प्रसङ्गवशादुपवर्णिताविति विशेषः । ग्रन्थोऽयं नषनलप्रेस राजकोट नगरे मुद्रितः ।।
५. भगवद्गीतैकदेशपरामर्शः ।
गीताया भेदवादे तात्पर्यं निरस्याद्वैतब्रह्मवादे तात्पर्यं साधितम् । मुद्रितश्चया गर्वर्नमण्टप्रेस गोण्टारपुरनगरे ।।

४४. कृष्णानन्दसरस्वती (1900 A. D.)
कैवल्यानन्दकृष्णानन्दयोश्शिष्योऽयं वाराणसीवासी एकोनविंशतिशतकीय इति ज्ञायते । १. अद्वैतसाम्राज्यम् N.S.P. 2. अज्ञानतिमिरदीपकः IOL 3. कैवल्यगाथा Kalpathi Press Bombay ४. गीतासारोद्धारः ; ५. खानुमूतिप्रकाशः (विलासः) (9976 BRd.) ६. ब्रह्मगीताव्याख्या - ``चित्प्रकाशिनी" (136 नासिकसूच्यां) ७. अध्यात्मभागवतव्याख्या ``चित्प्रकाशिनी" । नासिकूसूच्यां (108) अदसीयाः ग्रन्था दृश्यन्ते ।।

४५. कृष्णानुभूतियतिः (1500-1600 A. D.)
आनन्दानुभूतिं नमस्कुर्वाणोऽयं विबुधेन्द्रापरनामा आनन्दानुभूतिशिष्यः । केरलदेशजोऽयं स्वसामयिकौ केरलशासकौ राजराजरविवर्मनामानौ शारीरकमीमांसासूत्रसङ्ग्रहे निर्दिशति । केरलीयः वासुदेवकविनामा प्रसिद्धः रविवर्मणस्सभायामासीद्यस्य च कालः पञ्चदशशतकमध्यभाग इति कृष्णाचार्यसंस्कृतसाहित्यचरिते 252 पुटे दृश्यते । यदि कृष्णानुभूतिनिर्दिष्ट एवायं रविवर्मास्यात्तर्हि अस्यापि काल (1550 A. D.) इति वक्तुं शक्नुमः । मद्रासराजकीयहस्तलिखितपुस्तकालयस्थ शारीरकशास्त्रसंग्रहे विद्यमानमिदं - ``गीर्वाणेन्द्रसरस्वत्याः पादाब्ज हृदि विभ्रतः" इति पद्यञ्च नृसिम्हाश्रमिगुरुं गीर्वाणेन्द्रं स्मारयतीव ।
१. ब्रह्मसूत्राधिकरणन्यायानुक्रमणिका (R. 3305 B. MGOML)
२. शारीरकमीमांसाशास्त्रसंग्रहः (R. 2905 MGOML)
ब्रह्मसूत्रवृत्तिरूपोऽयं जीवब्रह्माभेदप्रतिपादकः ग्रन्थः अडयारपुस्तकालये अनन्तशयनपुस्तकालये विश्वभारतीशान्तिनिकेतनपुस्तकालये चामुद्रित उपलभ्यते ।
३. अधिकरणसंख्याश्लोकाः । (C. S. S. 1)

४६. केशवशास्त्री (1800-1850 A. D.)
स्वग्रन्थे आत्मानमानन्दाश्रमप्रशिष्यं लक्ष्मणपन्तशर्मणः राजारामशास्त्रिवाल शास्त्रिणोश्च शिष्यं प्रतिपादयन्नयं केशवशास्त्री एकोनविंशतिशतकीयः । तत्र लक्ष्मणपन्तशर्म आनन्दाश्रमशिष्यश्च ।
१. आत्मसोपानम् । ग्रन्थोऽयं 479 आनुष्ठुभैः पद्यैः घटितः प्रकरणग्रन्थः । गुरुशिष्यप्रणाल्या प्रवृत्तोऽयं ग्रन्थः आत्मनः नित्यशुद्धबुद्धमुक्तत्वं, ज्ञेयविषयाणां मिथ्यात्वं जीवन्मुक्तत्वोपपत्तिञ्च प्रतिपादयति । मुद्रितश्चायं वाराणसी पण्डितसंस्कृतग्रन्थमालायाम् (Vol. IV Pandit Series Vol. IV)

४७. कैवल्येन्द्रः (1550-1650 A. D.)
अनेन वेदान्तभूषणनामा ग्रन्थः कृतः । ग्रन्थोऽयं एतच्छिष्येण विद्येन्द्रसरस्वत्या स्वकृते वेदान्ततत्वसाराख्ये सरस्वतीमहालयस्थे (7575 TSML) ग्रन्थे निर्दिष्टः । (368 DCAL Vol. VI) नीलकण्ठवाजपेयीयायां सुखबोधिनीव्याख्या याञ्च निर्दिष्टः ।

४८. गणपतिशास्त्री (1850-1920 A. D.)
चोलदेशीयमन्नार्गुडिसमीपस्थपाङूगानाङुग्रामवासी राजुशास्त्र्यपरनामकत्यागराजशास्त्रिणः, पषवानेरीस्वामिनश्च प्राप्तन्यायवेन्दान्तव्याकरणशास्त्रः आशुकविर्गणपतिशास्त्री स्वीयसप्तदशमे वयस्येव कटाक्षशतककविरिति प्रसिद्धः । कोनेरिराजपुरवासिना साम्बशिवार्येण अद्वैतग्रन्थप्रकाशनाख्यं मुख्यं महत् प्रयोजनं मनसि कृत्वा कुम्भघोणे स्थापितस्य अद्वैतमञ्जरीग्रन्थमालां समारब्धवतः श्रीविद्यामुद्रणालयस्य ग्रन्थप्रकाशनाय नियुक्तेषु पण्डितेषु गणपतिशास्त्र्यप्यन्यतमः । द्वारकापीठाधिष्ठातृणा वेदान्तशास्त्रे समाधेयत्वेन (1905 A. D.) काले प्रकाशितानां सप्त प्रश्नानां उत्तरदानेन गणपतिशास्त्री वेदान्तकेसरीति बिरुदेन भूषितश्च ।
१. अथशब्दविचारः २. ईशावास्यविवृतिः ३. कटाक्षशतकम् ४. केनोपनिषद्विवृतिः ५. गुरुराजसप्ततिः ६. जीवविजयचम्पूः ७. दुर्गाशतकम्, ७. दुर्गाशतकम्. ८. ध्रुवचरितम् ९. नैर्गुण्यसिद्धिः १०. पार्थप्रहारः ११. वैदिकाभरणव्याख्या - मुकुरः १२. शारीरकमीमांसारहस्यम्, १३. श्रवणविधिवाक्यार्थश्चानेन कृताः ग्रन्थाः । केचित् अमुद्रिताः, केचन अद्वैतसमास्वर्णमहोत्सवपत्रिकायां प्रकाशिताः । वैदिकामरणव्याख्या तु अण्णामलैविश्वविद्यालये मुद्रिता ।

४९. गणेशशर्मा (1906 A. D.)
दक्षिणदेशजोऽयं दक्षिणार्काडजिल्लान्तर्गतदीक्षामङ्गलाग्रहारवासी मद्रपुरीसंस्कृतकलाशालाप्रधानाध्यापक्योः कृष्णतटाक - कृष्णशास्त्रि रामचन्द्रदीक्षितयोश्शिष्यः र्विशतिशतकीयोऽयं गणेशशर्मा ।
१. सुरेश्वरहृदयम् । ग्रन्थेऽस्मिन् शबलब्रह्मणः जगत्कारणत्वं, साक्षिशब्दार्थः भावाविद्या, अज्ञानशब्दनिर्वचनम्, भावरूपाज्ञाने सुरेश्वरसम्मतिः, सुषुप्त्यवस्थायां भावाविद्यासद्भावे भाष्यवार्तिकतात्पर्यनिर्णयः, वार्तिकमतेन दृष्टिसृष्टिवादः प्रतिपादितः । ग्रन्थोऽयं मद्रासलाजर्नलमुद्रणालये मुद्रितः ।।

५०. गीर्वाणेन्द्रसरस्वती (1600-1700 A. D.)
अमरेन्द्रसरस्वतीप्रशिष्यः विश्वेश्वरसरस्वतीशिष्यश्चायं गीर्वाणेन्द्रसरस्वती नृसिम्हाश्रमिबोधेन्द्रयत्योः गुरुश्चेति ज्ञायते । रघुनाथाश्रमिसामयिकः जगन्नाथाश्रमिसामयिकश्व ।।
%%% Chart
1. प्रपञ्चसारसङ्ग्रह (17637 TSML)
शाङ्करप्रपञ्चसारसङ्ग्रहात्मकोऽयं ग्रन्थः अमुद्रितस्सरस्वतीमहालये लभ्यते ।।

५१. गुरुमूर्तिशास्त्री (1850-1910 A. D.)
अनेनाद्वैतानन्दतीर्थकृताया ब्रह्मसूत्रतात्पर्यदीपिकायाः व्याख्यातात्पर्य विमर्शिनी रचिता । ग्रन्थोऽयं तेलुगु लिप्यां मुद्रितः । एकोनविंशतिशतकीयोऽयमिति ज्ञायते ।।

५२.
गुरुस्वामिशास्त्री (1911 A. D.)
चोलदेशीयकुम्भघोणसमीपस्थे वरहूर्ग्रामे लब्धजन्मायं गुरुस्वामिशास्त्री साहित्यवेदान्तपण्डितः मद्रपुरीसंस्कृतकलाशालाया प्राप्तविद्यः वैद्यनाथधर्माम्बिकयो पुत्रः मद्रपुरीसंस्कृतकलाशालाध्यापकेभ्यः बालसुब्रह्मण्यशास्त्रि वैद्यानाथशास्त्रि-रामचन्द्रदीक्षितेभ्यः प्राप्तविद्यः विंशतिशतकीय इति ज्ञायते ।
१. शारीकव्याख्याप्रस्थानानि ।।
विमर्शकसरण्या शाङ्करभाष्योपरि प्रवृत्तानां पद्मपादमण्डन-सुरेश्वर-विमुक्तात्म-प्रकटार्थकार-ज्ञानधनपाद-नृसिम्हभट्टोपाध्यायकृतानां व्याख्यानानां आशयभेदा उपवर्णिताः । ग्रन्थोऽयं बालमनोरमामुद्रणालये मुद्रितः ।
५३. गोपालः (1850 A. D.)
अस्य पिता मुद्रगलभट्टः । औत्तरोऽयं गोपालः एकोनविंशतिशतकीय इति ज्ञायते ।
१. विवेकामृतम् - (10. B. 6. AL) द्वाभ्यां प्रकरणाभ्यां विभक्तोऽयं प्रकरणग्रन्थः देहविचारं, तात्पर्यशोधनम्, मुक्तिस्वरूपविचारम्, ब्रह्मणस्सर्वात्मकत्वं तस्याद्वयत्वञ्च वर्णयति । ग्रन्थोऽयमडयारपुस्तकालये लभ्यते ।

५४. गोपालबालयतिः (1500-1600 A. D.)
``श्रीमद्यतीन्द्रमानम्य जगन्नाथश्रमं गुरुम्" इति एतदीये ग्रन्थे दर्शनात् बालगोपालापरनामायं गोपालवालयतिः नृसिम्हाश्रमिसामयिकः नृसिम्हाश्रमिसतीर्थ्यश्चेति षोडशशतकीयः । अस्य शिष्यस्स्वयम्प्रकाशाख्यश्शाङ्करैकश्लोकव्याख्यामधुमञ्जरीकारः ।
१. काठोपनिवच्छाङ्करभाष्यव्याख्या - भाष्यविवरणम् । (ASS. 7) ग्रन्थोऽयं आनन्दाश्रमे मुद्रितः ।
२. मधुमञ्जरी - (D. 4706 MGOML) शाङ्करमनीषापञ्चक व्याख्यात्मकोऽयंग्रन्थ अपूर्ण अमुद्रितश्च मद्रासराजकीयपुस्तकालये लभ्यते । 

५५. गोपालानन्दसरस्वती (माकिं 1700-1900 A. D.)
योगानन्दशिष्योऽयं गोपालनन्दसरस्वती वाराणसीवासीति ज्ञायते । गोपालानन्दमेनं दासगुप्तमहाशयस्सप्तदश - एकोनर्विशतिशतकमध्यवर्तिनं प्रतिपादयति (HIP. Vol. II Page 57)
1. अखण्डात्मप्रकाशिका - (R. 3891 MGOML)
निष्कामकर्मणश्चित्तशुद्धिः, चित्तशुद्धेस्संसारयाथात्म्यावबोधः तस्माद्वैराग्योत्पत्तिः, ततो मुमुक्षुत्वम्, तस्मादुपायोपेयेषणापूर्वकसर्वकर्मसन्यासः इति निरूप्य महावाक्यार्थपरिज्ञानं मननं योगाभ्यसाच्चित्तस्य प्रत्यक्प्रवणता, तस्माच्च सर्वाविद्योच्छेदः, तस्माच्च पूर्णाद्वितीयसच्चिदानन्दात्मना अवस्थितिरिति प्रतिपादितम् । प्रकरणग्रन्थोऽयं पूर्ण अमुद्रितश्च मद्रासराजकीयपुस्तकालये मैसूरपुस्तकालये चोपलभ्यते ।
२, ईशावास्योपनिषट्टीका (4527 B. R. D.)
३. वेदान्तामृतम् (4913 B. R. D.)

५६. गोविन्दभट्टः (1752 A. D.)
आहिताग्निरयं गोविन्दभट्टः विश्वनाथभट्टपुत्रः उत्तरभारतीय इति ज्ञायते । अनेनात्मार्कबोधाख्ये ग्रन्थे ग्रन्थनिर्माणकालः 1676 शकवत्सराणीति प्रतिपाद्यते । एवञ्चास्यकाल अष्टादशशतकमिति ज्ञायते ।
१. आत्मार्कबोधः (B. D. 285 R. A. S. Bomboy)
विंशत्यधिकशतपद्यैः षड्भिरध्यायैश्च पूर्णोऽयं ग्रन्थः प्रकरणग्रन्थकोटिमारोइति । अमुद्रितोऽयं रायलआसियाटिकसोसाइटि बाम्बेनगरस्थपुस्तकालये लभ्यते ।
२. वेदान्ततात्पर्य निवेदनम् (908 D. C. Panjab)

५७. गोविन्दानन्दसरस्वती (1885 A. D.)'
माधवानन्दसरस्वतीशिष्योऽयं गोविन्दानन्दसरस्वती वाराणसीवासीति ज्ञायते । अनेन ब्रह्मसुधाकारिकाख्यः ग्रन्थः कृतः । स च निर्णयसागरमुद्रणालये मुद्रितः ।

५८. गोविन्देन्द्रयतिः (1800-1900 A.D.)
नारायणेन्द्रशिष्योऽयमाधुनिकः । अनेन रचिते तत्वानुभवाख्येऽमुद्रिते (R. 47 B. MGOML) पुस्तके स्वानुभवपद्धत्यां जीवन्मुक्तस्थितिः वर्णिता । अन्योऽयं ``असङ्गात्मप्रकाशिकाख्यः" ग्रन्थः विश्वभारतीशान्तिनिकेतनपुस्तकालयेऽमुद्रितः (3035 V. B. S.) लभ्यते ।

५९. गौरीनाथशास्त्री (1850-1920 A.D.)
वन्दरपुरवासी स्वामिनाथशास्त्रिपौत्रः, नृसिम्हशास्त्रिपुत्रः शाण्डिल्यगोत्रजोऽयं सच्चिदानन्दसरस्वतीशिष्य इति ज्ञायते ।।
१. शाङ्करभाष्यगाम्भीर्यनिर्णयखण्डनम् । ग्रन्थोऽयं चोलदेशीयेन रामसुव्रह्मण्यशास्त्रिणा कृतस्य शाङ्करभाष्यनिर्णयाख्यग्रन्थस्य खण्डनात्मक अद्वैतग्रन्थः । ग्रन्थोऽयं वाणीविलासमुद्रणालये मुद्रितः ।।

६०. गङ्गाधरः (1200 A.D.)
अस्य पिता मनोरथः । अयं (1137 A. D.) काले आसीदिति ``यफि़ग्राफि़का इंडिकापत्रिकायाः द्वितीयभागे 333 पुटे दृश्यते । अनेन अद्वैतशतकमितिग्रन्थः कृत इति च दृश्यते ।।"

६१. गङ्गाधरभगवद्भक्ताकिङ्करः (1750-1850 A. D.)
अग्निहोत्रिवीरेश्वरसूरिपौत्रः सदशिवभट्टपुत्रः महाडकरइत्युपनामायं गङ्गाधरभगवद्भक्तविङ्करः वत्सर्षिगोत्रजः अष्टादशशतकीय इति निश्चीयते ।।
१. सुबोधिनी-शारीरकसूत्रसारार्थचन्द्रिकानामायं ग्रन्थः सूत्रवृत्तिरूप अमुद्रितः लन्दननगरस्थभारतकार्यालयपुस्तकालये उज्जयिन्याञ्च लभ्यते ।।

६२. गङ्गाधरेन्द्रसरस्वती (1780-1880 A. D.)
दक्षिणदेशजोऽयं गङ्गाधरेन्द्रसरस्वती सर्वज्ञसरस्वतीप्रशिष्य रामचन्द्रसरस्वतीशिष्यः अष्टदशशतकोत्तरार्धादारभ्य एकोनविंशतिशतकमध्यभागावधिके काले (1780-1880 A. D.) आसीदिति निश्चीयते । एतदीये स्वाराज्यसिद्धिग्रन्थे ``वस्वब्धिमुन्यवनिमानशक" इति ग्रन्थनिर्माणकाल (1728 श= 1826 A.D) इति दृश्यते ।।
१. वेदान्तसिद्धान्तसूक्तिमञ्जरी (C.S.S. 4) । अप्पय्यदीक्षितकृतसिद्धान्तलेशसारात्मकोऽयं ग्रन्थस्समन्वयाविरोधसाधनफलाख्यैश्चतुर्भिः परिच्छेदैः परिच्छिन्नः कल्कत्तासंस्कृतमालायां मुद्रितः । अस्य व्याख्यापि ``प्रकाशाख्या" अनेन कृता ।
२. सिद्धान्तचन्द्रिकाव्याख्या - ``उद्गारः" ।। रामब्रह्मेन्द्रापरनाम्ना रामभद्रानन्देन विरचितस्य सिद्धान्तचन्द्रिकाग्रन्थस्य व्याख्यात्मकोऽयं ग्रन्थः गोपालनारायण मुद्रणालये मुम्बय्यां मुद्रितः ।
३. स्वाराज्यसिद्धिः । पद्यात्मकोऽयं ग्रन्थः द्वैतखण्डनपूर्वकं अध्यारोपापवादकैवल्याख्यैस्त्रिभिः प्रकरणैः निरावरणस्वप्रकाशपरमात्मस्वरूपं प्रकाशयति । ग्रन्थोऽयं आर्यमतसंवर्द्धिनीमुद्रणालये (A. M. S. S. 25) मद्रासनगरे मुद्रितः । अस्य व्याख्या मूलकृत्कृता कैवल्यद्रुमाख्या, कृष्णशास्त्रिकृता परिमलाख्या च वर्तेते ।
४. निर्वाणाष्टकव्याख्या (264 TCL), 5. प्रणवकल्पप्रकाशः,
६. शुकाष्टकाव्याख्या (214 Nasik) च अनेन कृताः ग्रन्थाः ।।

६३. गङ्गापुरीभट्टारकः (1100-1200 A. D.)
न्यायरत्नदीपावली-इष्टसिद्धिविवरणकार आनन्दानुभव एवायमिति आनन्दानुभवमधिकृत्य विचारावसरे प्रतिपादितम् । गङ्गापुरीभट्टारककृतत्वेन त्रिपेदीमहाशयेन वर्णितः पदार्थतत्वनिर्णयाख्यः ग्रन्थः मद्रासराजकीयहस्तलिखितपुस्तकालये आनन्दानुभवोपज्ञ इति दृश्यते । आनन्दानुभवस्यैव सन्यासस्वीकारात्पूर्वं गङ्गापुरीभट्टारक इति स्यान्नाम । दासगुप्तस्तु (950-1050 A. D.) अस्य कालं वर्णयति ।
१. पदार्थतत्वनिर्णयः (R. 2981 MGOML) अस्य व्याख्या आनन्दगिरिकृता चास्ति ।।

६४. घनश्यामः (1706-1786 A.D.)
महाराष्ट्रजातीयः चौण्डाजीपण्डित - आर्यक - सर्वज्ञकवि - वश्यवाचसू - कण्ठीरव - सरस्वतीति उपनामभिः प्रसिद्धोऽयं घनश्यामः तञ्जावूरनगरवासी तुक्कोजीक्षितीशस्य (1729-35 A.D.) मन्त्री चासीत् । काशी - महादेवयोः पुत्रः चौण्डबालाजीपौत्रः मौनभार्गववशंजः ईशपण्डितस्य कनिष्ठभ्राता शाकम्भर्याश्च भ्राता सुन्दरीकमलयोः पतिः चन्द्रशेखर-गौवर्धनयोः पिता संस्कृते प्राकृते प्रान्तीय भाषायाञ्च बहूनां ग्रन्थानां कर्ता चेति (J. O. R. Vol III & IV) ज्ञायते ।
१. अद्वैतबोधः । ग्रन्थोऽयं विद्धशालभञ्जिकोपोद्धाते घनश्यामपत्न्या निर्दिष्टः । (TSML 4678)
२. प्रचण्डराहूदयः । ग्रन्थोऽयं बेलगामनगरे (1960) मुद्रितः ।

६५. चन्द्रशेखरभारतिः (2000 A.D.)
एते शृङ्गगिरिशङ्कराचार्यपीठाधिष्ठिताः चतुस्त्रिंशत्तमा आचार्याः नितान्तं विद्वांसः ज्ञानानिष्ठाश्चासन् । विवेकचूडामणिव्याख्या अदसीयः ग्रन्थः । व्याख्येयं तत्र तत्र सूत्रभाष्याद्युक्तानर्थान् विशदय्य मूलग्रन्थं सुलभयति । मुद्रितश्चायं ग्रन्थः ।

६६. चन्द्रिकाचार्यभिक्षुः (1800-1900 A. D.)
दक्षिणदेशवासी कृष्णानन्द-रामब्रह्मेन्द्र सरस्वत्योश्शिष्योऽयं चन्द्रिकाचार्यभिक्षुः गुरुणा रामब्रह्मेन्द्रेण ``रामब्रह्मेन्द्र" इति दत्तनाम मन्नार्गुडिराजुशास्त्र्यपराभिधत्यागराजमखिप्रोत्साहित अद्वैतसिद्धान्तगुरुचन्द्रिकां चकारेति एकोनविंशतिशतकीय इति च निश्चीयते ।
१. अद्वैतसिद्धान्तगुरुचन्द्रिका - 
ग्रन्थोऽयं भागद्वयपरिमितः प्राचीनान् निलीनविलीनान् अद्वैतसिद्धान्तान् श्रुतियुक्तिभ्यां च गवेषयति । ग्रन्थोऽयं ओरियण्टलमुद्रणालये मद्रासनगरे मुद्रितः । अस्य व्याख्या मूलकृतैव अमृतरसझरीनाम्नी कृता ।

६७. चिद्धनानन्दः (1895-1945 A.D.)
वाराणसीवासी रामकृष्णपरमहंससम्प्रदायगतोऽयं चिद्धनान्दः लक्ष्मणशास्त्रिद्राविडशिष्यः विंशतिशतकीय इति निर्णीयते । अच्युतानन्दोऽस्य दीक्षागुरुः ।
%%% Chart
१. ब्रह्मसूत्रभाष्यनिर्णयः ।
ग्रन्थोऽयं ब्रह्मसूत्रोपरि प्रवृत्तानां शाङ्कररामानुजभास्करमध्वनिम्बार्कादिविबिधभाष्यस्य गुणदोषविवेचनां कुर्वन् शाङ्करभाष्यस्यैव बादरायणव्याससम्मतत्व प्रदर्शयति । समालोचनात्मकोऽयं ग्रन्थः बोधायनादिवृत्तिसद्भावे संशयमुत्पादयति । ग्रन्थोऽयं काशीस्थरामकृष्णसेवाश्रममुद्रणालये मुद्रितः ।

६८. चिरञ्जीविभट्टाचार्यः (1675-1750 A.D.)
गौडदेशजोऽयं काश्यपगोत्रजः चिरञ्जीविभट्टः रामदेवचिरञ्जीव-वामदेवचिरञ्जीव इत्यपरनामा सामुद्रिकाचार्यपौत्र इति ज्ञायते । स्वकृतविद्वन्मोदतरङ्गिण्यां 
``द्वैताद्वैतमतादिनिर्णयविधिप्रोद्बुद्धबुद्धिश्रुतः
भट्टाचार्यशतावधान इति यो गौडोद्भवोऽभूत् कविः ।
विद्वन्मोदतरङ्गिणी ननु चिरञ्जीवेन तज्जन्मना"
इत्यादिवर्णनात् शतावधानभट्टाचार्यापराभिधानराघवेन्द्रस्य पुत्र इति निर्णीयते । स्वकृतकाव्यविलासे रघुदेवचरणध्यानात् रघुदेवन्यायालङ्कारशिष्य इति ज्ञायते ।
१. विद्वन्मोदतरङ्गिणी - 
ग्रन्थोऽयं श्रव्यकाव्यशैल्यां प्रणीतः विविधदर्शनसारसङ्ग्रहात्मकः शास्त्रप्रसिद्धैर्युक्तिजार्लैर्व्यावहारिकैश्च युक्तिजालैः वैष्णवशाक्तवैशेषिकनैय्यायिक अलङ्कारिकसिद्धान्तान् तदीयानाशयभेदांश्चोपवर्णयन् नास्तिकैस्साकं ततद्दार्शनिकानां वादप्रतिवादं सयुक्तिक प्रकाशयन् नास्तिकवादखण्डनाय वेदान्तशास्त्रस्य न्यायशास्त्रापेक्षित्ववर्णनपूर्वकं आत्मन अद्वितीयत्वं 
``ब्रह्मण्येव समुत्पद्य जगत्तत्रैव लीयते ।
बुद्बुदा इव तोयेषु शुक्तयो रजतेष्विव" ।
इत्यादिना प्रतिपादयति । प्रसङ्गवशात् हरिहराद्वैतभावनां प्रकाशयति । ग्रन्थोऽयं कल्कत्ताग्रन्थमालायां वेङ्कटेश्वरमुद्रणालये बम्बईनगर्यां काव्यप्रकाशमुद्रणालये वाराणस्यां च मुद्रितः ।
काव्यविलासः, शृङ्गारतटिनी, वृत्तरत्नावली, माधवचम्पूरिति ग्रन्थाश्च रचिताः ।

६९. चोक्कनाथदीक्षितः (1600-1650 A.D.)
सभानाथ सर्वक्रत्वपराभिधानोऽयं अग्निहोत्रभट्टपितामहः, चोलदेशीयः नारायणशास्त्रि - गणपत्यम्बयोः पुत्रः, नारायणसुब्रह्मण्यवैद्यनाथयोः, द्वादशाहयाजि बालपतञ्जल्यपरनाम्नः बालचन्द्रदीक्षितस्य च पिता, अस्यैव सुन्दरेशापरनामा च ज्ञायते । अस्य जामाता रामभद्रमखी । रामभद्रमखिनश्श्वशुरः गुरुश्चायं चोक्कनाथदीक्षितस्सप्तदशशतकीयः नल्लादीक्षितधर्मराजाध्वरीन्द्रसामयिकश्चेति निश्चीयते । वेङ्कटेशदीक्षितशिष्यश्चायं प्रसिद्धवैय्याकरणोऽपीति ज्ञायते । अस्यैव सञ्चारिमहाभाष्यमिति नाम प्रसिद्धम् ।
शाब्ददीपिका, घातुरत्नावली, भाष्यरत्नावलीति व्याकरणग्रन्थाश्चानेन कृताः । शाब्ददीपिकायाः शब्दकौमुदी नाम प्रसिद्धम् । अत्रायं वंशपरम्परावृक्षः-
%%% Chart
१. वेदान्तदीपिका (356 T. C. D. Vol. III)
विषयविदग्धापरनामायं ग्रन्थ अमुद्रितस्तिरुवनन्तपुरपुस्तकालये लभ्यते । शाङ्करभाष्यगतार्थसंग्राहकोऽयं सूत्रवृत्तिरूपः ग्रन्थः चौखाम्बामुद्रणालये मुद्रितश्च ।।

७०. जगज्जीवनः (1775-1850 A.D.)
अच्युताश्रमशिष्योऽयं अद्वैताश्रमप्रशिष्यः वाराणसीवासी जगज्जीवन इति ज्ञायते ।
१. ब्रह्मानन्दप्रकाशिका (145 Nasir)
२. वेदान्तसारसद्रत्नावलिः (1570 V. B. S.)
प्रकरणग्रन्थोऽयममुद्रितः विश्वभारतीशान्तिनिकेतअडयारपुस्तकालययोर्लभ्यते ।।

७१. जगन्नाथसरस्वती (1664 A. D.) कालात्प्राक् 
``हरिहरसरस्वती यद्गुरुरीड्यः परमहंसानाम्" इति ग्रन्थे दर्शनात् हरिहरसरस्वतीशिष्योऽयं जगन्नाथसरस्वतीति निश्चयः ।
१. अद्वैतामृतम् ।
सव्याख्योऽयं ग्रन्थः कवलाख्यैः पञ्चभिः परिच्छेदैः परिच्छिन्नः पद्यात्मकः वेदान्तशास्त्रप्रतिपादितं आत्मज्ञानोपायं, प्रपञ्चमिथ्यात्वं अद्वैतप्रक्रियाञ्च ज्ञान्यङ्गनाप्रश्नप्रतिवचनसरण्या प्रतिपादयति । प्रथमं काचनानङ्गविलासभूमिरङ्गना विवेकाश्रमनामकस्य ज्ञानिनस्सकाशमागत्य सन्यासविधिं ईषणात्रयत्यागञ्च निन्दन्ती ऐहिकविषयाभिलाषावश्यकतां प्रस्तौति । तस्यै उपदेशव्याजेन सर्वेऽपि अद्वैतसिद्धान्ताः प्रतिपादिता । ग्रन्थोऽयं चौखाम्बामुद्रणालये वाराणस्यां मुद्रितः । अमुद्रितस्तु सरस्वतीमहालये लन्दननगरस्थपुस्तकालये च लभ्यते । अस्य व्याख्या ``तरङ्गिणी" नाम्नी मूलकृतैव कृता ।
२. अद्वैतवादः - ग्रन्थोऽयं अप्रकाशितः कल्कत्तासूच्यास्तृतीये पुटे दृश्यते ।
३. वेदान्तरहस्यम् - ग्रन्थोऽयं हरप्रसादाशास्त्रिसूच्यामेव दृश्यते । (HPR. Vol. IV. No. 280)

७२. जगन्नाथाश्रमः (1465-1575 A. D.)
अद्वैतदीपिकायां नृसिम्हाश्रमिणा ``जगन्नाथाश्रममुनेश्चित्रं चरणरेणवः" इति नमस्कृतोऽयं नृसिम्हाश्रमविद्यागुरुर्जगन्नाथाश्रमः पञ्चदशशतकापरार्धादारब्धे षोडशशतकपूर्वार्धावधिके काले आसीदिति निश्चीयते । दक्षिणदेशीयेनानेन शाङ्करसूत्रभाष्यस्य काचन व्याख्या रचिता स्यादित्यूह्यते । गोविन्दानन्दकृतायां ``रत्नप्रभायां" पञ्चमपुटे युष्मदस्मद्वाच्यविचारावसरे ``आश्रमचरणास्तु टीकायोजनायां एवमाहुः" इति एतदीयः ग्रन्थः परामृष्टः ।
जगन्नाथाश्रमं वन्दे यतिं वेदान्तकोविदम् ।
पदवाक्यप्रमाणज्ञं स्वधर्मनिरतं सदा ।।
इति प्रक्रियाकौमुदीव्याख्यायां (P. 798, Bombay Sanskrit Series 82. Part 2) विट्ठलाचार्येण नमस्कृतोऽयमिति विट्टलदासस्यापि गुरुरिति ज्ञायते । (No. 148 D. C. A. L. Vol. VI) 
दासगुप्तमहाशयस्तु ग्रन्थस्यास्य ``भाष्यदीपिका" इति नामेति स्वीये भारतीयवेदान्तसाहित्येतिहासे (HIP. Vol. II Page 103) प्रतिपादयति । मुद्रितरत्नप्रभायान्तु ``टीकायोजना" इत्येब नाम दृश्यते 
१. सूत्रभाष्यव्याख्या - (भाष्यदीपिका टीकायोजना)

७३. जयकृष्णतीर्थः (1650-1750 A. D.)
सर्वेश्वरतीर्थशिष्योऽयं आनन्दगिरिं योगवासिष्ठञ्च प्रमाणयति । रामतीर्थादिवत् तीर्थान्तनामायं सप्तदशशतकीयस्स्यादिति परं ज्ञायते नान्यत्प्रबलतरं प्रमाणमुपलभामहे ।
१. ब्रह्मामृतम् - प्रकरणग्रन्थोऽयं 1027 श्लोकैः परिमितः गुरुशिष्यप्रणाल्या अद्वैतात्मकज्ञानाप्त्यै श्रवणमननध्यानसमाधीनां उपयोगं जगतस्सृष्टिप्रलयौ शुष्पतर्कवादात् ज्ञानानुत्पत्तिं च बहुविधाख्यायिकाप्रतिपादनसरण्या निरूपयन्
आत्मैव सर्वमिदमस्य विकारजालं मृत्स्वेव कुम्भविकृतः कटकं सुवर्णम् ।
माला फणीव निजहेतुमतत्व एव स्वाज्ञानमूलमखिलं मयि तत्वतो न ।।
इत्यादिना ब्रह्मात्मकत्वं प्रतिपादयन् व्रजमूषणदासकम्पन्यां काश्यां मुद्रितः ।

७४. जयरामः (1700 A. D.)
नागरवंशीयोऽयं जयरामपण्डितः माकिं सप्तदशशतकीय इति ज्ञायते । एतदीयमहावाक्यादर्शप्रतिलिपिकाल 1773 संवदिति बरोडापुस्तकालयसूच्याः ज्ञायते ।
१. ब्रह्मसूत्रार्थप्रकाशिका । 231 पञ्जाबविश्वविद्यालयसूच्यां दृश्यते ।
२. महावाक्यादर्शः 12424 बरोडासूच्यां दृश्यते ।

७५. ज्ञानघनः (950-1050 A. D.)
``विद्यावृष्टिसुपक्वशिष्ययतिसत्सस्यैः क्षमा शोभते ।
शश्वद्बोधघनस्य यस्य गुरवे तस्मै नमः श्रेयसे ।।"
इति स्वग्रन्थे बोधघनं प्रणमन्नयं ज्ञानघनः बोधघनशिष्य इति निर्णीयते । बोधघनस्यैव नित्यबोधघन इत्यपि नामान्तरं शृङ्गेरीसूच्यां दृश्यते । अस्य ज्ञानघनस्य शिष्यः ज्ञानोत्तम इति ख्यातः चित्सुखगुरुश्च । शृङ्गगिरिगुरुपरम्परासूच्यां ज्ञानघनकालः (848-910 A. D.) इति दृश्यते । परन्तु अनन्यानुभवशिष्यस्य प्रकाशात्मगुरुणा अनन्यानुभवेन आत्मतत्वमिति ग्रन्थः कृतः यश्च तत्वशुद्धिकारेण ज्ञानघनेन स्वग्रन्थे प्रथमपुटे ``अनन्यानुभवानन्दाद्वितीयात्मतत्वमधिकृत्य केषुचिदर्थेषु परिशोधनं विधीयते" इति अनन्यानुभव उद्धृत इति च श्रीकण्ठशास्त्री (I. H. Q. Vol XIV) वदति । दासगुप्तस्तु (H. I. P.) ग्रन्थे (1200-1300) शतकीय इति वर्णयति । परन्तु न्यू इण्डिया अण्डिक्वैरि पत्रिकायां (N. I. A. Vol II Page 62-72) दशमशतकीय इति निर्णीतम् ।
१. तत्वशुद्धिः । षटचत्वारिंशद्धिः प्रकरणैः पूर्णोऽयं प्रकरणग्रन्थः कुत्रचित् मण्डन मिश्रकृतां ब्रह्मसिद्धिं वाक्येन भावेनानुसरति । तत्वमस्यादिवाक्यजनितापरोक्षब्रह्मविद्याया अनाद्यविद्यापटलसमुत्पाटननित्यसिद्धनिरतिशय-आनन्दप्रत्यगद्वितीय परमात्मचैतन्यात्मना अवस्थानं अपवर्ग इति भाष्यकारमतं साधयति । ग्रन्थोऽयं मद्रासविश्वविद्यालयसंस्कतग्रन्थमालां मुद्रितः । अस्य व्याख्या उत्तमज्ञयतिकृता तिरुवनन्तपुरपुस्तकालये लभ्यते ।

७६. ज्ञानामृतयतिः (1350-1450 A. D.)
``निराकृतद्वैतकथं यतीश्वरं नमामि नाम्ना सदृशोत्तमामृतम् ।।" इति उत्तमामृतयतिं ``आनन्दारण्यपूज्यानांं पादपङ्कजपांसवः । पान्तु मां" इत्यानन्दारण्यञ्च नमस्कर्वन्नयं ज्ञानामृतयति उत्तमामृतआनन्दारण्ययोश्शिष्य इति भाति ।
अनेन स्वग्रन्थे पूर्वटीकाक्षरेभ्यः भावतत्वं गृहीत्वा व्याख्या निर्मीयत इति प्रतिज्ञातम् । अनेन स्वग्रन्थेषु उदयनाचार्यः भट्टाचार्यः ब्रह्मदत्तः, भर्तृप्रपञ्चः, भट्टमिश्रः, मण्डनमिश्रः, वाचस्पतिमिश्रः, गौडपादाचार्याश्च नामतो निर्दिष्टाः । (1350-1450 A.D.) कालवर्तिना भाव्यमित्यूह्मते एतदीयऐतरेयभाष्यटिप्पणात् अयं सायणाचार्यसामयिक इति ज्ञायते ।।
१. नैष्कर्म्यसिद्धिव्याख्या - विद्यासुरभिः (R. 3354 MGOML)
ज्ञानोत्तमकृतात् चित्सुखाचार्यकृताच्च व्याख्यानाद्भिन्नोऽयं ग्रन्थः पूर्ण अमुद्रितश्च मद्रास-अडयारपुस्तकालययोर्लभ्यते ।।
२. ऐतरेयभाष्यटिप्पणम् (D 332 MGOML)

७७. ज्ञानेन्द्रसरस्वती (1550-1650 A. D.)
अनेन पुरुषार्थसुधानिधौ वासुदेवेन्द्रः नमस्कृतः । अग्निहोत्रभट्टेन अद्वैतरत्नकोशपूरणी कृता, तत्र ``ज्ञानेन्द्रापरनामानं" इत्यादिना ज्ञानेन्द्रः वासुदेवेन्द्रश्च नमस्कृतौ । एवञ्चाग्निहोत्रभट्टः ज्ञानेन्द्रवासुदेवेन्द्रशिष्य इति उपपादितम् । अग्निहोत्रभट्टेन च स्वकृते अद्वैतरत्नकोशपूरणीग्रन्थे 49 तमे पुटे ``अस्मत्परमगुुचरणमतमेव सम्यक्," इति 72 तमे पुटे जिज्ञासाशब्दविचारे ``परमगुरुचरणास्तु जिज्ञासापदं तन्त्रेणोपात्तमित्याहु" रित्यादिना ज्ञानेन्द्रसरस्वतीग्रन्थ अनूद्यते । तस्मात् ज्ञानेन्द्रसरस्वती वासुदेवेन्द्रगुरुरग्निहोत्रभट्टप्राचार्यस्सप्तदशशतकपूर्वार्धा वधिककालवासीति सिध्यति ।।
१. पुरुषार्थसुधानिधिः (सूत्रवृत्तिः) (R. 2471 MGOML)
सर्ववेदान्तश्रुतिसारसंग्रहः सूत्रविषयवाक्यवृत्तिः वैय्यासिकब्रह्ममीमांसासारसंग्रहः सूत्रभाष्यसारसंग्रहापरनामा ग्रन्थः मद्रास-सरस्वतीमहालय-अडयार-अनन्तशयनपुस्तकालयेषु लभ्यते ।।

७८. ज्ञानोत्तमः I (1200 A. D. कालात्प्राक्‌)
स्वग्रन्थे नैष्कर्म्यसिद्धिविवरणे ``यमनियमनिलयधिषणो जयति श्रीसत्यबोधार्यः" इत्यादिना सत्यबोधाचार्यस्सादरं सूचितः । एवं ग्रन्थान्ते
चोलेषु मङ्गलमिति प्रथितार्थनाम्मि ग्रामे वसन् पितृगुरोरभिधां दधानः । ज्ञानोत्तमस्सकलदर्शनपारदृश्वा व्याख्यामिमां वितनुते स्फुटमिष्टसिद्धेः ।।
इत्यादिना स्वपरिचयश्च प्रदत्तः । अनेन उपनिषदः वाजसनेयसंहिता शाबरभाष्यं, तन्त्रवार्तिकञ्च प्रमाणत्वेन निर्दिष्टानि । एवञ्चायं स्वपितुरेव प्राप्तविद्यः चोलदेशीयः मङ्गलग्रामवासी ज्ञानोत्तमपुत्रः ज्ञानोत्तमशिष्यश्चेति ज्ञायते । स्वग्रन्थे निर्दिष्टः सत्यबोधाचार्यः अस्य दीक्षागुरुरिति ज्ञायते । दशमशतकीयस्य सर्वज्ञात्मनः कृतेः व्याख्यां कुर्वन्नयं दशमशतकादर्वाचीन इति तु निश्चयः । अस्यैव आश्रमस्वीकारादनन्तरं ज्ञानानन्द इति नामापि दृश्यते ।
चित्सुखाचार्यगुरुर्ज्ञानोत्तम अयमेवोतान्य इति न निर्णेतुं शक्यते । ग्रन्थान्तपुष्पिकायां `गौडेश्वराचार्यज्ञानोत्तमकृतायां' इति दर्शनात् मिश्रान्तोऽयं बंगालवासी गौडेश्वराचार्यापरनामा स्यादिति वक्तुं शक्यते । गौडेश्वरापराभिधस्सत्यानन्द इत्यपि व्यवहृियते । चित्सुखाचार्येण स्वग्रन्थे गौडेश्वराचार्यस्सत्यानन्दश्च नमस्कृतौ । तथा च चित्सुखाचार्यगुरुर्ज्ञानोत्तम एवायमिति केचित् वदन्ति । एवं सति चोलेषु मङ्गलमिति श्लोको विरुध्यते ।
दासगुप्तस्तु मङ्गलग्रामवासित्वेन वर्णितं ज्ञानोत्तमं नवीनं वदति । T. R. चिन्तामणिमहाशयस्तु चित्सुखगुरोरन्यं ज्ञानोत्तमिमं वदति ।
हिरियण्णामहाशयैस्तु - यौऽयं सत्यबोधापरनामा सत्यानन्दः चित्सुखाचार्येण निर्दिष्टः स चित्सुखाचार्यात्प्राचीनः । सर्वज्ञपीठाधिरोहणशीलानां काञ्चीकामकोठिपीठाधीशानां नामद्वयवत्वं प्राचीनपरम्परासिद्धम् । कामकोटिपरम्पराप्रामा ण्यात् ज्ञानोत्तमः शङ्कराच्चतुर्थः पीठाधिपतिः । चित्सुखकृता भावतत्वप्रकाशिकानाम्नी नैष्कर्म्यसिद्धिव्याख्या ज्ञानोत्तमकृतां नैष्कर्म्यसिद्धिचन्द्रिकामनुसरति । तस्माच्चित्सुखाचार्यगुरोर्भिन्नोऽयं ज्ञानोत्तमः द्वादशदशमशतकयोरन्तालवर्तीति निश्चीयते ।
१. इष्टसिद्धिविवरणम् - (G. O. S. 65.)
२. नैष्कर्म्यसिद्धिचन्द्रिका । ग्रन्थोऽयं बन्दरकार प्राच्यभाषानुसन्धानसमिति ग्रन्थमालायां मुद्रितः ।

७९. ज्ञानोत्तमः II (1100-1200 A. D.)
ज्ञानोत्तमोऽयं ज्ञानघनशिष्यः । चित्सुखाचार्यागुरुश्च चित्सुखाचार्यैस्तत्वप्रदीपिकायां ज्ञानोत्तमोऽयं गौडेश्वराचार्यशब्देन विशेषितः । तत्वप्रदीमिकायां, तात्पर्यटीकायां, भावतत्वप्रदीपिकायाञ्च चित्सुखाचार्येणास्य ज्ञानोत्तमस्य सत्यानन्द इत्यदि नामान्तरं निर्दिष्टम् । सोऽयं ज्ञानोत्तमः शृङ्गेरीगुरुपरम्परासूच्यनुसारं (919-953 A. D.) काले आसीदिति निर्णीयते ।
अयं ज्ञानोत्तमः नैष्कर्म्यसिद्धिः - इष्टसिद्धिव्याख्यातुः ज्ञानोत्तमाद्भिन्न इत्यपि ज्ञायते । अस्य द्वौ शिष्यौ । तयोरेकश्चित्सुखाचार्यः । अपरस्तु पञ्चपादिकाव्याख्यायाः ``तात्पर्यद्योतिन्याः" श्वेताश्वतरोपनिषद्दीपिकायाश्च कर्ता विज्ञानात्मा इति विज्ञानात्मकृतश्वेताश्वतरदीपिकायाः ज्ञायते ।
ज्ञानोत्तमस्य कालश्चित्सुखाचार्यात्किञ्चिदिव प्राचीनः । श्रीकण्ठशास्त्री तु स्वीये प्रबन्धे शृङ्गगिरिगुरुपरम्परां प्रदर्शयति । तत्रास्य कालः (910-953 A. D.) इति दृश्यते । स चायं भेदः शङ्करकालस्सप्तमशतकपूर्वार्ध इति सिद्धान्तं स्वीकृत्य वर्णित इति ज्ञायते ।
ज्ञानोत्तमकृताः ग्रन्था नोपलभ्यन्ते । परन्तु चित्सुखाचार्येण तत्वप्रदीपिकायां 392 तमे पुटे ``न्यायसुधा" निर्दिष्टा । प्रत्यक्स्वरूपाचार्येण ``आराध्यपादाः स्वगुरवः ज्ञानसिद्धिकाराः, तत्प्रणीतं वेदान्तप्रकरणं न्यायसुधा" इति व्याख्यातम् । अप्पय्यदीक्षितैश्च सिद्धान्तलेशसंग्रहे 269-270 तमे पुटे न्यायसुधा निर्दिष्टा । श्रीकण्ठशास्त्री ज्ञानसुधामपि निर्दिशति । तस्मात् ज्ञानसिद्धि न्यायसुधाज्ञानसुधाकारः ज्ञानोत्तमः द्वादशशतकीय इति निश्चीयते ।
१. न्यायसुधा Q.
२. ज्ञानसिद्धिः Q.
३. ज्ञानसुधा Q?

८०. ज्ञानोत्तममिश्रः III (1000-1200 A. D.)
कोऽयं ज्ञानोत्तममिश्रः । किं नैष्कर्म्यसिद्धिव्याख्याता ? मङ्गलग्रामवासी चोलदेशीयः ? आहोस्वित् गौडेश्वराचार्यापराभिधः गौडेषु वासी चित्सुखगुरुर्ज्ञानोत्तम् ? इति निर्णेतुं न शक्यते । परन्त्वमुद्रिते विद्याश्रीनामके ब्रह्मसूत्रभाष्यव्याख्याने ज्ञानघन आचार्यः निर्दिष्टः । ज्ञानोत्तमभट्टारक इत्यपि स्वस्य नामनिर्दिष्टम् । दासगुप्तेन तु ज्ञानोत्तममिश्र इति नाम निर्दिश्यते । नैष्कर्म्यसिद्धिव्याख्यायां ज्ञानोत्तममिश्र इति नाम निर्दिश्यते । नैष्कर्म्यसिद्धिव्याख्यायां ज्ञानोत्तमः मिश्रान्तशब्देन निर्दिष्टः । यद्येवं तर्हि मङ्गलग्रामवासी भवितुमर्हति । भट्टारकान्तनामश्रवणात् आनन्दबोधभट्टारकादिवत् अयमपि द्वादशशतकीयो भवितुमर्हति । 
१. विद्याश्रीः (R. 3783 MGOML)
अनादिरनन्तोऽयं सूत्रभाष्यव्याख्यात्मकः ग्रन्थ अमुद्रित मद्रासपुस्तकालये लभ्यते ।।
२. सुरेश्वरवार्तिकव्याखा । ग्रन्थोऽयं दासगुप्तमहाशयेन भारतीयदर्शनसाहित्येतिहासे H. I. P निर्दिष्टः । अस्य कालः दशमशतकमिति च निर्दिष्टम् ।।

८१. तत्वबोधभगवान् (1100-1200 A. D.)
``वन्दे वेदान्तशैलास्यं ज्ञानोत्तममृगोत्तमम्" इत्यादिना ज्ञानोत्तमः, ``श्री सत्यबोधशीतांशुमक्षीणकलमाश्रये" इत्यादिना सत्यबोधाचार्यः, ``तं वन्दे प्रज्ञारण्यमहामुनिम्" इति प्रज्ञारण्यश्चानेन नमस्कृताः । नैष्कर्म्यसिद्धिचन्द्रिकाकारेणापि सत्यबोधाचार्यः नमस्क्रियते । एवञ्चायं सत्यबोधाचार्यप्रशिष्यः ज्ञानोत्तमप्रज्ञाराण्यशिष्य इति निर्णेतुं शक्यते । एकादशद्वादशशतकमध्यवर्ती भवेदिति च निश्चीयते ।।
१. तत्वबोधः (R. 3344 MGOML) बौद्धमतं खण्डयन् अद्वैतं मण्डयन् शङ्कराचार्यं प्रमाणयत्ययं ग्रन्थः । अमुद्रितोऽयं सादिरनन्तः ग्रन्थः मद्रासराजकीय पुस्तकालये लभ्यते ।।

८२. तारकब्रह्माश्रमी (1700-1800 A. D.)
रामचन्द्राश्रमिणं नमस्यन्नयं तारकब्रह्माश्रमी दक्षिणदेशीयः रामचन्द्राश्रमिशिष्यस्सप्तदशशतकीयान्तवर्तीति निर्णीयते ।।
१. परिमलसङ्ग्रहः - (R. 2811 MGOML) अप्पय्यदीक्षितीयकल्पतरुव्याख्यापरिमलं संगृह्णाति । अमुद्रितोऽयं प्रथमाध्यायतृतीयपादान्तं मद्रासपुस्तकालये लभ्यते ।
२. तैत्तरीयोपनिषत्सारसंग्रहः । ग्रन्थोऽयमडयारपुस्तकालयस्थः ।।

८३. तारानाथशर्मा (1825-1900 A. D.)
कालिदाससार्वभौमभट्टाचार्यपुत्रोऽयं तारानाथशर्मा वात्स्यगोत्रजः वाराणसीवासीति ज्ञायते । अनेन ग्रन्थनिर्माणकालः अग्निनववाहेन्दुमितश्शकः (1793-1871 A. D.) इति निर्दिष्टः ।
१. सिद्धान्तबिन्दुसारः । मधुसूदनीयसिद्धान्तबिन्दुं संगृह्णात्ययं ग्रन्थः । सरस्वतीमुद्रणालये कल्कत्तायां मुद्रितश्च ।।

८४. ताराचरणशर्मा (1800-1900 A. D.)
आधुनिकोऽयं वायणसीवासी ताराचरणशर्मा । अनेन खण्डनखण्डखाद्यस्य टिप्पणीरूपः खण्डनपरिशिष्टाख्यः ग्रन्थः कृतः यश्च चौखाम्बायां मुद्रितः ।

८५. त्र्यम्बकभट्टः (1650-1750 A. D.)
त्र्यम्बकमखीति प्रसिद्धोऽयं भारद्वाजगोत्रोत्पन्नः बाबाजीयज्वनः पौत्रः गङ्गाधराध्वरिकृष्णाम्बिकयोः पुत्रः नरसिम्हयज्वनः कनीयान् भ्राता, परमार्थसारव्याख्यातू राघवानन्दस्य, गौडब्रह्मानन्दसरस्वत्याश्च शिष्यः शाहजीशरभोजीकाले चोलदेशवर्ती, पश्चात् मैसूर-पूनादिनगरेषु उवासेति ज्ञायते । अस्य शिष्यस्स्थाणुशास्त्रीति प्रसिद्धः घृतशौचदीपिकाकर्ता इति (R. 3854) मद्रासराजकीयपुस्तकात् ज्ञायते ।
१. अद्वैतवाक्यार्थः । अद्वैतप्रकरणापरनामायं ग्रन्थः कृष्णपुरसूच्यां (105) दृश्यते ।
२. अद्वैतसिद्धान्तवैजयन्ती । वाणीविलासमुद्रणालये (VVP) मुद्रितः ।।
३. तत्वनिरूपणम् । मैसूरसूच्यां दृश्यते ।
४. तत्वसंख्यानखण्डनम् । (29 G. 4 A.L.)
आनन्दतीर्थकृतस्य द्वैतसिद्धान्तपरतत्वसंख्यानस्य खण्डनपरोऽयं ग्रन्थ अडयार पुस्तकालये लभ्यते ।।
५. दृग्दृश्यसम्बन्धानुपपत्तिप्रकाशः । ग्रन्थोऽयं मैसूर पुस्तकालये लभ्यते ।।
६. प्रकृत्यधिकरणविचारः । ग्रन्थोऽयं मैसूरपुस्तकालयस्थः ।
७. प्रमाणतत्वम् । अयमपि पूर्वमिव ।
८. भाष्यानुप्रभा (41. C. 90. A. L.)
भाष्यभानुप्रभा इत्यपि नामान्तरमस्य श्रूयते । सूत्रभाष्यव्याख्यात्मकोऽयं ग्रन्थः भाष्यार्थसंग्राहकः क्रोडपत्रात्मकश्च अपूर्ण अमुद्रित अडयारपुस्तकालये लभ्यते । शृङ्गगिरिसूच्यां तस्य त्र्यम्बकवृत्तिरिति नामान्तरं दृश्यते ।।
९. शास्त्रारम्भसमर्थनम् । (21 G. 6 A. L.)
लघुशास्त्रारम्भमर्थनमिति नाम्ना च प्रसिद्धोऽयं ग्रन्थः ``वेदान्तानां प्रत्यगभिन्नब्रह्मण्येव तात्पर्यमिति निर्णायकन्यायग्रथनात्मकं ब्रह्ममीमांसाशास्त्रमारम्भणीयमिति" ब्रह्मानन्दसरस्वती प्रदर्शितदिशा समर्थयति । अडयारपुस्तकालयेऽमुद्रि तोऽस्ति ।।
१०. श्रुतिमतप्रकाशः (7810 B. R. d.)
११. श्रुतिमतोद्योतः । ग्रन्थोऽयं प्रथमद्वितीयमिथ्यात्वं सपरिकरं परिष्करोति । जगन्मिथ्यात्वानुमाने पक्षतावच्छेदकञ्च नव्यतर्कशैल्यां विशदयति । ग्रन्थोऽयं श्रीविद्याविलासमुद्रणालये कुम्भघोणे मुद्रितः । अस्य कामाक्षीकृता टिप्पणी च मुद्रिता ।।
१२. श्रुतिमतानुमानोपपत्तिः । (R. 2202 MGOML) ग्रन्थोऽयममुद्रितस्तेलुगुलिप्यां मद्रासपुस्तकालये लभ्यते ।।
१३. श्रुतिरत्नप्रकाशः । ग्रन्थोऽयं मिथ्यात्वसाधनानुमाननिरूपणपूर्वकं अविद्यां साधयति । दृश्यत्वपरिच्छिन्नत्वहेतू विचार्य दृश्यत्वहेतुं निर्दोषं साधयति । ईक्षत्यधिकरणार्थं च नव्यतर्कशैल्या वर्णयति । ग्रन्थोऽयं श्रीविद्याविलासमुद्रणालये कुम्भघोणे मुद्रितः । अस्याःटिप्पणी कामाक्षीकृता च वर्तते । बरोडास्थश्रुतिमतप्रकाशात् (7810 B. R. d.) भिन्नो वा उत नेति न ज्ञायते । वाल्मीकिरामायणव्याख्या धर्माकूतोऽप्यनेन कृतः ।।

८६. त्यागराजः माकिं (1750-1850 A. D.)
त्यागराजोऽयं काश्यपगोत्रजः आनन्दनाथशिष्य आन्ध्रदेशवसीति एतत्कृतग्रन्थदर्शनाज् ज्ञायते । आन्ध्रदेशजोऽपि 
``त्यागेशपदानन्दसमुद्रे भानुहिमांश्वोर्योगविशेषे ।
काले स्नात्वा पूतो भव रे न पुनर्जननं न पुरर्मरणम् ।।"
इति दर्शनात् त्यागेशभक्तिप्रदर्शनाच्च दक्षिणद्रविडदेशवासी स्याद्वेति संशीयते ।
मन्नार्गुडिवासिनस्त्यागराजदीक्षितादयं भिन्नः । अदसीयेषु बहुषु ग्रन्थेषु चत्वर एबाद्वैतपराः ।
१. उपदेशशिखामणिः । (7745 TSML)
शङ्कराचार्यकृतमोहमुद्गरवत् (भजगोविन्दम्) तदीयसंख्या तदीयवृत्तेनैव निर्मितश्शङ्कराद्वैतमतप्रदर्शकः प्रकरणग्रन्थः । ग्रन्थोऽयं अडयारपुस्तकालये सरस्वतीमहालये च लभ्यते । सरस्वतीमहालयवर्णनात्मकग्रन्थसूच्यास्त्रयोदशतमे भागे(DC. TSML Vol. XIII) मुद्रितश्च ।
२. पञ्चकोशविमर्शिनी (9 F. 43. AL)
भुजङ्गप्रयाताख्यवर्णछन्दसा बद्धैश्श्लोकैः त्यागराजात्मकपरमात्मस्वरूपनिरूपणपरया स्तुतिपद्धत्या अन्नमयादिपञ्चकोशान् वर्णयन् प्रकरणग्रन्थतामर्हति । अडयारपुस्तकालये लभ्यते च ।
३. स्वात्मस्फूर्तिविलासः सव्याख्यः । (9. F. 43. AL.)
सप्तत्रिंशद्भिश्श्लोकैः पूर्णोऽयं ग्रन्थश्शङ्कराचार्यकृतां स्वात्मप्रकाशिकां सदाशिवब्रह्मकृतं स्वानुभूतिप्रकाशञ्चानुकुर्वन् सोदाहरणं जीवब्रह्मणोरैक्यं प्रतिपादयति । ग्रन्थोऽयं अडयारपुस्तकालये सव्याख्याः लभ्यते ।
एवं शिवमीडेस्तोत्रम्, शान्तिस्तवः, नवमल्लिकास्तवः राजराजेश्वरीस्तवः, आर्यापञ्चदशी धर्माम्बिकास्तवः, शिवात्मस्फूर्तिविलासः, नवाक्षरीस्तोत्रम्, संवित्शतकम् इति ग्रन्थाश्च कृताः ।

८७. त्यागराजदीक्षितः (2000 A. D.)
दक्षिणदेशजोऽयं त्रिशिरपूर (तिरुच्चि) समीपस्थ-अम्मङ्गुडिग्रामवासौ विंशतिशतकीयः । अनेन ग्रन्थः (1953 A. D.) काले रचितः ।
१. आत्मलाभः (अद्वैतसारः) पद्यबद्धोऽयं लघुप्रकरणग्रन्थः सङ्कलनात्मकः नवसालपुरे कोवाप्रेटिवमुद्रणालये मुद्रितः ।

८८. दिवाकरः (1750-1850 A. D.)
बोधसारकर्तुर्नरहरेशिशष्योऽयं दाक्षिणात्य इति ज्ञायते । अनेन स्वग्रन्थे गजत्रिमुनिभूमिते शके (1738 श 1816 A. D.) बोधसारव्याख्या कृतेति निर्दिश्यते ।
१. बोधसारव्याख्या । चौखाम्बमुद्रणालये मुद्रिता ।।

८९. द्विजेन्द्रलालपुरकायस्थः (2000 A. D.)
जयपुरवासी दुलालानन्दपुत्र धीरेन्द्रकनिष्ठभ्राता जयकुमारतर्कवेदान्तसद्दर्शनतीर्थस्य पट्टाभिरामशास्त्रिणश्च शिष्योऽयं विंशतिशतकीयः । अनेन अद्वैतामृतसाराख्यः ग्रन्थः (1954 A. D.) काले निर्मितः ।
१. अद्वैतामृतसारः । लघुप्रकरणग्रन्थोऽयं पद्यबद्धः जयपुरे मुद्रितः ।।

९०. देवेन्द्रसरस्वती (1500-1600 A. D.)
नृसिम्हाश्रमिसामयिकोऽयं स्वानुभूतिप्रकाशाख्यप्रकरणग्रन्थकर्ता देवेन्द्रसरस्वतीति दासगुप्तः ।
१. स्वानुभूतिप्रकाशः । दासगुप्तेन (H. I. P. Vol. II Page 55) निर्दिष्टोऽयं ग्रन्थः ।।

९१. धनपतिसूरिः (1750-1850 A. D.)
पञ्जाबन्तर्गतरावलपिण्डीनाम्ना प्रसिद्धनगरे उत्पन्नोऽयं रामकुमारसूरिपुत्रः सारस्वतब्राह्मणकुलोत्पन्नः प्रत्यक्तत्वचिन्तामणिकारस्य सदानन्दव्यासवरस्य जामाता बालगोपालतीर्थस्य शिष्यः वाराणसीसंस्कृतकलाशालाया वेदान्ताध्यापकश्चासीत् । अस्याध्यापनकालः (1811 A. D.) इति ज्ञायते ।
१. अर्थदीपिका - वेदान्तपरिभाषाव्याख्यात्मकोऽयं ग्रन्थः मुद्रितः ।
२. भाष्योत्कर्षदीपिका - भगवद्गीताव्याख्यारूपोऽयं ग्रन्थः मुद्रितः ।
३. माधवीयशङ्करदिग्विजयटीका ।
४. शङ्करविजयदुन्दुभी । सदानन्दीयशङ्करदिग्विजयसारव्याख्या । विद्यारत्नाकरः (2458-59 IOL) रासपञ्चाध्यायीव्याख्या च कृता ।

९२. धर्मदत्तबच्चाशर्मा (1850-1920 A.D.)
मैथिलोऽयं बच्चाशर्मेति प्रसिद्धः धर्मदत्त एकोनर्विशतिशतकीय इति ज्ञायते । अनेन मधुसूदनसरस्वतीकृतायाः भगवद्गीताव्याख्यायाः गूढार्थतत्वालोक इति व्याख्या कृता निर्णयसागरमुद्रणालये मुद्रिता च ।।

९३. धर्मभट्टः (1650-1750 A. D.)
तिरुमलाचार्यसूनुः मुकुन्दाश्रमस्य मन्त्रवादीश्रीरामचन्द्रार्यस्य च शिष्योऽयं धर्मभट्टः अन्नम्भट्टादिवत् सप्तदशशतकापरार्वकालिकस्स्यादिति ज्ञायते । अन्नम्भट्टपितुः धर्मभट्टपितुश्च नामसादृश्यमपि अवघेयार्हो विषयः ।।
१. ब्रह्मामृतवर्षिणी ।। सूत्रवृत्तिग्रन्थोऽयं तेलुगुलिप्यां मुद्रितः । चौखाम्बमुद्रिते ग्रन्थे मुकुन्दगोविन्दशिष्य रामानन्दोऽस्य कर्तेति दृश्यते । आनन्दाश्रममुद्रिते तु ग्रन्थे रामकिङ्करधर्मोऽस्य कर्तेति दृश्यते । ग्रन्थेषु भेदोऽपि न दृश्यते । इण्डिया आफीस ग्रन्थालयस्थे (2268 IOL) ग्रन्थे तु स्वयप्रकाशानन्दशिष्यस्सदाशिवानन्दोऽस्य कर्तेति दृश्यते । अडयारपुस्तकालयस्थे तु पुस्तके (10 E. 41 A. L.) रामकिङ्करधर्म इति दृश्यते । मद्रासराजकीयपुस्तकालयस्थे तु D. 4689 MGOML ग्रन्थे धर्मभट्टः कर्तेति दृश्यते ।।

९४. धर्मय्यदीक्षितः (1550-1625 A. D.)
वेङ्कटेशदीक्षितपुत्रः मित्रमिश्रशिष्योऽयं धर्मय्यदीक्षितः वाराणसीवासी षोडशशतकापरार्धारब्धसप्तदशशतकपूर्वार्धान्ते काले उवासेति निश्चीयते । अत्र कारणम् - मुद्रितपुस्तके श्रीमत्परशुरामात्मजमित्रमिश्रप्रेरितस्य वेङ्कटभट्टसूनुधर्मय्यदीक्षितकृतौ इति दृश्यते । मित्रमिश्रेण च वीरमित्रोदयनामा ग्रन्थः प्रणीतः । स च ग्रन्थः काशीराजवंशोत्पन्नस्य बुण्डेलवंशजस्य वीरसिम्हनाम्नः काशीमूपालस्य प्रेरणया कृत इति ज्ञायते । वीरसिम्हश्च (1556-1603 A.D.) पर्यन्तं भारतराज्यं अधितिष्ठत अकबरस्यामात्येन अब्दुलफसलाख्येन हत इति इतिहासः । तस्मात् वीरमित्रोदयकर्तुर्मिश्रमित्रस्य शिष्योऽयं धर्मय्यस्सप्तदशशतकपूर्वार्धवासीति ज्ञायते । अदसीयः ग्रन्थः सरस्वतीभवनमालायां (S. B. S. 34) मुद्रितः ।
१. अद्वैतविद्यातिलकदर्पणम् ।  समरपुङ्खवीयाद्वैतविद्यातिलकव्याख्यात्मकोऽयं ग्रन्थः ।

९५. नटेशार्याः (1850-1910 A. D.)
भारद्वाजगोत्रजोऽयं मुडिकोण्डान् ग्रामवासी चोलदेशीयः नटेशार्यः रामस्वामिनागलक्ष्म्योः पुत्रः तिरुविशनल्लृररामसुब्रह्मण्यशास्त्रिशिष्य एकोनविंशतिशतकीय इति निश्चयः । 
१. अद्वैतसरणिः । वेङ्कटमणार्यकृतस्य चन्द्रिकाप्रकाशप्रसराख्यग्रन्थस्य खण्डनरूपोऽयं ग्रन्थः बालमनोरमामुद्रणालये मुद्रितः ।

९६. नरकण्ठीरवशास्त्री (1850-1950 A. D.)
अलङ्कारतर्कवेदान्तविचक्षणोऽयं वेङ्कटेश्वरसंस्कृतपाठशालाप्राध्यापक आसीत् । अनेनोमामहेश्वरकृतौ तत्वचन्द्रिका विरोधवरूधिनीग्रन्थौ प्रकाशितौ । किरणावलीव्याख्योपेतौ महावाक्यरत्नावलीतप्तचक्राङ्कविध्वंसावुभावपि ग्रन्थौ प्रकाशितौ ।
१. व्यासतात्पर्यनिर्णयदीपिका । अय्यण्णादीक्षितकृतव्यासतात्पर्यनिर्णयव्याख्या मुद्रिता च ।

९७. नरसिम्हभट्टः (1700-1800 A. D.)
रघुनाथभट्टपुत्रोऽयं नरसिम्हभट्टः रामभद्राश्रमनागेश्वरयोश्शिष्यः किम्मिरि (खिमुण्डि) वंशजस्य जगन्नाथनृपतेस्सामयिकः मिथिलावासी अष्टादशशतकीय इति ज्ञायते ।
१. अद्वैतचन्द्रिका । नृसिम्हाश्रमीयभेदधिक्कारव्याख्यात्मकोऽयं ग्रन्थः दासगुप्तेन निर्दिष्टः ।
२. ईशावास्यटीका । (481 C. C. P. B.) 
३. आनन्ददायिनीति ग्रन्थ अदसीय इति प्रसिद्धिः । परं नास्य कृतिरिति ज्ञायते ।

९८. नरहरिः (1725-1352 A. D.)
दाक्षिणात्योऽयं नरहरिः । अस्य शिष्यः दिवाकरनामा । अस्य शिष्येण बोधसारव्याख्या (1738 श 1875 A. D.) काले कृता । तस्मादयं नरहरिरपि अष्टादशशतकीयः ।
१. बोधसारः । विपुलकायेऽस्मिन् ग्रन्थे गुरुमहिमा मुनिचर्या जीवन्मुक्तलक्षणान्युपवर्णितानि । वाराणस्यां मुद्रितः ।

९९. नल्लासुधीः (1700-1800 A. D.)
नल्लाकविः, नल्लाबुधः, नल्लादीक्षितः, भूमिनाथ इत्यादिनामभिः प्रसिद्धोऽयं सुभद्रापरिणयनाटककर्ता नल्लासुधीः कौशिकगोत्रजः बालचन्द्रदीक्षितपुत्रः चोलदेशीयकण्डरमाणिक्कग्रामवासी परमशिवेन्द्रप्रशिष्यः सदाशिवब्रह्मेन्द्र-रामभद्र-दीक्षितयोः शिष्यः रत्नगिरीशदीक्षितपुत्रस्य वैद्यनाथदीक्षितस्यापि शिष्यः श्रीधरवेङ्कटेशशास्त्रिणां सामयिकः रामभद्रमखिवन्धुः सप्तदशशतकमध्यवर्तीति निर्णयः ।
१. अद्वैतरसमञ्जरी । ग्रन्थेऽत्र 45 पद्यैः सरसमधुरैश्शब्दैः नातिदीर्धैर्नातिहृस्वैश्चसमस्तशब्दैः बहुभिरुदाहरणैरर्थान्तरन्यासपूर्णैश्च वाक्यकदम्बैरद्वैतसिद्धान्ताः प्रतिपाद्यन्ते । ग्रन्थोऽयं वाणीविलासमुद्रणालये मुद्रितः । अस्य व्याख्या परिमलाख्या मूलकृतैव कृता । अन्या च व्याख्या अनन्तशयनपुस्तकालये लभ्यते ।
चित्तवृत्तिकल्याणम्, धर्मविजयचम्पूः, पदमञ्जरी, जीवन्मुक्तिकल्याणम्, शृङ्गारसर्वस्वभाणः, सुभद्रापरिणयनाटकञ्चानेन कृतानीति ज्ञायन्ते ।

१००. नागेशभट्टः (1674-1754 A. D.)
महाराष्ट्रविप्रकुलोत्पन्नोऽय शिवभट्टसतीदेव्योः पुत्रः भट्टोजिदीक्षितपौत्रस्य हरिदीक्षितस्य शिष्यः वैद्यनाथपायुकुण्डापराख्यस्य बालशर्मणश्च गुरुः शृङ्गिवेरपुराधीशाद्रामतो लब्धजीवक अपरे वयसि प्राप्तसन्यासः नागेशभट्ट इति ज्ञायते ।
रामराजस्य महानसे पाचनविभागे सूदपदव्यां नियुक्तोऽयं अदसीयपाण्डित्यमजानता राजपुरुषेण कदाचित् भुञ्जानेषु पण्डितेषु केनाप्यव्युत्पन्नेन पण्डितम्मानिना अज्ञात``पचेर्ब"सूत्रेणोक्तं पक्वार्थकं पक्तपदं श्रुत्वा अन्नपर्यवेषकेणानेन नागेशभट्टेन तडितस्स पण्डितः । ज्ञातवृत्तान्तो राजा नागेशभट्टं महान्तं विदुषं पात्वा सञ्जातपश्चात्तापः सभापण्डितप्रवरमेनमकार्षीदिति कथापि सम्प्रदायसमागता श्रूयते ।
१. वेदान्तभाष्यप्रतिपाद्यानि । अस्यैव वेदान्तभाष्यप्रदीपोद्योत इति नामान्तरं जयपुरपोटीखानासूच्यां दृश्यते । सूत्रवृत्तिरूपः उज्जयिनीसूच्यां जयपुरसूच्याञ्च लभ्यते । परिभाषेन्दुशेखरादिश्चानेन कृतः ।।

१०१. नानादीक्षितः (1600 A. D.)
प्रकाशानन्दनृसिम्हराघवेन्द्राणां शिष्योऽयं नानादीक्षित अधिकाश्युपविश्वेशमिति स्वग्रन्थे निर्दिशति । तस्माद्वाराणसीवासी प्रकाशानन्दशिष्योऽयं षोडशशतकीय इति निर्णीयते ।
१. सिद्धान्तदीपिका । प्रकाशानन्दकृतसिद्धान्तमुक्तावल्याः व्याख्यात्मकोऽयं ग्रन्थः जीवानन्दविद्यासागरमुद्रणालये कल्कत्तायां मुद्रितः ।

१०२. नारायणः (1300 A. D.)
``आनन्दशैलचरणाम्बुजष्ट्पद" इत्यात्मानं प्रकटीकुर्वन्नयं आनन्दशैल (आनन्दगिरि) शिष्योऽयमिति प्रतिभाति ।
१. प्रपञ्चसारार्थदीपः । ग्रन्थोऽयं (R. 3451 MGOML) लभ्यते ।

१०३.
रामतीर्थशिष्योऽयं वृषक्षेत्रवासी नारायणप्रियः सप्तदशशतकीयः ।
१. कैवल्यदीपिकाव्याख्या - ``स्नेहः" (R. 2934 MGOML)

१०४. नारायणतीर्थः (1600-1700 A. D.)
रामगोविन्द (गोविन्दानन्द) वासुदेवतीर्थयोश्शिष्योऽयं नारायणतीर्थः लघुचन्द्रिकाकार गौडव्रह्मानन्दगुरुस्सप्तदशशतकीय इति निश्चीयते । कृष्णलीलातरङ्गिणीकाराद्भिन्नोऽयम् । अनेन सिद्धान्तविन्दोर्व्याख्या कृतेति P. P. शास्त्रिण (7549 DC. TSML Vol XIII.) ग्रन्थे प्रवदन्ति ।
१. अद्वैतरत्नाकरः । (B. 1028) मैसूरसूच्यां दृश्यते ।
२. ब्रह्मसूत्रवृत्तिः (12717) बरोडासूच्यां दृश्यते ।
३. लघुव्याख्या । (B. S. S. 65) सिद्धान्तबिन्दुलघुव्याख्यात्मकोऽयं ग्रन्थः वाराणसीग्रन्थमालायां मुद्रितः ।
४. वेदान्तमन्दाकिनी (65 DC. Anup) माधवसरस्वतीकृतन्यायचूडामणिव्याख्यात्मकोऽयं ग्रन्थ अनृपसूच्यां दृश्यते ।
५. अद्वैतसुधा (3820 B. R. d.)
६. शारीरकमीमांसाभाष्यवार्तिकम् (A. S. I. C. S. S. I) गद्यमयोऽयं भाष्यस्य व्याख्यात्मकः वार्तिकग्रन्थः । वार्तिकशब्देन व्यवहृतोऽपि भाष्ये दुरुक्तचिन्तनं नास्ति । ग्रन्थोऽयमाशुतोषग्रन्थमालायां मुद्रितः ।
७. अद्वैतामृतकन्दम् । ग्रन्थोऽयं शारीरकमीमांसाभाष्यवार्तिक 19 तमे पुटेनिर्दिष्टः ।।
सिद्धान्तविन्दुमहद्व्याख्या, तत्वचन्द्रः तत्वत्रयनिरूपणम् न्यायकुसुमाञ्जलिकारिकाव्याख्या, भक्तिचन्द्रिका, भक्त्याधिकरणमाला, योगचन्द्रिका, योगसूत्रवृत्तिर्गूढार्थद्योतनिका, वेदस्तवटीका, वेदान्तविभावना सव्याख्या, सांख्यचन्द्रः, सांख्यकारिकाव्याख्या, सांख्यतत्वकौमुदीव्याख्या, दीधितीटीका, भाषापरिच्छेदटीकाश्चादसीयाः ग्रन्था इति श्रूयन्ते ।।

१०५. नारायणतीर्थः (1700 A. D.)
ब्रह्मविद्यातरङ्गिणीति कश्चन प्रकरणग्रन्थ अडयार पुस्तकालये लभ्यते । यद्यपि तत्र ग्रन्थकर्तुर्नाम न दृश्यते तथापि ग्रन्थस्यादिमश्लोकानां परिशीलनेन कृष्णलीलातरङ्गिणीकारः नारायणतीर्थ एवास्यापि कर्ता स्यादित्यूह्यते । पूर्वमीमांसायामपि भाट्टभाषाप्रकाशिका अनेन कृता विद्यते । नारायणतीर्थश्शिवरामानन्दतीर्थशिष्य आन्ध्रदेशजोऽपि दक्षिणदेशीयः कावेरीतीरस्थनडुक्कावेरीग्रामे वरहूरर्ग्रामे च उवासेति ज्ञायते । अस्य काल (1700 A. D.) इति कृष्णमाचार्यमतम् । त्यागराजशास्त्रिकृता व्याख्याप्यस्यास्ति ।

१०६. नारायणाश्रमः (1526-1600 A. D.)
जगन्नाथाश्रमगीर्वाणेन्द्रसरस्वत्योः प्रश्शिष्योऽयं अद्वैतदीपिकाकारनृसिम्हाश्रमिणश्शिष्योऽयं नारायणाश्रम अप्पय्यदीक्षितसामयिकः दक्षिणदेशवासी षोडशशतकीय इति ज्ञायते ।
१. अद्वैतदीपिकाविवरणम् । नृसिम्हाश्रमीयाद्वैतव्याख्यात्मकोऽयं ग्रन्थः पण्डितग्रन्थमालायां मुद्रितः ।।
२. अद्वैतसिद्धान्तसारसंग्रहः । (N.S.P.)
३. निगमान्तार्थचन्द्रिका (22 D. 86 A.L) प्रकरणग्रन्थोऽयं अडयारपुस्तकालये वेङ्कटेश्वरपुस्तकालये च लभ्यते ।।
४. भेदधिक्कारसत्क्रिया । (B.S.S 86, 92) नृसिम्हाश्रमीयभेदधिकारव्याख्या ।।

१०७. नारायणाश्रमी (1350 A. D. कालादर्वाचीनः)
एतेन प्रायशस्सर्वासामप्युपनिषदां दीपिकाः कृताः । कासुचित्सूपनिषत्सु दीपिका नैव कृतेति ज्ञायते । याश्चोपनिषदः शङ्कराचार्यैर्व्याकृताः तासां व्याख्यानावसरे नारायणाश्रमिणा शङ्करोक्त्युपजीविनेति कथ्यते याश्च शङ्करेणाव्याकृताः तासु श्रुतिमात्रोपजीविनेति कथ्यते ।। एतदीयायां माण्डूक्योपनिषद्दीपिकायां ``श्रीनाथात्मजभट्टरत्नाकरसृनुनारायणविरचिताद्वैताख्यप्रकरणे" इति दर्शनात् श्रीनाथपौत्रः रत्नाकरभट्टपुत्रोऽयं नारायणाश्रम इति ज्ञायते । अनेनानन्दगिरिः आरुणिकोपनिषद्दीपिकायां शङ्करानन्दश्च निर्दिश्येते । तस्मादानन्दात्मशिष्य शङ्करानन्दादर्वाक्तन स्सामयिको वा स्यादिति निश्चयः ।
१. उपनिषदां दीपिकाः ।।
अथर्वशिर-अथर्वशिख-अमृतबिन्दु-आत्मप्रबोध-आत्म-आरुणिक-काठक-कैवल्यक्षुरिक-गर्भ-चूलिक-जाबाल-तैत्तरीय-नारायण-नृसिम्हतापनी-प्रश्न-प्राणाग्निहोत्र-ब्रह्म-विद्या-ब्रह्म-मह - माण्डूक्य-मुण्डक - रामपूर्वतापिनी श्वेताश्वतर - षटचक्र-हंसोपनिषदां दीपिकाः कृताः । माण्डूक्यदीपिका अमुद्रिता सरस्वतीमहालये जयपुरसूच्याञ्च लभ्यते ।

१०८. नित्यानन्दाश्रमः (1620 A. D.)
पुरुषोत्तमाश्रमशिष्योऽय नित्यानन्दाश्रमः । उज्जयिनीस्थे छान्दोग्योपनिषन्मिताक्षराग्रन्थे (1749 सं 1690 A. D.) इति कालः दृश्यते । जयपुरपोटीखानापुस्तकालयस्थे तु ग्रन्थे 1976 इति काल निर्दिष्टः । स किं प्रतिलिपिकाल उत ग्रन्थनिर्माणकाल इति निर्णेतुं न शक्यते ।।
१. अद्वैततत्वदीपः (7534 TSML) प्रकरणग्रन्थोऽयममुद्रित अपूर्णस्सरस्वती महालये लभ्यते ।
२. मिताक्षरा ।। छान्दोग्योपनिषदां व्याख्यात्मकः ग्रन्थः बाम्बेसंस्कृतमुद्रणालये मुद्रितः ।।
३. मिताक्षरा ।। बृहदारण्यकोपनिषदां व्याख्यात्मकोऽयं ग्रन्थः आनन्दाश्रममुद्रणालये (A. S. S. 31) मुद्रितः ।।

१०९. निश्चलदासस्वामी (1800-1900 A. D.)
वारणासीवास्ययं निश्चलदासः विद्यारण्यकृतपञ्चदश्याः व्याक्यां रचयामास । ग्रन्थोऽयं दासगुप्तेन निर्दिष्टः ।
१. वृत्तिप्रभाकरः । विद्यारण्यीयपञ्चदशीव्याख्यात्मकोऽयं ग्रन्थः मुद्रितः ।।

११०. नीलकण्ठतीर्थः (1775-1875 A. D.)
``बालाभिधगुरुकृपयेति दर्शनात् बालतीर्थशिष्योऽयं अद्वैतसिद्धिकारं मधुसूदनसरस्वतीं, स्वाराज्यसिद्धिकारं गङ्गाधरेन्द्रञ्च प्रमाणयन् अष्टादशशतकवासीति निश्चीयते । केरलदेशानभिष्टुवन् केरलदेशवासितां प्रतिपादयति । अस्य शिष्या आत्मपञ्चककर्त्री आत्मयोगिनीनाम्नी सन्यासिनीति ज्ञायते ।।"
१. अद्वैतपारिजातम् । प्रकरणग्रन्थोऽयं 262 पद्यैः जीवन्मुक्तलक्षणं विशदयति । ग्रन्थोऽयं निर्णयसागरे मुद्रितः ।
२. गीतार्थप्रकाशः । (D. 2081 MGOML) लभ्यते । 
३. चित्सुधार्या । स्वाराज्यसर्वस्वापरनामायं ग्रन्थः सङ्ग्रहेणासत्कार्यवादं नास्तिकमतं, कपिलकाणादाद्यनीश्वरवादिमतं सौगतमतञ्च खण्डयन् अद्वैतमतं साधयति । ग्रन्थोऽयं पालकाडनगरे मुद्रितः ।
४. वेदान्तकतकः । 3200 पद्यैः पूर्णोऽयं ग्रन्थ अतिशिथिलः भारतकार्यालयपुस्तकालये (2402 IOL) लभ्यते । आत्मपञ्चकनामा ग्रन्थः अनेन रचितः । पालघाटमुद्रणालये मुद्रितश्च ।
अष्टाक्षरस्तोत्रम्, आत्मादर्शः, प्रश्नोत्तरमञ्जरी, श्रीशिवामृतम्, सनत्सुजातीयव्याख्या ; हरिषड्वदनम्,  हरिनवकम्, हरिभक्तिमकरनन्दः, इति ग्रन्थ अनेन विरचिताः ।

१११. नीलमेघशास्त्री (1850-1910 A. D.)
चोलदेशजोऽयं तिरुविनशल्लूरग्रामाभिजनः रामसुब्रह्मण्यशास्त्रिशिष्यः नीलमेघशास्त्रीति ज्ञायते ।
१. वेदान्तनवमालिका । ब्रह्मसूत्रलघुवृत्यपरनामायं ग्रन्थ ओरियण्टल पब्लिषिङ्हाउस मद्रासनगरे मुद्रितः ।

११२. नृसिम्हसरस्वती (1551-1650 A. D.)
स्वग्रन्थेषु कृष्णानन्दं रामानन्दञ्च नमस्कुर्वाणोऽयं नृसिम्हभारत्यपराभिधः वाराणसीवासीति वाराणस्यां (1558 A. D.) काले स्वीयां सुबोघिर्नी समापितवानिति च ज्ञायते । एवञ्चास्य कालष्षोडशसप्तदशशतकमध्य इति ज्ञायते ।
१. सुबोधिनी । सदानन्दकृतवेदान्तसारव्याख्यात्मकोऽयं ग्रन्थः विक्टेरियामुद्रणालये बाम्बेनगरे, सरस्वतीमुद्रणालये कल्कत्तायाञ्च मुद्रितः ।
२. वेदान्तडिण्डिमः । ग्रन्थस्यास्य प्रतिश्लोकं प्रतिचतुर्थचरणं ``इति वेदान्तडिण्डिम" इति दर्शनात् अस्याख्या अन्वर्था भवति । ग्रन्थोऽयं कल्कत्तायां मुद्रितः ।
३. विवेकमुकुरः । (22. D. 88. AL) देहत्रयं पञ्चकोशान् विश्वतैजसप्राज्ञांश्च विवेचयन्नयं प्रकरणग्रन्थ अमुद्रितः पूर्णश्च अडयारपुस्तकालये लभ्यते । योगपञ्चाशिकापि अनेन कृतेति ज्ञायते ।

११३. परमशिवेन्द्रसरस्वती (1600-1700 A. D.)
अभिनवनारायणेन्द्रसरस्वतीशिष्योऽयं परमशिवेन्द्रसरस्वती प्रसिद्धस्य सदाशिवब्रह्मेन्द्रस्य गुरुः षोडशसप्तदशशतकीयः त्र्यम्बकमखिशङ्करनारायणमखिनोस्सामयिक इति च निर्णीयते ।
१. दहरविद्याप्रकाशः । ब्रह्मसूत्रताण्डिश्रुत्यादिषु उपवर्णिता शाङ्करभाष्ये दहराधिकरणे प्रतिपादितेयं दहरविद्या ग्रन्थेऽस्मिन् साकल्येन निर्दिष्टा । र्त्रिशद्भिश्श्लोकैस्सङ्गृहीता । ग्रन्थोऽयं बालमनोरमा मुद्रणालये (B. M. P. S. 5) मुद्रितः ।
२. वेदान्तनामरत्नसहस्रव्याख्या । (7592 TSML)
स्वरूपानुसन्धानापरनामायं ग्रन्थ उपनिषत्प्रतिपादितानां ब्रह्मस्वरूपवाचकानां शब्दानां व्युत्पत्तिप्रदर्शनपर अमुद्रितस्सरस्वतीमहालये सव्याख्यः मद्रासराजकीयपुस्तकालये च लभ्यते ।
वेदसारशिवसहस्रनामव्याख्या, शिवगीतातात्पर्यप्रकाशिका चानेन कृतेति ज्ञायते ।

११४. परमानन्दतीर्थः (1300-1400 A. D.)
विद्यातीर्थप्रशिष्यः भारतीतीर्थशिष्यश्चायं परमानन्दतीर्थश्चतुर्दशशतकीय इति निर्णीयते ।
१. अवधूतगीताटीका । (10 F. 15. AL) अनन्तशयन-अडयार-मैसूर-मद्रास राजकीय-पुस्तकालयेषु लभ्यते ।
२. उत्तरगीताटीका । ग्रन्थोऽयं मैसूरपुस्तकालयेे लभ्यते ।
३. ब्रह्मविद्यासुधार्णवः (7560 TSML) प्रकरणग्रन्थोऽयं अडयारसरस्वतीमहालययोर्लभ्यते ।
४. वेदान्तशिक्षा, चिदानन्दब्रह्मविलासाख्या सूत्रवृत्तिश्च कृतेति ज्ञायते ।

११५. परमेश्वरभारती (1400 A. D.)
भारतीतीर्थशिष्योऽयं परमेश्वभारती चतुर्दशशतकीयः । वैय्यासिकसूत्रोपन्यास अनेन कृतः । तिरुवनन्तपुरपुस्तकालये लभ्यते । निजतत्वामृतसाख्य अपरो ग्रन्थ अडयारपुस्तकालये लभ्यते ।

११६. परमेष्ठिगुरुः (1725-1800 A. D.)
रामाश्रमापरनामायं नारायणश्रमगुरुः माधवाश्रमप्राचार्यः अष्टादशशतकीय इति माधवाश्रमप्राचार्यः अष्टादशशतकीय इति माधवाश्रमकृतात् स्वानुभवादर्शाज् ज्ञायत् । अनेन वेदान्तभूषणाख्यः ग्रन्थः कृतः । स च मध्वमतध्वान्तदिवाकराख्ये अमुद्रिते अडयारपुस्तकलयस्थे (I. A. 18. AL) निर्दिष्टः
१. वेदान्तभूषणम् Q.

११७. पाण्डुरङ्गपण्डितः (1750-1850 A. D.)
नारायणपण्डितपुत्रोऽयं पाण्डुरङ्गपण्डितः वाराणसीवासीति ज्ञायते । अनेन द्वौ ग्रन्थौ रचितौ ।
१. अद्वैतजलजातः - (11216 B. R. D.)
२. पञ्चरत्नप्रकाशः - (III. B. 122. R. A. S. B.) शाङ्करपञ्चरत्नव्याख्यात्मकोऽयं ग्रन्थः रायलासियाटिक सोसाइटि कल्कत्तासूच्यां दृश्यते ।

११८. पुरुषोत्तममिश्रः (1600-1700 A. D.)
कृष्णतीर्थप्रशिष्यः रामतीर्थशिष्यश्चायं पुरुषोत्तममिश्रस्सप्तदशशतकीय इति निर्णीयते । अग्निचित्पुरुषोत्तमाद्भिन्नो वा न वेति न ज्ञातुं पार्यते ।
१. सुबोधिनी । (2061 DC BUL Vol. II) संक्षेपशारीरकव्याख्यात्मकोऽयं ग्रन्थ अमुद्रित बम्बई युनिवर्सिटिपुस्तकालये लभ्यते । वाराणसीसंस्कृतविश्वविद्यालये (No. 28282 D. C.) उदयपुरसूच्यां (P. 160) च दृश्यते ।

११९. पुरुषोत्तमसरस्वती (1600-1700 A.D.)
श्रीधरसरस्वतीशिष्य अद्वैतसिद्धिकारमधुसूदनसरस्वत्याश्च शिष्य पुरुषोत्तसरस्वती षोडशसप्तदशशतकीय इति निर्णीयते । श्रीधरसरस्वती दीक्षागुरुः, मधुसूदनसरस्वती विद्यागुरुरिति विशेषः ।
१. बिन्दुसन्दीपनम् । सिद्धान्तबिन्दुव्याख्यात्मकोऽयं ग्रन्थः गुजरातिमुद्रणालये बाम्बेनगरे गैक्वार्डओरियण्टलग्रन्थमालायां (G.O.R.I.S.) च मुद्रितः ।।
२. अद्वैतसिद्धिसाधकः । बरोडापुस्तकालये लभ्यते ।
३. उपाघिखण्डनम् । (R. 3211 B. MGOML)

१२०. पूर्णानन्दः [पूर्णप्रकाशः] (1650-1750 A. D.)
ब्रह्मविद्याभरणकारस्याद्वैतानन्दसरस्वत्याः प्रशिष्यः पुरुषोत्तमानन्दसरस्वत्याश्शिष्योऽयं पूर्णप्रकाशानन्दापराभिधः पूर्णानन्दः स्वग्रन्थे ब्रह्मानन्दसरस्वती निर्दिशति । रामानन्दस्यापि प्रशिष्यस्स्यादिति ज्ञायते । अच्युतकृष्णानन्दसामयिकस्सतीर्थ्यश्च ।
१. अधिष्ठानविवेकः । (8. H. 54. AL) अमुद्रितोऽयं ग्रन्थ अडयारशृङ्गगिरिविद्यारण्यपुरसृचीषूपलभ्यते ।
२. तत्त्वम्पदार्थविवेकः । (R. 1382 MGOML & AL)
३. भाष्यरत्नप्रभाव्याख्या - पूर्णानन्दीया । ग्रन्थोऽयं मैसूरपुस्तकालस्थः । चतुस्सूत्रीपर्यन्तं मुद्रितः (KSS. 71) ।
४. श्रुतिसारः (7675 TSML) जीवन्मुक्तिविवेकव्याख्या दीपिकाख्या अदसीया लभ्यते (BORI)

१२१. पेत्तादीक्षितः (1625-1752 A. D.)
कौशिकगोत्रजाच्छन्दोविचितिकारात् पेद्दादीक्षितात् भिन्नोऽयं कौण्डिन्य गोत्रजः त्रिवेदीनारायणदीक्षितपुत्रः धर्मराजाध्वरिणः भ्रातुष्पुत्रः ऋक्शाखाध्यायी चोलदेशीयकण्डरमाणिक्कग्रामवासी सप्तदशशतकीय इति निर्णीयते । स्वपितृव्याद्धर्मराजाध्वरिण एवायं प्राप्तविद्य इति पदमञ्जरीव्याख्यामञ्जरीमकरन्दात् (Page 13 DC. AL. Vol VI) ज्ञायते ।
१. प्रकाशिका । वेदान्तपरिभाषाव्याख्यात्मकोऽयं ग्रन्थः अनन्तशयनग्रन्थमालायां (T. S. S. 93) मुद्रितः । तत्वचिन्तामणिप्रकाशसारोप्यनेन कृतः ।

१२२. प्रकाशानन्दः (1550-1650 A. D.)
अद्वैतानन्दस्य ज्ञानानन्दस्य च शिष्योऽयं प्रकाशानन्दः नानादीक्षितगुरुर्नृसिम्हाश्रमिसामयिकष्षोडशशतकीय इति च निर्णीयते । सिद्धान्तलेशसंग्रहेऽयं निर्दिश्यते च ।
१. सिद्धान्तमुक्तावलिः । वेदान्तसिद्धान्तमुक्ताक्ल्यपरनामायं ग्रन्थः वृत्तिसहितः पण्डितग्रन्थमालायां वाराणसीनगरे मुद्रितः । अस्य व्याख्या नानादीक्षितकृता सिद्धान्तदीपिकाख्या ।

१२३. प्रज्ञानानन्दः (1300-1400 A.D.)
``अनुभूतिस्वरूपं तं प्रणमामि परं गुरुम्" इत्युनुभूतिस्वरूपं प्रणमन्नयं अनुभूतिस्वरूपप्रशिष्य आनन्दगिरिनरेन्द्रनगर्योश्शिष्यश्चतुर्दशशतकीयादिमे भागे उवासेति निश्चीयते ।
%%% Chart
१. तत्वप्रकाशिका । आनन्दगिरीयतत्वालोकव्याख्यात्मकोऽयं ग्रन्थ अमुद्रितः पूर्णश्च अडयारपुस्तकालये (39-1-1 AL) बाम्बेनगरस्था रायलासियाटिकसोसाइटिपुस्तकालये च लभ्यते ।
२. पञ्चीकरणविवरणम् (R. 3324 MGOML)
३. त्रिपुटीटीका (III G. 51 R. A. S. Bengal)

१२४. प्रज्ञानाश्रमः (1600-1700 A. D. ?)
``बालमस्करीन्द्राय गुरवे प्रणतोऽस्म्यह" मिति वदन्नयं बालमस्करिशिष्य इति प्रतिभाति । बालमस्करीन्द्रोऽयं बालकृष्णानन्दसरस्वत्येवेति यदि निर्णीयते तर्हि सप्तदशशतकीयोऽयमिति निश्चयः । परन्तु बालमस्करिशब्दस्य बालसन्यासिरित्येवार्थस्फुरणात् बालसन्यासिरादिशङ्कर एव स्तूयते । गुरुशब्दश्चात्र बहुमानवाचक इति भाति ।
१. स्वात्मानन्दप्रकाशिका । बोधार्याव्याख्यात्मकोऽयं ग्रन्थ अमुद्रित अडयारपुस्तकालये लभ्यते । (41. C. 82. AL) ग्रन्थेऽस्मिन् बोधार्याकर्ता सर्वज्ञो भगवान् भाष्यकार इति दर्शनात् आदिशङ्कर एव कर्तां न तु सदाशिवबोधेन्द्र इति निश्चीयते । अत्र पञ्चकोशविवेकमहावाक्यविवेकवेदप्रामाण्यविवेकअहमर्थविवेकअहमर्थविवेक प्रकीर्णार्थसंग्राहकाख्यपञ्चकरणानि सन्ति ।

१२५. प्रत्यक्स्वरूपाचायँः (1400 A. D.)
प्रत्यक्प्रकाशविद्यागिर्योश्च शिष्योऽयं चतुर्दशशतकीय इति निर्णयः । अस्यैव प्रज्ञानात्मप्रकाश इति नामान्तरं दृश्यते ।
१. मानसनयनप्रसादिनी । (N.S.P.) चित्सुखाचार्यकृततत्वप्रदीपिकायाः व्याख्यात्मकोऽयं ग्रन्थः निर्णयसागरमुद्रणालये बाम्बेनगरे मुद्रितः ।

१२६. बलभद्रः (1600-1700 A. D.)
अद्वैतसिद्धिकारमधुसूदनसरस्वतीशिष्योऽयं बलभद्रस्सप्तदशशतकीय इति निर्णीयते ।
१. अद्वैतचन्द्रिका । अद्वैतसिद्धिव्याख्यात्मकोऽयं ग्रन्थः निर्णयसागरे मुद्रितः ।
२. सर्वसिद्धान्तरहस्यटीका । शाङ्करसर्वदर्शनसिद्धान्तसंग्रहाख्यग्रन्थव्याख्याभूतोऽयं ग्रन्थः दासगुप्तेन (HIP Vol. II 55) निर्दिष्टः ।।

१२७. बालकृष्णानन्दसरस्वती (1600-1700 A. D.)
वासुदेवेन्द्रसरस्वत्याः तोटकाचार्यकृतश्रुतिसारसमुद्धरणव्याख्यातुस्सच्चिदानन्दयोगिनश्चाशिष्योऽयं बालकृष्णानन्दः दक्षिणदेशवासी विशेषतः मल्लिकार्जुनक्षेत्रवासीति निश्चीयते ।।
१. अनुभवामृतम् । त्रयोदशस्वध्यायेषु 1500 पद्यैरद्वैतसिद्धान्तान् व्यासशुकाचार्य प्रश्नप्रतिवचनप्रणाल्या प्रतिपादयन्नयं अमुद्रितः ग्रन्थ अडयारपुस्तकालये (24 E. 37 AL) लभ्यते ।।
२. अद्वैतपञ्चरत्नव्याख्या - ``किरणावली" (R. 1613 1. MGOML & AL)

१२८. बालगोपालेन्द्रयतिः (1475-1575 A. D.)
जगन्नाथाश्रमिशिष्योयं बालगोपालेन्द्रः नृसिम्हाश्रमिसामयिकः । शाङ्करमनीषापञ्चकव्याख्या मधुमञ्जरी (D 4706 MGOML) अनेन कृता । बरोडाअनन्तशयन-नासिक-पुस्तकालयेषु लभ्यते च ।

१२९. बेल्लङ्गोण्ड-रामरायकविः (1850-1915 A.D.)
आन्ध्रदेशजोऽयं गुण्डूरजिल्लान्तर्गतनाराशरपेटतालुकान्तर्गत ``पानिदीपतु" ग्रामे आसीत् । अस्य जन्मकालः (1875 A. D.) इति पुरुगुल्लरामशास्त्रिसुब्रह्मण्यशास्त्रिणोश्शिष्योऽयं हयग्रीवोपासकः न्यायवेदान्तव्याकरणपण्डित अद्वैतीति च ज्ञायते ।
१. अद्वैतविजयः २. अद्वैतामृतम् ३. त्रिमतसमर्थनम् ४. देहदेहिभावनिर्मूलनम् ५. भगवद्गीताभाष्यार्थप्रकाशिका ६. मोक्षप्रसादः ७. शारीरकचतुस्सूत्रीविचारः ८. वेदान्तकौस्तुभः ९. वेदान्तसंग्रहः १०. वेदान्तदीपिका ११. वेदान्तदिनकरः १२. वेदान्ततत्वामृतम् १३. वेदान्तमक्तावलिः १४. अद्वैतान्यमतखण्डनम् इति ग्रन्था अनेन कृताः । एषु वेदान्तमुक्तावलिः शार्दूलविक्रीडितछन्दोबद्धर्ईशकेनकठप्रश्न-मुण्ड-माण्डूक्यतैत्तरीयछान्दोग्यऐतरेयबृहदारण्यककैवल्यश्वेताश्वतरोपनिषदां सारसङ्ग्राहकः 738 पद्यैः पूर्ण तेलुगुलिप्यां मुद्रितश्च । एनमधिकृत्य (Telugu Ency. Vol . III p. 1027) पुस्तकेऽधिकं विस्तृतम् । शंकराशांकरभाष्यविमर्शः, सिद्धान्तबिन्दुव्याख्या, अद्वैतन्यायमकरन्दः, सिद्धान्तकौमुदीव्याख्या-शरद्रात्रिः, चम्पूभागवतव्याख्या, अनर्धराघवव्याख्या, समुद्रमथनचम्पूः कृष्णलीलातरङ्गिणीत्यादि ग्रन्था अनेन रचिताः ।

१३०. बोधानन्दः (1700-1800 A. D.)
ब्रह्मानन्दयतिशिष्यः गुरुमूर्तिगुरुश्चायं बोधानन्दः द्रविडदेशीयपञ्चनद (तिरुवयार) क्षेत्रवासी अष्टादशशतकीय इति ज्ञायते ।
१. कैवल्यदीपिका (R. 2934 MGOML) अस्य व्याख्या नारायणप्रिययत्यपराभिधानदुर्गाप्रसादयतिना कृता स्नेहनाम्नी ।
२. बोधानन्दगीता (316 TCL)
ईशादिबृहदारण्यकान्तोपनिषदां भाष्यार्थं सङ्गृहात्ययं ग्रन्थः । द्वादशभिः परिछेदैः पूर्णोऽयं अमुद्रित अनन्तशयनपुस्तकालये लभ्यते ।

१३१. बोधेन्द्रयतिः (1500-1600 A. D.)
गीर्वाणेन्द्रसरस्वतीशिष्यः विश्वाधिकसरस्वतीशिष्यश्चायं नृसिम्हाश्रमिसतीर्थ्यश्लोलदेशीयः तञ्जपुरसमीपस्थगोविन्दपुरे प्राप्तसमाधिरिति ज्ञायते । केचित्तु बोधेन्दुगुरुर्गीर्वाणेन्द्रः नीलकण्ठदीक्षितगुरुर्गीर्वाणेन्द्र इति (1650-1750 A. D.) काले आसीदिति वदन्ति । पूर्वाश्रमेऽस्य पुरुषोत्तम इति नाम, काञ्चीमण्डलवासीति च केचिद्वदन्ति ।
१. अद्वैतभूषणम् । विवरणप्रस्थानानुयायी, पञ्चपादिकाविवरणार्थसंग्राहकः, विवरणप्रमेयसंग्रहापरनामा ब्रह्मविद्यापत्रिकायां मुद्रितः । अस्य व्याख्या वासुदेवेन्द्रकृता आनन्ददीपिकाख्या वर्तते ।
२. आत्मबोधव्याख्या । भावप्रकाशिकापरनामायं ग्रन्थ (7175 TSML) सरस्वती महालये लभ्यते । नामामृतरसायनम् हरिहराद्वैतभूषणम्, हरिहरभेदधिक्कारः, नामामृतरसोदयाख्याश्च ग्रन्थाः कृताः ।

१३२. ब्रह्मदेवपण्डितः (1700-1800 A. D.)
मल्लेश्वरसूर्याख्यात्स्वपितृव्यात् प्राप्तविद्यः पिन्निण्डिवंश्य आन्ध्रदेशजोऽयं ब्रह्मदेवपण्डित इति ज्ञायते ।
१. विशिष्टाद्वैतदूषणसारसंग्रहः । (29. L. 24. AL)

१३३. ब्रह्मानन्दः (1000-1100 A. D.)
सर्वज्ञात्मनः शिष्योऽयं स्वन्तेवासिने योगानन्दाय भगवत्पादशङ्करकथां दशसहस्राधिकैः पद्यैर्विवृणोति । अमुद्रितस्यास्य हस्तलिखिता मातृका कुम्भधोणस्थ महादेवशास्त्रिनिकटे लभ्यते । केचन अंशा श्रीभगवत्पादपूजाकल्पे मुद्रिताः ।
१. शङ्करविजयः (ब्रह्मानन्दीयशङ्करविजयः) ।

१३४. ब्रह्मानन्दभारती (1325-1425 A. D.)
आनन्दभारतीतीर्थस्य विद्यारण्यस्य च शिष्योऽयं ब्रह्मानन्दभारती चतुर्दशशतकीय इति निर्णीयते । अनेन रामानन्दभारती च नमस्कृतः ।
१. दृग्दृश्यविवेकव्याख्या - वाक्यसुधाटीका (B.S.S. 56) ग्रन्थेऽस्मिन् अद्वैतमकरनन्दकारः निर्दिश्यते ।
पुरुषार्थबोधः, प्रपञ्चसारटीका, अद्वैतमार्ताण्ड इत्यपरनामा श्रौतमार्ताण्डश्चानेेन कृताः ।

१३५. भजनानन्दः (1600-1700 A. D.)
अस्यैव भज्जुरामः, भजरामः, इत्यादि नाम दृश्यते । अनेन अद्वैतदर्पणाख्यः ग्रन्थः सव्याख्यः कृतः । अमुद्रितोऽयं ग्रन्थः पञ्जाबसूच्यां बरोडासूच्यां औधसूच्यां R. A. शास्त्रिसूच्यां च दृश्यते । पञ्चापसूच्यां (1745 सं 1689 A. D.) अस्य काल इति निर्दिश्यते ।
१. अद्वैतदर्पणम् सव्याख्यम् - (1726 B.R.D.)

१३६. भट्टवैद्यनाथः (1650-1750 A. D.)
रामचन्द्रतत्सदाख्यपुत्रोऽयं पदवाक्यप्रमाणाभिज्ञः दक्षिणदेशीयस्स्यादिति ज्ञायते । अनेन शास्त्रदीपिकाया व्याख्या प्रभानाम्नी (1767 सं 1745 A. D.) काले रचित इति हालसूच्यां (Page 174 No. XV) दृश्यते । जयपुरपोटीखानापुस्तकसूच्यां तु कल्पतरुमन्दारमञ्जरीलेखनकाल 1604 संवदिति दृश्यते । तस्मात् (1548 A. D.) अस्य काल इति वक्तव्यं भवति । लन्दननरस्थपुस्तकालयसूच्यां (1778 सं 1722 A. D.) इति प्रतिलेखकाल दृश्यते । अत्र निर्णयो न कर्तुं शक्यते ।
१. वेदान्तकल्पतरुमन्दारमञ्जरी । परिमलसंग्राहकः कल्पतरुव्याख्यात्मकः ग्रन्थः (2249 IOL VOL. IV) दृश्यते ।

१३७. भवदेवमिश्रः (1600-1650 A. D.)
महामहोपाध्यायबिरुदभूषितः सन्मिश्रभवदेवापरनामायं मिथिलावासिनः कृष्णदेवसन्मिश्रस्य पुत्रः स्वयं पाटनानगरवासी शाहजहाँनामकस्य सम्राजः सामयिकः सठक्कुरभवदेवशिष्य इति ज्ञायते । अनेन (1571 श 1649 A. D.) काले ग्रन्थः कृतः ।
१. ब्रह्मसूत्रव्याख्या चन्द्रिका (2270 IOL)

१३८. भवानीशङ्करानन्दः (1750-1850 A. D.)
रघुनाथयतीन्द्रशिष्योऽयं भवनीशङ्करानन्दस्सप्तदशशतकादारब्घे एकोनर्विशतिशतकान्ते काले उवासेति दासगुप्तमहाशयः । अद्वैतसिद्धान्तदीपिका । षड्भिः परिच्छेदैः पूर्णेऽत्र ग्रन्थे प्रथमेषु पञ्चसु प्रकरणेषु नैय्यायिक - वैशेषिक - सेश्वरसांख्य - निरीश्वरसांख्य - जैमिनीय - विशिष्टाद्वैत - द्वैतमतानि खण्डितानि । षष्ठे सिद्धान्तपरिच्छेदे स्वतन्त्रब्रह्माद्वैतवादः परिशीलितः । ग्रन्थोऽयं श्रीमहत्पुरे नक्षत्रशोधनमुद्रणालये मुद्रितः ।

१३९. भास्करदीक्षितः (1650-1750 A. D.)
अभिनवकालिदास इति प्रसिद्धस्य नृसिम्हाश्रमिशिष्यस्य उमामहेश्वराचार्यापराभिधानस्य पल्लकिचोरिग्रामाभिजनस्य चोलदेशवासिनः वेङ्कटपतियज्वनः पुत्रोऽयं भास्करदीक्षितः हारीतगोत्रजः चोलदेशीयकोनेरिराजपुरवासी उमामहेश्वरविजयराघवशास्त्रिणो सिद्धान्तसिद्धाञ्जनकारकृष्णानन्दसरस्वत्याश्च शिष्यः शाहजीराजेन (1684-1711 A. D.) दत्तस्य तिरुविशनलूराख्यग्रामस्य भागिष्वन्यतमः सप्तदशशतकापरार्धकालवासीति ज्ञायते । अस्य माता नाच्चमाम्बानाम्नी ।
%%% Chart
१. आत्मपरीक्षा (R. 756 MGOML) आत्मतत्वपरीक्षापरनामायं ग्रन्थः अहमर्थविचारं अखण्डार्थविचारं च विशेषतः परिशीलयन् अष्टाध्यायपर्यन्त अमुद्रितः मद्रास - सरस्वतीमहालय - मैसूरपुस्तकालयेषूपलभ्यते ।
२. तत्पमुद्राविद्रावणम् (7520 TSML) । विशिष्टाद्वैतदूषणपरोऽयं ग्रन्थः सरस्वतीमहालयमैसूरपुस्तकालययोर्लभ्यते ।
३. रत्नतूलिका । भास्करदीक्षितगुरुणा कृष्णानन्दसरस्वत्या कृतस्य सिद्धान्तसिद्धाञ्जनस्य व्याख्यात्मकोऽयं ग्रन्थ अनन्तशयनग्रन्थमालायां (TSS 47) मुद्रितः ।

१४०. भास्करशर्मा (1800-1900 A.D.)
अनेन ब्रह्मसूत्रभाष्यसंग्रहाख्यः ग्रन्थः कृतः । अमुद्रितोऽयं भाष्यसारभूतः ग्रन्थ उज्जैनपुस्तकसूच्यां दृश्यते ।

१४१. भास्करानन्दसरस्वती (1810-1900 A. D.)
कान्यकुब्जब्राह्मणोऽयं शाण्डिल्यगोत्रजः गोभिलसूत्री सामवेदी कौथुमशाखीयः पूर्वाश्रमे हेमकरमिश्रनामा मिश्रीलालमिश्रस्य पुण्यमत्याश्च पुत्रः अनन्तरामपण्डिताल्लब्धविद्यः दाक्षिणात्यात्पूर्णानन्दस्वामिनः लब्धदीक्षः काशीवासी, मिथिलायां लब्धजन्मा मणिराजचौबे इत्याख्यस्य दौहित्रः एकोनविंशतिशतकीय इति ज्ञायते ।
१. सव्याख्यं अनुभूतिविवरणम् २. ईशावास्यव्याख्या ३. उपनिषत्प्रसादः ४. कठोपनिषद्व्याख्या ५. केनोपनिषद्व्याख्या ६. ऐतरेयव्याख्या ७. माण्डूक्यव्याख्या ९. गौडपादकारिकाव्याख्या १०. तैत्तरीयव्याख्या ११ स्वाराज्यसिद्धिव्याख्या, इतीमे सर्वेऽपि ग्रन्थाः भारतीजीवनप्रेस वाराणस्यां मुद्रिताः ।

१४२. भैरवशर्मा (1750-1850 A. D.) 
तिलकोपनामायं भैरवशर्मा दक्षिणदेशजः वेदाक्षिवसुचन्द्रेऽब्दे स्वग्रन्थं कृतवानिति ज्ञायते ।
१. ब्रह्मसूत्रतात्पर्यविवरणम् । शङ्करभाष्यभावनावदिदं सूत्रविवरणं पण्डितनूतनग्रन्थमालायां (P. N. Vols. III & IV) मुद्रितम् ।

१४३. मल्लनाराध्यः (1400-1500 A. D.)
कोटीश्वरवंश्योऽयं विरूपाक्षपुत्रः विरूपाक्षशिष्यश्च मल्लनाराध्यः चतुर्दशपञ्चदशशतकीयो भवितुमर्हति । अनेन कृत अद्वैतरत्नाख्यः ग्रन्थ आनन्दतीर्थीयं द्वैतसिद्धान्तं खण्डयति । आनन्दतीर्थश्च (1199-1303 A. D.) पर्यन्तमासीदिति वदन्ति । (1500-1600 A. D.) कालीनेन नृसिम्हाश्रमिणा च अद्वैतरत्नस्य (अभेदरत्नस्य) व्याख्या तत्वदीपनाख्या कृता विद्यते । तस्मात् (1400-1500 A. D.) काले मल्लनाराध्य इति निर्णेतुं पार्यते ।।
१. अभेदरत्नम् (R. 5526 D. 4524 MGOML)
अद्वैतरत्नापरनामायं ग्रन्थः द्वैतमतखण्डनपर मुक्तिपरिच्छेदान्तं मद्रासराजकीयपुस्तकालये लभ्यते । अस्य व्याख्या नृसिम्हाश्रमकृता तत्वदीपनाख्या ।।

१४४. महादेवसरस्वती (1600-1700 A. D.)
वेदान्ती-महादेवः, महादेवभट्ट इत्यादिनाम्ना प्रसिद्धोऽयं महादेवसरस्वतीति ज्ञायते । गोपालबालयोगीति प्रसिद्धस्य अद्वैतानन्दसरस्वत्याः प्रशिष्यः स्वयम्प्रकाशानन्दसरस्वत्याश्शिष्यः ऋजुप्रकाशिकाकर्तुरखण्डानन्दस्य सतीर्थ्यश्चायं महादेवसरस्वती सुदर्शनेन्द्रापरनामा कामकोटिपीठाधीश आसीदिति वेदान्तसंग्रहव्याख्याया प्रकाशिकानाम्न्याः, जगद्गुरुपरम्परास्त्रोत्राच्च ज्ञायते । अस्य शिष्यावपि महादेवेन्द्रात्मबोधेन्द्रनामानौ । दासगुप्तमहाशयस्त्वेनं सप्तदशशतकीयं वर्णयति । श्रीकण्ठशास्त्रिणोऽपि षोडशशतकापरार्धकालवर्तिनमेनमङ्गीकुर्वन्ति । अनेन विष्णुसहस्रनामव्याख्या 1750 शके (1694 A. D.) काले लिखितेति जम्मूकाष्मीरसूचीपत्राज् ज्ञायते । तस्मादस्य काल (1600-1700 A. D.) भवति ।। न्यायसूत्राणां मितभाषिण्याख्या व्याख्यापि अनेन कृता वाराणस्यां दृश्यते ।
%%% Chart
१. अद्वैतग्रन्थः । वेङ्कटेश्वरपुस्तकालयसूच्यां दृश्यते ।।
२. अद्वैतप्रकाशः । मैसूरसूच्यां दृश्यते ।
३. अद्वैतभूषणम् । ग्रन्थोऽयं लालाचन्द्रसूच्यां 5953 दृश्यते ।
४. तत्वबोधः । ग्रन्थोऽयं विजयनगरसूच्यां 25 दृश्यते ।
५. तत्वानुसन्धानम् । (CSS. 985, 1083, 1155, 1127)
प्रकरणग्रन्थेऽस्मिन् प्राणायामादिविविधोपायानामपि अद्वैतब्रह्मसाक्षात्कारोपयोगित्ववर्णनपूर्वकं प्राप्तस्य ब्रह्मतत्वज्ञानस्य रक्षणोपायमावेदयन्ती ज्ञानरक्षा पद्धतिरपि विर्णिता । चतुर्भिः परिछेदैः पूर्णोऽयं ग्रन्थः सव्याख्याः कल्कत्तासंस्कृतमालायां चौखाम्बासंस्कृतमालायाञ्च । अस्य व्याख्या अद्वैतचिन्ताकौस्तुभाख्या च मुद्रिता ।
६. तत्वानुसन्धानव्याख्या - अद्वैतचिन्ताकौस्तुभः 
कुत्रचिदादर्शपुस्तकेषु अद्वैतचिन्तामणिरित्यपि नाम दृश्यते ।
७. परमामृतम् (8260 B. R. d.)
८. बृहदारण्यकभाष्यसारः तात्पर्यटीका । द्वितीयाध्यायचतुर्थब्राह्मणसारात्मकोऽयं ग्रन्थः सरस्वतीमहालये लभ्यते ।
९. विश्वेश्वरानुसन्धानम् । (1996 B. R. d.)
१०. वेदान्तसंग्रहव्याख्या - प्रकाशिका । स्वयम्प्रकाशेन कृतस्य वेदान्तसंग्रहस्य व्याख्यात्मकोऽयं ग्रन्थ अमुद्रितस्सरस्वतीमहालये (7622 TSML) लभ्यते ।
११. शारीरकसंक्षेपविवृतिः - प्रकाशिका । ग्रन्थोऽयं (186 DC. Anup.), जयपुरपोटीखानासूच्याञ्च (XXIV 97/2) लभ्यते । हरप्रसादशास्त्रिसूच्यान्तु संक्षेपशारीरककुतूहलमिति नाम दृश्यते । सांख्यवृत्तिसारः, अमरकोशव्याख्या जगद्गुरुपरम्परास्तोत्रम्, विष्णुसहस्रनामव्याख्या च अनेन कृताः ।

१४५. महीधारः (1550-1650 A. D.)
रत्नाकरपौत्रः पूर्णभद्रपुत्रः बालातन्त्रकर्तुः कल्याणभट्टस्य पिता रत्नेश्वरस्य शिष्यश्चायं वाराणसीवासी श्रीवत्सगोत्रजः मन्त्रमहोदधिकारः महीधरः (1598 A. D.) काले योगवासिष्ठसारविवृतिमकरोत् । अयं षोडशशतकीय इति बन्दरकार ओरियण्डलपत्रिकाया एकविंशतितमे भागे (A. B. O. R. I Vol XXI P. 248) वर्णितम् ।
१. योगवासिष्ठसारविवृतिः । बरोडालन्दनबाम्बेविश्वविद्यालयपुस्तकालयेषु लभ्यते ।

१४६.
महेश्वरतीर्थः (1350-1450 A.D.)
चिदानन्दघनस्य विद्यारण्यस्य च शिष्योऽयं महेश्वरतीर्थस्स्वग्रन्थे आनन्दगिरिं निर्दिशति । विद्यारण्यशिष्योऽयं चतुर्दशशतकीयः ।
१. बृहदारण्यकवार्तिकसारसंग्रहः । विद्यारण्यीयबृहदारण्यकवार्तिकसारव्याख्यात्मकोऽयं लघुसंग्रहाख्यः ग्रन्थः चौखाम्बामुद्रणालये मुद्रितः ।

१४७. महेश्वरानन्दसरस्वती (1665-1750 A. D. ?)
पुरुषोत्तमानन्दसरस्वतीशिष्योऽयं महेश्वरानन्दसरस्वतीति परं ज्ञायते । यद्ययं पुरुषोत्तमः मधुसूदनसरस्वतीशिष्यस्स्यात्तर्हि सप्तदशशतकापरार्घकालवासीति निर्णेतुं शक्यते ।
१. आत्मानात्मविवेचनम् (R. 1391 A MGOML)
२. महार्थमञ्जरी (614 TMPL)
३. महार्थमञ्जरीव्याख्या - परिमलः (614 TMPL)
४. सप्तदशप्रकरणम् । मैसूर पुस्तकालये लभ्यते ।

१४८. माधवतीर्थः (1825-1900)
चन्द्रिकाचार्यशिष्योऽयं माधवतीर्थश्शिवानन्देन्द्रसतीर्थ्यस्तिरुविशनल्लूर रामसुब्रह्मण्यशास्त्रिसामयिकः दक्षिणदेशीय एकोनविंशतिशतकीय इति निश्चयः ।
१. चन्द्रिकासारबोधः । अद्वैतसिद्धान्तगुरुचन्द्रिकाग्रन्थस्य सारभूतोऽयं पद्यात्मकः ग्रन्थः सव्याख्यः ओरियण्टलमुद्रणालये मद्रासनगरे मुद्रितः । अस्य व्याख्या शिवनन्दकृता स्वात्मादर्शाख्या ।

१४९. माधवसरस्वती (1475-1575 A. D.)
चण्डिकानदीतीरवासिनः गोकर्णक्षेत्रवासिनश्च विद्येन्द्रवनाख्यस्य यतिवर्यस्य शिष्यः सोदाक्षेत्रीयकपिलाश्रमवासी माधवसरस्वत्ययं अरसेन्द्रभूपसामयिकः कपिलाश्रमे वसन् स्वदेहावसानं सन्निकृष्टं ज्ञात्वा पम्पाक्षेत्रमगमत् । तुङ्गभद्रानदीतीर एवास्य भौतिकशरीरत्यागश्चाभूदिति ज्ञायते । अस्य जीवनकालः (1523 A. D.) इति (2007 DC BUL) अमुद्रितात् सर्वदर्शनाख्यादर्शग्रन्थाज् ज्ञायते ।
१. वेदान्तसारसर्वस्वम् । प्रकरणग्रन्थोऽयं (R. 3085 MGOML) अनन्तशयनपुस्तकालये च लभ्यते ।
२. सर्वदर्शनकौमुदी । मुद्रितश्चायं (T. S. S. 135)
३. वासिष्ठव्याख्या - वासिष्ठपञ्चिका
४. चिन्तामणिटीका मयूखमाला, 
५. सप्तपदार्थीटीका - मितभाषिणी
६. कुसुमाञ्जलिपद्यटीका-मन्दानुकम्पिनी,
७. अभिनवसप्तपदार्थी चास्य ग्रन्थाः ।
८. प्रक्रियाकौमुदीव्याख्या प्राकियासुधा (DC. AL. Vol VI No 157) अपि अनेन कृता ।

१५०. माधवसरस्वती (1550-1650 A. D.)
विश्वेश्वरसरस्वतीशिष्योऽयं माधवसरस्वती स्वग्रन्थं न्यायचूडार्णि (1578 A. D.) काले चकारेति ज्ञायते । किमयं माधवसरस्वती मधुसूदनसरस्वतीगुरुरूतान्य इति निर्णये प्रबलप्रमाणं नोपलभ्यते ।
१. न्यायचूडामणिः (64 DC. Anup) सूच्यां दृश्यते । अस्य व्याख्या वेदान्तमन्दाकिनीनाम्नी नारायणसरस्वतीकृता च विद्यते ।

१५१. माधवाश्रमः (1750-1850 A. D.)
नारायणााश्रमिशिष्योनयं माधवाश्रमः रामाश्रमापरनाम्नः परमेष्ठिगुरुरितिख्यातस्य वेदान्तभूषणकारस्य प्रशिष्यः वाराणसीवासीति ज्ञायते ।
१. स्वानुभवादर्शः । प्रकरणग्रन्थोऽयं 215 संख्यापरिमितैः प्रसादगाम्भीर्यादिगुणयुक्तैः मुमुक्षुजनमनोहरैः ललितशब्दगुम्फितैः पद्यैः ज्ञानयोगेनैव परमनिर्वाणकैवल्यस्वरूपं निजब्रह्मानन्दाद्वीतीयं परं ब्रह्म ज्ञातुमर्हतीति विवेचयन् सव्याख्यस्संस्कृतकलाशालाग्रन्थमालायां (CSS 171 & 256) मुद्रितः । अस्य व्याख्या अर्थप्रकाशिका नाम्नी मूलकृत्कृता य़

१५२. मुकुन्दमुनिः (1550-1650 A. D)
हरिनाथप्रशिष्यः रामनाथशिष्यः ब्रह्मामृतवर्षिणीकर्तुः रामकिङ्करधर्मस्य गुरुः महाराष्ट्रदेशीयः मुकुन्दानन्द-मुक्तिनाथ-मुकुन्दराजापरनामायं मुकुन्दमिनिस्सुज्ञानविंशतिकाराद्भिन्नस्सप्तदशशतकीय इति निर्णीयते ।
१. अष्टावक्रगीता । (C. C. P. B.)
२. तत्वबोधः । विवेकसिन्धुः परमार्थबोधः इति नामान्तराण्यस्य । ग्रन्थोऽयं (10 E 35 AL, 12429 B. R. d.) सरस्वतीमहालये नासिकपुस्तकालये च लभ्यते ।। परमामृतप्रकरणं, ब्रह्मावबोधः (2401 ; IOL) च अनेन कृतौ ग्रन्थौ ।

१५३. मोहनलालः (1850-1910 A. D.)
हीराधरपुत्रोऽयं मोहनलालः वाराणसीवासी प्रसिद्धस्य राममिश्रशास्त्रिणः मुकुन्ददासस्य च शिष्यः सिक्कधर्मप्रचारकस्य गुरुनानकस्य वंशजोऽपि शाङ्करेऽद्वैते बद्धादर इत्यत्र वेदान्तसिद्धान्तादर्श इति ग्रन्थकरणमेव प्रमाणम् । सांख्ये योगे वेदान्ते च पण्डितोऽयं (1887 A. D.) काले आसीदिति ज्ञायते ।
१. वेदान्तसिद्धान्तादर्शः । वेदान्तादर्शः बालबोधग्रन्थ इत्यपरनामायं ग्रन्थश्शाङ्करे वेदान्तशास्त्रे प्रविविक्षुः मध्यममन्दाधिकारिणोस्सुलभप्रवेशाय आदर्शसमेषु संज्ञाउत्पत्ति-प्रलय-यौक्तिकमतभेद-संज्ञकेषु चतुर्षु परिच्छेदेषु अद्वैतसिद्धान्तम् वर्णयति । ग्रन्थेऽस्मिन् अम्बिकादत्तगौडकृतः दुःखद्रुमकुठारः निर्दिष्टः ।। ग्रन्थोऽयं वाराणसी ग्रन्थमालायां मुद्रितः ।।

१५४. यज्ञेश्वरदीक्षितः (1600-1700 A. D.)
काश्यपगोत्रजोऽयं यज्ञेश्वरदीक्षितः बहवृचशाखाध्यायी चर्कूरिवंश्ययज्ञेश्वरपौत्रः, कोण्डुभट्टोपाध्यायगङ्गाम्बिकयोः पुत्रः तिरुमलैदीक्षितकनीयान् भ्राता यज्ञेश्वरकृष्णाश्रमयोः शिष्य इति अदसीयग्रन्थपरिशीलनात् ज्ञायतो । पञ्चपादिकाविवरणोज्जीवीन्यां ``नृसिम्हाश्रमयोगीन्द्रग्रन्थशाणनिकषात" इति दर्शनात् नृसिम्हाश्रमादर्वाचीन इति निश्चयः । न केवलं अद्वैते परं पूर्वमीमांसायामपि प्रभामण्डलनाम्नी शास्त्रदीपिकाव्याख्या कृता । दक्षिणदेशीयः चोलदेशीयशाहेन्द्रकालिकश्च ।।
१. पञ्चपादिकाविवरणोज्जीविनी । (R. 592 MGOML)
२. ईश्वरगीताभाष्यम् (8997 TSML)
शास्त्रदीपिकाव्याख्या ``प्रभामण्डलम्," अलङ्कारराघवः, अलङ्कारसूर्योदयः काव्यप्रकाशव्याख्या, चित्रबन्धरामायणव्याख्या च अनेन कृताः ।।

१५५. योगीश्वरः (1700-1900 A. D.)
अनेन स्वात्मयोगप्रदीपाख्यः ग्रन्थः कृत । अस्य व्याख्या अमरनन्दकृता । ग्रन्थोऽयं दासगुप्तेन (HIP Vol. II Page 57) निर्दिष्टः । परन्तु मद्रासराजकीयपुस्तकालये सव्याख्यस्स्वात्मयोगप्रदीप अमरान्दकृत इति त्रयोदशशतकीय इति च दृश्यते ।

१५६. रघुनाथशिरोमणिः (1477-1547 A. D.)
रघुनाथशिरोमणिरयं वङ्गदेशीयः नाडियावासी न्याये वेदान्ते नितरां निष्णातोऽपि, विशेषतः नैय्यायिकोऽपि
``ओं नमस्सर्वभूतानि विष्टभय परितिष्ठति ।
अखण्डानन्दबोधाय पूर्णाय परमात्मने ।।"
इति स्वग्रन्थेषु वदन्नयं अद्वैतब्रह्मणि विशेषमभिनिवेशं दर्शयतीति अद्वैतिकोटिमारोहति । विद्वन्मण्डले तार्किकशिरोमणिरिति बिरुदभूषितोऽयं रघुनाथकवेरिति दर्शनात्कविरपीति ज्ञायते । स्वीये चतुर्थ एव वयसि प्राप्तपितृवियोगोऽयं असहायया बन्धुजनवियक्तुया आतङ्कितया दारिद्रयदुःखोपहतया मात्रा परिश्रमानविगणय्य पालितः पोषितश्च । एकदा मात्रा परिश्रमानविगणय्य स्वपुत्रपालने वद्धपरिकरया स्वपुत्रः अङ्गाराहरणाय गृहान्तरं प्रेषितः । रिक्तहस्तोऽयं अङ्गारग्रहणसमये तापपीडानिवारणाय स्वहस्तयोस्तिकताः गृहीत्वा तासु अङ्गाराणि जग्राह । तादृशं बुद्धिकौशलं सन्दृश्य सन्तुष्टमनाः वासुदेवसार्वभौमाख्यनैय्यायिकाग्रणीः रघुनाथं सर्वाणि शास्त्राणि अध्यापयामास । गुरूपदिष्टानर्थान् न्यग्रोधवृक्षाध उपविश्य प्रतिरात्रं मननशीलोऽयं रघुनाथ अतितरां तुष्टेण स्वगुरुणा ``शिरोमणि" रिति बिरुदेन भूषित इति साम्प्रदायिकी कथा ।
अक्ष्णा काणोऽयं पक्षधरमिश्रसामयिक इचि च श्रूयते । अतएव `अभाग्यं गौडदेशस्य यत्र काणश्शिरोमणिः' इति कथनमपि श्रूयते । रघुनाथोऽयं मिथिलां गत्वा पक्षधरमिश्र इत्यपरनाम्नः जयदेवपण्डितस्य सकाशे न्यायविद्यायां महतीं नैपुणीमाससाद । कक्ष्यायां रघुनाथप्रश्नैः पराजितोऽयं पक्षधरमिश्रः रघुनाथाय चुकोप । गुरुकृतपरिभवासहिण्णुरयं रघुनाथः रात्रौ स्वगुरुं पक्षधरं जिघांसुः खडगहस्तः गुरोश्शयनागरं प्रविवेश । तत्र गतोऽयं रघुनाथश्लाघाप्रधानं स्वपत्न्या साकं पक्षधरकृतं वार्तालापं श्रुत्वा सञ्जातलञ्जः गुरुवे क्षमामयाचत इति कथा च श्रूयते ।
परन्तु पक्षधरकालः (1275 A. D.) इति सिद्धान्तः । एवञ्च रघुनाथस्य तत्सामयिकत्वे ऋते एतस्याः कथाया नान्यत्प्रबलप्रमाणमिति कथेयं पण्डितमण्डलीप्रसिद्धा किरातार्जुनी यकारभारवेः कथां मनसि कृत्वा केनापि प्रसिद्धिं यापितेति निर्णीयते ।।
१. खण्डनभूषामणिः ।
खण्डनखण्डखाद्याव्याख्यात्मकोऽयं खण्डनदीधित्यपरनामा ग्रन्थ अंशतः चौखाम्बामुद्रणालये मुद्रितः । अमुद्रितस्तु (R 4344 MGOML) लभ्यते ।।

१५७. रघुनाथसूरिः (1850 A. D.)
वासिष्ठगोत्रोत्पन्नः बहूवृचशाखाध्यायी आश्वलायनसूत्री रामचन्द्रसूरिपुत्रः रामशास्त्रिपिता महाराष्ट्रान्तर्गतवरहदृदेशीयः राघवाचार्यशिष्यश्चेति ज्ञायते । अस्य पुत्र रामशास्त्री भोरराजधान्यां राजाश्रितः नीतिवक्ता राजकीयवकीलासीत् । ग्रन्थकारोऽयं रघुनाथसूरिः पुण्यपत्तने राजकीयांग्लाधिकारिप्रार्थनावशात् न्यायाधीशपदमलञ्चकार । स्वस्यापरे वयसि नीरातीरसमीपस्थभोरमण्डलान्तर्गतराज्याधीश्वरान् पण्डितनानासाहेबपन्तसचिवान् समाश्रित् आसीत् । तदैव अद्वैतरत्नरक्षणकरण्डकोऽयं शङ्करपादभूषणग्रन्थः रचितः मुद्रितश्च ।
१. शङ्करपादभूषणम् । ब्रह्मसूत्राणां प्रथमाध्यायप्रथमपादः द्वितीयाध्यायाद्यपादश्च व्याख्यातः । द्वैतमतखण्डनपरोऽयं ग्रन्थः शिवविष्ण्वभेदप्रतिपादकः नव्यतर्कशैलीबद्धश्च आनन्दाश्रमे (ASS 101) मुद्रितः ।। भगवद्गीताव्याख्या शङ्करपदभूषणाख्या कृतेति भूमिकायाः ज्ञायते ।।

१५८.रघूत्तमसरस्वती (1684 A. D.)
अस्य गुरुस्स्वयम्प्रकाशः । प्राचार्यः ब्रह्मानन्दः । परात्परगुरुःपुरुषोत्तमानन्द इति परं ज्ञायते । अनेन स्वग्रन्थः (1605 श 1740 सं = 1684 A. D.) काले कृत इति जयपुरसूच्यां दृश्यते ।।
१. अद्वैतानन्दसागरः । सर्वविद्यासंग्रहापरनामायं ग्रन्थः जयपुरपोटीखानाराजपुस्तकालये तत्सूच्याञ्च (II 222) दृश्यते ।।

१५९. रत्नखेटश्रीनिवासः (1550-1650 A. D.)
भाष्यकारभवस्वामिभट्टवंशजोऽयं विश्वामित्रगोत्रजः कृष्णभट्टपौत्रः लक्ष्मीवल्लभपुत्रः राजचूडामणिदीक्षितपिता कामाक्षीपतिश्चायं रत्नखेटश्रीनिवासदीक्षितः चोलमण्डलाधिराजस्य शूरप्पनायकस्य सामयिकः काञ्चीमण्डलान्तर्गतशूरसमुद्रग्रामाभिजनः प्रसिद्धाप्पय्यदीक्षितानां सामयिकश्चेति निश्चीयते ।
१. अद्वैतकौस्तुभम् । ग्रन्थोऽयं राजचूडामणिदीक्षितकृतरुक्मिणीकल्याणस्य बालयज्ञदेवेश्वरकृतायां व्याख्यायां निर्दिष्टः ।
२. वादावलिः । वेदान्तवादावल्यपरनामायं ग्रन्थः षड्विंशतिभिरध्यायैः पूर्णः विशिष्टाद्वैत-द्वैतवादान् खण्डयति । अपूर्ण अमुद्रितश्च ग्रन्थः मद्रास (R. 3855 MGOML) सरस्वतीमहालये च लभ्यते । ग्रन्थेऽस्मिन् ``जन्माद्यस्ययत" इति सूत्रविवेचनावसरे ``इत्यस्मद्गुरुचरणकेशवपुरीचरणादय" इत्युक्तम् । एवञ्चास्य गुरुः केशवपुरीतिनाम्ना प्रसिद्धस्स्यादित्यू्ह्यते ।
३. भावनापुरुषोत्तमनाटकम् (T. S. M. L. 4427)

१६०. राघवानन्दः (1685 A. D.)
``विवेकनिकषोपले वहति राघवे कोलमू" मिति कृष्णपद्यां दर्शनात् राघवानन्दोऽयं स्वयम्प्रकाशानन्दप्रशिष्यः कृष्णानन्दरामभद्रानन्दयोश्शिष्यः केरल देशाधिपराघवाख्यनृपसामयिकस्स्वयं केरलवासीति ज्ञायते । अनेन लघुस्तुति व्याख्या 861 केरलवर्षेषु प्रणीतेति ज्ञायते । एवञ्चास्य काल (1685 A. D.) इति ज्ञायते ।
१. परमार्थसारव्याख्या विवरणम् (T. S. S. 12)
२. लघुस्तुतिव्याख्या (T. S. S. 60)
३. कृष्णपदी (भागवतव्याख्या) (R. 5205 MGOML)
४. त्रिपुरार्विशतिव्याख्या (R. 2827 MGOML)
५. भगवद्गीताव्याख्या (तत्वार्थचन्द्रिका) (284 T. C. D.)
६. तात्पर्यदीपिका । मुकुन्दमालाया व्याख्यानभूतोऽयं ग्रन्थः अण्णामलैविश्वविद्यालये मुद्रितः ।

१६१. राजचूडामणिदीक्षितः (1580-1650 A. D.)
सत्यमङ्गलापराभिवशूरसमुद्रग्रामामिजनस्य रत्नखेटश्रीनिवासदीक्षितस्य कामाक्ष्याश्च पुत्रः लक्ष्मीवल्लभपौत्रः कृष्णभट्टनप्ता विश्वामित्रगोत्रजः भाष्यकारभट्टवंशजः, राजचूडामणिदीक्षितः नागमाम्बागोविन्ददीक्षितयोः पुत्रस्य वेङ्कटेश्वरदीक्षितस्य अर्धनारीश्वरदीक्षितस्य च शिष्यश्चोलदेशवासीति ज्ञायते ।
%%% Chart
१. शास्त्रारम्भः । ब्रह्मविचाराय शास्त्रारम्भसमर्थनपरोऽयं ग्रन्थः वेङ्कटेश्वरपुस्त कालये 507 लभ्यते ।
ऐतरेयोपनिषदां व्याख्या, तन्त्रशिखामणिः, शास्त्रदीपिकाव्याख्या कर्पूरवर्तिका च कृताः ।

१६२. रामकिङ्करः (1600-1700 A.D.)
हरिनाथरामनाथयोः प्रशिष्यः मुकुन्दगोविन्दशिष्यश्चायं रामकिङ्करः षोडशसप्तदशशतकीय इति ज्ञायते ।
१. ब्रह्मामृतवर्षिणी । सूत्रवृत्तिरूपोऽयं ग्रन्थस्तेलुगुलिप्यां मुद्रितः । अस्य कर्ता धर्मभट्ट इति, रामानन्दसरस्वतीति, स्वयम्प्रकाशनन्दशिष्यसदाशिव इति बहुधा दृश्यते । परन्तु आनन्दाश्रममुद्रिते, अमुद्रिते बरोडास्थादर्शपुस्तके च रामकिङ्कर इत्येव दृश्यते ।

१६३. रामकृष्णः (1380-1480 A. D.)
विद्यातीर्थप्रशिष्यः भारतीतीर्थविद्यारण्यशिष्योऽयं रामकृष्णः मघ्वमतचपेटिका कर्तू रामकृष्णाद्भिन्नश्चतुर्दशशतकीय इति निश्चीयते ।
१. अद्वैतविवेकव्याख्या । नायं ग्रन्थः पञ्चदश्यन्तर्गतः । पञ्चदश्यां तादृशप्रकरणस्यादर्शनात् । परन्तु अन्योऽयं व्याख्यात्मकग्रन्थ अमुद्रितः वेङ्कटेश्वरपुस्तकालये लभ्यते ।
२. पञ्चदशीव्याख्या । तात्पर्यबोधिनी-पददीपिकापरनामायं ग्रन्थः निर्णयसागरमुद्रणालये वेङ्कटेश्वरमुद्रणालये मैसूरमुद्रणालये च मुद्रितः ।
३. महावाक्यविवेकव्याख्या । ग्रन्थोऽयं पञ्चदश्यन्तर्गतमहावाक्यविवेकव्याख्यात्माक एव । अमुद्रितोऽयं सरस्वतीमहालये (7568 T. S. M. L.) लभ्यते । केचित्तु शाङ्करमहावाक्यविवेकव्याख्यात्मक इति वदन्ति ।

१६४. रामकृष्णदीक्षितः (1625-1700 A. D.)
वेङ्कटनाथपौत्रः धर्मराजाध्वरिपुत्रः त्रिवेदीनारायणदीक्षितस्य भ्रातुष्पुत्रः कौण्डिन्यगोत्रजः ऋग्वेदाव्यायी चोलदेशीयकण्डरमाणिक्कग्रामवासी धर्मराजाध्वरिशिष्यश्चायं रामकृष्णदीक्षितः सप्तदशशतकीय इति निर्णयः ।
१. वेदान्तपरिभाषाव्याख्या शिखामणिः । ग्रन्थोऽयं वेङ्कटेश्वरमुद्रणालये मुद्रितः ।
२. विशिष्टाद्वैतभञ्जनम् । (R. 3236 B. MGOML) ग्रन्थोऽयं वेङ्कटेश्वरपुस्तकालये च लभ्यते ।
३. वेदान्तसारव्याख्या । ग्रन्थोऽयं दासगुप्तेन (H. I. P. Vol II 54) निर्दिष्टः । न्यायशिखामणिः (6228 TSML) मीमांसान्यायदर्पणञ्चानेन कृतमिति ज्ञायते ।

१६५. रामचन्द्रानन्दसरस्वती (1650-1750 A. D.)
नारायणानन्दसरस्वतीप्रशिष्यः गुरुचन्द्रिकाकर्तुः गौडब्रह्मानन्दसरस्वत्याश्शिष्यः बालकृष्णानन्दसतीर्थ्यश्चायं रामचन्द्रानन्दसरस्वती सप्तदशशतकापरार्धकालवासीति ज्ञायते ।
अनेन स्वाचार्यपरम्परा ग्रन्थ एव निर्दिष्टा ``पूर्णानन्द - पुरुषोत्तमानन्द - शिवरामानन्द - गोपालनन्द - स्वयम्प्रकाशानन्द - नारायणानन्द - ब्रह्मानन्द - रामचन्द्रानन्द" इति ।
१. ब्रह्मबोधिनी (54 A. 40. AL.) सदानन्दकृतवेदान्तसारव्याख्यात्मकोऽयं ग्रन्थ अडयारपुस्तकालये लभ्यते ।
२. भगवद्गीताव्याख्या - ``तत्वदीपिका" (D. 2068, R. 1921 MGOML) पदयोजनापरनामायं ग्रन्थः बरोडा-मद्रास-वेङ्कटेश्वर पुस्तकालयेषु लभ्यते । ग्रन्थोऽयं शाङ्करभाष्य - पौशाचभाष्य - श्रीधरीयव्याख्यासारभूतः । अनेन रामायणस्यापि व्याख्या कृतेति ग्रन्थादस्मात् ज्ञायते ।

१६६. रामचन्द्रेन्द्रसरस्वती (1765-1850 A. D.)
वासुदेवेन्द्रप्रशिष्यः वासुदेवेन्द्रशिष्यः उपनिषद्ब्रह्मापरनामकरामचन्द्रसरस्वतीसतीर्थ्यश्चायं रामचन्द्रेन्द्रसरस्वती काञ्चीपुरगतउपनिषद्ब्रह्मेन्द्रमठाधिष्ठाता अष्टादशशतकपारार्धादारब्धे काले आसीदिति निर्णीयते ।
१. कर्माकर्मविबेकः । (30. G. 13 AL) आनुष्ठुभैः पद्यैर्निर्मितोऽयं प्रकरणग्रन्थः कर्माकर्मणोर्विवेकं समष्टिव्यष्टिभेदेन मूलाविद्यातूलाविद्याभेदेन विद्याविद्याभेदञ्च निरूपयति । अडयारपुस्तकालये लभ्यते ।
२. तत्त्वम्पदार्थलक्ष्यैकशतकम् । (34. D. 16 AL) पद्यात्मकोऽयं छान्दोग्योपनिषद्गततत्वमसिमहावाक्यार्थं निरूपयति ।
३. तरङ्गः सव्याख्यः । ग्रन्थोऽयं अनन्तशयने लभ्यते ।
४. त्रिपात्तत्वविवेकः । (R. 4202 A. MGOML)
५. ब्रह्मतारषोडशसमाधिः । (34. D. 16. AL)
६. ब्रह्मप्रणवदीपिका । (34. D. 16. AL. D. 4646. MGOML) माण्डूक्योपनिषद्दिशा ओङ्कारोपासनां प्रदर्शयन्नयं समाधिनिष्ठस्य वरिष्ठत्वं प्रतिपादयति । बरोडापुस्तकालयेऽपि वर्तते ।
७. भेदतमोमार्ताण्डशतकम् । (34. D. 16. AL) ग्रन्थोऽयं जीवब्रह्मभेदखण्डनपरः दृश्यस्य द्रष्ट्रा चोभदं प्रतिपादयति ।
८. महावाक्यरत्नावलिः ।
विंशतिभिः प्रकरणैः पूर्णोऽयं ग्रन्थः अष्टोत्तरशतोपनिषद्भयः अष्टोत्तरसहस्रमहावाक्यानि प्रतिपादयति । ग्रन्थोऽयं वाराणसीमुद्रणालये मुद्रितः । अस्य व्याख्या मूलकृतत्कृता `प्रभाख्यापि । व्याख्याया अपि व्याख्या भासकलोचनाख्या वर्तते ।'
९. महावाक्यरत्नावलीव्याख्या - ``प्रभा" (34. G. 14. AL) अडयार - पञ्जाब शृङ्गगिरि - मैसूरसूचिषु दृश्यते । अस्य व्याख्या उपनिषद्ब्रह्मकृता ``भासकलो चनाख्या"
१०. विदेहमुक्तिप्रकरणम् । (34. D. 16. AL) जीवन्मुक्तिविदेहमुक्तिभेदप्रदर्शनपरग्रन्थ अमुद्रितः ।
११. श्लोकत्रयम् । (D. 4769. MGOML) त्रिभिः पद्यैरद्वैतसारं प्रदर्शयन्नयं ``विश्वंब्रह्मैवनेतरदिति" प्रतिपादयति । अमुद्रितोऽयं अडयारपुस्तकालयेऽपि लभ्यते ।

१६७. रामतीर्थः (1520-1620 A. D.)
कृष्णतीर्थशिष्यः सिद्धान्ततत्वकास्य अनन्तदेवप्रथमस्य संक्षेपशारीरकव्याख्यासुबोधिनीकारस्य पुरुषोत्तममिश्रस्य च गुरुः नृसिम्हाश्रमिसामयिकोऽयं रामतीर्थः इति निश्चीयते । अनेन ``जगन्नाथाश्रमाद्या ये गुरवो मे कृपालव" इति पञ्चीकरणविवरणव्याख्यायां तत्वचन्द्रिकायां निर्देशात् जगन्नाथाश्रमोऽयस्य गुरुरिति ज्ञायते । अनेन संक्षेपशारीरकव्याख्याता विश्ववेदः स्वकृतसंक्षेपशारीरक व्याख्याने निर्दिष्टः । अनेन मानसोल्लासव्याख्या 1630 सं 1574 A. D. काले कृत इति (1120 RAS Bombay) रायालासियाटिक सोसाइटिस्थात् आदर्शपुस्तकाज् ज्ञायते ।
१. उपदेशसाहस्रीव्याख्या - पदयोजनिका । (G. O. S. 21)
शाङ्करोपदेशसाहस्रीव्याख्यात्मकोऽयं ग्रन्थः निर्णयसागरमुद्रणालये बरोडाग्रन्थमालायां मैसूर पुस्तकालयमालायाञ्च मुद्रितः ।
२. नैष्कर्म्यसिद्धिसारार्थः । ग्रन्थोऽयं दासगुप्तमहाशयेन (HIP Vol. II) निर्दिष्टः ।
३. पञ्चपादिकाविवरणव्याख्या । दासगुप्तेन (HIP Vol. II 52) निर्दिष्टः ।
४. पञ्चीकरणविवरणव्याख्या । (10 E 34 AL)
आनन्दगिरिकृतपञ्चीकरणविवरणस्य व्याख्यात्मकोऽयं ग्रन्थ अमुद्रितः अडयारपुस्तकालये बरोडापुस्तकालये नासिकपुस्तकालये च लभ्यते । तत्वचन्द्रिका इति ग्रन्थस्य नाम ।
५. मानसोल्लासवृत्तान्तविलासः (R. 4261 c. MGOML)
सुरेश्वरकृतमानसोल्लासाख्यदक्षिणामूर्त्यष्टकव्याख्यायाः व्याख्यात्मकोऽयं ग्रन्थः अमुद्रितः मद्रास-बाम्बे-लन्दन-पुस्तकालयेषु लभ्यते । विश्वरूपसुरेश्वरैक्यमत्रोपवर्णितम् ।। जयपुरसूच्यामपि लभ्यते ।। मैसूर नगरे मुद्रितश्च (B.S. No. 6) 
६. मैत्रीयीउपनिषदव्याख्या
७. वाक्यार्थदर्पणम् (8940 B.R.D.)
८. वेदान्तसारव्याख्या - विद्वान्मनोरञ्जिनी सदानन्दीयवेदान्तसारव्याख्यात्मकोऽयं निर्णयसागरे मुद्रितः ।
९. शारीरकरहस्यार्थप्रकाशिका । अध्यायचतुष्टयवानयं सूत्रवृत्तिरूपः ग्रन्थः रायलासियाटिकसोसाइटि बाम्बे पुस्तकालये कल्कत्तसंस्कृतकलाशालापुस्तकालये बन्दरकारानुसन्धानालये च लभ्यते ।।
१०. संक्षेपशारीकव्याख्या - अन्वयार्थप्रकाशिका (B.S.S)
११. चिदानन्दलहरीटीका वेदान्तसारव्याख्यायां उल्लिखिता (P. No. 98 Nirnayasagar Edn) वस्तुतत्वप्रकाशोप्यनेन कृत इति ज्ञायते ।।

१६८. रामधनः (1753 A. D.)
अनेन ब्रह्मसूत्राणां संक्षिप्ता वृत्तिः कृता । रचनाकालः (1753 A. D.) प्रन्थोऽयं पञ्जाबसूच्यां 718 दृश्यते ।

१६९. रामनाथविद्वान् (1650-1750 A. D.)
शिवानन्दयतिशिष्योऽयं रामनाथविद्वान् स्वगुरुण शिवनन्देन कृतस्य आनन्ददीपस्य व्याख्याञ्चकार । दासगुप्तस्त्वेनं सप्तदश-एकोनविंशतिशतकयो रन्तरालवर्तिनमङ्गीकरोति । (H. I. P. II 57.)
१. आनन्ददीपिकाव्याख्या - विशुद्धदृष्टिः । (D. 4564 MGOML)

१७०. रामब्रह्मेन्द्रयोगी (1800-1900 A. D.)
वासुदेवेन्द्रप्रशिष्यः कृष्णानन्द - रामब्रह्मेन्द्र - उपनिषद्ब्रह्मेन्द्रसामयिकस्य सतीर्थ्यस्य च रामचन्द्रेन्द्रस्य शिष्यश्यन्द्रिकाचार्यसामयिकोऽयं रामब्रह्मेन्द्रः अष्टादशएकोनविंशतिशतकध्यवर्तीति निश्चीयते ।
१. स्वरूपदर्शनसिद्धाञ्जनम् (34 D. 16 AL.)
विदेहमुक्तिजीवन्मुक्तिविवेकं निर्विकल्पकसमाधिविवेकं प्रतिपादयन्नयं ग्रन्थः श्रवणमनननिदिध्यासनानां ज्ञानसाधनतां दर्शयति । अडयारपुस्तकालये लभ्यते च ।

१७१. रामब्रह्मेन्द्रसरस्वती (1800-1900 A. D.)
वासुदेवेन्द्रशिष्यः कृष्णानन्द - उपनिषद्ब्रह्मेन्द्र रामचन्द्रेन्द्राणां सतीर्थ्यः चन्द्रिकाचार्यगुरुश्चायं अष्टादशशैकोनविंशतिशतकीयः रामब्रह्मेन्द्राद्भिन्न इति ज्ञायते ।
१.सूत्रभाष्य सारसंग्रहः (27 D. 27 AL.)

१७२. रामभद्रदीक्षितः (1600-1700 A.D.)
कौण्डिन्यगोत्रजः शाहेन्द्रभूपसभापण्डितः चोक्कनाथमखिजामाता शिष्यश्च, यज्ञरामदीक्षितपुत्रः नीलकण्ठदीक्षितशिष्यश्चायं रामभद्रदीक्षितः तिरुविशनल्लूर ग्रामवासीति ज्ञायते ।
१. अद्वैतसंग्रहः । (7631 TSML)
षड्दर्शिनीसिद्धान्तसंग्रहान्तसंग्रहान्तर्गतोऽयं ग्रन्थः रामभद्रविद्वत्कृत िति अडयारपुस्तकालयस्यात् (19. M. 50 ग्र 25 AL) ग्रन्थाज् ज्ञायते । जानकीपरिणयपतञ्जलिचरित - उणादिमणिदीपिका - शब्दभेदनिरूपण - शृङ्गारतिलक - परिभाषावृत्तिव्याख्यादीनामयं कर्ता ।

१७३. रामवर्मा (1896-1915 A. D.)
केरलदेशान्तर्गतगोश्रीमहाराजोऽयं शेषस्य पुत्र इति ज्ञायते । अस्य राज्यशासनकालः (1896-1915 A. D.)
१. वेदान्तपरिभाषासंग्रहः । ग्रन्थोऽयं कोच्चिनसंस्कृतमालायां (COSS I.) मुद्रितः ।

१७४. रामशास्त्री (1875-1930 A. D.)
दक्षिणदेशजः हरिकेशवपुरग्रामवासी शालिवाटीनगराभिजनः (तिरुनेलवेली) रामशास्त्र्ययमिति ज्ञायते ।
१. कप्यासकौमुदी । छान्दोग्यगतकप्यासशब्दस्य विशिष्टाद्वैतिभिरुपवर्णितमर्थं सन्दूष्य शाङ्करभाष्यार्थोऽत्र वैय्याकरणपद्धत्या व्यवस्थाप्यते भास्करमुद्रणालयेऽनन्तशयने मुद्रितश्च ।

१७५. रामसुब्रह्मण्यशास्त्री (1850-1920 A. D.)
अश्वत्थनारायणशास्त्रिपौत्रः रामशङ्करशास्त्रिपुत्रः चोलदेशान्तर्गतशाहजग्रामवासी श्रीवत्सगोत्रोत्पन्नः शिवरामशास्त्रिशिष्यः चन्द्रिकाचार्यभिक्षुसामयिकोऽयं रामसुब्रह्मण्यशास्त्री महामहोपाध्यायबिरुदभूषितः एक्नोनविंशतिशतकान्तरालवर्तिति ज्ञायते ।
१. अनुभाष्यगाम्भीर्यनिर्णयः । भगवत्पादभाष्यसारभूतोऽयं शाङ्करभाष्योपरि कथितानां दूषणानां खण्डनपरः । आनन्दाश्रममुद्रणालये मद्रासनगरे मुद्रितः ।
२. न्यायभास्करखण्डनम् । ग्रन्थोऽयं अद्वैतसिद्धिव्याख्यालघुचन्द्रकायाः खण्डनपरस्य अनन्तार्यविरचितस्य न्यायभास्करस्य खण्डनपरः मुद्रितश्च चौखाम्बामुद्रणालये ।
३. न्यायरक्षामणिभाष्योक्तिविरोधग्रन्थः । ग्रन्थेऽस्मिन् अप्पय्यदीक्षितकृतन्यायरक्षामणौ ब्रह्मजिज्ञासासूत्रार्थकथनप्रकरणे प्रतिपादितस्य ``ब्रह्म प्रत्यगभिन्नं निर्विशेषं" इति वाक्यार्थश्शाङ्करभाष्यविरुद्ध इति प्रतिपादितम् । विनायकमुद्रणालये चिदम्बरनगरे मुद्रितोऽयं ग्रन्थः ।
४. न्यायेन्दुशेखरदोषयोगघटनग्रन्थः । ग्रन्थोऽयं विनायकमुद्रणालये चिदम्बरनगरे मुद्रितः । ग्रन्थेऽस्मिन् बिन्दुनाम्नी अप्पय्यदीक्षितकृतस्य परिमलस्य काचन टीका निर्दिष्टा । आभोगकारः लक्ष्मीनृसिम्हवाजपेयी कोट्टयूर्ग्रामवासीति च ज्ञायते ।
५. मध्वचन्द्रिकाखण्डनम् । व्यासरायकृतद्वैतपरमध्वचन्द्रिकायाः खण्डनपरोऽयं ग्रन्थः चौखाम्बायां मुद्रिताः ।
अथर्वशिखारहस्यम्, अद्वैतपद्यभाष्यम्, अद्वैतसिद्धान्तमण्डनम्, अद्वैताधिकरणस्रग्धरा, अनुपनीतसन्यासभङ्गः, आनान्तर्यवादः, ईशोपनिषद्विलासः, ऐतरेयोपनिषद्विलासः, ओङ्कारवादः, कठोपनिषद्विलासः, केनोपनिषद्विलासः, गायत्रीतत्वार्थविलासः, गीतोपनिषद्विलासः, छान्दोग्योपनिषद्विलासः, वेदान्तसूत्रमूक्तावलीव्याख्यात्मकं तत्वार्थविबोधनम्, तैत्तरीयोपनिषद्विलासः, दशोपनिषल्लधुप्रकाशः, निर्विशेषवादः, पञ्चरुद्रतात्पर्यनिर्णयः, पैङ्कलछन्दोरीत्यनुसारि रामस्तोत्रम्, प्रणवार्थनिर्णयः, प्रश्नोपनिषद्विलासः, ब्रह्मसूत्रतत्वार्थविलासः, ब्रह्मसूत्रविवरणम्, ब्रह्मसूत्रसारः, ब्रह्मसूत्रसद्वृत्तिः बृहदारण्यकोपनिषद्विलासः, भक्त्यानन्दप्रकाशः, भगवद्भक्तिप्रकाशः, भाट्टदीपिकाव्याख्या भाट्टकलपतरुः, मतरहस्यरत्नावलिः, मर्दनध्वंसमर्दनम्, माण्डूक्योपनिषद्विलासः, मुक्तिसुलभसोपानम्, मुण्डकोपनिषद्विलासः, रामकृतनलसेतुनिर्णयः, विष्णुतत्वरहस्यम्, विष्णुतत्वरहस्यमण्डनम्, विष्णुतत्वतात्पर्यम्, विष्णुद्वेषकरमतमर्दनम्, विष्णुसहस्रनामोपन्यासः, व्यामोहविद्रावणम्, शाब्दबोधवादः शास्त्रैक्यभङ्गवादः, शाङ्करभाष्यगाम्भीर्यनिर्णयः, श्वेताश्वतरोपनिषद्विलासः, सगुणनिर्गुणवादः, सज्जनमनोहरी, सर्वमतसंग्रहविलासः, सामगोष्ठीविचारः, सामस्वरविचारः, हिङ्कारमञ्जरी, हिम्सायज्ञमर्दनम्, इतीमे ग्रन्था अनेन कृताः । एषु केचन (R. 1808 MGOML) पुस्तकालये लभ्यन्ते ।।

१७६. रामभद्रानन्दसरस्वती (1600-1700 A. D.)
श्रीरामसंयमीन्दोरित्यस्य व्याख्यानावसरे गङ्गाधरेन्द्रेण रामभद्रानन्दस्यैव रामानन्द इत्यपि नामान्तरं प्रतिपाद्यते । रामभद्रानन्दोऽयं रामसंयमिकृष्णानन्दसरस्वत्योश्शिष्यः इति ज्ञायते । दासगुप्तमेनं रामानन्दसरस्वतीशिष्यं सप्तदशशतकीयं (HIP Vol. II 56) वदति । अष्टादशशतकीयेन गङ्गाधरेन्द्रेण व्याख्यायमानोऽयं ग्रन्थकारः अष्टादशशतकप्राक्तन इति निश्चयः ।
१. सिद्धान्तचन्द्रिका ।। वेदान्तसिद्धान्तचन्द्रिकापरनामायं ग्रन्थः पञ्चाशद्भिः पद्यैः पूर्णः बहुविषयभीषणसंसारकूटपन्नगग्रस्तं जीवलोकं मीमांसकभेकचमूः नैव मोचयेत् । अद्वैतसिद्धान्तराज्यस्य दौवारिकपदायैव सांख्ययोगन्यायशास्त्राण्यर्हन्ति इति शाङ्करमद्वैतमेव रक्षणार्हमिति प्रतिपादयति । गोपालनारायणमुद्रणालये मुद्रितः । अस्य व्याख्या गङ्गाधरेन्द्रसरस्वतीकृता उद्गाराख्या ।

१७७. रामाद्वयाचार्यः (1300-1400 A. D.)
अद्वयाश्रमशिष्योऽयं रामाद्वयाचार्यः स्वीये ग्रन्थे त्रयोदशशतकीयान् न्यायवेदान्ताचार्यान् निर्दिशति । वेदान्तकौमुद्याः स्वोपज्ञव्याख्यायां जनार्दनापरनामानमानन्दगिरिं निर्दिशति । वेदान्तकौमुद्याः कश्चनादर्शग्रन्थः रायलासियाटिकसोसाइटि कल्कत्तायां विद्यते । तत्र प्रतिलेखावसरः (1512 A. D.) इति दृश्यते । तस्मात् आनन्दगिरिकालात् (1320 A. D.) अर्वाग्भव (1512 A. D.) कालात्प्राक्तन इति च सिध्यति ।।
१. वेदान्तकौमुदी । प्रकरणग्रन्थोऽयं कल्पतरु-अमलानन्द-इष्टसिद्धिकारप्रकटार्थकारांश्च निर्दिशन्नयं मद्रासविश्वविद्यालयसंस्कृतमालायां (M.U.S.S. 20) मुद्रितः । 
अस्य व्याख्याः - मूलकारकृता काचन रायल आसियाटिक सोसाइटि कल्कत्तायां विद्यते । अपरा काचनाज्ञातकर्तृका बन्दरकारप्राच्यभाषानुसन्धानालयपुस्तकालये BORI लभ्यते । भावदीपिका इति नाम वाराणसीसंस्कृतविश्वविद्यालयसूच्यां दृश्यते । (Vol. VII) जीवकौमुदीनामा ग्रन्थः वेदान्तकौमुद्यां 93 पुटे निर्दिष्टः ।।

१७८. रामाध्वरीन्द्रः (1650-1750 A. D.)
नृसिम्हाश्रमिशिष्यस्य भक्तनाथस्य पौत्रः कृष्णयज्वनः पुत्रः नारायणाध्वरिशिष्यश्चायं रामाध्वरी सप्तदशशतकापरार्धकालवासीति निश्चीयते ।
१. तत्वविवेकदीपनव्याख्या अद्वैतरत्नकोशपालिनी ।। (R. 1513 MGOML) द्वितीयपरिच्छेदानन्तमडयारपुस्तकालये बरोडा मैसूरपुस्तकालययोर्लभ्यते ।

१७९. रामानन्दतीर्थः (1300-1400 A. D.) 
``ऋजुविवरणभावस्य रचिता भावदीपिका । भारतीतीर्थशिष्येण रामानन्दाभिघेन च ।।" इति दर्शनात् भारतीतीर्थशिष्योऽयं रामानन्दतीर्थश्चतुर्दशशतकीय इति निर्णीयते ।।
१. ऋजुविवरणव्याख्या - त्रय्यन्तभावदीपिका (R. 2256 MGOML) विष्णुभट्टोपाध्यायकृतपञ्चपादिकाविवरणव्याख्याऋजुविवरणव्याख्यात्मकोऽयं ग्रन्थः मद्रासपुस्तकालये लभ्यते ।।
२. अद्वैतनिर्णयसंग्रहः
३. अद्वैतरहस्यम् । ग्रन्थाविमौ राजेन्द्रलालासूच्यां दृश्येते ।।

१८०. रामानन्दसरस्वती (1685-1785 A. D.)
स्वयम्प्रकाशप्रशिष्यः रामभद्रानन्दप्रशिष्यः राघवानन्दशिष्यश्चायं रामानन्दसरस्वती स्वग्रन्थे लघुचन्द्रिकाकारं गौडब्रह्मानन्दं निर्दिशन् सप्तदशशतकापरार्धकाले उवासेति निश्चीयते ।।
१. आत्मतत्वविवेकसारः । ग्रन्थोऽयं मैसूरपुस्तकालये लभ्यते ।
२. तत्वमस्यखण्डार्थनिरूपणम् । (R. 2921 MGOML)
३. त्रिपुरतापिनीव्याख्या (R. 611 BMGOML and Mysore)
४. भगवद्गीताभाष्यव्याख्या - गीताशयः (6939 d. BRd)
५. पञ्चदशीव्याख्या - विशुद्धदृष्टिः (D. 4564 MGOML)

१८१. रामानन्दभिक्षुः (1700-1800 A. D.)
स्वयम्प्रकाशेन्द्रशिष्योऽयं रामानन्दभिक्षुरच्युतकृष्णानन्दसतीर्थ्यः अष्टादशशतकीयश्च भवति ।
१. मानमालाविवरणम् । अच्युतकृष्णानन्दकृतायाः मानमालायाः व्याख्यात्मकोऽयं अडयारपुस्तिकामालायां मुद्रितः ।।

१८२. रामेन्द्रयोगी (1700-1850 A. D.)
गीर्वाणेन्द्रमुनिशिष्योऽयं रामेन्द्रयोगी सप्तदशशतकादारब्धे एकोनविंशत्यन्ते शतके उवासेति निर्णयः । अदसीयः ``जगन्मिथ्यात्वदीपिकाख्यः" चतुर्दशप्रकरणपूर्णः ग्रन्थः (D. 4576 MGOML) लभ्यते ।। 

१८३. रामेश्वरभारती (1320-1400 A. D.)
विद्याशङ्करप्रशिष्य आत्मभारतीशिष्यश्चायं रामेश्वरभारतीति ज्ञायते । विद्याशङ्करस्यैव शृङ्गगिरिपीठारोहणात्पूर्वं विद्यातीर्थ इति प्रसिद्धिः । पीठारोहणादनन्तरभेव विद्याशङ्करः शङ्करानन्द इत्यपि नामान्तरमिति ज्ञायते । विद्याशङ्करपीठारोहणकालः (1280 A. D.) इति शृङ्गगिरिपरम्पराप्रमाण्यात् ज्ञायते । एवञ्च विद्याशङ्करप्रशिष्योऽयं चतुर्दशशतकावसानवासीति निश्चीयते ।
१. वैय्यासिकसूत्रोपन्यासः । शाङ्करभाष्यतट्टीकादिसारसंग्रहात्मकोऽयं सूत्रवृत्तिग्रन्थः अनन्तशयन - अडयार - सरस्वतीमहालय - मैसूर - मद्रास (D. 4693 MGOML) पुस्तकालयेषु लभ्यते । व्याससूत्रदीपिका इत्यपि नामान्तरम् ।

१८४. रङ्गनाथः 1620 कालादर्वाचीनः
``आनन्दाश्रमपादाब्जपरागाणां कृपाबलात् ।" इति दर्शनादानन्दाश्रमशिष्य इति ज्ञायते । ``विद्यातीर्थकृतैश्लोकैर्नृसिम्हाश्रमसूक्तिभिः । सन्दृब्धा" इति विद्यातीर्थनृसिम्हाश्रणिणौ प्रमाणयन्नयं तयोरर्वाचीन इति निश्चयः । यद्ययं आनन्दरससागरकर्तुरानन्दाश्रमस्य शिष्यस्स्यात्तर्हि कालेऽस्य (1600 A. D.) षोडशशतकात् अर्वाचीन इति वक्तुं शक्यते । अनेन विद्वज्जनमनोहारिणीनाम्नी सूत्रवृत्तिः कृतेति लन्दननगरस्थपुस्तकाज् ज्ञायते भट्टोजिदीक्षितस्य भ्रात्रा रङ्गोजिभट्टेनापि `अद्वैतचिन्तामणि' नाम्नी सूत्रवृत्तिरारचिता सरस्वतीभवने मुद्रिता च । तस्यां विद्वज्जनमनोहारिणीति नाम न दृश्यते । उभयोरानन्दाश्रमशिष्यत्वं समान दृश्यते । तस्मादयं रङ्गनाथो रङ्गोजिभट्टो वा उतान्य इति संशय उदेति ।
१. ब्रह्मसूत्रवृत्तिः - ``विद्वज्जप्तपनोहरी" (2267 DC. Vol. IOL)

१८५. रङ्गनाथसूरिः (1600-1700 A. D.)
आत्रेयगोत्रजोऽयं रङ्गनाथसूरिर्दक्षिणदेशीयः श्रीरङ्गनगरसमीपस्थित `चोलकनल्लूर' ग्रामवासी कावेरीनदीतीरलब्धवासः, विवरणोपन्यासरत्नप्रभाकारस्य रामानन्दसरस्वत्याः प्रशिष्यः वासुदेवेन्द्रस्य च प्रशिष्यः सिद्धान्तसिद्धाञ्जनकारस्य कृष्णानन्दसरस्वत्याश्शिष्यः, भास्करदीक्षितस्य सिद्धान्तसिद्धाञ्जनप्रथमप्रतिलेखकस्य रामानन्दस्य सतीर्थ्यः सप्तदशशतकीय इति भवति । अनेन स्वाचार्यपरम्परा निर्दिष्टा-
%%% Chart
अस्य शिष्यः भावाज्ञानप्रकाशिकाकारश्शिवरामाख्य इति (24. E 7. AL Mss) ज्ञायते ।
१. पुरुषार्थरत्नाकरः । (28 H. 17. AL. 5774 MGOML) विवेकनामभिः पञ्चदशभिरध्यायैः सीमितोऽयं ग्रन्थः जीवन्मुक्तगतिं ज्ञानाघिकारिणं तत्वम्पदार्थविचारञ्च कुर्वन् प्रकरणग्रन्थतामर्हति ।

१८६. रङ्गराजाध्वरी (1500-1550 A.D.)
भारद्वाजगोत्रजः काञ्चीपुरान्तर्गताडयप्पलग्रामवासी, प्रसिद्धाप्पय्यदीक्षि तानां पिता वक्षस्थलाचार्यापरनाम्न आचार्यदीक्षितस्य तोतारम्बीनाम्न्यां उत्पन्नः बोम्मराजसभापण्डितप्षोडशतकपूर्वार्धवासी रङ्गराजाध्वरीति ज्ञायते ।
%%% Chart
१. अद्वैतविद्यामुकुरः । (6686 A. B. R. D.) अद्वैतमुकुरनामायं ग्रन्थः बरोडापुस्तकालये लभ्यते । ग्रन्थोऽयं नलचरितनाटके नीलकण्ठदीक्षितेन निर्दिष्टः । मैसूरपुस्तकालये च लभ्यते ।
२. विवरणदर्पणम् । (7064 TSML) पञ्चपादिकाविवरणव्याख्यात्मकोऽयं ग्रन्थः सरस्वतीमहालये अपूर्ण उपलभ्यते । रूपकरहस्यमप्यस्य कृतिरिति वदन्ति ।

१८७. रङ्गोजिभट्टः (1600-1700 A. D.)
लक्ष्मीधरपुत्रः भट्टोजिदीक्षितकनीयान् भ्राता कोण्डुभट्टपिता आनन्दाश्रमभट्टोजिदीक्षितयोश्शिष्यः महाराष्ट्रविप्रकुलोत्पन्नः काशीवासी रङ्गोजिभट्टोऽयं नृसिम्हाश्रमिणोऽपि शिष्य इति वदन्ति । भट्टोजिदीक्षितकाल (1650 A. D.) इति निरूपितत्वात्त्तदनुजस्यापि स एव कालः । जयपुरसूच्यां प्रतिलेखनकाल (1568 A. D.) इति दृश्यते ।
१. अद्वैतशास्त्रसारोद्धारः । ग्रन्थोऽयं वाराणसीसंस्कृतकालाशालाहस्तलिखितवर्णनात्मकग्रन्थसूच्यां (3. DC. Skt c. Benaras) दृश्यते ।
२. अद्वैतचिन्तामणिः । (S. B. T. S. 2.)
अद्वैतसिद्धान्तपरमफलस्य उपायोपवर्णनपूर्वकं द्वैतसिद्धान्तबद्धपरिकरनैय्यायिकादिप्रतिद्वन्द्विमतनिरसनमुखेन अद्वैतमतं प्रतिपादितम् । महावाक्यार्थवर्णनपुरस्सरं सैद्धान्तिकी अखाण्डार्थता प्रतिपादिता । परिच्छेदद्वयपूर्णोऽयं सरस्वतीभवनग्रन्थमालायां मुद्रितः ।

१८८. लक्ष्मणसूरिः (1820-1920 A. D.)
महामहोपाध्यायविरुदभूषितस्साहित्यवेदान्तादिषु प्रकाण्डपण्डितः आशुकविरयं लक्ष्मणसूरिः भारतराज्यर्वोत्तमप्राड्विवाकन्यायाधीशस्य T.L. वेङ्कट्रामार्यस्य पिता दक्षिणदेशीयः एकोनविंशतिशतकीय इति ज्ञायते । अस्य पिता मुत्तुसुब्बा भारती । अस्य पत्नी लक्ष्मीनाम्नी । नैत्रपकाश्यपगोत्रोद्भवोऽयं युजुर्वेदाध्यायी तिरुनेल्वेलीसमीपस्थ ``हरिकेशनल्लूर्" ग्रामाभिजन इति ज्ञायते ।।
१. उपनिषत्संक्षेपवार्तिकम् । अनुष्ठुभैः छन्दोभिस्सर्वासामुपनिषदां भावार्थग्राहकं वार्तिकं प्रणीतम् । तेषु माण्डूक्योपनिषद्वार्तिकं परं मुद्रितम् । अन्ये ग्रन्था T. L. वेङ्कट्रामार्यगहे वर्तन्ते ``शङ्करभगवद्पादाभ्युदयः" भारतसाराद्याश्च ग्रन्था अनेन निर्मिताः ।

१८९. लक्ष्मीनाथझा (20th Cent. A. D.)
मिथिलासमीपस्थ दरभङ्गामण्डलान्तर्गत मधुवनीसीमास्थसीमाख्यग्रामवासी, हीराशङ्करझात् प्राप्तव्याकरणशास्त्रः, भवद्गीतागूढार्थदीपिकाकारात् बच्चाशर्माख्यात् धर्मदत्तात् प्राप्तपूर्वोत्तरमीमांसान्यायदर्शनः (1926 A. D.) कालात् काशीहिन्दूविश्वविद्यालयदर्शनविभागप्रधानाध्यापकोऽयं लक्ष्मीनाथझा स्वग्रन्थौ (1874 श 1952 A. D.) काले समापितवानिति ज्ञायते ।
१. भामतीव्याख्या `प्रकाशः' । भामतीव्याख्यात्मकोऽयं ग्रन्थः भामत्यर्थान् सुलभं बोधयति । ग्रन्थोऽयं वाराणस्यां ग्रन्थकृतैव मुद्रापितः ।
२. भामतीव्याख्या `विकासः' । ग्रन्थोऽयं नव्यतर्कशैल्यां प्रकाशाख्यव्याख्यायामनुक्तानर्थान् प्रतिपादयति । पाण्डित्यपूर्णोऽयं महान् ग्रन्थः तत्र तत्र पञ्चावयववाक्यैः क्वचित् परिष्कारप्रधानैः वाक्यैश्च भामती - तद्वयाख्यागतानर्थान् सविस्तरमुपर्णयति । ग्रन्थोऽयमपि वाराणस्यां मुद्रितः । व्युत्पत्तिवादस्यापि अनेन प्रकाशाख्या व्याख्या कृतेति ज्ञायते ।

१९०. लक्ष्मीनृसिम्हः (1650-1750 A. D.)
कोण्डुभट्टरमाम्बयोः पुत्रः महीधरवंशजोऽयं लक्ष्मीनृसिम्हः नारायणेन्द्रसरस्वतीशिष्यः कल्पतरुव्याख्यापरिमलकारादर्वाचीन इति तु निर्णयः । यद्ययं कोण्डुभट्टः रङ्गोजीभट्टस्य पुत्रः भट्टोजिदीक्षितस्य भ्रात्रीयस्यात् तर्हि लक्ष्मीनृसिम्होऽपि रङ्गोजिभट्टपौत्र इति सप्तदशशतकीय इति निणेतुं पार्यते । तिरुविशनलूर रामसुब्रह्मण्यशास्त्रिकृते न्यायेन्दुशेखरदोषयोगधटनग्रन्थे लक्ष्मीनृसिम्हवाजपेयी कोट्टयूरर्ग्रामवासीति निर्दिश्यते । एवञ्चायं दाक्षिणात्ये भवति ।
१. कल्पतरुव्याख्या आभोगः
अप्पय्यदीक्षितीयपरिमलार्थानुसारिणीयं कल्पतरुव्याख्या मद्रासराजकीयहस्तलिखितग्रन्थमालायां MGOMLS मुद्रितः । अस्य टिप्पणी श्रीरामशास्त्रिणा रचिता । सिद्धान्तकौमुदीव्याख्या विलासः तर्कदीपिका, चानेन कृताविति ज्ञायते ।

१९१. लिङ्गनसोमयाजी रायप्रोलु (2000 A. D.)
आत्रेयगोत्रजः गुण्टूराख्यान्ध्रनगरजः श्रीरमणराज्यलक्ष्म्योः पुत्रः कल्याणानन्द भारतीस्वामिशिष्योऽयं लिङ्गनसोमयाजी विंशतिशतकीय इति ज्ञायते ।
१. पञ्चदशीव्याख्या - कल्याणपीयूषः । सुमधुरकाव्यशैलीबद्धोऽयं ग्रन्थः लक्ष्मीपवरमुद्रणालये तेनालिनगरे मुद्रितः ।
२. तैत्तरीयवार्तिकव्याख्या - कल्याणविवरणम् । सुरेश्वराचार्यकृततैत्तरीयवार्तिकव्याख्यात्मकः भाष्यार्थसङ्ग्राहकश्च ग्रन्थः शारदामुद्रणालये भटनवल्लीनगरे मुद्रितः ।।

१९२. लोकनाथाध्वरी (1650-1750 A. D.)
अद्वैतदीपिकाकतुर्नृसिम्हाश्रमिणः पूर्वाश्रमे नप्ता नारायणशास्त्रिपुत्रः चोलदेशान्तर्गतआलङ्गुडिग्रामवासी चायं लोकनाथाध्वरी सप्तदशशतकापरार्घारब्दे काले उवासेति निश्चीयते ।
१. अद्वैतमुक्तासरः (R. 2985 MGOML, R. 2985 B. MGOML)
परिच्छेदचतुष्टयपूर्णोऽयं ग्रन्थः द्वैतवादखण्डनपर अप्रकाशितश्च मद्रासवेङ्कटेश्वर - अनन्तशयन - पुस्तकालयेषु लभ्यते । अस्य व्याख्या कान्तिनाम्नी मद्रास (R. 2985 B. MGOML) अनन्तशयनपुस्तकालययोर्लभ्यते ।

१९३. वनमालिमिश्रः (1800-1900 A. D.)
अनेन ब्रह्मसूत्रसिद्धान्तमुक्तावलिरिति वृत्तिग्रन्थः कृत स च चौखाम्बामुद्रणालये मुद्रितः ।

१९४. वरदपण्डितः (1750-1850 A. D.)
भट्टनारायणपुत्रः हारीतगोत्रजः वरदराजपण्डितः विद्यानन्दपूज्यपादशिष्यः वाराणसीवासी सप्तदशशतकापरार्धकालवासीति ज्ञायते ।
१. खण्डनखण्डखाद्यव्याख्या - खण्डनमण्डनम् (R. 31618 MGOML) चतुर्थपरिच्छेदान्तमपूर्ण अमुद्रितश्चायं ग्रन्थः मद्रासअडयारअनन्तशयनबरोडापुस्तकालयेषु लभ्यते ।
२. वेदान्तहृदयम् (R. 3600 MGOML)

१९५. वर्धमानः (1400 A. D.)
नव्यन्यायप्रवर्तकस्य तत्वचिन्तामणिकर्तुः गङ्गेशोपाध्यायस्य पुत्रोऽयं वर्धमानोपाध्यायः बहूनां व्याख्याग्रन्थानां निर्माता चतुर्दशशतकीयः । अनेन कृतः खण्डनखाद्यप्रकाश अमुद्रित आसियाटिक् सोसाईटि कल्कत्तापुस्तकालये लभ्यते । अदसीया अन्ये ग्रन्था अस्मदीयदर्शनमञ्जर्यां (p. 25) द्रष्टव्याः ।

१९६. वाञ्छेश्वरः (1760-1830 A. D.)
दक्षिणदेशवासी शाहजग्रामभागी महिषशतकव्याख्यातुः वाञ्छेश्वरस्य प्रपितामहः नरसिम्हस्य पितामहः महादेवपिता तञ्जपुःशासकप्रतापसिम्हस्य राजसभापण्डितः अष्टादशशतकापरार्धकालवासीति निश्चयः । विश्वामित्रगोत्रजः आश्वलायनसूत्री वाञ्छेश्वरसुधीपौत्रः शरभोजीसभाविद्वान् ईश्वरश्रीनिवासअहोविलानां शिष्य इति अनेन हिरण्यकेशीयश्रौतसूत्रव्याख्या वस्वग्निअद्रिक्षितिमितशके (1738 श 1816 A. D.) काले कृतेति (2072 D. C. TSML Vol. IV) दृश्यते । कर्णाटकाभिजनोऽयमित्यपि केचित् ।
१. ब्रह्मसूत्रार्थचिन्तामणिः ।।
सव्याख्योऽयं ग्रन्थः सूत्रवृत्तिरूपः कुम्भधोणशङ्करमठपुस्तकालये मैसूरपुस्तकालये च लभ्यते ।।
भाट्टचिन्तामणिः, दत्तकचिन्तामणिः, श्राद्धचिन्तामणिः, हिरण्यकेशीयश्रौतसूत्रव्याख्या च अनेन कृताः ग्रन्थाः ।।

१९७. वासुदेवब्रह्म (1800-1900 A. D.)
अनिरुद्धपुत्रः हृषीकेशाश्रमशिष्यश्चायमुत्तरभारतीयः वासुदेवब्रह्माख्य एकोनविंशतिशतकीय इति ज्ञायते ।।
१. बृहदारण्यकव्याख्या - प्रकाशिका (19981 B.R.D.)
२. सच्चिदनुभवप्रदीपिका (3822 B.R.D.) जयपुसूच्यां बरोडासूच्याञ्च दृश्यते । अस्य कालः (1883 A. D.) इति ।

१९८. वासुदेवब्रह्मेन्द्रः (1850-1905 A. D.)
रामचन्द्रेन्द्रापरनाम्नामुपनिषद्ब्रह्मेन्द्राणां प्रशिष्यः कृष्णानन्देन्द्रसरस्वतीशिष्यः कावेरीतीरवर्तिपञ्चनदक्षेत्रवासी वैद्यानाथशास्त्रिगुरुरयं वासुदेवब्रह्मेन्द्र एकोनविंशतिशतकीय इति निश्चीयते ।
१. शास्त्रसिद्धान्तलेशतात्पर्यसंग्रहः । (N.S.P.) सिद्धान्तलेशसंग्रहसारभूतोऽयं ग्रन्थः निर्णयसागरमुद्रणालये मुद्रितः ।।

१९९. वासुदेवशास्त्री अभ्यङ्करः (1850 - 1920 A. D.)
नागोजिभट्टप्रशिष्यभास्कराचार्यपौत्रः प्रशिष्यः शिष्यश्चायं वासुदेवशास्त्रीअभयङ्करः रामशास्त्रिणां शिष्यः पुण्यपुरवासी एकोनविंशतिशतकीय इति निश्चीयते । अत्रेयं परम्परा - नागोजीभट्टः - नीलकण्ठः - भास्कराचार्यः - रामशास्त्री - अभ्यंकरः ।
१. अद्वैताङ्कुरः (ASS 109) भगवद्गीताप्रथमद्वितीयाध्यायव्याख्यात्मकोऽयं ग्रन्थ आनन्दाश्रमुद्रणालये मुद्रितः ।
२. अद्वैतामोदः । (ASS 84)
ग्रन्थस्यास्य श्रीभाष्यकृता कृतस्य विशिष्टाद्वैतसिद्धान्तभूतस्य ग्रन्थस्य तद्दूषणानाञ्च दूषणं विषयः । अद्वैतमतप्रतिष्ठापनं मुख्यं फलम् । अद्वैतस्वरूपकथनम्, सूत्रकारात्प्रागपि मायावादस्य विद्यमानता, निर्विशेषब्रह्मवादः सूत्रकारसम्मतः, अवच्छेदवादः, आभासवादः, प्रतिबिम्बवादः, एकजीववादः, अनिर्वचनीयख्यातिः, मायाशब्दार्थ इतीमे विषयाः सूपपादिताः । आनन्दाश्रमे मुद्रितः । कायपरिशुद्धिरप्यस्य कृतिः । सिद्धान्तबिन्दुव्याख्या - ``बिन्दुप्रपातः" (BORIS 2) अनेन कृतःमुद्रितश्च ।

२००. वासुदेवेन्द्रः (1800-1900 A. D.)
रामचन्द्रापरनाभ्नामुपनिषद्ब्रह्मेन्द्राणां शिष्येष्वन्यतमोऽयं कृष्णानन्दसतीर्थ्यः वासुदेवप्रशिष्य एकोनविंशतिशतकीय इति ज्ञायते ।
१. प्रत्यक्तत्वप्रकाशिका (R. 1168 MGOML) प्रकरणग्रन्थोऽयं अध्यारोपापवादविवरणं प्रत्यगात्मवर्णनां महावाक्यार्थविचारञ्च कुर्वन्नमुद्रितः पूर्णः मद्रास-अडयारपुस्तकालययोः मैसूरपुस्तकालये च लभ्यते ।
२. ब्रह्मसूत्रविषयवाक्यविवरणम् । मैसूरपुस्तकालये लभ्यते ।
३. अद्वैतभूषणव्याख्या - आनन्ददीपिका । बोधेन्द्रकृताद्वैतभूषणव्याख्यात्मकः ग्रन्थः मैसूरपुस्तकालये लभ्यते ।

२०१. वासुदेवेन्द्रमुनिः (1700-1765 A. D.)
उपनिषद्ब्रह्मेन्द्रापरनामकस्य रामचन्द्रेन्द्रस्य रामब्रह्मेन्द्रस्य कृष्णानन्दस्य च गुरुः चन्द्रिकाचार्यरामब्रन्द्रयोः प्राचार्यः कृष्णानन्देन्द्रवासुदेवेन्द्रप्राचार्यश्चायं दक्षिणदेशवासी वासुदेवेन्द्रमुनिः वासुदेवेन्द्रयोगिशिष्यः अष्टादशशतकीय इति निश्चयः । एतद्रचिते तत्वबोधाख्ये लेखनकाल (1823 सं 1767 A. D.) इति निर्दिष्टः ।
%%% Chart
१. तत्वबोधः । प्रश्नोत्तरपद्धत्या अद्वैतसिद्धान्तप्रदर्शनपरोऽयं प्रकरणग्रन्थः वाराणसीचौखाम्बामुद्रणालये मुद्रितः ।
२.कर्माकर्मविवेकनौका (R. 4209 D MGOML)
३. विवेकमकरन्दः । ग्रन्थोऽयं नासिकसूच्यां लभ्यते ।
४. वैराग्यपञ्चकम् (7769 TSML)
५. शास्त्रदीपिकाव्याख्या ग्रन्थोऽयं बाम्बेविश्वविद्यालयवर्णनात्मकग्रन्थसूच्याः ज्ञायते ।।

२०२. वासुदेवेन्द्रसरस्वती (1800-1900 A. D.)
अनेन मनीषापञ्चकव्याख्या कृता । व्याख्येयं सदाशिवब्रह्मेन्द्रसरस्वतीकृतायास्सङ्गहरूपा दृश्यते । तस्मात्तदर्वाचीनोऽयमिति निश्चयः ।
१. मनीषापञ्चकव्याख्या पञ्चरत्नविवृतिः । शङ्करगुरुगुलपत्रिकायां वाणीविलासमुद्रणालये मुद्रितः ।।

२०३. विज्ञानात्मा (1100-1120 A. D.)
विज्ञानाश्रमः, परमानन्दमस्कर्यपरनामायं विज्ञानात्मा ज्ञानोत्तमशिष्यः चित्सुखाचार्यसतीर्थ्यः द्वादशशतकवासीति निश्चीयते । शृङ्गगिरिसूच्यां ज्ञानोत्तमकालः (910-953 A. D.) इति दृश्यते । श्रीकण्ठशास्त्रिभिरप्युपपादितम् ।
१. तैत्तरीयोपनिषद्विवृत्तिः (R. 3208 MGOML)
२. नारायणोपनिषद्विवरणम् (1505 TSML)
३. पञ्चपादिकाव्याख्या (R. 4336 MGOML)
४. प्रणवार्थप्रकाशिका (307 T. C. D.)
५. प्रपञ्चसारवृत्तिः (R. 4466 MGOML)
६. श्वेताश्वतरोपनिषद्दीपिका (R 1476 MGOML)
ग्रन्थोऽयं मद्रासनासिकविश्वभारतीशान्तिनिकेतनअनन्तशयनपुस्तकालयेषु लभ्यते । तिरुपतिपुस्तकालयेऽपि लभ्यते । पञ्चपादिकाविवरणस्यापि गूढार्थगीपिकाख्या व्याख्या कृतेति श्रूयते ।

२०४. विट्टलेशोपाध्यायः (1700-1800 A. D.)
गूर्जरदेशीयोऽयं गादाधरीयव्याख्यातुः कृष्णभट्टस्य शिष्यश्च विट्टलेशोपाध्यायः अष्टादशशतकीय इति ज्ञायते ।
१. लघुचन्द्रिकाव्याख्या - विठ्ठलेशीया । (N. S. P)

 ``"

ऽ  ?
``" ``" ``" ``" ``" ``" ``" ``" ``" ``" ``" ``" ``" ``" ``" ``" ``" ``" ``" ``" ``" ``"
`' `' `' `' `' `' `' `' `' `' 
 
ऽ  ।   ॥ ?
