\chapter{अद्वैतग्रन्थकाराः}
प्रथमे प्रकरणे प्रसिद्धतमानां अद्वैताचार्यपदव्यपदेश्यानां अद्वैतग्रन्थकर्तृणां तिथिक्रमानुसारी निर्देशः कृतः । प्रकरणेऽस्मिन् द्वितीये प्रसिद्धाः अद्वैतग्रन्थ प्रणेतारः तेषां ग्रन्थाश्च वर्णमालाक्रमेण निरप्यन्ते ।

१. अखण्डधामा (1250-1350 A.D.)
अस्याखण्डधाम्नाः गुरुरखण्डानुभूतिः । पञ्चपादिकाविवरणव्याख्याता तत्वदीपनकार अखण्डानन्दसरस्वती च अखण्डानुमूत्यानन्दगिर्योः शिष्यः । यदि स एव तत्वदीपनकारगुरुरखण्डानुभूतिरस्य गुरुस्तर्हि आनन्दगिरिसामयिकस्त्रयोदशचतुर्दशशतकयोरपरपूर्वभागे आसीदिति सिध्यति । अनेन स्वग्रन्थे श्रुतयः परं प्रमाणत्वेनोल्लिखिता नान्ये प्रबन्धकाः प्रबन्धा वा ।
१. उपदेशसाहस्रीव्याख्या - गूढार्थदीपिका ।
शाङ्करोपदेशसाहस्रीगद्यभागव्याख्यात्मकोऽयं ग्रन्थः नातिविस्तृतस्सुगमश्चामुद्रितः मद्रासपुस्तकालये R. 2793 MGOML लभ्यते ।

२. अग्निचित्पुरुषोत्तमः (1600 A.D.)
संक्षेपशारीरकव्याख्यातुष्षोडशशतकीयस्य रामतीर्थस्य शिष्योऽयं पुरुषोत्तमदीक्षित इति ग्रन्थाज् ज्ञायते । अनेन सर्वज्ञात्मकृतसंक्षेपशारीरकस्य सुबोधिनीनाम्नी अन्वर्थसंज्ञा व्याख्या कृता या च आनन्दाश्रममुद्रणालये (A. S. S. 83) मुद्रिता ।

३. अग्निहोत्रभट्टः (1600-1700 A.D.)
दाक्षिणात्योऽयं अग्निहोत्रभट्टः शब्दकौमुदीव्याख्यातुर्द्वादशाहेज्यस्य पुत्रः । चोक्कनाथशास्त्रिणः पौत्रः । रामान्वयप्रसूतस्यास्याग्निहोत्रभट्टस्य माता भवानीनाम्नी । मीमांसान्यायवेदान्तेषु निष्णातोऽयं ज्ञानेन्द्रसरस्वतीवासुदेवेन्द्रसरस्वतीगण्डनृसिम्हसृरि-कालहस्तीशयज्वनां शिष्यः । ज्ञानेन्द्रगण्डनृसिम्हावुभावपि वासुदेवेन्द्रशिष्यौ अग्निहोत्रभट्टस्य गुरू चेति ज्ञायते । कालहस्तीशयज्वा तु अग्निहोत्रमट्टस्य भावुकः गुरुश्च । तेषु ``वासुदेवाभिद्यं सौम्यं पचञ्जलिमहं भजे" इति दर्शनात् व्याकरणशास्त्रे गुरुर्वासुदेवेन्द्रः, वेदान्ते ज्ञानेन्द्रसरस्वती, न्याये गण्डनृसिम्हः, मीमांसायां कालहस्तीशयज्वा इत्यादि ऊहितुं अस्त्यवकाशः ।
सरस्वतीमहालयस्थादर्शग्रन्थान्तपुष्पिकायाः (6095 DC TSML) अग्निहोत्रभट्टकालस्प्तप्तदशशतकार्धावधिकष्षोडशशतकापरार्धादिकश्चेति ज्ञायते ।
१. अद्वैतरत्नकोशपूरणी (तत्वविवेचनी) ।
नृसिम्हाश्रमिकृतस्य तत्वविवेकदीपनापरनाम्नः तत्वविवेकव्याख्यात्मकस्य अद्वैतरत्नकोशस्य व्याख्याभूतोऽयं अद्वैतरत्नकोशपूरण्याख्यः ग्रन्थः मैसूरराजकीयविश्वविद्यालयसंस्कृतग्रन्थमालायां (O.R.I.S.S. 96 Mysore) मुद्रितः ।
तर्कोज्जीविनीनामा केशवमिश्रकृतायाः तर्कभाषाया व्याख्यायाः तत्वबोधिन्याख्यायाः व्याख्याभूतः ग्रन्थः कृतः । गङ्गेशोपाध्यायकृतस्य तत्वचिन्तामणे र्व्याख्यायाः पक्षधरमिश्रकृतायाः आलोकाख्यायाः व्याख्याः ``तत्वचिन्तामण्यालोकस्फूर्तिरिति कृता । सिद्धान्तकौमुदीव्याख्या सुमनोरमाख्या अनेन कृता उतनेति संशयः । प्रामाण्यवादाख्यः कश्चन ग्रन्थः (D. 4272 MGOML) अदसीय लभ्यते ।"

४. अच्युतरायमोडकः (1775-1839 A.D.)
नासिक (पञ्चवटी) क्षेत्रवासी अच्युतशर्मायं नारायणसाठे - अन्नपूर्णयोः पुत्रः । नारायणसाठे इत्याख्यस्यास्य पितुस्सन्यासाश्रमस्वीकारादनन्तरं अद्वैतसच्चिदानन्देन्द्रसरस्वतीति नाम । अद्वैतसच्चिदानन्दस्य शिष्यष्षष्ठिनारायण इति प्रसिद्धः अच्युतशर्मण अद्वैतदर्शने गुरुः । शिवभक्त्युपपदेशकः शैवदर्शनगुरुर्महादेवबु आख्यः । जनस्थानक्षेत्रवासी रघुनाथभट्टश्चास्य गुरुः । पितुरपि प्राप्तविद्योऽयम् । क्रैस्तवीयाष्टादशशतकापरार्धादारभ्य एकोनर्विशतिशतकपूर्वार्धान्तं चतुष्षष्ठितमास्सम । अस्य जीवनकाल इति महावाक्यार्थमञ्जरी शङ्करदिग्विजयव्याख्याया अद्वैतराज्यलक्ष्म्याख्यायाश्चावगम्यते ।
१. अद्वैतजलजातम् २. अद्वैतराज्यलक्ष्मीः ३. अद्वैतविद्याविनोदः (317 BRD, 12 Nasir)  ४. अवैदिकमततिरस्कारः ५. अद्वैतामृतमञ्जरी ६. जीवन्मुक्तिविवेकव्याख्या (पूर्णानन्देन्दुकौमुदी (ASS 20)) 7. पञ्चदशीव्याख्या (ASS 20) 8. बोधैक्यसिद्धिः (आमबोधव्याख्या) ९. महावाक्यार्थमञ्जरी (II. D. 14. A. L.) एते नव ग्रन्था अपि अद्वैतसिद्धान्तसम्बद्धाः । मद्रपुरीविश्वविद्यालये महावाक्यार्थमञ्जरी मुद्रिता (A. O. R. Vol 24 Part II)
अपरे च - अच्युतशतकम् , (नीतिशतपत्रम्) अकल्पितचिदम्बरीस्तोत्रम्, ईशदेशिकविवेचनमञ्जरी, कृष्णलीलामृतकाव्यम्, गीतासीतापतिः, गोदालहरी, (N. S. P.), दृश्यविषयताखण्डनम् (12378 B. R. D.) निरञ्जनमञ्जरी, प्रारब्धध्वान्तसंहृतिः, भागीरथीचम्पूः, भामिनीविलासव्याख्या, मतोपन्यासः, (न स्वतन्त्रः ग्रन्थः साहित्यसारस्य चतुर्थः अध्यायः) रामगीताचन्द्रिका, विष्णुपादलक्षणम्, वेदान्तामृतचिद्रत्नचषकम् (322 B. R. D.), साहित्यसारः . (N. S. P.), सौभाग्यकल्पद्रुमः सौन्दर्यलहरीव्याख्या, सदाचारः सव्याख्यः (10845 B. R. D.), हेरम्बचरणामृतलहरी च ग्रन्थाः कृताः । अधोनिर्दिष्टाः ग्रन्था एतत्कृता उत नेति न निश्चेतुं पार्यन्ते-स्मरहरविजयभागः, कृष्णशतकम्, नीतिशतकम्, रतिनीतिमुकुलम्, विशुद्धमाधवनाटकम्, मृत्युञ्जयचम्पूः, दुःखक्षयेन्दूदयः, द्वैतव्यक्तिक्षयः, मणिमयादर्शः, मुक्तिरमालङ्क्रिया, ईशकेशादिपादान्तस्तुतिः, चिच्चिन्तामणिचिन्तनम्, कारुण्यलहरी, अवयवोक्तिप्रत्युक्तिमञ्जरी, सम्यमसत्कृती, जगद्विजयः, हिरण्यकेशीयाह्निकम्, भूभृदुद्वाहः, शिवस्तवनमञ्जरी, प्रियव्रतचरितचन्द्रिका, शिवस्तुतिमुक्ताभरणम्, सिद्धान्तरत्नसिद्धान्तः, रेवापीयूषलहरी, हरि भक्तिस्प्तरामृतसिन्धुसारः, अमरप्रथमकाण्डटीका, अर्थद्वयात्मिका अमरुशतकटीका, गोवर्धनसप्तशतीटीका, सोपानपञ्चकव्याख्या, सदाशिवोक्तार्थटीका, स्वप्नमन्त्रत्रयीटीका, शृङ्गारकलिकाटीका, स्वकृतबोधैक्यसिद्धिटीका सप्तशतीटीका च । वेदान्त संग्रहनामा ग्रन्थः मद्रपुरीपुस्तकालयेऽमुद्रित उपलभ्यते ।
एषु महावाक्यार्थमञ्जरी पञ्चाशीतिभिरानुष्ठुभैः पद्यैः पूर्णः तत्वमस्यादि महावाक्यानामर्थं प्रसाधयति मद्रपुर्थां विश्वविद्यालये मुद्रिता च ।

५. अद्वयानन्दः (1500-1600 A.D.)
वेदान्तसारकर्तुस्सदानन्दस्य गुरुः शङ्करानन्दप्रशिष्यः प्रथमसदानन्दस्य शिष्योऽयमद्वयानन्दः पञ्चदशषोडशशकीय इति ज्ञायते । अदसीयः वेदान्तसंग्रहाख्यः प्रकरणग्रन्थ अमुद्रित अडयारपुस्तकालये लभ्यते ।

६. अनन्तदेवः (1600 A.D.)
महाराष्ट्रजोऽयमनन्तदेवः देवोपनामककुलप्रसूतः गोदावरीतीरग्रामाभिजनः, एकनाथपौत्रः, आपदेवपुत्रः, द्वितीयानन्तदेवप्रपितामहः, द्वितीयापदेवपितामहः, भट्टनारायण-रामतीर्थयोश्शिष्य इति ``श्रुतं यच्छ्रीरामतीर्थेभ्यः सम्प्रदायसमागतम्" इति एतदीये सिद्धान्ततत्वे, एतदीयपौत्रकृतस्मृतिकौस्तुभे च दर्शनाज्जायते । पूर्वोत्तरमीमांसापण्डितोऽयं मीमांसान्यायप्रकाशिकाकारस्यापदेवस्य पिता गुरुश्च ।
केचित्तु - न्यायप्रकाशोपान्तिमश्लोके ``गोविन्दगुरुपादयो" रितिदर्शनात् भ्रान्ताः आपदेवः गोविन्दशिष्य इति वदन्ति । अथवा अनन्तदेवस्यैव गोविन्ददेव इति नामान्तरं स्यादिति वदन्ति ।
%%% Chrat
१. सिद्धान्ततत्वम् - (P. S. 22)
अयं ग्रन्थः श्रवणादिसाधनपरिपाकसमुत्पन्नतत्वज्ञानस्य अज्ञानतत्कार्यबाधेसति सत्यज्ञानानन्दाखण्डाद्वितीयं वस्तुमात्रमवशिष्यत इति, विक्षेपशक्तिसम्बद्धं चैतन्यं ईश्वरः, आवरणशक्तिसम्बद्धं चैतन्यं जीव इति जीवेश्वरपरिष्कृर्ति कुर्वन्नयं प्रकरणग्रन्थकोटिमारोहति । ग्रन्थोऽयं पण्डितग्रन्थमालायां वाराणस्यां मुद्रितः । अस्य व्याख्या सम्प्रदायनिरूपणाख्यापि अनेनैव कृता ।
२. सिद्धान्ततत्वव्याख्या - सम्प्रदायनिरूपणम् तत्त्वप्रक्रियेतिनामान्तरम् (7547 TSML) मनोरञ्जननाटकम्,  भगवन्नामकौमुदीव्याख्या - ``प्रकाशः" (328 TCD) च अनेन कृतः ।

७. अनन्तभट्टः (1654 A.D.)
यदुभट्टापरनाम्नः दादूभट्टस्य पुत्रोऽयं अनन्तभट्टः बीकानेराधीशेन अनूपसिम्हेन प्रोत्साहितः ग्रन्थमेनञ्चकारेति ज्ञायते । अस्य पुत्रः वैद्यनाथभट्टनामा । १. अद्वैतरत्नाकरः - ग्रन्थोऽयं वेङ्कटेश्वरमुद्रणालये मुद्रितः । अस्य व्याख्या अमरदासवर्मणा कृता रत्नप्रभानाम्नी मुद्रिता च । शास्त्रमालाविवृतिरपि अनेन कृतेति ज्ञायते ।

८. अनुभवानन्दः (1600-1700 A.D.)
अनुभवानन्दोऽयं दक्षिणदेशवासी प्रसिद्धस्य सिद्धान्तसिद्धाञ्जनकर्तुः कृष्णानन्दसरस्वत्याश्शिष्येषु अन्यतम इति अमुद्रिते एतदीये कोशरत्नप्रकाशाख्ये ग्रन्थे ``कृष्णानन्दयतीश्वरं गुरुवरम्" इति, ``गुरुचोदितः कोशरत्नप्रकाशाख्यां व्याख्यां कुर्वे यथामति" इति च दर्शनाज् ज्ञायते । यद्यपि अनेन स्वग्रन्थारम्भे महेश्वरानन्द, शङ्करानन्द, कैवल्यतीर्थ शुद्धानन्दयति पूर्णानन्दाद्याः वहवो नमस्कृतास्तथापि ग्रन्थस्यादिमध्यान्तपुष्पिकाभ्यः कृष्णानन्दसरस्वत्येवास्स्य गुरुरिति निश्चीयते । सिद्धान्तचन्द्रकाव्याख्याता रामानन्दयतिः कृष्णानन्दसरस्वतीकृतानुष्ठानपद्धतिव्याख्याता अप्पाध्वरी, रत्नतृलिकाकारः भास्करदीक्षितश्चास्य सतीर्थ्या इति निश्चीयते ।
अस्य सतीर्थ्येन भास्तरदीक्षितेन शाहजीप्रथमः तञ्जपुरमण्डलाधिराजः स्वपोषक इति रत्नतूलिकायां निर्दिश्यते । शाहजीप्रथमकालः (1684-1711 A. D.) इति इतिहासविदः । एवञ्च गुरोश्शिष्याणां सतीर्थ्यानाञ्च कालस्सप्तदशशतकमिति ज्ञायते ।
१. कोशरत्नप्रकाशः - (7502 TSML)
अमुद्रितोऽयं ग्रन्थः नृसिम्हाश्रमिकृतस्वीयतत्वविवेकव्याख्याभूतस्य तत्व विवेकदीपनापरनाम्न अद्वैतरत्नकोशाख्यग्रन्थस्य व्याख्यात्मकः सरस्वतीमहालये लभ्यते ।
२. प्रभामण्डलम् - (6932 TSML) शास्त्रदीपिकाव्याख्यात्मकोऽयं ग्रन्थः यज्ञनारायणविरचिताद्भिन्न अमुद्रितश्च सरस्वतीमहालये लभ्यते ।

९. अन्नम्भट्टः (1600-1700 A.D.)
आन्ध्रदेशजोऽयं कृष्णानन्दीतीरोपान्तग्रामवासी (चितूर) अद्वैतविद्याचार्यराघवसोमयाजिकुलोत्पन्नः तिरुमलाचार्यसूनुः कौशिकागोत्रजः, सिद्धान्तकौमुदीव्याख्यासिद्धान्तरत्नाकरकर्तू रामकृष्णभट्टस्य सर्वदेवस्य च कनीयान् भ्राता, व्याकरणाचार्यंशेषकृष्णपुत्रस्य शेषवीरेश्वरापराभिधस्य शेषविश्वेश्वरस्य शिष्यः, वेदान्ते ब्रह्मानन्दसरस्वत्याः शिष्यश्चेति एतत्कृतग्रन्थाज् ज्ञायते । अयं काश्यां शास्त्राण्यधीती उवासेति एतत्कृतविश्वेश्वरध्यानात् अवगम्यते । ``काशीगमनमात्रेण नान्नम्भट्टायते द्विज" इत्यन्नम्भट्टप्रशंसनपरा किंवदन्ती च प्रसिद्धा । अस्य शिष्यः वेदाद्रिसूरिरिति प्रसिद्धः वेदान्तपरिभाषाच्याख्यातत्वबोधिनीकारः । न्याये व्याकरणे च एतदीयाः ग्रन्था उपलभ्यन्ते ।
१. मिताक्षरा-
ब्रह्मसूत्रवृत्तिग्रन्थोऽयं भामतीकल्पतरुवैय्यासिकन्यायमालाप्रदर्शितवर्त्मना ब्रह्मसूत्रार्थं वर्णयन् वाक्यविन्यासेन कल्पतरुं न्यायमालामनुसरति । ग्रन्थश्चायं मद्रासराजकीयहस्तलिखितपुस्तकालयमालायां (MGOMLS 19) मुद्रितः ।
तर्कसङ्ग्रहः, तर्कसङ्ग्रहदीपिका, तर्कभाषा, तत्वबोधिनीटीका, जयदेवकृतत्वचिन्तामणिव्याख्या सिद्धाञ्जनम्, महाभाष्यप्रदीपव्याख्या राणकोज्जीविनी, तत्वचिन्तमणिदीधितिव्याख्या सुबुद्धिमनोहरा, तन्त्रवार्तिकव्याख्या सुबोधिनी, स्वरविवेकः, तत्वविवेकदीपनव्याख्याश्च कृता इति ज्ञायते ।

१०. अभिनवनारायणेन्द्रसरस्वती (1600-1700 A.D.)
अयमभिनवनारायणः ज्ञानेन्द्रसरस्वतीशिष्यः, कैवल्येन्द्रसरस्वतीप्रशिष्यः सदाशिवब्रह्मेन्द्रगुरोः परमशिवेन्द्रस्य गुरुः, अग्निहोत्रभट्टस्य सतीर्थ्यश्चेति ज्ञायते । प्रसिद्धनृसिम्हाश्रमिणां शिष्यस्य नारायणाश्रमिण अपरे वयसि सामयिकोऽसाविति कारणादेवास्य अभिनवनारायणत्वं सार्थकं भवति । अनेन स्वीये छान्दोग्यभाष्यव्याख्याने विद्याप्रकाशाख्यः ग्रन्थः बहुवारं प्रमाणीकृतः ।
%%% Chart
१. ऐतरेयोपनिषद्भाष्यटीका - (R. 1475 MGOML)
२. कठोपनिषद्भाष्यटीका - (XXI old h)
३. केनोपनिषद्भाष्यटीका - (XXI 26 Oudh)
४. छान्दोग्योपनिषद् भाष्यटीका (R. 1662 MGOML)
५. पञ्चीकरणभावप्रकाशिका (R. 1492 B. MGOML)
६. प्रश्नोपनिषद्भाष्यटीका (D. 621 MGOML)
७. मुण्डकोपनिषद्भाष्यटीका (XXI 26 Oudh)
८. वार्तिकाभरणम् (B. S. S.)
सुरेश्वराचार्यकृतपञ्चीकरणवार्तिकव्याख्यात्मकोऽयं ग्रन्थः वाराणस्यां मुद्रितः । अनेन पञ्चरत्नव्याख्या कल्पवल्लीनाम्नी कृतेति वदन्ति ।

११. अभिनवसदाशिवब्रह्मेन्द्रः (1800-1900 A.D.)
कर्माकर्मविवेकतत्वम्पदार्थल्क्ष्यैकशतकादिग्रन्थप्रणेतुर्वासुदेवेन्द्रशिष्यरामचन्द्रेन्द्रस्य शिष्योऽयं अभिनवसदाशिवः आत्मानं स्वीयाभिनवत्वविशेषणेनैव प्रसिद्धसदाशिवब्रह्मेन्द्रस्य पश्चाद्भवं दक्षिणदेशजमावेदयति ।
१. पञ्चीकरणम् (D. 4572 MGOML)

१२. अमरदासः (1800-1900 A.D.)
अनेन वेदान्तपरिभाषा-व्याख्यामणिप्रभायाः विषयपरिच्छेदे (Page 298) स्वस्य दीक्षागुरुः श्रीचन्द्र इति निर्दिश्यते । ब्रह्मविज्ञानोऽस्य विद्यागुरुः । नानकवादनात् गुरुसिक्खमतप्रवर्तकनानकसम्प्रदायानुगतोऽपि अद्वैतसम्प्रदाये महानादर अनेन प्रदर्श्यते । अमरदासोऽयं एकोनविंशतिशतकीय इति ग्रन्थेभ्यः निर्णीयते ।
१. वेदान्तपरिभाषा-शिखामणि-व्याख्या-मणिप्रभा ग्रन्थोऽयं वेङ्कटेश्वर स्ट्रीममुद्रणालये मुद्रितः ।
२. ईश-ऐतरेय-कठ-केन-तैत्तरीय-प्रश्न-माण्डूक्य-मुण्डकोपनिषदां व्याख्या मणिप्रभा अनेन कृता । अतएवायं मणिप्रभाकार इत्यपि प्रसिद्धः ।
उपनिषदां व्याख्या मणिप्रभा चौखाम्बामुद्रणालये मुद्रिता ।

१३. अमरानन्दः (1225-1300 A.D.)
कर्णाटकदेशाभिजनस्यापि काशीवासिनोऽस्यामरानन्दस्य पूर्वाश्रमे पितुर्नाम कुमारेश्वर इति ज्ञायते । कर्णाटकदेशशासितुः होयलावंशजस्य त्रयोदशशतकीयस्य नरसिम्हपुत्रसोमेश्वरस्य सामयिकोऽयममरानन्दः अमरानन्दप्रशिष्यस्य जगदाराध्येत्यपरनामकानुपमसुखशिष्यस्य विश्वनाथापराख्यनिरूपमबोधस्य शिष्य इति ग्रन्थेषु दर्शनात् अस्य परात्परगुरुः अमरानन्दप्रथमः, परमगुरुर्जगदाराध्येत्यपरनामा अनुपमसुखः, विश्वनाथापरनामा निरुपमबोधः गुरुरिति निश्चीयते ।
१. स्वात्मयोगप्रदीपः - (R. 3428 MGOML)
आत्माद्वैतप्रतिपादकोऽयं ग्रन्थः जीवेश्वरस्वरूपं वर्णयन् तत्त्वम्पदविचारं सपरिकरं प्रतिपादयति । गौडपादाचार्यं भट्टाचार्यञ्च प्रमाणयति । अस्य व्याख्यापि मूलकृतैव स्वात्मयोगप्रदीपप्रबोधिनी नाम्नी कृता अनुद्रिता च वर्तते ।
२.स्वात्मयोगप्रदीपव्याख्या - प्रबोधिनी (R. 3428 MGOML) विष्णुपुराणव्याख्या विष्णुवल्लभानाम्न्यपि अदसीयः ग्रन्थः ।

१४. अमृतानन्दमुनिः (1400 A.D.)
``अपरं दक्षिणामूर्तिं तमानन्दगिरिं भजे" इति आनन्दशैलाङ्घ्रिसरोजभृङ्गमाराद्भजे यादवशक्रशेलमिति आनन्दगिरियादवेन्द्रगिर्योर्नमस्कृतिश्रवणात् अमृतानन्दोऽयं आन्दगिर-यादवेन्द्रगिर्योश्शिष्य इति ज्ञायते । यादवेन्द्रगिरिस्तु आनन्दगिरेरपि शिष्य इति ज्ञायते ।
१. न्यायदीपावलीव्याख्या - न्यायविवेकः
आनन्दबोधाचार्यकृतन्यायदीपावलीव्याख्यात्मकोऽयं ग्रन्थः प्रथमानुमानपर्यन्तं मुद्रितश्चौखाम्बामुद्रणालये । अमुद्रितश्चापूर्णः ग्रन्थः सरस्वतीमहालये लभ्यते ।

१५. अल्लालसूरिः (1300-1400 A.D.)
कोटिकलाग्रामवासी नागमाम्बात्रिविक्रमाचार्ययोः पुत्र अनन्तार्यप्रज्ञानारण्ययोश्शिष्योऽयमल्लालसूरिस्स्वग्रन्थे कल्पतरुकारं अमलानन्दं व्यासाश्रमशब्देन निर्दिशन् चित्सुखाचार्यं प्रमाणयति । अनेन स्वग्रन्थारम्भे वाचस्पतिमिश्रः व्यासाश्रमः प्रज्ञानारण्यश्च सनामग्रहणं निर्दिष्टाः ।
``अक्षुण्णायामरण्यान्यां भामत्यां वर्त्म यो व्यधात् ।
सुगमं प्रणमामस्तं व्यासाश्रममुनीश्वरम् ।।"
इति व्यासाश्रमः निर्दिष्टः । व्यासाश्रम एवामलानन्द इति केचित् । केचित्तु तर्योर्मेदमामनन्ति । तेषां मतेन सुगम इति भामतीव्याख्या काचनासीद्या नाममात्रेण प्रसिद्धेदानीमिति वक्तव्यमापतति ।
यदि व्यासाश्रमामलानन्दावभिन्नौ (632 TMPL Mss.) तर्हि अल्लालकाल (1300 A. D.) कालादर्वाग्भवः । स्वग्रन्थे परिमलकारस्याप्पय्यदीक्षितास्यानुल्लेखात् अप्पय्यदीक्षितात्प्राचीन इति निश्चीयते । बरोडापुस्तकालयस्थे (13768 B. R. D.) भामतीतिलकादर्शपुस्तकेतु तस्य प्रतिलिपिकाल (1335 A. D.) इति दृश्यते । तस्मात् (1335 A. D.) कालात्प्राक्तन इत्यत्र तु न संशयः । यदि व्यासाश्रमामलानन्दौ भिन्नौ तर्हि अमलानन्दव्यासाश्रमात् सुगमकारः व्यासाश्रमभिन्न इति बरोडास्थप्रतिलिपिकालस्यायमर्थ आपतति यत् कल्पतरुकारादपि प्राचीन इति । केचित्तु अल्लालसूरिः (1600-1700 A. D.) काले आसीदिति वदन्ति । अन्ये तु (1756 A.D.) कालादर्वाग्भव इति वदन्ति । तेषां मतेन बरोडास्थ प्रतिलिपिकाल विरुध्यते ।
१. भामतीतिलकम् (R. 4190 MGOML)
भामतीव्याख्यात्मकोऽयं ग्रन्थ अमुद्रित अपूर्णश्च बहुत्रोपलभ्यते ।

१६. आत्मस्वरूपः (1500-1250 A.D.)
अप्रकाशिते एतत्कृते प्रबोधपरिशोधिन्याख्ये ग्रन्थे ``श्रीनृसिम्हस्वरूपस्य शिष्येणेयं मयेरिता" इति दर्शनात् नृसिम्हस्वरूपशिष्योऽयमात्मस्वरूपः अनुमूतिस्वरूपादिवत्स्वरूपान्तनामा द्वादशशतकादारभ्य त्रयोदशशतकपूर्वार्धावधिककालिकस्स्यादिति ज्ञायते । अत्रेमानि कारणानि -
आत्मस्वरूपरचितौ द्वौ ग्रन्थावुलभ्येते । प्रबोधपरिशोधिनीनामा पञ्चपादिका व्याख्यात्मकः कश्चन ग्रन्थः । द्वादशशतकीयेन आनन्दनुभवेन कृतस्य पदार्थतत्वनिर्णयस्य व्याख्यात्मक अपरो ग्रन्थः । तयोः प्रबोधपरिशोधिन्यां पञ्चपादिकाविवरणाचार्यः, आचार्यसुन्दरपाण्ड्यः, गौडाचार्यः प्रभाकराः भाट्टाश्च निर्दिष्टाः । 31-37 पत्रपर्यन्तमनिर्वचनीयख्यातिस्सम्यङिनरूपिता । न तत्र खण्डनादिग्रन्थो वा ग्रन्थकर्ता वोल्लेखार्हों नोल्लिखितः । पदार्थतत्वनिर्णयटीकायां भाट्टः किरणावली, उदयनः, न्यायसारः, भूषणं, ब्रह्मसिद्धिः, लीलावतीकारः, प्रमाणमाला, प्रशस्तपादभाष्यम्, कुसुमाञ्जलिः, इष्टसिद्धिकारश्चोद्धृताः । प्रमाणमालाकारः आनन्दबोध 11-12 शतकीयः । दशमशतकीय इष्टमिद्धिकारः । आनन्दगिरिश्चित्सुखो वा न र्निदिष्टः । तस्मात्तयोः पूर्वतनः आनन्दबोधादर्वाक्तन इति निश्चीयते । नान्यदत्र प्रमाणमुपलभ्यते ।
१. पञ्चपादिकाव्याख्या - प्रबोधपरिशोधिनी । (R. 3225 MGOML) अमुद्रित उपलभ्यते ।
२. पदार्थतत्वनिर्णयटीका - (R. 4219 MGOML) मध्यलुप्तोऽयं ग्रन्थ अमुद्रितः ।

१७. आदिवेङ्कटयोगी (1700-1800 A.D.)
अयमादिवेङ्कटयोगी भारद्वाजगोत्रजः गन्नेपुडीग्रामाभिजनः कोण्डयपुत्रः सुब्बय्यामात्यपुत्रः रामचन्द्रसरस्वतीप्रशिष्यः स्वयम्प्रकाशसरस्वत्याश्शिष्यः आन्ध्रदेशज इति ज्ञायते । ``अङ्गीकृता हि मत्कृतिरखिलज्ञैर्बालकृष्णयतिवर्यैः । शाहजियाग्रहारस्थितविद्वन्मुख्यसेवितपदाब्जैः ।" इति वदन्नयं ग्रन्थकृत् सिद्धान्तसिद्धाञ्जनकाराणां शाहजग्रामवासिनां बालकृष्णसरस्वतीनां सामयिक इति ज्ञायते ।
१. व्रह्मविन्निधिः - (29 G. 29 AL. R. 4362 MGOML)
अमुद्रितोऽयं पूर्णः ग्रन्थः त्रयस्त्रिंशद्भिः प्रकरणः परिमितः जगन्मिथ्यात्व, गुरुशिष्यलक्षणम्, महावाक्यार्थविचारम् , इन्द्रियजयोपायम् , ध्यानस्वरूपम् , जीवस्वरूपम् ,  सुषुप्तिमृत्योंर्मेदम्, मुक्तिस्वरूपम्, जीवन्मुक्तिविदेहमुक्तिनिरूपणम् , अद्वैतब्रह्मस्वरूपम् , ब्रह्माविचाराधिकारिनिरूपणञ्च कुर्वन् समग्राद्वैतवेदान्तप्रकरणग्रन्थ प्रतिपादितान् विषयान् साकल्येनैकत्र प्रतिपादयति । ग्रन्थोऽयं अडयार - मद्रासराजकीयहस्तलिखितपुस्तकालययोर्लभ्यते ।

१८.आनन्दबोधेन्द्रसरस्वती (1780-1850 A. D.)
सर्वज्ञसरस्वती - रामचन्द्रसरस्वत्योः प्रशिष्यः स्वाराज्यसिद्धिकारस्य गङ्गाधरेन्द्रसरस्वत्याश्शिष्य इति ज्ञायते । अनेन ऋतुरसतुरगमहीशक विकारि शुभवत्सरस्य शिशिरर्तोः इति स्वग्रन्थनिर्माणकाल (1766-श 1i842 A.D.) निर्दिश्यते ।
१. योगवासिष्ठव्याख्या-तात्पर्यप्रकाशः (N. S. P.) योगवासिष्ठव्याख्यत्मकोऽयं ग्रन्थः निर्णयसागरमुद्रणालये मुद्रितः ।

१९. आनन्दरायमखी (1684-1728 A.D.)
गङ्गाधराध्वरिणः पौत्रः, नृसिम्हरायपुत्रः त्र्यम्बकरायज्येष्ठभ्राता चायं आनन्दरायः शाहजीशरभोजीसचिवश्चेति सप्तदशाष्टादशशतकीयः इति ज्ञायते । अस्य कृतिर्वेदान्ततत्वप्रधाना विद्यापरिणयाख्या । मुद्रिताचेयं निर्णयसागरे ।

२०. आनन्दस्वरूपभट्टारकः (1300 A.D.)
अयमानन्दस्वरूपभट्टारक आनन्दात्मशिष्यः । आनन्दात्मा शङ्करानन्दस्यापि गुरुः । एवञ्च शङ्करानन्दसतीर्थ्योऽयं त्रयोदशशतकीय इति निश्चयः ।
१. वाक्यदीपिका (R. 3324 C. MGOML)
वेदोत्तमभट्टारकरचिताया बृहद्वाक्यवृत्याः व्याख्यारूपोऽयं ग्रन्थ अमुद्रितः मद्रपुरीपुस्तकालये लभ्यते ।

२१. आनन्दाश्रमः (1650-1750 A.D.)
सच्चिदानन्दचिद्धनानन्दाश्रमवरशिष्यपरमहंसपरिव्राजकाचार्यआनन्दाश्रमेत्यादिग्रन्थात् सच्चिदानन्दचिद्धनानन्दशिष्योऽयं आनन्दाश्रमः महाराष्ट्रदेशज इति ज्ञायते । अस्याचार्यस्य विश्वेश्वर इति नामान्तरमपि स्यादिति ग्रन्थतो ज्ञायते । एतद्विरचिते आनन्दरससागराख्ये ग्रन्थे शङ्कराचार्यः, मानसोल्लासः, दशश्लोकी, योगवासिष्ठञ्च प्रमाणत्वेन वर्णितानि ।
कालनिर्णये प्रबलतरप्रमाणं नोपलभ्यते । परन्त्वस्य शिष्येण कृतायां मध्वसिद्धान्तभञ्जिन्यां भट्टोजिदीक्षितः निर्दिष्टः । यद्ययं भट्टोजिः प्रसिद्धस्स्यात् तर्हि तस्य सप्तदशशतकीयत्वेन तदर्वाग्भवोऽयमानन्दाश्रम इति निर्णेतुं शक्यते ।
१. आनन्दरससागरः (R. 7543. R. 5749 MGOML)
त्रिपञ्चाशद्भिः प्रकरणैः पूर्णोऽयं प्रकरणग्रन्थः अखण्डार्थत्वं जगन्मिथ्यात्वं आत्मानात्मविचारञ्च कुर्वन् मद्रासराजकीयपुस्तकालये लभ्यते ।

२२. आपदेवः (1600-1700 A.D.)
दाक्षिणात्योऽयं आपदेवः महाराष्ट्रदेशीयः देवोपनामककुलप्रसूतः गोदावरीतीरग्रामाभिजनः एकनाथनप्ता आपदेवप्रथमस्य पौत्रः, सिद्धान्ततत्वकारस्य अनन्तदेवस्य पुत्रः स्मृतिकौस्तुभकारस्य अनन्तदेवस्य पितामहः स्वपितुरेवाधीतदर्शन इति ज्ञायते ।
केचित्तु मीमांसान्यायप्रकाशोपान्त्यश्लोके ``गोविन्दगुरुपादयो" रिति दर्शनात् भ्रान्ताः गोविन्दशिष्य आपदेव इति अनन्तदेवस्यैव गोविन्द इति नामान्तरमिति वा वदन्ति ।
१. वेदान्तसारव्याख्या - बालबोधिनी (V. V. S.)
ग्रन्थोऽयं सदानन्दकृतवेदान्तसारव्याख्यात्मकः । ग्रन्थस्यास्य तत्वदीपिका इत्यपि नामान्तरं दृश्यते । ग्रन्थोऽयं वाणिविलासमुद्रणालाये श्रीरङ्गनगरे मुद्रितः । मीमांसान्यायप्रकाशोऽपि अनेन कृतः ।

२३. ईश्वरतीर्थः (1098-1146 A.D.)
शृङ्गगिरिशङ्करपीठपरम्परागतोऽयमीश्वरतीर्थः नृसिम्हगिरिशिष्यः, नृसिम्हतीर्थगुरुश्चेति शृङ्गगिरिगुरुपरम्परासूच्याः ज्ञायते । भारतीयैतिहासिकत्रैमासिकपत्रिकायां (IHQ Vol. XIV) प्रतिपादितश्च ।
१. शतश्लोकी - (5539 C.C.P.B.) वैराग्यप्रकरणापराभिधानोऽयं ग्रन्थ अमुद्रितः मध्यप्रान्तीयबरार्ग्रन्थसूच्यां लभ्यते ।

२४. उत्तमश्लोकः (1250-1350 A.D.)
उत्तमश्लोकोऽयं शुद्धानन्दशिष्य इति ग्रन्थाज् ज्ञायते । आनन्दगिरिस्तु शुद्धानन्दस्यापि शिष्यः । यद्येवं तर्हि आनन्दगिरिसामयिकस्सतीर्थ्यश्चेति सिध्यति । श्रीकण्ठशास्त्री तु आनन्दगिरिकालं 1114-1228 A. D. इति (IHQ Vol. XIV) वदति । लक्ष्मीधरगुरोरनन्तान्दगिरिरिति नाम । लक्ष्मीधरस्य शिष्येषु शुद्धानन्दः कश्चन आसीत् । यस्य च शिष्यस्स्वयन्प्रकाश इति प्रतिपादयति । यद्येवं शुद्धानन्दशिप्यौ स्वयम्प्रकाशोत्तमश्लोकौ द्वावपि सामयिकौ सतीर्थ्यौ चेति (1400 A.D.) कालीनाविति सिध्यति । ग्रन्थेषु विश्वनाथस्थ विशालनयनानाथस्य च नमस्कृतिप्रकाशनात् वाराणसीवासीति ज्ञायते ।
१. लघुवार्तिकम् (B. S. S. 205)
वेदान्तसूत्रलघुवार्तिकापरनामायं ग्रन्थः आनुष्ठुभेण छन्दसा रचितः, कुत्र चिदेकेन पादेन एकैकस्याधिकरणस्य सङ्ग्राहकः मुद्रितश्च ।
२. न्यायसुधा (B. S. S. 205)
लघुवार्तिकव्याख्यात्मकोऽयं ग्नर्थः लघुवाक्यसुधा लघुन्यायसुधा इत्यपि व्यवह्नियते ।

२५. उत्तमज्ञयतिः (1100-1200 A.D.)
``यन्नामश्रवणाद्भीता वादिनो मोहिता भृशम् ।
तस्मै ज्ञानोत्तमाख्याय जगन्मोहभिदे नमः ।।"
इति तत्वशुद्धिव्याख्यायां दर्शनात् उत्तमज्ञस्यास्य गुरुर्ज्ञानोत्तम इति ज्ञायते । ज्ञानोत्तमोऽयं चोलदेशीयात् मङ्गलग्रामवासिनः ज्ञानोत्तमाद्भिन्नः न्यायसुधादिग्रन्थ प्रणेता विज्ञानात्मचित्सुखज्ञानगिरिगुरुरिति च परिशीलनाज् ज्ञायते । ज्ञानघन शिष्यस्य ज्ञानोत्तमस्य काल (1100-1200 A.D.) एवाञ्चास्यापि कालस्य एव ।
श्रीकण्ठशास्त्रिणां मतेन (IHQ Vol. XIV) ज्ञानघनः प्रकाशात्मनः सामयिक इति ज्ञानघनप्रशिष्यस्योत्तमज्ञस्य कालः (958-1038 A.D.) इति सिध्यति ।
%%% Chart
१. तत्वशुद्धिव्याख्या - (754 C, O. L. 291 TCD) ज्ञानघनविरचितस्य तत्वशुद्धिग्रन्थस्य व्याख्यात्मकोऽयं पूर्णग्रन्थः तिरुवन्तपुरपुस्तकाये लभ्यते ।
२. पञ्चपादिकाव्याख्या - वक्तव्यप्रकाशिका (56. A. Sringeri Math) अमुद्रितोऽयं ग्रन्थः शृङ्गगिरिमठपुस्तकालये लभ्यते ।



 ``" ``" ``" ``" ``" ``" ``" ``" ``" ``" ``" ``" ``" ``" ``" ``" ``" ``" ``"

ऽ  ?
``" ``" ``" ``" ``" ``" ``" ``" ``" ``" ``" ``" ``" ``" ``" ``" ``" ``" ``" ``" ``" ``"
`' `' `' `' `' `' `' `' `' `' 

ऽ  ।   ॥ ?
