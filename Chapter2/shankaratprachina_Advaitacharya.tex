\chapter{शङ्करात्प्राचीना अद्वैताचार्याः}
द्वितीयो भागः
अद्वैताचार्याः, अद्वैतग्रन्थप्रणेतारश्च ।
अद्वैताचार्याः अद्वैतमतप्रतिपादकग्रन्थविशेषप्रणेतारश्च कालभेदेन द्वेधा विभक्तुं शक्यन्ते -``शङ्करभगवत्पादेभ्यः प्राचीनाः, शङ्करभगवत्पादेभ्य अर्वाचीनश्चेति ।" तेष्वपि शङ्करभगवत्पादेभ्यः प्राचीनेषु आचार्येषु १. ब्रह्मसूत्रकारात् शङ्करभगवत्पादेभ्यश्च प्राचीनाः, २. ब्रह्मसूत्रकारात् अर्वाचीनाः शङ्करभगवत्पादेभ्यः प्राचीनाश्चेति विभागोऽपि कर्तुं शक्यते । अद्वैतमतसिद्धान्तस्य प्राचीनतमत्वात् ब्रह्मसूत्रेषु बहूनां वृत्तिग्रन्थानां सत्वानुमानाच्च । शङ्करभगवत्पादकृतभाष्यप्रभावात् बौद्धप्रभावाच्च प्राचीनं वृत्यादिकं विनष्टम् । अथवा शाङ्करभाष्येण तेषां सम्पूर्णतया गतार्थत्वात् तेषां संरक्षणे औद सीन्यवशाच्च विनष्टं जातम् । तत्वचन्द्रिकाकारेणोमामहेश्वरेण तत्वचन्द्रिकायां (R 5156 MGOML) ``शङ्करभगवत्पादैः स्वीये भाष्ये एकोनशतं सृत्रवृत्तिग्रन्थाः परामृष्टाः, केचित् खण्डिताश्चेति" निर्दिश्यते । शाङ्करभाष्यदर्शकाणां सर्वेषाञ्च मनसि ईदृशी चिन्ता स्वाभाविकी यत् `शाङ्करभाष्ये शङ्कराचार्येण ततोऽपि प्राचीनवृत्तिकारस्य व्याख्यानं बहुषु स्थलेषु खण्डितम् । तदर्थं तत्र तत्र युक्तिरपि प्रदर्शिता । परन्तु तासां वृत्तीनां नाम, वृत्तिकारादीनां नाम वा न कुत्रापि निर्दिष्टम् । परन्तु व्याख्यानादिकर्तृभिः तेषां नाम तत्र तत्र निर्दिष्टं क्वचित् क्वचित् । शाङ्करभाष्ये पूर्वपक्षत्वेन गृहीता एव भास्कररामानुजादिभिः सिद्धान्तत्वेन स्वीकृता वर्तन्ते । तस्मात् शङ्करभाष्ये पूर्वपक्षत्वेनोपन्यस्तानां मतवादानां मध्ये कियन्तो मतवादाः पूर्वैरुद्भाविताः? कियन्तो वा बौधायनादिभिः प्रकल्पिताः? कति वा शङ्करभगवता स्वयं समुद्भाविताः? किं ते शाङ्करभाष्यानुगाः ? उत शङ्करसिद्धान्तेन अंशतस्सदृशाः ? इत्यादिविषया न निर्णेतुं शक्यन्ते । तथा च बोधायनोपवर्षादिभिस्सह भास्कररामानुजादीनां सम्बन्धो यथा दुर्निर्णेय स्तथा वृत्तिकारादीनामपीत्येवावगम्यते । एवमपि बोधायनोपवर्षभर्तृप्रपञ्चभर्तृहरि ब्रह्मनन्दिसुन्दरपाण्डयद्रविडाचार्यब्रह्मदत्तादिवेदान्ताचार्याः ब्रह्मसूत्रेषु वृत्तिग्रन्थान् प्राणैषुरिति परं ज्ञायते । व्याख्यानपरम्परैवात्र प्रमाणम् । एतेषां ग्रन्थानामिदानीमनुपलम्भात् ।'
एवं ब्रह्मसूत्रेवपि बादरायणव्यासेन केचनाचार्याः नाम्ना निर्दिष्टाः . तेषाञ्च ग्रन्था नोपलभ्यन्ते । तथापि तेषां मतवादादिकं विविधपत्रिकादिप्रमाणानुसारं यथाकथञ्चित विवृणीतुं प्रयतामहे । तेषु आचार्येषु ``अष्टावक्र - आत्रेय - आश्म रथ्य - उपवर्ष औहुलौमि - काशकृत्स्न-कार्ष्णाजिनि-जैमिनिबादर्यादयः व्यासात्प्राचीनाचार्यविभागे, आचार्यसुन्दरपाण्डूय द्रविडाचार्य ब्रह्मदत्त ब्रह्मनन्दि भर्तृप्रपञ्च भर्तृ रि बदिरायणापरव्यासाचार्याः शङ्करात्माचीनाचार्यविभागे, च विभक्तुं शक्यन्ते । एवमनिर्णीताः अज्ञातसमयाः विभिन्नविचारलक्ष्यीभूता अपरे अध्यात्मरामायणकारआदिशेषकाश्यपजाम्बवतदत्तात्रेययोगवासिष्ठ कारशु कसनत्सुजातादयश्च वर्तन्ते । तेषां सर्वेषां इतिहासादिकं प्रकरणेऽस्मिन् वर्णम लाक्रमेण निरूप्यते ॥"

१. अष्टावक्रः
अष्टावक्रोऽयं महर्षिः सुजाताकहोलयोः पुत्रः, उद्दालकस्य दौहित्रः, श्वेतकेतोस्स्वस्रीयः, वदान्यजामाता, सुपभाभर्तेति ज्ञायते । उद्दालकनामा महान् ऋषिंः वहोलाय स्वशिष्याय स्वकन्यां सुजातानाम्नीं वैवाहिकेन विधिना ददौ । शिप्यतध्ये अधीय नं स्वपितरं कहोलं सुजातागर्भस्थः अग्निकल्पशिशशुः रात्रिन्दिवं कहो रुस्य ध्ययनशीलतां च प्रति अधिचिक्षेप । क्रुद्धः मातामह उद्दालकस्स्वदौहित्रं ``यस्मात त्वं कुक्षौ वर्तमानोऽधिक्षिपसि तस्मात् त्वं अष्टकृत्वः वक्रो भवितासि" इति शशाप । तथा स वक्र एवाभ्यजायत अष्टावक्र इति प्रथितश्च । स्वपत्नीप्रेरणया वित्तार्जनार्थं जनरूपुरं गत अष्टावक्रपिता कहोलः जनकपुरद्वारपालेन बन्दिना वादे पराजितः जले मिमज्य मृतश्च । मातृपकाशात् पितृवृत्तान्तं श्रुत्वा मातुलेन श्वेतकेतुता सह अष्टावक्रः जनकपुरं गतः, वादेषु द्वारपालं जनकञ्च जित्वा ``समङ्गापरनाम्नीं मधुविलानदी स्नात्वा अष्टावक्रेदेहं समीकृत्य जरर्तरूपधारीण्या उत्त दिगभिपानिदेवतायाः धर्मो देशं स्वीकृत्य वदान्यकन्यां सुप्रभानाम्नी विधिनोपयेमे । इति कथा महाभारतादिप्रसिद्धा । अनेन कृतः ग्रन्थः-"
(क) अष्टावक्रगीता
अष्टावक्रसृक्तम्, अवधूतानुभूतिः, इत्यादिनाम्ना प्रसिद्धोऽयं ग्रन्थः ए विंशतिभिाध्यायैः पूर्ण अष्टवक्रजनकसंवादरूपेण अद्वैतवेदान्तसिद्धान्तान् ब्रह्मणोऽद्वितीयत्वं चिन्मयत्वञ्च प्रतिपादयति । ग्रन्थोऽयं आष्टेकरकम्पनि पूनानगरे मुद्रितश्च । अस्य व्य ख्याः विश्वेश्वरकृता - दीपिघख्या, पूर्णानन्दतीर्थकृता काचन व्याख्या, मुकुन्दमुनिकृता अन्या व्याख्या, भासुरानन्दकृता अपरा व्याख्या इत्येवं चतस्रः व्याख्या उपलभ्यन्ते ।
२. आचार्य सुन्दरपाण्ड्यः (600 A.D.)
शङ्करभगवत्पादेभ्यः प्राचीनोऽयं सुन्दरपाण्डयाचार्यः दक्षिणद्रविडदेशीयः, मधुरानगरवासीति ज्ञायते । पूर्वोत्तरमीमांसयोः प्रकाण्डपण्डितेनानेन ब्रह्मसूत्रणां वार्तिकं विरचितमिति, कैस्तवीय षष्ठशतकात् प्राचीनः, षष्ठशतकीयो वा इति निश्चीयते । अत्रेमानि कारणानि-
स्वर्गीयमहामहोपाध्याय कुप्पुस्वामिशास्त्रिणः ``जर्नल आफ ओरियण्टल रिसर्च मद्रास" पत्रिकायाः प्रथमे भागे (J.O.R.I.) एवमभिपयन्ति । शङ्करभगवत्पादैस्स्वोये ब्रह्मसूत्रमाष्ये सनन्वय - अधिकरणभाष्यान्ते अपिवाहुः-
``गौणमिथ्य त्मनोऽपत्वे पुत्रदेहादिबाधनत् ।
सद्ब्रह्म त्माहं इत्येव बोधकार्थं कथं भवेत् ॥
अन्वेष्टव्यात्मबिज्ञानात् प्राक् प्रमातृत्वमात्मनः ।
अन्विष्टस्स्यात् प्रमातैव पाप्मदोषादिवर्जितः ॥
देहात्मप्रत्ययो यद्वत् प्रप्ताणत्वेन कल्पितः ।
लौकिकं तद्वदेवेदं प्रनाणं त्वात्मनिश्चयात् ॥" इति 
श्लोकत्रयमुदाहृतम् । अत्र भामतीकारैः ``अत्रैव ब्रह्मविदां गाथामुदाहरति" इत्यवतारितम् । पञ्चपादिकाकारैः ``प्रसिद्धपेतत् ब्रह्मविदां पूर्वोक्तं न्याय संक्षेपतः श्लोकैस्संगृह्नणाति" इत्यवतारितम् । पञ्चपादिकाव्याख्यात्रा नरसिम्हस्वरूपशिप्येण आत्मस्वरूपेण स्वीयप्रबोवपरिशोधिन्यां `श्लोकत्रयं सुन्दरपाण्डयाचार्यप्रणीतं प्रमाणयति' इत्यवतारितम् ।
माधवनन्त्रिणा विरचितायां सूतसंहितयाः व्याख्यायां तात्पर्यदीपिकाख्यायां ``देहात्मप्रत्ययो यद्वत् प्रप्ताणत्वेन कल्पितः । लौकिकं तद्वदेवेदं प्रमाणन्त्वात्मनिश्चयात् ।" इत्ययं श्लोक उद्धृतः । एतच्छ्रलोकविवरणावसरे तथा सुन्दर पाण्ड्यवार्तिकमपीति अवतारिका कृता । 
एवं त्रयोदशशतकीयेनामलानन्देन स्वरविते कल्पतरुग्रन्थे (3 - 3 - 25 Page 755 N.S.P. Edn.) ``आह चात्र निदर्शनमाचार्यसुन्दरपाण्ड्यः" इति-
निःश्रेण्यारोहणप्राप्यं प्राप्तिमात्रोपपादि च ।
एकमेव फलं प्राप्तुं उभावारोहतो यदा ॥
एकसोपानवर्त्येको भूमिष्ठश्चापरस्तयोः ।
उभयोश्च जवस्तुल्यः प्रतिबन्धश्च नान्तरा ॥
विरोधिनोस्तदैको हि तत्फलं प्राप्नुयात्तयोः ।
प्रथमेन गृहीतेऽस्मिन् पश्चिमोऽवतरेन्मुधा ॥ इति 
श्लोकत्रयमुपपादितम् । एतदेव श्लोकत्रयं कुमरिलभट्टैर्बलाबलाधिकरणे तन्त्रवार्तिके (BSS Page 852-853) आह चेत्यादिना प्रतिपादितम् । एवञ्च शङ्करभगवत्पादेभ्यः कुमरिलभट्टाच्च प्राचीन इति सिध्यति ।
आचार्यसुन्दरपाण्ड्यकृतः नीतिद्विषष्ठिकाख्यः ग्रन्थः कश्चन नीतिपरः मद्रपुर्यां प्रकाशितः । तत्रस्थाः बहवः श्लोकाः त्रयोदशशतकीयेन जल्हणेन सूक्तिमुक्तावल्यां, पञ्चदशशतकीयेन वल्लभदेवेन सुभाषितावल्यां, पञ्चदशशतकीयेन शार्ङ्गधरेण स्वीयशार्ङ्गधरपद्धत्यां पोतयार्येण प्रसङ्गरत्नावल्यां, पेद्दिभट्टेन सूक्तिवारिधौ, द्वादशशतकीयेन कलिङ्गराजापरनाम्ना सूर्यपण्डितेन कुलशेखरसूक्तिरत्नहाराख्ये ग्रन्थे चोदाहृताः । विशेषतः सूर्यपण्डितेन ``आचार्यसुन्दरपाण्ड्यकृता" इति निर्दिष्टाश्च । पञ्चतन्त्रकर्ता विष्णुशर्मा क्रैस्तवीयषष्ठशतकादर्वाचीन इति विमर्शकसिद्धान्तः । तेनापि नीतिद्विषष्ठिकास्थाः 29, 30, 48, श्लोकाः स्वीये ग्रन्थे उद्धृताः । एवं नीतिद्विषष्ठिकाग्रन्थावसाने कश्चन श्लोकः - ``इमां काञ्चनपीठस्थां समेत्य कवयो भुवि । आर्यां सुन्दरपाण्ड्यस्य स्नापयन्ति वधूमिव ।" दृश्यते । क्रैस्तवीय षष्ठशतकादारभ्य द्रविडेदेशे मधुरायां द्रविडसङ्घस्स्थापितः । अभ्यर्हिंत कविं तत्कृतिञ्च तत्सङ्घस्थाः विद्वांसः सङ्धपूजिते काञ्चनपीठे निवेश्य कनकाभिषेकमकुर्वन्निति तमिलसाहित्ये प्रसिद्धम् । तादृशस्तत्कारः ग्रन्थस्यास्यनीतिद्विषष्ठिकाख्यस्यापि प्रवृत्त इत्येवास्मात् पद्यात् ज्ञायते ।
``श्रीमच्छकाब्देऽब्धिशशिसायकसम्मिते । राजा माधववर्माभूत् विख्यातो धरणीतले ॥" इति प्रसङ्गरत्नावल्याख्ये पोतयार्यकृते ग्रन्थे दर्शनात्, पेण्डयाल सुब्रह्मण्यशास्त्रिभिः प्रकाशितात् पुलिवूरुशिलाशासनप्रमाणाच्च 514 शके 592 A.D. काले विष्णुकुण्डिनवंश्यः माधववर्मापरनामा जनाश्रयाख्यः कृष्णानदीतीरान्ध्रदेशाधीश आसीदिति ज्ञायते । तेन राज्ञा कृता कृतिः जानाश्रयीति च प्रसिद्धा । तस्मिन् जानाश्रयीत्यपरनामके छन्दोग्रन्थे नीतिद्विषष्ठिकायाः चतुर्विशतितमः ``चारित्रनिर्मल जलः सत्पुरुषनदोऽक्षयो भवतु नित्यम् । यस्य विभवारविन्दे विद्वद्भमराः कृतविनोदाः ॥" इति शलोक उद्धृतः ।
तस्मात्-कुमरिलभट्टेन, शङ्कराचार्येण, विष्णुशर्मणा, जनाश्रयेण, माधवमन्त्रिणा च प्रमाणीकृतोऽयं सुन्दरपाण्ड्याचार्य क्रैस्तवीयषष्ठशतकात् प्राचीनः, षष्ठशतकीयो वेति निश्चप्रचमभ्युपगम्यते ।
स्वर्गीय महामहोपाध्याय कुप्पुस्वामिशास्त्रिण आचार्यसुन्दरपाड्यमेनं क्रैस्तवीयाष्टमशतकीयं प्रवदन्ति । तमिलसाहित्ये प्रसिद्धः (अषकेसरी) एवायमिति जर्नल आफ ओरियण्टल पत्रिकायाः प्रथमे भागे J. O. R. I. निरूपयन्ति । इतिहासनिपुणाः K. A. नीलकण्ठशास्त्रिणस्तु जर्नल आफ ओरियण्टलरिसर्चपत्रिकायाः प्रथमे भागे (J. O. R. Madras-1) सप्तमशतकमध्यकाले सुन्दरपाण्ड्याचार्य आसीदिति प्रतिपादयन्ति । सर्वथापि शङ्करभगवत्पादेभ्यः प्राचीनोऽयमित्येवास्मत्सिद्धान्तः । ``श्रीमान् सुन्दरपाड्यः श्रुति स्मृति प्रसृत सत्पदार्थज्ञ" इति नीतिद्विषष्ठिकायां दर्शनात आचार्यसुन्दरपाण्ड्योऽयं पूर्वोत्तरमीमांसादिषु निष्णात इति ज्ञायते । शाङ्करभाष्ये उद्धृत्य प्रमाणीकृतत्वात् वेदान्तेऽनेन ब्रह्मसूत्राणां किमपि वार्तिकं पद्यबंद्ध कृतं स्यादिति निश्चीयते । कुमरिल भट्टैरुद्धृतोऽयं पूर्वमीमांसायामपि ग्रन्थप्रणेता इत्यभ्यूह्यते ।
अस्यैवाचार्यसुन्दरपाण्ड्यस्य द्रविडाचार्य इत्यपि नामान्तम्, व्यावहारिकनाम वा स्यादित्यूह्यते । अत्रेदं कारणं भवति-अभिनवद्रविडाचार्यापरनाम्ना अष्टादशशतकीयेन बालकृष्णानन्दसरस्वत्या विरचिते पद्यबद्धे शारीरकमीमांसाभाष्यवार्तिके आशुतोषग्रन्थमालामुद्रिते शङ्करभाष्यस्थस्य `अपिचे'ति ग्रन्थस्यावतरणसमये ``कथितार्थपरं द्रविडार्यकृतां अपिचेति गुरुर्वदतीह कथाम्" (ASI Page 403) निर्दिष्टम् । तस्मात् आचार्यसुन्दरपाण्ड्योऽयं द्रविडाचार्य इत्यपि व्यवहृतस्स्या दिति निश्चीयते ।
३. आत्रेयः
व्यासात्पूर्वतनेषु वेदान्ताचार्येषु अन्यतमोऽयमात्रेयः । बादरायणव्यासनिमिंतेषु ब्रह्मसूत्रेषु स्वामिनः फलश्रुतेरित्यात्रेयः 3-4-44 इति आत्रेयोऽयं निर्दिष्टः । यज्ञे अङ्गाश्रीतोपासना यज्ञस्वामिना एवं ऋत्विग्भिश्च कर्तव्या । अत्र फलविषये संशयः । किं उपासनाजन्यफलभाक् यजमानः ? उत ऋत्विक् ? इति । अत्रात्रेयमतन्तु अङ्गाश्रितोपासनाफलभाग्यजनमान एवेति । अदसीयः वेदान्तग्रन्थस्तु नोपलभ्यते ।
४. आदिशेषः (परमार्थसारकारः)
``वेदान्तशास्त्रमखिलं विलोक्य शेषस्तु जगदाधार" इति परमार्थमारे दृश्यते । साघवानन्दकृतायां व्याख्यायां ``भगवता जगदाधरेण आदिशेषेण" इति दृश्यते । तस्य त् सहस्रफणामणिमणिमण्डल आदिशेष एवास्य परमार्थसारस्य कर्तेति साम्प्रदायिकविश्व सः । परन्तु विधुशेखरभट्टाचार्यास्स्वपम्पादिवि ``गौडपदीयं आगमशास्त्र" मिति ग्रन्थे गौडपादाचार्यकालात् भास्कराचार्यकालस्य च मध्यवर्तिना केतापि आदिशेषनाम्ना ग्रन्थरचना कृतेति परमार्थसारग्रन्थकारकालः (500-800 A.D.) इति प्रवदन्ति ।
परमार्थसारः - (TSS 12)
आर्यावृत्तघटितैः पद्यैरद्वैतपरमार्थसारान् शिष्योपदेशशैल्यां प्रतिपादयन्नयं ग्रन्थः अभिनवगुप्ताचार्यकृतात् परमार्थसाराद्भिन्नः मुद्रितश्च चौखाम्बामुद्रणालये । अनन्तशयनग्रन्थावल्याञ्च सव्याख्योऽयं मुद्रितः । अस्य व्याख्याः- १ राघवानन्द मुनिकृता विवरणनाम्नी काचन २ वासुदेवयतिकृता अन्या प्रकाशि कानाम्नी व्याख्या अमुद्रिता (R. 4149 C. MGOML) लभ्यते ।
५. आश्मरथ्यः
ब्रह्मसूत्रकारात् बादरायणव्यासात् पूर्वतनोऽयं आश्मरथ्यः ब्रह्मसूत्रेषु वैश्वानराधिकरणे ``अभिव्यक्तेरित्याश्मरथ्यः 1-2-29, एवं वाक्यान्वयाधिकरणे प्रतिज्ञासिद्धेर्लिङ्गमित्याशमरथ्यः" 1-1-29 इति वारद्वयं निर्दिष्टः ।
उपनिषत्सु ईश्वरः प्रादेशमात्रः प्रतिपादितः । अस्य उपपतिरनेनैवं क्रियते ``परमेश्वर अनन्तः । भक्तानुग्रहार्थं प्रादेशमात्रादुद्भवति । हृदयादिषु उपलब्धियोग्येषु प्रदेशेषु उपलभ्यमानोऽयमिति प्रादेशमात्र इति च । भेदाभेदवाद्ययम् । कार्यावस्थायां विज्ञानात्मा परमात्मनः भिन्नः । कारणावस्थायान्तु अभिन्न इत्यस्य सिद्धान्त इति ज्ञायते ।"
६. उपवर्षाचार्यः (100 BC - 200 AD)
उपवर्षाचार्योऽयं पूर्वोत्तरमीमांसयेर्वृत्तिकारः, शङ्करभगवत्पादेभ्यः शबरस्वामिनोऽपि प्राक्तन इति ज्ञायते । शङ्करभगवत्पादैस्स्वीये सूत्रभाष्ये ऐकात्म्याधिकरणे 3-3-53 सूत्रे ``अत एव च भगवतोपवर्षेण प्रथमतन्त्रे आत्मास्तित्वाभिधानप्रसक्तौ शारीरके वक्ष्याम" इत्युद्धारः कृत इति सबहुमानमुपवर्षाचार्य आवेदितः । प्रकटार्थकारैरपि ``अतएवेत्य दि" भाष्यव्याख्यानावसरे `वृत्तिकारवचनं गमकमित्याह इत्येव शाङ्करभाष्यमवतारितम् । एवमानन्दमयाघिकरणे शङ्कराचार्यैः प्रथमं वृत्तिकारमतानुसारेण सूत्राणि व्याख्यातानि । अनन्तरं' `इदं त्विह वक्तव्यम्' इत्यादिना अधिकरणान्ते वृत्तिकारमतं पूर्वपक्षीकृत्य सिद्धान्तविधया स्वीयसिद्धन्तः प्रदर्शितः । एवमन्यत्रापि `अन्ये त्वाहुः' `अपरे त्वाहुः' इत्यादिना शङ्करभगवत्पादर्वत्तिकारमतमनूदितम् ।
शबरस्वामिना च ``वर्णा एव तु शब्दाः" इति भगवानुपवर्ष इति उपवर्षाचार्यः प्रमाणीकृतः । एवञ्च ब्रह्मसूत्राणां वृत्तिकार उपवर्षाचार्य अद्वैतमतैकदेशी प्र. चीन इति निर्णीयते । तादृशी उपवर्षीया वृत्तिस्तु नोपलभ्यते कुत्रापीदानीम् ।
ग्रन्थानुपलव्धेरेव विशिष्टाद्वैतिनः उपवर्षाचार्यस्यैव बोधायनकृतकोटिरित्यपि नामान्तरमिति वर्णयन्तस्स्वमतसंरक्षकबोधायनवृत्तिसत्यत्वसंरक्षणाय मणिमेखलादि द्रविडभाषाग्रन्थअवन्तिसुन्दरीग्रन्थप्रपञ्चहृदयग्रन्थमुखेन बोधायनकृतकोटिउपवषत्रयस्य अभिन्नतां साधयितुं प्रकटप्रयत्नमकुर्वन् । परन्तु तेषां प्रयत्नो विफल इति बोध यनकृतकोटिउपवर्षाचार्याः भिन्ना एवेति बोधायनवृत्तिस्तु नास्त्त्ये वेते ब्रह्मश्री पोलकं श्रीरामशास्त्रिभिरस्मद्गुरुचरणैः स्वीये द्रविडात्रेयदर्शने प्रतिपादितम् ।
आचार्य भगवद्दत्तैस्तु स्वीये ``भारतवर्ष का बृहद् इतिहास" नामके ग्रन्थे प्रथमभागे 84 पुटे शबरस्वामिनां कालः विक्रम चतुर्थशतकात् प्राचीन इति प्रतिपादितम् । तस्मादुपवर्षकालः 200 A.D. कालात् प्राक्तन इति तु निर्णीयते ॥
उपवर्षः वर्षोपाध्यायस्य कनिष्ठभ्राता पाणिनीयवृत्तिकृत् कात्यायनस्य श्वशुरः, उपकोशाया जनकः, महापद्भनन्दस्य प्रधानमन्त्रीति 500 A. D. काले आसीदिति च ``वृद्धत्रय्यां" गुरुपादशर्महालदारः ।
७. औहुलोमिः
व्यासात्पूर्वतनेषु आचार्येषु औडुलोमिरप्यन्यः । औडुलोमिरयं ब्रह्मसूत्रेषु वाक्यान्वयाधिकरणे ``उत्क्रमिष्यत एवं भावादौडुलौभिः" 1 - 4 - 21 इति, स्वाम्यधिकरणे ``आर्त्विज्यमौडुलोमिस्तस्मै हि परिक्रीयते" 3 - 4 - 45 इति, ब्राह्माधिकरणे ``चितिमात्रेण तदात्मकत्वादित्यौडुलोमि" 4 - 4 - 6 इति च स्थलत्रये निर्दिष्टः ।
संसार-मोक्षकालभेदेन जीवब्रह्मणोर्भेदाभेदवादी अयमौडुलोमिः । दृश्यप्रपञ्चेऽज्ञानवशात् जीवब्रह्मणोर्भेदः । मुक्तावस्थायान्तु उभयोरप्यभेद इत्यस्य मतं स्यादित्युह्यते । भामतीकारोऽपि मतमेतदीयं प्रतिपादयति । 
८. काशकृत्स्नः
अविकृतः परमेश्वरो जीवः, नान्य इति सिद्धान्तवाद्ययं काशकृत्स्नः ब्रह्मसूत्रेषु वाक्यान्वयाधिकरणे ``अवस्थितेरिति काशकृत्स्नः" 1 - 4 - 22 निर्दिष्टः ॥
९. काश्यपः
नायं ब्रह्मसूत्रकारैर्निर्दिष्टः । परन्तु ``शाण्डिल्यभक्तिसूत्रे तामैश्वर्यपरां काश्यपः परत्वात्" No. 29 इति निर्दिष्टः ।
१०. काष्णाजिनिः
छान्देग्योपनिषदां पञ्चमाध्याये श्रूयमाणस्य ``रमणीयचरणा" इति ग्रन्थस्य व्याख्यानावसरे कार्ष्णाजिनिमतं ब्रह्मसूत्रे निर्दिष्टम् । कृतात्ययाधिकरणे ``चरणादिति चेन्नोपलक्षणार्थेति कार्ष्णाजिनि" 3 - 1 - 9 सूत्रेण निर्दिष्टः ॥
११. जाम्बवान्
एतत्कृतत्वेन प्रसिद्धस्य प्रणवमहाभाष्याख्यग्रन्थस्य उपान्त्यवाक्यात् रामचन्द्रभक्तो भगवान् जाम्बवानेवायमिति प्रतीयते । यद्येवं तर्हि प्रजापति पुत्रोऽयं त्रेतायुगादारभ्य वर्तमानश्चिरञ्जीवी जाम्बवतीपिता भगवतः कृष्णस्य श्वशुरश्चेति निर्णेतुं शक्यते ।
प्रणवमहाभाष्यम् -
प्रणवार्थप्रकाशकोऽयं ग्रन्थः माण्डूत्योपनिषदन्तर्गतं ओङ्कारोपासनार्थ प्रदर्शयति । आह्निकत्रयपूर्ण अमुद्रितोऽयं ग्रन्थःतिरुवनन्तपुरपुस्तकालये 306 TCD दृश्यते ॥
१२.जैमिनिः
जैमिनिरयं ब्रह्मसूत्रेषु वैश्वानराधिकरणे देवताधिकरणे बालाक्यधिकरणे फलाधिकरणे पुरुषार्थाधिकरणे परामर्शाधिकरणे तद्भूताधिकरणे कार्याधिकरणे ब्राह्माधिकरणे अभावाधिकरणे च निर्दिष्टः । बादरायणस्य साक्षाच्छिष्यः 300 B.C कालात्पूर्वतन इति च सिद्धान्तः ।
१३. दत्तात्रेयः
पातिव्रत्यधर्मपरायणा अनसूया स्वपतिं अत्रिमुनिं स्वशिरसि वहन्ती निशीथे स्वाश्रमात् देशान्तरं जगाम । सूचिभेद्ये तमसि मध्येमार्गं गच्छन्ती सा शूलारोपितं माण्ढव्यमबुध्वा स्वपतिं माण्ढव्यशरीरे घर्षितवती । निष्कारणं पीडामुत्पादयन्तं कमित्यज्ञात्वा माण्ढव्यः ``सूर्योदयादतन्तरं पीडोत्पादकस्य मृतिर्भवतु" इति शशाप । पतिपरायणाऽनसूया ``सूर्योदय एव मा भूदिति" शशाप । अन्धकारावृते च जगति, सूर्ये च अनुदिते यज्ञक्रियादिकर्मलोपात् भीताः देवाः ब्रह्मणा प्रेरितास्सूर्योदयाय अनसूयां प्रार्थयामासुः । उदिते च सूर्ये अनसूयायाः पातिव्रत्यधर्मेण तुष्टेन इन्द्रेण प्रार्थितः भगवान् विष्णुरनसूयायां अत्रेः पुत्रत्वेनावततार । सोऽयं पुत्रः दत्तात्रेयः । दत्तात्रेयस्य प्रसादेन कार्तवीर्यार्जुतस्सचराचरं भूमण्डलं शश स । दत्तात्रेयः । दत्तात्रेयस्य प्रसादेन कार्तवीर्यार्जुतस्सचराचरं भूमण्डलं शश स । दत्तात्रेयः निमिनामकस्य पिता श्रीमतः पितामहश्चेति कथा महाभारते सभापर्वणि अनुशासनपर्वणि च प्रसिद्धा ।
(क) वेदान्तसारः -
क्वचित् क्वचित् अस्यैव अवधूतगीता इति नामान्तरमिति च दृश्यते । दत्तात्रेयकार्तिकेयंसवादरूपेऽस्मिन् ग्रन्थे प्रथमपरिच्छेदे अद्वैतब्रह्मवर्णना, द्वितीयादारभ्य सप्तमपरिच्छेदान्तं स्वात्मसंवित्युपदेशश्च दृश्यते । अमुद्रितोऽयं सम्पूर्णग्रन्थः सरस्वतीमहालये (7589 TSML) दृश्यते ।
(ख) अवधूतगीता -
दत्तगीता, जीवन्मुक्तिगीता, इत्यादिकं नाम अस्यैव ग्रन्थस्य दृश्यते । गोरक्षदत्तात्रेयसंवादरूपेऽस्मिन् ग्रन्थे जीवन्मुक्तस्य स्वरूपं सम्यगुपवर्णितम् । मुद्रितश्चायं ग्रन्थ अ ष्टेकरकम्पनि पूनानगरे गीतासङ्ग्रहे । अस्याः व्याख्या पर मानन्दगीर्थकृता अमुद्रिता वर्तते ।
(ग) प्रबोधचन्द्रिका -
ग्रन्थोऽयं मध्यप्रान्तीयबरार्ग्रन्थसूच्यां दृश्यते ।
(घ) स्वात्मसंवित्युपदेशः-
ग्रन्थोऽयममुद्रितः बरोडापुस्तकालये (996 BRD) लभ्यते । जीवन्मुक्तलक्षणमप्यस्य कृतिरिति ज्ञायते ।
१४. द्रविडाचार्याः
अद्वैतसम्प्रदायप्रवर्तकेषु पूर्वाचार्येषु अन्यतमा एते द्रविडाचार्या एतद्युगारम्भ एव सम्भूताः । एते भाष्यकारा इति प्रसिद्धाः । छान्दोग्योपनिषदां अर्थविवरणात्मकं सूत्ररूपवाक्यनिचयपण्डितं वाक्यनामकं ग्रन्थं ब्रह्मतन्दिनः प्राणैषु । द्रविडाचार्यैः स्वव्याख्येयवाक्यग्रन्थानुपारं सविशेषनिर्विशेषभेदेन द्विरूपं ब्रह्म न्यरूपि ।
बृहदाण्यकोपनिषद्भाष्ये द्वितीयाध्यायप्रथमब्राह्मणविवरणे शङ्कराचार्यैः द्रविडाचार्याः प्रमाणीकृताः । ``अत्र हि सम्प्रदायविदः आख्यायिकां सम्प्रचक्षते" इति । आनन्दगिरिणापि ``तत्वमस्यादिवाक्यमैक्यपरं, तच्छेषस्सूष्टय दिवाक्यम्" इत्युक्तेऽर्थे द्रविाडाचार्यसम्मतिमाह - ``अत्र चेति" इत्यवतारिका प्रदीयते । बृ दाण्यकवार्तिकेऽपि - ``आचक्षते तथाचात्र केचिदाख्यायिकां शुभाम् । यथाभिलषितार्थोऽयं यथा सम्भाव्यते स्फुटः ।" इति । अत्राप्यानन्दगिरिणा ``द्रविडाचार्य प्रणीतामाख्य यि कामवतारयति" इत्यवतारिता । एवं तोटकाचार्यः श्रुतिसारसमुद्धरणे विषयेऽस्मिन्नेव द्रविडाचार्यान् निर्दिशति -``द्रविडोऽपि च तत्वमसीति वचो विनिवर्तकमेव निरूपितवान् । शबरेण विवर्धितराजशिशोर्निजजन्मविदुक्तिनिदर्शनतः।" इति । एवं छान्दोग्योपनिषदश्शाङ्करभाष्योपक्रमे एवं दृश्यते - ``ओमित्येतदक्षरमित्यष्टाध्यायी छान्दोग्योपनिषत् । तस्याः संक्षेपत अर्थजिज्ञासुभ्यः ऋजुविवरणं अल्पग्रन्थमिदं आरभ्यते । इति । आनन्दगिरिणा " ``अथ पाठक्रममाश्रित्यापि द्राविडं भाष्यं प्रणीतम् , तत्किमनेन इत्याकाङ्क्षायामाह अल्पग्रन्थ" मिति अवतरणिका प्रदत्ता । मधुसूदनसरस्वत्या कृतायां संक्षेपशारीरकटीकायां ब्रह्मप्तन्दिविरचितवाक्यानां सूत्ररूपाणां भाष्यकर्ता द्रविडाचार्य इति निर्दिश्यते । नृसिम्हाश्रमिकृतायां संक्षेपशारीरकटीकायां ``भाष्यकृदद्रविडाचार्यवचनात्" इति निर्दिश्यते । रामतीर्थेनापि नन्दिकृतग्रन्थभाष्यकारः द्रविडाचार्य इति निर्दिश्यते । छान्दोग्यभाष्ये तृतीयाध्याये मधुविद्य विवरणे शङ्कराचार्यैः ``अत्रोक्तः परिहार आचार्यैः" इति आचार्यशब्देन द्रविडाचार्यः निर्दिष्टः । आनन्दगिरिणाऽपि द्रविडाचार्योक्तं उपपादयतीत्यवतारिका दीयते । सूत्रभाष्ये ज्योतिश्चरणाधिकरणे ``व्याचक्षत" इति शब्देन द्रविडाचार्याः निर्दिश्यन्ते । भामत्यां समन्वयाधिकरणे ``यथाहुर्दविडाचार्या इति द्रविडाचार्याः" प्रमाणीकृताः । सूत्रभाष्ये समन्वयाधिकरणभाष्यान्ते ``गौणमिथ्यात्मनोऽसत्वे पुत्रदेहादिबाधनात्" इत्यादि श्लोकत्रयमुद्धृत्य प्रप्ताणीकृतम् । बालकृष्णानन्दसरस्वत्या प्रकाशिते शारीरकमीमांसाभाष्यवार्तिके पूर्वोक्तभाष्यस्यावतारिकाप्रदानसप्तये ``कथितार्थपरां द्रविडार्यकृतां अपि चेति गुरुवर्दतीहकथाम् ॥" इति निर्दिश्यते । गौणमिथ्यात्मन इत्यादि पद्यं द्रविडाचार्यकृतमिति निर्दिष्टम् । म. ग. कुप्पुस्वामि शास्त्रिभिश्च प्राच्यभाषासंशोघनपत्रिकायाः प्रथमे भागे J. O. R. Vol I Madras पूर्वोक्तपद्यत्रयं आचार्यसुन्दरपाण्ड्यकृतमिति निरूपितम् । द्रविडाचार्यस्यैव सुन्दरपाण्ड्य इति स्यान्नाम् ।
एवञ्चाद्वैताचार्यैः निर्दिष्टाः द्रविडाचार्याः नूतनसम्प्रदायप्रवर्तकत्वेन प्रसिद्धाः गौडपादसामयिकाः तत्समानमेव पूज्याः गौडपादाचार्या इव इमेऽपि सन्यासपरम्परायाः प्रवर्तका आसन् । स च सम्प्रदायश्शङ्कराचार्यैः स्वान्तेवासिषु केषुचन उरीकृत एव परन्तु भगवत्पादादिक्रमेण नोरीकृतः । किन्तु सम्प्रदायान्तरद्वारा । अत एव गौडपादानारभ्य सङ्कलितासु स्वाचार्यपरम्पारासु द्रविडाचार्यः न निर्दिष्टः इति प्रतिभाति । एतैः । बृहदारण्यकवाक्यभाष्यं 2 छान्दोग्यभाष्वमपि कृतं स्यात् ॥
१५.बादरिः
ब्रह्मसूत्रेषु वैश्वानराधिकरणे, कृतात्ययाधिकरणे कार्याधिकरणे अभावाधिकरणे च निर्दिष्टोऽय बादरिः । वैदिककर्मणि सर्वेषामधिकारं प्रादेशमात्रे दृदये वर्तमानत्वात् प्रादेशमात्र ईश्वर इति सिद्धान्तं, ``रमणीयचरणा" इति छान्दोग्यवाक्यस्थस्य चरणशब्दस्य कर्मपरत्वं, छान्देग्यस्य 4-15-5 य एनान् ब्रह्म गमयतीति बाक्यस्थब्रह्मशब्दस्य कार्यब्रह्मवाचकत्वं ईश्वरभावापन्नस्य विदुषश्शरीरेन्द्रियमनसां असत्वञ्च वदन् अयं बादरिर्व्यासात्प्राचीनेषु प्रसिद्धाचार्येषु अन्यतमः ॥
१६. ब्रह्मदत्तः
जीवाः ब्रह्मण उत्पद्यन्ते मोक्षपर्यन्तावस्थायिनश्चेति ब्रह्मदत्तः । वेदान्तेषु अस्य मतं औपनिषदाभासशब्देन व्यवहारार्हं भवति । ज्ञानकाण्डब्रह्मकाण्डयोर्मुख्यं फलभेकमेवेति ब्रह्मदत्तसिद्धान्तः । ``अहं ब्रह्मास्मि" इति वाक्यार्थज्ञानमेव मोक्षोपयोगीति ब्रह्मदत्तः । ब्रह्मदत्तपते उपनिषदः ध्याननियोगप्रधानाः । अद्वेतिनां मते मोक्षो दृष्टफलः । ब्रह्मदत्तमते मोक्ष अदृष्टफलः । ब्रह्मदत्तमते `तत्वमसि' वाक्यात् ``आत्मा वारे द्रष्टव्य" इति वाक्यमेवोपादेयार्हम् ।
ज्ञानकर्मसमुच्चयवादी अयं ब्रह्मदत्तः ब्रह्मसूत्राणां ज्ञानकर्मसमुच्चयपरां व्याख्यां कृतवान् स्यात् । ब्रह्मदत्तश्च यामुनाचार्येण सिद्धित्रयेऽनुपादेयत्वेन निर्दिष्टः । सुरेश्वराचार्यकृतनैष्कर्म्यसिद्धेः ज्ञानामृतकृतायां विद्यासुरभिनाम्न्यां व्याख्यायां मद्रासराजकीयहस्तलिखित पुस्तकालयस्थायां (R. 3354 MGOML) ब्रह्मदत्तः निर्दिष्टः । आनन्दगिरिणापि बृहदारण्यकसम्बन्धवार्तिकव्याख्यानावसरे (P. 220 ASS 16) ब्रह्मदत्तः निर्दिष्टः 
किमयं ब्रह्मदत्तः ज्ञानकर्मसमुच्चयवादी ? उत न ? इति हिरियण्णामहाशयैः (J. O. R. Vol. 12 Madras) पत्रिकायां विमृष्टम् ॥
१७. ब्रह्मनन्दी
एते ब्रह्मनन्द्याचार्या अद्वैताचार्येषु प्राचीनेषु अन्यतमाश्श्ङ्करात् पूर्वमासन्निति ज्ञायते । अदसीयाः ग्रन्था नोपलभ्यन्ते । तथापि अद्वैताचार्यैशशङ्करादर्वाक्तनैः प्रमाणीकृताः । 
भामतीव्याख्याने कल्पतरौ ``इयञ्चोगदानपरिणामादिभावा न विकारोभिप्रायेण" इत्यादि भामतीग्रन्थस्य व्याख्यानावसरे (Page 421 VVS Edn) ब्रह्मनन्दी निर्दिष्टः । संत्रक्षपशारीरके तृतीयपरिच्छेदे (श्लोक संख्या 217-221) ``आत्रेय वाक्यमपि संव्यवहारमात्रम्" इत्यादिना आत्रेय अत्रिगोत्रजः ब्रह्मनन्दी अनूदितः । मधुसूदनसरस्वत्या स्वटीकायां ``छान्दोग्यवाक्यकारेण ब्रह्यनन्दिना" इत्यवतारितम् । ``सिद्धन्तु निवर्तकत्वात्" इति वाक्यं शङ्कराचार्येः माण्डूक्योपनिषदां भाष्ये वैतथ्यप्रकरणे ``न निरोधो न चोत्पत्तिः" इति कारिकाविवरणावसरे निर्दिष्टम् । इष्टसिद्धौ (Page 72) ``सिद्धन्तु निवर्तकत्वादिति चोक्तं वाक्यं ज्ञानोत्तमकृतायां इष्टसिद्धिटीकायां ब्रह्मनन्दीयमिति निर्दिष्टम् । आनन्दगिरिणा योगवासिष्ठव्याख्यात्रा आनन्दबोधेन च द्रविडशब्देन ब्रह्मनन्दी निर्दिश्यते । पञ्चपादिकाविवरणेऽष्टमवर्णके ``सिद्धन्तुनिवर्तकत्वात्" इति वाक्यं प्रमाणत्वेन स्वीकृतम् । खण्डनखण्डखाद्यव्याख्यायां विद्यासागर्यां सिद्धन्तु निवर्तकत्वादितिवाक्यं उदाहृतम् । नृसिम्हाश्रमिकृतायां संक्षेपपशारीरकटीकायां ``ब्रह्मतन्दिनापिछान्दोग्यषष्ठव्याख्यानावसरे उक्तम्" इति निर्दिष्टम् । रामतीर्थकृतायां संक्षेपशारीरकटीकायां ``ब्रह्मनन्दिनाप्याचार्येण छान्दोग्य भाष्ये उक्तम्" नन्दिकृत भाष्यकारः द्रविडाचार्य इति च निर्दिश्यते । यामुुुनाचार्येण ``आत्मसिद्धौ" ``आचार्यटङ्क - भर्तृप्रपञ्च - भर्तृमित्र भर्तृहरि ब्रह्मदत्त - शङ्कर-श्रीवत्साङ्क - भास्करादि विरचितसितासितविविधनिबन्धनश्रद्धाविप्रलब्धबुद्धयः न यथावत् अन्यथा च प्रति पद्यन्ते ।" इति टङ्क अपरिग्र ह्यत्वेन निर्दिष्टः । टङ्क एव ब्रह्मनन्दीति श्रुतप्रकाशिकाचार्येण वेङ्कटनाथेन तात्पर्यदीपिकायां ``अत्र भाष्यकारः ब्रह्मनन्दिवाक्यव्याक्याता द्रविडाचार्यः" इति द्रविडाचार्यव्याख्येयग्रन्थकर्त टङ्काख्यः ब्रह्मनन्दीत्युक्तम् ॥
तस्मात् ब्रह्यनन्द्यभिन्नः टङ्काख्योऽयं अपरिग्राह्यत्वेन यामुनाचार्येण निर्दिष्टः विविर्तवादावलम्बी नाद्वैतमतविरोधीति स्पष्टं प्रतीयते । अनेन निर्मितः छान्देग्यवाक्यनामा ग्रन्थस्तु न कुत्रापि लभ्यते ॥
१८. भर्तृप्रपञ्चः
भर्तृप्रपञ्चोऽयं वेदान्तसाहित्ये शनैश्शनैर्म्लानयशास्सञ्जातः . शङ्कराचार्यसिद्धान्तात् भर्तृप्रपञ्चसिद्धान्तःभिद्यते । भर्तृप्रपञ्चः भेदाभेदवादी । शङ्कर अभेदवादी । भर्तृप्रपञ्चः ज्ञानकर्मसमुच्चयवादी । शङ्कराचार्यः ज्ञानवादी । मतस्य दर्शनग्रन्थानाञ्च जीवेश्वराणांं आत्मनश्च प्रतिपादने एव तात्पर्यमिति भर्तृप्रपञ्चः । भर्तृप्तपञ्चसिद्धान्तः प्रमाणसमुच्चयताम्नापि व्यवहर्तुं शक्यते । भर्तृप्रपञ्चः भोग एव मोक्षहेतुः न वैराग्यम् , वस्तुतत्वानुभव एव विरक्तेस्मुगमः पन्था इति चाभिप्रैति । ``तत्वमसि" वाक्यात् ``आत्मानमेवलोकमुपासीत" इति वाक्यमेव भर्तृप्रपञ्चमते उपादेयतरमिति ज्ञायते ।
ज्ञानकर्मसमुच्चयवादी भर्तृप्रपञ्चोऽय द्वैताद्वैतवादीति च निश्चयः । शङ्करभगवत्पादैरयं बृहदारण्यकभाष्ये ``औपनिषदम्मन्य" इति नाम्नानूद्य खण्डितः । आनन्दगिरिणापि भाष्यव्याख्यायां तत्र तत्र निर्दिष्टः । तस्मात् नायं विशुद्धाद्वैतवादी परन्तु अद्वैतैकदेशीति परं वक्तुं अर्हः । अनेन कठोषनिषदां बृहदाण्यकस्य च भाष्यमारचितं स्यादिति ज्ञायते ।
एनमधिकृत्य ``इण्डियन् आण्डिक्वैरिपत्रिकायां (I. A. Part 53 Page 77, 1924) हिरियण्णामहाशयेन सविस्तरं प्रतिपादितम् ।"
१९. भर्तृहरिः
वेदविदामलङ्कार इति प्रसिद्धोऽयं भर्तृहरिः ``नचागमादृते धर्मस्तर्केण व्यवतिष्ठते" इति वाक्यपदीयब्रह्मकाण्डे वदन् स्वस्य वैदिकधर्मावलम्बित्वं प्रकटयति । वाक्यपदीयकर्ता भर्तृः रिः वसुरातशिष्य इति ज्ञायते । ``ईस्टिङ् भारत यात्रा" ग्रन्थानुसारं भर्तृहरिसमयस्सप्तमशतकापरार्धावधिक (600-700 A. D.) इति निर्णीयते । भारतपण्डितमण्डलीप्रसिद्धा तु कथा- ``भर्तृहरिः विक्रमादित्यभ्राता" इति । तन्त्रवार्तिके कुमरिलभट्टः वाक्यपदीयं वाक्यं खण्डयति (1 - 3 - 871) काशिकायां (4 - 3 - 88) वाक्यपदीयः निर्दिष्टः । तस्मात् ताभ्यां पूर्वतन इति न संशयः । कुन्हन राजामहाशयस्तु (I. H. Q. Vol. XIV) भर्तृहरि पञ्चमशतकीयं वदन्ति । म. म. कुप्पुस्वामि शास्त्रिणः ब्रह्मसिद्धि भूमिकायां षष्ठशतकापरार्धादारब्धे सप्तमशतकापरार्धावधिके काले भर्तृहरिरासीदिति प्रवदन्ति । शबर स्वामिनोऽपि प्राचीनोऽयमिति भगवद्दत्तजीकृत वैदिकवाङ्मयेतिहासे दृश्यते ।
यामुनाचार्येण सिद्धित्रये भर्तृहरिकृतस्य सूत्रव्याख्यानस्य अनुपादेयत्व प्रदर्शनात् भर्तृहरिणापि ब्रह्मसूत्रवृत्तिः कृताह इति ज्ञायते । केचित्तु शब्दाद्वैत वादिनमेनं वदन्ति । वाक्यपदीयब्रह्मकाण्डमेव भर्तृहरिणः वेदान्तित्वे प्रमाणम् ।
२०. वाल्मीकिः (योगवासिष्ठकारः)
अध्यात्मविद्यायाः अद्वैतवेदान्तसिद्धान्तस्य च प्राचीनतमोऽयं ग्रन्थः योगवासिष्ठमिति मतिरस्माकम् । रामतीर्थस्वामिनः ग्रन्थमेनं भूमण्डलान्तर्गतेषु ग्रन्थेषु अत्युत्तमं ब्रह्मसाक्षात्कारकरञ्चेति वदन्ति । प्रस्थानत्रथी साधनावस्थोपयोगिनी । योगवासिष्ठन्तु सिद्धावस्थायामपि पठनार्हं ग्रन्थरत्नम् । ग्रन्थश्चायं अनेकदृष्टान्तोपाख्यानादिभिर्युक्तिभिश्च अद्वैतसिद्धान्तं प्रतिपादयति ।
ग्रन्थस्यास्याद्भुतस्य रचनाविषयेऽस्ति महान् विदुषां मतभेदः । चित्तशुद्धि समुत्पादनाय पूर्वरामायणम्, सञ्जातचित्तशुद्धेः पुरुषस्य जिज्ञासाशान्त्यै आत्मानात्मविवेचनपरं अद्वैतसिद्धान्तकोशभूतमिंद उत्तररामायणापराभिघं योगवासिंष्ठ वाल्मीकिना प्रणीतमिति तु साम्प्रदायिकी पण्डितमण्डितवार्ता । ``ऋषिभिर्बहुधागीत" मिति गीताया (XIII. 3.) ऋषिभिरित्यस्य व्याख्यानावसरे ``वसिष्ठादिभिरिति" शाङ्करं भाष्यम् इति च प्रमाणं प्रवदन्ति । यदि वाल्मीकिरेवास्य कर्ता स्यात् तर्हि वाल्मीकिस्सुकन्याच्यवनयोः पुत्र इति पौराणिकी प्रसिद्धिरिति कालादिनिर्णयो न कर्तुं शक्यते ।
आधुनिकेषु प्राच्यप्रतीच्यभाषाप्रवीणेषु विमर्शकवरेषु च अस्य रचनाकाल विषये महान् मतभेद आशयभेदश्चवरीवर्ति । डाक्टर फर्कुहार प्रभृतय आधुनिकाः द्वादशत्रयोदशशतकमध्यमस्य रचनाकाल इति मन्वते । शिवप्रसाद भट्टाचार्यास्तु (900-1110 A. D.) दशमैकादशशतकमध्यमस्य रचनाकाल इति वर्णयन्ति । भारतीय साहित्येतिहासलेखकानां जर्मन पण्डितानां विण्टर्निट महाशयानां नवमशतकमिति । दिवानजी महाशयास्तु योगवासिष्ठरचनास्थानं काष्मीरदेशः, योगवासिष्ठरचनाकालः दशमशतकमिति निश्चिन्वन्ति । आत्रेयपहाशयास्तु कालिदामात् अर्वावीने भर्तृहरिगौडपादशङ्करसुरेश्वरप्रभृतिभ्य अद्वैतवेदान्ताचायभ्यः प्राचीने न काले योगवासिष्ठं प्रणीतमिति सिद्धान्तयति । डाक्टर. वे. राघवमहोदयाश्च गीतायाः योगवासिष्ठस्य च साम्यप्रतिपादकानि द्विनवतिंसख्याकानि उद्धरणानि प्रतिपाद्य राजशेखरात् अनन्तरभाविनि नवमशतकादारब्धे त्रयोदशशतकान्ते च काले योगवासिष्ठं प्रणीतमिति जर्नल आफ ओरियण्टल पत्रिकायां प्रतिपादयन्ति । दासगुप्तमहाशयस्तु नवमशतकीयेन काष्मीरिणा अभिनन्देन लघुयोगवासिष्ठनामा ग्रन्थः प्रणीत इति तत्कालात्पूर्वतनोऽयं ग्रन्थ इति (HIP Vol II) प्रतिपादयति ।
शङ्कराचार्यकृतविवेकचूडामणौ, विद्यारण्यस्वामिकृतपञ्चदश्यां, जीवन्मुक्तिविवेके च, भर्तृहरिकृतवाक्यपदीयवैराग्यशतकयोः, प्रकाशानन्दानां वेदान्त सिद्धान्तमुक्तावल्यां च योगवासिष्टीयश्लोकाः दृश्यन्ते । उपनिषत्स्वपि योगवासिष्ठीयश्लोकाः दृश्यन्ते । कतिपयोपनिषदः योगवासिष्ठश्लोकसंग्रहा एवेति च आत्रेयमहाशयेन सविस्तरमुपपादितम् ।
योगवासिष्ठग्रन्थे न कोऽपि ग्रन्थः ग्रन्थकारो वा उद्धृतः प्रमाणत्वेन निर्दिष्टश्च । द्वित्रिस्थलेषु परं `बुद्धः' जिनः इत्यादिशब्दाः (Page 33, 669, 729, Vol I NSP Edn) दृश्यन्ते । परन्तु तत्रापि व्याख्यात्रानन्दबोधेन ``प्रव्रजितः" इत्येवार्थः क्रियते । प्रथमभागे 454 तमे पुटे ``यत्प्राप्तं शङ्करादिभिः" इत्यादिना शङ्करः निर्दिष्टः । अत्र व्याख्यात्रा न व्याख्यातम् । कोऽयं शङ्करः ? किं शङ्कर भगवत्पादः ? उत भगवान् भवानीपतिः ? प्रकरणवशात्सु शङ्करभगवत्पाद इत्येवास्माकं प्रतीयते । एवस्मिन्नेव स्थले 714 तमे पुटे प्रथमभागे ``बृहदारण्यकादिषु" इति बृहदारण्यकोपनिषत् नाम्ना निर्दिश्यते । प्रथमभागे 594 तमे पुटे ``श्रीशैलाचार्यपुत्रेण" इति श्रीशैलाचार्यः निर्दिष्टः । एवमादिप्रमाणैश्शङ्करादनन्तरभावित्वमस्य ग्रन्थस्य वक्तुं शक्यते परन्तु प्रक्षिप्ता इमे श्लोका इति वज्रकुठारप्रक्षेपभीत्या न तथापि वक्तुं शक्यते ।
``अयं प्रपञ्चो मिथ्यैव सत्यं ब्रह्माहमद्वयम् । अत्र प्रनाणं वेदान्ता गुरवोऽनु भवस्तथा ।" (Page 181 Vol I) इत्यादिभिरसंख्यैः पद्यैः जगन्मिथ्यात्वं जीवब्रह्मैक्यं, ब्रह्मणो नामरूपबहिर्भूतत्वं, जीवन्मुक्ति, अजातवादः, अनिर्वचनीयतावादः, इत्यादय अद्वैतसिद्धान्ताः काव्यशैल्यां प्रतिपादिता इति ग्रन्थोऽयं अद्वैतवेदान्तसाहित्यास्यादिमं महाकाव्यामित्येव मदीयस्मिद्धान्तः । एतादृशे अद्वैतवेदान्तशास्त्रकाव्ये आध्यात्मिकमहाकाव्यापरनामके अद्वैतसिद्धान्तवर्णना, जगन्मिथ्यात्ववर्णना च एतादृशी वर्तते यया प्रभाविताः केचन विद्वासः योगवसीष्ठे बौद्धमतस्य सर्वश्न्यवादस्य प्रभावं वर्णयन्ति । परन्तु नैतत्सत्यम् । योगवासिष्ठस्य रचयिता न साधकः । परन्तु अद्वैतानान्दानुभवी महान् सिद्धः । तादृसस्य सिद्धस्य स्वीयानुभवैकप्रमाणे ग्रन्थेऽस्मिन् अद्वैतसिद्धान्ताः सिद्धावस्थानुकूला एव प्रतिपादिता इति तु निश्चयः । ग्रन्थोऽयं तात्पर्यप्रकाशव्याख्यासहितः निर्णयसागरमुद्रणालये मुद्रितः । अस्य व्याख्याः एतत्सम्बद्धाश्च ग्रन्थाः-
(क) अद्वयाख्याकृता - योगवासिष्ठपददीपिका । ग्रन्थोऽयं कल्कत्ता रायल आसियाटिका सूच्यां दृश्यते ।
(ख) आनन्दबोधकृतः - तात्पर्यप्रकाशः । ग्रन्थोऽयं निर्णयसागरमुद्रणालये मुद्रितः ।
(ग) अभिनन्दकृतः - योगवासिष्ठसंक्षेपः (लघुयोगवासिष्ठम्) ग्रन्थोऽयं आत्मसुखेन वासिष्ठचन्द्रिकाव्याख्यया व्याख्यातः । मुम्मुडिदेवेन संसारतरणि व्याख्यया च व्याख्यातः । ग्रन्थोऽयं सव्याख्यः निर्णयसागरमुद्रणालये मुद्रितः ।
(घ) महीधरकृतः - योगवासिष्ठसारः सव्याख्यः । ग्रन्थोऽयं बरोडापुस्तकालये लन्दनपुस्तकालये बाभ्बेविश्वविद्यालयहस्तलिखितपुस्तकालये च लभ्यते । 
(ङ) माधवसरस्वतीकृता - वासिष्ठपञ्चिका । ग्रन्थोऽयं अनन्तशयनपुस्तकालये लभ्यते ।
(च) काष्मीरपण्डिकृतम् - ज्ञानवासिष्ठम् ।
(छ) कृष्णय्यकृतः - ज्ञानवासिष्ठसारसमुच्चयः । इमौ द्वावपि ग्रन्थौ मद्रासराजकीयहस्तलिखितपुस्तकालये लभ्येते ।
%%% Chart
एतेषु केचन ग्रन्थास्तेलुगुलिप्यामेव सन्ति ।
२१. शुकः
शुकाचार्यापरनामानः बहवो वेदान्तिन आसन्निति ज्ञायते । कुत्रचित् शुकभगवत्पाद इति, कुत्रचित् शुकयोगीति नामानि बहुनि श्रूयन्ते । शुकाष्टाककर्ता शुक अन्यः ब्रह्मसूत्रभाष्यकर्ता शुक अन्य इत्येव ज्ञातुं पार्यते । अत्र प्रबलतरप्रमाणानि तु नैवोपलभ्यन्ते । यदि अद्वैतसम्प्रदायप्रवर्तकाचार्येषु परिगणितश्शुकाचार्यस्स्यात् तर्हि शुकाष्टककर्ताय महाभारतादिप्रसिद्धः कृष्णद्वैपायनात् शुकीरूपधारिण्यां घृताच्यांं जातः महान् ज्ञानीति सिघ्यति । एवञ्चास्य कालादिकथनं दुश्शकम् । इदन्तयाऽनिर्णीतत्वात् ।
(क) शुकाष्टकम् - व्यासपुत्राष्टकमित्यपरनामाय ग्रन्थः जीवन्मुक्तमहिमानं प्रदर्शयति । ग्रन्थश्चायं मुद्रितः । अस्य व्याख्यापि गङ्गाधरेन्द्रसरस्वतीकृता नासिकसूच्यां दृश्यते ।
(ख) ज्ञानबोधः - अमुद्रितोऽयं पूर्णग्रन्थ अडयारपुस्तकालये (9. B. 23) सरस्वतीमाहालये च लभ्यते ।
(ग) ब्रह्मसूत्रवृत्तिः - अस्य कर्ता शुकभगवत्पादाचार्य इति निर्दिष्टम् । ग्रन्थोऽयमुद्रितः पञ्जाब सूूच्यां 719 दृश्यते ।
२२. सनत्सुजातः
ब्रह्मणो मानसः पुत्रोऽयमिति महाभारतादिषु प्रसिद्धिः । तस्मादस्य कालनिर्णयो न शक्यते । महाभारतयुद्धस्य कालः क्रिस्तो पूर्वमिति विविधमतभेदेन विमर्शः कृतः । तस्मात् क्रिस्तोः पूर्ववर्ती युगान्तरीयस्सनत्सुजात इति परं वक्तुं शक्यते ।
(क) सनत्सुजातीयम् - मुद्रीतोऽयं ग्रन्थः बहुत्र मुद्रणालयेषु । अस्य व्याख्या शङ्करभगवत्कृता च वाणीविलासमुद्रणालये आनन्दश्रममुद्रणालये च मुद्रिता ।
एवं व्यासात् शङ्करभगवत्पादाच्च प्राक्तनाः प्रायशः सर्वे वेदान्ताचार्या निरूपिताः । अद्वैतसम्प्रदायस्य प्रवर्तकाचार्याणां परम्परा नारायणादारब्घेति प्रसिद्धा । प्रसिद्धा चेयं परम्परा-
``नारायणं पद्भभुवं वसिष्ठं शक्तिञ्च तत्पुत्रपराशरञ्च ।
व्यासं शुकं गौडपदं महान्तं गोविन्दयोगीन्द्रम् ।
अथास्य शिष्यं श्री शङ्कराचार्यमथास्य पद्मपादञ्च हस्तामलकञ्च शिष्यम् ।
तं तोटकं वार्तिककारमन्यान् अस्मद्गुरून् सन्ततमानतोऽस्मि ।" इति पद्येन ।
एतेषु वेदान्ताचार्येषु पराशरशक्त्योर्विषये न किमपि ज्ञातुं शक्यते । तस्मात् इतः परं बादरायणव्यासादारब्धाः क्रैस्तवीयर्विशतिशतकान्ताः अद्वैतवेदान्तसाहित्ये प्रसिद्धतमप्रमाणभूतग्रन्थप्रणेतारः मूलग्रन्थव्याख्याग्रन्थप्रणेतराश्चाद्वैताचार्याः कालक्रमेण निरूप्यन्ते ।
२३. बादरायणव्यासः (400 A. D. कलात् प्राक्)
अष्टादशपुराणानां रचयिता कृष्णद्वैपायनापरनामा व्यास एव बादरायण व्यास इति ब्रह्मसूत्राणां रचयितेति च साम्प्रदायिकाः । केचित्तु ब्रह्मसूत्रकारः बादरायणव्यासोऽयं अष्टादशपुराणकर्तुः कृष्णद्वैपायनाद्भिन्न इति वर्णयन्ति । पाणिनेरप्ययं ब्रह्मसूत्रकारः प्राक्तनः । यतः ``पाराशर्यशिलालिभ्यां भिक्षुनटसूत्रयो" रिति पाणिनिना ब्रह्मसूत्राणि भिक्षुसूत्रनाम्ना निर्दिष्टानि । ब्रह्मसूत्रपदैश्चैव हेतुमद्भिर्विनिश्चितैः (गीता. 13.4) इति बादरायणकृतान्येव ब्रह्मसूत्राणि विनिर्दिष्टानि, इति भगवद्गीताव्याख्यातृश्रीधरस्वामिनामभिप्रायः । केचित्तु गीतायां निर्दिष्टानि ब्रह्मसूत्राणि नैतानि, परन्त्वन्यानि तानि च नष्टानीति वदन्तः पुराणादिकर्तुः व्यासात् ब्रह्मसूत्रकारमन्यं मन्वते ।
सर्वकारणकारणं एकमेवाद्वितीयं सत्यज्ञानानन्दस्वरूपं अनन्तं असङ्गम्, नित्यं च ब्रह्म, तज्जानान्मोक्ष इत्यादिकं सर्वं अलौकिकविषयान्तर्गतम् । एतादृशेऽलौकिक विषये अनाद्यपौरुषेयश्रुतिरेव प्रमाणम् । व्यासाचार्यास्तु अलौकिकविषये अनाद्यपौरुषेयश्रुतिमेव प्रमाणमावेदयन्ति । बादरायणेन ब्रह्मसूत्रमुखेन तत्वान्युपदिश्यन्ते । तादृशं श्रुतिसम्मतमद्वैतमेवेति प्रदर्शयितुं भगवान्नारायण एव कृष्णद्वैपायनापरबादरायणव्यासो भूत्वा ब्रह्मसूत्राणि विरचयामास । बादरायणव्यासोऽयमद्वैतीत्यत्र प्रमाणानि कानिचन दृश्यन्ते-शाण्डिल्यभक्तिसूत्रे 30 आत्मैकपरां बादरायणः" इति दृश्यते । एवं बादरायणव्यास एव अष्टादशपुराणानां कर्ता, स च अधिकारिभेदं मनसि कृत्वा मन्दाधिकारिणां तत्वोपदेशाय अरूपं निर्गुणं अनिर्वचनीयं च आत्मस्वरूपं सरूपं सगुणं स्तुतिविषयं कृत्वा पुराणानि रचयामासेति वदन्त पद्यमिदं व्यासकृतत्वेन प्रसिद्धं प्रमाणयन्ति -
``रूपं रूपविवर्जितस्य भवतो ध्यानेन यत्कल्पितम्,
स्तुत्यानिर्वचनीयताखिलगुरो दूरीकृता यन्मया ।
व्यापित्वञ्च निराकृतं तु भवतो यत्तीर्थयात्रादिना
क्षन्तव्यं जगदीश तद्विकलतादोषत्रयं मत्कृतम् ॥" इति ।
तस्मात् अष्टादशपुराणादिकर्तुः ब्रह्मसूत्रकर्तुश्च बादरायणव्यासस्य कालादि निर्णेतुं न शक्यते । तथापि प्रो. हिरियण्णामहाशयास्तु ब्रह्मसूत्रनिर्माणकालः (400 A.D.) इति वदन्ति ।
(क) ब्रह्मसूत्राणि - शारीरकमीमांसासूत्रमित्यपरनामायं ग्रन्थः समन्वयाविरोधसानफलाख्यैश्चतुर्भिरध्यायैः पूर्णः । प्रत्यध्यायं चत्वारः पादा विद्यन्ते । प्रतिपादं बहून्यधिकरणानि । प्रत्यधिकरणं भिन्नविषयम् । एषु सूत्रेषु बादरायणेन उपनिषदां तदर्थानां च सङ्कलनं कृतम् । शङ्कराचार्यमतेन सूत्रसमष्टिसंख्या 555,अधिकरणसंख्या 192 । अस्य भाष्यं शङ्करभगवत्पादैः कृतं शारीरकमीमांसाभाष्यमित्याख्यम् । यदाधारं कृत्वा परश्शतानि ग्रन्थरत्नानि अद्वैतवेदान्तसाहित्ये प्रकाशन्ते ।
(ख) सिद्धान्तदर्शनम् - पूर्वोत्तराम्नायभेदात् द्विविधा हि मीमांसा । तत्रोत्तरमीमांसा पुनर्द्वेधा वादिबुबुत्सुप्रतिपादनभेदात् । तत्राद्या ब्रह्मसूत्राख्या, द्वितीया सिद्धान्तसूत्रात्मिका । एवञ्च ``अथातो ब्रह्मजिज्ञासा" इत्यादिकं वादिप्रतिपादनाय कृतम् । इदन्तु ``सिद्धान्तदर्शनं" बुबुत्सुप्रतिपादनाय कृतमिति विशेषः । सूत्ररूपेऽस्मिन् ग्रन्थे समग्राद्वैत्तसिद्धान्ताः प्रतिपाद्यन्ते । केचित्तु साम्प्रदायिका अपि अस्य ग्रन्थस्य बादरायणव्यासप्रणीतत्वे सन्दिह्यन्ति । ग्रन्थोऽयं आनन्दाश्रम मुद्रणालये मुद्रितः । अस्य भाष्यं विश्वदेवेन रचितं ``निरञ्जनभाष्या"ख्यमपि मुद्रितम् ।
(ग) भगवद्गीता - (महाभारतान्तर्गता) एतामधिकृत्य स्विस्तरं गीताप्रस्थाने प्रतिपादितमिति नेह प्रतन्यते ।
२४. गौडपादाचार्यः (500 A.D.)
गौडपादाचार्या इमे शुकमुनीन्द्रशिष्या इति, अद्वैताचार्यपरम्परायां महनीयतमा इति च ``नारायणं पद्भभुवमि"त्यादिश्लोकात् ज्ञायते । नृसिम्हतापनीयोपनिषदां गौडपादकृते व्याख्याने ``इति श्रीपरमहंसपरिव्राजकाचार्यश्रीमच्छुकमुनीन्द्रशिष्य गौडपादविरचिते उत्तरतापनीयोपनिषद्विवरणे प्रथमः खण्डः, नवमः खण्डः" इति (D. 581, 582 MGOML) ग्रन्थेऽमुद्रिते दृश्यते । एवं श्वेताश्वतरोपनिषदां शाङ्करभाष्ये ``तथाच शुकशिष्यो गौडपादाचार्यः" (Page 30, ASS 17) इति दृश्यते । एवमेव लक्ष्मणशास्त्रिविरचिते गुरुवंशकाव्ये (12. VSS) दृश्यते । तस्मान् शुकशिष्यो गौडपादाचार्य इति नीश्चीयते ।
गौडपादाचार्यस्य स्थानं नाद्यापि निश्चितम् । विषयेऽस्मिन् विभिन्ना एव विचारा दृश्यन्ते । परन्तु मान्डूक्यकारिकाशङ्करभाष्यव्याख्याने आनन्दगिरीये अलातशान्तिप्रकरणस्थस्य ``तं वन्दे द्विपदां वर"मिति पद्यांशस्य व्याख्याने एवं दृश्यते -``आचार्यो हि पुरा वदरिकाश्रमे नरनारायणाधिष्ठिते नारायणं भगवन्तमभिप्रेत्य तपो महदतप्यत ।" इति (Page 157 ASS 10) दृश्यते । तस्मात् बदरिकाश्रम एव गौडपादस्य स्थानमिति ज्ञायते । सप्तदशशतकीयेन बालकृष्णान्दसरस्वत्या स्वीये शारीरकमीमांसाभाष्यवार्तिके तु ``गौडचरणाः कुरुक्षेत्रगता हीरारावतीनदीतीरभवगौडजातिश्रेष्ठाः, देशविशेषभवजातिनाम्नैव प्रसिद्धाः द्वापरयुगमारभ्यैव समाधिनिष्ठत्वेन आधुनिकैरपरिज्ञातविशेषाभिधानास्सामान्यनाम्नैव लोकविख्याता" (Page 6. AS.I) इति प्रतिपादितम् । एवञ्च गौडपादः कुरुक्षेत्रवासी गौडजात्युत्पन्न इति गौडपादीयनामान्तरापरिज्ञाने च कारणं सूचितम् । केचित्तु गौडपादाचार्यं गौडदेशभवं वदन्ति । अत एव देशनाम्ना तेषां व्यवहारः । यथागौडब्रह्मानन्द इत्यादि । एवञ्चैतत्पक्षे गौडपादः बंगालदेशानां उत्तरभागवर्तीति सिध्यति ।
भारतीयाद्वैतवेदान्तपरम्पराप्रामाण्येन गौडपादश्शुकशिष्यः, शङ्करश्च गौडपादशिष्यः, गौडपादेनानुगृहीतश्चेति शङ्करात्पूर्वतनो वा शङ्करकालपर्यन्तजीवी वेति निश्चीयते । यदि वयं म. म. कुप्पुस्वामिशास्त्रिमहाशयानां सिद्धान्तमनुसृत्य 632-661 A.D. कालवर्तिनं शङ्करमभ्युपगच्छामस्तर्हि तैरेव प्रतिपादितसिद्धान्तमनुसृत्य गौडपादकालः (520-620 A.D.) इति, इच्छामात्रशरीरत्यागिनां गौडपादाचार्याणां कालश्शङ्कराचार्यानुग्रहपर्यन्तमिति वा स्वीकर्तव्यम् । एतेन शङ्करदिग्विजयादिवचनञ्च सङ्गतं भवति । विधुशेखरभट्टाचार्यास्तु स्वसम्पादिते `आगमशास्त्र' ग्रन्थोपोद्धाते एवं वदन्ति -``द्वितीयशतकादारभ्य चतुर्थशतकपर्यन्तानां बौद्धपण्डितानां ग्रन्थस्य गौडपादकारिकायाश्च शब्दसाम्यदर्शनात् गौडपादस्तदर्वाग्भव इति तथाच तन्मतरीत्या गौडपादकालः (500 A.D.) इति सिध्यति । यद्येवं गौडपादाचार्यस्य शङ्कराचार्यप्राचार्यत्वं कथम् ? किमन्योऽयं गौडपादः ? उतान्योऽयं शङ्करः ? न वा शङ्करः गौडपादेनानुगृहीतः, नापि प्रशिष्य इत्यभ्युपगम्य शङ्करदिग्विजयादिग्रन्थानामप्रामाण्यं स्वीकर्तव्यम् ? आहोस्वित् शङ्कराचार्यानुग्रहकालपर्यन्तजीवी गौडपादः ? इत्यादयस्संशयविशेषाः स्वतस्समुद्भवन्ति ।"
गुरुपादहालदारस्तु ``वृद्धत्रय्यां" (Page 307) गौडपादाचार्यः गोविन्दभगवत्पादशिष्यः शङ्कराचार्यकामदेवभूपालयोः परमगुरुः वेदान्तसम्प्रदायप्रवर्तकः, माण्डूक्यकारिकाकृदद्वैती (700 A. D.) कालवर्तीति प्रतिपादयति ।
(क) माण्डूक्यकारिका - (ASS. 10) गौडपादकारिकाभिधेऽस्मिन् माण्डूक्योपनिषदां व्याख्यात्मके ग्रन्थे चत्वारि प्रकरणानि सन्ति । तत्र प्रथमेऽऽगमाख्यप्रकरणे एकोनत्रिंशत्, द्वितीये वैतथ्याख्ये अष्टात्रिंशत, तृतीयेऽद्वैताख्ये अष्टाचत्वारिंशत्, चतुर्थेऽलातशान्तिप्रकरणे शतमिति 215 कारिकास्सन्ति । मुद्रितश्चायं ग्रन्थ आनन्दाश्रममुद्रणालये ।
आगमप्रकरणम् - इदमागमप्रकरणं माण्डूक्योपनिषदां भावार्थरूपं आगममूलकत्वात् अन्वर्थनाम । प्रकरणेऽस्मिन् अकारोकारमकारैः प्रतिपादितेभ्यः वैश्वानर हिरण्यगर्म-ईश्वरेभ्यः, जाग्रत्स्वप्नसुषुप्त्यवस्थाभ्यश्च भिन्नं तदनुगतं साक्षिरूपं च परमात्मतत्वं ``तुरीय" इति नाम्ना वर्णितम् ।
वैतथ्यप्रकरणम् - द्वितीयेऽस्मिन् प्रकरणे दृश्यप्रपञ्चस्य मायामयत्वं मिथ्यात्वञ्च सयुक्तिकं साधितम् । आत्मा एक एव नित्यः, तस्मिन् विविधकल्पनावशात् प्रपञ्चस्तोत्पत्तिरिवि विकल्पो भवति । अस्य मूलकारणं माया । मायाकल्पितजगतः गन्धर्वनगरवत् असत्यत्वमिति प्रतिपाद्य ``न निरोधोनचोत्पत्ति" रित्यादिना अखण्डचिद्धनानन्दआत्मतत्वादन्यस्यासत्वं प्रतिपादितम् ।
अद्वैतप्रकरणम्-तृतीयेऽद्वैताख्यप्रकरणेऽस्मिन् अनेकाभिस्सुदृढाभिर्युक्ति भिरद्वैतत्वं साधितम् । आत्मनि सुखदुःखभावना नितरां असङ्गता । यथा बालाः धूलिधूमादिसंसर्गेणाकाशं मलिनमामनन्ति, वस्तुतः यथा च आकाशो मालिन्यशून्यः तथैवात्मनोऽपि सुखित्वदुःखित्वकथनं बालबुद्धिविलासतुल्यमिति प्रतिपादितम् । असङ्गोह्यात्मा । माया हि द्वैतकल्पनायाः कारणम् । अमृतस्य मर्त्यत्वं, मर्त्यस्य अमृतत्वञ्चासङ्गतम् । अत अमृतस्यात्मनः यदि उत्पत्तिस्स्वीक्रियते तर्हि मर्त्यत्व धर्म आपद्येत इति आत्मनः उत्पत्ति - जातिः नास्ति इति प्रतिपादितम् । अयमेव गौडपादाचार्याणां अजातिवादः । एतच्च 1-17, 2-31, 32, 3-4, श्लोकेषु प्रतिपादितम् । अयमजातिबादः गौडपादात् प्राचीनस्य बौद्धाचार्यस्य दिङ्नागस्य माध्यमिकवृत्तौ, पालिभाषाप्रणीतबौद्धग्रन्थेषु च समुपलब्धेस्ततो गृहीत इति केचिद्वदन्ति । पालीभाषाया अपि प्राचीनासूपनिषत्सु ``अजायमानो वहुधा व्यजायत" इत्यादिदर्शनात् तेषामुक्तेरनुपपत्तौ भारतीयाः प्रमाणम् ।
``अलातशान्तिप्रकरणम्"- चतुर्थेऽस्मिन् प्रकरणे यथा अलाते भ्रमिते सति गोलाकारप्रतीतिर्जायते परन्तु सा गोलाकारभ्रमणजन्या एव न वस्तुतः, एवं जगदादि मायाकल्पितमेव । मनसो व्यापारादेव तस्योत्पत्तिः, मनसः निरोघे च स नास्त्येव । यथा च भ्रमणादिक्रियाशान्तौ गोलाकारकप्रतीतेश्शान्तिः, एवं मनस अमनीभावात् जगतश्शान्तिः । जगदुत्पत्तिलयौ प्रतीत्यप्रतीती उभावपि भ्रान्तिजनितावेव । परमार्थतः परमार्थतत्वं पारमार्थिकमिति प्रतिपादितम् । अद्वैतवेदान्तस्य प्राणभूताऽनिर्वचनीयख्यातिरपि प्रकरणेऽस्मित् प्रतिपादिता । ``विपर्यासात् यथा जाग्रदित्यादिना (4-41) एवं ``न निर्गतास्ते विज्ञानादित्या" दिना (4-52) ``उभेह्यन्योन्यं दृश्येते" इत्यादिना च (4-67) पद्येन प्रदर्शिता ।
प्रकरणस्यास्य भाषा ``विज्ञप्ति" रित्यादिपारिभाषिकशब्दैः पूर्णा । एवं मङ्गलाचरणश्लोके ``तं वन्दे द्विपदां वरम्" इत्यत्र द्विपदां वरशब्दश्च प्रयुक्तः । एते शब्दाः बुद्धमतग्रन्थेषु दृश्यन्त इति केचन बुद्धमतमेव गौडपादः वेदान्तापदेशेन प्रतिपादयतीति प्रच्छन्नबौद्धा अद्वैतिन इति वदन्ति ।
परन्तु शब्दसाम्यं नात्र प्रमाणमकिञ्चित्करञ्च । यत अध्यात्मशास्त्राणां पारिभाषिकशब्दाः न केवलं बौद्धानां स्वम् । परन्तु ते सर्वदर्शनसामान्याः । तेषां प्रयोगे यथा गौडपादस्य तथा बौद्धानां यथा बौद्धानां तथा गौडपादस्येति सर्वेषामधिकास्समस्ति । द्विपदांवर शब्दस्य पुराणादिष्वपि भूरिशः प्रयोगः दृश्यते । भारतरामायणादीनां बुद्धादपि प्राचीनत्वं प्रसिद्धमेव । नलभीमार्जुनभीष्मादिषु शब्दोऽयं प्रयुक्तः दृश्यते । महाभारते नारायणीयपर्वाध्याये द्विपदांवरार्थकं द्विपदां वरिष्ठपदं प्रयुक्तं दृश्यते । न वा एतत्पदं कोषग्रन्थेषु बुद्धपरत्वेन व्याख्यातम् । तस्मात् यौगिकश्शब्द एवैषः न तु योगरूढः ।
माण्डूक्यकारिकाचेयं शङ्करभगवत्पादैर्व्याख्यातम् । आनन्दगिरिव्याख्योपेतं भाष्यं आनन्दाश्रममुद्रणालये मुद्रितम् । स्वयम्प्रकाशानन्दसरस्वतीकृता मिताक्षरानाम्नी माण्डूक्यकारिकाव्याख्या वाराणस्यांं (BSS 48) मुद्रिता । उपनिषद्ब्रह्मकृता व्याख्या अडयारपुस्तकालये मुद्रिता । अनुभूतिस्वरूपाचार्यकृतं गौडपादीयभाष्यटिप्पणं (R. 2911 MGOML) लभ्यते । अज्ञातकर्तृकः गौडपादीयविवेकनामा ग्रन्थोऽपि (ई. 3882 d MGOML) लभ्यते । गौडपादाचार्यप्रणीतत्वेन प्रसिद्धाः ग्रन्थाः-
(ख) उत्तरगीताव्याख्या - ग्रन्थोऽयं तिरुपति सृच्यां (DCVORIT) अनन्तशयनपुस्तकालये (275 TCL) जयपुर पोटीखानासूच्यां (XXXIII 74/4) च दृश्यते ।
(ग) पञ्चीकरणवार्तिकम् - बरोडासूच्यां (13325 c BRD) दृश्यते ।
(घ) नृसिम्हतापनीयभाष्यम् - (D. 581 MGOML)
(ङ) अनुगीताभाष्यम् - ग्रन्थोऽयं नासिक सूच्यां दृश्यते ।
(च) श्रीविद्यारत्नसूत्रम् - (275 TCL)
(छ) दुर्गासप्तशती व्याख्या - ग्रन्थोऽयं तन्त्रदर्शनाचार्येण भास्कररायेण दुर्गासप्तशती व्याख्याने निर्दिष्टः ।
(ज) सुभगोदयः - (275 TCL)
(झ) सांख्याप्रवचन भाष्यम् ?
२५. मण्डनमिश्रः (750-850 A.D.)
मण्डनमिश्रोऽयं कुमरिलभट्टस्य शिष्यश्शङ्कराचार्यकाले प्रसिद्धः पूर्वमीमांसापण्डितः कर्मनिष्ठश्चेति प्रसिद्धिः । अस्यैव विश्वरूप इति नामन्तरम् । मण्डनमिश्रस्यैव शङ्कराचार्यात् आश्रमस्वीकारपूर्वकशिष्यत्वस्वीकारादनन्तरं सुरेश्वराचार्य इति नामेति सम्प्रदायविदः । ``जागोपि" महाशयेन नैष्कर्म्यसिद्धिभूमिकायां मण्डनमिश्र-विश्वरूपसुरेश्वराणां ऐक्यमङ्गीकृतम् । सप्तदशशतकीयेन बालकृष्णानन्दसरस्वत्या कृते शारीरकमीमांसाभाष्यवार्तिके च मण्डनमिश्रसुरेश्वरविश्वरूपाणामैक्यमेव वर्णितम् ।
दासगुप्तमहाशयास्तु सुरश्वरविश्वरूपावभिन्नौ मण्डनमिश्रस्तु अन्य एवेति (H. I. P. Vol. II) ग्रन्थे निर्दिशन्ति । हिरियण्णामहाशयास्तु (J. R. A. S. 1924) रायलासियाटिक सोसाइटि पत्रिकायाश्चतुर्विशतितमे भागे सुरेश्वरः मण्डनादन्य इति निश्चिन्वन्ति । म. म. कुप्पुस्वामिशास्त्रिणश्च स्वसम्पादितब्रह्मसिद्धि भूमिकायां सुरेश्वरब्रह्मसिद्धिकारयोस्सिद्धान्तगतभेदमुपवर्ण्य सुरेश्वरादन्यं ब्रह्मसिद्धिकारं मण्डनं वर्णयन्ति स्म ।
बिब्लियोथिकाइण्डिकासीरीजमुद्रितायां पराशरस्मृतिव्याख्यायां (Page 51) बृहदारण्यकवार्तिकात् उद्घृतम् । तच्चोद्धरणं विश्वरूपाचार्यकृतग्रन्थादित्युक्तम् । एवं विद्यारण्यैः विवरणप्रमेयसंग्रहे (Page 92) बृहदारण्यकवार्तिकात् (IV. 8.) उद्धरणं दत्तम् । तत्रापि सुरेश्वरः विश्वरूपशब्देनैव निर्दिष्टः । तस्मात् विश्वरूपसुरेश्वरावभिन्नौ, मण्डनस्त्वन्य इति निश्चयः ।
ब्रह्मसिद्धिकारेणानेन शङ्करात्पूर्वतनाः ग्रन्थाः प्रमाणत्वेन निर्दिष्टाः । वाचस्पतिमिश्रेण च ब्रह्मसिद्धिं प्रमाणं कृत्वा ब्रह्मतत्वसमीक्षा कृता । अत एव मण्डनपृष्ठसेवी वाचस्पतिरिति च प्रसिद्धिः । तस्मात् वाचस्पतिकालिको वा, तस्मात् पूर्वतनो वा भवितुमर्हति । शङ्कराचार्यसामयिक इति तु सम्प्रदायविदः । दासगुप्तमहाशयेन नवमशतकीयोऽयमिति बर्ण्यते । कुप्पुस्वामिशास्त्रिणस्तु (615-695 A. D.) इति सप्तमशतकीयं वर्णयन्ति ।
ब्रह्मसिद्धिः - ग्रन्थोऽयं मद्रासराजकीय पुस्तकालये (MGOMLS 4) सव्याख्यः मुद्रितः । अस्याः व्याख्या वाचस्पतिमिश्रकृता ``ब्रह्मतत्वसमीक्षा" चित्सुखाचार्यकृता ``अभिप्रायप्रकाशिका" आनन्दपूर्णविद्यासागरकृता टीकारत्नापरनामा ``भावशुद्धिः" शङ्खपाणिकृता ब्रह्मसिद्धिटीका चेति ग्रन्थाः वर्तन्ते ।
(ख) विभ्रमविवेकः - 162 पद्यैः पूर्णोऽयं ग्रन्थः पञ्चख्यातिव्याख्यात्मकः । मुद्रितश्चायं जर्नलआफ ओरियण्टल पत्रिकायां (J. O. R.) मद्रासनगरे । अन्येऽपि ग्रन्थाः मीमांसाशास्त्रे कृताः ।
26. शङ्करभगवत्पादः (788-820 A. D.)
शङ्कराचार्यवतारसमये धर्मपरिस्थितिः-
सनतनकालत एव प्रबलप्रमाणपूर्वकं आत्यन्तिकनिःश्रेयसाधिगमसाधनत्वेन श्रुतिपुराणभगवद्गीतोपनिषद्भिः महद्भिराचार्यैश्च निरूपितोऽयमद्वैतसिद्धान्तः । स च क्रैस्तवीयचतुर्थशतकात्पूर्वमुत्पन्नेन शाक्यमुन्यपरनाम्ना गौतम बुद्धेन तात्कालिकपरिस्थित्यनुसारं प्रवर्तितस्य बहूनां धारापतीनां प्रवेशात् अतिमहर्ती वृद्धिमाप्तस्य बौद्धमतस्य, तच्छाखान्तरस्य जैनमतस्य च प्रसरणेन प्रसारणेन च, धर्मपाखण्डानां स्वार्थपराणां धर्मकर्मकितवानां भारतीयार्याणां केषाञ्चित् आचारेण च, निःस्वार्थधर्मप्रचारकाचार्याभावेन, कर्मनिष्ठानां निरीश्वरमीमांसकानां प्रबलप्रोत्साहनेन कर्मभरभारपीडिते च लोके, धर्मनाम्ना तत्र तत्र हिंसाप्रधानेषु कर्मसूज्जृम्भतामाप्तेषु, बुद्धमतानुयायिनां विहारविहारिणां आहार विहारपराणां भिक्षूणां धर्माभासेन च जनतायां अनादिवैदिकमतं प्रति द्वेषे स्मुत्पन्ने अत्यधिकसम्पत्तिसमृद्धिवशात् अत्यधिकसुखानुभवाच्च स्थिरनित्यानन्देप्सावति च लोके विभिन्नमतप्रवर्तकाचार्योपदेशानां अनैक्यपराणां बलेन भ्रान्ते आविले च समाजे प्राचीनोऽयं उपनिषत्प्रतिपादितस्सर्वधर्मसमन्वयपर अद्वैतसिद्धान्तमार्गः ह्नासोन्मुख इबाभृत् । तादृशं अद्वैतमतं परित्रातुं भगवतः परमेश्वरस्य साक्षादवतार भूताश्शङ्कराचार्या इति प्रसिद्धिः ।
शङ्कराचार्यप्रभावः -
देवप्रार्थनया शङ्करावतारभूतैश्शङ्कराचार्यैरद्वैतात्मवादः सर्वत्र प्रकाशितः प्रसारितश्च ।
``अष्टवर्षे चतुर्वेदी द्वादशे सर्वशास्त्रवित् ।
षोडशे कृतवान् भाष्यं द्वात्रिंशे मुनिरत्यगादिति" ॥
प्रसिद्धाभाणकानुसारं वयस्यल्पे एव सर्वत्र शिष्यगणैस्साकं सञ्चारं कृत्वा अद्वैतात्म वादः पुनरुज्जीवितः । शङ्कराचार्यैः एकोनशतसंख्याका ब्रह्मसूत्रव्याख्याः खण्डिता इति उमामहेश्वरेण तत्वचन्द्रिकायां वर्णितम् । बौद्धजैनमतानि तर्क पातञ्जलादि दर्शनानि कर्मब्रह्मवादिमतानि अन्यानि च वैदिकप्रमाणप्रबलयुक्तिभिस्सप्रमाणाभिः प्रौढाभिश्च रीतिभिस्सञ्चूर्णितानि ।
जीवब्रह्मणोरैक्यं, ब्रह्मण एव सत्यत्वं, नामरूपाणां मिथ्यात्वं, सर्वदेशकालवस्त्वपरिच्छिन्नसत्ताकत्वमेव सत्यत्वं, मिथ्यारूपस्य जगत आत्मनि कल्पितत्वं, कल्पनाप्यज्ञानेन वस्तुतः ब्रह्मव्यतिरिक्तसत्ताकं जगत नास्त्येवेत्यादीन् सर्वधर्म समन्वयकरान् जगतश्शान्ति प्रदान् सर्वथा सर्वदा कल्याणप्रदान् कालत्रययोग्यान् सिद्धान्तान् सर्वत्र सम्यगुपदिश्य अद्वैतसिद्धान्तसंरक्षणाय चतसृषु दिक्षु मठान् संस्थाप्य तत्र तत्र स्वीयान् अद्वैतवादकुशलान् शिष्यान् संयोज्य पक्षवेदर्षिमिते शके 742-820 A. D. काले भगवान् शङ्कराचार्यस्स्वधाम प्रपेदे ।
शङ्कराचार्यपितामहः विद्याधिराज इति विश्रुतनामधेयः । राजशेखरराजेन पालिते ``कालटि" नामके ग्रामेऽयमुवास । कालटिग्रामोऽयं केरलदेशान्तर्गतः । शङ्कर - भगवान् स्वावताराय कैलासात् स्वपद्भयामेवागच्छदिति वदन्तः ``काल + अडि" इति तमिलभाषापदं तथैब व्यवहृतं तद्ग्रामस्यान्वर्थनाम चाभूदिति वर्णयन्ति । तद्वास्तव्यस्य विद्याधिराजस्य पुत्रः शिवगुरुरिति प्रसिद्धः परमविद्वानासीत् ॥
शङ्कराचार्यपिता
शङ्कराचार्याणां पितुर्नाम शिवगुरुरिति । बाल्ये एवाधीतविद्योऽयं चतुर्थाश्रमाभिलाषी सञ्जातः, पितुराचार्यस्य चोपदेशेन स्वीकृतगार्हर्स्थ्य अन्वर्थनामास त् । तस्य पत्नी सतीनाम्नीति शङ्करविजयव्याख्याया अवगम्येते । तयोः पुत्रमुखकमल प्रेक्षणसुखं चिरेणापि नाभूत् । अत उभावपि शिवमाराधयामासतुः । तुष्टो भगवान् शङ्कर अनयोः पुत्रत्वमूरीचकार ।
शङ्कराचार्याणां गोत्रम् -
शङ्कराचार्याणां गोत्रमात्रेयगोत्रम् । बृहदारण्यकोपनिषद्वार्तिके शङ्करशिष्यसुरेश्वराचार्यकृते `तं वन्देऽत्रिकुलोद्भवम्' इति दृश्यते ॥
शङ्कराचार्याणां गुरुः -
शङ्कराचार्याणां गुरुः गौडपादशिष्यः गोविन्दभगवत्पादः । शङ्कराचार्यरचितेषु ग्रन्थेषु ``गोविन्दभगवत्पादशिष्येणेति तत्र तत्र दृश्यते । गोविन्दं परमानन्दं मद्गुरुं प्रणतोऽस्म्यहम्" इति विवेकचूडामणौ च दृश्यते । गोविन्दभगवत्पादश्च सोमोद्भवातीरे कस्याञ्चित् गुहायामुवास । शङ्करश्च दूरादेव तं दृष्ट्वा सप्रश्रयं सभक्त्युन्मेषं साञ्जलिबन्धं तुष्टाव । तदनु गोविन्दभगवत्पादश्शङ्करं शिष्यत्वेन स्वीकृत्य साम्प्रदायिकैस्तत्वमसीत्यादि श्रुतिवचनैः ``ब्रह्मतत्वमाचरे"ति उपदिदेश । एवञ्च विद्यागुरुर्गोविन्दभगवत्पाद इति सिध्यति ।
``वृद्धत्रय्यां" गुरुपादहालदारस्तु हैहयराजस्य कामदेवस्य शङ्कराचार्यस्य च गुरुरयं गोविन्दभगवत्पादः परमयोगी सनातनधर्मावलम्बी वेदान्तसम्प्रदायप्रवर्तकः ``रसहृदयाख्य वैद्यग्रन्थप्रणेता सप्तम नवमशतकमध्यावर्तीति प्रतिपादयति ।"
शङ्कराचार्यपरमगुरुः -
गोविन्दभगवत्पादाचार्याणां आचार्यः शङ्कराचार्याणां परमाचार्यः गौडपादाचार्यः । गौडपादाचार्योऽयं गौडदेशीयः । वादप्रतिवादे गौडपादाचार्यैः गौडीया विद्वांसः पराजिताः । गौडपादाचार्योऽयं शुकाचार्यशिष्यः । शङ्कराचार्यस्य सूत्रभाष्यप्रणयने गौडपादाचार्याणां सम्पतिरनुग्रहश्चाभूतामिति शङ्करदिग्विजये श्रूयते । माण्डूक्यकारिकाभाष्येऽलातशान्तिप्रकरणे ``तं पूज्याभिपूज्यं परमगुरुममुं पादपार्तैनतोऽस्मि" इत्यादिना परमगुरोर्नमस्कृतिरावेदिता । ``अमुम्" इत्यस्य व्याख्यानावसरे आनन्दगिरिणापि ``पुरोदेशे सन्निहितत्वेनापरोक्षत्वं सृचितम्" इति व्याख्यातम् । सूत्रभाष्ये द्वितीयाध्यायस्य प्रथमपादे नवमसूत्रभाष्यावसरे गौडपादाचार्यस्य ``अनादिमायया" इत्यादिनी कारिका ``तदुक्तमाचार्येणेत्यादिना" परामृष्टः । एतत्सर्वं शङ्कराचार्याणां गौडपादशिष्यत्वेऽथवा अनुग्रहप्राप्तौ प्रमाणम् ॥
शङ्करशिष्याः-
पद्भपाद - हस्तामलक - तोटक - सुरेश्वराख्याश्चत्वार एतेषां शिष्याः । ते च क्रमेण पञ्चपादिका - विवेकमञ्जरी - श्रुतिसारसमुद्रण - बृहदारण्यकोपनिषत्तैत्तरीयकवार्तिकादिकृतः ॥
शङ्कराचार्यकालः- 
शङ्कराचार्याश्च खेन्दुहयमिते शालिवाहनशके प्रादुरभूवन् । ``व्योमभूवाजिसंख्याङ्के शालिवाहनके शुभे । विभवेऽब्दे शुक्लपक्षे वैशाखे दशमीतिथौ । शङ्करः प्रादुरासीत् ब्राह्मण्यस्थितिगुप्तये ।" इति परम्परागतञ्च पद्ममत्र प्रमाणं भवति । बालकृष्णब्रह्मानन्दकृतशङ्करदिग्विजये ``सहस्रद्वितयादूर्ध्वं एकोनर्विशके । शते । एकादशोनंसख्याके वत्सरे कलिमानतः । निधिनागेभवह्न्यब्दे विभवे शङ्करोदयः । कलौ तु शालिवाहस्य सखेन्दुशतसप्तके ।" इति दृश्यते एवञ्च शङ्करोत्पत्तिकालः(788 A. D.) एवं ``कल्यब्दे चन्द्रनेत्राङ्गगुणसंख्ये सुहायने । विहारिनामके तस्मिन् वैशाख्यां शिवतामगात् ।" शालिवाहशकेह्यब्धिहयसंख्ये स शङ्करः । विकारिनामके तस्मिन् वैशाख्यां शिवतामगात् । इत्यादिभिश्च प्रमाणैः (742) शके शङ्करनिर्याणमिति तेषां जीवनकालस्त्रयस्त्रिंशद्वत्सराण्येवेति निश्चीयते । द्विसहस्रवत्सरेभ्यः पूर्वमिति साम्प्रदायिकाः । क्रिस्तोः पूर्वं पञ्चमशतकं (500 BC.) जनन कालः ब्रह्मीभावश्च (476 BC.) इति नारायणशास्त्रिणः ``एज आफ शङ्कर" नामके ग्रन्थे वदन्ति ।
केरलोत्पत्तिनामके ग्रन्थे (400 A. D.) काले शङ्करावतारः । जीवन कालश्च 38 वर्षाणीति प्रतिपाद्यते । K. T. तेलाङ् महाशयस्तु (550-590 A. D.) शङ्करकाल इति  (I. A. Val. XIII Page 95) प्रतिपादयति । डा. बेर्नल महाशयस्तु सामविधानब्राह्मणभूमिकायां शङ्करकालं (652-680 A. D.) इति वदति । डा. फलीटमहाशयस्तु नेपालवंशावलीनामके प्रबन्धे  (Page 118-123) नेपालदेवस्य वृषदेवाख्यस्य काले शङ्कर आसीत् । वृषदेवश्च (630-655 A. D.) काले आसीदिति वर्णयति । सूर्यनारायणरावमहाशयस्तु (I. A. Vol. XLIII Page 272) अष्टमशतकीयश्शङ्कर इति प्रतिपादयति । डा. बंदरकार महाशयस्तु सूत्रभाष्ये (2-4-1, 4-3-5) बलवर्मा निर्दिष्टः । बलवर्मणश्च काल (767-785 A. D.) इति स एव शङ्करकाल इति वदति । (I. A. Vol. XLI Page 200) पत्रिकायाम् । प्रो. टेलीतु (I. A. Vol. IX Page 263) शङ्करकाल (788-820 A. D.) इति वदति । वेङ्कटेश्वराचार्यस्तु (J. R. A. S. 1916 Page 153) शङ्कराचार्यकालः (805-897 A. D.) इति वदति । पताक महाशयस्तु (3889-3921) (788-820 A. D.) Flf (I. A. Vol. XI page 174-175) वर्णयति । बालकृष्णप्पिल्लायमहाशयस्तु संक्षेपशारीरके मनुकुलादित्यः निर्दिष्टः । संक्षेपशारीरकारः सर्वज्ञात्मा शङ्करप्रशिष्यस्सुरेश्वरशिष्यश्चेति (978 A.D.) कालिकेन भास्कररविवर्मणा शिलाशासने शङ्करस्य नाम निर्दिष्टमिति शङ्करकालः दशमशतकादिमः भाग इति (I. A. Vol. I Page 136) प्रतिपादयति । K. A. नीलकण्ठशास्त्री तु ``येनाधीतानि शास्त्राणि भगवच्छङ्कराह्वयात् । निःशेषसूरिमूर्धालिमालालीढाङ्घ्निपङ्कजात् । सर्वविद्यैकनिलयो वेदवित् विप्रसम्भवः । शासको यस्य भगवान् रुद्रो इवापरः ।" इति पद्यं शिवसोमकालिकायाश्शिलालेखायाः ज्ञायते । शिवसोमश्च (877-889 A. D.) कालिकस्य इन्द्रवर्मणो गुरुरिति शङ्करकाल (877 A. D.) पर्यन्तं स्याद्वेति सन्देग्धि । प्रतिपादितञ्चैतत् (J. O. R. Vol. XI Page 265) । अडयार पुस्तकाल्यस्थेऽमुद्रितेऽज्ञातकर्तृके ब्रह्ममीमांसाशास्त्रसंग्रहे 845 सं (788 A. D.) शङ्करकाल इति दृश्यते । श्रीकण्ठशास्त्री तु भारतीयैतिहासिकत्रैमासिकपत्रिकायाश्चतुर्दशतमे भागे (I. H. Q. Vol. XIV Page 401) धर्मकीर्तिसामयिकश्शङ्कराचार्य इति (620 A. D.) कालान्नार्वाचीन इति प्रतिपादयति । T. R. चिन्ताामणिमहाशयस्तु (560-650 A. D.) कालमध्यवर्ती शङ्कर इति प्रतिपादयति । म. म. कुप्पुस्वामि शास्त्रिणश्च (632-664 A. D.) इति प्रतिपादयन्ति । 
शङ्कराचार्यग्रन्थाः-
यद्यपि शङ्कराचार्यकृता इति बहवो ग्रन्थाः मुद्रिताः प्रसिद्धाश्च तथापि ते सर्वे शङ्कराचार्यकृता इत्यत्र न प्रमाणम् । परन्तु तत्सिम्हासनारूढैश्शिष्यप्रशिष्यैर्विरचित इति ज्ञेयम् । परन्तु अधोनिर्दिष्टाः ग्रन्थाशशङ्करकृता इत्यत्र न संशयः ॥
१. अद्वैतपञ्चरत्नम् - (S. M. E. Vol 16)
सोपानपञ्चकापरनामायं ग्रन्थश्शङ्कराचार्यस्मारकग्रन्थावल्यां श्रीरङ्गनगरे मुद्रितः । अस्य व्याख्याः - कृष्णानन्दसरस्वतीकृता ``किरणावली," विमलभूधरकृता व्याख्या, अज्ञातकर्तृका दीधीतिनाम्नी व्याख्या, अज्ञातकर्तृका अपरा व्याख्या च विद्यन्ते ॥
२. अद्वैतानुभूतिः- (S. M. E. Vol 16)
प्रकरणग्रन्थोऽयं वाणीविलासमुद्रणालये मुद्रितः ।
३. अध्यात्मविद्योपदेशविधिः -
अज्ञानबोधिनी `अध्यात्मविद्योपदेशः' `आत्मज्ञानोपदेश' इत्यादिनामान्तरमस्य दृश्यते । प्रकारणग्रन्थोऽयं चौखाम्बामुद्रणालये मुद्रितः । अस्य व्याख्याः आनन्दगिरिकृता, अनन्तराममुनिकृता ``सम्प्रदायतिलकम्," पुराणानुभवकृता ``दीपकनाम्नी," पूर्णानन्दकृता व्याख्या च विद्यन्ते ॥
४. अनात्मश्रीविगर्हणम् - (S. M. E. Vol 16) प्रकरणग्रन्थोऽयं मुद्रितः ।
५. अपरोक्षानुभूतिः - (S. M. E. Vol 15)
147 पद्यैः पूर्णोऽयं प्रकरणग्रन्थश्शाङ्करप्रकरग्रन्थावल्यां मुद्रिताः अस्यैव ``अपरोक्षानुभवामृत" मित्यपि नामान्तरं श्रूयते । अस्य व्याख्याः-चण्डेश्वर शर्मकृता `दीपिका,' नित्यानन्दानुचरकृतं विवरणम्, बालगोपालयतिकृता व्याख्या विद्यारण्यकृता `व्याख्या' अज्ञातकर्तृका काचन व्याख्या च वर्तते ॥
६. आत्मबोधः- (S. M. E. Vol 15)
अज्ञानबोधिनी बोधार्यापरनामायं प्रकरणग्रन्थः 68 पद्यैः पूर्णः जीवात्मनोरभेदं जीवन्मुक्तदशाञ्च प्रतिपादयति । मुद्रितश्चायं ग्रन्थश्शाङ्करप्रकरणग्रन्थावल्याम् ।
अस्य व्याख्याः- अद्वयानन्दकृता व्याख्या चिदानन्दकृता ``आत्मबोधलहरी," पद्मपादाचार्यकृता `वेदान्तसारः,' बोधेन्द्रकृता `भावप्रकाशिका,' भासुरानन्दकृता व्याख्या, मधुसूदनसरस्वतीकृता व्याख्या, रघुनाथसरस्वतीकृता व्याख्या, रामानन्दतीर्थकृता व्याख्या, विश्वेश्वरपण्डितकृता दीपिका च विद्यन्ते । कृष्णानन्द सरस्वतीकृता व्याख्या ग्रन्थलिप्यां मुद्रितः । ब्रह्मानन्दकृता आत्मबोधव्याख्या, चित्सुखशिष्यकृता व्याख्या, अद्वयानन्दसरस्वतीकृता व्याख्या, अद्वैतानन्दसरस्वतीकृता व्याख्या च विद्यन्त इति श्रूयते । हलषसूच्यांं तु विद्यारण्यकृता व्याख्या च निर्दिष्टा । रघुनाथसरस्वतीकृता व्याख्या तु (AL. BRD) पुस्तकालये लभ्यते । आनन्दगिरिकृता टीका वाराणसीविश्वविद्यालये अमुद्रिता विद्यते । (D. C. VII P. No. 130)
७. आत्मानात्मविवेकः - (VVP)
प्रकरणग्रन्थोऽयं वाणीविलासमुद्रणालये मुद्रितः । अस्य व्याख्याः- पूर्णानन्दतीर्थकृता व्याख्या, वासुदेवयतिकृता व्याख्या, सदाशिवेन्द्रसरस्वतीकृता `प्रकाशिका' स्वयम्प्रकाशयतिकृता व्याख्या, च विद्यन्ते । सायमाचार्येण पद्भपादाचार्येण व्याख्या कृतेति श्रूयते । अज्ञातकर्तृका वेदान्तचूर्णिकानाम्नी च विद्यते ।
८. उपदेशसाहस्री - (S. M. E. Vol 14)
गद्यपद्यात्मकभागद्वयमितोऽयं प्रकरणग्रन्थः अद्वैतवेदान्तखनिः । मुद्रितश्चायं शाङ्करप्रकरणग्रन्थावल्याम् । अस्य व्याख्याः- आनन्दगिरिकृता व्याख्या, अखण्डधामकृता ``गूढार्थदीिका" बोधनिधिकृता व्याख्या रामतीर्थकृता `पदयोजनिका' च विद्यन्ते । त्र्यम्वकभट्टकृता व्याख्या उज्जैनसूच्यां दृश्यते । अज्ञातकर्तृकव्याख्या मद्रासराजकीयपुस्तकालये लभ्यते । 
९. एकश्लोकः - (S. M. E. Vol. 16)
किं ज्योतिःश्लोक इत्यपरनामायं ग्रन्थः शाङ्करप्रकरणग्रन्थावल्यां मुद्रितः । दिग्विजययात्रायै विश्वं सञ्चरन् कदाचित् शङ्कराचार्यः कस्मिंश्चित् ग्रामविशेषे, कूश्माण्डमिव पाण्डुराङ्गं, उलूकमिव सूर्याबलोकनाक्षमं, लज्जयावनतमुखं मां त्राहि त्राहि इति पुनः पुनः प्रणमन्तं कञ्चन कुष्ठिनं सा धनचतुष्टययुतमवलोक्य संसारसङ्कटान्मोचयन् कृती कर्तुं परमकारुणिकोऽयं भगवान् शङ्कराचार्यः प्रश्नप्रतिवचनप्रणाल्या श्लोकमेनं निबबन्ध इति साम्प्रदायिकी कथा श्रूयते । अस्य व्याख्या - स्वयम्प्रकाश यतिकृता ``तत्वदीपनाख्या" (स्वात्मदीपनम्) विद्यते ।
१०. काशीपञ्चकम् । वाणीविलासमुद्रणालये मुद्रितः ।
११. कौपीनपञ्चकम् - (S. M. E. Vol. 16)
यतिपञ्चकापरनामायं ग्रन्थः श्रीरङ्गक्षेत्रे मुद्रितः ।
१२. ज्ञानाङ्कुशम् सविवरणम् -
मनोनिग्रहोपायप्रतिपादनपरोऽयं ग्रन्थः अद्वैतसभापत्रिकायां मुद्रितः ।
१३. दशश्लोकी - (S. M. E. Vol. 15)
अद्वैतदशकम्, निर्वाणदशकम्, इत्यपरनामायं ग्रन्थ वाराणस्यां वाणीविलासमुद्रणालये च मुद्रितः । अस्य व्याख्यामधुसूदनसरस्वतीकृता सिद्धान्तबिन्दुनाम्नी प्रसिद्धा । विश्वेश्वरकृतापि व्याख्या विद्यते । कुत्रचिगस्यैव चिदानन्दस्तवराजः, चिदानन्ददशश्लोकीत्यपि नामान्तरं श्रूयते ।
१४. निर्वाणषट्कम् - (S. M. E. Vol. 16)
१५. पञ्चरत्नमालिका - (S. M. E. Vol. 16)
आत्मपञ्चिका, अद्वैतपञ्चिका, उपदेशपञ्चकम् , पञ्चरत्नकारिका इत्यपरनामायं ग्रन्थः वाणीविलासमुद्रणालये मुद्रितः । अस्य व्याख्याः अभिनवनारायणेन्द्रकृता, पाण्डुरङ्गपण्डितकृता प्रकाशाख्या, सदाशिवब्रह्मकृता, सुब्रह्मण्यकृता, दीधितिनाम्नी अज्ञातकर्तृका, मुमुक्षुजनकल्पवल्लीनाम्नी अज्ञातकर्तृका च व्याख्या विद्यन्ते । ग्रन्थोऽयं शङ्कराचार्यकृत इति प्रसिद्धिः । परन्तु अमुद्रिते मद्रासराजकीयहस्यलिखितपुस्तकालयस्थेऽऽदर्शापुस्तके (D. 4632 MGOML)  ``शङ्काराचार्यः प्रकटयति" इति दर्शनात् नायं शङ्करकृतिरिति प्रतिभाति ।
१६. पञ्चीकरणम् - (S. M. E. Vol 16)
वेदान्तसारपञ्चीकरणनामायं ग्रन्थश्शाङ्करप्रकरणग्रन्थावल्यां मुद्रिताः । अस्य व्याख्याः - अन्तरारामकृता ``समाधिप्रक्रिया", अभिनवनारायणेन्द्रकृता भावप्रकाशिका, आनन्दज्ञानकृतं विवरणम्, प्रज्ञानानदयतिकृतं बिवरणम् , रामतीर्थकृता तत्वचन्द्रिका, स्वयम्प्रकाशयतिकृतं विवरणम् , अज्ञातकर्तृका तत्वपञ्चिका, अज्ञातकर्तृका व्याख्या, सुरेश्वराचार्यकृतं पञ्चीकरणवार्तिकम् ।
१७. प्रबोधसुधाकरः - (S. M. E. Vol 16)
प्रकरणग्रन्थोऽयं वाणीविलासमुद्रणालये मुद्रितः । ग्रन्थोऽयं नृसिम्हचम्पूकर्त्रा षोडशशतकीयेन दैवज्ञसूर्यसूरिपण्डितेन कृतस्स्यादिति अडयार बुल्लट्टन पत्रिकायाः प्रथमे भागे, (AOR Vol VI) पत्रिकायाञ्च प्रतिपादितम् ।
१८. प्रश्नोत्तररत्नमाला - (S. M. E. Vol 16)
१९. प्रौढानुभूतिप्रकरणम् - (S. M. E. 16)
सप्तदशभिः पद्यैः पूर्णोऽयं ग्रन्थः जीवन्मुक्तस्य स्वानुभवप्रकटनपद्धत्या रचितः । वाणीविलासमुद्रणालये मुद्रितश्च । 
२०. ब्रह्मानुचिन्तनम् - (S. M. E. Vol 16)
ब्रह्मानुसन्धानापरनामायं ग्रन्थः वाणीविलासमुद्रणालये मुद्रितः ।
२१. मनीषापञ्चकम् - (S. M. E. Vol 16)
कदाचित् शङ्कराचार्यः काशीपुरीं प्रति ययौ । तत्र शङ्काराचार्यस्य ज्ञानपरीक्षायै भगवान् चण्डालस्वरूपमुपादाय समागतः । चण्डालस्वरूपं दृष्ट्ववा गच्छ गच्छ इति वदन्तं शङ्कराचार्यं स चण्डालः ``अन्नमयादन्नमयं अथवा चैतन्यादेव चैतन्यं, यतिवर किं दूरीकर्तुं वाञ्छसि ? शरीरयोरनयोरन्नकार्यत्वादिति पप्रच्छ ।" तत्प्रश्नस्य प्रतिवचनाय शङ्काराचार्यैर्मनीषापञ्चकं प्रणीतमिति कथा प्रसिद्धा । प्रतिचरणं ``मनीषा मम" इति दर्शनात् अस्य मनीषापञ्चकमिति नाम । अद्वैतसिद्धान्तसारभूतोऽयं प्रकरणग्रन्थश्शाङ्करग्रन्थमालायां मुद्रितः । अस्य व्याख्याः-गोपालबालयतिकृता ``मधुमञ्जरी", नृसिम्हाश्रमिकृता `मधुमञ्जरी' वासुदेवन्द्रकृता व्याख्या, सदाशिवब्रह्मेन्द्रकृता तात्पर्यदीपिका, अज्ञातकर्तृकं लघुविवरणम्, अज्ञातकर्तृका व्याख्या च विद्यन्ते । 
२२. मायापञ्चकम् -
मायाविवरणापरनामायं ग्रन्थः मायाया अधटितधटनापटीयस्त्वं वर्णयति । ग्रन्थोऽयं शाङ्करप्रकरणग्रन्थावल्यां (S. M. E. 16) मुद्रितः ।

२३. मोहमुद्गरः (S. M. E. Vol. 16)
द्वादशमञ्जरिकापरनामायं ग्रन्थः भजगोविन्दस्तोत्रमित्यपि व्यपदिश्यते । मुद्रितोऽयं ग्रन्थः । अस्य व्याख्या द्वादशमञ्जरी ``मकरन्द" नाम्नी स्वयम्प्रकाशयतिकृता तिरुवनन्तपुरपुस्तकालये मद्रासराजकीयहस्तलिखितपुस्तकालये च लभ्यते ॥
२४. लघुवाक्यवृत्तिप्रकरणम् - (S. M. E. Vol. 16)
प्रकरणग्रन्थोऽयं वाणीविलासमुद्रणालये मुद्रितः । रामानन्दसरस्वतीकृता प्रकाशिका नाम्नी, अज्ञातकर्तृका ``पुष्पाञ्जलि" नाम्नी च व्याख्या वर्तते । तत्र पुष्पाञ्जलिः रामभद्रयतिशिष्यरामानन्दकृता इति हालसूच्याः 107 पुटे दृश्यते ।
२५. वाक्यवृत्तिः- (ASS 80, S. M. E. Vol. 15)
गुरुशिष्यकथासरण्या विरचितोऽयं प्रकरणग्रन्थस्त्रिपञ्चाशद्भिः पद्यैः पूर्णो जीवब्रह्मणोरैक्यं महावाक्यार्थञ्च प्रतिपादयति । अस्य व्याख्याः आनन्दज्ञानकृता व्याख्या, आन्दस्वरूपभट्टारककृता वाक्यदीपिका, रामानन्दसरस्वतीकृता प्रकाशिका, विश्वेश्वरपण्डितकृता प्रकाशिका, अज्ञातकर्तृका लघुटीका च ॥
२६. विवेकचूडामणिः - (S. M. E. Vol. 14)
आत्मानात्मविवेकचूडामण्यपरनामायं ग्रन्थः वाणीविलासमुद्रणालये मुद्रितः । अस्य व्याख्या रत्नस्वामिशिष्येण हरिनाथभट्टेन कृता काचन वाराणस्यां मुद्रिता । अपरा अज्ञातकर्तृका च विद्यतेऽमुद्रिता । शृङ्गगिरिशङ्करपीठाधीशैः श्रीमच्चन्द्रशेखरभारतीस्वामिभिः कृता काचन व्याख्या पाण्डित्यपूर्णा मुद्रिता लभ्यते ।
२७. वेदान्तसारः- (S. M. E. Vol. 15)
वेदवेदान्तसारः, सर्ववेदान्तसिद्धान्तसारसङ्ग्रह इत्यपरनामायं ग्रन्थः 124 पद्यैः पूर्णः वाणीविलासमुद्रणालये मुद्रितः ॥
२८. शतश्लोकी - (S. M. E. 15)
वेदान्तशतश्लोक्यपरनामायं ग्रन्थ आनन्दज्ञानेन व्याख्यातः, स्रग्धरावृत्त घटितश्च वाणीविलासमुद्रणालये मुद्रितश्च । अज्ञातकर्तृका व्याख्या काचन बरोडापुस्तकालये बाम्बे युनिवर्सिटिपुस्तकालये च लभ्यते । लन्दननगरहस्तलिखित पुस्तकालयेऽपि लभ्यते ।
२९. षट्पदी - (G. N. P. B)
आर्यावृत्तघटितैः पद्यैः पूर्णोऽयं ग्रन्थ अद्वैतब्रह्मात्मकभगवत्स्तुतिरूपः । मुद्रितश्चायं गोपालनारायणमुद्रणालये बाम्बे नगरे । अस्य व्याख्याः कविसरोजभिक्षुकृता, वैकुण्ठशिष्यकृता, शङ्करानन्दतीर्थकृता च षट्पदमञ्जरी नाम्नी प्रसिद्धा ॥
३०. सदाचारप्रकरणम् - (S. M. E. Vol. 16)
सदाचारानुसन्धानमित्यपि व्यपदिश्यतेऽयं ग्रन्थः । आत्मसाक्षात्कारोपायभूतं सदाचारं प्रतिपादयन्नयं ग्रन्थः वाणीविलासे मुद्रितः । ग्रन्थोऽयं अच्युतशर्मामोडकेन शुद्धधर्मपद्धत्याख्यया व्याख्यया व्याख्यातः ।
३१. सनत्सुजातीयभाष्यम् - (S. M. E. Vol. 13)
महाभारतोद्योगपर्वान्तर्गतस्य विधुरप्रार्थनाप्रेरितसनत्सुजातधृतराष्ट्रसंवादात्मकस्य ग्रन्थस्य व्याख्यारूपोऽयं ग्रन्थः वाणीविलासमुद्रणालये मुद्रितः काण्डद्वयातीतयोगिना मोक्षसाम्राज्यलक्ष्मीतन्त्रनाम्न्या व्याख्यया, नीलकण्ठकृतया च व्याख्यया व्याख्यातः ।
३२. सर्ववेदान्तसिद्धान्तसंग्रहः - (S. M. E. Vol. 15)
सर्वदर्शनसिद्धान्तः, सर्वसिद्धान्तसंग्रहः, वेदान्तशास्त्रसिद्धान्तसंग्रहः, वेदान्तसिद्धान्तदीपिका, सर्ववेदान्तसिद्धान्तसरसंग्रह इत्यपरनामायं ग्रन्थः द्वादशभिः प्रकरणैः पूर्णः भारतीयानां विभिन्नानां दर्शनानां सिद्धान्तान् प्रतिपादयति । मुद्रितश्चायं वाणीविलासमुद्रणालये । अस्य व्याख्या शेषगोविन्दकृता च विद्यते ।
ग्रन्थोऽयं शङ्करादर्वाचीनेन केनापि कृतं स्यादिति (ABORI Vol. XII Page 253) पत्रिकायां प्रतिपादितम् । ``अर्थतोऽप्यद्वयानन्दमतीतद्वैत लक्षणम् । आत्माराममहं वन्दे श्रीगुरुं शिवविग्रहम्" । इत्यमुद्रिते अडयारपुस्तकालये च दर्शनात् प्रथमसदानन्दप्रशिष्येण अद्वयानन्दशिष्येण द्वितीयसदानन्देन कृतस्स्यादिति (J. O. R. Vol. XIII, AOR. Vol. VI) पत्रिकयोः प्रतिपाद्यते ।
३३. स्वरूपानुसन्धानाष्टकम् - (S. M. E. Vol. 16)
भुजङ्गप्रयातछन्दोबद्धोऽयं विज्ञाननौकापरनामा ग्रन्थ अष्टभिः पद्यैः पूर्णः वाणीविलासमुद्रणालये मुद्रितः । अस्य व्याख्या आनन्दज्ञानकृत ``स्वरूपविवरण"नाम्नी, पद्मपादकृता ``स्वरूपानुभव"नाम्नी, श्रीकुदकृता ``पदव्याख्या"नाम्नी च विद्यन्ते ॥
३४. स्वात्मनिरूपणम् - (S. M. E. Vol. 16)
स्वात्मप्रकाशिका, स्वात्मानन्दप्रकरणम् , वेदान्तार्या, अनुभूतिरत्नमाला, स्वात्मानन्दप्रकाशिका, इत्यादिनामभिः व्यपदिश्यमानोऽयं ग्रन्थः देहेन्द्रियादिभिन्न आत्मस्वरूपमेकं प्रतिपादयति । मुद्रितश्चायं शाङ्करप्रकरणग्रन्थावल्यां वाणीविलासमुद्रणालये । अस्य व्याख्या सच्चिदानन्दसरस्वतीकृता अमुद्रिता विद्यत इत्यन्यत्र प्रतिपादितम् ॥
३५. हस्तामलकीयभाष्यम् -
हस्तामलकाचार्यकृतविवेकमञ्जरीव्याख्यात्मकोऽयं ग्रन्थः वाणी विलासमुद्रणालये मुद्रितः ।
३६. ईशावास्योपनिषद्भाष्यम् - (ASS 5)
ग्रन्थोऽयमानन्दाश्रममुद्रणालये मुद्रितः । अस्य व्याख्याः - आनन्दगिरिकृता शिवानन्दयतिकृता च विद्येते ।
३७. ऐतरेयोपनिषद्भाष्यम् - (ASS 11)
मुद्रितश्चायं ग्रन्थ आनन्दाश्रममुद्रणालये । अस्य व्याख्याः - अभिनवनारायणेन्द्रकृता, आनन्दज्ञानकृता, ज्ञानामृतयतिकृता, उपनिषद्ब्रह्मेन्द्रकृता, विद्यातीर्थकृता, च विद्यन्ते । नृसिह्मकृता, बालकृष्णदासकृताश्च भाष्यव्याख्या विद्यन्त इति वदन्ति । परन्तु ताः कुत्र लभ्यन्त इति न ज्ञायते । सीतानाथतत्वभूषणेन काचन व्याख्या कृता । सा च भारतकार्यलयमुद्रितपुस्तकालयपुस्तकसूच्यां लन्दननगरस्थायां (IOLPC Vol II Part I Page 64) दृश्यते ।
३८. कठोपनिषद्भाष्यम् - (ASS 7)
मुद्रितश्चायं ग्रन्थ आनन्दाश्रममुद्रणालये । अस्य व्याख्याः - अच्युतकृष्णानन्दकृता, अभिनवनारायणेन्द्रकृता, आनन्दगिरिकृता, गोपालबालयतिकृता, शिवानन्दयतिकृत, श्रीधरशास्त्रिकृता च विद्यन्ते ।
३९. केनोपनिषद् भाष्यम् - (ASS 6)
पदवाक्यभाष्यभेदेन भिन्नं भाष्यद्वयमपि आनन्दाश्रमे मुद्रितम् । अस्य व्याख्या अभिनवनारायणेन्द्रकृता, आनन्दगिरिकृता, शिवानन्दयतिकृता, श्रीधरशास्त्रीपाठककृता च व्याख्याः विद्यन्ते ।
४०. छान्दोग्योपनिषद् भाष्यम् - (ASS 14)
मुद्रितोऽयं ग्रन्थ आनन्दाश्रममुद्रणालये । अस्य व्याख्याः-अभिनवनारायणेन्द्रकृता, आनन्दगिरिकृता, च विद्येते । नरेन्द्रपुरीकृता काचन भाष्यटिप्पणी मद्रासराजकीय पुस्तकालये लभ्यते ।
४१. तैत्तरीयोपनिषद् भाष्यम् - (ASS 12)
ग्रन्थोऽयं आनन्दाश्रमे वाणीविलासे च मुद्रितः । अस्य व्याख्याः आनन्दगिरिकृता, अच्युतकृष्णानन्दकृता वनमालाख्या, सुरेश्वराचार्यकृतं वार्तिकम्, अभिनवद्रविडाचार्यबालकृष्णानन्दसरस्वतीकृतं भाष्यविवरणमिति व्याख्याः विद्यन्ते ।
४२. नृसिम्हतापनीय भाष्यम् - (ASS. 30)
(D. 581. MGOML) ग्रन्थोऽयं वाणीविलासेऽऽनन्दाश्रमे च सुद्रितः ।
४३. प्रश्नोपनिषद्भाष्यम् - (ASS 18)
आनन्दाश्रमे मुद्रितः । अभिनवनारायणेन्द्रकृता आनन्दगिरिकृता उप निषद्ब्रह्मेन्द्रकृता, शिवानन्दयतिकृता अज्ञातकर्तृका व्याख्याः विद्यन्ते ॥
४४. ब्रह्मसूत्रभाष्यम् - (N. S. P.)
शारीरकमीमांसाभाष्यापरनामायं ग्रन्थः निर्णयसागरमुद्रणालयेऽन्यत्र च मुद्रितः । अस्य व्याख्याः - अद्वैतानन्दकृतं ब्रह्मविद्याभरणम् , अनन्तकृष्णाशास्त्रिकृतः प्रदीपः, अनन्यानुभवकृता शारीरकन्यायमणिमाला, अनुभूतिस्वरूपाचार्यकृत प्रकटार्थविवरणम्, आनन्दगिरिकृतः न्यायनिर्णयः, उपनिषद्ब्रह्मेन्द्रकृतः भाष्यसिद्धान्तसंग्रहः, कृष्णशास्त्रिकृतानुगुण्यसिद्धिः, रामानन्दकृता रत्नप्रभा, चित्सुखाचार्यकृता भाष्यभावप्रकाशिका, ज्ञानोत्तमकृता विद्याश्रीः, त्र्यम्बकशास्त्रिकृता भाष्यानुप्रभा (भाष्यभानुप्रभा) नारायणसरस्वतीकृतं भाष्यवार्तिकम्, पद्मपादाचार्यकृता पञ्चपादिका, बालकृष्णानन्दकृतं भाष्यवार्तिकम् , ब्रह्मानन्दयतिकृतं भाष्यवार्तिकम् , ब्रह्मानन्दयतिकृतः भाष्यार्थसंग्रहः, वाचस्पतिमिश्रकृता भामती, शिवनारायणानन्दकृता सुबोधिनी, कृष्णानुभूतिकृतः भाष्यसिद्धान्तसंग्रहः, प्रकाशात्मकृता भाष्यन्यायसंग्रहाख्याश्च व्याख्याः प्रसिद्धतमाः ॥
ब्रह्मसूत्रभाष्यम् (भामती प्रस्थानम्)
%%% Chart
पञ्चपादिकाप्रस्थानम् (विवरणप्रस्थानम्)
%%% Chart
४५. बृहदारण्यकोपनिषद्भाष्यम् - (ASS 16)
आनन्दाश्रममुद्रणालये वाणीविलासमुद्रणालये च मुद्रितः । अस्य आनन्दगिरिकृता, शिवानन्दयतिकृता, महादेवेन्द्रसरस्वतीकृता, सुरेश्वराचार्यकृतं वार्तिकञ्चेति व्याख्यास्सन्ति ।
बृहदारण्यकोपनिषत्
%%% Chart
४६. भगवद्गीताभाष्यम् - (N. S. P.)
निर्णयसागर-वाणीविलास-आनन्दाश्रममुद्रणालयेषु मुद्रितः । अस्य व्याख्याः - अनुभूतिस्वरूपाचार्यकृता, आनन्दगिरिकृता, केशवसाक्षिभगवत्पादकृता, रामानन्दकृता, भागवतानन्दकृताश्च विद्यन्ते ।
४७. माण्डूक्योपनिषद्भाष्यम् - (ASS 10) 
सव्याख्योऽयं ग्रन्थः वाणीविलासमुद्रणालयेऽऽनन्दाश्रममुद्रणालये च मुद्रितः आनन्दगिरिकृता मधुरानाथशुक्लकृता, राघवानन्दकृता, अज्ञातकर्तृकाश्च व्याख्यास्सन्ति ।
४८. माण्डूक्यकारिकाभाष्यम् - (ASS 10)
गौडपादकारिकाभाष्यापरनामायं ग्रन्थ आनन्दाश्रमे मुद्रितः । अस्य आनन्दगिरिकृता, अनुमूतिस्वरूपकृताश्च व्याख्याः विद्यन्ते ॥
माण्ड्क्योपनिषत्
%%% Chart
४९. मुण्डकोपनिषद् भाष्यम् (ASS 9)
ग्रन्थोऽयं वाणीविलासआनन्दाश्रममुद्रणालययोर्मुद्रितः । अस्य व्याख्या - अभिनवनारायणेन्द्रकृता, आनन्दगिरिकृता, शिवानन्दयतिकृताश्च विद्यन्ते ॥
५०. श्वेताश्वतरोपनिषद्भाष्यम् (ASS 17)
भाष्यमिदं शङ्कराचार्यकृतमिति प्रसिद्धं आनन्दाश्रमे मुद्रितञ्च । परन्तु नेदं शङ्कराचार्यकृतमिति वर्णितं उनिषत्प्रस्थाने ।
स्तोत्रग्रन्थाः
५१. अच्युताष्टकम् ।
५२. अन्नपूर्णाष्टकम् ।
५३. अर्धनारीश्वरस्तोत्रम् ।
५४. आनन्दलहरी ।
५५. उमामहेश्वरस्तोत्रम् ।
५६. कनकधारास्तोत्रम् ।
५७. कल्याणवृष्टिस्तवः ।
५८. कालभैरवाष्टकम् ।
५९. काशीपञ्चकम् ।
६०. कृष्णाष्टकम् ।
६१. गणेशपञ्चरत्नम् ।
६२. गणेशभुजङ्गम् ।
६३. गुर्वष्टकम् ।
६४. गोविन्दाष्टकम् ।
६५. गौरीशतकम् ।
६६. गङ्गाष्टकम् ।
६७. जगन्नाथाष्टकम् ।
६८. त्रिपुरसुन्दरीवेदपादस्तोत्रम् ।
६९. त्रिपुरसुन्दरीमानसपूजास्तोत्रम् ।
७०. त्रिपुरसुन्दर्यष्टकम् ।
७१. दक्षिणामूर्तिवर्णमालास्तोत्रम् ।
७२. दक्षिणामूर्त्यष्टकम् ।
७३. दशश्लोकीस्तुतिः ।
७४. देवीचतुःषष्ठ्युपचारस्तोत्रम् ।
७५. देवीभुजङ्गम् ।
७६. नवरत्नमालिका ।
७७. नर्मदाष्टकम् ।
७८. निर्गुणमानसपूजा ।
७९. पाण्डुरङ्गाष्टकम् ।
८०. प्रातस्स्मरणस्तोत्रम् ।
८१. भवानीभुजङ्गम् ।
८२. भगवन्मानसपूजा ।
८३. भ्रमराम्बाष्टकम् ।
८४. मणिकर्णिकाष्टकम् ।
८५. मृत्युञ्जयमानसपूजास्तोत्रम् ।
८६. मन्त्रमातृकापुष्पमालास्तवः ।
८७. मोहमुद्गरः ।
८८. मीनाक्षीस्तोत्रम् ।
८९. यमुनाष्टकम् ।
९०. रामभुजङ्गप्रयातम् ।
९१. लक्ष्मीनृसिम्हकरुणास्तोत्रम् ।
९२. ललितात्रिशतीभाष्यम् ।
९३. ललितापञ्चरत्नम् ।
९४. विष्णुसहस्रनामभाष्यम् ।
९५. विष्णुभुजङ्गप्रयातम् ।
९६. विष्णुपादादिकेशान्तवर्णनम् ।
९७. वेदसारशिवस्तोत्रम् ।
९८. शारदाभुजङ्गप्रयाताष्टकम् ।
९९. शिवपञ्चाक्षरनक्षत्रमालास्तोत्रम् ।
१००. शिवनामावल्यष्टकम् ।
१०१. शिवकेशादिपादान्तवर्णनम् ।
१०२. शिवपादादिकेशान्तवर्णनम् ।
१०३. शिबभुजङ्गम् ।
१०४. शिवापराधक्षमापनस्तोत्रम् ।
१०५. शिवानन्दलहरी ।
१०६. षट्पदीस्तोत्रम् ।
१०७. सुब्रह्मण्यभुजङ्गम् ।
१०८. सौन्दर्यलहरी ।
अस्य व्याख्याः- अरिच्छित्कृता ``सुधाविद्योतिनी" । अमुद्रिता व्याख्येयं तिरुवनन्तपुरपुस्तकालये (C. O. L. 116 F.) लभ्यते । अस्यां व्याख्यायां सौन्दर्यलहरी प्रवरसेनेन कृतेति दृश्यते । प्रवरसेनाख्यः काश्चित् क्षत्रिय आसीत् । तस्य पिता द्रमिडाख्यः कश्चन राजा । माता वेदवतीनाम्नी । शुकाख्यः मन्त्री स च प्रवरसेनजननकालरीत्या द्रमिडस्य राज्यच्युतिमुवाच । भीतः द्रमिडः स्वपुत्रं प्रवरसेनं गिरिमूर्धनि तत्याज । क्षुधिताय तस्मै बालाय प्रवरसेनाय भगवती परमेश्वरी स्तन्तं पाययामास । स्तन्यपानलब्धवैदुष्यः प्रवरसेनः भगवतीस्तोत्ररूपां सौन्दर्यलहरी चकार । अत एव ``तव स्तन्यं मन्ये" इत्यादिना निर्दिश्यते । प्रवरसेनस्य कश्चित्पुत्रः वन्यायां भार्यायामुत्पन्नः । तस्य नाम अरिच्छिदिति । अनेन कृतेयं सुधाविद्योतिनीति कथापि दृश्यते ।
अन्ये तु ``तव स्तन्यं मन्ये" इति श्लोके दृश्यमानः द्रविडशिशुशब्दः तमिलभाषाशैवसाहित्यप्रसिद्धः तिरुज्ञानसम्बन्धनायनारिति वदन्तः पूर्वोक्तां कथामधिक्षिपन्ति ।
कौवल्याश्रमकृता सौभाग्यवर्धिनी, नरसिम्हकृता व्याख्या, डिण्डिमरामकविकृता व्याख्या, लक्ष्मीधरकृता लक्ष्मीधरा, अज्ञातकर्तृका विद्वान्प्ननोरमा, सदाशिवकृता व्याख्याश्च मद्रासराजकीहस्तलिखितपुस्तकालये लभ्यन्ते ।
१०९. हरिस्तुतिः ।
११०. हनूपत्पञ्चरत्नम् ।
१११. सुवर्णमालास्तोत्रम् ।
११२. द्वादशलिङ्गस्तोत्रम् ।
अन्योऽपि प्रपञ्चसाराख्यः तन्त्रशास्त्रग्रन्थः विरचितः । एवमन्येऽपि शङ्कराचार्यकृता इति मुद्रिता अमुद्रिताश्च बहवः ग्रन्था उपलभ्यन्ते । परन्तु नामान्तरेण निर्दिष्टा एत एव ग्रन्था इति विभावनीयम् ।
२७.पद्मपादाचार्यः (800 A. D.)
शङ्करभगवत्पादशिष्यतल्लजेष्वन्यतमोऽयं पद्मपादाचार्यः आश्रमस्वीकारात्पूर्वं सनन्दनापरनामा विमलनामकस्य ब्राह्मणस्य पुत्रः, नद्यास्तीरे स्थितोऽयं उत्तरतीरवर्तिना गुरुणाऽहूतः सत्वरागमनाय नदीजले पादौ निक्षिप्य पाथोरुहेषु क्रमेण पदं विनिक्षिप्याजगाम । तादृशीमनितरसाधरणीं गुरुभक्तिं दृष्ट्वा तुष्ट आचार्यस्तस्य पद्मपाद इति नाम व्यतानीदिति कथा प्रसिद्धा । स्वगुरोस्सकाशादनेन वारत्रयं भाष्यमपाठि । आचार्यानुज्ञातेनानेन भाष्यस्य पञ्चपादिका नाम्नी टीका व्यरचि ।
टीकामेतां पूर्वमीमांसापण्डितस्य कर्मनिष्ठस्य कस्यचित् स्वबन्धोर्गृहे निक्षिप्य पद्भपादाचार्यः तीर्थयात्रायै जगाम । द्वेषात् स बन्धुः लोकापवादभीतेश्च पञ्चपादिकां भस्मसात् कर्तुमिच्छन् स्वगृहमेव भस्मसात् चकार । तीर्थयात्रायास्समागतेन पद्भपादाचार्येण पुनरपि स्वगुरुणा शङ्कराचार्येण स्मृतिपथमानीतं यद्यदुक्तं तत्सर्वं विलिखितमिति निर्मूला दन्तकथा श्रूयते ।
शङ्कराचार्यशिष्योऽयं अष्टमशतकीय इति तु सामान्यसिद्धन्तः । म. म. कुप्पुखामिशास्त्रिणां तु सप्तमशतकमिति (625-705 A.D) विशेषसिद्धान्तः । अस्य सतीर्थ्याः सुरेश्वरहस्तामलकतोटकाचार्याः प्रसिद्धाः । अद्वैतवेदान्तसाहित्ये पद्भपादाचार्यः विशिष्टप्रस्थानस्य प्रवर्तकः । तच्च प्रस्थानं पञ्चपादिकाप्रस्थानमिति प्रसिद्धम् ।
१. पञ्चपादिका-(V. N. S. S. 3)
सूत्रभाष्यव्याख्यात्मकोऽयं ग्रन्थश्चतुस्सूत्र्यन्त एवोपलभ्यते । नवभिः वर्णकैपूर्णोऽयं ग्रन्थः विजयनगरसंस्कृतग्रन्थमालायां, निर्णयसागरमुद्रणालये, वाणीविलासमुद्रणालये, कल्कत्तासंस्कृतग्रन्थमालायां, च मुद्रितः । ग्रन्थस्य नामपर्यालोचने पञ्चभिः पादैर्भाव्यम् । परन्तु प्रथमेऽध्याये प्रथमे पादे चतुस्सूत्र्यन्त एवोपलभ्यते । अस्य व्याख्याः - आत्मसर्वज्ञकृता ``प्रबोधपरिशोधिनी," आनन्दपूर्वविद्यासागरकृता पञ्चपादिकाव्याख्या उत्तमज्ञयतिकृता वक्तव्यप्रकाशिका, नृसिम्हाश्रमिकृता वेदान्तरत्नकोशः, प्रकाशात्मयतिकृतं पञ्चपादिका विवरणम्, रामानन्दकृता त्रय्यन्तभावदीपिका, धर्मराजाध्वरिकृता पञ्चपादिकाव्याख्या, विज्ञानात्मकृता तात्पर्यद्योतिनी, अज्ञातकर्तृका च पञ्चपादिका व्याख्याः ग्रन्थेऽस्मिन् प्रतिपादिताः ॥
२. विज्ञानदीपिका- (A. U. S. S. J)
पद्यबद्धोऽयं ग्रन्थः । ग्रन्थेऽस्मिन् सकलवेदान्ततत्वान्यवलोङ्य विविधकर्मपाशबन्धनच्छेदेनैव मुक्तिर्भवितुमर्हतीति निश्चित्य विश्वजनीनं मोक्षमार्गं सुलभं कर्तुं केनोपायेन कर्मनिर्मुक्तिर्भविष्यतीति ज्ञापनमुखेन फलानुसन्धानरहितकर्मानुष्ठानद्वारा साक्षात् परम्परया वा विज्ञानं जायत इति तत्वज्ञानप्रदीपिकेयं प्रदीपिता । अस्य व्याक्यापि मूलकृतैव कृता विद्यते । मुद्रितश्चायं ग्रन्थः अलहाबाद विश्वविद्यालयग्रन्थमालायाम् ।
३. आत्मबोधव्याख्या-(D. 4558 MGOML)
शाङ्करात्मबोधव्याख्यात्मकोऽयं ग्रन्थः मद्रासराजकीयपुस्तकालये अमुद्रित उपलभ्यते । बरोडापुस्तकालयेऽपि लभ्यते ।
४. आत्मानात्मविवेकव्याख्या - (686 C. C. P. B.)
५. कठोपनिषद्भाष्यम् - (742 C. C. P. B.)
६. कर्मनिर्णयः - (686 C. C. P. B.)
७. तत्वमसिपञ्चकम् - (77 Nasik Vol. XXVI 52) ग्रन्थोऽयं नासिक सूच्यां दृश्यते ।
८. प्रपञ्चसारव्याख्या - (686 C. C. P. B.)
९. स्वरूपानुभवः - (7730 TSML.)
एते ग्रन्थाः पद्भपादकृता इति निर्दिष्टाः । परन्तु प्रबलप्रमाणानि नोपलभ्यन्ते । आदर्शग्रन्थाश्च नावलोकितुं पार्यन्ते च ।
२८. सुरेश्वराचार्यः (800-900 A. D.)
``श्रीमच्छङ्करपादपद्मयुगलं संसेव्य लब्ध्वोचिवान् " इति नैष्कर्म्यसिद्धौ वदन्नयं सुरेश्वराचार्यः शङ्कराचार्यशिष्यः पद्मपादतोटकहस्तामलकानां सतीर्थ्यश्च ।
पूर्वाश्रमे शोणानदीतीरवासी पञ्चगौडान्तर्गतः कुमरिलभट्टजामाता पूर्वकाण्डप्रवर्तकः मण्डनमिश्र इति ख्यातः विश्वरूप एव सन्यासस्वीकारादनन्तरं सुरेश्वर इति प्रसिद्ध इति साम्प्रदायिका वदन्ति । जागोपमहाशयेन नैष्कर्म्यसिद्धिभूमिकायां मण्डनमिश्रसुरेश्वरविश्वरूपाणामैक्यंमङ्गीक्रियते । सप्तदशशतकीयेन बालकृष्णानन्दसरस्वत्या कृते शारीरकमीमांसाभाष्यवार्तिके च त्रयाणामैक्यमेवोपवर्णितम् । विद्यारण्यैः विवरणप्रमेयसंग्रहे बृहदारण्यकवार्तिकादुद्धरणं दत्तम् । तत्रापि विश्वरूपशब्देन सुरेश्वरः निर्दिष्टः ।
दासगुप्तमहाशयस्तु सुरेश्वरविश्वरूपावभिन्नौ मण्डनमिश्रस्त्वन्य इति वदति । हिरियण्णामहाशयस्तु (J.R. A. S. 1924) पत्रिकायां सुरेश्वरः मण्डनादन्य इति निश्चिनोति । म. म कुप्पुस्वामिशास्त्रिणस्तु सुरेश्वरब्रह्मसिद्धिकारयोस्सिद्धान्तगत भेदमुपवर्ण्य ब्रह्मसिद्धिकारः सुरेश्वरादन्य इति प्रतिपादयन्ति । संक्षेपशारीरक कर्ता सर्वज्ञात्मा सुरेश्वर (देबेश्वर) शिष्य इति प्रसिद्धिः । श्रीकण्ठशास्त्री तु नायं सर्वज्ञात्मगुरुरिति (I. H. Q. Vol . XIV) वदति । दासगुप्तस्तु सर्वज्ञात्मा सुरेश्वरशिष्य इत्येव वदति । श्रीकण्ठशास्त्रिणा प्रदर्शितायां शृङ्गगिरिगुरुपरम्परायां नित्यबोधघनाभिघः नित्यबोधाचार्यस्सुरेश्वरशिष्य इति निर्दिष्टम् । नित्यबोधाचार्यकालश्च (773 - 848 A.D.) पर्यन्तमिति च । तस्मात् सर्वज्ञात्मन एव नित्यबोधाचार्य इत्यपि नामान्तरं स्यादिति स्वीकारोऽपि सुष्ठु लगति ।
शङ्कराचार्यकाल एव सुरेश्वरकालः । तस्मात् नवमशतकीयस्सुरेश्वरः । कुप्पुस्वामिशास्त्रिणश्च  (602 - 700 A.D.) इति वदन्ति । शृङ्गगिरिगुरुपरम्परायां सुरेश्वरकालः (695 - 777 A.D.) इति दृश्यते ।
१. तैत्तरीयोपनिषद्वार्तिकम् - (A. S. S. 13)
शाङ्करतैत्तरीयभाष्यस्य पद्यमयी वार्तिकनाम्नी व्याख्या आनन्दाश्रममुद्रणालये वाराणसीग्रन्थमालायाञ्च मुद्रिता । अस्या व्याख्याः - आनन्दगिरिकृता वार्तिकटीका, विश्वानुभवकृता वार्तिकसङ्गतिः, लिङ्गनसोमयाजिकृतम् - कल्याणविवरणम् , विद्यन्ते ॥
तैत्तरीयोपनिषत्
%%% Chart
२. नैष्कर्म्यसिद्धिः - (B. S. P. S. 38.)
प्रकरणग्रन्थोऽयं बाम्बे संस्कृतप्राकृतग्रन्थमालायां मुद्रितः । बन्दरकार प्राच्यभाषाग्रन्थमालायाञ्च मुद्रितः । अस्य व्याख्याः - अखिलात्मकृतं विवरणम्, ज्ञानोत्तमकृता चन्द्रिका, ज्ञानामृतयतिकृता विद्यासुरभिः, चित्सुखकृता भावतत्वप्रकाशिका, अज्ञातकर्तृका सम्बन्धोक्तिरिति विद्यन्ते ।
३. प्रणवार्थकारिकाः - (551 TMPL)
अमुद्रितोयं ग्रन्थ अनन्तशयनराजगृहपुस्तकालये लभ्यते ।
४. पञ्चीकरणवार्तिकम् -
ग्रन्थोऽयं कामकोटि सूच्यां दृश्यते । कामकोटिकोशस्थाने मुद्रितश्च । नासिकसूच्यान्तु अस्य व्याख्याया एव वार्तिकाभरणमिति नाम दृश्यते । अभिनवनारायणेन्द्र कृता वार्तिकाभरणनाम्नी, शिवनारायणानन्दकृता विवरणदीपिकानाम्नी च व्याख्ये विद्येते ।
५. बृहदारण्यकोपनिषद्भाष्यवार्तिकम् - (ASS. 16)
अस्य व्याख्याः - आनन्दगिरिकृता - शास्त्रदीपिका, आनन्दपूर्णकृता न्यायकल्पलतिका, नृसिम्हाश्रमिकृतं न्यायतत्वविरणम् , विद्यारण्यकृतः वार्तिकसारः, विश्वानुभवकृता सम्बन्धोक्तिश्च विद्यन्ते ।
६. मानसोल्लासः
शाङ्करद क्षिणामूर्तिस्तोत्रव्यात्मकोऽयं ग्रन्थः अडयारपुस्तकालये मुद्रितः । अस्य व्याख्या रामतीर्थकृता मानसोल्लासवृत्तान्तविलासनाम्नी विद्यते ।
७. मोक्षनिर्णयः - (R. 2603 A. MGOML)
काशीमोक्षनिर्णयापरनामायं ग्रन्थ अमुद्रितः मद्रासमैसूरपुस्तकालययोर्लभ्यते ।
८. वेदान्तसारवर्तिकराजसंग्रहः - (7736 TSML) ग्रन्थोऽयं (DC TSML Vol. XIII) मुद्रितः ।
९. लघुवार्तिकम् - अस्य कर्ता न सुरेश्वरः परन्तु उत्तमश्लोक इति निश्चयः ।
२९. तोटकाचार्यः (800-900 A. D.)
शङ्करभगवत्पादशिष्येष्वन्यतमोऽयं तोटकाचार्य उदङ्कापरनामा शिलादपुत्र इति शङ्करविजयात् ज्ञायते । गिरिरितिनामान्तरमप्यस्यैव । तोटकाचार्य इति नाम तु तोटकवृत्तघटितग्रन्थप्रणयितृत्वेन सञ्जातम् ।
उदङ्कापरनामा गिरिनामकोऽयं शङ्कराचार्ये निष्कामभक्तिमान् नितान्तं श्रद्धधानश्च मूत्वा गुरुचरणसेवापर आसीत् । कस्मिंश्चिद् दिने गिरिनामायं शिष्यः दिनचर्यादिकं कृत्वा भाष्यपाठस्यागमने चिरयति स्म । तदागमनप्रतीक्षावन्तं शङ्कराचार्यं पण्डितप्रकाण्डाः तीक्ष्णप्रतिभावन्त पद्मपादादयः विद्यामदमत्ताः गिरिं अनाघ्रातशास्त्रगन्धं वदन्तः तदनागमनं भाष्यपाठारम्भस्य न प्रतिरोधकं भवतु इति निवेदयामासुः । आचार्यस्तु पद्भपादादिदर्पपरिहाराय तोटकाचार्यं मनसा अनु जगृहे । नातिचिरादागतः तोटकाचार्यंः आचार्यकटाक्षवीक्षावाप्तवैदग्ध्यः भाष्यपाठान्ते लब्धगुर्वनुज्ञः तोटकवृत्तधटितैः शलोकैः सर्वमद्वैतवेदान्ततत्वं पपाठ । ततः प्रभृति तोटकाचार्याख्यापि अस्य समञ्जातेति साम्प्रदायिकी कथा ।
अनेन स्वग्रन्थे द्रविडः 106 तमे श्लोके, ईश्वरगुप्तः 51 तमे श्लोके च निर्दिष्टौ । द्रविडाचार्यमधिकृत्य पूर्वमेवोपपादितम् । ईश्वरगुप्तमधिकृत्य न किमपि ज्ञातुं पार्यते ।
१. श्रुतिसारसमुद्धरणम् - (ASS 103)
प्रकरणग्रन्थोऽयं गुरुशिष्यकथाश्रवणपद्धत्या अद्वैतसिद्धान्तान् प्रदर्शयन् एकोनाशीत्यधिकशतसंख्याकैः पद्यैः (179) पूर्ण आनन्दानाश्रमे मुद्रितश्च, श्लोकेष्वेतेषु प्रक्षिप्तभागा अपि सन्तीति प्रतिभाति । सच्चिदानन्दयोगिकृता तत्व प्रदीपिका, पूर्णत्मकृष्णकृता व्याख्या, अज्ञातकर्तृनामधेया श्रुतिसारसमुद्धरण सम्बन्धोक्तिरीति व्याख्यास्सन्ति ।
केचित्तु नामैकदेशे नामग्रहणमिति न्यायमनुसन्दधानाः आनन्दगिरिरेव गिरिशब्देन व्यवहृत इति तोटकाचार्यानन्दगिर्योरैक्यं मन्वानाः आनन्दगिरिश्शङ्कर शिष्य इति प्रवदन्ति । परन्तु प्रमाणं किमपि नोपलभामहे ।
३०. हस्तामलकः (800-900 A. D.)
हस्तामलकाचार्योऽयं शङ्कराचार्यशिष्यः अस्य पिता प्रभाकरः । प्रभाकरोऽयं पूर्वमीमांसायां गुरुमतप्रवर्तक इति केचित् । सुरेश्वरादीनामिव हस्तामलकस्यापि कालः नवमशतकमिति विनिश्चीयते ।
श्रीकैवल्याख्ये ग्रामे वसतः प्रभाकरस्य पुत्रत्वेन आविर्भूतदेहः जन्मान्तरानुष्ठितश्रवणादिना निवृत्ताज्ञानः दिव्ययोगीन्द्रः मूढजनैः पिशाचत्वेनाभिमतः जन्मना मूकः, शङ्करभगवत्पादविजययात्रासमये स्वपित्रा शङ्करसन्निधावानीतः, शङ्करभगवता कस्त्वमिति पृष्टः स्वानुभवं वदन् , अन्येषां मुमुक्षूणां अनुभवो मूयादिति प्राणिनामनुकम्पया द्वादशश्लोकैरात्मतत्वं उपनिषदर्थञ्च समधिगम्य प्रतिवचनछलेन विवेकमञ्जरीमाहेति, तादृशप्रतिवचनेन तुष्टश्शङ्करस्तं स्वच्छात्रेषु अन्यतमं स्वीचकारेति पण्डितप्रसिद्धा साम्प्रदायिकी कथानुसन्धेया ।
१. अनुभववेदान्तप्रकरणम् - (D. 4538 MGOML)
नित्योपलब्धिस्वरूपोऽहमात्मा इति प्रतिचतुर्थचरणं दृश्यमानोऽयं ग्रन्थः मद्रासराजकीयपुस्तकालयेऽमुद्रितः प्राप्यते ।
२. विवेकमञ्जरी - (V. V. P.)
हस्तामलकश्लोकाः, हस्तामलकस्तोत्रम् , वेदान्तसिद्धान्तदीपिका, हस्तामलकीयमित्यादिना प्रसिद्धोऽयं ग्रन्थः वाणीविलासमुद्रणालये मुद्रितः । अस्य आनन्दप्रकाशकृता व्याख्या, शङ्कराचार्यकृतं भाष्यम् , स्वयम्प्रकाशमुनिकृता व्याख्या, अज्ञातकर्तृका च व्याख्यास्सन्ति ।
३१. वाचस्पतिमिश्रः (842 A. D.)
वाचस्पतिमिश्रस्य आचार्यसन्मिश्रवाचस्पतिः, षड्दर्शिनीटीकाकृदाचार्यवाचस्पतिरिति नामान्तराणि । माघवीयशङ्करविजये शङ्कराचार्याणां प्रघानशिष्यः पद्मपादाचार्य एवान्यस्मिन् जन्मनि वाचस्पतिमिश्र आसीदिति दृश्यते । वेदधर्मशास्त्र पुराणतन्त्रव्याकरणज्योतिष-छन्दोन्यायवैशेषिक-सांख्ययोगमीमांसा वेदान्तचिकित्साशास्त्रादौ वाचस्पतिनाम्ना बहवः ग्रन्था प्रणीता विद्यन्ते । बहुषु वाचस्पतिषु दार्शनिकौ द्वौ प्रसिद्धतमौ । तयोरपि एकः तत्वचिन्तामणिप्रकाश-न्यायतत्वलोकाख्य-न्यायसूत्रवृत्ति-न्यायसूत्रोद्धारादिकर्ता वाचस्पतिः । अन्यश्च भामती-ब्रह्मतत्वसमीक्षा न्यायसूचीनिबन्धादिकर्ता षड्दर्शिनीटीकाकार आचार्यवाचस्पतिरिति । सोऽयं भामतीप्रस्थानमस्मादेव प्रस्थितमिति विशेषः ।
वाचस्पतिमिश्रदेशः -
वाचस्पतिमिश्रोऽयं कस्मिन् देशे आविरभूदिति गवेषणायां मण्डननिश्र मुरारिमिश्रपार्थसारथिमिश्रसुचरितमिश्रादयस्सर्वेऽपि मैथिला इति मिश्रान्तनामसम्बन्धात् मिथिलावासीत्यनुमातुं शक्यते । केचित्तु नेपालप्रान्ते भामानामकः कश्चन ग्रामः । तत्र वसन् वाचस्पतिमिश्र भामतीग्रन्थं तद्देशप्रसिध्यै तद्देशस्मरणाय वा चकारेति वदन्ति ।
भम्प्रदायसमागता कथा तु वाचस्पतिमिश्रस्य भामतीनाम्नी पत्नी आसीत् । तया तस्य विवाहकाले काचन पण्डितपरिषत् सञ्जाता । वेदान्तशास्त्रे वादस्समजनि । वेदान्तेतरविदुषां वादः युक्तयश्च प्रबला अद्वैतिनां युक्तयः दुर्बलाश्चासन् । ततः प्रभृति अद्वैतवेदान्तयुक्तीरतिप्रबलास्सप्रमाणाश्च कर्तुं यत्नं कुर्वतस्तदर्थं ग्रन्थस्यास्य भामत्याख्यस्य रचयितुर्नवीनं वयः गतम् । ग्रन्थनिर्माणे मग्नस्यास्यापरे वयसि काचन नातिवृद्धा नष्टप्राययौवना स्त्री समायाता । ताञ्च स्वपत्ननीमक्षतयोनिमसञ्जातसन्तर्ति स्वञ्च ग्रन्थनिर्माणमग्नं दृष्ट्वा स्वग्रन्थस्य भामतीति स्वपत्न्याः नाम चकारेति प्रसिद्धा ।
कान्यकुब्जाधिपतेर्युक्तिदीपिकाकारस्य वृद्धभोजस्य सभापण्डित आश्रितश्चायं वाचस्पतिमिश्रः षड्दर्शिनीटीकाकृत् नवमशतकीय इति गुरुपादहालदारः वृद्धत्रय्यां वदति ।
वाचस्पतिमिश्राणामाचार्यः-
तात्पर्यटीकायां वाचस्पतिमिश्रेण ``त्रिलोचनगुरून्नीतमार्गानुगमनोन्मुखै" रिति कथनात् त्रिलोचनमिश्रः वाचस्पतिगुरुरिति ज्ञायते । केचित्तु ``मार्ताण्डतिलकस्वामिमहागणपतीन्वयम् ।" इति भामत्यां मङ्गलाचरणात् तिलकस्वामी आचार्य इति वदन्ति । परन्तु अमलानन्देन ``समस्तशब्दोऽयं मार्ताण्डतिलकस्वामि" शब्दः देवतापर इति व्याख्यातम् । तिलकस्वामिशब्दश्च सुब्रह्मण्यपर इति प्रसिद्धिः । तस्मात्तिलकस्वामिशिष्य इत्यसमञ्जसमिति भाति । उदयनाचार्यै र्न्यायवार्तिकतात्पर्यपरिशुध्यां वाचस्पतिस्त्रिलोचनशिष्य इति प्रतिपादितम् वर्धमामानोपाध्यायैश्च न्यायनिबन्धप्रकाशे उदयनोक्तिरादृता । दासगुप्तस्तु विद्यातरुरपि गुरुरस्येति (HIP Vol. 107) वदति ॥
वाचस्पतिमिश्राश्रयदाता :-
वाचस्पतिमिश्रैर्भामत्यन्ते ``नरेश्वरा यच्चरितानुकारभिच्छन्ति कर्तुं न च पारयन्ति । तस्मिन् महीये महनीयकीर्तौ श्रीमन्नृगेऽकारि मया प्रबन्धः ॥" इति स्वाश्रयदाता निर्दिष्टः । नृगोऽयं राघवदेवपण्डितपौत्रेण दामोदरपुत्रेण शार्ङ्गधरेण ``शार्ङ्गघरपद्धत्यां" विशिष्टराजवंशवर्णनप्रस्तावे निर्दिष्टः । नातोऽधिकं ज्ञायते ॥
वाचस्पतिमिश्राणां कालः-
वाचस्पतिमिश्रैःस्वीये न्यायसृचीनिबन्धे न्यायसृचीनिबन्धोऽयं अकारि सुधियां मुदे । श्रीवाचस्पतिमिश्रेण क्स्वङ्कवसुवत्सर इति निर्दिष्टम् । यदि शकाब्दस्स्यादयं तर्हि 898-976 A. D. इति दशमशतकमस्य कालः । यदि विक्रमाब्दस्स्यात् तर्हि 893 सं 842 A. D. इति  सिध्यति । सर्वथा नवमशतकमध्यवर्तीति तु निश्चयः । दशमशतकीयेन रत्नकीर्तिनाम्ना बुद्धपण्डितेन अपोहसिद्धौ (Page 7 ) भामतीकारः निर्दिष्टः । तस्मादप्यस्य दशमशतकपूर्ववर्तित्वं सिद्यति ॥
१. ब्रह्मसूत्रभाष्यव्याख्या भामती (N. S. P.)
वाराणस्यां , निर्णयसागरमुद्रणालये, वाणीविलासे च मुद्रितोऽयं ग्रन्थः । अस्य व्याख्याः - अखण्डानन्दसरस्वतीकृता ऋजुप्रकाशिका, अच्युतकृष्णानन्दकृता भावदीपिका अल्लालसूरिकृता भामतीनिलकम् , अज्ञातकर्तृका भामतीविलासः, अमलानन्दकृतः, कल्पतरुरिति व्याख्यापञ्चकमत्र निर्दिष्टम् ।
भामाती
%%% Chart
2. ब्रह्मतत्वसमीक्षा-
ग्रन्थोऽयं न्यायवर्तिकतात्पर्यटीकायास्तृतीयेऽधिकरणे द्वितीयाह्निके, भामत्यां (Page 64 VVP. Edn) च निर्दिष्टः । ग्रन्थस्तु नोपलभ्यते मण्डन मिश्रकृतब्रह्मसिद्धिव्याख्यात्मकोऽयमिति वदन्ति ।
अन्येऽपि न्यायवार्तिकतात्पर्यटीका, न्यायसूचीनिवन्धनम्, तत्ववैशारदी, सांख्यतत्वकौमुदी, तत्वबिन्दुः, न्यायकणिकादयः ग्रन्थाः न्यायमीमांसायोगशास्त्रेष्वारचिताः ।
३२. विमुक्तात्मा (950-1050 A. D.)
अव्ययात्मभगवत्पादशिष्योऽयं विमुक्तात्मा (1050) कालिकेनानन्दबोधयतिना प्रमाणमालायां (1150 A. D.) कालिकेन यामुनाचार्येण सिद्धित्रये (1150) कालिकेन सङ्खकेन श्रीकण्ठचरिते चतुर्दशशतकीयेन रामाद्वयेन वेदान्तकौमुद्यां निर्दिष्टः । नवमशतकीयः भास्कराचार्यः विमुक्तात्मना खण्ड़यते । एवञ्चायं विमुक्तात्मा दशमशतकीयापरार्धकाले उवासेति निश्चीयते । दासगुप्तमहाशयस्तु द्वादशशतकीयमेनं वर्णयति । हिरियण्णामहाशस्तु (JOR. Vol. V Page 32) पत्रिकायां (850 A. D.) कालात्प्राचीनो न भवतीति वदति ।
संक्षेपशारीरककर्त्रा चतुर्दशतमे पद्ये चतुर्थाध्याये संक्षेपशारीरके मुक्तिकोविदशब्देन विमुक्तात्मा निर्दिष्टः । इष्टसिद्धिकारविमुक्तात्म एव मुक्तिकोविदशब्देन प्रतिपादित इति संक्षेपशारीरकसारसंग्रहकर्त्रा मधुसूदनसरस्वत्या, नृसिम्हाश्रमिणा रामतीर्थेन च प्रतिपादितम् । तस्मात् सर्वज्ञात्मनः पूर्वतनस्स्यादिति निर्णीयते ।
१. इष्टसिद्धिः (G. O. S. S. 65)
अष्टभिः परिच्छेदैः पूर्णोऽयं ग्रन्थः भास्करमतं खण्डयति । मुद्रितश्चायं बरोडाग्रन्थमालायाम् । अनुभूतिस्वरूपकृतं आनन्दानुभवकृतं ज्ञानोत्तमकृतञ्च विवरणमस्य विद्यते ।
२. प्रमाणवृत्तिनिर्णयः ।
ग्रन्थोऽयं इष्टसिद्धौ (Page 37) निर्दिष्टः ।
३३. प्रकाशात्मा (1000 A. D.)
स्वयम्प्रकाशानुभवापरनामायं प्रकाशात्मा अनन्यानुभवशिष्यः तत्वशुद्धिकारस्य ज्ञानधनस्य सामयिक इति च ज्ञायते । अस्य गुरुणा अनन्यानुभवेन आत्मतत्वमिति ग्रन्थः कृतः, यश्च ज्ञानघनेन तत्वशुद्धौ उद्धृत इति श्रीकण्ठशास्त्री (I. H. Q. Vol XIV) ।
वेदान्ते प्रस्थानद्वयस्य प्रचारे प्रकाशात्मा भामतीकारवाचस्पतिश्च कारणम् । अनयोः प्रस्थानभेदस्य कारणं-एकेश्वरवाद अविद्याश्रयादिवादयोरङ्गीकारानङ्गीकार एवेति च प्रसिद्धम् ।
विशिष्टाद्वैतिना रामानुजेन श्रीभाष्ये परिशीलितोऽयं प्रकाशात्मा तस्मात् पूर्वतन इति तु निश्चयः । परन्तु दासगुप्तः (1200-1300 A. D.) अस्य काल इति वदति । श्रीकण्ठशास्त्री तु (950-1050 A. D.) काल एवास्येति प्रतिपादयति (I. H. Q. Vol XIV) ।
१. पञ्चपादिकाविवरणम् (V. N. S. S. 5)
पद्मपादीयपञ्चपादिकाव्याख्यात्मकोऽयं ग्रन्थः विजयनरसंस्कृतग्रन्थमालायां, कल्कत्तासस्कृतग्रन्थमालायां च मुद्रितः ।
अस्य व्याख्या :-
पञ्चपादिकाविवरणम्
%%% Chart
एषु तत्वदीपनमखण्डानन्दमुनिकृतम् , विवरणदर्पणं अमलान्दकृतम्, भावद्योतनिका चित्सुखकृता, टीकारत्नं आनन्दपूर्णकृतम्, ऋजुविरणं विष्णुभट्टोपाध्यायकृतम्, विवरणप्रमेयसंग्रहः विद्याण्यकृतः, भावप्रकाशिका नृसिम्हाश्रमिकृता, विवरणोपन्यासः रामामन्दसरस्वतीकृतः, विवरणोज्जीविनी यज्ञेश्वरदीक्षितकृता, रामतीर्थकृता व्याख्या, कृष्णकृता व्याख्या, परिव्राजककृता भावप्रकाशिकेति च ज्ञेयम् ।
२. शाब्दनिर्णयः (T. S. S. 53)
वृत्तिसंवलितकारिकामयोऽयं ग्रन्थः (प्रकरणग्रन्थः) शब्दात् ब्रह्मापरोक्ष ज्ञानं जायत इत्युपवर्णयति । तिरुववनन्तपुरपुस्तकमालायां मुद्रितश्च । अस्य व्याख्या आनन्दबोधभट्टारककृता शाब्दनिर्णयदीपिकाख्या च मद्रासहस्तलिखितपुस्तकालये लभ्यते ।
३. शारीरकन्यायसंग्रहः (M. U. S. S.)
ब्रह्मसूत्रवृत्तिरूपोऽयं ग्रन्तः 193 न्यायान् प्रतिपादयति । ग्रन्थोऽयं मद्रासविश्वविद्यालयमालायां मुद्रितः ।
४. लौकिकन्यायसंग्रहः । ? अयं कुत्र लभ्यत इति न ज्ञायते । परन्तु प्रकाशात्मकृत इति श्रूयते ।
३४. सर्वज्ञात्मा (1050 A. D.)
``श्रीदेवेश्वरपादपङ्कजरजःसम्पर्कपूताशय" इति वदन्नयं सर्वज्ञात्मा देवेश्वरशिष्य इति निश्चीयते । कोऽयं देवेश्वर इति संशये सुरेश्वर एव देवेश्वर शब्देन निर्दिष्ट इति सुरेश्वरशिष्यश्शङ्कराचार्यप्रशिष्य इति च बहवो वदन्ति । एवं संक्षेपशारीरकस्य मधुसूदनसरस्वतीव्याख्याअन्वयार्थप्रकाशिकायां दृश्यते च सर्वज्ञात्मायं कामकोटिपीठाधिष्ठाता चासीदिति च वदन्ति ।
सर्वज्ञात्मना संक्षेपशारीरकोपान्त्यश्लोके ``श्रीमत्यक्षतशासने मनुकुलादित्ये भुवं शासति" इति स्वसामायिको राजा निर्दिष्टः । तस्य राज्ञो नाम श्रीमानिति केचित् । परन्तु मनुकुलादित्यनामा राजा (1027 A.D.) काले अनन्तशयनराज्यं प्रशशास । तथा च तत्कालीनस्तद्देशीयश्चेति वदन्ति ।
देवेश्वरो न सुरेश्वरः, परन्तु सर्वज्ञात्मकृत प्रमाणलक्षणप्रमाणेन श्रेष्ठानन्दस्य प्रशिष्यः, देवानन्दस्य शिष्योऽयं देवेश्वर अन्य एवेति T. R. चिन्तामणि प्रभृतयः वदन्ति । श्रीकण्ठशास्त्रिभिश्च (IHQ Vol XIV & J.O.R. 1937) पत्रिकायां तथैव निरूपितम् ।
दासगुप्तमहाशयैस्तु (HIP. Vol. II 112) सुरेश्वरशिष्य एवायं सर्वज्ञात्मा, सर्वज्ञात्मनः नित्यबोधाचार्य इति नामान्तरमपीति प्रतिपाद्य नवमशतकीयोऽयं सर्वज्ञात्मा इति प्रतिपाद्यते । नायं सुरेश्वरशिष्यः परन्तु इष्टसिद्धिकारात् विमुक्तात्मनः अर्वाचीन इति श्रीकण्ठशास्त्रिणा प्रतिपाद्यते ।
%%% Chart
१. संक्षेपशारीरकम् - (S. B. S. 69)
समन्वयाविरोधसाधनफलाख्यैश्चतुर्भिरध्यायैः पूर्णोऽयं प्रकरणन्थः पद्यमयः भाष्यार्थविचारणपरः सरस्वतीभवनग्रन्थमालायां आनन्दाश्रममुद्रणालये च मुद्रितः । अस्य व्याख्याः-नृसिम्हाश्रमिकृता तत्वबोधिनी, पुरोषोत्तमदीक्षितकृता सुबोधिनी, प्रत्यग्विष्णुकृता व्याख्या, मधुसूदनसरस्वतीकृतः सारसङ्ग्रहः, राघवानन्दकृता विद्यामृतवर्षिणी, रामतीर्थकृता अन्वयार्थप्रकाशिका, विश्ववेदकृतस्मद्धान्तदीपः, वेदानन्दकृता संक्षेपशारीकसंबन्धोक्तिरिति व्याख्यास्सन्ति । महादेववेदान्तिकृतः लघुसंक्षेपशारीरकम् (संक्षेपशारीरककुतूहलम्) इति ग्रन्थः हरप्रसादशास्रिसूच्यां दृश्यते । (H. P. R. III No 261)
2. पञ्चप्रक्रिया - (MUSBS 4)
ग्रन्थोऽयं मद्रासराजकीयहस्तलिखितपुस्तकालयेऽपि (R 1361 B.) लभ्यते । अस्य व्याख्या पूर्णविद्यकृत । नायं संक्षेपशारीरककारकृतिरिति दासगुप्तः ॥
३. प्रमाणलक्षणम् - (D. 1576 MGOML)
३५. आनन्दबोधः (1050-1150 A. D.)
आनन्दबोधोऽयमात्मावासशिष्यः उत्तरभारतदेशवासीति शब्दनिर्णयव्याख्याया अवगम्यते । प्रकाशात्मनश्शाब्दनिर्णयस्य व्याख्यां कुर्वन् आनन्दगिरिणा व्याख्यातोऽयं तयोरुभयोर्मध्यकालवर्तीति ज्ञायते । ``एतदेवोक्तं गुरुभिरिति" इष्टसिद्धिकारः विमुक्तात्मा आनन्दबोधेन निर्दिश्यत इति आनन्दबोधोऽयं विमुक्तात्मशिष्य इत्यपि केचिद्वदन्ति ।
१. न्यायदीपावलिः-(BSS. 38, 62, 87, 117)
पञ्चावयववाक्यैर्नैय्यायिकप्रक्रियानिरसतनपूर्वकं भेदवादं खण्डयन् अद्वैतप्रक्रियां साधयति । अस्य व्याख्या - अमृतानन्दमुनिना कृता न्यायविवेकाख्या, सुखप्रकाशकृता व्याख्या, अनुभूतिस्वरूपकृता चन्द्रिकाख्या, आनन्दगिरिकृतां वेदान्तविवेकाख्याश्च विद्यन्ते ।
२. न्यायमकरन्दः - (BSS. 38, 62, 87, 117)
नैय्यायिकाभिमतं भेदवादं खण्डयति वाचस्पतिमिश्रमम्डनमिश्रपञ्चपादिकाविवरणकारान् प्रमाणयति । अनिर्वचनीयख्यातिं जीवब्रह्मणोरैक्यं मोक्षस्वरूपं च प्रतिपादयति । नानानिबन्धकुसुमप्रभवावदातेति वदन्नयं ग्रन्थः प्राचीनान् सिद्धान्तान् संगृहणाति ।
३. प्रमाणमाला - (BSS. 38)
न्यायमकरन्दप्रतिपादितान् सिद्धान्तान् संगृह्नणन् वाचस्पतिमिश्रकृतां ब्रह्मतत्वसमीक्षां प्रमाणयन् प्रकरणग्रन्थतामहेत्ययं ग्रन्थः । अस्य व्याख्याऽनुभूतिस्वरूपाचार्यकृता निबन्धनाख्या । अन्या काचन अज्ञातकर्तृनामधेया व्याख्या वाराणस्यां विद्यते ।
४. शाब्दनिर्णयव्याख्या - ग्रन्थोऽयं मद्रासराज़कीयपुस्तकालये लभ्यते ।
३६. श्रीहर्षः (1075-1175 A. D.)
श्रीहीरपण्डितमामल्लदेव्योः पुत्रोऽयं श्रीहर्षः कान्यकुब्जदेशवासी जयचन्द्रसभापण्डितः द्वादशशतकीय निश्चयः ।
हीरपण्डितः कान्यकुब्जाधीशस्य (कन्नैज) हड़तालवंश्यस्य विजयचन्द्रस्य सभायां पण्डित आसीत् । अस्य पत्नी मामल्लदेवी । एनमधिकृत्य सम्प्रदायसमागता कथेयं प्रसिद्धा - ``श्रीहीरपण्डितः वादे उदयनाचार्येण पराभूतः । श्रीहीरः स्वन्यक्कृतिपरिहाराय स्वन्यक्कृतिप्रतीकाराय वा श्रीहर्षमादिश्य ममार । तस्मात् श्रीहर्षः" चिन्तामणिमन्त्रोपासनया लब्धवाग्वैभवः उदयनं वादे पराजित्य स्वपितुरानृण्यतां गतः ।
श्रीहर्षसामयिक राजा-
`ताम्बूलद्वयमासनञ्च लभते यः कान्यकुब्जेश्वरात्' इति खण्डनखाद्ये दर्शनात् श्रीहर्षोऽयं विजयचन्द्रपुत्रस्य जयचन्द्रस्य सभायां पण्डित आसीत् । अस्य स्थानं कान्यकुब्ज इति च निर्धार्यते । विषयेऽस्मिन् 
गोविन्दनन्दनतया च वपुःश्रिया च
मास्मिन् नृपे कुरुत कामधियं तरुण्यः ।
अस्त्रीकरोति जगतां विजये स्मरः
स्त्रीः अस्त्रीजनः पुनरनेन विधीयते स्त्रीः ॥
इति प्रसिद्धं पद्यं श्रूयते । गोविन्दनन्दनतया चेति शब्दस्वारस्यात् जयचन्द्रपितुर्विजयचन्द्रस्य गोविन्द इति नामान्तरं स्यादिति संशयोपि ॥
प्रकाण्डपण्डितः वाग्मी चिन्तामणिमन्त्रप्रसादलब्धवैदुष्यः श्रीहर्ष इति सार्वजनीनम् । अनेन नक्तंदिवं अनशनो भूत्वा चिन्तामणिमन्त्राराधना कृता । मन्त्रप्रसादात् अनेन तादृशी कवित्वशक्तिः एतादृशी वाग्वैखरी समासादिता यया कृताया एतदीयकविताया अर्थावगमे, एतदीयशास्त्रार्थविचारस्य तत्वावगमे वा पण्डिता कुण्ठिता अभूवन् । तदर्थं श्रीहर्षः यामिन्यास्त्रितीये यामे प्रतिदिनं स्नात्वा माहिषदधिमिश्रमन्नं भक्षयामास । तादृशाहारदोषात् सञ्जातबुद्धिप्रतिभादिमान्द्यः परिणतः । एतादृश्यां परिणतावस्थायां अनेन नैवधीयचरितं खण्डनखण्डखाद्यञ्च प्रणीतमिति कथापि सम्प्रदायसमागता श्रूयते ।
श्रीहषोंऽयं काव्यप्रकाशकारस्य मम्भटभट्टस्य भागिनेय इति केषाञ्चिन्मतम् । विषयेऽस्मित् ``तव वर्त्मनि वर्ततां शिवं पुनरस्तु त्वरितं समागप्तः" इति पद्यांशे सन्ध्यश्लीलदोषप्रयुक्तं (तव वर्त्म निवर्तताम्) मम्मटश्रीहर्षयोर्विवादं प्रमाणयन्ति ॥
जयचन्द्रसामयिकोऽयम् । देहलीशाय पृथ्वीराजाय द्रुह्य असूयावतश्च जयचन्द्रस्य प्रेरणया यवनराजः गोरीनामकः देहलीनगरे पृथ्वीराजोपरि सेनाक्रमण चकारेति, महान् भयावहो युद्धस्समजनीति चेतिहासः । स च युद्धः 1191 A. D. द्वादशशतकान्ते इति च वदन्ति । एवञ्चास्य कालः द्वादशशतकमिति ।
1348 A. D. कालिकेन राजशेखरेण रचिते प्रबन्धकोशाख्ये ग्रन्थे श्रीहर्षः परिशीलितः । श्रीहर्षेण एकादशशतकस्य पूर्वार्धवासी महिमभट्टः तत्कृतिर्व्यक्तिविवेकश्च खण्डनखण्डखाद्ये निर्दिष्टः । एकादशशतकस्योत्तरार्धवासिना 1050 A. D. प्रसिद्धेन धारापतिना भोजराजेन निर्मिते सरस्वतीकण्ठाभरणे प्रख्यातकवि नामसु श्रीहर्षनामनिर्देशो नास्ति । प्रसिद्धनैय्यायिकः तत्वचिन्तामणिकारः गङ्गेशोपाध्यायः द्वादशशतकीयः श्रीहर्षं निर्दिशति । एवञ्च महिमभट्टात् भोजराजात् अर्वाक्तनः, राजशेखरात् गङ्गेशोपाध्यायाच्च प्राक्तन इति सिघ्यति ।
केचित्तु उदयनाचार्यसामयिक इति वदन्ति । श्रीहर्षेणोदयनाचार्यः खण्डने खण्ड्यते । उदयनकालश्च तर्काम्बराङ्कप्रमितशकः (906-984 A. D.) इति लक्षणावल्याः ज्ञायते । तस्मादनन्तरभावीत्येव निर्णेंतु शक्यते ।
१. खण्डनखण्डखाद्यम् (B. S. S. 81, 109)
शून्यवादिमतं नैय्यायिकमतञ्च खण्डयति । अद्वैतवेदान्तशास्त्रे नव्यन्यायशैलीप्रवेशजं यशः प्रथमत अस्यैव ग्रन्थस्य । न्यायशास्त्रीयतर्कशैल्या न्यायसिद्धान्ताः खण्डिताः । नैय्यायिकाभिमतान् । पदार्थान् निरस्य, तदभिमतं द्वैतवादमपि निरस्य अद्वैतवादस्स्थापितः । सर्वेषामलौकिकानां लौकिकानाञ्च पदार्थानां अनिर्वचनीयता प्रमाणैर्निरूपिता । अत एवास्य ग्रन्थस्य ``अनिर्वचनीयतासर्वस्व"मिति व्यपदेशोऽपि युज्यते । ग्रन्थस्यास्य खण्डनखण्डं, खण्डनखाद्यं,स्वाद्यखंडनं, खण्डनमित्यपि नामान्तराणि श्रूयन्ते . ग्रन्थोऽयं चौखाम्बामुद्रणालये मुद्रितः । अस्य व्याख्याः ।
शङ्करमिश्रकृता आनन्दवर्धिनी, चित्सुखकृत्ता काचन व्याख्या, पद्मनाभपण्डितकृता खण्डनटीका, भवनाथविरचितं खण्डनमण्डनम्, परमानन्दविरचितं खण्डनमण्डनम्, वरदराजकृतं खण्डनमण्डनम्, चरित्रसिंहकृतः खण्डनमहातर्कः, प्रगल्भमिश्रकृतं खण्डनखण्डमम्, गोकुलनाथकृतः खण्डनकुठारः, वाचस्पतिकृतः खण्डनोद्धारः, रघुनाथशिरोमणिकृता दीधितिः, वर्धमानकृतः प्रकाशः, विद्याभरणकृता विद्याभरणी, आनन्दपूर्णकृता विद्यासागरी, अनुमूतिस्वरूपकृता शिष्यहितै षिणी, शुभङ्करकृतं श्रीदर्पणम्, अज्ञातकर्तृकं अद्वैतबोधामृतम्, अज्ञातकर्तृका शारदा, सूर्यनारायणशुक्लकृता खण्डनरत्नमालिका चेति । आसु खण्डनकुठारः, खण्डनोद्धारश्च श्रीहर्षदोषोद्धाराय कृतौ । आनन्दवर्धना तात्पर्यमात्रप्रकाशिका, न तु प्रतिपदव्याख्यारूपा ।
नैषधीयचनरितमित्यादयः इतरे नव-काव्य-ग्रन्था अपि रचिता इति नैषधीयचरितात् ज्ञायते ।
३७. आनन्दानुभवः (1100-1200 A. D.)
``बन्दे तस्य पदारविन्दयुगलं नाराणज्योतिष" इति स्वग्रन्थेषु वदन्नयं आनन्दानुभवः नारायणज्योतिश्शिष्य इति निश्चीयते । एतेन कृताः त्रयः ग्रन्थाउपलभ्यन्ते । तेषु पदार्थतत्वनिर्णयः आनन्दानुभवकृत इति मद्रासपुस्तकालयस्थात् पुस्तकात् ज्ञायते । परन्तु ``तर्कसंग्रह" भूमिकायां त्रिपेदीमहाशयेन ``पदार्थतत्वनिर्णयः" गङ्गापुरीभट्टारककृत इति वर्णितम् । आनन्दानुभवस्यैव आश्रमस्वीकरात्पूर्वं गङ्गापुरीभट्टारक इति नामान्तरं स्यादिति त्रिपेदीमहाशयवचनमपि सम्भाव्यते ।
दासगुप्तस्तु (950-1050 A. D.) अस्य कालं वदति । न्यायमकरन्दकारादानन्दबोधादर्वाचीनोऽयमिति (IHQ Vol. XIV) श्रीकण्ठशास्त्रिभिः प्रतिपादितम् । विमुक्तात्मकृतस्य इष्टसिद्धिग्रन्थस्य व्याख्यां कुर्वन् आनन्दगिरिणा व्याख्यातेऽयं तयोर्मध्यकालिक इति तु निश्चयः ।
१. इष्टसिद्धिविवरणम् - (39. A. 3. AL) ग्रन्थोऽयमडयारपुस्तकालये लभ्यते अमुद्रितश्च ।
२. न्यायरत्नदीपावली - (38. H. 15. AL)
समन्वय-अविरोध साधन - फलाख्यैश्चतुर्भिरध्यायैः पूर्णोऽयं प्रकरणग्रन्थः अमुद्रित अडयारपुस्तकालये लभ्यते । अस्य व्याख्या आनन्दगिरिणा कृता । मुद्रितश्चायं मद्रपुर्याम् (MGOMLS)
३. पदार्थतत्वनिर्णयः - (R. 2981 MGOML, 41. A. 39. AL)
ग्रन्थोऽयं नैय्यायिकाभिमतान् वैशेषिकाभिमताँश्च पदार्थान् खण्डयति । अद्वैतदिशा पदार्थान् विभजते च । ग्रन्थोऽयममुद्रितः मद्रासपुस्तकालये लभ्यते । अस्य व्याख्या आनन्दगिरिणा, आत्मस्वरूपेण च कृता ।
३८. अनुभूतिस्वरूपाचार्यः (1100-1300 A. D.)
स्वग्रन्थेषु हयग्रीवं प्रणमन्नयं अनुभूतिस्वरूपः दक्षिणदेशीयोऽपि सन् न्यायवेदान्ताध्ययनाय गुजरातिप्रान्तीयकठियावारप्रान्तीयद्वारकादिक्षेत्रेषु उवास । न्यायव्याकरणवेदान्तेषु निष्णातोऽयं आनन्दगिरेः, प्रज्ञानानन्दस्य, नरेन्द्रनगर्याख्यस्य च गुरुरिति ज्ञायते । स्वरचिते खण्डनखण्डव्याख्याने शिष्यहितैषिष्याख्ये श्रीहर्षं प्रणमन्नय श्रीहर्षशिष्य इति ख्यातः ।
वेदान्तवेद्यस्यात्मनः बहुत्वासम्भवे वाच्ये सति पुंसूशब्दस्य सप्तमीबहुवचने कगागमो भविष्यतीति स्वोच्चरितपुंक्षु-इति रूपस्य साधुत्वप्रदर्शनायानेन एकरात्र्यामेव सारस्वतसूत्राणि सवृत्तीनि रचितानीति कथा पण्डितसम्प्रदायसमागता श्रूयते ।
द्वादशशतकीयस्य आनन्दबोधाचार्यकृतस्य न्यायमकरन्द-न्यायदीपावली - प्रमाणमालाख्यग्रन्थत्रयस्यापि व्याख्यां कुर्वन् एकादशशतकीयप्रथमपादोत्पन्नस्य इष्टसिद्धिग्रन्थस्य व्याख्यां रचयन् आनन्दगिरिणा अभिवन्द्यमानोऽयं अनुभूतिस्वरूपः क्रैस्तवीयद्वादशशतकमध्यादारभ्य त्रयोदशशतकमध्यपर्यन्तमुवासेति ज्ञायते ।
१. इष्टसिद्धिविवरणम् - (TCL 656) अमुद्रितोऽयं ग्रन्थ अनन्तशयनपुस्तकालये लभ्यते ।
२. गौडपादीयभाष्यटिप्पणी - (R. 5911 MGOML)
माण्डूक्यकारिकाशाङ्करभाष्यव्याख्यात्मकोऽयं ग्रन्थः मद्रासहस्तलिखितपुस्तकालये लभ्यते ।
३. चन्द्रिका - (D. 15306 MGOML)
आनन्दबोधीयन्यायदीपावलिव्याख्यात्मकोऽयं ग्रन्थः मद्रासपुस्तकालये वीकानेरपुस्तकालये चामुद्रित उपलभ्यते ।
४. निबन्धनम् (प्रमाणरत्नमाला व्याख्या) - (R. 3268 MGOML)
5. प्रकटार्थविवरणम् - (MUS 9)
ब्रह्मसूत्रशाङ्करभाष्यव्याख्यात्मकोऽयं ग्रन्थः मद्रासविश्वविद्यालयपुस्तकमालायां मुद्रितः । पञ्चपादिकाविवरणस्य गूढार्थत्वेन व्याख्यापेक्षतां प्रकटार्थविवरणस्य तदनपेक्षतां च प्रकटीकुर्वन्नयं अन्वर्थनामा ग्रन्थः ``वाचस्पतिस्तु मण्डनपृष्ठसेवी"ति भामतीखण्डनपरः उदयनसुन्दरपाण्ड्याचार्यब्रह्मप्रकाशिकाकारान् प्रमाणयन् भाष्यार्थं विशदयति ।
यद्यपि भुद्रिते ग्रन्थे अनुभूतिस्वरूपाचार्यनाम न निर्दिष्टम्, नापि कल्पतर्वादौ प्रकटार्थविवरणखण्डनावसरे अनुभूतिस्वरूपनाम निर्दिश्यते, प्रकटार्थविवरणस्य द्वादशशतकापरभागकालिकत्वं बदता दासगुप्तेनापि ननिर्दिष्टम्, तथापि बन्दरकाररजतजयन्तीस्मारकपत्रिकायां राघवमहाशयेन प्रकटार्थकार अनुभूति स्वरूप इति सिद्धान्तिम् ।
६. भगवद्गीताभाष्यटिप्पणी (231 T. C. L.) ग्रन्थोऽयं अडयारपुस्तकालये अनन्तशयन पुस्तकालये च प्राप्यते ।
७. शिष्यहितैषिणी (खण्डनखण्डखाद्यव्याख्या)
ग्रन्थोऽयं जैनबन्दरगारपुस्तकालयसूच्या 27 पुटे दृश्यते । सारस्वतप्रक्रियापि अदसीया कृतिः ।
३९. चित्सुखाचार्यः (1120-1220 A. D.)
चित्सुखप्राचार्यः -
चित्सुखाचार्यस्य प्राचार्यः ज्ञानधन इति ख्यातः । अस्यैव बौधधन इत्यपि नामान्तरम् । बोधधनापरनाम्ना तत्वशुद्धिः कृता ज्ञानधनेन । ज्ञानघनोऽयं नित्यबोधधनशिष्य इति शृङ्गगिरिसूच्यां दृश्यते । अस्य कालश्च (848-910 A. D.) इति । एवञ्च ज्ञानघनः नित्यबोधघनशिष्यः । ज्ञानघनस्य शिष्यः ज्ञानोत्तम इति च ज्ञायते ।
चित्सुखाचार्यगुरुः-
ज्ञानधनशिष्यः ज्ञानोत्तमः चित्सुखाचार्यगुरुः । `तत्वप्रदीपिका' अन्ते अयं ज्ञानोत्तमः गौडेश्वराचार्यशब्देन विशेषितः । ज्ञानोत्तमाख्यं तद्वन्दे सत्यानन्दपदोदितम् इति तत्वप्रदीपिकायां, तात्पर्यटीकायां, भावतत्वप्रदीपिकायां च ज्ञानोत्तमस्यैव सत्यानन्द इति नामान्तराण्यपि दृश्यन्ते । सोऽयं ज्ञानोत्तमः शृङ्गगिरि गुरुपरम्परायां (910-953 A. D.) काले आसीदिति ज्ञायते ।
अद्वैतवेदान्तसाहित्ये ज्ञानोत्तमद्वयं वर्तते । तयोरेकः चित्सुखगुरुः प्रकृतः । अन्यस्तु नैष्कर्म्यसिद्धि-इष्टसिद्धिव्याख्याकारः चोलदेशीयः मङ्गलग्रामवासी दाक्षिणात्यः । चित्सुखगुरुश्च गौडदेशीयः । T. R. चिन्तामणिश्च चित्सुखाचार्यगुरोर्ज्ञानोत्तमात् नैष्कर्म्यसिद्धिव्याख्याता ज्ञानोत्तमो भिन्न इत्येव वदति । हिरियण्णा च चिन्तामणिमतमाराधयन् एवं वदति योऽयं सत्यानन्दापरनामा ज्ञानोत्तमः चित्सुखाचार्येण निर्दिष्टः स चित्सुखाचार्यात् प्राचीनः । सर्वज्ञपीठामधिरोहमाणानां काञ्चीकामकोटिपीठाधीशानां नामद्वयवत्वं प्राचीनपरम्परासिद्धम् । एवञ्च कामकोटिगुरुपरम्पराप्रामाण्यात् ज्ञानोत्तमः शङ्करात् चतुर्थपीठाधिपतिः । तत्प्रणीतां नैष्कर्म्यसिद्धिव्याख्यां चित्सुखीया नैष्कर्म्यसिद्धिव्याख्या अनुसरतीति । दासगुप्तस्तु मङ्गलग्रामवासिनमेनं नवीनंवदति ।
गौडेश्वराभिधानेन चित्सुखगुरुणा ज्ञानोत्तमेन न्यायसुधा, ज्ञानसिद्धिरिति द्वौ ग्रन्थौ प्रणिताविति तत्वप्रदीपिकाया (Page 392) ज्ञायते । अप्पय्यदीक्षितैश्च सिद्धान्तलेशसंग्रहे (269-270) न्यायसुधा निर्दिष्टा । श्रीकण्ठशास्त्रिभिः (1HQ Vol. XIV P 401) ज्ञानसुधाख्योपि ग्रन्थः निर्दिश्यते । एवञ्च न्यायसुधा ज्ञानसिद्धि - ग्रन्थकारः गौडेश्वरापराभिधः ज्ञानोत्तमः चित्सुखाचार्यगुरुरिति ज्ञायते ।
चित्सुखकालः -
1268 - 1369 (A. D.) कालवर्तिना वेदान्तदेशिकेन चित्सुखाचार्यस्स्वग्रन्थे निर्दिष्टः । एवं (1187 A. D.) कालवर्तिनः श्रीहर्षस्य कृतेः खण्डनखण्डन खाद्यस्य चित्सुखाचार्येण व्याख्या कृता । एवं (1050-1150 A. D.) कालवर्तिन आनन्दबोधस्य कृतेः व्याख्या कृता । तस्मात् (1120-1220 A. D.) चित्सुखकाल इति सिध्यति ।
परन्तु शृङ्गगिरिमठीयगुरुपरम्परासूची यदि प्रामाणिकी तत्र दृश्यमान ज्ञानोत्तमकालश्च (953 A. D.) यदि प्रामाणिकाः, यदि स एव ज्ञानोत्तमश्चित्सुखाचार्यगुरुस्तर्हि (1053 A. D.) चित्सुखसमय इति सिध्यति । श्रीकण्ठशास्त्रिभिस्तु स्वीये प्रबन्धे चित्सुखकालः (1180 A. D.) इति वर्णितम् । परंतु ज्ञानोत्तमः कः इति न निर्णीतम् । यदि स एव स्यात्तर्हि द्वादशशतकात् शतवर्षेभ्यः पूर्वमिति वक्तव्यमापतति । स चायं सन्देहः शङ्कराचार्यस्य कालस्सप्तमशतकपूर्वार्घ इति सिद्धान्तस्वीकारमूलक इति प्रत्यक्षम् । सर्वथा चित्सुखः द्वादशशतकीय इति निश्चयः । 
चित्सुखसतीर्थ्यः-
श्वेताश्वतरोपनिषद्दीपिकायां विज्ञानात्मना ``प्रत्यस्ताखिलभेदाय जगद्विभ्रमसाक्षिणे । ज्ञानोत्तममुनीन्द्राय नमः प्रत्यक्स्वरूपिणे" इति ज्ञानोत्तमः नमस्कृतः । एवञ्च पञ्चपादिकाव्याख्यातात्पर्यद्योतिनी-श्वेताश्वतरोपनिषद्दीपिकाकारः विज्ञानात्मा चित्सुखसतीर्थ्य इति निर्णीयते ।
चित्सुखशिष्याः-
न्यायापदेशमकरन्दव्याख्यायां, अधिकरणरत्नमालायाम्, तत्वप्रदीपिका व्याख्यायाञ्च भावद्योतनिकायाम्, सुखप्रकाशेन चित्सुखः नमस्कृतः । एवमानन्दगिरेर्दीक्षागुरुः शुद्धानन्दश्च चित्सुखशिष्य इति श्रीकण्ठशास्त्रिभिः प्रतिपादितम् । तस्मात् सुखप्रकाशशुद्धानन्दौ चित्सुखशिष्याविति निर्णीयते ।
चित्सुखप्रशिष्याः-
``सुखप्रकाशयतिनं तं नौमि विद्यागुरुम्" इत्यमलानन्दैरुक्तत्वात् कल्पतरुकार अमलानन्दः सुखप्रकाशशिष्यः चित्सुखाचार्यप्रशिष्यश्च । एवं ``शुद्धानन्दपदाम्भोजभृङ्गायितमना" इत्यादिना आनन्दगिरिणा वर्ण्यमानश्शुद्धानन्दोऽपि आनन्दगिरिदीक्षागुरुरिति निर्णीयते । तस्मात् आनन्दगिरिरपि चित्सुखाचार्य प्रशिष्यो भवितुमर्हति ।
अत्रायं शिक्षादीक्षागुरुशिष्यपरम्परावृक्षः-
%%% Chart 
चित्सुखग्रन्थाः-
१. अधिकरणमञ्जरी (J. O. R. Vol V)
``रामग्रहेन्दुसंख्याता न्यायाश्शारीरकाश्रया" इत्यादिना शङ्करभगवत्पाद भाष्यानुसारं अधिकरणार्थं अधिकरणसंख्याश्च प्रतिपादव्यन्नयं ग्रन्थः प्राच्यभाषाशोधन पत्रिकायां (J. O. R. Vol V) मद्रासनगरे मुद्रितः ।
२. अधिकरणसङ्गतिः (J. O. R. Vol VII)
भाष्योपवर्णिताधिकरणानां सङ्गतिप्रदर्शनपरोऽयं ग्रन्थकः (J. O. R. Vol VII) मुद्रितः । अमुद्रितस्तु (R. 3305 MGOML) लभ्यते ।
३. अभिप्रायप्रकाशिका (R. 3853 MGOML)
मण्डनमिश्रीयब्रह्मसिद्धि व्याख्यात्मकोऽयं ग्रन्थ अनादिः सान्तश्च मद्रासहस्तलिखितपुस्तकालये लभ्यते ।
४. खण्डनभावप्रकाशिका-
श्रीहर्षीयखण्डव्याख्यात्मकोऽयं ग्रन्थः अंशतः चौखाम्बामुद्रणालये मुद्रितः ॥
५. तत्वप्रदीपिका (N. S. P.)
समन्वय-अविरोधसाधनफलाख्यैश्चतुर्भिः परिच्छेदैः पूर्णोऽयं ग्रन्थः न केवलं अद्वैतसिद्धान्तरक्षकः परं अद्वैतप्रकाशकः व्युत्पादकश्च । निर्णयसागरमुद्रणालये मुद्रितश्च । नयनप्रसादिनी, भावद्योतनिकेति च द्वे व्याख्ये विद्येते ।
६. न्यायमकरन्दटीका (B. S. S. 38, 62, 87, 117)
आनन्दबोधकृतन्यायमकरन्दव्याख्यात्मकोऽयं ग्रन्थः चौखाम्बामुद्रणालये मुद्रितः । ग्रन्थेऽस्मिन् आनन्दवोधं ब्रह्मप्रकाशिकाकारखण्डनपरं चित्सुखः (Page 346) वर्णयति ।
७. प्रमाणरत्नमालाव्याख्या (R. 3273 MGOML)
तात्पर्यटीका, सम्बन्धोक्तिः निबन्धनमित्यादय नामविशेषाः श्रूयन्ते । आनन्दबोधीय प्रमाणरत्नमालाव्याख्यात्मकोऽयं ग्रन्थ अमुद्रितः मद्रासपुस्तकालये अनन्तशयनपुस्तकालये सरस्वतीमहालये च लभ्यते । ब्रह्मविद्यापत्रिकायां मुद्रितश्च ।
८. भावतत्वप्रकाशिका (R. 3271 MGOML)
नैष्कर्म्यसिद्धिव्याख्यात्मकोऽयं ग्रन्थः अमुद्रितः मद्रासराजकीयपुस्तकालये अडयारपुस्तकालये च लभ्यते । व्याख्यैषा ज्ञानोत्तमकृतां चन्द्रिकामनुकरोति ।
९. भावद्योतनिका (R. 4305 MGOML)
तात्पर्यदीपिकापरनामायं ग्रन्थः पञ्चपादिकाविवरणव्याख्यात्मकः मद्रास राजकीयपुस्तकालये लभ्यते । अचिरादेव प्रकाशमेष्यति ।
१०. भाष्यभावप्रकाशिका (R. 3020, R. 5140, MGOML)
अध्यासभाष्यान्तोऽयं सूत्रभाष्यव्याख्यात्मकः ग्रन्थः मद्रास अडयार मैसूर पुस्तकालयेषु लभ्यते ।
११. वेदान्तसिद्धान्तकारिकामञ्जरी (R. 1492 MGOML) प्रकरणग्रऽन्थोऽयं मद्रासपुस्तकालये लभ्यते ।
१२. विष्णुपुराणव्याख्या (5511 DC. TSML)
१३. ब्रह्मस्तुतिः
१४. सद्दर्शनसंग्रहः --- अनयोश्चित्सुखकर्तृत्वे प्रमाणं न ज्ञायते ।
४०. अमलानन्दः (1247-1347 A. D.)
``आनन्दात्मयतीश्वरं तमनिशं वन्दे गुरूणां गुरुम्" इति आनन्दात्मानं प्रणमन्नयं (632 TMPL) व्यासाश्रम इत्यपरनामा अमलानन्द आनन्दात्मप्रशिष्य इति, ``यथार्थानुभवानन्दपदगीतं गुरुं नुम" इति अनुभवानन्दं प्रणमन्नयं अनुभवानन्दशिष्य इति च ज्ञायते । ``सुखप्रकाशशशिनं तं नौमि विद्यागुरुं" इति सुखप्रकाशं विद्यागुरुत्वेन वर्णयन् अयं सुखप्रकाशात् प्राप्तविद्य इति च ज्ञायते । तस्मात् दीक्षागुरुः अनुभवानन्दः, दीक्षाप्रगुरुरानन्दात्मा, विद्याप्रगुरुश्चित्सुख विद्यागुरुस्सुखप्रकाश इति सिध्यति । एतेन अमलानन्दगुरुरयं अनुभवानन्दः कोशरत्नप्रकाशप्रभामण्डलकारात् अनुभवानन्दात् भिन्नश्शङ्करानन्दसामयिक इति, अमलानन्दश्च आनन्दात्मप्रशिष्यशङ्करानन्दशिष्यविद्यारण्यसामयिक इति च सिध्यति ।
``श्रीजैत्रदेवात्मजे कृष्णे क्ष्माभृति भूतलं सह महादेवेन सम्बिभ्रति" इति देवगिरिक्ष्मापतिकृष्णस्य तद्भ्रातुर्महादेवस्य सामयिकत्वदर्शनात् क्रैस्तवीयत्रयोदशशतकापरकालिक इति निश्चियते । महाराष्ट्रदेशज इति केचित् ।
%%% Chart
१. कल्पतरुः- (N. S. P.)
भामतीव्याख्यात्मकोऽयं ग्रन्थः प्रकटार्थकारं खण्डयन् भामतीपक्षमक्षुण्णं साधयति । निर्णयसागरे मुद्रितः । अस्य व्याख्या अप्पय्यदीक्षितकृता परिमलाख्या लक्ष्मीनृसिम्हीया आभोगाख्या, वैद्यनाथकृता कल्पतरुमन्दारमञ्जरी च विद्यते ।
२. शास्त्रदर्पणम् - (VVSS 7)
अधिकरणरचनासु भामतीमनुसरन् सूत्रवृत्तिग्रन्थोऽयं भामत्यर्थान् संगृह्नाति । वाणीविलासे मुद्रितः ।
३. पञ्चपादिकादर्पणम् - पञ्चपादिकाव्याख्यात्मकोऽयं ग्रन्थः कृत इति सम्प्रदायसमागता कथा श्रूयते ।
आनन्दगिरिः (1260-1320 A. D.)
आनन्दज्ञानापराभिधस्यास्य आनन्दगिरेस्सन्यासाश्रमस्वीकारात्पूर्वं जनार्दन इति नाम । गुजरातप्रान्तजोऽयं सन्यासस्वीकारादनन्तरं द्वारकाशाङ्करपीठाधीश आसीत् । केचिदेनमान्ध्रदेशजं वदन्ति । अन्ये तु चेरदेशजं वदन्ति ।
``तं वन्देऽनुभवस्वरूपयमिनं स्मृत्याखिलाभीष्टदम्" इति तत्वालोके दर्शनात् सारस्वतप्रक्रियाइष्टसिद्धिविवरणप्रकटार्थविवरणादिकार अनुभूतिस्वरूपाचार्यः विद्यागुरुरिति ``शुद्धानन्दपदाम्भोज भृङ्गायितमना जगौ" इति बहुत्र दर्शनात् शुद्धानन्दः दीक्षागुरुरिति निर्णयः ।
सारस्वतप्रक्रियाव्याख्यातुर्नरेन्द्रनगरीति प्रसिद्धस्य सामयिकोऽयम् । तत्वदीपनकार अखण्डानन्दसरस्वती, तत्वालोकव्याख्याता तत्वप्रकाशिकाकारः प्रज्ञानानन्दश्च आनन्दगिरिशिष्येषु प्रसिद्धतरौ । एकादशीनिर्णयव्याख्याता अच्युतानन्दोऽपि शिष्योऽयमस्येति ज्ञायते । अदसीयायां पदार्थतत्वनिर्णयव्याख्यातर्क विवेकाख्यायां ``कलिङ्गदेशाधिपतौ नरेन्द्रे भुवं प्रशासत्यमरेन्द्रतुल्ये । नृसिम्हदेवे जगदेकवीरे नरोत्तमेऽकारि मया प्रबन्धः" । इति दर्शनात् कलिङ्गदेशाधिप नृसिम्हदेवस्य सामयिकश्चेति ज्ञायते ।
%%% Chart
आनन्दगिरिशिष्येण अद्वयगिरिणा प्रपञ्चभारव्याख्या विज्ञानचन्द्रिकानाम्नी कृता या अडयारपुस्तकालये लभ्यते ।
आनन्दगिरिणा प्रायस्सर्वाण्यपि शाङ्करभाष्याणि व्याख्यातानि । परन्तु ऐतरेय-प्रश्नोपनिषदां भाष्यस्य या व्याख्या आनन्दगिरिकृतेति मुद्रिता सा वस्तुतः नानन्दगिरिकृता । यतः-ऐतरेयप्रश्नोपनिषद्वयाव्याख्या अन्ते विद्यारण्यीया दीपिका निर्दिष्टा । विद्यारण्यश्चानन्दगिरेरर्वाचीन इति तु निश्चयः । तस्मात् न प्रसिद्ध आनन्दगिरिरस्य कर्ता । श्रीकण्ठशास्त्रिभिः (I. H. Q. Vol. XIV) अनन्तानन्दगिरिति कश्चन लक्ष्मीघरगुरुरूपवर्णितः । तेन कृता स्यादिति सशयः ।
१. आत्माज्ञनोपदेशविधिटीका - (R. 3380 H. MGOML) अमुद्रितोऽयं ग्रन्थः मद्रासराजकीयपुस्तकालये पञ्चाब सृच्याञ्चलक्ष्यते ।
२. ईशावास्यशाङ्करभाष्यटिप्पणी - (ASS 5)
३. उपदेशसाहस्री व्याख्या - (7207 TSML)
अमुद्रितोऽयं ग्रन्थः सरस्वतीमहालये, मद्रासपुस्तकालये (R 380 b.-MGOML) अडयारपुस्तकालये (33. M. 25. AL) लन्दनपुस्तकालये (2279 IOL) च लभ्यते ।
४. उपसदनव्याख्या - (R 3380 E. MGOML)
५. काठकोपनिषद्भाष्यव्याख्या - (ASS 7)
६. केनोपनिषद्भाष्यव्याख्या - (ASS 6)
७. गोविन्दाष्टकटीका (XXII 9) नासिकसूच्यां दृश्यते ।
८. छान्दोग्यभाष्यव्याख्या - (ASS 14)
९. तर्कविवेकः (पदार्थतत्वनिर्णयव्याख्या) - (R. 4342 MGOML)
आनन्दानुभवीयपदार्थतत्वनिर्णयव्याख्यात्मकोऽयं ग्रन्थ अमुद्रितः मद्रास पुस्तकालये अनन्तशयनसूच्यां 302 (T. C. D.) दृश्यते ।
१०. तत्वालोकः - (1105 DCRAS Bombay Mss. 229 27287 DC Benaras Sanskrit Viswa Vidyalaya Vol. VII Page No. 56.)
न्यायवैशेषिकप्रतिपादितं सिद्धान्तं दूषयन्नयं ग्रन्थः भास्कराचार्यनिम्बार्काचार्यौ निर्दिशन् प्रकटार्थकारमार्गानुसारी भागद्वयपरिमितः रायल आसियाटिक सोसाइटि बाम्बे नगरे, अपूर्णः अडयारपुस्तकालये च लभ्यते । अस्य व्याख्या प्रज्ञानानन्दकृता तत्वप्रकाशिकाख्या वर्तते ।
११. तर्कसङ्ग्रहः - (G. O. S. 3)
१२. त्रिपुटीविवरणम् - (R 3380 d. MGOML)
१३. तैत्तरीयभाष्यव्याख्या - (ASS 13)
१४. पञ्चप्रकरणव्याख्या ``पञ्चप्रक्रियाव्याख्या" - (MUSBS 4)
ग्रन्थोऽयं सर्वज्ञात्ममुनिकृतपञ्चप्रक्रियाव्याख्यात्मकः मद्रासविश्वविद्यालय बुल्लट्टिनग्रन्थमालायां प्रकाशितः ।
१५. तैत्तरीयवार्तिकटीका - (ASS 12)
१६. पञ्चीकरणविवरणम् -
रामतीर्थकृततत्वचन्द्रिकाव्याख्यासहितोऽयं चौखाम्बामुद्रणालये मुद्रितः । अमुद्रितस्तु मद्रास-अनन्तशनादिषु लभ्यते च ।
१७. बृहदारण्यकभाष्यव्याख्या - (ASS 15)
१८. बृहदारण्यकवार्तिकव्याख्या - (ASS 16)
१९. भगवद्गीताभाष्यविवेचनम् - (ASS 34)
२०. माण्डूक्योपनिषद्भाष्यव्याख्या - (ASS 10)
२१. माण्डूक्यकारिकाव्याख्या - (ASS 10)
२२. मुण्डकभाष्यव्याख्या - (ASS 9)
२३. वाक्यवृत्तिव्याख्या - (R 3380 A. R. 3324 b. MGOML)
२४. वेदान्तविवेकः (न्यायदीपावलीव्याख्या) - (R 4439 b. R 3604, MGOML) अनन्तशयनपुस्तकालयेऽपि (365 TCD) लभ्यते ।
२५. शारीरकन्याययनिर्णयः (सूत्रभाष्यव्याख्या) - N.S.P
२६. स्वरूपविवरणम् (R. 3380 C MGOML, XXI 37/8) जयपूरसृची ।
आनन्दगिरिकृतत्वे सन्दिग्धाः ग्रन्थाः
२७. एेतरेयभाष्यव्याख्या - (ASS 11) २८. गुरुस्तुतिः । २९. चूलिकोपनिषद्व्याख्या ३०. प्रश्नोपनिषद्भाष्यव्याख्या - (ASS 8) ३१. बृहच्छङ्करविजयः । ३२. मितभाषिणी । ३३. शतश्लोकीव्याख्या - (Mysore Edn BSS 20) ३४. शङ्कराचार्यावतारकथा । ३५. हरिमीडेस्तुतिव्याख्या (76 d.) शृङ्गरिरिसृच्यां, (254) नासिक्सूच्याञ्च दृश्यते ।
४२. अखण्डानन्दमुनिः (1250-1350 A. D.)
अयमखण्डानन्दमुनिरखण्डानुभूत्यान्दगिर्योश्शिष्यः । अखण्डानुभूतिर्दीक्षागुरुः । आनन्दज्ञानः विद्यागुरुः । ग्रन्थस्यारम्भे चतुर्थवर्णकावसाने च उभयोरपि नामनिर्देशः कृतः । आनन्दगिरिर्बोधपृथ्वीघरशब्देन निर्दिष्टः । आन्दगिरिजन्मस्थानं रत्नाचल इति च निर्दिश्यते । अयमौत्तरः विष्णुभट्टोपाध्यायसतीर्थ्यश्चेति निश्चयः ।
म. म. अनन्तकृष्णशास्त्रिणस्स्वसम्पादिते ब्रह्मसूत्रभाष्यग्रन्थे तत्वदीपनकारस्याखण्डानन्दमुनेः भामतीव्याख्याऋजुप्रकाशिकाकर्तुः अखण्डानन्दानुभूतेरैक्यं सम्भावयन्ति । ऋजुप्रकाशिकाकर्तुः गुरोः स्वयम्प्रकाशता गौणीति च सम्भावयन्ति ।
परन्तु इग्मिडिजगदेकरायकालिकस्य षोडश-सप्तदश-शतकीयस्य ऋजुप्रकाशिकाकर्तुरखण्डानन्दसरस्वत्याः तत्वदीपनकारात् त्रयोदशचतुर्दशशकीयात् आनन्दगिरिशिष्यात् अखण्डानन्दमुनेर्भिन्नता प्रतीयते । केशवमिश्रकृतायाः तर्कभाषाया व्याख्या गोवर्धनेन कृता । गोवर्धनकृताया व्याख्यायाः व्याख्या ऋजुप्रकाशिकाकारेण अखण्डानन्देन रचिता । गोवर्धनकालस्तु (1560 A. D.) इति प्रसिद्धम् । तस्मात् ऋजुप्रकाशिकाकारस्तत्वदीपनकारात् भिन्न एव । तत्वानु सन्धानकारस्य महादेवसरस्वत्यास्सतीर्त्योऽयं ऋजुप्रकाशिकाकारः । महादेवसरस्वती तु षोडशशतकापरार्धादारब्धे समये आसीत् । तस्माच्चायं भिद्यते ऋजुप्रकाशिकाकारात् । ऋजुप्रकाशिकाकर्त्रा नृसिम्हाश्रमीयस्य अद्वैरत्नकोशस्य व्याख्या रत्नकोशप्रकाशिकानाम्नी मैसूरपुस्तकालये कृतेति दृश्यते । नृसिम्हाश्रमश्च (1550 A. D.) काले आसीदिति प्रसिद्धम् । तस्मादपि कारणात् ऋजुप्रकाशिकाकाराद्भिन्नोऽयं तत्वदीपनकार इति निश्चीयते ।
१. तत्वदीपनम् (C. S. S. 1)
पञ्चपादिकाविवरणव्याख्यारूपोऽयं चतुस्सूत्रीभागपूर्णः ग्रन्थः कल्कत्तासंस्कृतमालायां विजयनगरग्रन्थमालायां वाराणसीग्रन्थमालायाञ्च मुद्रितः । विनायक कृतस्तत्वदीपनसारोऽपि बरोडासूच्यां दृश्यते ।
४३. आनन्दपूर्णः (1275-1350 A. D.)
विद्यासागरापराभिधोऽयं आनन्दपूर्ण अभयानन्दशिष्य इति एतदीयकृतिषु अभयानन्दस्तुतेर्ज्ञायते । बृहदाण्यकोपनिषद्भाष्यवार्तिकव्याख्यायां न्यायकल्पलति कायां ``श्रीमते गुरवे श्वेतगिरये स्यात् नमक्रिया" इति दर्शनात् श्वेतगिरिर्दीक्षा गुरुरभयानन्दो विद्यागुरुरिति च निश्चियते । आशुतोषग्रन्थमालाप्रथमपुष्पात् कृष्णानन्दसरस्वतीकृतात् शारीरकमीमांसाभाष्यवार्तिकाख्यात् ग्रन्थात् `यदि रत्नप्रभाब्रह्मामृतवर्षिणीकारः गोविन्दानन्दः तर्हि गोविन्दानन्दस्य परमेष्ठिगुरुरानन्दपूर्णोऽयं इति, यदि रामानन्दः रत्नप्रभाकारस्तर्हि रामानन्दपरमेष्ठिगुरोर्गुरुरानन्दपूर्णोऽयमिति' निर्णीयते । आनन्दपूर्णस्य शिष्यः पुरुषोत्तमानन्दसरस्वतीति च ।
एतदीयायां न्यायचन्द्रिकायां न्यायकल्पलतिकायाञ्च गोकर्णक्षेत्रनिर्देशात् गोकर्णक्षेत्रवासीति निश्चयः । न्यायचन्द्रिकाव्याख्यान्यायप्रकाशिकायां स्वरूपानन्दमुनिकृतायां ``येन व्याकरणाटवी स्कुटतरं संप्लाविता लीलया" इति दर्शनात् व्याकरणेऽपि आनन्दपूर्णः प्रकाण्डपण्डित इति ज्ञायते । अमुद्रितस्य न्यायकल्पलतिकाग्रन्थस्य प्रतिलेखनकालः (1499 सं  1443 A. D.) इति (R. 5283 MGOML) ग्रन्थात् ज्ञायते । एवञ्च ग्रन्थकर्ता (1443 A. D.) कालात्पूर्वतन इति तु निश्चयः । राघवमहोदयस्तु वीये प्रबन्धे (A. O. R. Madras Vol. IV Part I) कामदेवभूपालकालिकोऽयं विद्यासागरः (1350 A. D.) इति प्रतिपादयति । दासगुप्तस्तु (1600 A. D.) कालिकं वदति । टेलाङ् महाशयस्तु महाविद्याविडम्बनभू मिकायां (1529-1600 A. D.) सामयिकं प्रवदति ॥
१. खण्डनफक्किकाविभजनम् (B.S.S 81, 109)
खण्डनखण्डखाद्यव्याख्यात्मकोऽयं ग्रन्थः विद्यासागर्यपराभिधः चैखाम्बामुद्रणालये मुद्रितः ।
२. टीकारत्नम् (R. 3403 M.G.O.M.L) पञ्चपादिकाविवरणव्याख्यात्मकोऽयं ग्रन्थ मद्रासनगरे लभ्यते ॥
३. न्यायचन्द्रिका (R. 2931 M.G.O.M.L, 298-299 DCTCL)
अमुद्रितोऽयं परिच्छेदचतुष्टयात्मकस्सिद्धान्तलेशसंग्रहे उद्धृतः । ग्रन्थोऽयं न्यायवैशेषिकमीमांसादिमतं खण्डयन् प्रमाणानां लक्षणं प्रतिपादपयन् प्रकरणग्रन्थता मर्हति । अस्य व्याख्या न्यायप्रकाशिकाख्या स्वरूपानन्दमुनिकृता अनन्तशयने लभ्यते ॥
४. न्यायकल्पलतिका (R. 5283 MGOML)
बृहदारण्यकभाष्यवार्तिकव्याख्यात्मकोऽयं ग्रन्थः । अचिरादेव तिरुपति केन्द्रीयविद्यालयात् प्रकाशतामेष्यति ।
५. बृहदारण्यकव्याख्या
ग्रन्थोऽयं कुत्रत्य इति न ज्ञायते । परन्तु प्रबोधचन्द्रोदयव्याख्यायाः नन्दिल्लीगोपप्रणीतायाः चन्द्रिकाख्यायाः Page 204 (N.S.P. Edn.) ``बृहदारण्यक भाष्ये मधुकाण्डे विद्यासागरीतोऽवगन्तव्यमिति ग्रन्थात् ज्ञायते ॥"
६. पञ्चपादिकाव्याख्या (2261 DC IOL Vol. IV)
पञ्चपादिकाविवरणानुसारी भाववर्णनपरः स्वतन्त्रव्याख्यारूप इति ज्ञायते - ग्रन्थोऽयं लन्दनपुस्तकालये लभ्यते ॥
७. भावशुद्धिः `ब्रह्मसिद्धिव्याख्या' (R 3967 MGOML) ८. समन्वयसूत्रवृत्तिः ९. न्यायसारव्याख्या (व्याख्यारत्नम्) १०. पुरुषार्थबोधः ११. प्रक्रियामञ्जरी (काशिकावृत्तिव्याख्या) १२. मोक्षधर्मव्याख्या १३. महाविद्याविडम्बनव्याख्या ॥ एते ग्रन्था अपि विरचिताः केचन मुद्रिता अपरेऽमुद्रिताश्च ॥
४४. शङ्करानन्दः (1275-1350 A. D.)
एतद्रचितायां नृसिम्हतापनीयदीपिकायां कृता मया वेदचतुष्टयेऽपि प्रसिद्धशाखोपनिषत्पदेषु । व्याख्या तया तुष्यतु सर्वजीवः आनन्दआत्माद्वय ईश्वरोऽयम् । इति, एवमितरग्रन्थेषु चानन्दात्मा नमस्कृतः । एवञ्चायं आनन्दात्मशिष्यः । विवरणप्रमेयसंग्रहे पञ्चदश्यादौ च विद्यारण्येन शङ्करानन्दः नमस्क्रियते । एवाञ्चायं विद्यारण्यगुरुः । एतदीये आत्मपुराणे ``इदं गुरुस्स्वशिष्याय कावेरीतीरवासिने उक्तवान् श्रद्धधानाय स्नेहादेव च केवलम् ॥" इति दर्शनात् द्रविडान्वयसम्भूतस्स्यादिति प्रतिभति ।
अयं विद्यातीर्थशिष्यः विद्यारण्यगुरुश्चोलदेशीयमध्यार्जुनग्रामाभिजनश्चेति सम्प्रदायसमागता जनश्रुतिः । विद्यातीर्थस्यैव आनन्दात्मा इति नामान्तरं स्यात् । अथवा आनन्दात्मा विद्युगुरुः, विद्यातीर्थः दीक्षागुरुरिति स्यात् । सूर्यनारायणशास्त्री तु विद्यातीर्थस्यैव विद्याशङ्करतीर्थः शङ्करानन्द इति नामान्तरं इति वदति । दासगुप्तस्तु विद्यारण्यामलानन्दयोर्गुरुश्शङ्करानन्द इति वदति । श्रुङ्गगिरिगुरुपरम्परायां नृसिम्हतीर्थशिष्यः विद्यातीर्थ इति दृश्यते । यदि विद्यातीर्थ एव शङ्करानन्दस्स्यात्तार्हि नृसिम्हतीर्थस्यैव आनन्दात्मा इति नाम भाव्यम् । अथवा आनन्दात्मा विद्यागुरुः, नृसिम्हतीर्थः दीक्षागुरुरिति वक्तव्यं भवति । ``आनन्दात्मयतीश्वरं तमनिशं वन्दे गुरूणां गुरुम्" इति दर्शनात् आनन्दात्मनः प्रशिष्य अमलानन्द इति तु निश्चयः । एवं ``यथार्थानुभवानन्दपदगीतं गुरुं नमः" इति दर्शनात् अनुभवानन्दः गुरुरिति ज्ञायते । ``सुखप्रकाशयतिनं तं नौमि विद्यागुरु"मिति दर्शनात् सुखप्रकाशः विद्यागुरुरिति च निश्चयः । तस्मात् किमयं अमलानन्दः अनुभवानन्दशिष्य उत शङ्करानन्दशिष्य इति संशय उदेति । सर्वथा सामयिक इति तु निश्चयः ।
शङ्करानन्दस्य शिष्यः सदानन्दप्रथमः । सदानन्दप्रथमस्य शिष्य अद्वैतानन्दसरस्वती । अद्वैतानन्दसरस्वत्याश्शिष्यः सदानन्दद्वितीय इति A. O. R. पत्रिकायां (Vol VI Part I) प्रतिपादितम् ।
आनन्दात्मविद्यातीर्थयोश्शिष्यः, भारतीकृष्णतीर्थविद्यारण्ययोर्गुरुः मध्यार्जुनक्षेत्र (तिरुविशरनल्ल्) वासी, सन्यासस्वीकारतत्पूर्वं अस्य पितरौ वेङ्कटेशसुब्बाम्बा-वाञ्छेश्वरौ इति च ज्ञायते ।
%%% Chart
१. अथर्वशिर उपनिषद्दीपिका
२. अथर्वशिखा उपनिषद्दीपिका
३. अमृतनादोपनिषद्दीपिका
४. अमृतबिन्दूपनिषद्दीपिका
५. आत्मपुराणम् (उपनिषद्रत्नम्) ग्रन्थोऽयं गोपालनारायणमुद्रणालये वाम्बेनहरे मुद्रितः । अस्य व्याख्या ``रामकृष्णकृता सत्प्रसवाख्या" ।
६. आरुणिंकोपनिषद्दीपिका
७. ईशावास्यदीपिका
८. ऐतरेयदीपिका 
९. कठोपनिषद्दीपिका
१०. केनदीपिका
११. कैवल्यदीपिका
१२. कौषीतकीदीपिका
१३. क्षुरिकदीपिका
१४. गर्भदीपिका
१५. छान्दोग्यदीपिका
१६. जाबालदीपिका
१७. तैत्तरीयदीपिका
१८. नारायणोपनिषद्दीपिका
१९. नृसिम्हतापनीदीपिका
२०. पञ्चक्रोशयात्रामञ्जरी
२१. परमहंसोपनिषद्दीपिका
२२. प्रश्नोषनिषद्दीपिका
२३. बृहदारण्यकदीपिका
२४. ब्रह्मविद्योपनिषद्दीपिका
२५. ब्रह्मसूत्रदीपिका (B. S. S. 91)
२६. ब्रह्मसूत्रवृत्तिः (A. S. S. 67) तात्पर्यबोधिन्यपरनामायं ग्रन्थः ।
२७. ब्रह्मोपनिषद्दीपिका
२८. भगवद्गीताव्याख्या (तात्पर्यबोधीनी) (N. S. P.)
२९. महोपनिषद्दीपिका
३०. माण्डूक्योपनिषद्दीपिका
३१. मुण्डकोपनिषद्दीपिका
३२. शिवगीताव्याख्या
३३. श्वेताश्वतरदीपिका
३४. श्रुतिगीताव्याख्या 
३५. हंसोपनिषद्दीपिका । आसु दीपिकासु काश्चन आनन्दाश्रममुद्रणालये ``उपनिषदां समुच्चये" मुद्रिता । अमुद्रितास्सर्वा अपि शृङ्गगिरिमठपुस्तकालये मद्रासराजकीयहस्तलिखितपुस्तकालये च लभ्यन्ते ।
४५. भारतीतीर्थः (1280-1350 A. D.)
``प्रणम्य परमात्मानं श्रीविद्यातीर्थरूपिणं । वैय्यासिकन्यायमाला श्लोकैस्संगृह्यते स्फुटम्" इति वैय्यासिकन्यायमालायां दर्शनात् भारतीतीर्थगुरुर्विद्यातीर्थ इति ज्ञायते । ``स्वमात्रयानन्दयदत्र जन्तून् सर्वात्मभावेन तथा परत्र । यच्छङ्करानन्दपदं हृदब्जे विभ्राजते तद्यतयो विशन्ति" इति विवरणप्रमेयसग्रहे दर्शनात् (1802 A. C. O. Lm 352 A. T. C. D.) इमे विद्यातीर्था एव विद्याशङ्करतीर्थ इति शङ्करानन्द इति वा प्रसिद्धा (1280 A. D.) काले शृगगिरिपीठाधीशा आसन् । विद्यातीर्थेन च रुद्रप्रश्नस्य भाष्यं कृतम् । तस्मात् विद्यातीर्थापराभिधशङ्करानन्दस्य शिष्योऽयं भारतीतीर्थ इति निश्चीयते ।
भारतीतीर्थशिष्याः-
विद्यारण्यस्यैव आश्रमस्वीकारात् पूर्वं माधवाचार्य इति नाम माधवाचार्येण ``कालनिर्णये" सोऽहं प्राप्य विवेकतीर्थपदवीमाम्नायतीर्थे परं मज्जन् सज्जन सङ्गतीर्थनिपुणस्सद्वृत्तितीर्थं श्रयन् । लब्धामाकलयन् प्रभावलहरीं श्रीभारतीतीर्थतो विद्यातीर्थमुपाश्रयन् हृदि भजे श्रीकण्ठमव्याहतम् । इति भारतीतीर्थः नमस्कृतः । एवञ्च विद्यारण्यगुरुषु भारतीतीर्थोऽपि अन्य इति सिध्यति । ब्रह्मानन्दभारत्याख्य अपरश्शिष्योऽपि भारतीतीर्थस्य । ब्रह्मानन्दभारत्या दृग्दृश्यविवेकस्य व्याख्या कृता । महावाक्यदर्पणकारः कृष्णानन्दभारती च भारतीतीर्थशिष्यः ।
भारतीतीर्थोऽयं विद्यातीर्थादनन्तरं (1333 A. D.) काले शृङ्गगिरिमठपीठाधिप आसीदिति ज्ञायते । बहूनां ग्रन्थानां भारतीतीर्थस्स्वतन्त्रः कर्ता । कतिपयानां पञ्चदश्यादीनां ग्रन्थानां विद्यारण्येन सम्मिलितः कर्तेति च ज्ञायते ।
१. दृग्दृश्यविवेकः - (B.S.S. 56)
वाक्यसुधापरनामायं ग्रन्थः नाम्नोऽनुरूपं दृश-आत्मनः, दृश्यस्य-जगतश्च मार्मिकं विवेचनं दृष्ट्टदृश्ययोस्सम्बन्धञ्च वर्णयति । यद्यपि ग्रन्थमेन शङ्कराचार्यकृतमिति केचिद्वदन्ति । तथैव चौखाग्बायां मुद्रितश्च तथापि ब्रह्मानन्दभारतीकृतायां व्याख्यायां भारतीतीर्थकृत इत्येव दृश्यते । ग्रन्थोऽयं मुद्रितः वाराणसी ग्रन्थमालायां रत्नपिटकग्रन्थमालायाञ्च ।
२. वैय्यासिकन्यायमाला (अधिकरणरत्नमाला) (ASS. 23)
ब्रह्मसूत्रेषु शङ्करोभिमतानां अधिकरणानां वर्णनपरोऽयं ग्रन्थः द्वाभ्यां श्लोकाभ्यां पूर्वपक्षसिद्धान्तौ प्रदर्शयन् प्रत्यधिकरणं श्लोकद्वयमिति सरणिमनुसरति । अस्य व्याख्यापि न्यायमालाविस्तराख्या अनेनैव कृता । मुद्रितश्चायं आनन्दाश्रमे । 
३. पञ्चकोशविवेकः - (7177 TSML)
४. उपनिषत्संक्षेपवार्तिकम् । ग्रन्थोऽयं वाक्यसुधाटीकायां निर्दिश्यते ।
५. माण्डूक्योपनिषदीपिका ?
४६. विद्यारण्यः (1296-1386 A. D.)
भारद्वाजगोत्रजोऽयं विद्यारण्यः माधवाचार्यापरनामा सङ्गमराजमहामन्त्रिणः मायणस्य श्रीमत्याश्च पुत्रः, सायणभोगनाथयोस्सिङ्गलायाश्च भ्राता, अद्वैतमकरन्दकारस्य लक्ष्मीधरस्य मातुलः, विजयनगराधीशबुक्कणक्ष्मापतिसामयिकः, विद्यातीर्थ भारतीतीर्थ - श्रीकण्ठाचार्य शङ्करानन्दानां शिष्यः, विद्यातीर्थनृसिम्हतीर्थप्रशिष्यः, कृष्णानन्दभारती-ब्रह्मानन्दभारती-रामकृष्णानां गुरुः, विजयनगरवासी, कर्णाटकब्राह्मणः, विजयनरसाम्राज्यस्थापनधुरीण इति ज्ञायते ।
विद्यरण्यस्यैव सन्यासग्रहणात्पूर्वं माधवाचार्य इति नाम । विद्यारण्यः (1380-1386 A. D.) कालपर्यन्तं शृङ्गगिरिपीठाधीश आसीदिति शृङ्गगिरीगुरुपरम्पराया ज्ञायते । `परित्यक्ता देवी विविधविधिसेवाकुलतया मया पञ्चाशीतेरधिकमुपनीते तु वयसि' इति देव्यपराधक्षमापनस्तोत्रे दर्शनात् पञ्चाशीत्यधिकवर्षजीवी विजयनगर निर्माणे बद्धश्रद्धः राजकीयकार्यनिर्वहणे प्रगल्भश्चेति ज्ञायते । आश्रमस्वीकारात् पूर्वं माधवाचार्यापरनाम्ना अनेन क्रूरदारिद्रयपीडितेन विद्यारण्येन स्वीयदारिद्रयविमोचनाय सम्पत्करीं महालक्ष्मीं प्रति महत्तप अतप्यत । प्रसन्ना देवी आविभूय एवमुवाच-त्वदीय पूर्वकर्मवशात् नाहमस्मिन् जन्मनि तुभ्यं ऋद्धिं दास्यामि, परन्तु जन्मान्तरे नूनं दास्यामि इति । तद्वचनं श्रुत्वा सपदि स्वीकृतसंन्यासः माधवाचार्यः लक्ष्मी प्रत्युवाच प्राप्तसन्यासस्य मम इदमेव अन्यत् जन्म । तस्मात् श्रियं देहि इति । तच्छुत्वा तुष्टया देव्या दत्तं प्रभूतं धनं सन्यासिनो मम धेनन किमिति मत्वा विजयनगराभिवृद्धौ विजयनगरसाम्राज्यस्थापनायाञ्च व्ययितवानिति साम्प्रदायिकी कथापि श्रूयते ।
विजयनगरे माधवत्रयप्तासीत् । सायणात्मजो माघवः कश्चन । माधवमन्त्री द्वितीयः । विद्यारण्यापराभिधमायणपुत्रः माधवाचार्यस्तृतीयः । तेषु विद्यारण्यापराभिध एव शृङ्गगिरिपीठमध्युवास ।
वंशवृक्षः
%%% Chart
गुरुशिष्यपरम्परावृक्षः
%%% Chart
अमलानन्दसामयिकोऽयम् । विद्यारण्यामलानन्दौ शङ्करानन्दानुभवानन्दयोः शिष्यौ । आनन्दात्मनोऽपि शिष्याविति दासगुप्तः (HIP Vol II 57, 58)
१. अनुभूतिप्रकाशः N. S. P.
द्वादशोपनिषदां सारात्मकोऽयं र्विशतिभिरध्यायैः परिमितः ग्रन्थः निर्णयसागरमुद्रणालये मुद्रितः ॥
२. अपरोक्षानुभूतिदीपिका (Mysore G O. M. L. S. 20)
शाङ्करापरोक्षानुभूतिव्याख्यात्मकोऽयं ग्रन्थः मैसूर बिब्लियोथिकासंस्कृतमालायां मुद्रितः ॥
३. उपनिषत्कारिकाः (विद्याप्रकाशः) ।
ग्रन्थोऽयं 549 पञ्जाव सृच्यां दृश्यते ।
४. ऐतरेयोपनिषद्दीपिका (ASS XI)
दीपिकैषा आनन्दज्ञानकृतायां ऐतरेयशाङ्करभाष्यटिप्पण्यां (Page 28) निर्दिष्टा । आनन्दज्ञानकाल (1260-1320 A. D.) तस्मात् भिन्नकालीनैषा कथं आनन्दगिरिणा निर्दिष्टा भवेत् एवञ्च ऐतरेयभाष्यटिप्पणं न प्रसिद्धानन्दज्ञानकृतिरिति कल्पनमुचितम् । शुद्धानन्दादिगुरुवन्दनादेरभावाच्च ।
५. जीवन्मुक्तिविवेकः (ASS 20)
गद्यपद्यात्मकैः पञ्चभिः प्रकरणैः पूर्णेऽस्मिन् ग्रन्थे जीवन्मुक्ति-वासनाक्षय-मनोनाश-स्वरूपसिद्धिप्रयोजनविद्वत्सन्याससम्बद्धाः विषयाः विचारिताः । विरक्तिद्वैविध्यं, न्यासद्वैविध्यं, मुक्तिद्वैविध्यं, तयोर्भेदञ्च वर्णयन्नयं ग्रन्थः जीवतो मुक्तस्य च स्वाभाविकभेदं च वर्णयन् प्रकरणग्रन्थतामर्हति । ग्रन्थोऽयं वाराणस्यां आनन्दाश्रमे च मुद्रितः । अस्य व्याख्याः - अच्युतशर्मकृता, पूर्णानन्दकृता दीपिका, सदेकानन्दकृतस्साराख्यश्च वर्तन्ते ॥
६. तैत्तरीयलघुदीपिका (R 1968 MGOML)
७. नृसिम्होत्तरतापिनीदीपिका (R 3615 MGOML)
८. पञ्चदशी (N. S. P.)
अद्वैतवेदान्तस्य परमोपकारी पञ्चदशमिः प्रकरणैः पूर्णोऽयं प्रकरणग्रन्थः सव्याख्यः निर्णयसागरमुद्रणालये मुद्रितः । ग्रन्थोऽयं विद्यारण्यभारतीतीर्थयोरुभयोरपि कृतिरिति सम्प्रदायः । पञ्चदशीव्याख्यात्रा निश्चलदासस्वामिना ``प्राथमिकादश परिच्छोदा एव विद्यारण्यनिर्मिताः" इति प्रतिपाद्यते । रामकृष्णेन तु सप्तम परिच्छेदारम्भे एव भारतीतीर्थकृत इति निर्दिश्यते । अपरे तु प्रथमपरिच्छेदषट्कं विद्याण्यकृतं, अन्यत् परिच्छेदनवकं भारतीतीर्थकृतमिति प्रतिपादयन्ति । अस्य व्याख्याः - निश्चलदासकृतः वृ्त्तप्रभाकरः, रामकृष्णकृता तात्पर्यबोधीनी (पददीपिका) रामानन्दकृता विशुद्धदृष्टिः, सदानन्दकृता व्याख्या, अच्युतशर्मकृता पृर्णानन्देन्दुकौमुदी, लिङ्गनसोमयाजिकृता कल्याणपीयूषाख्या, अज्ञातकर्तृका तत्वबोधिनी च विद्यन्ते ।
९. ब्रह्मविदाशीर्वादपद्धतिः
द्विपञ्चाशद्भिः आशीर्लिङ्प्रत्ययान्तै भूधात्वन्तैर्वाक्यैः अद्वैतसिद्धान्तान् गुर्वनुग्रहप्रणाल्या प्रतिपादयन्नयं ग्रन्थः विद्याविनोदिनीमुद्रणालये तञ्जपुरे मुद्रितः ।
१०. बृहदारण्यकटीका 127 नासिकसूच्यां दृश्यते ।
११. बृहदारण्यकवार्तिकसारः (B. S. S. 205, 208, 243, 244)
यद्यपि मुद्रितात् ग्रन्थादस्मात् विद्यारण्यः कर्तेति न प्रतीयते । तथापि व्याख्यायां ``येनोद्घृतो वार्तिकाब्धेस्सारः विबुधतुष्टिदः । अविद्यातज्जतापध्नं विद्याण्यगुरुं भजे" इति दृश्यते । अज्ञातकर्तृका वार्तिकसारव्याख्या वार्तिकसारसंग्रहापरनाम्नी (397 T. C. D.) दृश्यते । महावाक्यविवरणमपि अस्य कृतिरिति वदन्ति । परन्तु पञ्चदश्यन्तर्गतमिति भाति ।
१२. विद्यारत्नदीपिका (2992 G. V. B. S.)
१३. विवरणप्रमेयसंग्रहः (V. N. S. S. 7.)
अस्यैव विवरणोपन्यास इति नामान्तरमिति सिद्धान्तलेशसंग्रहात् (Page 68) ज्ञायते । पञ्चपादिकाविवरणार्थान् संगृह्णात्ययं ग्रन्थः । विजयनगरसंस्कृत मालायां मुद्रितः ।
१४. सर्वदर्शनसंग्रहः १५. पराशरमाधवः १६. कालमाधवः १७. शङ्करदिग्विजयकाव्याम् । ग्रन्थोऽयं न विद्याण्यकृतिः । परन्तु अभिनवकालिदासकृतिरिति सिद्धान्तः ।
१८. सङ्गीतसारः
``सङ्गीतसारं समवेक्ष्य विद्यारण्याभिधश्रीचरणप्रणीतम्" इति गोविन्ददीक्षितकृतायां सङ्गीतसुधायां द्वितीयेऽध्याये दर्शनात्, चतुर्दण्डीप्रकाशिकायां वीणाप्रकरणे निर्देशाच्च ज्ञायते ॥
४७. माधवमन्त्री (1300-1400 A. D.)
विजयनगरराजसभायां माधवाचार्यत्रयमासीत् । तेषु विद्यारण्यापरनामा माधवाचार्य एकः । सायणमाधव अपरः । तृतीयः प्रचण्डपण्डितः प्रतापी योद्धा माधवनामा मन्त्री । सर्वेषां नामसाम्यात् कस्यचित्कार्यकलाप अन्यस्मिन् आरोपितो भवति । परन्तु यफिग्राफिका कर्नाटिका (Vol 7) शिकारपुर (Page 281) एवं (Vol 8) प्रमाणात् माधवमन्त्री माधवाचार्याद्भिन्न इति निश्चीयते ।
माधवमन्त्री आङ्गिरसगोत्रोत्पन्नः । चावुण्डोऽस्य पिता । माचाम्बिका अस्यमाता । उपनिषन्मार्गप्रवर्तकाचार्य इति विरुदभूषितः । काशीविलासक्रियाशक्तिशिष्यः । शिवाद्वैततत्ववेत्ता सन्नपि शङ्कराद्वैतपक्षपातीति सूतसंहिताव्याख्यातात्पर्यदीपिकायाः ज्ञायते । अत्र प्रमाणम्  ।  ``गोत्रे योऽङ्गिरसां प्रचण्डतपसश्चावुण्डपृथ्वीसुरप्रेष्ठादुद्भवमेत्य नीतिसरसौ दत्तां धियं धैषणीम् । सूरिस्सन्नपि सर्वदा नवमनः प्रह्लाददानोचितां यद्भूयः कवितां व्यनक्ति तनुते नो कस्य तेनाद्भुतम् ॥ I ॥ ``यः कृत्वाखिलभूतमौपनिषदं दूर्वावदूकोन्मदव्यालातङ्कददुर्नयोग्रगहनोत्सादेन वत्मोंज्वलम् । ब्राह्मं धाम सुदूरमप्यविरतं प्रस्थापयन्नप्लवात् आर्यां स्वेन बुधैरुपनिषन्‌मार्गंप्रतिष्ठागुरुः" ॥ II ॥ यः साक्षाद्गिरिशावतारवपुषः काशीविलासेशितुस्सोद्भासाद्भृतया कटाक्षकलया नीतः प्रथां शाम्भवीम् । जेता शक्तिभिरीशितात्मभिरिमं चामुञ्च लोकं जवात् आजैर्षीत् कियतोपरान्तविषयान् यत्सास्तु कास्य स्तुतिः ॥ III ॥ ``तस्या (बृक्कराज) स्ति शस्तयशसो नयशौर्यमुख्यैः ख्यातो गुणैर्जगति माधव इत्यमात्यः" यो ब्रह्मजिम्हदमनाधिकृतः पवित्रं क्षत्रं च जैत्रमभयाय भुवो विभर्ति (Vol 7) ``प्रज्ञाबलेन गुरुमप्यतिसन्धदानो मन्त्री महानजनि माधवनामधेयः" । (Vol 8)
माधवमन्त्री हरिहरप्रथमस्यावरजस्य मारप्पस्यापि मन्त्री आसीदिति (1347 A. D.) वर्षीयात् शिलालेखात् ज्ञायते । मारप्प-बुक्कप्रथम-बुक्क-द्वितीयानाञ्च मन्त्री आसीदिति ज्ञायते । माधवमन्त्री न केवलं विद्वान् परन्तु शौर्यसम्पन्नः योद्धा शत्रुप्तानमर्दनकारी च । अस्य निधनकालः  1391 A. D. इति ।
माधवाचार्य-माधवमन्त्रीभेदतालिका
माधवाचार्यः			माधवमन्त्री
भारद्वाजगोत्रजः					आङ्गिरसगोत्रजः
मायणश्रीमत्योः पुत्र					चावुण्ड्यमाचाम्बिकयोः पुत्र
सायणभोगनाथभ्राता						-
विद्यातीर्थभारतीतीर्थ-					काशीविलासक्रियशक्तिशिष्यः
श्रीकण्ठशिष्यः							-
पराशरमाधवीयादिकर्ता				तात्पर्यदीपिका B. M. P.
सूतसंहिताव्याख्यात्मकोऽयमद्वैतग्रन्थः 	बालमनोरमामुद्रणालये मुद्रित्ः ॥
४८. लक्ष्मीधरः (1406-1500 A. D.)
सरस्वतीमहालयपुस्तकालयमद्रपुरीसर्वकारपुस्तकालयस्थहस्तलिखितपुस्तकप्रामाण्यात् (7641, 4269, 6930, TSML R No 1424, MGOML) लक्ष्मीधरद्वयमासीदिति ज्ञायते । चरूकूरिवश्यः काश्यपगोत्रजः ऋग्वेदी मीमांसाद्वयपारगः कविः, आश्रमस्वीकारादनन्तरं रामान्द इति प्रसिद्धः कृष्णाश्रमपादशिष्यः लक्ष्मीधरः कश्चन । अस्य भ्राता कोण्डुभट्टः षड्दशिनीविवेककारः । अस्य ज्येष्ठभ्रातृपुत्रेण यज्ञानारायणेन शास्त्रदीपिकाव्याख्या प्रभामण्डलम्, गीतगोविन्दव्याख्या श्रुतिरञ्जिनी, जयदेवीयप्रसन्नराघवव्याख्या - अभीष्टार्थदायिनी षड्भाषाचन्द्रिका च कृताः । अनेन लक्ष्मीधरेण कृता इष्टार्थकल्पवल्लीनाम्नी अनर्धराघवव्याख्या सरस्वतीमहालये लभ्यते ।
%%% Chart
अद्वैतमकरन्दकर्ता तु नरसिम्हसूरिपुत्रः अनन्तानन्दरघुनाथयतिशिष्योऽयं लक्ष्मीधरः सन्यासग्रहणादनन्तरं कृष्णेन्द्रनाम्ना प्रसिद्ध इति (7641 Mss TSML R. 1424 MGOML) ज्ञायते अस्य गुरोरनन्तानन्दरधुनाथस्य अनन्तानन्दगिरिरित्यपरं नाम । अनन्तानदगिरिरिकालः (1380 A. D.) । लक्ष्मीधरोऽयं सायणभोगनाथविद्यारण्यानां भगिन्याः सिम्हलानाम्न्याः पुत्रः । कर्णाटककविमधुरावासी चायं अनन्तानन्दगिरिशिष्यः लक्ष्मीधरः देवरायप्रथमस्य राज्यशासनकाले आसीत् । देवरायप्रथमस्य राज्यशासनकालः (1406 A. D.) इति लक्ष्मीधरोऽयं चतुर्दशशतकप्रारम्भकाले आसीदिति श्रीकण्ठशास्त्रिभिः भारतीयैतिहासिकत्रैमासिकपत्रिकायां (IHQ Vol XIV) प्रतिपादितम् ।
एतत्कृतायाः भगवन्नामकौमुद्याः आपदेवात्मजेन अनन्तदेवभट्टेन व्याख्या कृता । भट्टवंशश्च (1650 - A. D.) काले वाराणस्यामासीत् । अनन्तदेवस्य कालोऽपि (1600-1700 A. D.) इति प्रतिप्रादितम् । भारतीतीर्थकृतायाः वाक्यसुधायाः व्याख्या तच्छिष्येण ब्रह्मानन्दभारत्या कृता । तेन च 46 पुटे लक्ष्मीधरः निर्दिष्टः । ब्रह्मानन्दभारत्याः कालस्तु (1325-1425 A. D.) इति निश्चयः । तस्मादयं लक्ष्मीधरः (1406 A. D.) काल एवासीदिति निश्चीयते । चतुर्दशशतकान्तादारब्धे पञ्चदशशतकावसाने काले आसीदिति तु निश्चयः ।
१. अद्वैतमकरन्दः (V. V. P.)
प्रकरणग्रन्थोऽयं वाणीविलासमुद्रणालये श्रीरङ्गनगरे भुद्रितः । अस्य व्याख्याः-मूलकारेण कृता व्याख्या, स्वयम्प्रकाशयतिकृता । रसाभिव्यञ्जिका, पूर्णानन्दतीर्थकृता-व्याख्या 328 नार्तवेस्टप्रान्तसूच्यां दृश्यते । वंशीधरकृतव्याख्याग्रन्थोऽयं लालाचन्द्रसूच्यां दृश्यते । वासुदेवसार्वभौमकृता-अपरा काचन व्याख्या (28038, D.C. B.S.V. Vol. VII P. 128) राजेन्द्रलालसूच्यां दृश्यते । हरिरामकृता व्याख्या नार्तवेस्ट सूच्यां लभ्यते । विज्ञानदीपिकानाम्नी अज्ञातकर्तृका कामकोटिपुस्तकालये लभ्यते । अज्ञातकर्तृका काचन व्याख्या उपनिषद्ब्रह्मेन्द्रसूच्यां दृश्यते ।
भगवन्नामकौमुदी, भागवतव्याख्या - अमृततरङ्गिणी च अनेन कृताविति ज्ञायते ।
४९. सदानन्दसरस्वती (1500-1600 A.D.)
शङ्करानन्दस्य प्रथमसदानन्दस्य च प्रशिष्यः अद्वयानन्दशिष्यश्चायं सदानन्दसरस्वती नृसिम्हाश्रमिसामयिकः षोडशशतकीय इति निश्चयः ।
१. वेदान्तसारः (N.S.P.)
विभिन्नमतसिद्धान्तोपवर्णनपूर्वकं अद्वैतमतस्य श्रैष्ठ्यत्वमुपादेयत्वञ्च वर्णयन्नयं सव्याख्यः प्रकरणग्रन्थः निर्णयसागरमुद्रणालये वाणीविलासमुद्रणालये वेङ्कटेश्वरमुद्रणालये च मुद्रितः । अस्य व्याख्या - आपदेवकृता बालबोधिनी, नृसिम्हसरस्वतीकृता सुबोधिनी, रामचन्द्रानन्दकृता ब्रह्मबोधिनी, रामतीर्थकृता विद्वन्मनोरञ्जिनी, रामशरणशास्त्रिकृता भावबोधिनी च ।
२. वेदान्तसिद्धान्तसारसंग्रहः (R. 1939 D. MGOML)
अद्वैतसिद्धान्ते विभिन्नग्रन्थप्रतिपादितान् सिद्धान्तान् सङ्गृह्णाति । अमुद्रितोऽयं अडयारपुस्तकालये अनन्तशयनपुस्तकालये मद्रासराजकीयपुस्तकालये च लभ्यते ।
५०. नृसिम्हाश्रमी (1500-1600 A. D.)
जगन्नाथाश्रममुनेश्चित्र चरणरेणवः । वाचामगोचरेऽप्यर्ये मूकं वाचालयन्ति माम् ॥ कल्याणगुणसम्पूर्णं निर्वाणविभवालयम् । गीर्वाणेन्द्रसरस्वत्याश्चरणं शरणं भजे ॥
इति दर्शनात् नृसिम्हाश्रमिणः विद्यागुरुर्जगन्नाथाश्रमः, दीक्षागुररुर्गीर्वाणेन्द्रसरस्वतीति निश्चीयते । काञ्चीमण्डलान्तर्गतपुरुषोत्तमपुरवासीति ज्ञायते । अयमप्पय्यदीक्षितगुरुः काञ्चीमण्डलान्तर्गतचोलङ्गिपुरवासीति आश्रमस्वीकारात्पूर्वं सच्चिदानन्दशास्त्रीत्यस्यैव नामान्तरमिति वदन्ति । अयमेव नृसिम्हाश्रमी सगुणभक्तान् अप्पय्यदीक्षितान् निगुणसक्तान् अद्वैतनिष्ठानकरोदिति सम्प्रदायः ।
अस्य सामयिकेषु वेदान्तसारकर्त्ता सदानन्दसरस्वती, प्रकाशानन्दः, नानादीक्षितः, अद्वैतानन्दः, भट्टेजिदीक्षितः, अप्यप्यदीक्षितपिता रङ्गराजाध्वरी च प्रसिद्धाः । अस्य शिष्येषु प्रसिद्धाः नारायणाश्रमधर्मराजाध्वरिगुरुवेङ्कटनाथ भट्टोजिदीक्षितकनीयसभ्राता (रंगोजिमट्टः) धर्मराजाध्वरी च ।
अनेन अब्दे वेदवियद्रसेन्दुगुणिते विक्रमाब्दे (1603-1547 A. D.) अद्वैतदीपिका रचितेति ज्ञायते । एवञ्च षोडशशतके नृसिम्हाश्रमी अतिप्रसिद्धः अद्वैतवेदान्ताचार्येष्वतिप्रतिष्ठितः वेदान्तसिद्धान्तसाराभिज्ञ इति प्रसिद्ध इति निश्चीयते ।
%%% Chart
१. अद्वैतदीपिका (P. S. B.)
ग्रन्थोऽयं साक्षिविवेकविभागप्रक्रियौपनिषददीपिकानन्ददीपिकाख्यैः परिच्छेदैः परिच्छिन्नः पण्डितग्रन्थमालायां वाराणस्यां मुद्रितः । अस्य व्याख्याः - नारायणाश्रमकृतं विवरणम्, सदानन्दव्यासवरकृता व्याख्या, सुन्दरराजकृता प्रकाशाख्या च ।
२. अद्वैतसिद्धान्तविजयः (R. 165 C. MGOML)
अमुद्रितोऽयं मद्रासारजकीयपुस्तकालये अज्ञातकर्तृकव्याख्यया सह लभ्यते ।
३. अद्वैतानुसन्धानम् - ग्रन्थोऽयममुद्रितः बङ्गालराजकीयग्रन्थसूच्यां जयपुरपोटीखाना सूच्याञ्च दृश्यते ।
४. तत्वादीपनम् - (D. 4525 MGOML)
मल्लणाराध्यकृताभेदरत्नस्य व्याख्यात्मकोऽयं ग्रन्थ अमुद्रितः मद्रासराजकीयपुस्तकालये लभ्यते । अखण्डानन्दमुनिकृताद्भिन्नोऽयं ग्रन्थः । अस्य कर्ता किमयं नृसिम्हाश्रम उतान्य इति न निर्णतु पायते ।
५. तत्वबोधिनी - (S.B.T.S. 69)
संक्षेपशारीरकव्याख्यात्मकोऽयं ग्रन्थस्सरस्वतीभवनग्रन्थमालायां मुद्रितः ।
६. तत्वविवेकः - (P.S.B)
ग्रन्थोऽयं नव्यनैय्यायिकसरण्या विवरणप्रस्थानानुसारं अद्वैतसिद्धान्तान् प्रतिपादयन् वैशेषि कादिमतं खण्डयति । ग्रन्थोऽयं पण्डितग्रन्थमालायां मैसूरराजकीयग्रन्थमालायाञ्च मुद्रितः । अस्य व्याख्या मूलकारेण कृता तत्वविवेकदीपनापराभिधा अद्वैतरत्नकोशनाम्नी मुद्रिता । अस्य भट्टोजिदीक्षितकृता ``वाक्यमाला" नाम्नी व्याख्याप्यस्तीति ज्ञायते ।
७. तत्वविवेकदीपनम् -
तत्वविवेकदीपननामायं ग्रन्थस्तत्वविवेकव्याख्यात्मकः । अस्यैव ``अद्वैतरत्नकोश" इत्यपि नामान्तरम् । मुद्रितश्चायं ग्रन्थः मैसूरराजकीयपुस्तकालयमालायाम् । अस्य व्याख्याः - अग्निहोत्रभट्टकृता अद्वैतरत्नकोशपूरणी, अखण्डानन्दयतिकृता अद्वैतरत्नकोशदीपिका, अन्नम्भट्टकृता व्याख्या, कलहस्तीशयज्वकृता भावप्रकाशिका, रामाध्वरिकृता अद्वैतरत्नकोशपालिनी, अनुभवानन्दकृता कोशरत्नप्रकाशाख्या, शाश्वतानन्दतीर्थकृता भावार्थप्रकाशिका च ।
%%% Chart
८. तत्वम्पदार्थशोधनप्रकारः (7507 T. S. M. L.) अपूर्ण अमुद्रितश्चायं ग्रन्थः सरस्वतीमहालये लभ्यते ।
९. नृसिम्हविज्ञापना (S. B. T. S. 52)
एद्यमयोऽयं ग्रन्थ अद्वैतवेदान्तसाहित्यरत्नम् । ग्रन्थेऽस्मिन् जीवेश्वरसाक्षिणां स्वरूपम्, बिम्बप्रतिबिम्बवादः अविद्यास्वरूपः, सर्वश्रुतीनामद्वैते ऐकमत्यादिकं सरलया शैल्या नृसिम्हप्रार्थनामुखेन निरूपितम् । सरस्वतीभवनग्रन्थमालायां मुद्रितश्च ।
१०. पञ्चपादिकाटीका (R. 2626 MGOML)
वेदान्तरत्नकोशापराभिध अपूर्ण अमुद्रितश्चायं ग्रन्थः मद्रासपुस्तकालये सरस्वतीमहालये मैसूरपुस्तकालये च लभ्यते ।
११. पञ्चपादिकाविवरणव्याख्या (D. 4669 MGOML)
भावप्रकाशिकापरनामायं ग्रन्थः अडयारपुस्तकालये मैसूरपुस्तकालये मद्रासपुस्तकालये च लभ्यते ।
१२. भावाज्ञानप्रकाशिका (23. G. 10. ग्र. 18 AL)
१३. मधुमञ्जरी (7329 T. S. M. L)
मनीषापञ्चकव्याख्यात्मकोऽयं ग्रन्थस्सरस्वतीमहालये लभ्यते ।
१४. भेदधिक्कारः (B. S. S. 86)
नाम्नैव प्रतिपादितविषयोऽयं ग्रन्थ अन्तःकरणातिरिक्त अहमितिप्रतीयमानो जीवः परस्मान् न भिद्यते चेतनत्वात्, पदार्थत्वात् ब्रह्मवदिति प्रयोगमुखेन मध्वसिद्धान्तप्रतिपादितं नैय्यायिकवैशेषिकसिद्धान्तसाधितञ्च भेदवादं तेषां युक्त्यैव धिक्करोति । अस्य व्याख्या नारायणाश्रमिकृता कालहस्तीशकृता च ।
%%% Chart
१५. वाचारम्भणप्रकरणम् (R. 2251 MGOML)
१६. वैदिकसिद्धान्तसंग्रहः (D. 4750 MGOML)
अपूर्ण अमुद्रितश्चायं ग्रन्थश्शिवविष्णुरुद्राणां परस्मात् ब्रह्मण एव सद्गुणरूपेणाविर्भूतत्वं वर्णयन् मूर्तित्रयाद्वैतं वर्णयति । ग्रन्थोऽयं मद्रासराजकीयपुस्तकालये लभ्यते ।
५१. अप्पय्यदीक्षितः (1520-1593 A. D.)
संस्कृतवाङ्प्तये विशेषतः दक्षिणदेशे च अप्पय्यनाम्ना बहवः प्रसिद्धा आसन् । तेषु प्रसिद्धतरौ द्वौ यावधिकृत्य नात्र विचारः क्रियते । प्रसिद्धतमः चतुरधिकशताधिकग्रन्थप्रणेतृत्वेन प्रसिद्धः अद्वैतपक्षपाती अप्पय्यदीक्षित एवात्र विचारार्हः ।
अप्पय्यदीक्षितस्य पितामहः-
अप्पय्यदीक्षितस्य पितामह आच्चान् दीक्षितेत्यपरनामकः आचार्यदीक्षितः । काञ्चीपुरसमीपवर्तिनि अडयप्पलनाम्न्यग्रहारे उवास । वक्षस्थलाचार्य इत्यपि नामा न्तरमस्य । वक्षस्थलाचार्य इति नाम विजयनगराधीश्वरेण कृष्णराजमहीपतिना प्रदत्तमिति वदन्ति । कदाचित् विजयनगराधीशः सपत्नीकः सपरिवारः काञ्चीनगरमाजगाम । तदोपासीनवरदराजस्य महीपतेः पत्नीं वरदराजसन्निधौ वीक्ष्य वक्षस्थलगणपत्युपासकेन स्वीकृतनृपतिप्रार्थनेन आचार्यदाक्षितेन एवमावेदितम् - ``काञ्चित् काञ्चनगौराङ्गी वीक्ष्य साक्षादिव श्रियम् । वरदस्संशयापन्नः वक्षस्थलमवैक्षत । इति तादृशमुपश्लोकं श्रुत्वा काममामोदितमानसेन चमत्कृतमनसा राज्ञा" ``वक्षस्थलाचार्य" इत्याहूत आचार्यदीक्षितः । प्रतिपादितञ्चैतत् अप्पय्यदीक्षितैश्चित्रमीमांसायां सन्देहालङ्कारध्वन्युदाहरणावसरे । आचार्यदीक्षित अद्वैतचित्सुखमहाम्बुधि मग्नभाव इति च अप्पय्यदीक्षितैर्न्यायरक्षामणौ आवेदितः । आचार्यदीक्षितस्य द्वे भार्ये आस्ताम् । तयोः प्रथमा सजातीया । द्वितीया श्रीवैष्णवकुलतिलकश्रीवैकुण्ठाचार्यवंश्यश्रीरङ्गराजाचार्यदुहिता विप्रकन्या तोतारम्बीनाम्नी । अस्यामाचार्यदीक्षितस्य पुत्राश्चत्वार अभूवन् । तेषु मातामहप्रार्थनाप्रसन्नपितृदत्तरङ्गराजाभिधानः रङ्गराजाध्वरीति प्रसिद्धः कश्चन ।
अप्पय्यदीक्षितपिता रङ्गराजाध्वरी-
अद्वैतमुकुर-विवरणदर्पणादिप्रबन्धकृत् सर्वविद्याविशारदश्च । प्रपञ्चितञ्चैतत् न्यायरक्षामणौ अप्पय्यदीक्षितैः, नलचरिते नीलकण्ठदीक्षितैश्च । अस्य द्वौ पुत्रावास्ताम् । तयोर्ज्यायान् अप्पय्यदीक्षितः कनीयानाचार्यदीक्षितः । यस्य नाम कालक्रमेण विकृतिं गतं आच्चान् दीक्षित इत्यापि व्यवहृतम् ।
अप्पय्यदीक्षितस्य नामान्तराणि -
स्वापित्रा रङ्गराजाघ्वरिणा पुत्रः `अप्पा' इति `अय्या' इति सस्नेहमाह्वानात् अप्पय्यदीक्षित अप्पदीक्षित इति व्यवहृतः । परन्तु पित्रा नामकरणसंस्कारकाले ``विनायक" इति नाम कृतमिति वदन्ति । अप्पय्यदीक्षितस्य ``अवधानियज्वा" इत्यपि नामान्तरमिति अप्पय्यदीक्षिकृतायाः न्यायसिद्धान्तमञ्जरीव्याख्यायाः ज्ञायते इति दासगुप्तः (H. I. D. II 230)
अप्पय्यदीक्षितगुरवः -
अप्पय्यदीक्षितस्स्वपितुरेवावाप्तसर्वविद्य इति रंगराजाध्वर्येवास्य गुरुरिति सार्वजनीनम् । परन्तु ``अप्यय्यदीक्षितैर्वाराणस्यां कस्मिंश्चित् धार्मिके विषये दत्तायां व्यवस्थायां स्वपित्रा रंगराजाध्वरिणा साकं स्वेन नृसिम्हाश्रमिणां सविधे शास्त्राध्ययनं कृतमित्युक्तमिति सूर्यनारायणशुक्लमहोदयैर्नृसिम्हविज्ञापनाभूमिकायां लिखितम् । एवञ्च नृसिम्हाश्रमी चास्य गुरुरिति केचित् । प्राकृतमणिदीपिकाख्यः ग्रन्थः न प्रसिद्धाप्यय्यदीक्षितस्य कृतिरिति केचित् । केचित्तु तां प्रसिद्धदीक्षितीयां मत्वा एवं वदन्ति प्राकृतमणिदीपिकायां" ``ध्यायामि पादपद्भे सच्चिदानन्दशास्त्रिण" इति दर्शनात् सच्चिदानन्दशास्त्रीत्यपरो गुरुरप्यस्येति । अपरे आश्रमस्वीकारात्पूर्वं नृसिम्हाश्रमिण एव सच्चिदानन्दशास्त्रीति नामान्तरमिति वदन्ति ।
अप्पय्यदीक्षितकालः -
शिवानन्दयोगिरचिते अप्पय्यदीक्षितचरिते अप्पय्यकाल (1553 A. D.) दृश्यते । भट्टात्मजरामजयन्तपण्डितेन (1564 A. D.) इति वर्ण्यते । (1520 A. D.) इति आंग्लविमर्शकवरेण्या वदन्ति । (1550 A. D.) इति ब्रह्मविद्यापत्रिकायां प्रतिपादितम् । मानवल्लीगङ्गाधरशास्त्रिणश्च (1550 A. D.) इत्येव वदन्ति । हरिहरशास्त्रिणस्सिद्धान्तलेशसंग्रहभूमिकायां (1587 A. D.) इति वदन्ति । अप्पय्यदीक्षितेन्द्रविजयकाव्ये (18 - 9 - 1553 A.D.) इति दृश्यते । हालास्यनाथशास्त्रिभिः कुवलयानन्दभूमिकायां (1552-1626 A.D.) इति प्रतिपादितम् । भारतीयशिलाशासनपत्रिकायाः (Ep. Ind. Vol XII 1554-1626 A.D.) इति ज्ञायते । षोडश शतकमध्यकालिक इति दासगुप्तः (H. I. P. Vol II. P. 230) । इतिहास-शिलाशासन-सम्प्रदाय - साहित्यजन्मपत्रादिप्रमाणात् अप्पय्यदीक्षितकाल (1520-1593 A.D.) इति अप्पय्यदीक्षितवंश्यैर्महालिङ्गशास्त्रिभिः प्रतिपादितम् (J. O. R. Vol III P. 160)
अप्पय्यदीक्षिताः चिदम्बरक्षेत्रं गता चिदम्बरेशसन्निधौ ``आभाति हाटकसभानटपादपद्मज्योतिर्मयो मनसि मे तरुणारुणोऽयमिति" 
``चिदम्बरमिदं पुरं प्रथितमेव पुण्यस्थलम् ।
वयांसि मम सप्ततेरुपरि नैव भोगे स्पृहा ।
सुताश्च विनयोज्वलास्सुकृतयश्च काश्चित्कृताः ।
न किश्चिदहं अर्थये शिवपदं दिदृक्षे परम् ॥"
इति वदन्त एवैक्यं गताः, प्रार्थनानुपदं सायुज्यसिद्धिरिति सम्प्रदायसमागता कथा श्रूयते । तस्मात् अप्पय्यदीक्षितजीवनकाल माकिं द्वासप्ततिरिति निर्णीयते ।
अप्पय्यदीक्षितभ्राता आचार्यदीक्षितः -
अप्पय्यदीक्षितस्य कनीयसः भ्रातुर्नाम आचार्यदीक्षित इति । अप्पय्यदीक्षितस्य काचन कनीयसी भगिनी आसीदित्यपि ज्ञायते । आचार्यदीक्षितस्य पञ्च पुत्रा आसन् । तेषु द्वितीयः पुत्रः नीलकण्ठदीक्षितनामा प्रसिद्धः नीलकण्ठविजयचम्प्वादिकर्ता । 
अप्पय्यदीक्षितधर्मपत्नी-
अप्पय्यदीक्षितभार्या मङ्गलनायिकानाम्नी प्रसिद्धानां रत्नखेटश्रीनिवासदीक्षितानां पुत्रीति ज्ञायते ।
अप्पय्यदीक्षितापत्यानि -
अप्पय्यदीक्षितस्य नीलकण्ठ-उमामहेश्वर-चन्द्रावतंसनामानः त्रयः पुत्राः मरकतवल्ली-मङ्गलाम्बाख्ये द्वे कन्ये चोदभवन् ।
अप्पय्यदीक्षितशिष्येषु प्रधानतमाः-
अप्पय्यशिष्येषु प्रसिद्धः भट्टोजिदीक्षितः । अप्पय्यदीक्षितप्रेरणया तत्वः कौस्तुभं शाङ्करभाष्याब्धेरुद्धधार । नालकण्ठदीक्षितोपि प्रसिद्धः ।
अप्पय्यदीक्षितसामयिकाः केचन विद्वांसः -
विद्यापरिणयजीवानन्दादिप्रणेता आनन्दरायमखी, रत्नकेतूदयसुभद्रापरिणयादिप्रबन्धकारः वीरराघवयज्वापरनामा बालकवि, मल्लिकामारुतादिप्रकरणकर्ता उद्दण्डापरनामा सार्वाभौमकविः, चन्द्रगिरिमहीपतिगुरुः श्रीवैष्णवः ताताचार्यः, भाट्टदीपिका-भाट्टरहस्यादिकृत् वारणसीवसी पण्डितरजजगन्नाथपितृपेरुभट्टगुरुः सन्यासस्वीकारादनन्तरं श्रीधरेन्द्रयतीति प्रसिद्धश्च खण्डदेवः, यात्राप्रबन्धकृत् समरपुङ्गवदीक्षितः, कमलिनीकलहंस - आनन्दराघवभावनापुरुषोत्तम-भैष्मीपरिणय-काव्यदर्पण-तन्त्रशिखामणिप्रभृतिग्रन्थनिर्माता सत्यमङ्गलवासी रत्नखेटदीक्षितपुत्र राजचूडामणिः, विश्वगुणादर्शचम्पू-प्रद्युम्नानन्दनाटकादिकृत् ताताचार्यभागिनेयः वेङ्कटाध्वरी, शतदूषणीव्याक्याकारः दोड्डयाचार्यः नीलकण्ठदीक्षितमन्त्रगुरुर्गीर्वाणयोगी, वार्तिकाभरणकर्ता नीलकण्ठदीक्षितविद्यागुरुर्वेङ्कटेश्वरमखी च अप्पय्यदीक्षितसामयिकाः ॥
केचित्तु रसगङ्गाधर-चित्रमीमांसाखण्डनकारः सम्राट्शाहजहाँप्रेमपात्रं जगन्नाथपण्डितोऽपि, ब्रह्मतत्वप्रकाशिकाकारस्सदाशिवब्रह्मेन्द्रश्च सामयिकाविति वदन्ति ।
एतादृशं शिवाद्वैतपक्षपातिनं सर्वतन्त्रनिष्णातं प्रकाण्डपण्डितं अत्युत्कटसाधकं अप्पय्यदीक्षितं नृसिम्हाश्रम्येव अद्वैतपक्षपातिनमद्वैतप्रचारबद्धदीक्षञ्चा कुर्वन्निति सम्प्रदायः । अप्पय्यदीक्षितेन यज्ञेश्वरदीक्षितेन साकं न्यायशास्त्रमप्यधीतमिति जानकीनाथकृतन्यायसिद्धान्तमञ्जर्याख्यग्रन्थस्य व्याख्याया ज्ञायते । अप्पय्यदीक्षितश्चतुरधिकशताधिकप्रबन्धकर्ता भारद्वाजगोत्रज इति प्रसिद्धिः । तेषु अद्वैतग्रन्थाः ।
१. सिद्धान्तलेशसङ्ग्रहः (A. M. S. 5)
शङ्करादारब्धानां नृसिम्हाश्रम्यन्तानां आत्मैक्यसिद्धौ सन्नह्यतां अद्वैताचार्याणां प्रक्रियास्सिद्धान्तांश्च प्रतिपादयन्नयं ग्रन्थः न केवलं सिद्धान्तप्रदर्शनेन परन्तु ऐतिहासिकदृष्ट्यापि बहूपकरोति । अद्वैत्तमञ्जरीग्रन्थमालायां मुद्रितश्च । अस्य व्याख्याः - अच्युतकृष्णानन्दकृता कृष्णालङ्काराख्या, राघवानन्दकृता सिद्धान्तकौमुदी, रामचन्द्रपूज्यपादकृता सिद्धान्तसूक्तिमञ्जरी, विश्वनाथकृता व्याख्या, वासुदेवब्रह्मकृतस्संग्रहसारश्च विद्यन्ते । दासगुप्तेन (H. I. P. Vol II 220) गङ्गाधरेन्द्रसरस्वत्या सिद्धान्तबिन्दुसीकर इति, रामचन्द्रयज्वना गूढार्थप्रकाश इति, विश्वनाथतीर्थेन धर्मय्यदीक्षितेन च व्याख्याः कृता इति प्रतिपाद्यते । P. P. शास्त्रिणश्च सिद्धान्तलेशसंग्रहस्य मधुसूदनसरस्वत्या काचन व्याख्या कृतेति (7535 DC. T. S. M. L. Vol XII) ग्रन्थे प्रतिपादयन्ति ।
२. न्यायरक्षामणिः (S. V. P.)
शारीरकन्यायरक्षामणिरिति प्रसिद्धोऽयं ग्रन्थः ब्रह्मसूत्रवृत्तिरूपः भाष्यं भामतीञ्चानुसरति । प्रत्यधिकरणं पूर्वपक्षसिद्धान्तौ मीमांसान्यायप्तनुसृत्य सविस्तरमुपवर्णितौ । प्रतिसूत्रं दलप्रयोजनमस्य ग्रन्थस्य महान् विशेषः । प्रथमाध्यायान्तमुपलभ्यमानोऽयं ग्रन्थः भाष्यस्य क्रोडपत्ररूपः वाणीविलासे मुद्रितश्च ।
३. परिमलः । अमलानन्दकृतभामतीव्याख्याकल्पतरुव्याख्यारूपोऽयं ग्रन्थः मुद्रितः । अस्य संग्रहोऽपि विद्यते ।
४. मध्वतन्त्रमुखमर्दनम् - (S.V.P.)
आनन्दतीर्थीयां द्वैतपरां शारीरकशास्त्रप्राथमिकपञ्चाधिकरणप्रक्रियां तन्मतमर्यादयैव स्रग्धराधटितपद्यैः खण्डयन्नयं ग्रन्थः मध्वमुखमर्दनं माध्वमुखभङ्ग इति नाम्ना प्रसिद्धः । अस्य व्याख्यापि मूलकत्कृता व्यध्वविध्वंसनाख्या मुद्रिता च ।
५. व्यध्वविध्वंसनम् - (S.V.P.)
६. नयमञ्जरी - (V.V.P.)
अद्वैतमतानुसारेण सूत्रार्थविवरणं ग्रन्थेऽस्मिन् उपलभ्यते । अप्पय्यदीक्षितैरेतैः कृते चतुर्मतसारसंग्रहे परिच्छेदचतुष्टयात्मके उपलभ्यमानः अद्वैतपरिच्छेदार्थसंग्रहात्मकः 182 पद्यैः कल्पितः ग्रन्थ वाणीविलासे मुद्रितः ।
७. अधिकरणकुञ्चिका - भारतीसदनमुद्राक्षरशालायां कार्वेटनगरे मुद्रितः ।
८. चतुर्मतसारसंग्रहः- (V.P.P.)
चतुर्भिः परिच्छेदैः परिवलृप्तः । प्रथमे न्यायमुक्तावल्याख्ये द्वैतिनां मतं, द्वितीये नममयूखमालिकाख्ये विशिष्टाद्वैतिमतं, तृतीये नयमणिमालाख्ये श्रीकण्ठाचार्यमतं शिवाद्वैतं प्रतिपादितम् । चतुर्थे नयमञ्जर्याख्ये सिद्धान्तपद्धत्या अद्वैतमतमुपवर्णितम् ।
९. वादनक्षत्रमालिका - (V.V.P.)
पूर्वोत्तरमीमांसावादनक्षत्रमालिकेति प्रसिद्धोऽयं ग्रन्थः सप्तर्विशतिभिः वादकोटिभिः प्रणीतः । तत्र प्रथमेषु अष्टभिः कोटिभिः पूर्वमीमांसायां पूर्वपक्षसिद्धान्ताः प्रतिपादिताः । द्वितीयेषु एकोनर्विशतिभि कोटिभिः शाङ्करभाष्योपरि प्रवृत्तानां पूर्वपक्षाणां खण्डनं कृतमिति पूर्वोत्तरमीमांसावादनक्षत्रमालिकेत्यन्वर्थनामा ।
अप्पय्यदीक्षितरचितग्रन्थसूची -
१. अद्वैतनिर्णयः 
२. अधिकरणकुञ्चिका
३. अधिकरणमाला
४. अधिकरणसारावलिः
५. अनुग्रहाष्टकम् 
६. अमरकोषव्याख्या
७. अरुणाचलेश्वरस्तुतिः 
८. आत्मार्पणस्तुतिः
९. आदित्यस्तोत्ररत्नम् 
१०. आनन्दलहरी
११. आनन्दलहरीव्याख्या चन्द्रिका 
१२. आर्याशतकम् अस्य व्याख्या `राघवशर्मणा' कृता पूनानगरे मुद्रिता च ।
१३. उपक्रमपराक्रमः
१४. कुवलयानन्दः
१५. कृष्णध्यानपद्धतिः
१६. कृष्णध्यानपद्धतिव्याख्या
१७. गङ्गाघराष्टकम् 
१८. चतुर्मतसारससंग्रहः
१९. चित्रपटः (लघुवार्तिकम्)(J. O. R. Vol III) मुद्रितः ।
२०. चित्रमीमांसा
२१. जयोल्लासनिधिः
२२. णत्वसमर्थनम्
२३. तत्वमुक्तावलिः
२४. तप्तमुद्राविद्रावणम् 
२५. तान्त्रिकमीमांसा (A. O. R. Vol VI 3)
२६. तिङन्तशेषसंग्रहः
२७. दशकुमारचरितसंग्रहः
२८. दशकोटिः
२९. दुर्गाचन्द्रकलास्तुतिः
३०. दुर्गाचन्द्रकलाविवरणम् 
३१. धर्ममीमांसापरिभाषा
३२. नयमञ्जरी
३३. नयमयूखमालिका
३४. नामसंग्रहमाला
३५. नामसग्रहमालाव्याख्या
३६. निग्रहाष्टकम्
३७. न्यायमुक्तावलिः
३८. न्यायमुक्तावलीव्याख्या
३९. न्यायरक्षामणिः
४०. न्यायरत्नमाला
४१. न्यायरत्नमालाव्याख्या
४२. पञ्चरत्नस्तुतिः
४३. पञ्चरत्नस्तुतिव्याख्या
४४. पञ्चस्वरवृत्तिः
४५. पादुकासहस्रव्याख्या
४६. परिमल (कल्पतरुव्याख्या)
४७. पूर्वमीमांसाविषयसंग्रहदीपिका (J. O. R. Vol IX 321)
४८. प्रबोधचन्द्रोदयटीका
४९. प्राकृतचन्द्रिका
५०. ब्रह्मतर्कस्तवः 
५१. ब्रह्मतर्कस्तवविवरणम्
५२. भक्तिशतकम्
५३. भस्मवादावलिः
५४. भारततात्पर्यसंग्रहः
५५. भारततात्पर्यसंग्रहव्याख्या
५६. मणिमालिका
५७. मतसारार्थसंग्रहः
५८. मध्वतन्त्रमुखमर्दनम्
५९. व्यध्वविध्वंसनम्
६०. मार्गबन्धुच्म्पूः
६१. मार्गबन्धुपञ्चरत्नम् 
६२. मार्गसहायलिङ्गस्तुतिः
६३. मार्गसहायस्तोत्रम्
६४. मानसोल्लासः
६५. यादवाभ्युदयव्याख्या
६६. योगमार्ताण्डः
६७. रत्नत्रयपरीक्षा
६८. रत्नत्रयपरीक्षाव्याख्या चन्द्रिका
६९. रामायणतात्पर्यसंग्रहः
७०. रामायणतात्पर्यसंग्रह-व्याख्या
७१. रामायणतात्पर्यनिर्णयः
७२. रामायणसारसंग्रहः
७३. रामायणतात्पर्यनिर्णयः
७४. रामायणसारः
७५. रामानुजमतखंडनम्
७६. रामानुजशृङ्गभङ्ग
७७. लक्षणरत्नावली सव्याख्या J.O.R. Vol IV 242
७८. वरदराजस्तवः
७९. वरदराजस्तवविवरणम्
८०. वसुमतीचित्रसेनविलासः
८१. वादनक्षत्रमालिका
८२. विधिरसायनम्
८३. विधिरसायनव्याख्या सुखोपयोगिनी
८४. विष्णुतत्वरहस्यम्
८५. वीरशैवम्
८६. वृत्तवार्तिकम् 
८७. व्याकरणवादनक्षत्रमाला
८८. शब्दप्रकाशः
८९. शान्तिस्तवः
९०. शास्त्रदीपिकाव्याख्या
९१. शिखरिणीमाला
९२. शिवकर्णामृतम्
९३. शिवकल्पदुमः
९४. शिवतत्वविवेकः
९५. शिवध्यानपद्धतिः
९६. शिवध्यानपद्धतिव्याख्या
९७. शिवपुराणतामसत्वखण्डनम्
९८. शिवपूजाविधिः
९९. शिवार्कमणिदीपिका
१००. शिवार्चनाचन्द्रिका
१०१. शिवार्चना चन्द्रिकाव्याख्या ``बालचन्द्रिका"
१०२. शिवाद्वैतनिर्णयः
१०३. शिवोत्कर्षचन्द्रिका
१०४. शिवोत्कर्षमञ्जरी
१०५. शैवकल्पद्रुमः
१०६. श्रीविद्यातत्वविवरणम्
१०७. सिद्धान्तरत्नाकरः
१०८. सिद्धान्तलेशसंग्रहः
१०९. स्तोत्ररत्नाकरः
११०. हरिवंशसारचरितव्याख्या
१११. हरिहरस्तुतिः 
११२. हंससन्देशटीका
एषु नयमञ्जरी, नयमयूखमालिका उभावपि चतुर्मतसारसंग्रहान्तर्भूतौ वसुमतीचित्रसेननाटकं तु नैतेनाप्पय्यदीक्षितेन कृतम् । परन्तु तन्त्रसिद्धान्तदीपिकादूरूहशिक्षा-प्राकृतमणिदीपिका-अतिदेशलक्षणपुनराक्षेपादिकर्त्रा प्रसिद्धापय्यदीक्षितपौत्रेण कृतमिति (AOR Vol VI 3. & JOR Vol II 247) पत्रिकासु प्रतिपादितम् । एवं दशकुमारचरितसारोऽपि अप्पय्यदीक्षितवंश्यैः कृतमिति (AOR Vol VI 3) प्रतिपादितम् । अमुद्रिते जयोल्लासनिधिनामके भागवतव्याख्याने तत्कर्ता श्रीनिवासपुत्रः श्रीवत्सगोत्रज अप्पय्य इति (6742 DC I.O.L Vol II) दृश्यते । प्रसिद्धस्तु भारद्वाजगोत्रज इति नैतद्ररचनेति ज्ञायते ।
णत्वसमर्थनन्तु न कुत्राप्युपलभ्यते परन्तु वेङ्कटेशबिरचिते स्वात्मानुभूतिमणिदर्पणे (21. J. 2. AL Mss) परामृष्टः ।
५२.धर्मराजाध्वरी (1550-1650 A. D.)
यदन्तेवासिपञ्चास्यैर्निरस्ता भेदिवारणाः । तं नौमि नृसिम्हाख्यं यतीन्द्रं परमं गुरुम् ॥ इति नृसिम्हाश्रमिण, ``श्रीमद्वेङ्करनाथाख्यान् वेलाङ्गुडिनिवासिनः" । इति वेङ्कटनाथञ्च नमस्कुर्वन्नयं धर्मराजाध्वरी नृसिम्हाश्रमिप्रशिष्यः वेङ्कटनाथपुत्रः रामकृष्णदीक्षितपिता त्रिवेदीनारायणदीक्षितभ्राता कौण्डिन्यगोत्रजः, ऋग्वेदाध्यायी चोलदेशस्थकण्डरमाणिक्कग्रामवासीति सप्तदशशतकीय इति च निश्चीयते । अस्यशिष्येण नारायणशस्त्रिणा कृता महाभाष्यप्रदीपव्याख्या मद्रपुरीपुस्तकालये लभ्यते । (R. 39 MGOML)
१. अद्वैतवेदान्तपरिभाषा-
अष्टभिः परिच्छेदैः पूर्णोऽयं ग्रन्थः प्रत्यक्षानुमानोपमानशब्दार्थापत्ति-अनुपलब्धिरूपाणि प्रमाणानि सपरिकरं निरूप्य वेदान्तस्य विषयप्रयोजने च विशदयति । प्रकरणग्रन्थोऽयं अनन्तशयन - कल्कत्ता - अडयार - आदिमुद्रणालये मुद्रितः । अस्य व्याख्याः - शिवदत्त - धनपतिसूरि - जीवानन्द विद्यासागरकृताः अर्थदीपिकाः, अनन्तकृष्णशास्त्रि पेद्दादीक्षितकृताः प्रकाशिकाः, कृष्णनाथकृता आशुबोधिनी, वेदान्द्रिसूरिकृता तत्वबोधिनी, नारायणभट्टकृतं भूषणम्, अमरदासकृता मणिप्रभा, रामकृष्णदीक्षितीयः शिखामणिः, रामवर्मकृतः परिभाषासंग्रहश्चेति विद्यन्ते ।
२. पञ्चपादिकाव्याख्या - पदयोजनिका । अपूर्णोऽयं ग्रन्थः मैसूरपुस्तकालये (आ 237) लभ्यते ।
तर्कचूडामणि-तत्वचिन्तामणिप्रकाशिका न्यायसिद्धान्तदीपव्याख्या-युक्तिसंग्रहादयश्च ग्रन्थाः कृता इति ज्ञायते ।
५३. भट्टोजिदीक्षितः (1550-1650 A.D.)
ऋग्वेदान्तर्गताश्वलायनसूत्रशाकलशाखाध्यायी भट्टकुलावतंसः लक्ष्मीधरपण्डितपुत्रः शेषकृष्णस्य शिष्यः अप्पय्यदीक्षितान्नृसिम्हाश्रमिणश्च प्राप्ताद्वैतविद्यः रङ्गोजिभट्टभ्राता कौण्डुभट्टस्य पितृव्यः भानुदीक्षितवीरेश्वरयोः पिता हरिदीक्षितस्य पितामहश्चायं भट्टोजिदीक्षितः सारस्वतब्राह्मणकुलोत्पन्न इति महाराष्ट्रदेशज इति निश्चीयते । केचित्वेनं दाक्षिणात्यं वदन्ति । दक्षिणयात्रासमये विजयनगराधीशितुः प्रेरणाया मध्वमतविध्वंसः, नृसिम्हाश्रमिण अप्पय्यदीक्षितस्य च प्रेरणया तत्वकौस्तुभश्चानेन कृत इति सम्प्रदायपरम्परागता कथा ।
वंशवृक्षः
%%% Chart
१. तत्वकौस्तुभः - (V.V.P.)
परिच्छेदत्रयपरिमितोऽयं ग्रन्थः आनन्दतीर्थीयद्वैतविचारदूषणपरः । प्रथमपरिच्छेदे मध्वाचार्यसिद्धान्तद्षणं द्वितीये ब्रह्मसूत्रार्थवर्णनम्, तृतीये शङ्करसिद्धन्तप्रदर्शनमिति विवेकः । ग्रन्थोऽयं शङ्करगुरुकुलपत्रिकायां श्रीरङ्गनगरे वाणीविलासमुद्रणालये मुद्रितः ।
२. वाक्यमाला - (1978 B.R.D.)
नृसिम्हाश्रमीयवेदान्तत्वविवेकस्य व्याख्यात्मकोऽयं ग्रन्थ अमुद्रितः बरोडापुस्तकालये लभ्यते । दासगुप्तेन (HIP Vol II 217) निर्दिष्टश्च ।
३. मध्वमतविध्वंसः -(7561 TSML)
आशौचनिर्णयः, आशौचदशश्लोकीव्याख्या, कालनिर्णयः, गौत्रप्रवरनिर्णयः, ग्रहणश्राद्धविधिः, चतुर्विशतिस्मृतिव्याख्या, तन्त्राधिकारिनिर्णयः, तिथिनिर्णयः, त्रिस्थलीसेतुः, प्रौढमनोरमा, शब्दाकौस्तुभः, सिद्धान्तकौमुद्याद्याः ग्रन्थाश्च कृताः ।
५४. गोविन्दानन्दसरस्वती (1550-1650 A.D.)
द्रविडान्वयजः दाक्षिणात्योऽयं गोविन्दानन्दः शिवरामप्रशिष्यः गोपालसरस्वतीशिष्यः रामानन्दगुरुः, ब्रह्मविद्याभरणकारस्य अद्वैतानन्दस्य प्राचार्यः, षोडशशतकापरार्धादारब्धे काले आसीदिति ज्ञायते । अनेन भाष्यस्य रत्नप्रभाख्या व्याख्या कृतेति वदन्ति । परन्तु गोविन्दवाणीचरणकमलगो निर्वृतोऽयं यथालिः इति ग्रन्थे दर्शनात् गोविन्दानन्दो नास्य कर्ता भवति । परन्तु तच्छिष्येण रामानन्देनैव कृता । गुरुकृतत्वेन व्यावहारस्तु आदरातिशयेन इति केचिद्वदन्ति । अतएव रत्नप्रभायाः रामानन्दीयमिति व्यवहारोऽपि सङ्गच्छते । गोविन्दानन्दरामानन्दयो र्गुरुशिष्यपरम्परावृक्षः-
%%% Chart
१. योगमणिप्रभा (R. 3885 MGOML) रामानन्दकृत इत्येव (R 3885) मातृकायां दृश्यते । रामानन्दगोविन्दानन्दयोरैक्यं सम्पाद्य गोविन्दानन्दकृत इति व्यवहारः संगच्छते ।
२. सप्तविधानुपपत्तिभङ्गः । 44 शृङ्गगिरिसृच्यां दृश्यते । ३. भाष्यरत्नप्रभारामानन्दप्रस्तावे लिखितमस्माभिः ।
५५. रामानन्दसरस्वती (1570-1650 A.D.)
शिवरामानन्दगोपालानन्दसरस्वत्योः प्रशिष्यः, गोविन्दानन्दशिष्यश्चायं स्वयम्प्रकाशानन्दसरस्वतीशिष्योऽपि षोडशशतकापरार्घकालवासीति निश्चीयते । अनेन रत्नप्रभायां जगन्नाथाश्रमिकृता भाष्यदीपिका (Page 5 N. S. P. Edn.) निर्दिष्टा ।
१. भाष्यरत्नप्रभा (N. S. P.)
सूत्रभाष्यव्याख्यात्मकोऽयं ग्रन्थः निर्णयसागरे मुद्रितः । अस्य व्याख्याः - अच्युतकृष्णानन्दकृता व्याख्या, प्रकाशानन्दकृता व्याख्या, स्वयम्प्रकाशशिष्यकृता व्याख्या, पूर्णानन्दकृता व्याख्या, अज्ञातकर्तृका अभिव्याक्ता च ।
२. ब्रह्मसूत्रविषयवाक्यवृत्तिः - (R. 2471 MGOML)
३. विवरणोपन्यासः - (B.S.S. 55)
पञ्चपादिकाविवरणतात्पर्यसंग्राहकोऽयं ग्रन्थः चौखाम्बायां मुद्रितः । गद्यात्मकविवरणेन साकं अर्थसंग्राहकश्लोकाश्च विद्यन्ते ।
५६. मधुसूदनसरस्वती (1565-1665 A.D.)
मधुसृदनसरस्वत्याः कुलपूरुषः राममिश्रः-
मुहम्मदगोरीनाम्नः यवनराजस्याक्रमणात् भीताः ब्राह्मणाः (1194 A.D.) काले कान्यकुब्जान् त्यक्त्वा पलायनपरा आसन् । तेषु अध्वर्युः काश्यपगोत्रजः राममिश्राख्यः ब्राह्मणस्सपरिवारः पूर्वस्यां दिशि आससार । स च वङ्गदेशेषु फरीदपुरान्तर्गतकोटालिपाडा नामनि ग्रामे लब्धवास उवास । स एव राममिश्रः मधु सृदनस्य कुलपूरुष इति ऐतिहासिकपरिशीलनात् ज्ञायते । वैदिकवादमीमांसा हरिलीलाप्रामाण्यात् राममिश्रपुत्रः गोपालमिश्रः, गोपालमिश्रस्य पुत्रः माधवमिश्रः, माधवमिश्रस्य पुत्रः माधवमिश्रः, माधवमिश्रस्य पुत्रः सनातनमिश्रः, सनातनमित्रपुत्रः गणपतिमिश्चः, गणपतिमित्रपुत्रः गुणार्णवाचार्यः, गुणार्णवपुत्रः मधुसूदनपिता पुरन्दराचार्यः इति ज्ञायते ।
प्रमोदनपुरन्दराचार्यापरनाम्नः पुरन्दराचार्यस्य वासस्थानं माधवपाशाभि धक्षेत्रम् । अस्य चत्वारः पुत्रा उदभूवन् । ते च श्रीनाथयादवानन्द-कमलजनयन वागीशगोस्वम्याख्याः चत्वारः ।
यादवानन्दन्यायाचार्य इति प्रसिद्धोऽयं कमलजनयनभ्राता प्रतापादित्यस्य सभाचूडामणिरासीत् । तस्य पाण्डित्यमपूर्वं श्रुत्वा मुग्धेन राज्ञा यादवानन्द अविलम्बसरस्वतीति नाम्ना अलञ्चक्रे । अस्यैव माधवसरस्वतीति प्रसिद्धिरासीत् । किमयमेव माधवसरस्वर्ती मधुसृदनसरस्वत्याः विद्यागुरुरुतान्यो वा वृत्तरत्नाकरटीकाकर्तुः नारायणभट्टस्य पितुः रामेश्वरभट्टस्य शिष्यः माधवसरस्वतीति सम्यक् नावधारयितुं पार्यते ।
कमलजनयनः (मधुसूदनसरस्वती) -
पुरन्दराचार्यस्य तृतीयः पुत्रः कमलजनयनाख्यः नवद्वीपे न्यायाशास्त्रमधीतवान् । अस्य वैदुष्यं जानन्नपि माधवपाशाभिधजनपदाधिपः तदीयवासार्हभूभागप्रदाने पराङ्मुख आसीत् । मनस्वी मधुसूदनः नृपतेरेवंविधाचरणमवलोक्य जातखेदः नितरां विषयवैराग्यमनुभवन् स्वीकृतपित्राज्ञः विश्वेश्वरसरस्वत्यास्सन्यासाश्रमं स्वीचकार । आश्रमगुरवः विश्वेश्वरसरस्वत्यः कमलजनयनमेनं मधुसूदनसरस्वत्यभिधानेन भूषयामासुः । ततः प्रभृति तन्नाम्नैव प्रसिद्धिं गतः ग्रन्थांश्चकार ।
%%% Chart
मधुसूदनसरस्वत्याः गुरवश्शिष्याश्च-
मधुसूदनसरस्वत्या प्रायस्सर्वेष्वपि ग्रन्थेषु ``श्रीरामविश्वेश्वरमाधवानां" इति निर्दिश्यते । अद्वैतसिध्यन्ते
``श्रीमाधवसरस्वत्यो जयन्ति यमिनां वराः ।
वयं येषां प्रसादेन शास्त्रार्थे परिनिष्ठिताः ॥"
माधवसरस्वती निर्दिष्टः । एवञ्च विश्वेश्वरसरस्वती दीक्षागुरुः, माधवसरस्वती विद्यागुरुरिति सिध्यति । श्रीरामसरस्वती तु विश्वेश्वरसरस्वतीगुरुरिति श्रीरामसरस्वती मधुसृदनस्य परमगुरुरिति केचित् । हरिरामतर्कवागीशानामन्तिके गदाधरचक्रवर्तिना साकं मधुसूदनेन न्यायशास्त्रमधीतमिति हरिराम एव श्रीरामशब्देन निर्दिष्ट इति प्रतिभाति ।
मधुसूदनसरस्वत्या स्वीये संक्षेपशारीरकव्याख्याने रामतीर्थः केचित्पदेनानूद्यते । एवञ्च रामतीर्थादनन्तरवर्त्ती सामयिको वायं भवितुमर्हति । केचित्तु इदमेव प्रमाणं मत्वा रामतीर्थोऽप्यस्य गुरुरिति वदन्ति । तादृशकथनस्य प्रबलतर प्रमाणाभादेव रामतीर्थस्य शिष्यवर्णनमसङ्गतमिति निश्चयः
मधुसूदनसरस्वतीगुरुविषये परिशीलनावसरे इण्डियन् अण्डिक्वैरि I. A. 1912-9 नामक पत्रिकायां एवं दृश्यते ।
रामेश्वरभट्टनामा पण्डितः राज्ञा वितीर्णानि पारितोषिकादीनि हस्त्यश्वादीनि स्वीकृत्य कल्य एव द्वारकां प्रतस्थे । तस्मिन्नेव मार्गे त्रिपञ्चाशदधिकचतुर्दशतमे शकाब्दे (1531 A. D.) तस्य एकः पुत्रस्समजनि । तस्य च नारायणभट्ट इति महती प्रसिद्धिरभूत् । रामेश्वरभट्टानां जन्मसमये रामेश्वरभट्टः वयसा वृद्ध आसीत् । स च प्रतिष्ठाननगरमध्युवास । अन्ते वाराणसीमाजगाम । रामेश्वरभट्टस्य त्रयः प्रधानशिष्या आसन् - अनन्तभट्टः, दामोदरभट्टः, माधवसरस्वतीति । तेषु माधवसरस्वत्याः शिष्य मधुसूदनसरस्वतीति ।
सिद्धान्तबिन्दुव्याख्याने पुरुषोत्तमसरस्वतीविरचिते बिदुसन्दीपनाख्ये ``विद्यागुरुं गुरुमिव सुराणां मधुसृदनम्" इति दर्शनात् पुरुषोत्तमसरस्वती मधुसूदनशिष्य इति ज्ञायते । ``बहुयाचनया मयायमल्पो बलभद्रस्य कृते कृतो निबन्धः" इति श्लोकांशस्य व्याख्यानावसरे ``बलभद्रो भट्टाचार्यः, कश्चन सम्यक् भक्त" इति दर्शनात् बलभद्रोऽपि शिष्यः इति ज्ञायते । शेषगोविन्दनिर्मितायां शङ्करसर्वसिद्धान्तरहस्यटीकायां ``यत्प्रसादाधीनसिद्धिपुरुषार्थचतुष्टयम् । सरस्वत्यवतारं तं वन्दे श्रीमधुसूदनम् । गुरुणा मधुसृदनेन" इति च दर्शनात् शेषगोविन्दोऽपि मधुसूदनशिष्य इति ज्ञायते ।
%%% Chart
मधुसूदनसामयिकाः विशिष्टाः केचन कवयः -
अवधीभाषाकविसम्राजः रामचरितमानसकर्तुः भक्तकवेः तुलसीदासस्य सामयिकोऽयं मधुसूदनः । अनयोर्मैत्रीविषये
आनन्दकानने काचित् जङ्गमस्तुलसीतरुः ।
कवितामञ्जरी (कवितावलि) यस्य मुखपद्माद् विनिस्सृता ।
इति पद्यमपि प्रसिद्धम् । भारतसम्राजः अकबरस्य सामयिकोऽयं देहलीसभायां अकबरेणाहूतः वादे विदुषः पराभूय (1600 A.D.) काले
``वेत्ति पारं सरस्वत्याः मधुसूदनसरस्वती ।
मधुसूदनसरस्वत्याः पारं वेत्ति सरस्वती" ॥ इति 
गाथाञ्च लेभे । भारतसम्राजः अकबरस्य प्रतिद्वन्द्विनः देशभक्तशिखामणेः मातृभूमिस्वतन्त्र्याय त्यक्तसर्वस्वस्य आजानवीरस्य राणाप्रतापस्य सामयिकः । नव्यनैय्यायिकस्य न्यायशास्त्रविपश्चितः गदाधरभट्टाचार्यस्य सतीर्थ्योऽपि भेदवादखण्डने अस्यानितरसाधारर्णी प्रतिभां विलोक्य गदाधरभट्टाचार्योऽपि चकम्पे । अत एव नवद्वीपे समायाते मधुस्दनवाक्पत्तौ । चकम्पे तर्कवागीशः कातरोऽभूत् गदाधर ॥ ``इति फणितिरपि प्रसिद्धा । अद्वैतदीपिकादिकर्ता नृसिम्हाश्रमी अप्पय्यदीक्षितभट्टोजिदीक्षितादयः अपरवयस्या सामयिका आसन्निति च वक्तुं शक्यते॥" 
मधुसूदनसरस्वतीकालः -
(1678 A.D.) काले आसीदिति हरिलीलाविवेकभूमिकायां दृश्यते । रामतीर्थष्षोडशशतकीयः केचित्पदेन संक्षेपशारीरकव्याख्याने अनूदित इति मधुसूदनसरस्वती सप्तदशशतकापरार्धकालिक इति (J. O. R. Vol II Page 101) उपवर्णितम् । ईश्वरचन्द्रमहाशयैः मधुसूदनसरस्वतीकाल (1532-1620 A.D.) इति प्रतिपाद्यते । महाविद्याविडम्बनाख्ये ग्रन्थे तिलाङ्गमहाशयैः सप्तदशशतकीय इति प्रतिपाद्यते । गोपीनाथकविराजमहाशयेन स्वीयसंस्कृतहस्तलिखितग्रन्थसूचीभूमिकायां (1918-1919) मधुसूदनसरस्वती षोडशशतकापरार्धकालिक इति प्रतिपाद्यते । दिवानजीमहाशयस्तु (A. B. O. R. I. Vol VIII Part II) षोऽशशतकापरार्धकालिकं वदति । सर्वथा (1565-1665 A.D.) काले आसीदिति निश्चयः । मधुसूदनसरस्वत्याः जीवनकालः पञ्चाधिकशततमवयःपर्यन्तमिति (105) वाराणस्यामेवास्य ब्रह्मलोकवासस्समजनीति च सम्प्रदायः ।
मधुसूदनसरस्वतीग्रन्थाः-
१. अद्वैतरत्नरक्षणम् (N. S. P.)
शङ्करमिश्रकृतभेदरत्नस्य खण्डनात्मक अद्वैतसिद्धिसारभूतश्चायं ग्रन्थः निर्णयसागरमुद्रणालये मुद्रितः ।
२. अद्वैतसिद्धिः (N. S. P., Mysore S. S. 75)
व्यासतीर्थकृतन्यायामृताख्यास्य द्वैतपरग्रन्थस्य नव्यनैय्यायिकसरण्या खण्डनपरोऽयं महान् ग्रन्थः निर्णयसागरमुद्रणालये मैसूरग्रन्थमालायाञ्च मुद्रितः । अस्य व्याख्याः गौडब्रह्मानन्दकृता गुरुचन्द्रिका, लघुचन्द्रिका, सदासुखकृता शरच्चन्द्रिका, बलभद्रकृता अद्वैतचन्द्रिका, पुरुषोत्तमकृतः अद्वैतसिद्धिसाधकः, अज्ञातकर्तृनामकव्याख्याद्वयञ्च वर्तन्ते ।
%%% Chart
३. आत्मबोधव्याख्या 526 पञ्चाबसूच्यां दृश्यते ।
४. ईश्वरप्रतिपत्तिप्रकाशः (TSS. 73)
तनीयान् ग्रन्थोऽयं ईश्वरं प्रति याः वादिविप्रतिपत्तयः ताः निराकृत्य युक्त्या सम्प्रतिपत्तिं प्रकाशयन् अद्वैतमतवर्णनेन सामापयति । ग्रन्थोऽयं तिरुवनन्तपुर मालायां मुद्रितः ।
५. गृढार्थदीपिका (N.S.P.)
भगवद्गीताव्याख्यात्मकोऽयं ग्रन्थः निर्णयसागरे मुद्रितः । व्याख्यामेनामनु सृत्य धर्मदत्तबज्जाशर्मणापि काचन व्याख्या लिखिता ।
६. प्रस्थानभेदः - (ASS 5)
सर्वदर्शनसंग्रहवत् ग्रन्थोऽयमपि विभिन्नपस्थानान्युपपाद्य शाङ्करप्रस्थाने वैशिष्टयं प्रतिपादयन् सर्वेषा प्रस्थानानां साक्षात् परम्परया वा अद्वितीये परमात्मनि तात्पर्यं निरूपयति । आनन्दाश्रममुद्रणालये मुद्रितश्च ।
७. वेदान्तकल्पलतिका - (S.B.S. 3)
ग्रन्थेऽस्मिन् मोक्षशास्त्रापरपर्यायस्य वेदान्तदर्शनस्योपेयतया व्यवस्थितस्य मोक्षपदार्थस्य स्वरूपनिर्णयाय तत्साधनादिप्रदर्शनाय च विभिन्नवाद्यभ्युपगतमोक्षतत्साधनादिस्वरूप्रदर्शनपूर्वकं निराकरणीया अंशा निराकृताः । मुद्रितश्चायं सरस्वतीभवनग्रन्थमालायाम् ।
८. शारीरकसारसंग्रहः 
संक्षेपशारीरकव्याख्यात्मकोऽयं ग्रन्थः चौखाम्बामुद्रणालये मुद्रितः । विश्ववेद प्रत्यग्विष्णुकृते संक्षेपशारीकव्याख्येऽत्र निर्दिष्टे ।
९. सिद्धान्तबिन्दुः - (BSS. 65 AMS. 3)
शाङ्करदशश्लोकीव्याख्यात्मकोऽयं ग्रन्थः उपोद्धात-त्वम्पादार्थतत्पदार्थतत्वमसि निर्णयविभागभेदेन चतुर्भिः परिच्छेदैः पूर्णः मुद्रितश्च । अस्य व्याख्याः तारानाथकृतः सिद्धान्तबिन्दुसारः, नारायणीया लघुटीका, पुरुषोत्तमकृतं सन्दीपनम्, पूर्णानन्दीयः तत्वविवेकः, ब्रह्मानन्दीया न्यायरत्नावली, सच्चिदानन्दशिवलालकृता व्याख्या च विद्यन्ते ।
%%% Chart 
१०. सिद्धान्तलेशसंग्रहव्याख्या ?
ग्रन्थोऽयं हरिलीलाविवेकभूमिकायां निर्दिष्टः । P. P. शास्त्रिणश्च सरस्वतीमहालयवर्णनात्मकहस्तलिखितग्रन्थसूच्याः त्रयोदशमे भागे (7535 DC. T. S. M. L Vol XIII) निर्दिशन्ति ।
अन्येऽपि आनन्दमन्दाकिनी - भक्तिरसायन - हरिलीलाविवेकव्याख्याद्याः बहवः ग्रन्था अनेन कृताः । जटाद्यष्टाविकृतिविवृतिरिति ग्रन्थस्तु नानेन कृत इति ज्ञायते ।
५७. गौड-ब्रह्मानन्दसरस्वती (1600-1700 A.D.)
गद्यमयशारीरकमीमांसाभाष्यवार्तिककर्तुः नारायणतीर्थस्य परमानन्दतीर्थस्य च शिष्यः शिवरामानन्दगोपालानन्दयोः रत्नप्रभाकारगोविन्दानन्दसरस्वत्याः प्रशिष्यः, बालकृष्णानन्दसरस्वतीत्र्यम्बकभट्टरामचन्द्रानन्दसरस्वतीनां गुरुरयं गौडब्रह्मानन्दसरस्वती सप्तदशशतकीयः । अस्य गुरुणा नारायणतीर्थेन सिद्धान्त बिन्दोः व्याख्या कृता इति (7540 DC. T.S.M.L. Vol XIII) ग्रन्थे P. P. शास्त्रिणः वदन्ति । अयमेव वाक्यमुधाव्याख्याता ब्रह्मानन्दसरस्वती । अस्य गुरुः परमानन्दतीर्थ एव आनन्दगुरुशब्देन निर्दिष्ट इति केचित् ।
%%% Chart
काञ्चीमण्डलान्तर्गतशिवरामानन्दात् प्राप्ताद्वैतविद्य इत्यपि केचित् ।
१. अद्वैतसिद्धान्तविद्योतनम् (S. B. S. S. 51)
ग्रन्थोऽयं सप्तविधासु ख्यातिषु अनिर्वचनीयख्यातिरेवाद्वैतिनां परमस्सिद्धान्त इति प्रतिपादयति । स चायमनिर्वचनीयख्यातिवादः भाष्यभामतीखण्डनखण्डखाद्यादिषूपवर्णितोऽपि साकल्येनास्मिन्नेव ग्रन्थे वर्णित इति तु विशेषः । ग्रन्थोऽयं सरस्वतीभवनग्रन्थमालायां मुद्रितः ।
२. ईशावास्यरहस्यम् (A. S. S. 5)
३. गुरुचन्द्रिका (Mysore S. S. 75)
बृहद्ब्रम्नानन्दीयापरनामायं ग्रन्थः मधुसूदनसरस्वतीकृतायाः अद्वैतसिद्धेर्व्याख्यात्मकः । मधुसूदनसरस्वतीकृता अद्वैतसिद्धिः व्यासतीर्थकृतन्यायामृत दूषणाय प्रवृत्ता । अद्वैतसिद्धिदूषणाय रामाचार्येण न्यायामृततरङ्गिणी कृता । न्यायामृततरङ्गिणीदूषणाय अद्वैतसिद्धिव्याख्याव्याजेन ब्रह्मानन्दसरस्वत्या गुरुचन्द्रिका कृता । ग्रन्थोऽयं मैसूर संस्कृतग्रन्थमालायां मुद्रितः । 
४. न्यायरत्नावली (B. S. S. 65)
मधुसूदनसरस्वतीकृतस्य सिद्धान्तबिन्दोर्व्याख्यात्मकोऽयं ग्रन्थः वाराणसीग्रन्थमालायामद्वैतमञ्जरीग्रन्थमालायाञ्च मुद्रितः ।
५. लघुचन्द्रिका (N. S. P.)
गुरुचन्द्रिकासंग्राहकोऽयं ग्रन्थः अद्वैतसिद्धिव्याख्यात्मकः मुद्रितः निर्णयसागरमुद्रणालये । अस्य व्याख्या विट्ठलेशमिश्रकृता । विठ्ठलेशीया । शारदोल्लासाख्याऽपरा काचन व्याख्या लाहूरपण्डितराधाकृष्णसूच्यां दृश्यते ।
६. वेदान्तसूत्रमुक्तावलिः (A. S. S 77)
सूत्रवृत्तिरूपोऽयं ग्रन्थ आनन्दाश्रमे मुद्रितः । अस्यां निर्णयदर्पणाख्यः ग्रन्थः निर्दिष्टः । अस्य व्याख्या तत्त्वार्थविबोधनाख्या रामसुब्रह्मण्यशास्त्रिकृता हलषूसूच्यां (H. Z. 1542) दृश्यते ।
७. मीमांसाचन्द्रिका (259 S. B. D.)
सरस्वतीभवनामुद्रितग्रन्थसूच्यां दृश्यते । मुण्डकोपनिषद्रहस्यं, ईशावास्यश्लोकार्थः, मीमांसारहस्याख्याश्च ग्रन्थाः कृता इति श्रूयते ।
५८. अखण्डानन्दसरस्वती (1600-1700 A. D.)
अस्याखण्डानन्दसरस्वत्याः पूर्वाश्रमे रङ्गनाथ इति नाम । आन्ध्रदेशीयोऽयं अद्वैतरत्नकोशव्याख्यायाः, भावप्रकाशिकायाः भेदधिक्कारविवृतेश्च कर्तुः कालहस्तीशयज्वनः यज्ञाम्बानाम्न्याश्च पुत्रः । नलगन्तुवंशजोऽयं स्वयम्प्रकाशानन्दस्य शिष्यः । अस्याश्रयदाता सामयिकश्च इम्मिडिजगदेकरायः (जगदेकरायद्वितीयः) । अस्य शासनकालः क्रैस्तवियषोडशशतकात् सप्तदशशतकमिति  S. कृष्णस्वाम्यय्यङ्गारलिखितविजयनगरचरित्रग्रन्थात् कर्णाटकदेशान्तर्गतशिलाशासनप्रतिपादक ग्रन्थस्य प्रथमभागात् भूमिका आफ़ यफि़ग्राफि़का कर्णाटिका आफ मैसूर (Part I Page 27-28) च ज्ञायते ।
तत्वदीपनकाराद्भिन्नोऽयं अखण्डानन्द इति तत्वदीपनकाराखण्डानन्दसरस्वतीप्रतावे प्रतिपादितम् ।
१. ऋजुप्रकाशिका भामतीव्याख्या (C.S.S. 1)
ग्रन्थस्यास्य शैली सरलतमा सुबोधाऽर्थपूर्णा च । ग्रन्थेऽस्मिन् अद्वैतरत्नकोशप्रकाशिका निर्दिष्टा । ग्रन्थोऽयं चौखाम्बामुद्रणालयेऽपि मुद्रितः ।
२. रत्नकोशप्रकाशिका (भावप्रकाशिका) 47 शृङ्गगिरिसूची 
ग्रन्थोऽयममुद्रितः मैसूरपुस्तकालये शृङ्गगिरिसृच्याञ्च दृश्यते । ग्रन्थोऽयं नृसिम्हाश्रमीयतत्वेवेकदीपनव्याख्यात्मकः । अस्यैव अद्वैतरन्तकोशदीपिका इत्यापि नामान्तरम् ।
तर्कभाषाप्रकाशः, तर्कभाषातत्वबोधिनीव्याख्या, न्यायसिद्धान्तदीपिका अपि अदसीयन्यायशास्त्रीयाः ग्रन्थाः मैसूरपुस्तकालये लभ्यन्ते ।
५९. अद्वैतानन्दबोधेन्द्रः (1700 A. D.)
अस्य पूर्वाश्रमे सीतापतिरिति नाम । अस्य पिता प्रेमनाथदीक्षितः । माता पार्वती । दक्षिणदेशे पञ्चनदक्षेत्रे जन्म इति ``प्रेमेशस्य पिनाकिनी तटभुवः सृनुस्स सीतापति" इति पुण्यश्लोकमञ्जर्यां तथा ``जप्येशस्य कृपाभरात् समुदभूत् यः पार्वतीगर्भजः" इति एतदीयशान्तिविवरणे च दर्शनात् ज्ञायते ।
एष किलाद्वैतान्धबोधेन्द्रः काञ्चीकामकोटिपीठाधिपतिभिः भूमानन्दापराभिधानचन्द्रशेखरेन्द्रसरस्वतीभिः दीक्षित इति ``भूमानन्दपराभिधानविलसच्छ्रीचन्द्रचूडाश्रमिप्रेक्षावाप्तसमस्तवित्पदतया श्रीकामकोटीश्वरः ।" इति ``विद्यांददौ चिन्मर्यी भूमानन्दपदस्थितो गुरुरसौ" इति शान्तिविवरणे (Page 219) दर्शनात् ज्ञायते । अस्य विद्यागुरू रामानन्दसरस्वतीति ``रामानन्दमुनिं मुनीन्द्रनिकरैरासेवितं सर्वदा" इति ब्रह्मविद्याभरणे P. 1 दर्शनान्निश्चीयते । अयमेव ``चिद्विलासः" आनन्दबोधाचार्य इत्यादिभिर्नामभिर्व्यपदिष्ट इति सदाशिवब्रह्मेन्द्रकृतगुरु रत्नमालिकायां ``कलये हृदि संश्रितं स्वभासा विलयं चिद्वियतीह चिद्विलासम्" इति दर्शनाज् ज्ञायते ।
अस्य कालः कल्यब्देषु (4268-4301-1168-1201 A. D.) इति सम्प्रदायविदां मतम् । श्रीहर्षाभिनवगुप्तादिसामयिकश्चेति केचित् ।
परन्तु अद्वैतानन्दबोधकालः सप्तदशशतकस्य आदिमः भाग इति विमर्शकसिद्धान्तः । अत्रैषा युक्तिः -
अद्वैतानन्दसरस्वती अद्वैतवाणी अद्वैतानन्दवोधेन्द्र एते त्रयोऽपि एक एव । सप्तदशशतकावसानसामयिकेन अच्युतकृष्णान्दतीर्थेन स्वीये भाष्यरत्नप्रभाव्याख्याने ``अद्वैतानन्दवाण्याख्यं तं वन्दे शमवारिधिम्" इति अद्वैतानन्दवाण्यै नाम आवेदितम् । अच्युतकृष्णानन्दगुरुस्स्वयम्प्रकाशानन्दः । स्वयम्प्रकाशगुरुरद्वैतानन्दवाणीति युक्तमेव प्रगुरोर्नमस्काराविष्करणम् । अद्वैतानन्दसरस्वत्याः गुरु रामानान्दसरस्वती । रामानन्दश्च गोविन्दानन्दशिष्यः । नारायणतीर्थ रघुनाथ तीर्थसतीर्थ्यश्च । नारायणतीर्थेन पञ्चदशशतके (1592 A. D.) शारीरकाभाष्यवार्तिकं कृतमिति अच्युतकृष्णानन्दनाराणसरस्वत्योः मध्यमः काल एव अद्वैतानन्दबोधस्य, स च सम्भाव्यते सप्तदशशतकस्यादिमः भाग इति ।
%%% Chart
१. ब्रह्मविद्याभरणम् (A. M. S. 6)
ब्रह्मविद्याभरणाख्योऽयं ब्रह्मसूत्रशाङ्करभाष्यव्याख्यात्मकः समग्रचतुरध्यायी भामतीभावप्रकाशनपर अद्वैतमतस्याभरणमेव । ग्रन्थोऽयमद्वैतमञ्जरीग्रन्थमालायां मुद्रितः ।
शान्तिविवरणम् गुरुप्रदीपः, उभावपि एतद्विरचितौ ग्रन्थाविति ज्ञायते ॥ अध्यात्मचन्द्रिका आत्मबोधटीकापि अनेन कृताविति वदन्ति ।
६०. कृष्णानन्दसरस्वती (1600-1700 A.D.)
दक्षिणदेशीयस्यास्य कृष्णानन्दसरस्वात्याः पूर्वाश्रमे पितुर्नाम विश्वनाथः । माता अंम्बा ``नाम्नीति अडयार पुस्तकालयस्थसिद्धान्तसिद्धाञ्जनहस्तलिखितपुस्तकाजू ज्ञायते । स्वग्रन्थे जागर्ति त्रिजगद्गुरुर्मनसि नः श्रीरामभद्रो यमी" इति रामभद्रयमिनं ``तं पर्येमि प्रणतिभिरहं वासुदेवं यतीन्द्रम्" इति वासुदेवेन्द्रञ्च नमस्कुर्वन्नयं रामभद्रवासुदेवयोश्शिष्य इति निर्णीयते । अयं प्रकाशानन्दसरस्वत्याः प्रशिष्यश्चेति ``स्वयम्प्रकाशं परमगुरुं नौमि योगीन्द्रम् इति ग्रन्थादेव ज्ञायते । अस्य शिष्येषु भास्करदीक्षित-रामनन्दसरस्वती-रघुनाथसूरि-प्रज्ञानाश्रमी-अनु-भवानन्द अय्याध्वरिणः विशिष्टाः प्रसिद्धतराश्च अनेनैव सिद्धान्तसिद्धाञ्जने पस्वाचार्यरम्परा निर्दिश्यते -
%%% Chart
शाहजीप्रथमेन तञ्जपुरशासकेन मानितोऽयं कृष्णानन्दसरस्वतीति ग्रन्थादेव ज्ञायते । शाहजीप्रथमकालस्तु (1684-1711 A.D.) इति निश्चयः । रामानन्दाख्येनास्य शिष्येण (1670 A.D.) काले प्रतिलेखः कृत इति प्रतिलेखकस्य स्वस्याचार्यः मूलग्रन्थकार इति ग्रन्थत एव प्रतिपादितम् । एवञ्जास्यकालस्सप्तदशशतकापरार्धादारब्ध इति निर्विवादोऽयं विषयः ।
१. कृष्णालङ्काः (B. S. S. 36)
सिद्धान्तलेशसंग्रहव्याख्यात्मकोऽयं ग्रन्थः चौखाम्बामुद्रणालये अद्वैतमञ्जरीग्रन्थमालायाञ्च मुद्रितः ।
२. सिद्धान्तसिद्धाञ्जनम् (T. S. S. 47)
विशिष्टाद्वैतदूषणपरोऽयं ग्रन्थः विविधान् भिन्नाचार्योपज्ञान् वेदान्तसिद्धान्तान् परिशीलयन् शाङ्कराद्वैतसिद्धान्तं साधयति । ग्रन्थोऽयं तिरुवनन्तपुरसंस्कृतग्रन्थमालायां मुद्रितः । अस्य व्याख्या भास्करदीक्षितकृता रत्नतूलिकानाम्नी ।
३. अनुष्ठानपद्धतिः, ४. शिवतत्वमालिका, ५. वेदान्तवादार्थ (7516 T. S. M. L.) इमे ग्रन्थास्सरस्वतीमहालये विद्यन्ते । ६. प्रणवनिर्णयप्रकाशिका तुातिरुवनन्तपुरपुस्तकाले लभ्यते । ७. शास्त्रनिर्णयनामा ग्रन्थः मैसूरपुस्तकालये लभ्यते ।
६१. कृष्णानन्दसरस्वती (1600-1700 A.D.)
``श्रीनारायणरूपांस्तानखण्डानन्ददेशिकान् । नत्वा" इत्यादिना अखण्डानन्दसरस्वतीन् नमस्कुर्वन्नयं कृष्णानन्दसरस्वती अखण्डानन्दशिष्य इति निश्चीयते ।
अथ कोऽयमखण्डानन्दसरस्वती ? किं तत्वदीपनकारः ? उत षोडशसप्तदश शतकोत्तरार्धपूर्वार्धर्कालिकः ऋजुप्रकाशइकाकार अखण्डानन्दसरस्वती ? । त्रिपेदी महाशयस्तु स्वसम्पादिततर्कसंग्रहभूमिकायां तत्वदीपनऋजुप्रकाशिकाकारौ उभावपि अखण्डानन्दौ न भिन्नाविति साधयति । एवञ्च तन्मतेन स एवास्य गुरुरिति सिध्यति । श्रीकण्ठशास्त्री तु स्वीये (I. H. Q. XIV) लेखे तत्वदीपनकारादखण्डानन्दात् ऋजुप्रक'शिकाकारं भिन्नं वर्णयति । एवं सति कृष्णानन्दयतिरयं सप्तददशशतकीय इति सिध्यति । अन्ये तु अन्योऽयमखण्डानन्द इति वदन्ति ।
अनेन कृष्णानन्दसरस्वत्या स्वीये ग्रन्थे 104, 92, पुटे विद्यारण्य उद्धृतः । विद्यारण्यस्य कालस्तु (1294-1384 A. D.) इति निरूपितम् । आनन्दगिरिस्तु विद्यारण्यात् पूर्वतनः । यद्येवं आनन्दगिरिप्रशिष्योऽयं कृष्णानन्दसरस्वती विद्यारण्यस्यापरे वयसि सामयिक इति वक्तव्यं भवति । आनन्दगिरिप्रशिष्योऽयं कृष्णानन्दः गुजरातदेशवासी । एवं च गुजरातदेशपर्यन्तं विद्यारण्यस्य कृतेः प्रसिद्धिः विद्यारण्यकाल एवाभ्युपेयो भवति । प्रायः ग्रन्थकर्तृणां ग्रन्थस्य च प्रसिद्धिः । कालविलम्बमर्हति । यद्येवं तर्हि विद्यारण्यं प्रमाणीकुर्वन्नयं न तत्वदीपनकारशिष्य इति वक्तव्यं भवेत् । ऋजुप्रकाशिकाकार शिष्य इत्यभ्युपगमे तु शतकद्वये विद्यारण्यस्य ग्रन्थानां महती प्रसिद्धिस्सम्भाव्यते ।
१. ब्रह्मतत्वसुबोधनी - (33. F. 4. AL)
महावाक्यानि दहरविद्यां, जीवन्मुक्तस्वरूपं अध्यारोपापवादौ च सपरिकरं विचारयन्नयं प्रकरणग्रन्थ अमुद्रितः पूर्व अडयारमद्रासपुस्तकालयपञ्चाबसूच्यादिषु लभ्यते ।
६२. बालकृष्णानन्दसरस्वती (1600-1700 A.D.)
``श्रुतिनगराधीशेन अभिनवद्रविडार्यबालकृष्णेन । रचितेयं मोदयतात् कैलासेशंं कृतिर्महादेवम् ।" इति ग्रन्थे दर्शनात्, श्रुतिनगरस्य वेदपुरीति प्रसिध्या च काञ्चीमण्डलान्तर्गत वेदपुरीवासीति ज्ञायते । अभिनवद्रविडाचार्य इत्यस्य नामान्तरमपि द्रविडदेशत्वमस्य द्रढयति । 
``सूत्रभाष्यकृदनुग्रहपक्षलम्बनादमृतमान्तरमाप्तुम् ।
वैनतेन इव पुष्करदेशं जन्मयोगमतरं जलराशिम् ॥"
इति कथनात् पुष्करदेशेऽनेाश्रमस्वीकार कृतः इति निश्चीयते ।
प्राचीना हि द्रविडाचार्याः शङ्करात् प्राचीनाश्छान्दोग्योपनिषदां बृहदारण्यकोपनिषदाञ्च वाक्याभिधस्य ब्रह्मनन्दिकृतस्य व्याख्यानस्य भाष्यकृत्वेन प्रसिद्धाः । अभिनवद्रविडाचार्यस्तु स्वग्रन्थमहिम्ना स्वविद्यया अध्यापनसामर्थ्येन अनुष्ठाननिष्ठया च पूजितः गौडैः परमहंसैः अभिनवद्रविडाचार्य शब्देनेति ज्ञायते । सिद्धान्तवर्णनेनापि अयमभिनवत्वमर्हति । तमसः चक्षुःप्रभारूपतया वर्णनम्, आकाशसरूपत्वाप्रत्यक्षत्वादिवर्णनञ्चास्याभिनवसिद्धान्तः । तस्मादिदमपि अभिनवद्रविडाचार्यत्वप्रसिद्धौ मुख्यं निदानं भवितुमर्हति ।
साहित्यशास्त्रेऽस्य गुरुर्वेङ्कटकविः । व्याकरणे नागेशभट्टः । ज्योतिश्शास्त्रे स्वयम्प्रकाशतीर्थस्तिप्पन्नयज्वा च । न्यायशास्त्रे सिद्धान्तसिद्धाञ्जनकर्तुः गुरुर्वासुदेवान्दसरस्वती । वेदान्तेऽद्वैतसिद्धिव्याख्याता गौडब्रह्मानन्दसरस्वती । दीक्षागुरुः श्रीधरानन्दसरस्वती । अनेन पूर्णानन्दारब्धा श्रीधरान्ता च गुरु परम्परा निर्दिश्यतेऽत्र ग्रन्थे -
%%% Chart
महादेवः श्रीमान् समधिगततत्वो वटुवरः 
महाकैलासेशः सकलयतिसंसेनवनपरः ।
विशेषं जानीतां विशदवचनैर्भाष्यवचसां
इतीयं सञ्जाता जगति कृतिरग्र्या विजयताम् ॥
इति ग्रन्थे दर्शनात् महादेवकैलासेशाख्यौ द्वावस्य शिष्याविति ज्ञायते ।
अथास्य कालः-
ब्रह्मानन्दसरस्वतीकालश्च नागेशभट्टकालश्च सप्तदशशतकमिति निर्णयः । गौडब्रह्मानन्दश्च नारायणसरस्वतीशिष्यः । नारायणसरस्वत्या च गद्यमयं व्रह्मसूत्र भाष्यवार्तिकं ``चत्वार्यब्दसहस्राणि तथा रसशतानि च । नवतिस्त्रीणयथाब्दस्य व्यतीतानि यथा कलेः ॥" इति 4693-1670 श 1592 A.D, काले कृतमिति निर्दिश्यते । एवञ्च तत्प्रशिप्योऽयं बालकृष्णानन्दरसप्तदशशतकीय एव भवितुमर्हति ।
अभिनवद्रविडाचार्यापरनामायं बालकृष्णानन्दः श्रीधरानन्दशिष्यः सिद्दान्तसिद्धाञ्जनकारकृष्णानन्दसायिकः, लघुचन्द्रिकाकारब्रह्मानन्दसरस्वतीशिष्यः श्रुतिनगरापराभिधवेदपुरी(तिरुवेत्तियूर)ग्रामजस्सप्तदशशतकीय इति च साम्प्रदायिकाः । वेदपुर्यपरापराभिधोऽयं तिरुवेत्तियूर्ग्रामः काञ्चीनगरस्य षोडशयोजने वर्तमानः नार्तार्काड (उत्तरार्काड)न्तर्गत इति प्रसिद्धिः ।
१. शारीरकमीमांसाभाष्यवार्तिकम् (A. S. 1)
उक्तानुक्तदुरुक्तचिन्तनं हि वार्तिकम् । तच्च चिन्तनं सूत्रविवरणमपि भवति । भाष्यविवरणमपि । सिद्धान्तलेशसंग्रहादिषूपवर्णितान् प्रक्रियाविशेषान् भाष्यारोपणप्रयत्नं कुर्वाणोऽयं बालकृष्णानन्दः वार्तिकलक्षणाङ्गभूतस्थचिन्तनस्य अनुरूपं ग्रन्थं प्रतिपादयति । भामतीकाराणां अवच्छेदवादः, विवरणकाराणां ईश्वरभावापत्तिरेव मुक्तिरिति मतम्, एवमन्येऽपि विषयविशेषाः प्रतिपादिताः । ग्रन्थकारोऽयं वार्तिकमतानुसारी सुरेश्वरविश्वरूपमण्डनमिश्राणां ऐक्यमङ्गीकरोति । ग्रन्थोऽयमाशुतोषग्रन्थमालायां मुद्रितः ।
२. तैत्तरीयोपनिषद्विवरणम् (भाष्यविवरणम्) (R. 383 MGOML) लन्दनपुस्तकालये बाम्बेरायलासियाटिकपुस्तकालये च लभ्यते । ईशकेनकठछान्दोग्य प्रश्नोपनिषदां व्याख्याऽपि अनेन कृता इति ज्ञायते ।
शिक्षादीक्षादिगुरुपरम्परा
%%% Chart
६३. सदानन्दकाष्मीरी (1650-1750 A.D.)
काश्मीरदेशाभिजनोऽयं सदानन्दः ब्रह्मानन्दसरस्वत्याश्शिष्यः वेदान्तसारकर्तुस्सदानन्दत् भिन्नश्चेति ज्ञायते । अनेन कृताया अद्वैतब्रह्मसिद्धेः कस्मिंश्चित् हस्तलिखितग्रन्थे 1761 संवत्सरे ग्रन्थसमाप्तिरिति दृश्यते । एवञ्चास्य काल अष्टादशशतकस्य पूर्वार्धार्धावधिक इति ज्ञायते ।
अनेन कृतायां अद्वैतब्रह्मसिद्धौ अद्वैतमतविरुद्धानां आस्तिकनास्तिकानां बहूनि मतानि खण्डितानि । परन्तु वल्लभोपज्ञं मतं न खण्डितम् । तस्माद्विशुद्धाद्वैतमतप्रवर्तकात् पञ्चदशशतकीयात् (1479 A.D.) वल्लभाचार्यादयं पूर्वतन इति केचिद्वदन्ति ।
१. अद्वैतब्रह्मसिद्धिः (C.U.)
षण्णामास्तिकानां नास्तिकानाञ्च दर्शनानां संक्षेपत उपन्यासपूर्वकं तत्खण्डनम् , अद्वैतसिद्धान्तस्याबाधितत्वञ्च मुद्गरप्रहाराख्येषु चतुर्षु परिच्छेदेषु वर्णितम् । मुद्रितश्चायं ग्रन्थः कल्कत्ताविश्विविद्यालये ।
२. स्वरूपप्रकाशः । ग्रन्थोऽयं अद्वैतब्रह्मसिद्धौ निर्दिष्टः ।
६४. अच्युतकृष्णानन्दतीर्थः (1650-1750 A.D.)
दाक्षिणात्योऽयमच्युतकृष्णान्दः स्वयम्प्रकाश-अद्वैतानन्दसरस्वत्योश्शिष्यः । अद्वैतवाणीति अद्वैतानन्दसरस्वत्या एव नामान्तरमिति अनेन भाष्यरत्नप्रभाव्याख्याने व्यपदिश्यते । अद्वैतानन्दश्च रामानन्दस्य शिष्यः । एवञ्चाद्वैतवाणीं नमस्कुर्वाणोऽयं अच्युतकृष्णः रामानन्दप्रशिष्यः स्वयम्प्रकाशाद्वैतानन्दयोः शिष्य इति सिध्यति । स्वयम्प्रकाशानन्दश्च अद्वैतानन्दनामपि शिष्याः । अस्य कालस्सप्तदशशतकापरार्धादिमः अष्टादशशतकपूर्वार्धश्चेति ज्ञायते ।
१. कठोपनिषच्छाङ्करभाष्यटीका 1278 ग्र 22 प (GOML Mysore)
२. कृष्णालङ्कारः (A. M. S. 5)
सिद्धान्तलेशसंग्रहव्याख्यात्मकोऽयं ग्रन्थ अद्वैतमञ्जरीग्रन्थमालायां कुम्भघोणे मुद्रितः ।
३. वनमाल । तैत्तरीयोपनिषछाङ्करभाष्यव्याख्यात्मकोऽयं ग्रन्थः वाणीविलासे मुद्रितः ।
४. भावदीपिका-भामतीव्याख्या (39 E. 9 A. L.)
कल्पतरुपरिमलसंग्रहरूपोऽयं ग्रन्थ अमुद्रित अडयारपुस्तकालयस्थः ।
५. भाष्यरत्नप्रभाभागव्याख्या (R. 2782 MGOML)
जिज्ञासासूत्रमारभ्य आनन्दमयाधिकरणपर्यन्तेयं रत्नप्रभाव्याख्या अमुद्रिता मद्रासराजकीयपुस्तकालये अडयारपुस्तकालये च (26 M. 20 A. L.) लभ्यते । विश्वभारतीशान्तिनिकेतनपुस्तकालये अनन्तशयनपुस्तकालये लभ्यते ।
६. मानमाला-
प्रमाणप्रमेयप्रमाप्रमातृनामभिः प्रकरणैः पदार्थानां भेदनिर्वचनपूर्वकं प्रमाणस्वभावं वर्णयन्नयं प्रकरणग्रन्थ अडयार पुस्तिकामालायां मुद्रितः । अस्य व्याख्या रामानन्दभिक्षुविरचिता विवरणाख्या च ।
६५. अय्यण्णादीक्षितः (1700-1800 A.D.)
दाक्षिणात्योऽयं द्रविडदेशवासी अय्यण्णादीक्षितः स्वग्रन्थे ``तस्मै श्रीधरवेङ्कटेशगुरवे कुर्वे प्रणामायुतम् ।" इति श्रीधरवेङ्कटेशन्नमस्करोति । एवञ्च तिरुविशनल्लूराख्यचोलदेशीयग्रामवासिन अय्यावालिति प्रख्यातस्य श्रीधरवेंकटार्यस्य शिष्यः, राधामाधवसंवादकर्तुः वेंकटपतेः पुत्रः, सुदन्ताकल्याणकर्तुः नरहरिसूनोः भ्रातृव्यः सदाशिवब्रह्मेन्द्रसामयिकः क्रैस्वतीयाष्टादशशतकीय इति ज्ञायते ।
१. व्यासतात्पर्यनिर्णयः (V.V.P.)
शाङ्करभाष्यवर्णितं ब्रह्मसूत्रार्थं व्यासाभिमतं वर्णयन् परिच्छेदद्वयवानयं ग्रन्थः अद्वैत एव व्यासाभिमतं युक्तिभिः प्रमाणैश्च साधयति । मुद्रितश्चायं वाणीविलासमुद्रणालये ।
२. अद्वैतमतप्रकाशः (त्रिमतैक्यप्रकाशान्तर्गप्तः)
३. जीवन्मुक्तिविवेकः 
४. आत्मानात्मविवेकश्च अनेन कृता इति ज्ञायते । व्यासतात्पर्यनिर्णयस्य आधुनिकेन नरकण्ठीरवाख्यशास्त्रिणा कृता च व्याख्या वर्तते ।
६६. सदाशिवब्रह्मेन्द्रसरस्वती (1700-1800 A.D.)
योगशक्तिसम्पन्नस्य महतस्मिद्धस्यास्य सदाशिवब्रह्मेन्द्रस्य पूर्वश्रमे शिवरामकृष्ण इति नाम । मधुरानगरवासिनस्सोमनाथावधानिनः पार्वत्याश्च पुत्रोऽयम् । प्रसिद्धरामभद्रदीक्षितसतीर्थ्योऽयं आश्रमस्वीकारादनन्तरं सदाशिवब्रह्मेन्द्र इति प्रसिद्धिंगतः । दहरविद्याप्रकाशिकाकर्तुः परमशिवेन्द्रस्य शिष्यः अद्वैतरसमञ्जरीकर्तुर्नल्लादीक्षितस्य गुरुः, तिरुविशनल्लूर अय्यावालिति प्रसिद्धस्य श्रीधरवेङ्कटेशशास्त्रिणः, महाभाष्यं गोपालकृष्णशास्त्रिणश्च सतीर्थ्यः, शरभोजिप्रथमस्य सामयिकः अष्टादशशतकीय इति निर्णीयते । अखण्डकावेरीतिरस्थिते त्रिशिरपुर्यन्तर्गते नेरूरग्रामे अयं समाधिनिष्ठश्च । सदाशिवब्रह्मेन्द्रोयं नवसालपुरराजाय विजयरघुनाथतोण्डमानाख्याय मन्त्रानुपदिदेश । तत आरभ्यैव प्रतिनवरात्रं पण्डितपरिषत्प्रचलति इति च वदन्ति ।
कृष्णमाचार्यकृते संस्कृतसाहित्येतिहासे तु सदाशिवब्रह्मेन्द्रकालः षोडशशतकीय इति दृश्यते ।
गुरुशिष्यपरम्परा
%%% Chart
१. आत्मविद्याविलासः (VVP)
द्विषष्ठिभिः पद्यैः पूर्णोऽयं ग्रन्थ आत्मज्ञानानुभवं प्रदर्शयति । आर्याच्छन्दोघटितैः पद्यैः पूर्णोऽयं ग्रन्थः वाणीविलासमुद्रणालये मुद्रितः । ग्रन्थोऽयं शरभोजीप्रथमस्यानुग्रहार्थं मलहारीपण्डितप्रार्थनावशात् सदाशिवब्रह्मेन्द्रेण रचित इति सरस्वतीमहालयस्थात् आत्मविद्याविलासपुस्तकात् (7685 TSML) ज्ञायते ।
२. आत्मानात्मविवेकप्रकाशिका (23. C. 15. AL)
अमुद्रितोऽयं ग्रन्थ अडयारपुस्तकालये लभ्यते । ग्रन्थेऽस्मिन् ``अथाद्वैतविज्ञानेऽस्मत्परमगुरु सदाशिबब्रह्मेन्द्र कृते आत्मानात्मविवेकप्रकाशिकार्या उदाह्रियन्ते" इति दर्शनात् आत्मानात्मविवेक अस्य प्रसिद्धसदाशिवब्रह्मेन्द्रस्य परमगुरुणा सदाशिवेन्द्रेण कृतस्स्यादिति ज्ञायते । अस्य व्याख्या प्रशिष्येणानेन कृता ।
३. आत्मानुसन्धानम् । मुद्रितश्चायं सरस्वतीमहालये (7746 DC TSML Vol. XIII)
४. कैवल्योपनिषद्दीपिका (R. 1492 D. MGOML)
५. नवमणिमाला । मुद्रितश्चायं (7734 DC T.S. M. L. Vol XIII)
६. ब्रह्मसूत्रभाष्यसिद्धान्तसङ्ग्रहः - (ALPS 19)
बह्मसूत्रसिद्धान्तविवृतिरित्यपरनामायं ग्रन्थः शङ्कराचार्यकृतसूत्रभाष्यस्य प्रत्यधिकरणसारार्थं प्रतिपादयति । ग्रन्थकृतानेन प्रथमद्वितीयचतुर्थाध्यायानां यथाक्रमं विराङ्विश्वबीजतुर्याख्या समाख्याता । ग्रन्थोऽयमडयारपुस्तिकालयमालायां मुद्रितः ।
७. ब्रह्मसूत्रार्याद्विशतिका - द्विशतैः पद्यैस्सूत्रार्थविवरणकार्ययं ग्रन्थ निर्णयसागरमुद्रणालये अडयार पुस्तकलये च मुद्रितः ।
८. लिङ्गभङ्गमुक्तिशतकम् - सव्याख्यम् सव्याख्योऽयं ग्रन्थ 87 पञ्जाबसूच्यां दृश्यते ।
९. महावाक्यरत्नावलीप्रभालोचनम् -
भासकलोचनापरनामायं ग्रन्थः रामचन्द्रेन्द्रसरस्वतीकृतायाः महावक्यरत्नावलीव्याख्यायाः प्रभाख्याया व्याख्यारूपः । अमुद्रितोऽयं ग्रन्थः मैसूरपुस्तकालये लभ्यते ।
१०. परमाद्वैतसिद्धान्तपरिभाषापि अनेन कृतेति ज्ञायते ।
त्यागराजशास्त्री (राजुशास्त्री) (1815-1904 A.D)
आधुनिकोऽयं प्रकाण्डपण्डित मन्नार्गुडिरानुशास्व्यपराभिधः त्यागराजशास्त्री अप्पय्यदीक्षितवंशजः । अस्य पितामहस्त्यागराजाभिधः । अस्य पिता मार्गसहायापरामिध अप्पय्य (अप्पा)दीक्षिताख्यः । अस्य माता मरकतवल्लीनाम्नी । भारद्वाजगोत्रजोऽयं बाल्यादारभ्य स्वपितामहात् प्राप्तकाव्यव्युत्पत्तिः मन्नार्गुडिवासिनः नारायणसरस्वस्याः वेदान्तशास्त्रं, कुम्भघोणवासिन रघुनाथशास्त्रिणः मीमांसां मन्नार्गुडिवासिनः गोपालशास्त्रिणः व्याकरणञ्चाधीयाय । तात्कालिकेषु विद्वत्सु अद्वैतवेदान्ते नितरं निष्णातोऽयं अद्वैतसभाप्रवर्तकाग्रगण्यश्रेयः प्राप । तिरुवारूर्समीपस्थे कृत्तम्भाडिनामकग्रामे समुत्पन्नोऽयं मन्नार्गुडिग्रामस्थं स्वमातुलगृहं गत्वा तत्रैव विद्वत्सङ्गदत्तचित्तः स्वकालमनैषीत् । अद्वैतसभारजतजयन्तीस्मारकपत्रिकाप्रमाणात् अस्य काल एकोनविंशतिशतकमिति (1815-1904 A.D.) इति ज्ञायते । अस्य शिष्येषु हरिहरशास्त्री प्रसिद्धः ।
१. न्यायेन्दुशेखरः (S. V. P. K.)
चन्द्रिकाप्रसादनापरनामायं ग्रन्थ तार्किकश्रीमदनन्तार्यविरचितस्य अद्वैतसिद्धि-चन्द्रिकादिदूषणपरस्य न्यायभास्करनामकग्रन्थस्य खण्डनपरः चन्द्रिकाप्रसादनपूर्वकं अद्वैतसिद्धान्तसाधकः । मुद्रितश्चायं ग्रन्थः शारदाविलासमुद्रणालये कुम्भघोणनगरे ।
२. ब्रह्मविद्यातरङ्गिणीव्याख्या-श्रीमन्नारायणयोगीन्द्रैः प्रणीतस्य अद्वैतसिद्धान्त प्रतिषादकस्य ब्रह्मविद्यातरङ्गिण्याख्यस्य प्रकरणग्रन्थस्य विस्तृतविवरणरूपः ।
३. सद्विद्याविलासः सव्याख्याः - छान्दोग्यषष्ठाध्यायार्थसङ्ग्राहकः पद्यात्मकः भाष्यतद्व्याख्यानानुसारं विस्पष्टार्थप्रकाशकः ।
४. वेदान्तवादसङ्ग्रहः- ग्रन्थोऽयं अद्वैतसिद्धान्तीयावान्तरमतभेदविषयकान् पूर्वोत्तरपक्षान् आवेदयति ।
एवं - १. दुर्जनोक्तिनिरासः । २. उपाधिविचारः . ३. व्यावहारिकशब्दसाधुत्वविचारः । ४. प्रातिबन्ध्यशब्दसाधुत्वविचारः । ५. स्तोत्रग्रन्थाः । ६. सुन्दरेशलीलासङ्गहः । ७. पुरुषार्थबोधसंग्रहः । ८. गङ्गाष्टकम् । ९. आर्तिहराष्टकम् । १०. नागरखण्डार्थसंग्रहः । ११. शिवमहिमकलिकास्तुतिव्याख्या । १२. इदंव्याक्या । १३. दीक्षितनवरत्नमालिका । १४. स्मार्तशिवरात्रिनिर्णयः । १५. राजगोपालोत्सवानुक्रमणिका । १६. त्यागेशनटनस्मरणसन्तानम् । १७. अर्धनारीश्वराष्टकम् । १८. सत्याख्या चतुश्लोकी । १९. कावेरीनवनरत्नमालिका । २०. सव्याख्यः त्यागराजस्तवः । २१. सव्याख्या श्लोकद्वयी । २२. शिवतत्वविवेकदीपिका । २३. तत्वार्थचन्द्रिका (सामरुद्रसंहिताभाष्यम्) २४. ताम्रपर्णीस्तुतिः । एवं वहवो ग्रन्थाः प्रणीता इति ज्ञायन्ते ॥
६९. हरिहरशास्त्री (1800-1900 A.D.)
राजुशास्त्रीति प्रसिद्धस्य त्यागराजशास्त्रिणश्शिष्यः चिदम्बरनगरवासी एकोनर्विशतिशतकीयोऽयं हरिहरशास्त्री वेदान्तशास्त्रनिष्णातः । अस्य शिष्येषु पोलकं श्रीरामशास्त्रिणः दण्डपाणिस्वामिदीक्षिताश्च प्रसिद्धाः ।
१. न्यायेन्दुशेखरः -
राजुशास्त्रिकृतन्यायेन्दुशेखरोत्तरभागात्मकोऽयं ग्रन्थः अद्वैतसिद्धान्तसाधकः ब्रह्मविद्यापत्रिकायां कुम्भघोणे मुद्रितः ।
७०. अनन्तकृष्णशास्त्री (1886 A.D.)
एते हि अनन्तकृष्णशास्त्रिणः महामहोपाध्यायाख्यराजकीयविरुदसम्मानिताः पालक्काडन्तर्गतसुब्रह्मण्योपाध्यायपुत्राः न केवलं ग्रन्थकरणेन शतभूषणीकाराः परन्तु जन्मनापि शतभूषणी (नूरणि) काराः । नृरणिग्रामाभिजना एते चित्तूरपाठशालायां साहित्यमधीत्य 1904 तमे वत्सरे चिदम्वरक्षेत्रे हरिहरशास्त्रिभ्यः समधीतव्याकरणाः मद्राससंस्कृतकलाशालाया अधीतपूर्वोत्तरमीमांसाशास्त्राः तिरुपति संस्कृतकलाशालाया अध्यापका आसन् । अनन्तरं एते कल्कत्तासंस्कृतकलाशालावेदान्तप्राध्यापका भूत्वा पञ्चदशवत्सरेभ्य अनन्तरं बम्बई भारतीविद्याभवनस्य गीताविद्यालये प्राध्यापका आसन् । एतेषां गुरवः पञ्चापगेशशास्त्रिणः, वेङ्कटसुब्बाशास्त्रिणश्च धन्यतमाः । एतैः बहवः ग्रन्थास्सम्पादिताः कृताश्च । परन्तु स्वतन्त्रा एव ग्रन्था अत्र निर्दिश्यन्ते ।
१. वेदान्तपरिभाषाप्रकाशिका । वेदान्तपरिभाषाव्याख्यात्मकोऽयं ग्रन्थः कल्कत्ताविश्वविद्यालये मुद्रितः ।
२. अद्वैतमार्ताण्डः । ग्रन्थोऽयं व्याससिद्धान्तमार्ताण्डादिद्वैतिग्रन्थं खण्डयन् अद्वैतं पुष्णाति । ग्रन्थोऽयं वणिकप्रेस कल्कत्तायां मुद्रितः ।
३. वेदान्तरक्षामणिः । ग्रन्थस्यस्य श्रीभाष्यसमालोचनमित्यपि नामान्तरम् । ग्रन्थेऽस्मिन् त्रिपाठीमहोदयोद्भावितानांं अद्वैतदूषणानां खण्डनं श्रीभाष्यसमालोचना अद्वैतसिद्धान्त एव श्रुतेरैदम्पर्यञ्च वादमुखेन प्रतिपादितानि । ग्रन्थोऽयं विश्वमित्रमुद्रणालये कल्कत्तायां मुद्रितः ।
४. शारीरकभाष्यटिप्पणी - प्रदीपः (CSS 1)
५. अद्वैतदीपिका । ग्रन्थोऽयं महामहोपाध्यायश्रीरामसुब्बाशास्त्रिकृतस्य व्यासरायरचितमध्वचन्द्रिकाखण्डनपरस्य चन्द्रिकाखण्डनाख्यग्रन्थस्य निरसनाय मध्वचन्द्रिकासमर्थनाय च गौडगिरिवेङ्कट्रमणार्यसम्पादितस्य चन्द्रिकाप्रकाशप्रसराख्यस्य ग्रन्थस्य सिद्धान्तान्, उत्तरादिमठाधीशमध्वस्वामिकृतं `चन्द्रिकामण्डन'ञ्च खण्डयति, शाङ्करमतमक्षुण्णं साधयति च । ग्रन्थोऽयं वणिकमुद्रणालये कल्कत्तायां मुद्रितः ।
६. चतुर्ग्रन्थिसङ्ग्रहः - (CSS)
७. भगवद्गीता भारतीयदर्शनानि च । ग्रन्थेऽस्मिन् नवमेऽध्याये भगवद्गीताया अद्वैतमत एव तात्पर्यमित्यवधारितम् । ग्रन्थोऽयं भारतीयविद्याभवनग्रन्थमालायां (BVS 4) बम्बई नगरे मुद्रितः ।
८. शतभूषणी (शतदूषणीपरीक्षणम्) । विशिष्टाद्वैतग्रन्थरत्नस्य शतदूषण्याख्यस्य अद्वैतनिन्दापरस्य ग्रन्थस्य खण्डनपरोऽयं ग्रन्थः । यस्य कृते एते अनन्तकृष्णशास्त्रिणः कामकोटिपीठाधिपतिभिः शतभूषणीति विरुदेन भूषिताः । ग्रन्थोऽयं मद्रासनगरे (P. G. Pal) मुद्रणालये मुद्रितः ।
९. शतभूषण्यनुवन्धः । एवं विवाहरहस्यमीमांसा, अधियाननिर्णयमीमांसाशास्त्रसंग्रहः, सनातनधर्मप्रदीपः, कामप्रदीपप्रभाव्याख्या, सौगन्ध्यविमर्शाख्याश्च ग्रन्था विरचिताः ।
७१. श्रीरामशास्त्री (1900-1968 A.D.)
पोलकं - श्रीरामशास्त्रिण इति प्रसिद्धा इमे चोलदेशीय नन्निलग्रामसमीपस्थ पोलकाख्याग्रहारे प्राप्तजन्प्तानः वेदान्तसाहित्यादिषु दर्शनेषु पण्डिताः शास्त्ररत्नाकर इति विरुदेन सम्मानिताः ।
अद्वैतसभापण्डिता इमे हरिहरशास्त्री-दण्डपाणिस्वामि-वेङ्कटरामशास्त्रिभ्यः प्राप्तविद्याः मद्राससंस्कृतकलाशालायाः वेदान्तप्रधानाध्यापकाश्च राजन्ते स्म । एतेषां पिता सुन्दरशास्त्री स्वस्यापरे वयसि प्राप्तसन्यासाश्रमः । एतेषां माता शिवरामलक्ष्मीनाम्नी । वारक्यान्वयजा एते स्वग्रन्थं द्रविडात्रेयदर्शनं षट्पञ्चाशदधिकपञ्चसहस्रेषु कलिवर्षेषु चक्रुरिति ज्ञायते ।
एते न केवलं शास्त्राणामध्यापने परन्तु साहित्यरसानुभवे च निष्णाताः । शास्त्रज्ञत्वकवित्वयोस्सुन्दरसामञ्जस्यमेतेषु महान् विशेषः । एते न केवलं दर्शनतत्वज्ञाः परन्तु विमर्शकवरेण्याश्च । एतेषां प्रतिभा यथा शास्त्ररण्यां तथा विमर्शनसरण्याञ्च असाधारणी विद्यते । शिवाद्वैते शाङ्कराद्वैते च पारङ्गता इमे रसिकोत्तमाः अध्ययनशीलाश्च ।
१. द्रविडात्रेयदर्शनम् । शङ्करात् प्राचीनयोः ब्रह्मनन्दिद्रविडाचार्ययोः सिद्धान्तप्रदर्शनपरोऽयं ग्रन्थः (B. G. Pal) मुद्रणालये मुद्रितः ।
२. चतुर्मतसामरस्यम् । कामकोटिकोशस्थाने मुद्रितः ।
३. आभोगटिप्पणी ॥ मद्रासराजकीयहस्तलिखितपुस्तकालये मुद्रितः ॥
अन्ये च बहवः शैवग्रन्थाः एतैस्सम्पादिताः ॥
७२. जगदीश्वरशास्त्री (20 th cent. A.D.)
कुम्भघोणसमीपस्थ-इञ्जिक्कोल्लैग्रामाभिजनः भज्ञरामदीक्षितपुत्रः कृष्णयज्ञस्वामि - वेङ्कटरामशास्त्रिभ्योऽधीतशास्त्रः अद्वैतसभापण्डित दक्षिणदेशीयश्च । अनेन कृताः ग्रन्थाः- १. निर्गुणतत्वनिर्णयः २. चिदचिच्छारीरकब्रह्मसिद्धिः ३. सप्तविधानुपपत्तिप्रकाशः सर्वे ग्रन्थाः अद्वैतसभायां मुद्रिताः ।
