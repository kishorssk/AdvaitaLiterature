\chapter{शङ्करात्प्राचीना अद्वैताचार्याः}
द्वितीयो भागः
अद्वैताचार्याः, अद्वैतग्रन्थप्रणेतारश्च ।
अद्वैताचार्याः अद्वैतमतप्रतिपादकग्रन्थविशेषप्रणेतारश्च कालभेदेन द्वेधा विभक्तुं शक्यन्ते -``शङ्करभगवत्पादेभ्यः प्राचीनाः, शङ्करभगवत्पादेभ्य अर्वाचीनश्चेति ।" तेष्वपि शङ्करभगवत्पादेभ्यः प्राचीनेषु आचार्येषु १. ब्रह्मसूत्रकारात् शङ्करभगवत्पादेभ्यश्च प्राचीनाः, २. ब्रह्मसूत्रकारात् अर्वाचीनाः शङ्करभगवत्पादेभ्यः प्राचीनाश्चेति विभागोऽपि कर्तुं शक्यते । अद्वैतमतसिद्धान्तस्य प्राचीनतमत्वात् ब्रह्मसूत्रेषु बहूनां वृत्तिग्रन्थानां सत्वानुमानाच्च । शङ्करभगवत्पादकृतभाष्यप्रभावात् बौद्धप्रभावाच्च प्राचीनं वृत्यादिकं विनष्टम् । अथवा शाङ्करभाष्येण तेषां सम्पूर्णतया गतार्थत्वात् तेषां संरक्षणे औद सीन्यवशाच्च विनष्टं जातम् । तत्वचन्द्रिकाकारेणोमामहेश्वरेण तत्वचन्द्रिकायां (R 5156 MGOML) ``शङ्करभगवत्पादैः स्वीये भाष्ये एकोनशतं सृत्रवृत्तिग्रन्थाः परामृष्टाः, केचित् खण्डिताश्चेति" निर्दिश्यते । शाङ्करभाष्यदर्शकाणां सर्वेषाञ्च मनसि ईदृशी चिन्ता स्वाभाविकी यत् `शाङ्करभाष्ये शङ्कराचार्येण ततोऽपि प्राचीनवृत्तिकारस्य व्याख्यानं बहुषु स्थलेषु खण्डितम् । तदर्थं तत्र तत्र युक्तिरपि प्रदर्शिता । परन्तु तासां वृत्तीनां नाम, वृत्तिकारादीनां नाम वा न कुत्रापि निर्दिष्टम् । परन्तु व्याख्यानादिकर्तृभिः तेषां नाम तत्र तत्र निर्दिष्टं क्वचित् क्वचित् । शाङ्करभाष्ये पूर्वपक्षत्वेन गृहीता एव भास्कररामानुजादिभिः सिद्धान्तत्वेन स्वीकृता वर्तन्ते । तस्मात् शङ्करभाष्ये पूर्वपक्षत्वेनोपन्यस्तानां मतवादानां मध्ये कियन्तो मतवादाः पूर्वैरुद्भाविताः? कियन्तो वा बौधायनादिभिः प्रकल्पिताः? कति वा शङ्करभगवता स्वयं समुद्भाविताः? किं ते शाङ्करभाष्यानुगाः ? उत शङ्करसिद्धान्तेन अंशतस्सदृशाः ? इत्यादिविषया न निर्णेतुं शक्यन्ते । तथा च बोधायनोपवर्षादिभिस्सह भास्कररामानुजादीनां सम्बन्धो यथा दुर्निर्णेय स्तथा वृत्तिकारादीनामपीत्येवावगम्यते । एवमपि बोधायनोपवर्षभर्तृप्रपञ्चभर्तृहरि ब्रह्मनन्दिसुन्दरपाण्डयद्रविडाचार्यब्रह्मदत्तादिवेदान्ताचार्याः ब्रह्मसूत्रेषु वृत्तिग्रन्थान् प्राणैषुरिति परं ज्ञायते । व्याख्यानपरम्परैवात्र प्रमाणम् । एतेषां ग्रन्थानामिदानीमनुपलम्भात् ।'
एवं ब्रह्मसूत्रेवपि बादरायणव्यासेन केचनाचार्याः नाम्ना निर्दिष्टाः . तेषाञ्च ग्रन्था नोपलभ्यन्ते । तथापि तेषां मतवादादिकं विविधपत्रिकादिप्रमाणानुसारं यथाकथञ्चित विवृणीतुं प्रयतामहे । तेषु आचार्येषु ``अष्टावक्र - आत्रेय - आश्म रथ्य - उपवर्ष औहुलौमि - काशकृत्स्न-कार्ष्णाजिनि-जैमिनिबादर्यादयः व्यासात्प्राचीनाचार्यविभागे, आचार्यसुन्दरपाण्डूय द्रविडाचार्य ब्रह्मदत्त ब्रह्मनन्दि भर्तृप्रपञ्च भर्तृ रि बदिरायणापरव्यासाचार्याः शङ्करात्माचीनाचार्यविभागे, च विभक्तुं शक्यन्ते । एवमनिर्णीताः अज्ञातसमयाः विभिन्नविचारलक्ष्यीभूता अपरे अध्यात्मरामायणकारआदिशेषकाश्यपजाम्बवतदत्तात्रेययोगवासिष्ठ कारशु कसनत्सुजातादयश्च वर्तन्ते । तेषां सर्वेषां इतिहासादिकं प्रकरणेऽस्मिन् वर्णम लाक्रमेण निरूप्यते ।।"

१. अष्टावक्रः
अष्टावक्रोऽयं महर्षिः सुजाताकहोलयोः पुत्रः, उद्दालकस्य दौहित्रः, श्वेतकेतोस्स्वस्रीयः, वदान्यजामाता, सुपभाभर्तेति ज्ञायते । उद्दालकनामा महान् ऋषिंः वहोलाय स्वशिष्याय स्वकन्यां सुजातानाम्नीं वैवाहिकेन विधिना ददौ । शिप्यतध्ये अधीय नं स्वपितरं कहोलं सुजातागर्भस्थः अग्निकल्पशिशशुः रात्रिन्दिवं कहो रुस्य ध्ययनशीलतां च प्रति अधिचिक्षेप । क्रुद्धः मातामह उद्दालकस्स्वदौहित्रं ``यस्मात त्वं कुक्षौ वर्तमानोऽधिक्षिपसि तस्मात् त्वं अष्टकृत्वः वक्रो भवितासि" इति शशाप । तथा स वक्र एवाभ्यजायत अष्टावक्र इति प्रथितश्च । स्वपत्नीप्रेरणया वित्तार्जनार्थं जनरूपुरं गत अष्टावक्रपिता कहोलः जनकपुरद्वारपालेन बन्दिना वादे पराजितः जले मिमज्य मृतश्च । मातृपकाशात् पितृवृत्तान्तं श्रुत्वा मातुलेन श्वेतकेतुता सह अष्टावक्रः जनकपुरं गतः, वादेषु द्वारपालं जनकञ्च जित्वा ``समङ्गापरनाम्नीं मधुविलानदी स्नात्वा अष्टावक्रेदेहं समीकृत्य जरर्तरूपधारीण्या उत्त दिगभिपानिदेवतायाः धर्मो देशं स्वीकृत्य वदान्यकन्यां सुप्रभानाम्नी विधिनोपयेमे । इति कथा महाभारतादिप्रसिद्धा । अनेन कृतः ग्रन्थः-"
(क) अष्टावक्रगीता
अष्टावक्रसृक्तम्, अवधूतानुभूतिः, इत्यादिनाम्ना प्रसिद्धोऽयं ग्रन्थः ए विंशतिभिाध्यायैः पूर्ण अष्टवक्रजनकसंवादरूपेण अद्वैतवेदान्तसिद्धान्तान् ब्रह्मणोऽद्वितीयत्वं चिन्मयत्वञ्च प्रतिपादयति । ग्रन्थोऽयं आष्टेकरकम्पनि पूनानगरे मुद्रितश्च । अस्य व्य ख्याः विश्वेश्वरकृता - दीपिघख्या, पूर्णानन्दतीर्थकृता काचन व्याख्या, मुकुन्दमुनिकृता अन्या व्याख्या, भासुरानन्दकृता अपरा व्याख्या इत्येवं चतस्रः व्याख्या उपलभ्यन्ते ।
२. आचार्य सुन्दरपाण्ड्यः (600 A.D.)
शङ्करभगवत्पादेभ्यः प्राचीनोऽयं सुन्दरपाण्डयाचार्यः दक्षिणद्रविडदेशीयः, मधुरानगरवासीति ज्ञायते । पूर्वोत्तरमीमांसयोः प्रकाण्डपण्डितेनानेन ब्रह्मसूत्रणां वार्तिकं विरचितमिति, कैस्तवीय षष्ठशतकात् प्राचीनः, षष्ठशतकीयो वा इति निश्चीयते । अत्रेमानि कारणानि-
स्वर्गीयमहामहोपाध्याय कुप्पुस्वामिशास्त्रिणः ``जर्नल आफ ओरियण्टल रिसर्च मद्रास" पत्रिकायाः प्रथमे भागे (J.O.R.I.) एवमभिपयन्ति । शङ्करभगवत्पादैस्स्वोये ब्रह्मसूत्रमाष्ये सनन्वय - अधिकरणभाष्यान्ते अपिवाहुः-
``गौणमिथ्य त्मनोऽपत्वे पुत्रदेहादिबाधनत् ।
सद्ब्रह्म त्माहं इत्येव बोधकार्थं कथं भवेत् ।।
अन्वेष्टव्यात्मबिज्ञानात् प्राक् प्रमातृत्वमात्मनः ।
अन्विष्टस्स्यात् प्रमातैव पाप्मदोषादिवर्जितः ।।
देहात्मप्रत्ययो यद्वत् प्रप्ताणत्वेन कल्पितः ।
लौकिकं तद्वदेवेदं प्रनाणं त्वात्मनिश्चयात् ।।" इति 
श्लोकत्रयमुदाहृतम् । अत्र भामतीकारैः ``अत्रैव ब्रह्मविदां गाथामुदाहरति" इत्यवतारितम् । पञ्चपादिकाकारैः ``प्रसिद्धपेतत् ब्रह्मविदां पूर्वोक्तं न्याय संक्षेपतः श्लोकैस्संगृह्नणाति" इत्यवतारितम् । पञ्चपादिकाव्याख्यात्रा नरसिम्हस्वरूपशिप्येण आत्मस्वरूपेण स्वीयप्रबोवपरिशोधिन्यां `श्लोकत्रयं सुन्दरपाण्डयाचार्यप्रणीतं प्रमाणयति' इत्यवतारितम् ।
माधवनन्त्रिणा विरचितायां सूतसंहितयाः व्याख्यायां तात्पर्यदीपिकाख्यायां ``देहात्मप्रत्ययो यद्वत् प्रप्ताणत्वेन कल्पितः । लौकिकं तद्वदेवेदं प्रमाणन्त्वात्मनिश्चयात् ।" इत्ययं श्लोक उद्धृतः । एतच्छ्रलोकविवरणावसरे तथा सुन्दर पाण्ड्यवार्तिकमपीति अवतारिका कृता । 
एवं त्रयोदशशतकीयेनामलानन्देन स्वरविते कल्पतरुग्रन्थे (3 - 3 - 25 Page 755 N.S.P. Edn.) ``आह चात्र निदर्शनमाचार्यसुन्दरपाण्ड्यः" इति-
निःश्रेण्यारोहणप्राप्यं प्राप्तिमात्रोपपादि च ।
एकमेव फलं प्राप्तुं उभावारोहतो यदा ।।
एकसोपानवर्त्येको भूमिष्ठश्चापरस्तयोः ।
उभयोश्च जवस्तुल्यः प्रतिबन्धश्च नान्तरा ।।
विरोधिनोस्तदैको हि तत्फलं प्राप्नुयात्तयोः ।
प्रथमेन गृहीतेऽस्मिन् पश्चिमोऽवतरेन्मुधा ।। इति 
श्लोकत्रयमुपपादितम् । एतदेव श्लोकत्रयं कुमरिलभट्टैर्बलाबलाधिकरणे तन्त्रवार्तिके (BSS Page 852-853) आह चेत्यादिना प्रतिपादितम् । एवञ्च शङ्करभगवत्पादेभ्यः कुमरिलभट्टाच्च प्राचीन इति सिध्यति ।
आचार्यसुन्दरपाण्ड्यकृतः नीतिद्विषष्ठिकाख्यः ग्रन्थः कश्चन नीतिपरः मद्रपुर्यां प्रकाशितः । तत्रस्थाः बहवः श्लोकाः त्रयोदशशतकीयेन जल्हणेन सूक्तिमुक्तावल्यां, पञ्चदशशतकीयेन वल्लभदेवेन सुभाषितावल्यां, पञ्चदशशतकीयेन शार्ङ्गधरेण स्वीयशार्ङ्गधरपद्धत्यां पोतयार्येण प्रसङ्गरत्नावल्यां, पेद्दिभट्टेन सूक्तिवारिधौ, द्वादशशतकीयेन कलिङ्गराजापरनाम्ना सूर्यपण्डितेन कुलशेखरसूक्तिरत्नहाराख्ये ग्रन्थे चोदाहृताः । विशेषतः सूर्यपण्डितेन ``आचार्यसुन्दरपाण्ड्यकृता" इति निर्दिष्टाश्च । पञ्चतन्त्रकर्ता विष्णुशर्मा क्रैस्तवीयषष्ठशतकादर्वाचीन इति विमर्शकसिद्धान्तः । तेनापि नीतिद्विषष्ठिकास्थाः 29, 30, 48, श्लोकाः स्वीये ग्रन्थे उद्धृताः । एवं नीतिद्विषष्ठिकाग्रन्थावसाने कश्चन श्लोकः - ``इमां काञ्चनपीठस्थां समेत्य कवयो भुवि । आर्यां सुन्दरपाण्ड्यस्य स्नापयन्ति वधूमिव ।" दृश्यते । क्रैस्तवीय षष्ठशतकादारभ्य द्रविडेदेशे मधुरायां द्रविडसङ्घस्स्थापितः । अभ्यर्हिंत कविं तत्कृतिञ्च तत्सङ्घस्थाः विद्वांसः सङ्धपूजिते काञ्चनपीठे निवेश्य कनकाभिषेकमकुर्वन्निति तमिलसाहित्ये प्रसिद्धम् । तादृशस्तत्कारः ग्रन्थस्यास्यनीतिद्विषष्ठिकाख्यस्यापि प्रवृत्त इत्येवास्मात् पद्यात् ज्ञायते ।
``श्रीमच्छकाब्देऽब्धिशशिसायकसम्मिते । राजा माधववर्माभूत् विख्यातो धरणीतले ।।" इति प्रसङ्गरत्नावल्याख्ये पोतयार्यकृते ग्रन्थे दर्शनात्, पेण्डयाल सुब्रह्मण्यशास्त्रिभिः प्रकाशितात् पुलिवूरुशिलाशासनप्रमाणाच्च 514 शके 592 A.D. काले विष्णुकुण्डिनवंश्यः माधववर्मापरनामा जनाश्रयाख्यः कृष्णानदीतीरान्ध्रदेशाधीश आसीदिति ज्ञायते । तेन राज्ञा कृता कृतिः जानाश्रयीति च प्रसिद्धा । तस्मिन् जानाश्रयीत्यपरनामके छन्दोग्रन्थे नीतिद्विषष्ठिकायाः चतुर्विशतितमः ``चारित्रनिर्मल जलः सत्पुरुषनदोऽक्षयो भवतु नित्यम् । यस्य विभवारविन्दे विद्वद्भमराः कृतविनोदाः ।।" इति शलोक उद्धृतः ।
तस्मात्-कुमरिलभट्टेन, शङ्कराचार्येण, विष्णुशर्मणा, जनाश्रयेण, माधवमन्त्रिणा च प्रमाणीकृतोऽयं सुन्दरपाण्ड्याचार्य क्रैस्तवीयषष्ठशतकात् प्राचीनः, षष्ठशतकीयो वेति निश्चप्रचमभ्युपगम्यते ।
स्वर्गीय महामहोपाध्याय कुप्पुस्वामिशास्त्रिण आचार्यसुन्दरपाड्यमेनं क्रैस्तवीयाष्टमशतकीयं प्रवदन्ति । तमिलसाहित्ये प्रसिद्धः (अषकेसरी) एवायमिति जर्नल आफ ओरियण्टल पत्रिकायाः प्रथमे भागे J. O. R. I. निरूपयन्ति । इतिहासनिपुणाः K. A. नीलकण्ठशास्त्रिणस्तु जर्नल आफ ओरियण्टलरिसर्चपत्रिकायाः प्रथमे भागे (J. O. R. Madras-1) सप्तमशतकमध्यकाले सुन्दरपाण्ड्याचार्य आसीदिति प्रतिपादयन्ति । सर्वथापि शङ्करभगवत्पादेभ्यः प्राचीनोऽयमित्येवास्मत्सिद्धान्तः । ``श्रीमान् सुन्दरपाड्यः श्रुति स्मृति प्रसृत सत्पदार्थज्ञ" इति नीतिद्विषष्ठिकायां दर्शनात आचार्यसुन्दरपाण्ड्योऽयं पूर्वोत्तरमीमांसादिषु निष्णात इति ज्ञायते । शाङ्करभाष्ये उद्धृत्य प्रमाणीकृतत्वात् वेदान्तेऽनेन ब्रह्मसूत्राणां किमपि वार्तिकं पद्यबंद्ध कृतं स्यादिति निश्चीयते । कुमरिल भट्टैरुद्धृतोऽयं पूर्वमीमांसायामपि ग्रन्थप्रणेता इत्यभ्यूह्यते ।
अस्यैवाचार्यसुन्दरपाण्ड्यस्य द्रविडाचार्य इत्यपि नामान्तम्, व्यावहारिकनाम वा स्यादित्यूह्यते । अत्रेदं कारणं भवति-अभिनवद्रविडाचार्यापरनाम्ना अष्टादशशतकीयेन बालकृष्णानन्दसरस्वत्या विरचिते पद्यबद्धे शारीरकमीमांसाभाष्यवार्तिके आशुतोषग्रन्थमालामुद्रिते शङ्करभाष्यस्थस्य `अपिचे'ति ग्रन्थस्यावतरणसमये ``कथितार्थपरं द्रविडार्यकृतां अपिचेति गुरुर्वदतीह कथाम्" (ASI Page 403) निर्दिष्टम् । तस्मात् आचार्यसुन्दरपाण्ड्योऽयं द्रविडाचार्य इत्यपि व्यवहृतस्स्या दिति निश्चीयते ।
३. आत्रेयः
व्यासात्पूर्वतनेषु वेदान्ताचार्येषु अन्यतमोऽयमात्रेयः । बादरायणव्यासनिमिंतेषु ब्रह्मसूत्रेषु स्वामिनः फलश्रुतेरित्यात्रेयः 3-4-44 इति आत्रेयोऽयं निर्दिष्टः । यज्ञे अङ्गाश्रीतोपासना यज्ञस्वामिना एवं ऋत्विग्भिश्च कर्तव्या । अत्र फलविषये संशयः । किं उपासनाजन्यफलभाक् यजमानः ? उत ऋत्विक् ? इति । अत्रात्रेयमतन्तु अङ्गाश्रितोपासनाफलभाग्यजनमान एवेति । अदसीयः वेदान्तग्रन्थस्तु नोपलभ्यते ।
४. आदिशेषः (परमार्थसारकारः)
``वेदान्तशास्त्रमखिलं विलोक्य शेषस्तु जगदाधार" इति परमार्थमारे दृश्यते । साघवानन्दकृतायां व्याख्यायां ``भगवता जगदाधरेण आदिशेषेण" इति दृश्यते । तस्य त् सहस्रफणामणिमणिमण्डल आदिशेष एवास्य परमार्थसारस्य कर्तेति साम्प्रदायिकविश्व सः । परन्तु विधुशेखरभट्टाचार्यास्स्वपम्पादिवि ``गौडपदीयं आगमशास्त्र" मिति ग्रन्थे गौडपादाचार्यकालात् भास्कराचार्यकालस्य च मध्यवर्तिना केतापि आदिशेषनाम्ना ग्रन्थरचना कृतेति परमार्थसारग्रन्थकारकालः (500-800 A.D.) इति प्रवदन्ति ।
परमार्थसारः - (TSS 12)
आर्यावृत्तघटितैः पद्यैरद्वैतपरमार्थसारान् शिष्योपदेशशैल्यां प्रतिपादयन्नयं ग्रन्थः अभिनवगुप्ताचार्यकृतात् परमार्थसाराद्भिन्नः मुद्रितश्च चौखाम्बामुद्रणालये । अनन्तशयनग्रन्थावल्याञ्च सव्याख्योऽयं मुद्रितः । अस्य व्याख्याः- १ राघवानन्द मुनिकृता विवरणनाम्नी काचन २ वासुदेवयतिकृता अन्या प्रकाशि कानाम्नी व्याख्या अमुद्रिता (R. 4149 C. MGOML) लभ्यते ।
५. आश्मरथ्यः
ब्रह्मसूत्रकारात् बादरायणव्यासात् पूर्वतनोऽयं आश्मरथ्यः ब्रह्मसूत्रेषु वैश्वानराधिकरणे ``अभिव्यक्तेरित्याश्मरथ्यः 1-2-29, एवं वाक्यान्वयाधिकरणे प्रतिज्ञासिद्धेर्लिङ्गमित्याशमरथ्यः" 1-1-29 इति वारद्वयं निर्दिष्टः ।
उपनिषत्सु ईश्वरः प्रादेशमात्रः प्रतिपादितः । अस्य उपपतिरनेनैवं क्रियते ``परमेश्वर अनन्तः । भक्तानुग्रहार्थं प्रादेशमात्रादुद्भवति । हृदयादिषु उपलब्धियोग्येषु प्रदेशेषु उपलभ्यमानोऽयमिति प्रादेशमात्र इति च । भेदाभेदवाद्ययम् । कार्यावस्थायां विज्ञानात्मा परमात्मनः भिन्नः । कारणावस्थायान्तु अभिन्न इत्यस्य सिद्धान्त इति ज्ञायते ।"
६. उपवर्षाचार्यः (100 BC - 200 AD)
उपवर्षाचार्योऽयं पूर्वोत्तरमीमांसयेर्वृत्तिकारः, शङ्करभगवत्पादेभ्यः शबरस्वामिनोऽपि प्राक्तन इति ज्ञायते । शङ्करभगवत्पादैस्स्वीये सूत्रभाष्ये ऐकात्म्याधिकरणे 3-3-53 सूत्रे ``अत एव च भगवतोपवर्षेण प्रथमतन्त्रे आत्मास्तित्वाभिधानप्रसक्तौ शारीरके वक्ष्याम" इत्युद्धारः कृत इति सबहुमानमुपवर्षाचार्य आवेदितः । प्रकटार्थकारैरपि ``अतएवेत्य दि" भाष्यव्याख्यानावसरे `वृत्तिकारवचनं गमकमित्याह इत्येव शाङ्करभाष्यमवतारितम् । एवमानन्दमयाघिकरणे शङ्कराचार्यैः प्रथमं वृत्तिकारमतानुसारेण सूत्राणि व्याख्यातानि । अनन्तरं' `इदं त्विह वक्तव्यम्' इत्यादिना अधिकरणान्ते वृत्तिकारमतं पूर्वपक्षीकृत्य सिद्धान्तविधया स्वीयसिद्धन्तः प्रदर्शितः । एवमन्यत्रापि `अन्ये त्वाहुः' `अपरे त्वाहुः' इत्यादिना शङ्करभगवत्पादर्वत्तिकारमतमनूदितम् ।
शबरस्वामिना च ``वर्णा एव तु शब्दाः" इति भगवानुपवर्ष इति उपवर्षाचार्यः प्रमाणीकृतः । एवञ्च ब्रह्मसूत्राणां वृत्तिकार उपवर्षाचार्य अद्वैतमतैकदेशी प्र. चीन इति निर्णीयते । तादृशी उपवर्षीया वृत्तिस्तु नोपलभ्यते कुत्रापीदानीम् ।
ग्रन्थानुपलव्धेरेव विशिष्टाद्वैतिनः उपवर्षाचार्यस्यैव बोधायनकृतकोटिरित्यपि नामान्तरमिति वर्णयन्तस्स्वमतसंरक्षकबोधायनवृत्तिसत्यत्वसंरक्षणाय मणिमेखलादि द्रविडभाषाग्रन्थअवन्तिसुन्दरीग्रन्थप्रपञ्चहृदयग्रन्थमुखेन बोधायनकृतकोटिउपवषत्रयस्य अभिन्नतां साधयितुं प्रकटप्रयत्नमकुर्वन् । परन्तु तेषां प्रयत्नो विफल इति बोध यनकृतकोटिउपवर्षाचार्याः भिन्ना एवेति बोधायनवृत्तिस्तु नास्त्त्ये वेते ब्रह्मश्री पोलकं श्रीरामशास्त्रिभिरस्मद्गुरुचरणैः स्वीये द्रविडात्रेयदर्शने प्रतिपादितम् ।
आचार्य भगवद्दत्तैस्तु स्वीये ``भारतवर्ष का बृहद् इतिहास" नामके ग्रन्थे प्रथमभागे 84 पुटे शबरस्वामिनां कालः विक्रम चतुर्थशतकात् प्राचीन इति प्रतिपादितम् । तस्मादुपवर्षकालः 200 A.D. कालात् प्राक्तन इति तु निर्णीयते ।।
उपवर्षः वर्षोपाध्यायस्य कनिष्ठभ्राता पाणिनीयवृत्तिकृत् कात्यायनस्य श्वशुरः, उपकोशाया जनकः, महापद्भनन्दस्य प्रधानमन्त्रीति 500 A. D. काले आसीदिति च ``वृद्धत्रय्यां" गुरुपादशर्महालदारः ।
७. औहुलोमिः
व्यासात्पूर्वतनेषु आचार्येषु औडुलोमिरप्यन्यः । औडुलोमिरयं ब्रह्मसूत्रेषु वाक्यान्वयाधिकरणे ``उत्क्रमिष्यत एवं भावादौडुलौभिः" 1 - 4 - 21 इति, स्वाम्यधिकरणे ``आर्त्विज्यमौडुलोमिस्तस्मै हि परिक्रीयते" 3 - 4 - 45 इति, ब्राह्माधिकरणे ``चितिमात्रेण तदात्मकत्वादित्यौडुलोमि" 4 - 4 - 6 इति च स्थलत्रये निर्दिष्टः ।
संसार-मोक्षकालभेदेन जीवब्रह्मणोर्भेदाभेदवादी अयमौडुलोमिः । दृश्यप्रपञ्चेऽज्ञानवशात् जीवब्रह्मणोर्भेदः । मुक्तावस्थायान्तु उभयोरप्यभेद इत्यस्य मतं स्यादित्युह्यते । भामतीकारोऽपि मतमेतदीयं प्रतिपादयति । 
८. काशकृत्स्नः
अविकृतः परमेश्वरो जीवः, नान्य इति सिद्धान्तवाद्ययं काशकृत्स्नः ब्रह्मसूत्रेषु वाक्यान्वयाधिकरणे ``अवस्थितेरिति काशकृत्स्नः" 1 - 4 - 22 निर्दिष्टः ।।
९. काश्यपः
नायं ब्रह्मसूत्रकारैर्निर्दिष्टः । परन्तु ``शाण्डिल्यभक्तिसूत्रे तामैश्वर्यपरां काश्यपः परत्वात्" No. 29 इति निर्दिष्टः ।
१०. काष्णाजिनिः
छान्देग्योपनिषदां पञ्चमाध्याये श्रूयमाणस्य ``रमणीयचरणा" इति ग्रन्थस्य व्याख्यानावसरे कार्ष्णाजिनिमतं ब्रह्मसूत्रे निर्दिष्टम् । कृतात्ययाधिकरणे ``चरणादिति चेन्नोपलक्षणार्थेति कार्ष्णाजिनि" 3 - 1 - 9 सूत्रेण निर्दिष्टः ।।
११. जाम्बवान्
एतत्कृतत्वेन प्रसिद्धस्य प्रणवमहाभाष्याख्यग्रन्थस्य उपान्त्यवाक्यात् रामचन्द्रभक्तो भगवान् जाम्बवानेवायमिति प्रतीयते । यद्येवं तर्हि प्रजापति पुत्रोऽयं त्रेतायुगादारभ्य वर्तमानश्चिरञ्जीवी जाम्बवतीपिता भगवतः कृष्णस्य श्वशुरश्चेति निर्णेतुं शक्यते ।
प्रणवमहाभाष्यम् -
प्रणवार्थप्रकाशकोऽयं ग्रन्थः माण्डूत्योपनिषदन्तर्गतं ओङ्कारोपासनार्थ प्रदर्शयति । आह्निकत्रयपूर्ण अमुद्रितोऽयं ग्रन्थःतिरुवनन्तपुरपुस्तकालये 306 TCD दृश्यते ।।
१२.जैमिनिः
जैमिनिरयं ब्रह्मसूत्रेषु वैश्वानराधिकरणे देवताधिकरणे बालाक्यधिकरणे फलाधिकरणे पुरुषार्थाधिकरणे परामर्शाधिकरणे तद्भूताधिकरणे कार्याधिकरणे ब्राह्माधिकरणे अभावाधिकरणे च निर्दिष्टः । बादरायणस्य साक्षाच्छिष्यः 300 B.C कालात्पूर्वतन इति च सिद्धान्तः ।
१३. दत्तात्रेयः
पातिव्रत्यधर्मपरायणा अनसूया स्वपतिं अत्रिमुनिं स्वशिरसि वहन्ती निशीथे स्वाश्रमात् देशान्तरं जगाम । सूचिभेद्ये तमसि मध्येमार्गं गच्छन्ती सा शूलारोपितं माण्ढव्यमबुध्वा स्वपतिं माण्ढव्यशरीरे घर्षितवती । निष्कारणं पीडामुत्पादयन्तं कमित्यज्ञात्वा माण्ढव्यः ``सूर्योदयादतन्तरं पीडोत्पादकस्य मृतिर्भवतु" इति शशाप । पतिपरायणाऽनसूया ``सूर्योदय एव मा भूदिति" शशाप । अन्धकारावृते च जगति, सूर्ये च अनुदिते यज्ञक्रियादिकर्मलोपात् भीताः देवाः ब्रह्मणा प्रेरितास्सूर्योदयाय अनसूयां प्रार्थयामासुः । उदिते च सूर्ये अनसूयायाः पातिव्रत्यधर्मेण तुष्टेन इन्द्रेण प्रार्थितः भगवान् विष्णुरनसूयायां अत्रेः पुत्रत्वेनावततार । सोऽयं पुत्रः दत्तात्रेयः । दत्तात्रेयस्य प्रसादेन कार्तवीर्यार्जुतस्सचराचरं भूमण्डलं शश स । दत्तात्रेयः । दत्तात्रेयस्य प्रसादेन कार्तवीर्यार्जुतस्सचराचरं भूमण्डलं शश स । दत्तात्रेयः निमिनामकस्य पिता श्रीमतः पितामहश्चेति कथा महाभारते सभापर्वणि अनुशासनपर्वणि च प्रसिद्धा ।
(क) वेदान्तसारः -
क्वचित् क्वचित् अस्यैव अवधूतगीता इति नामान्तरमिति च दृश्यते । दत्तात्रेयकार्तिकेयंसवादरूपेऽस्मिन् ग्रन्थे प्रथमपरिच्छेदे अद्वैतब्रह्मवर्णना, द्वितीयादारभ्य सप्तमपरिच्छेदान्तं स्वात्मसंवित्युपदेशश्च दृश्यते । अमुद्रितोऽयं सम्पूर्णग्रन्थः सरस्वतीमहालये (7589 TSML) दृश्यते ।
(ख) अवधूतगीता -
दत्तगीता, जीवन्मुक्तिगीता, इत्यादिकं नाम अस्यैव ग्रन्थस्य दृश्यते । गोरक्षदत्तात्रेयसंवादरूपेऽस्मिन् ग्रन्थे जीवन्मुक्तस्य स्वरूपं सम्यगुपवर्णितम् । मुद्रितश्चायं ग्रन्थ अ ष्टेकरकम्पनि पूनानगरे गीतासङ्ग्रहे । अस्याः व्याख्या पर मानन्दगीर्थकृता अमुद्रिता वर्तते ।
(ग) प्रबोधचन्द्रिका -
ग्रन्थोऽयं मध्यप्रान्तीयबरार्ग्रन्थसूच्यां दृश्यते ।
(घ) स्वात्मसंवित्युपदेशः-
ग्रन्थोऽयममुद्रितः बरोडापुस्तकालये (996 BRD) लभ्यते । जीवन्मुक्तलक्षणमप्यस्य कृतिरिति ज्ञायते ।
१४. द्रविडाचार्याः
अद्वैतसम्प्रदायप्रवर्तकेषु पूर्वाचार्येषु अन्यतमा एते द्रविडाचार्या एतद्युगारम्भ एव सम्भूताः । एते भाष्यकारा इति प्रसिद्धाः । छान्दोग्योपनिषदां अर्थविवरणात्मकं सूत्ररूपवाक्यनिचयपण्डितं वाक्यनामकं ग्रन्थं ब्रह्मतन्दिनः प्राणैषु । द्रविडाचार्यैः स्वव्याख्येयवाक्यग्रन्थानुपारं सविशेषनिर्विशेषभेदेन द्विरूपं ब्रह्म न्यरूपि ।
बृहदाण्यकोपनिषद्भाष्ये द्वितीयाध्यायप्रथमब्राह्मणविवरणे शङ्कराचार्यैः द्रविडाचार्याः प्रमाणीकृताः । ``अत्र हि सम्प्रदायविदः आख्यायिकां सम्प्रचक्षते" इति । आनन्दगिरिणापि ``तत्वमस्यादिवाक्यमैक्यपरं, तच्छेषस्सूष्टय दिवाक्यम्" इत्युक्तेऽर्थे द्रविाडाचार्यसम्मतिमाह - ``अत्र चेति" इत्यवतारिका प्रदीयते । बृ दाण्यकवार्तिकेऽपि - ``आचक्षते तथाचात्र केचिदाख्यायिकां शुभाम् । यथाभिलषितार्थोऽयं यथा सम्भाव्यते स्फुटः ।" इति । अत्राप्यानन्दगिरिणा ``द्रविडाचार्य प्रणीतामाख्य यि कामवतारयति" इत्यवतारिता । एवं तोटकाचार्यः श्रुतिसारसमुद्धरणे विषयेऽस्मिन्नेव द्रविडाचार्यान् निर्दिशति -``द्रविडोऽपि च तत्वमसीति वचो विनिवर्तकमेव निरूपितवान् । शबरेण विवर्धितराजशिशोर्निजजन्मविदुक्तिनिदर्शनतः।" इति । एवं छान्दोग्योपनिषदश्शाङ्करभाष्योपक्रमे एवं दृश्यते - ``ओमित्येतदक्षरमित्यष्टाध्यायी छान्दोग्योपनिषत् । तस्याः संक्षेपत अर्थजिज्ञासुभ्यः ऋजुविवरणं अल्पग्रन्थमिदं आरभ्यते । इति । आनन्दगिरिणा " ``अथ पाठक्रममाश्रित्यापि द्राविडं भाष्यं प्रणीतम् , तत्किमनेन इत्याकाङ्क्षायामाह अल्पग्रन्थ" मिति अवतरणिका प्रदत्ता । मधुसूदनसरस्वत्या कृतायां संक्षेपशारीरकटीकायां ब्रह्मप्तन्दिविरचितवाक्यानां सूत्ररूपाणां भाष्यकर्ता द्रविडाचार्य इति निर्दिश्यते । नृसिम्हाश्रमिकृतायां संक्षेपशारीरकटीकायां ``भाष्यकृदद्रविडाचार्यवचनात्" इति निर्दिश्यते । रामतीर्थेनापि नन्दिकृतग्रन्थभाष्यकारः द्रविडाचार्य इति निर्दिश्यते । छान्दोग्यभाष्ये तृतीयाध्याये मधुविद्य विवरणे शङ्कराचार्यैः ``अत्रोक्तः परिहार आचार्यैः" इति आचार्यशब्देन द्रविडाचार्यः निर्दिष्टः । आनन्दगिरिणाऽपि द्रविडाचार्योक्तं उपपादयतीत्यवतारिका दीयते । सूत्रभाष्ये ज्योतिश्चरणाधिकरणे ``व्याचक्षत" इति शब्देन द्रविडाचार्याः निर्दिश्यन्ते । भामत्यां समन्वयाधिकरणे ``यथाहुर्दविडाचार्या इति द्रविडाचार्याः" प्रमाणीकृताः । सूत्रभाष्ये समन्वयाधिकरणभाष्यान्ते ``गौणमिथ्यात्मनोऽसत्वे पुत्रदेहादिबाधनात्" इत्यादि श्लोकत्रयमुद्धृत्य प्रप्ताणीकृतम् । बालकृष्णानन्दसरस्वत्या प्रकाशिते शारीरकमीमांसाभाष्यवार्तिके पूर्वोक्तभाष्यस्यावतारिकाप्रदानसप्तये ``कथितार्थपरां द्रविडार्यकृतां अपि चेति गुरुवर्दतीहकथाम् ।।" इति निर्दिश्यते । गौणमिथ्यात्मन इत्यादि पद्यं द्रविडाचार्यकृतमिति निर्दिष्टम् । म. ग. कुप्पुस्वामि शास्त्रिभिश्च प्राच्यभाषासंशोघनपत्रिकायाः प्रथमे भागे J. O. R. Vol I Madras पूर्वोक्तपद्यत्रयं आचार्यसुन्दरपाण्ड्यकृतमिति निरूपितम् । द्रविडाचार्यस्यैव सुन्दरपाण्ड्य इति स्यान्नाम् ।
एवञ्चाद्वैताचार्यैः निर्दिष्टाः द्रविडाचार्याः नूतनसम्प्रदायप्रवर्तकत्वेन प्रसिद्धाः गौडपादसामयिकाः तत्समानमेव पूज्याः गौडपादाचार्या इव इमेऽपि सन्यासपरम्परायाः प्रवर्तका आसन् । स च सम्प्रदायश्शङ्कराचार्यैः स्वान्तेवासिषु केषुचन उरीकृत एव परन्तु भगवत्पादादिक्रमेण नोरीकृतः । किन्तु सम्प्रदायान्तरद्वारा । अत एव गौडपादानारभ्य सङ्कलितासु स्वाचार्यपरम्पारासु द्रविडाचार्यः न निर्दिष्टः इति प्रतिभाति । एतैः । बृहदारण्यकवाक्यभाष्यं 2 छान्दोग्यभाष्वमपि कृतं स्यात् ।।
१५.बादरिः
ब्रह्मसूत्रेषु वैश्वानराधिकरणे, कृतात्ययाधिकरणे कार्याधिकरणे अभावाधिकरणे च निर्दिष्टोऽय बादरिः । वैदिककर्मणि सर्वेषामधिकारं प्रादेशमात्रे दृदये वर्तमानत्वात् प्रादेशमात्र ईश्वर इति सिद्धान्तं, ``रमणीयचरणा" इति छान्दोग्यवाक्यस्थस्य चरणशब्दस्य कर्मपरत्वं, छान्देग्यस्य 4-15-5 य एनान् ब्रह्म गमयतीति बाक्यस्थब्रह्मशब्दस्य कार्यब्रह्मवाचकत्वं ईश्वरभावापन्नस्य विदुषश्शरीरेन्द्रियमनसां असत्वञ्च वदन् अयं बादरिर्व्यासात्प्राचीनेषु प्रसिद्धाचार्येषु अन्यतमः ।।
१६. ब्रह्मदत्तः
जीवाः ब्रह्मण उत्पद्यन्ते मोक्षपर्यन्तावस्थायिनश्चेति ब्रह्मदत्तः । वेदान्तेषु अस्य मतं औपनिषदाभासशब्देन व्यवहारार्हं भवति । ज्ञानकाण्डब्रह्मकाण्डयोर्मुख्यं फलभेकमेवेति ब्रह्मदत्तसिद्धान्तः । ``अहं ब्रह्मास्मि" इति वाक्यार्थज्ञानमेव मोक्षोपयोगीति ब्रह्मदत्तः । ब्रह्मदत्तपते उपनिषदः ध्याननियोगप्रधानाः । अद्वेतिनां मते मोक्षो दृष्टफलः । ब्रह्मदत्तमते मोक्ष अदृष्टफलः । ब्रह्मदत्तमते `तत्वमसि' वाक्यात् ``आत्मा वारे द्रष्टव्य" इति वाक्यमेवोपादेयार्हम् ।
ज्ञानकर्मसमुच्चयवादी अयं ब्रह्मदत्तः ब्रह्मसूत्राणां ज्ञानकर्मसमुच्चयपरां व्याख्यां कृतवान् स्यात् । ब्रह्मदत्तश्च यामुनाचार्येण सिद्धित्रयेऽनुपादेयत्वेन निर्दिष्टः । सुरेश्वराचार्यकृतनैष्कर्म्यसिद्धेः ज्ञानामृतकृतायां विद्यासुरभिनाम्न्यां व्याख्यायां मद्रासराजकीयहस्तलिखित पुस्तकालयस्थायां (R. 3354 MGOML) ब्रह्मदत्तः निर्दिष्टः । आनन्दगिरिणापि बृहदारण्यकसम्बन्धवार्तिकव्याख्यानावसरे (P. 220 ASS 16) ब्रह्मदत्तः निर्दिष्टः 
किमयं ब्रह्मदत्तः ज्ञानकर्मसमुच्चयवादी ? उत न ? इति हिरियण्णामहाशयैः (J. O. R. Vol. 12 Madras) पत्रिकायां विमृष्टम् ।।
१७. ब्रह्मनन्दी
एते ब्रह्मनन्द्याचार्या अद्वैताचार्येषु प्राचीनेषु अन्यतमाश्श्ङ्करात् पूर्वमासन्निति ज्ञायते । अदसीयाः ग्रन्था नोपलभ्यन्ते । तथापि अद्वैताचार्यैशशङ्करादर्वाक्तनैः प्रमाणीकृताः । 
भामतीव्याख्याने कल्पतरौ ``इयञ्चोगदानपरिणामादिभावा न विकारोभिप्रायेण" इत्यादि भामतीग्रन्थस्य व्याख्यानावसरे (Page 421 VVS Edn) ब्रह्मनन्दी निर्दिष्टः । संत्रक्षपशारीरके तृतीयपरिच्छेदे (श्लोक संख्या 217-221) ``आत्रेय वाक्यमपि संव्यवहारमात्रम्" इत्यादिना आत्रेय अत्रिगोत्रजः ब्रह्मनन्दी अनूदितः । मधुसूदनसरस्वत्या स्वटीकायां ``छान्दोग्यवाक्यकारेण ब्रह्यनन्दिना" इत्यवतारितम् । ``सिद्धन्तु निवर्तकत्वात्" इति वाक्यं शङ्कराचार्येः माण्डूक्योपनिषदां भाष्ये वैतथ्यप्रकरणे ``न निरोधो न चोत्पत्तिः" इति कारिकाविवरणावसरे निर्दिष्टम् । इष्टसिद्धौ (Page 72) ``सिद्धन्तु निवर्तकत्वादिति चोक्तं वाक्यं ज्ञानोत्तमकृतायां इष्टसिद्धिटीकायां ब्रह्मनन्दीयमिति निर्दिष्टम् । आनन्दगिरिणा योगवासिष्ठव्याख्यात्रा आनन्दबोधेन च द्रविडशब्देन ब्रह्मनन्दी निर्दिश्यते । पञ्चपादिकाविवरणेऽष्टमवर्णके ``सिद्धन्तुनिवर्तकत्वात्" इति वाक्यं प्रमाणत्वेन स्वीकृतम् । खण्डनखण्डखाद्यव्याख्यायां विद्यासागर्यां सिद्धन्तु निवर्तकत्वादितिवाक्यं उदाहृतम् । नृसिम्हाश्रमिकृतायां संक्षेपपशारीरकटीकायां ``ब्रह्मतन्दिनापिछान्दोग्यषष्ठव्याख्यानावसरे उक्तम्" इति निर्दिष्टम् । रामतीर्थकृतायां संक्षेपशारीरकटीकायां ``ब्रह्मनन्दिनाप्याचार्येण छान्दोग्य भाष्ये उक्तम्" नन्दिकृत भाष्यकारः द्रविडाचार्य इति च निर्दिश्यते । यामुुुनाचार्येण ``आत्मसिद्धौ" ``आचार्यटङ्क - भर्तृप्रपञ्च - भर्तृमित्र भर्तृहरि ब्रह्मदत्त - शङ्कर-श्रीवत्साङ्क - भास्करादि विरचितसितासितविविधनिबन्धनश्रद्धाविप्रलब्धबुद्धयः न यथावत् अन्यथा च प्रति पद्यन्ते ।" इति टङ्क अपरिग्र ह्यत्वेन निर्दिष्टः । टङ्क एव ब्रह्मनन्दीति श्रुतप्रकाशिकाचार्येण वेङ्कटनाथेन तात्पर्यदीपिकायां ``अत्र भाष्यकारः ब्रह्मनन्दिवाक्यव्याक्याता द्रविडाचार्यः" इति द्रविडाचार्यव्याख्येयग्रन्थकर्त टङ्काख्यः ब्रह्मनन्दीत्युक्तम् ।।
तस्मात् ब्रह्यनन्द्यभिन्नः टङ्काख्योऽयं अपरिग्राह्यत्वेन यामुनाचार्येण निर्दिष्टः विविर्तवादावलम्बी नाद्वैतमतविरोधीति स्पष्टं प्रतीयते । अनेन निर्मितः छान्देग्यवाक्यनामा ग्रन्थस्तु न कुत्रापि लभ्यते ।।
१८. भर्तृप्रपञ्चः
भर्तृप्रपञ्चोऽयं वेदान्तसाहित्ये शनैश्शनैर्म्लानयशास्सञ्जातः . शङ्कराचार्यसिद्धान्तात् भर्तृप्रपञ्चसिद्धान्तःभिद्यते । भर्तृप्रपञ्चः भेदाभेदवादी । शङ्कर अभेदवादी । भर्तृप्रपञ्चः ज्ञानकर्मसमुच्चयवादी । शङ्कराचार्यः ज्ञानवादी । मतस्य दर्शनग्रन्थानाञ्च जीवेश्वराणांं आत्मनश्च प्रतिपादने एव तात्पर्यमिति भर्तृप्रपञ्चः । भर्तृप्तपञ्चसिद्धान्तः प्रमाणसमुच्चयताम्नापि व्यवहर्तुं शक्यते । भर्तृप्रपञ्चः भोग एव मोक्षहेतुः न वैराग्यम् , वस्तुतत्वानुभव एव विरक्तेस्मुगमः पन्था इति चाभिप्रैति । ``तत्वमसि" वाक्यात् ``आत्मानमेवलोकमुपासीत" इति वाक्यमेव भर्तृप्रपञ्चमते उपादेयतरमिति ज्ञायते ।
ज्ञानकर्मसमुच्चयवादी भर्तृप्रपञ्चोऽय द्वैताद्वैतवादीति च निश्चयः । शङ्करभगवत्पादैरयं बृहदारण्यकभाष्ये ``औपनिषदम्मन्य" इति नाम्नानूद्य खण्डितः । आनन्दगिरिणापि भाष्यव्याख्यायां तत्र तत्र निर्दिष्टः । तस्मात् नायं विशुद्धाद्वैतवादी परन्तु अद्वैतैकदेशीति परं वक्तुं अर्हः । अनेन कठोषनिषदां बृहदाण्यकस्य च भाष्यमारचितं स्यादिति ज्ञायते ।
एनमधिकृत्य ``इण्डियन् आण्डिक्वैरिपत्रिकायां (I. A. Part 53 Page 77, 1924) हिरियण्णामहाशयेन सविस्तरं प्रतिपादितम् ।"
१९. भर्तृहरिः
वेदविदामलङ्कार इति प्रसिद्धोऽयं भर्तृहरिः ``नचागमादृते धर्मस्तर्केण व्यवतिष्ठते" इति वाक्यपदीयब्रह्मकाण्डे वदन् स्वस्य वैदिकधर्मावलम्बित्वं प्रकटयति । वाक्यपदीयकर्ता भर्तृः रिः वसुरातशिष्य इति ज्ञायते । ``ईस्टिङ् भारत यात्रा" ग्रन्थानुसारं भर्तृहरिसमयस्सप्तमशतकापरार्धावधिक (600-700 A. D.) इति निर्णीयते । भारतपण्डितमण्डलीप्रसिद्धा तु कथा- ``भर्तृहरिः विक्रमादित्यभ्राता" इति । तन्त्रवार्तिके कुमरिलभट्टः वाक्यपदीयं वाक्यं खण्डयति (1 - 3 - 871) काशिकायां (4 - 3 - 88) वाक्यपदीयः निर्दिष्टः । तस्मात् ताभ्यां पूर्वतन इति न संशयः । कुन्हन राजामहाशयस्तु (I. H. Q. Vol. XIV) भर्तृहरि पञ्चमशतकीयं वदन्ति । म. म. कुप्पुस्वामि शास्त्रिणः ब्रह्मसिद्धि भूमिकायां षष्ठशतकापरार्धादारब्धे सप्तमशतकापरार्धावधिके काले भर्तृहरिरासीदिति प्रवदन्ति । शबर स्वामिनोऽपि प्राचीनोऽयमिति भगवद्दत्तजीकृत वैदिकवाङ्मयेतिहासे दृश्यते ।
यामुनाचार्येण सिद्धित्रये भर्तृहरिकृतस्य सूत्रव्याख्यानस्य अनुपादेयत्व प्रदर्शनात् भर्तृहरिणापि ब्रह्मसूत्रवृत्तिः कृताह इति ज्ञायते । केचित्तु शब्दाद्वैत वादिनमेनं वदन्ति । वाक्यपदीयब्रह्मकाण्डमेव भर्तृहरिणः वेदान्तित्वे प्रमाणम् ।
२०. वाल्मीकिः (योगवासिष्ठकारः)
अध्यात्मविद्यायाः अद्वैतवेदान्तसिद्धान्तस्य च प्राचीनतमोऽयं ग्रन्थः योगवासिष्ठमिति मतिरस्माकम् । रामतीर्थस्वामिनः ग्रन्थमेनं भूमण्डलान्तर्गतेषु ग्रन्थेषु अत्युत्तमं ब्रह्मसाक्षात्कारकरञ्चेति वदन्ति । प्रस्थानत्रथी साधनावस्थोपयोगिनी । योगवासिष्ठन्तु सिद्धावस्थायामपि पठनार्हं ग्रन्थरत्नम् । ग्रन्थश्चायं अनेकदृष्टान्तोपाख्यानादिभिर्युक्तिभिश्च अद्वैतसिद्धान्तं प्रतिपादयति ।
ग्रन्थस्यास्याद्भुतस्य रचनाविषयेऽस्ति महान् विदुषां मतभेदः । चित्तशुद्धि समुत्पादनाय पूर्वरामायणम्, सञ्जातचित्तशुद्धेः पुरुषस्य जिज्ञासाशान्त्यै आत्मानात्मविवेचनपरं अद्वैतसिद्धान्तकोशभूतमिंद उत्तररामायणापराभिघं योगवासिंष्ठ वाल्मीकिना प्रणीतमिति तु साम्प्रदायिकी पण्डितमण्डितवार्ता । ``ऋषिभिर्बहुधागीत" मिति गीताया (XIII. 3.) ऋषिभिरित्यस्य व्याख्यानावसरे ``वसिष्ठादिभिरिति" शाङ्करं भाष्यम् इति च प्रमाणं प्रवदन्ति । यदि वाल्मीकिरेवास्य कर्ता स्यात् तर्हि वाल्मीकिस्सुकन्याच्यवनयोः पुत्र इति पौराणिकी प्रसिद्धिरिति कालादिनिर्णयो न कर्तुं शक्यते ।
आधुनिकेषु प्राच्यप्रतीच्यभाषाप्रवीणेषु विमर्शकवरेषु च अस्य रचनाकाल विषये महान् मतभेद आशयभेदश्चवरीवर्ति । डाक्टर फर्कुहार प्रभृतय आधुनिकाः द्वादशत्रयोदशशतकमध्यमस्य रचनाकाल इति मन्वते । शिवप्रसाद भट्टाचार्यास्तु (900-1110 A. D.) दशमैकादशशतकमध्यमस्य रचनाकाल इति वर्णयन्ति । भारतीय साहित्येतिहासलेखकानां जर्मन पण्डितानां विण्टर्निट महाशयानां नवमशतकमिति । दिवानजी महाशयास्तु योगवासिष्ठरचनास्थानं काष्मीरदेशः, योगवासिष्ठरचनाकालः दशमशतकमिति निश्चिन्वन्ति । आत्रेयपहाशयास्तु कालिदामात् अर्वावीने भर्तृहरिगौडपादशङ्करसुरेश्वरप्रभृतिभ्य अद्वैतवेदान्ताचायभ्यः प्राचीने न काले योगवासिष्ठं प्रणीतमिति सिद्धान्तयति । डाक्टर. वे. राघवमहोदयाश्च गीतायाः योगवासिष्ठस्य च साम्यप्रतिपादकानि द्विनवतिंसख्याकानि उद्धरणानि प्रतिपाद्य राजशेखरात् अनन्तरभाविनि नवमशतकादारब्धे त्रयोदशशतकान्ते च काले योगवासिष्ठं प्रणीतमिति जर्नल आफ ओरियण्टल पत्रिकायां प्रतिपादयन्ति । दासगुप्तमहाशयस्तु नवमशतकीयेन काष्मीरिणा अभिनन्देन लघुयोगवासिष्ठनामा ग्रन्थः प्रणीत इति तत्कालात्पूर्वतनोऽयं ग्रन्थ इति (HIP Vol II) प्रतिपादयति ।
शङ्कराचार्यकृतविवेकचूडामणौ, विद्यारण्यस्वामिकृतपञ्चदश्यां, जीवन्मुक्तिविवेके च, भर्तृहरिकृतवाक्यपदीयवैराग्यशतकयोः, प्रकाशानन्दानां वेदान्त सिद्धान्तमुक्तावल्यां च योगवासिष्टीयश्लोकाः दृश्यन्ते । उपनिषत्स्वपि योगवासिष्ठीयश्लोकाः दृश्यन्ते । कतिपयोपनिषदः योगवासिष्ठश्लोकसंग्रहा एवेति च आत्रेयमहाशयेन सविस्तरमुपपादितम् ।
योगवासिष्ठग्रन्थे न कोऽपि ग्रन्थः ग्रन्थकारो वा उद्धृतः प्रमाणत्वेन निर्दिष्टश्च । द्वित्रिस्थलेषु परं `बुद्धः' जिनः इत्यादिशब्दाः (Page 33, 669, 729, Vol I NSP Edn) दृश्यन्ते । परन्तु तत्रापि व्याख्यात्रानन्दबोधेन ``प्रव्रजितः" इत्येवार्थः क्रियते । प्रथमभागे 454 तमे पुटे ``यत्प्राप्तं शङ्करादिभिः" इत्यादिना शङ्करः निर्दिष्टः । अत्र व्याख्यात्रा न व्याख्यातम् । कोऽयं शङ्करः ? किं शङ्कर भगवत्पादः ? उत भगवान् भवानीपतिः ? प्रकरणवशात्सु शङ्करभगवत्पाद इत्येवास्माकं प्रतीयते । एवस्मिन्नेव स्थले 714 तमे पुटे प्रथमभागे ``बृहदारण्यकादिषु" इति बृहदारण्यकोपनिषत् नाम्ना निर्दिश्यते । प्रथमभागे 594 तमे पुटे ``श्रीशैलाचार्यपुत्रेण" इति श्रीशैलाचार्यः निर्दिष्टः । एवमादिप्रमाणैश्शङ्करादनन्तरभावित्वमस्य ग्रन्थस्य वक्तुं शक्यते परन्तु प्रक्षिप्ता इमे श्लोका इति वज्रकुठारप्रक्षेपभीत्या न तथापि वक्तुं शक्यते ।
``अयं प्रपञ्चो मिथ्यैव सत्यं ब्रह्माहमद्वयम् । अत्र प्रनाणं वेदान्ता गुरवोऽनु भवस्तथा ।" (Page 181 Vol I) इत्यादिभिरसंख्यैः पद्यैः जगन्मिथ्यात्वं जीवब्रह्मैक्यं, ब्रह्मणो नामरूपबहिर्भूतत्वं, जीवन्मुक्ति, अजातवादः, अनिर्वचनीयतावादः, इत्यादय अद्वैतसिद्धान्ताः काव्यशैल्यां प्रतिपादिता इति ग्रन्थोऽयं अद्वैतवेदान्तसाहित्यास्यादिमं महाकाव्यामित्येव मदीयस्मिद्धान्तः । एतादृशे अद्वैतवेदान्तशास्त्रकाव्ये आध्यात्मिकमहाकाव्यापरनामके अद्वैतसिद्धान्तवर्णना, जगन्मिथ्यात्ववर्णना च एतादृशी वर्तते यया प्रभाविताः केचन विद्वासः योगवसीष्ठे बौद्धमतस्य सर्वश्न्यवादस्य प्रभावं वर्णयन्ति । परन्तु नैतत्सत्यम् । योगवासिष्ठस्य रचयिता न साधकः । परन्तु अद्वैतानान्दानुभवी महान् सिद्धः । तादृसस्य सिद्धस्य स्वीयानुभवैकप्रमाणे ग्रन्थेऽस्मिन् अद्वैतसिद्धान्ताः सिद्धावस्थानुकूला एव प्रतिपादिता इति तु निश्चयः । ग्रन्थोऽयं तात्पर्यप्रकाशव्याख्यासहितः निर्णयसागरमुद्रणालये मुद्रितः । अस्य व्याख्याः एतत्सम्बद्धाश्च ग्रन्थाः-
(क) अद्वयाख्याकृता - योगवासिष्ठपददीपिका । ग्रन्थोऽयं कल्कत्ता रायल आसियाटिका सूच्यां दृश्यते ।
(ख) आनन्दबोधकृतः - तात्पर्यप्रकाशः । ग्रन्थोऽयं निर्णयसागरमुद्रणालये मुद्रितः ।
(ग) अभिनन्दकृतः - योगवासिष्ठसंक्षेपः (लघुयोगवासिष्ठम्) ग्रन्थोऽयं आत्मसुखेन वासिष्ठचन्द्रिकाव्याख्यया व्याख्यातः । मुम्मुडिदेवेन संसारतरणि व्याख्यया च व्याख्यातः । ग्रन्थोऽयं सव्याख्यः निर्णयसागरमुद्रणालये मुद्रितः ।
(घ) महीधरकृतः - योगवासिष्ठसारः सव्याख्यः । ग्रन्थोऽयं बरोडापुस्तकालये लन्दनपुस्तकालये बाभ्बेविश्वविद्यालयहस्तलिखितपुस्तकालये च लभ्यते । 
(ङ) माधवसरस्वतीकृता - वासिष्ठपञ्चिका । ग्रन्थोऽयं अनन्तशयनपुस्तकालये लभ्यते ।
(च) काष्मीरपण्डिकृतम् - ज्ञानवासिष्ठम् ।
(छ) कृष्णय्यकृतः - ज्ञानवासिष्ठसारसमुच्चयः । इमौ द्वावपि ग्रन्थौ मद्रासराजकीयहस्तलिखितपुस्तकालये लभ्येते ।
%%% Chart
एतेषु केचन ग्रन्थास्तेलुगुलिप्यामेव सन्ति ।
२१. शुकः
शुकाचार्यापरनामानः बहवो वेदान्तिन आसन्निति ज्ञायते । कुत्रचित् शुकभगवत्पाद इति, कुत्रचित् शुकयोगीति नामानि बहुनि श्रूयन्ते । शुकाष्टाककर्ता शुक अन्यः ब्रह्मसूत्रभाष्यकर्ता शुक अन्य इत्येव ज्ञातुं पार्यते । अत्र प्रबलतरप्रमाणानि तु नैवोपलभ्यन्ते । यदि अद्वैतसम्प्रदायप्रवर्तकाचार्येषु परिगणितश्शुकाचार्यस्स्यात् तर्हि शुकाष्टककर्ताय महाभारतादिप्रसिद्धः कृष्णद्वैपायनात् शुकीरूपधारिण्यां घृताच्यांं जातः महान् ज्ञानीति सिघ्यति । एवञ्चास्य कालादिकथनं दुश्शकम् । इदन्तयाऽनिर्णीतत्वात् ।
(क) शुकाष्टकम् - व्यासपुत्राष्टकमित्यपरनामाय ग्रन्थः जीवन्मुक्तमहिमानं प्रदर्शयति । ग्रन्थश्चायं मुद्रितः । अस्य व्याख्यापि गङ्गाधरेन्द्रसरस्वतीकृता नासिकसूच्यां दृश्यते ।
(ख) ज्ञानबोधः - अमुद्रितोऽयं पूर्णग्रन्थ अडयारपुस्तकालये (9. B. 23) सरस्वतीमाहालये च लभ्यते ।
(ग) ब्रह्मसूत्रवृत्तिः - अस्य कर्ता शुकभगवत्पादाचार्य इति निर्दिष्टम् । ग्रन्थोऽयमुद्रितः पञ्जाब सूूच्यां 719 दृश्यते ।
२२. सनत्सुजातः
ब्रह्मणो मानसः पुत्रोऽयमिति महाभारतादिषु प्रसिद्धिः । तस्मादस्य कालनिर्णयो न शक्यते । महाभारतयुद्धस्य कालः क्रिस्तो पूर्वमिति विविधमतभेदेन विमर्शः कृतः । तस्मात् क्रिस्तोः पूर्ववर्ती युगान्तरीयस्सनत्सुजात इति परं वक्तुं शक्यते ।
(क) सनत्सुजातीयम् - मुद्रीतोऽयं ग्रन्थः बहुत्र मुद्रणालयेषु । अस्य व्याख्या शङ्करभगवत्कृता च वाणीविलासमुद्रणालये आनन्दश्रममुद्रणालये च मुद्रिता ।
एवं व्यासात् शङ्करभगवत्पादाच्च प्राक्तनाः प्रायशः सर्वे वेदान्ताचार्या निरूपिताः । अद्वैतसम्प्रदायस्य प्रवर्तकाचार्याणां परम्परा नारायणादारब्घेति प्रसिद्धा । प्रसिद्धा चेयं परम्परा-
``नारायणं पद्भभुवं वसिष्ठं शक्तिञ्च तत्पुत्रपराशरञ्च ।
व्यासं शुकं गौडपदं महान्तं गोविन्दयोगीन्द्रम् ।
अथास्य शिष्यं श्री शङ्कराचार्यमथास्य पद्मपादञ्च हस्तामलकञ्च शिष्यम् ।
तं तोटकं वार्तिककारमन्यान् अस्मद्गुरून् सन्ततमानतोऽस्मि ।" इति पद्येन ।
एतेषु वेदान्ताचार्येषु पराशरशक्त्योर्विषये न किमपि ज्ञातुं शक्यते । तस्मात् इतः परं बादरायणव्यासादारब्धाः क्रैस्तवीयर्विशतिशतकान्ताः अद्वैतवेदान्तसाहित्ये प्रसिद्धतमप्रमाणभूतग्रन्थप्रणेतारः मूलग्रन्थव्याख्याग्रन्थप्रणेतराश्चाद्वैताचार्याः कालक्रमेण निरूप्यन्ते ।
२३. बादरायणव्यासः (400 A. D. कलात् प्राक्)
अष्टादशपुराणानां रचयिता कृष्णद्वैपायनापरनामा व्यास एव बादरायण व्यास इति ब्रह्मसूत्राणां रचयितेति च साम्प्रदायिकाः । केचित्तु ब्रह्मसूत्रकारः बादरायणव्यासोऽयं अष्टादशपुराणकर्तुः कृष्णद्वैपायनाद्भिन्न इति वर्णयन्ति । पाणिनेरप्ययं ब्रह्मसूत्रकारः प्राक्तनः । यतः ``पाराशर्यशिलालिभ्यां भिक्षुनटसूत्रयो" रिति पाणिनिना ब्रह्मसूत्राणि भिक्षुसूत्रनाम्ना निर्दिष्टानि । ब्रह्मसूत्रपदैश्चैव हेतुमद्भिर्विनिश्चितैः (गीता. 13.4) इति बादरायणकृतान्येव ब्रह्मसूत्राणि विनिर्दिष्टानि, इति भगवद्गीताव्याख्यातृश्रीधरस्वामिनामभिप्रायः । केचित्तु गीतायां निर्दिष्टानि ब्रह्मसूत्राणि नैतानि, परन्त्वन्यानि तानि च नष्टानीति वदन्तः पुराणादिकर्तुः व्यासात् ब्रह्मसूत्रकारमन्यं मन्वते ।
सर्वकारणकारणं एकमेवाद्वितीयं सत्यज्ञानानन्दस्वरूपं अनन्तं असङ्गम्, नित्यं च ब्रह्म, तज्जानान्मोक्ष इत्यादिकं सर्वं अलौकिकविषयान्तर्गतम् । एतादृशेऽलौकिक विषये अनाद्यपौरुषेयश्रुतिरेव प्रमाणम् । व्यासाचार्यास्तु अलौकिकविषये अनाद्यपौरुषेयश्रुतिमेव प्रमाणमावेदयन्ति । बादरायणेन ब्रह्मसूत्रमुखेन तत्वान्युपदिश्यन्ते । तादृशं श्रुतिसम्मतमद्वैतमेवेति प्रदर्शयितुं भगवान्नारायण एव कृष्णद्वैपायनापरबादरायणव्यासो भूत्वा ब्रह्मसूत्राणि विरचयामास । बादरायणव्यासोऽयमद्वैतीत्यत्र प्रमाणानि कानिचन दृश्यन्ते-शाण्डिल्यभक्तिसूत्रे 30 आत्मैकपरां बादरायणः" इति दृश्यते । एवं बादरायणव्यास एव अष्टादशपुराणानां कर्ता, स च अधिकारिभेदं मनसि कृत्वा मन्दाधिकारिणां तत्वोपदेशाय अरूपं निर्गुणं अनिर्वचनीयं च आत्मस्वरूपं सरूपं सगुणं स्तुतिविषयं कृत्वा पुराणानि रचयामासेति वदन्त पद्यमिदं व्यासकृतत्वेन प्रसिद्धं प्रमाणयन्ति -
``रूपं रूपविवर्जितस्य भवतो ध्यानेन यत्कल्पितम्,
स्तुत्यानिर्वचनीयताखिलगुरो दूरीकृता यन्मया ।
व्यापित्वञ्च निराकृतं तु भवतो यत्तीर्थयात्रादिना
क्षन्तव्यं जगदीश तद्विकलतादोषत्रयं मत्कृतम् ।।" इति ।
तस्मात् अष्टादशपुराणादिकर्तुः ब्रह्मसूत्रकर्तुश्च बादरायणव्यासस्य कालादि निर्णेतुं न शक्यते । तथापि प्रो. हिरियण्णामहाशयास्तु ब्रह्मसूत्रनिर्माणकालः (400 A.D.) इति वदन्ति ।
(क) ब्रह्मसूत्राणि - शारीरकमीमांसासूत्रमित्यपरनामायं ग्रन्थः समन्वयाविरोधसानफलाख्यैश्चतुर्भिरध्यायैः पूर्णः । प्रत्यध्यायं चत्वारः पादा विद्यन्ते । प्रतिपादं बहून्यधिकरणानि । प्रत्यधिकरणं भिन्नविषयम् । एषु सूत्रेषु बादरायणेन उपनिषदां तदर्थानां च सङ्कलनं कृतम् । शङ्कराचार्यमतेन सूत्रसमष्टिसंख्या 555,अधिकरणसंख्या 192 । अस्य भाष्यं शङ्करभगवत्पादैः कृतं शारीरकमीमांसाभाष्यमित्याख्यम् । यदाधारं कृत्वा परश्शतानि ग्रन्थरत्नानि अद्वैतवेदान्तसाहित्ये प्रकाशन्ते ।
(ख) सिद्धान्तदर्शनम् - पूर्वोत्तराम्नायभेदात् द्विविधा हि मीमांसा । तत्रोत्तरमीमांसा पुनर्द्वेधा वादिबुबुत्सुप्रतिपादनभेदात् । तत्राद्या ब्रह्मसूत्राख्या, द्वितीया सिद्धान्तसूत्रात्मिका । एवञ्च ``अथातो ब्रह्मजिज्ञासा" इत्यादिकं वादिप्रतिपादनाय कृतम् । इदन्तु ``सिद्धान्तदर्शनं" बुबुत्सुप्रतिपादनाय कृतमिति विशेषः । सूत्ररूपेऽस्मिन् ग्रन्थे समग्राद्वैत्तसिद्धान्ताः प्रतिपाद्यन्ते । केचित्तु साम्प्रदायिका अपि अस्य ग्रन्थस्य बादरायणव्यासप्रणीतत्वे सन्दिह्यन्ति । ग्रन्थोऽयं आनन्दाश्रम मुद्रणालये मुद्रितः । अस्य भाष्यं विश्वदेवेन रचितं ``निरञ्जनभाष्या"ख्यमपि मुद्रितम् ।
(ग) भगवद्गीता - (महाभारतान्तर्गता) एतामधिकृत्य स्विस्तरं गीताप्रस्थाने प्रतिपादितमिति नेह प्रतन्यते ।
२४. गौडपादाचार्यः (500 A.D.)
गौडपादाचार्या इमे शुकमुनीन्द्रशिष्या इति, अद्वैताचार्यपरम्परायां महनीयतमा इति च ``नारायणं पद्भभुवमि"त्यादिश्लोकात् ज्ञायते । नृसिम्हतापनीयोपनिषदां गौडपादकृते व्याख्याने ``इति श्रीपरमहंसपरिव्राजकाचार्यश्रीमच्छुकमुनीन्द्रशिष्य गौडपादविरचिते उत्तरतापनीयोपनिषद्विवरणे प्रथमः खण्डः, नवमः खण्डः" इति (D. 581, 582 MGOML) ग्रन्थेऽमुद्रिते दृश्यते । एवं श्वेताश्वतरोपनिषदां शाङ्करभाष्ये ``तथाच शुकशिष्यो गौडपादाचार्यः" (Page 30, ASS 17) इति दृश्यते । एवमेव लक्ष्मणशास्त्रिविरचिते गुरुवंशकाव्ये (12. VSS) दृश्यते । तस्मान् शुकशिष्यो गौडपादाचार्य इति नीश्चीयते ।
गौडपादाचार्यस्य स्थानं नाद्यापि निश्चितम् । विषयेऽस्मिन् विभिन्ना एव विचारा दृश्यन्ते । परन्तु मान्डूक्यकारिकाशङ्करभाष्यव्याख्याने आनन्दगिरीये अलातशान्तिप्रकरणस्थस्य ``तं वन्दे द्विपदां वर"मिति पद्यांशस्य व्याख्याने एवं दृश्यते -``आचार्यो हि पुरा वदरिकाश्रमे नरनारायणाधिष्ठिते नारायणं भगवन्तमभिप्रेत्य तपो महदतप्यत ।" इति (Page 157 ASS 10) दृश्यते । तस्मात् बदरिकाश्रम एव गौडपादस्य स्थानमिति ज्ञायते । सप्तदशशतकीयेन बालकृष्णान्दसरस्वत्या स्वीये शारीरकमीमांसाभाष्यवार्तिके तु ``गौडचरणाः कुरुक्षेत्रगता हीरारावतीनदीतीरभवगौडजातिश्रेष्ठाः, देशविशेषभवजातिनाम्नैव प्रसिद्धाः द्वापरयुगमारभ्यैव समाधिनिष्ठत्वेन आधुनिकैरपरिज्ञातविशेषाभिधानास्सामान्यनाम्नैव लोकविख्याता" (Page 6. AS.I) इति प्रतिपादितम् । एवञ्च गौडपादः कुरुक्षेत्रवासी गौडजात्युत्पन्न इति गौडपादीयनामान्तरापरिज्ञाने च कारणं सूचितम् । केचित्तु गौडपादाचार्यं गौडदेशभवं वदन्ति । अत एव देशनाम्ना तेषां व्यवहारः । यथागौडब्रह्मानन्द इत्यादि । एवञ्चैतत्पक्षे गौडपादः बंगालदेशानां उत्तरभागवर्तीति सिध्यति ।
भारतीयाद्वैतवेदान्तपरम्पराप्रामाण्येन गौडपादश्शुकशिष्यः, शङ्करश्च गौडपादशिष्यः, गौडपादेनानुगृहीतश्चेति शङ्करात्पूर्वतनो वा शङ्करकालपर्यन्तजीवी वेति निश्चीयते । यदि वयं म. म. कुप्पुस्वामिशास्त्रिमहाशयानां सिद्धान्तमनुसृत्य 632-661 A.D. कालवर्तिनं शङ्करमभ्युपगच्छामस्तर्हि तैरेव प्रतिपादितसिद्धान्तमनुसृत्य गौडपादकालः (520-620 A.D.) इति, इच्छामात्रशरीरत्यागिनां गौडपादाचार्याणां कालश्शङ्कराचार्यानुग्रहपर्यन्तमिति वा स्वीकर्तव्यम् । एतेन शङ्करदिग्विजयादिवचनञ्च सङ्गतं भवति । विधुशेखरभट्टाचार्यास्तु स्वसम्पादिते `आगमशास्त्र' ग्रन्थोपोद्धाते एवं वदन्ति -``द्वितीयशतकादारभ्य चतुर्थशतकपर्यन्तानां बौद्धपण्डितानां ग्रन्थस्य गौडपादकारिकायाश्च शब्दसाम्यदर्शनात् गौडपादस्तदर्वाग्भव इति तथाच तन्मतरीत्या गौडपादकालः (500 A.D.) इति सिध्यति । यद्येवं गौडपादाचार्यस्य शङ्कराचार्यप्राचार्यत्वं कथम् ? किमन्योऽयं गौडपादः ? उतान्योऽयं शङ्करः ? न वा शङ्करः गौडपादेनानुगृहीतः, नापि प्रशिष्य इत्यभ्युपगम्य शङ्करदिग्विजयादिग्रन्थानामप्रामाण्यं स्वीकर्तव्यम् ? आहोस्वित् शङ्कराचार्यानुग्रहकालपर्यन्तजीवी गौडपादः ? इत्यादयस्संशयविशेषाः स्वतस्समुद्भवन्ति ।"
गुरुपादहालदारस्तु ``वृद्धत्रय्यां" (Page 307) गौडपादाचार्यः गोविन्दभगवत्पादशिष्यः शङ्कराचार्यकामदेवभूपालयोः परमगुरुः वेदान्तसम्प्रदायप्रवर्तकः, माण्डूक्यकारिकाकृदद्वैती (700 A. D.) कालवर्तीति प्रतिपादयति ।
(क) माण्डूक्यकारिका - (ASS. 10) गौडपादकारिकाभिधेऽस्मिन् माण्डूक्योपनिषदां व्याख्यात्मके ग्रन्थे चत्वारि प्रकरणानि सन्ति । तत्र प्रथमेऽऽगमाख्यप्रकरणे एकोनत्रिंशत्, द्वितीये वैतथ्याख्ये अष्टात्रिंशत, तृतीयेऽद्वैताख्ये अष्टाचत्वारिंशत्, चतुर्थेऽलातशान्तिप्रकरणे शतमिति 215 कारिकास्सन्ति । मुद्रितश्चायं ग्रन्थ आनन्दाश्रममुद्रणालये ।
आगमप्रकरणम् - इदमागमप्रकरणं माण्डूक्योपनिषदां भावार्थरूपं आगममूलकत्वात् अन्वर्थनाम । प्रकरणेऽस्मिन् अकारोकारमकारैः प्रतिपादितेभ्यः वैश्वानर हिरण्यगर्म-ईश्वरेभ्यः, जाग्रत्स्वप्नसुषुप्त्यवस्थाभ्यश्च भिन्नं तदनुगतं साक्षिरूपं च परमात्मतत्वं ``तुरीय" इति नाम्ना वर्णितम् ।
वैतथ्यप्रकरणम् - द्वितीयेऽस्मिन् प्रकरणे दृश्यप्रपञ्चस्य मायामयत्वं मिथ्यात्वञ्च सयुक्तिकं साधितम् । आत्मा एक एव नित्यः, तस्मिन् विविधकल्पनावशात् प्रपञ्चस्तोत्पत्तिरिवि विकल्पो भवति । अस्य मूलकारणं माया । मायाकल्पितजगतः गन्धर्वनगरवत् असत्यत्वमिति प्रतिपाद्य ``न निरोधोनचोत्पत्ति" रित्यादिना अखण्डचिद्धनानन्दआत्मतत्वादन्यस्यासत्वं प्रतिपादितम् ।
अद्वैतप्रकरणम्-तृतीयेऽद्वैताख्यप्रकरणेऽस्मिन् अनेकाभिस्सुदृढाभिर्युक्ति भिरद्वैतत्वं साधितम् । आत्मनि सुखदुःखभावना नितरां असङ्गता । यथा बालाः धूलिधूमादिसंसर्गेणाकाशं मलिनमामनन्ति, वस्तुतः यथा च आकाशो मालिन्यशून्यः तथैवात्मनोऽपि सुखित्वदुःखित्वकथनं बालबुद्धिविलासतुल्यमिति प्रतिपादितम् । असङ्गोह्यात्मा । माया हि द्वैतकल्पनायाः कारणम् । अमृतस्य मर्त्यत्वं, मर्त्यस्य अमृतत्वञ्चासङ्गतम् । अत अमृतस्यात्मनः यदि उत्पत्तिस्स्वीक्रियते तर्हि मर्त्यत्व धर्म आपद्येत इति आत्मनः उत्पत्ति - जातिः नास्ति इति प्रतिपादितम् । अयमेव गौडपादाचार्याणां अजातिवादः । एतच्च 1-17, 2-31, 32, 3-4, श्लोकेषु प्रतिपादितम् । अयमजातिबादः गौडपादात् प्राचीनस्य बौद्धाचार्यस्य दिङ्नागस्य माध्यमिकवृत्तौ, पालिभाषाप्रणीतबौद्धग्रन्थेषु च समुपलब्धेस्ततो गृहीत इति केचिद्वदन्ति । पालीभाषाया अपि प्राचीनासूपनिषत्सु ``अजायमानो वहुधा व्यजायत" इत्यादिदर्शनात् तेषामुक्तेरनुपपत्तौ भारतीयाः प्रमाणम् ।
``अलातशान्तिप्रकरणम्"- चतुर्थेऽस्मिन् प्रकरणे यथा अलाते भ्रमिते सति गोलाकारप्रतीतिर्जायते परन्तु सा गोलाकारभ्रमणजन्या एव न वस्तुतः, एवं जगदादि मायाकल्पितमेव । मनसो व्यापारादेव तस्योत्पत्तिः, मनसः निरोघे च स नास्त्येव । यथा च भ्रमणादिक्रियाशान्तौ गोलाकारकप्रतीतेश्शान्तिः, एवं मनस अमनीभावात् जगतश्शान्तिः । जगदुत्पत्तिलयौ प्रतीत्यप्रतीती उभावपि भ्रान्तिजनितावेव । परमार्थतः परमार्थतत्वं पारमार्थिकमिति प्रतिपादितम् । अद्वैतवेदान्तस्य प्राणभूताऽनिर्वचनीयख्यातिरपि प्रकरणेऽस्मित् प्रतिपादिता । ``विपर्यासात् यथा जाग्रदित्यादिना (4-41) एवं ``न निर्गतास्ते विज्ञानादित्या" दिना (4-52) ``उभेह्यन्योन्यं दृश्येते" इत्यादिना च (4-67) पद्येन प्रदर्शिता ।
प्रकरणस्यास्य भाषा ``विज्ञप्ति" रित्यादिपारिभाषिकशब्दैः पूर्णा । एवं मङ्गलाचरणश्लोके ``तं वन्दे द्विपदां वरम्" इत्यत्र द्विपदां वरशब्दश्च प्रयुक्तः । एते शब्दाः बुद्धमतग्रन्थेषु दृश्यन्त इति केचन बुद्धमतमेव गौडपादः वेदान्तापदेशेन प्रतिपादयतीति प्रच्छन्नबौद्धा अद्वैतिन इति वदन्ति ।
परन्तु शब्दसाम्यं नात्र प्रमाणमकिञ्चित्करञ्च । यत अध्यात्मशास्त्राणां पारिभाषिकशब्दाः न केवलं बौद्धानां स्वम् । परन्तु ते सर्वदर्शनसामान्याः । तेषां प्रयोगे यथा गौडपादस्य तथा बौद्धानां यथा बौद्धानां तथा गौडपादस्येति सर्वेषामधिकास्समस्ति । द्विपदांवर शब्दस्य पुराणादिष्वपि भूरिशः प्रयोगः दृश्यते । भारतरामायणादीनां बुद्धादपि प्राचीनत्वं प्रसिद्धमेव । नलभीमार्जुनभीष्मादिषु शब्दोऽयं प्रयुक्तः दृश्यते । महाभारते नारायणीयपर्वाध्याये द्विपदांवरार्थकं द्विपदां वरिष्ठपदं प्रयुक्तं दृश्यते । न वा एतत्पदं कोषग्रन्थेषु बुद्धपरत्वेन व्याख्यातम् । तस्मात् यौगिकश्शब्द एवैषः न तु योगरूढः ।
माण्डूक्यकारिकाचेयं शङ्करभगवत्पादैर्व्याख्यातम् । आनन्दगिरिव्याख्योपेतं भाष्यं आनन्दाश्रममुद्रणालये मुद्रितम् । स्वयम्प्रकाशानन्दसरस्वतीकृता मिताक्षरानाम्नी माण्डूक्यकारिकाव्याख्या वाराणस्यांं (BSS 48) मुद्रिता । उपनिषद्ब्रह्मकृता व्याख्या अडयारपुस्तकालये मुद्रिता । अनुभूतिस्वरूपाचार्यकृतं गौडपादीयभाष्यटिप्पणं (R. 2911 MGOML) लभ्यते । अज्ञातकर्तृकः गौडपादीयविवेकनामा ग्रन्थोऽपि (ई. 3882 d MGOML) लभ्यते । गौडपादाचार्यप्रणीतत्वेन प्रसिद्धाः ग्रन्थाः-
(ख) उत्तरगीताव्याख्या - ग्रन्थोऽयं तिरुपति सृच्यां (DCVORIT) अनन्तशयनपुस्तकालये (275 TCL) जयपुर पोटीखानासूच्यां (XXXIII 74/4) च दृश्यते ।
(ग) पञ्चीकरणवार्तिकम् - बरोडासूच्यां (13325 c BRD) दृश्यते ।
(घ) नृसिम्हतापनीयभाष्यम् - (D. 581 MGOML)
(ङ) अनुगीताभाष्यम् - ग्रन्थोऽयं नासिक सूच्यां दृश्यते ।
(च) श्रीविद्यारत्नसूत्रम् - (275 TCL)
(छ) दुर्गासप्तशती व्याख्या - ग्रन्थोऽयं तन्त्रदर्शनाचार्येण भास्कररायेण दुर्गासप्तशती व्याख्याने निर्दिष्टः ।
(ज) सुभगोदयः - (275 TCL)
(झ) सांख्याप्रवचन भाष्यम् ?
२५. मण्डनमिश्रः (750-850 A.D.)
मण्डनमिश्रोऽयं कुमरिलभट्टस्य शिष्यश्शङ्कराचार्यकाले प्रसिद्धः पूर्वमीमांसापण्डितः कर्मनिष्ठश्चेति प्रसिद्धिः । अस्यैव विश्वरूप इति नामन्तरम् । मण्डनमिश्रस्यैव शङ्कराचार्यात् आश्रमस्वीकारपूर्वकशिष्यत्वस्वीकारादनन्तरं सुरेश्वराचार्य इति नामेति सम्प्रदायविदः । ``जागोपि" महाशयेन नैष्कर्म्यसिद्धिभूमिकायां मण्डनमिश्र-विश्वरूपसुरेश्वराणां ऐक्यमङ्गीकृतम् । सप्तदशशतकीयेन बालकृष्णानन्दसरस्वत्या कृते शारीरकमीमांसाभाष्यवार्तिके च मण्डनमिश्रसुरेश्वरविश्वरूपाणामैक्यमेव वर्णितम् ।
दासगुप्तमहाशयास्तु सुरश्वरविश्वरूपावभिन्नौ मण्डनमिश्रस्तु अन्य एवेति (H. I. P. Vol. II) ग्रन्थे निर्दिशन्ति । हिरियण्णामहाशयास्तु (J. R. A. S. 1924) रायलासियाटिक सोसाइटि पत्रिकायाश्चतुर्विशतितमे भागे सुरेश्वरः मण्डनादन्य इति निश्चिन्वन्ति । म. म. कुप्पुस्वामिशास्त्रिणश्च स्वसम्पादितब्रह्मसिद्धि भूमिकायां सुरेश्वरब्रह्मसिद्धिकारयोस्सिद्धान्तगतभेदमुपवर्ण्य सुरेश्वरादन्यं ब्रह्मसिद्धिकारं मण्डनं वर्णयन्ति स्म ।
बिब्लियोथिकाइण्डिकासीरीजमुद्रितायां पराशरस्मृतिव्याख्यायां (Page 51) बृहदारण्यकवार्तिकात् उद्घृतम् । तच्चोद्धरणं विश्वरूपाचार्यकृतग्रन्थादित्युक्तम् । एवं विद्यारण्यैः विवरणप्रमेयसंग्रहे (Page 92) बृहदारण्यकवार्तिकात् (IV. 8.) उद्धरणं दत्तम् । तत्रापि सुरेश्वरः विश्वरूपशब्देनैव निर्दिष्टः । तस्मात् विश्वरूपसुरेश्वरावभिन्नौ, मण्डनस्त्वन्य इति निश्चयः ।
ब्रह्मसिद्धिकारेणानेन शङ्करात्पूर्वतनाः ग्रन्थाः प्रमाणत्वेन निर्दिष्टाः । वाचस्पतिमिश्रेण च ब्रह्मसिद्धिं प्रमाणं कृत्वा ब्रह्मतत्वसमीक्षा कृता । अत एव मण्डनपृष्ठसेवी वाचस्पतिरिति च प्रसिद्धिः । तस्मात् वाचस्पतिकालिको वा, तस्मात् पूर्वतनो वा भवितुमर्हति । शङ्कराचार्यसामयिक इति तु सम्प्रदायविदः । दासगुप्तमहाशयेन नवमशतकीयोऽयमिति बर्ण्यते । कुप्पुस्वामिशास्त्रिणस्तु (615-695 A. D.) इति सप्तमशतकीयं वर्णयन्ति ।
ब्रह्मसिद्धिः - ग्रन्थोऽयं मद्रासराजकीय पुस्तकालये (MGOMLS 4) सव्याख्यः मुद्रितः । अस्याः व्याख्या वाचस्पतिमिश्रकृता ``ब्रह्मतत्वसमीक्षा" चित्सुखाचार्यकृता ``अभिप्रायप्रकाशिका" आनन्दपूर्णविद्यासागरकृता टीकारत्नापरनामा ``भावशुद्धिः" शङ्खपाणिकृता ब्रह्मसिद्धिटीका चेति ग्रन्थाः वर्तन्ते ।
(ख) विभ्रमविवेकः - 162 पद्यैः पूर्णोऽयं ग्रन्थः पञ्चख्यातिव्याख्यात्मकः । मुद्रितश्चायं जर्नलआफ ओरियण्टल पत्रिकायां (J. O. R.) मद्रासनगरे । अन्येऽपि ग्रन्थाः मीमांसाशास्त्रे कृताः ।
26. शङ्करभगवत्पादः (788-820 A. D.)
शङ्कराचार्यवतारसमये धर्मपरिस्थितिः-
सनतनकालत एव प्रबलप्रमाणपूर्वकं आत्यन्तिकनिःश्रेयसाधिगमसाधनत्वेन श्रुतिपुराणभगवद्गीतोपनिषद्भिः महद्भिराचार्यैश्च निरूपितोऽयमद्वैतसिद्धान्तः । स च क्रैस्तवीयचतुर्थशतकात्पूर्वमुत्पन्नेन शाक्यमुन्यपरनाम्ना गौतम बुद्धेन तात्कालिकपरिस्थित्यनुसारं प्रवर्तितस्य बहूनां धारापतीनां प्रवेशात् अतिमहर्ती वृद्धिमाप्तस्य बौद्धमतस्य, तच्छाखान्तरस्य जैनमतस्य च प्रसरणेन प्रसारणेन च, धर्मपाखण्डानां स्वार्थपराणां धर्मकर्मकितवानां भारतीयार्याणां केषाञ्चित् आचारेण च, निःस्वार्थधर्मप्रचारकाचार्याभावेन, कर्मनिष्ठानां निरीश्वरमीमांसकानां प्रबलप्रोत्साहनेन कर्मभरभारपीडिते च लोके, धर्मनाम्ना तत्र तत्र हिंसाप्रधानेषु कर्मसूज्जृम्भतामाप्तेषु, बुद्धमतानुयायिनां विहारविहारिणां आहार विहारपराणां भिक्षूणां धर्माभासेन च जनतायां अनादिवैदिकमतं प्रति द्वेषे स्मुत्पन्ने अत्यधिकसम्पत्तिसमृद्धिवशात् अत्यधिकसुखानुभवाच्च स्थिरनित्यानन्देप्सावति च लोके विभिन्नमतप्रवर्तकाचार्योपदेशानां अनैक्यपराणां बलेन भ्रान्ते आविले च समाजे प्राचीनोऽयं उपनिषत्प्रतिपादितस्सर्वधर्मसमन्वयपर अद्वैतसिद्धान्तमार्गः ह्नासोन्मुख इबाभृत् । तादृशं अद्वैतमतं परित्रातुं भगवतः परमेश्वरस्य साक्षादवतार भूताश्शङ्कराचार्या इति प्रसिद्धिः ।
शङ्कराचार्यप्रभावः -
देवप्रार्थनया शङ्करावतारभूतैश्शङ्कराचार्यैरद्वैतात्मवादः सर्वत्र प्रकाशितः प्रसारितश्च ।
``अष्टवर्षे चतुर्वेदी द्वादशे सर्वशास्त्रवित् ।
षोडशे कृतवान् भाष्यं द्वात्रिंशे मुनिरत्यगादिति" ।।
प्रसिद्धाभाणकानुसारं वयस्यल्पे एव सर्वत्र शिष्यगणैस्साकं सञ्चारं कृत्वा अद्वैतात्म वादः पुनरुज्जीवितः । शङ्कराचार्यैः एकोनशतसंख्याका ब्रह्मसूत्रव्याख्याः खण्डिता इति उमामहेश्वरेण तत्वचन्द्रिकायां वर्णितम् । बौद्धजैनमतानि तर्क पातञ्जलादि दर्शनानि कर्मब्रह्मवादिमतानि अन्यानि च वैदिकप्रमाणप्रबलयुक्तिभिस्सप्रमाणाभिः प्रौढाभिश्च रीतिभिस्सञ्चूर्णितानि ।
जीवब्रह्मणोरैक्यं, ब्रह्मण एव सत्यत्वं, नामरूपाणां मिथ्यात्वं, सर्वदेशकालवस्त्वपरिच्छिन्नसत्ताकत्वमेव सत्यत्वं, मिथ्यारूपस्य जगत आत्मनि कल्पितत्वं, कल्पनाप्यज्ञानेन वस्तुतः ब्रह्मव्यतिरिक्तसत्ताकं जगत नास्त्येवेत्यादीन् सर्वधर्म समन्वयकरान् जगतश्शान्ति प्रदान् सर्वथा सर्वदा कल्याणप्रदान् कालत्रययोग्यान् सिद्धान्तान् सर्वत्र सम्यगुपदिश्य अद्वैतसिद्धान्तसंरक्षणाय चतसृषु दिक्षु मठान् संस्थाप्य तत्र तत्र स्वीयान् अद्वैतवादकुशलान् शिष्यान् संयोज्य पक्षवेदर्षिमिते शके 742-820 A. D. काले भगवान् शङ्कराचार्यस्स्वधाम प्रपेदे ।
शङ्कराचार्यपितामहः विद्याधिराज इति विश्रुतनामधेयः । राजशेखरराजेन पालिते ``कालटि" नामके ग्रामेऽयमुवास । कालटिग्रामोऽयं केरलदेशान्तर्गतः । शङ्कर - भगवान् स्वावताराय कैलासात् स्वपद्भयामेवागच्छदिति वदन्तः ``काल + अडि" इति तमिलभाषापदं तथैब व्यवहृतं तद्ग्रामस्यान्वर्थनाम चाभूदिति वर्णयन्ति । तद्वास्तव्यस्य विद्याधिराजस्य पुत्रः शिवगुरुरिति प्रसिद्धः परमविद्वानासीत् ।।
शङ्कराचार्यपिता
शङ्कराचार्याणां पितुर्नाम शिवगुरुरिति । बाल्ये एवाधीतविद्योऽयं चतुर्थाश्रमाभिलाषी सञ्जातः, पितुराचार्यस्य चोपदेशेन स्वीकृतगार्हर्स्थ्य अन्वर्थनामास त् । तस्य पत्नी सतीनाम्नीति शङ्करविजयव्याख्याया अवगम्येते । तयोः पुत्रमुखकमल प्रेक्षणसुखं चिरेणापि नाभूत् । अत उभावपि शिवमाराधयामासतुः । तुष्टो भगवान् शङ्कर अनयोः पुत्रत्वमूरीचकार ।
शङ्कराचार्याणां गोत्रम् -
शङ्कराचार्याणां गोत्रमात्रेयगोत्रम् । बृहदारण्यकोपनिषद्वार्तिके शङ्करशिष्यसुरेश्वराचार्यकृते `तं वन्देऽत्रिकुलोद्भवम्' इति दृश्यते ।।
शङ्कराचार्याणां गुरुः -
शङ्कराचार्याणां गुरुः गौडपादशिष्यः गोविन्दभगवत्पादः । शङ्कराचार्यरचितेषु ग्रन्थेषु ``गोविन्दभगवत्पादशिष्येणेति तत्र तत्र दृश्यते । गोविन्दं परमानन्दं मद्गुरुं प्रणतोऽस्म्यहम्" इति विवेकचूडामणौ च दृश्यते । गोविन्दभगवत्पादश्च सोमोद्भवातीरे कस्याञ्चित् गुहायामुवास । शङ्करश्च दूरादेव तं दृष्ट्वा सप्रश्रयं सभक्त्युन्मेषं साञ्जलिबन्धं तुष्टाव । तदनु गोविन्दभगवत्पादश्शङ्करं शिष्यत्वेन स्वीकृत्य साम्प्रदायिकैस्तत्वमसीत्यादि श्रुतिवचनैः ``ब्रह्मतत्वमाचरे"ति उपदिदेश । एवञ्च विद्यागुरुर्गोविन्दभगवत्पाद इति सिध्यति ।
``वृद्धत्रय्यां" गुरुपादहालदारस्तु हैहयराजस्य कामदेवस्य शङ्कराचार्यस्य च गुरुरयं गोविन्दभगवत्पादः परमयोगी सनातनधर्मावलम्बी वेदान्तसम्प्रदायप्रवर्तकः ``रसहृदयाख्य वैद्यग्रन्थप्रणेता सप्तम नवमशतकमध्यावर्तीति प्रतिपादयति ।"
शङ्कराचार्यपरमगुरुः -
गोविन्दभगवत्पादाचार्याणां आचार्यः शङ्कराचार्याणां परमाचार्यः गौडपादाचार्यः । गौडपादाचार्योऽयं गौडदेशीयः । वादप्रतिवादे गौडपादाचार्यैः गौडीया विद्वांसः पराजिताः । गौडपादाचार्योऽयं शुकाचार्यशिष्यः । शङ्कराचार्यस्य सूत्रभाष्यप्रणयने गौडपादाचार्याणां सम्पतिरनुग्रहश्चाभूतामिति शङ्करदिग्विजये श्रूयते । माण्डूक्यकारिकाभाष्येऽलातशान्तिप्रकरणे ``तं पूज्याभिपूज्यं परमगुरुममुं पादपार्तैनतोऽस्मि" इत्यादिना परमगुरोर्नमस्कृतिरावेदिता । ``अमुम्" इत्यस्य व्याख्यानावसरे आनन्दगिरिणापि ``पुरोदेशे सन्निहितत्वेनापरोक्षत्वं सृचितम्" इति व्याख्यातम् । सूत्रभाष्ये द्वितीयाध्यायस्य प्रथमपादे नवमसूत्रभाष्यावसरे गौडपादाचार्यस्य ``अनादिमायया" इत्यादिनी कारिका ``तदुक्तमाचार्येणेत्यादिना" परामृष्टः । एतत्सर्वं शङ्कराचार्याणां गौडपादशिष्यत्वेऽथवा अनुग्रहप्राप्तौ प्रमाणम् ।।
शङ्करशिष्याः-
पद्भपाद - हस्तामलक - तोटक - सुरेश्वराख्याश्चत्वार एतेषां शिष्याः । ते च क्रमेण पञ्चपादिका - विवेकमञ्जरी - श्रुतिसारसमुद्रण - बृहदारण्यकोपनिषत्तैत्तरीयकवार्तिकादिकृतः ।।
शङ्कराचार्यकालः- 
शङ्कराचार्याश्च खेन्दुहयमिते शालिवाहनशके प्रादुरभूवन् । ``व्योमभूवाजिसंख्याङ्के शालिवाहनके शुभे । विभवेऽब्दे शुक्लपक्षे वैशाखे दशमीतिथौ । शङ्करः प्रादुरासीत् ब्राह्मण्यस्थितिगुप्तये ।" इति परम्परागतञ्च पद्ममत्र प्रमाणं भवति । बालकृष्णब्रह्मानन्दकृतशङ्करदिग्विजये ``सहस्रद्वितयादूर्ध्वं एकोनर्विशके । शते । एकादशोनंसख्याके वत्सरे कलिमानतः । निधिनागेभवह्न्यब्दे विभवे शङ्करोदयः । कलौ तु शालिवाहस्य सखेन्दुशतसप्तके ।" इति दृश्यते एवञ्च शङ्करोत्पत्तिकालः(788 A. D.) एवं ``कल्यब्दे चन्द्रनेत्राङ्गगुणसंख्ये सुहायने । विहारिनामके तस्मिन् वैशाख्यां शिवतामगात् ।" शालिवाहशकेह्यब्धिहयसंख्ये स शङ्करः । विकारिनामके तस्मिन् वैशाख्यां शिवतामगात् । इत्यादिभिश्च प्रमाणैः (742) शके शङ्करनिर्याणमिति तेषां जीवनकालस्त्रयस्त्रिंशद्वत्सराण्येवेति निश्चीयते । द्विसहस्रवत्सरेभ्यः पूर्वमिति साम्प्रदायिकाः । क्रिस्तोः पूर्वं पञ्चमशतकं (500 BC.) जनन कालः ब्रह्मीभावश्च (476 BC.) इति नारायणशास्त्रिणः ``एज आफ शङ्कर" नामके ग्रन्थे वदन्ति ।
केरलोत्पत्तिनामके ग्रन्थे (400 A. D.) काले शङ्करावतारः । जीवन कालश्च 38 वर्षाणीति प्रतिपाद्यते । K. T. तेलाङ् महाशयस्तु (550-590 A. D.) शङ्करकाल इति  (I. A. Val. XIII Page 95) प्रतिपादयति । डा. बेर्नल महाशयस्तु सामविधानब्राह्मणभूमिकायां शङ्करकालं (652-680 A. D.) इति वदति । डा. फलीटमहाशयस्तु नेपालवंशावलीनामके प्रबन्धे  (Page 118-123) नेपालदेवस्य वृषदेवाख्यस्य काले शङ्कर आसीत् । वृषदेवश्च (630-655 A. D.) काले आसीदिति वर्णयति । सूर्यनारायणरावमहाशयस्तु (I. A. Vol. XLIII Page 272) अष्टमशतकीयश्शङ्कर इति प्रतिपादयति । डा. बंदरकार महाशयस्तु सूत्रभाष्ये (2-4-1, 4-3-5) बलवर्मा निर्दिष्टः । बलवर्मणश्च काल (767-785 A. D.) इति स एव शङ्करकाल इति वदति । (I. A. Vol. XLI Page 200) पत्रिकायाम् । प्रो. टेलीतु (I. A. Vol. IX Page 263) शङ्करकाल (788-820 A. D.) इति वदति । वेङ्कटेश्वराचार्यस्तु (J. R. A. S. 1916 Page 153) शङ्कराचार्यकालः (805-897 A. D.) इति वदति । पताक महाशयस्तु (3889-3921) (788-820 A. D.) Flf (I. A. Vol. XI page 174-175) वर्णयति । बालकृष्णप्पिल्लायमहाशयस्तु संक्षेपशारीरके मनुकुलादित्यः निर्दिष्टः । संक्षेपशारीरकारः सर्वज्ञात्मा शङ्करप्रशिष्यस्सुरेश्वरशिष्यश्चेति (978 A.D.) कालिकेन भास्कररविवर्मणा शिलाशासने शङ्करस्य नाम निर्दिष्टमिति शङ्करकालः दशमशतकादिमः भाग इति (I. A. Vol. I Page 136) प्रतिपादयति । K. A. नीलकण्ठशास्त्री तु ``येनाधीतानि शास्त्राणि भगवच्छङ्कराह्वयात् । निःशेषसूरिमूर्धालिमालालीढाङ्घ्निपङ्कजात् । सर्वविद्यैकनिलयो वेदवित् विप्रसम्भवः । शासको यस्य भगवान् रुद्रो इवापरः ।" इति पद्यं शिवसोमकालिकायाश्शिलालेखायाः ज्ञायते । शिवसोमश्च (877-889 A. D.) कालिकस्य इन्द्रवर्मणो गुरुरिति शङ्करकाल (877 A. D.) पर्यन्तं स्याद्वेति सन्देग्धि । प्रतिपादितञ्चैतत् (J. O. R. Vol. XI Page 265) । अडयार पुस्तकाल्यस्थेऽमुद्रितेऽज्ञातकर्तृके ब्रह्ममीमांसाशास्त्रसंग्रहे 845 सं (788 A. D.) शङ्करकाल इति दृश्यते । श्रीकण्ठशास्त्री तु भारतीयैतिहासिकत्रैमासिकपत्रिकायाश्चतुर्दशतमे भागे (I. H. Q. Vol. XIV Page 401) धर्मकीर्तिसामयिकश्शङ्कराचार्य इति (620 A. D.) कालान्नार्वाचीन इति प्रतिपादयति । T. R. चिन्ताामणिमहाशयस्तु (560-650 A. D.) कालमध्यवर्ती शङ्कर इति प्रतिपादयति । म. म. कुप्पुस्वामि शास्त्रिणश्च (632-664 A. D.) इति प्रतिपादयन्ति । 
शङ्कराचार्यग्रन्थाः-
यद्यपि शङ्कराचार्यकृता इति बहवो ग्रन्थाः मुद्रिताः प्रसिद्धाश्च तथापि ते सर्वे शङ्कराचार्यकृता इत्यत्र न प्रमाणम् । परन्तु तत्सिम्हासनारूढैश्शिष्यप्रशिष्यैर्विरचित इति ज्ञेयम् । परन्तु अधोनिर्दिष्टाः ग्रन्थाशशङ्करकृता इत्यत्र न संशयः ।।
१. अद्वैतपञ्चरत्नम् - (S. M. E. Vol 16)
सोपानपञ्चकापरनामायं ग्रन्थश्शङ्कराचार्यस्मारकग्रन्थावल्यां श्रीरङ्गनगरे मुद्रितः । अस्य व्याख्याः - कृष्णानन्दसरस्वतीकृता ``किरणावली," विमलभूधरकृता व्याख्या, अज्ञातकर्तृका दीधीतिनाम्नी व्याख्या, अज्ञातकर्तृका अपरा व्याख्या च विद्यन्ते ।।
२. अद्वैतानुभूतिः- (S. M. E. Vol 16)
प्रकरणग्रन्थोऽयं वाणीविलासमुद्रणालये मुद्रितः ।
३. अध्यात्मविद्योपदेशविधिः -
अज्ञानबोधिनी `अध्यात्मविद्योपदेशः' `आत्मज्ञानोपदेश' इत्यादिनामान्तरमस्य दृश्यते । प्रकारणग्रन्थोऽयं चौखाम्बामुद्रणालये मुद्रितः । अस्य व्याख्याः आनन्दगिरिकृता, अनन्तराममुनिकृता ``सम्प्रदायतिलकम्," पुराणानुभवकृता ``दीपकनाम्नी," पूर्णानन्दकृता व्याख्या च विद्यन्ते ।।
४. अनात्मश्रीविगर्हणम् - (S. M. E. Vol 16) प्रकरणग्रन्थोऽयं मुद्रितः ।
५. अपरोक्षानुभूतिः - (S. M. E. Vol 15)
147 पद्यैः पूर्णोऽयं प्रकरणग्रन्थश्शाङ्करप्रकरग्रन्थावल्यां मुद्रिताः अस्यैव ``अपरोक्षानुभवामृत" मित्यपि नामान्तरं श्रूयते । अस्य व्याख्याः-चण्डेश्वर शर्मकृता `दीपिका,' नित्यानन्दानुचरकृतं विवरणम्, बालगोपालयतिकृता व्याख्या विद्यारण्यकृता `व्याख्या' अज्ञातकर्तृका काचन व्याख्या च वर्तते ।।
६. आत्मबोधः- (S. M. E. Vol 15)
अज्ञानबोधिनी बोधार्यापरनामायं प्रकरणग्रन्थः 68 पद्यैः पूर्णः जीवात्मनोरभेदं जीवन्मुक्तदशाञ्च प्रतिपादयति । मुद्रितश्चायं ग्रन्थश्शाङ्करप्रकरणग्रन्थावल्याम् ।
अस्य व्याख्याः- अद्वयानन्दकृता व्याख्या चिदानन्दकृता ``आत्मबोधलहरी," पद्मपादाचार्यकृता `वेदान्तसारः,' बोधेन्द्रकृता `भावप्रकाशिका,' भासुरानन्दकृता व्याख्या, मधुसूदनसरस्वतीकृता व्याख्या, रघुनाथसरस्वतीकृता व्याख्या, रामानन्दतीर्थकृता व्याख्या, विश्वेश्वरपण्डितकृता दीपिका च विद्यन्ते । कृष्णानन्द सरस्वतीकृता व्याख्या ग्रन्थलिप्यां मुद्रितः । ब्रह्मानन्दकृता आत्मबोधव्याख्या, चित्सुखशिष्यकृता व्याख्या, अद्वयानन्दसरस्वतीकृता व्याख्या, अद्वैतानन्दसरस्वतीकृता व्याख्या च विद्यन्त इति श्रूयते । हलषसूच्यांं तु विद्यारण्यकृता व्याख्या च निर्दिष्टा । रघुनाथसरस्वतीकृता व्याख्या तु (AL. BRD) पुस्तकालये लभ्यते । आनन्दगिरिकृता टीका वाराणसीविश्वविद्यालये अमुद्रिता विद्यते । (D. C. VII P. No. 130)
७. आत्मानात्मविवेकः - (VVP)
प्रकरणग्रन्थोऽयं वाणीविलासमुद्रणालये मुद्रितः । अस्य व्याख्याः- पूर्णानन्दतीर्थकृता व्याख्या, वासुदेवयतिकृता व्याख्या, सदाशिवेन्द्रसरस्वतीकृता `प्रकाशिका' स्वयम्प्रकाशयतिकृता व्याख्या, च विद्यन्ते । सायमाचार्येण पद्भपादाचार्येण व्याख्या कृतेति श्रूयते । अज्ञातकर्तृका वेदान्तचूर्णिकानाम्नी च विद्यते ।
८. उपदेशसाहस्री - (S. M. E. Vol 14)
गद्यपद्यात्मकभागद्वयमितोऽयं प्रकरणग्रन्थः अद्वैतवेदान्तखनिः । मुद्रितश्चायं शाङ्करप्रकरणग्रन्थावल्याम् । अस्य व्याख्याः- आनन्दगिरिकृता व्याख्या, अखण्डधामकृता ``गूढार्थदीिका" बोधनिधिकृता व्याख्या रामतीर्थकृता `पदयोजनिका' च विद्यन्ते । त्र्यम्वकभट्टकृता व्याख्या उज्जैनसूच्यां दृश्यते । अज्ञातकर्तृकव्याख्या मद्रासराजकीयपुस्तकालये लभ्यते । 
९. एकश्लोकः - (S. M. E. Vol. 16)
किं ज्योतिःश्लोक इत्यपरनामायं ग्रन्थः शाङ्करप्रकरणग्रन्थावल्यां मुद्रितः । दिग्विजययात्रायै विश्वं सञ्चरन् कदाचित् शङ्कराचार्यः कस्मिंश्चित् ग्रामविशेषे, कूश्माण्डमिव पाण्डुराङ्गं, उलूकमिव सूर्याबलोकनाक्षमं, लज्जयावनतमुखं मां त्राहि त्राहि इति पुनः पुनः प्रणमन्तं कञ्चन कुष्ठिनं सा धनचतुष्टययुतमवलोक्य संसारसङ्कटान्मोचयन् कृती कर्तुं परमकारुणिकोऽयं भगवान् शङ्कराचार्यः प्रश्नप्रतिवचनप्रणाल्या श्लोकमेनं निबबन्ध इति साम्प्रदायिकी कथा श्रूयते । अस्य व्याख्या - स्वयम्प्रकाश यतिकृता ``तत्वदीपनाख्या" (स्वात्मदीपनम्) विद्यते ।
१०. काशीपञ्चकम् । वाणीविलासमुद्रणालये मुद्रितः ।
११. कौपीनपञ्चकम् - (S. M. E. Vol. 16)
यतिपञ्चकापरनामायं ग्रन्थः श्रीरङ्गक्षेत्रे मुद्रितः ।
१२. ज्ञानाङ्कुशम् सविवरणम् -
मनोनिग्रहोपायप्रतिपादनपरोऽयं ग्रन्थः अद्वैतसभापत्रिकायां मुद्रितः ।
१३. दशश्लोकी - (S. M. E. Vol. 15)
अद्वैतदशकम्, निर्वाणदशकम्, इत्यपरनामायं ग्रन्थ वाराणस्यां वाणीविलासमुद्रणालये च मुद्रितः । अस्य व्याख्यामधुसूदनसरस्वतीकृता सिद्धान्तबिन्दुनाम्नी प्रसिद्धा । विश्वेश्वरकृतापि व्याख्या विद्यते । कुत्रचिगस्यैव चिदानन्दस्तवराजः, चिदानन्ददशश्लोकीत्यपि नामान्तरं श्रूयते ।
१४. निर्वाणषट्कम् - (S. M. E. Vol. 16)
१५. पञ्चरत्नमालिका - (S. M. E. Vol. 16)
आत्मपञ्चिका, अद्वैतपञ्चिका, उपदेशपञ्चकम् , पञ्चरत्नकारिका इत्यपरनामायं ग्रन्थः वाणीविलासमुद्रणालये मुद्रितः । अस्य व्याख्याः अभिनवनारायणेन्द्रकृता, पाण्डुरङ्गपण्डितकृता प्रकाशाख्या, सदाशिवब्रह्मकृता, सुब्रह्मण्यकृता, दीधितिनाम्नी अज्ञातकर्तृका, मुमुक्षुजनकल्पवल्लीनाम्नी अज्ञातकर्तृका च व्याख्या विद्यन्ते । ग्रन्थोऽयं शङ्कराचार्यकृत इति प्रसिद्धिः । परन्तु अमुद्रिते मद्रासराजकीयहस्यलिखितपुस्तकालयस्थेऽऽदर्शापुस्तके (D. 4632 MGOML)  ``शङ्काराचार्यः प्रकटयति" इति दर्शनात् नायं शङ्करकृतिरिति प्रतिभाति ।
१६. पञ्चीकरणम् - (S. M. E. Vol 16)
वेदान्तसारपञ्चीकरणनामायं ग्रन्थश्शाङ्करप्रकरणग्रन्थावल्यां मुद्रिताः । अस्य व्याख्याः - अन्तरारामकृता ``समाधिप्रक्रिया", अभिनवनारायणेन्द्रकृता भावप्रकाशिका, आनन्दज्ञानकृतं विवरणम्, प्रज्ञानानदयतिकृतं बिवरणम् , रामतीर्थकृता तत्वचन्द्रिका, स्वयम्प्रकाशयतिकृतं विवरणम् , अज्ञातकर्तृका तत्वपञ्चिका, अज्ञातकर्तृका व्याख्या, सुरेश्वराचार्यकृतं पञ्चीकरणवार्तिकम् ।
१७. प्रबोधसुधाकरः - (S. M. E. Vol 16)
प्रकरणग्रन्थोऽयं वाणीविलासमुद्रणालये मुद्रितः । ग्रन्थोऽयं नृसिम्हचम्पूकर्त्रा षोडशशतकीयेन दैवज्ञसूर्यसूरिपण्डितेन कृतस्स्यादिति अडयार बुल्लट्टन पत्रिकायाः प्रथमे भागे, (AOR Vol VI) पत्रिकायाञ्च प्रतिपादितम् ।
१८. प्रश्नोत्तररत्नमाला - (S. M. E. Vol 16)
१९. प्रौढानुभूतिप्रकरणम् - (S. M. E. 16)
सप्तदशभिः पद्यैः पूर्णोऽयं ग्रन्थः जीवन्मुक्तस्य स्वानुभवप्रकटनपद्धत्या रचितः । वाणीविलासमुद्रणालये मुद्रितश्च । 
२०. ब्रह्मानुचिन्तनम् - (S. M. E. Vol 16)
ब्रह्मानुसन्धानापरनामायं ग्रन्थः वाणीविलासमुद्रणालये मुद्रितः ।
२१. मनीषापञ्चकम् - (S. M. E. Vol 16)
कदाचित् शङ्कराचार्यः काशीपुरीं प्रति ययौ । तत्र शङ्काराचार्यस्य ज्ञानपरीक्षायै भगवान् चण्डालस्वरूपमुपादाय समागतः । चण्डालस्वरूपं दृष्ट्ववा गच्छ गच्छ इति वदन्तं शङ्कराचार्यं स चण्डालः ``अन्नमयादन्नमयं अथवा चैतन्यादेव चैतन्यं, यतिवर किं दूरीकर्तुं वाञ्छसि ? शरीरयोरनयोरन्नकार्यत्वादिति पप्रच्छ ।" तत्प्रश्नस्य प्रतिवचनाय शङ्काराचार्यैर्मनीषापञ्चकं प्रणीतमिति कथा प्रसिद्धा । प्रतिचरणं ``मनीषा मम" इति दर्शनात् अस्य मनीषापञ्चकमिति नाम । अद्वैतसिद्धान्तसारभूतोऽयं प्रकरणग्रन्थश्शाङ्करग्रन्थमालायां मुद्रितः । अस्य व्याख्याः-गोपालबालयतिकृता ``मधुमञ्जरी", नृसिम्हाश्रमिकृता `मधुमञ्जरी' वासुदेवन्द्रकृता व्याख्या, सदाशिवब्रह्मेन्द्रकृता तात्पर्यदीपिका, अज्ञातकर्तृकं लघुविवरणम्, अज्ञातकर्तृका व्याख्या च विद्यन्ते । 
२२. मायापञ्चकम् -
मायाविवरणापरनामायं ग्रन्थः मायाया अधटितधटनापटीयस्त्वं वर्णयति । ग्रन्थोऽयं शाङ्करप्रकरणग्रन्थावल्यां (S. M. E. 16) मुद्रितः ।

२३. मोहमुद्गरः (S. M. E. Vol. 16)
द्वादशमञ्जरिकापरनामायं ग्रन्थः भजगोविन्दस्तोत्रमित्यपि व्यपदिश्यते । मुद्रितोऽयं ग्रन्थः । अस्य व्याख्या द्वादशमञ्जरी ``मकरन्द" नाम्नी स्वयम्प्रकाशयतिकृता तिरुवनन्तपुरपुस्तकालये मद्रासराजकीयहस्तलिखितपुस्तकालये च लभ्यते ।।
२४. लघुवाक्यवृत्तिप्रकरणम् - (S. M. E. Vol. 16)
प्रकरणग्रन्थोऽयं वाणीविलासमुद्रणालये मुद्रितः । रामानन्दसरस्वतीकृता प्रकाशिका नाम्नी, अज्ञातकर्तृका ``पुष्पाञ्जलि" नाम्नी च व्याख्या वर्तते । तत्र पुष्पाञ्जलिः रामभद्रयतिशिष्यरामानन्दकृता इति हालसूच्याः 107 पुटे दृश्यते ।
२५. वाक्यवृत्तिः- (ASS 80, S. M. E. Vol. 15)
गुरुशिष्यकथासरण्या विरचितोऽयं प्रकरणग्रन्थस्त्रिपञ्चाशद्भिः पद्यैः पूर्णो जीवब्रह्मणोरैक्यं महावाक्यार्थञ्च प्रतिपादयति । अस्य व्याख्याः आनन्दज्ञानकृता व्याख्या, आन्दस्वरूपभट्टारककृता वाक्यदीपिका, रामानन्दसरस्वतीकृता प्रकाशिका, विश्वेश्वरपण्डितकृता प्रकाशिका, अज्ञातकर्तृका लघुटीका च ।।
२६. विवेकचूडामणिः - (S. M. E. Vol. 14)
आत्मानात्मविवेकचूडामण्यपरनामायं ग्रन्थः वाणीविलासमुद्रणालये मुद्रितः । अस्य व्याख्या रत्नस्वामिशिष्येण हरिनाथभट्टेन कृता काचन वाराणस्यां मुद्रिता । अपरा अज्ञातकर्तृका च विद्यतेऽमुद्रिता । श्रृङ्गगिरिशङ्करपीठाधीशैः श्रीमच्चन्द्रशेखरभारतीस्वामिभिः कृता काचन व्याख्या पाण्डित्यपूर्णा मुद्रिता लभ्यते ।
२७. वेदान्तसारः- (S. M. E. Vol. 15)
वेदवेदान्तसारः, सर्ववेदान्तसिद्धान्तसारसङ्ग्रह इत्यपरनामायं ग्रन्थः 124 पद्यैः पूर्णः वाणीविलासमुद्रणालये मुद्रितः ।।
२८. शतश्लोकी - (S. M. E. 15)
वेदान्तशतश्लोक्यपरनामायं ग्रन्थ आनन्दज्ञानेन व्याख्यातः, स्रग्धरावृत्त घटितश्च वाणीविलासमुद्रणालये मुद्रितश्च । अज्ञातकर्तृका व्याख्या काचन बरोडापुस्तकालये बाम्बे युनिवर्सिटिपुस्तकालये च लभ्यते । लन्दननगरहस्तलिखित पुस्तकालयेऽपि लभ्यते ।
२९. षट्पदी - (G. N. P. B)
आर्यावृत्तघटितैः पद्यैः पूर्णोऽयं ग्रन्थ अद्वैतब्रह्मात्मकभगवत्स्तुतिरूपः । मुद्रितश्चायं गोपालनारायणमुद्रणालये बाम्बे नगरे । अस्य व्याख्याः कविसरोजभिक्षुकृता, वैकुण्ठशिष्यकृता, शङ्करानन्दतीर्थकृता च षट्पदमञ्जरी नाम्नी प्रसिद्धा ।।
३०. सदाचारप्रकरणम् - (S. M. E. Vol. 16)
सदाचारानुसन्धानमित्यपि व्यपदिश्यतेऽयं ग्रन्थः । आत्मसाक्षात्कारोपायभूतं सदाचारं प्रतिपादयन्नयं ग्रन्थः वाणीविलासे मुद्रितः । ग्रन्थोऽयं अच्युतशर्मामोडकेन शुद्धधर्मपद्धत्याख्यया व्याख्यया व्याख्यातः ।
३१. सनत्सुजातीयभाष्यम् - (S. M. E. Vol. 13)
महाभारतोद्योगपर्वान्तर्गतस्य विधुरप्रार्थनाप्रेरितसनत्सुजातधृतराष्ट्रसंवादात्मकस्य ग्रन्थस्य व्याख्यारूपोऽयं ग्रन्थः वाणीविलासमुद्रणालये मुद्रितः काण्डद्वयातीतयोगिना मोक्षसाम्राज्यलक्ष्मीतन्त्रनाम्न्या व्याख्यया, नीलकण्ठकृतया च व्याख्यया व्याख्यातः ।
३२. सर्ववेदान्तसिद्धान्तसंग्रहः - (S. M. E. Vol. 15)
सर्वदर्शनसिद्धान्तः, सर्वसिद्धान्तसंग्रहः, वेदान्तशास्त्रसिद्धान्तसंग्रहः, वेदान्तसिद्धान्तदीपिका, सर्ववेदान्तसिद्धान्तसरसंग्रह इत्यपरनामायं ग्रन्थः द्वादशभिः प्रकरणैः पूर्णः भारतीयानां विभिन्नानां दर्शनानां सिद्धान्तान् प्रतिपादयति । मुद्रितश्चायं वाणीविलासमुद्रणालये । अस्य व्याख्या शेषगोविन्दकृता च विद्यते ।
ग्रन्थोऽयं शङ्करादर्वाचीनेन केनापि कृतं स्यादिति (ABORI Vol. XII Page 253) पत्रिकायां प्रतिपादितम् । ``अर्थतोऽप्यद्वयानन्दमतीतद्वैत लक्षणम् । आत्माराममहं वन्दे श्रीगुरुं शिवविग्रहम्" । इत्यमुद्रिते अडयारपुस्तकालये च दर्शनात् प्रथमसदानन्दप्रशिष्येण अद्वयानन्दशिष्येण द्वितीयसदानन्देन कृतस्स्यादिति (J. O. R. Vol. XIII, AOR. Vol. VI) पत्रिकयोः प्रतिपाद्यते ।
३३. स्वरूपानुसन्धानाष्टकम् - (S. M. E. Vol. 16)
भुजङ्गप्रयातछन्दोबद्धोऽयं विज्ञाननौकापरनामा ग्रन्थ अष्टभिः पद्यैः पूर्णः वाणीविलासमुद्रणालये मुद्रितः । अस्य व्याख्या आनन्दज्ञानकृत ``स्वरूपविवरण"नाम्नी, पद्मपादकृता ``स्वरूपानुभव"नाम्नी, श्रीकुदकृता ``पदव्याख्या"नाम्नी च विद्यन्ते ।।
३४. स्वात्मनिरूपणम् - (S. M. E. Vol. 16)
स्वात्मप्रकाशिका, स्वात्मानन्दप्रकरणम् , वेदान्तार्या, अनुभूतिरत्नमाला, स्वात्मानन्दप्रकाशिका, इत्यादिनामभिः व्यपदिश्यमानोऽयं ग्रन्थः देहेन्द्रियादिभिन्न आत्मस्वरूपमेकं प्रतिपादयति । मुद्रितश्चायं शाङ्करप्रकरणग्रन्थावल्यां वाणीविलासमुद्रणालये । अस्य व्याख्या सच्चिदानन्दसरस्वतीकृता अमुद्रिता विद्यत इत्यन्यत्र प्रतिपादितम् ।।
३५. हस्तामलकीयभाष्यम् -
हस्तामलकाचार्यकृतविवेकमञ्जरीव्याख्यात्मकोऽयं ग्रन्थः वाणी विलासमुद्रणालये मुद्रितः ।
३६. ईशावास्योपनिषद्भाष्यम् - (ASS 5)
ग्रन्थोऽयमानन्दाश्रममुद्रणालये मुद्रितः । अस्य व्याख्याः - आनन्दगिरिकृता शिवानन्दयतिकृता च विद्येते ।
३७. ऐतरेयोपनिषद्भाष्यम् - (ASS 11)
मुद्रितश्चायं ग्रन्थ आनन्दाश्रममुद्रणालये । अस्य व्याख्याः - अभिनवनारायणेन्द्रकृता, आनन्दज्ञानकृता, ज्ञानामृतयतिकृता, उपनिषद्ब्रह्मेन्द्रकृता, विद्यातीर्थकृता, च विद्यन्ते । नृसिह्मकृता, बालकृष्णदासकृताश्च भाष्यव्याख्या विद्यन्त इति वदन्ति । परन्तु ताः कुत्र लभ्यन्त इति न ज्ञायते । सीतानाथतत्वभूषणेन काचन व्याख्या कृता । सा च भारतकार्यलयमुद्रितपुस्तकालयपुस्तकसूच्यां लन्दननगरस्थायां (IOLPC Vol II Part I Page 64) दृश्यते ।
३८. कठोपनिषद्भाष्यम् - (ASS 7)
मुद्रितश्चायं ग्रन्थ आनन्दाश्रममुद्रणालये । अस्य व्याख्याः - अच्युतकृष्णानन्दकृता, अभिनवनारायणेन्द्रकृता, आनन्दगिरिकृता, गोपालबालयतिकृता, शिवानन्दयतिकृत, श्रीधरशास्त्रिकृता च विद्यन्ते ।
३९. केनोपनिषद् भाष्यम् - (ASS 6)
पदवाक्यभाष्यभेदेन भिन्नं भाष्यद्वयमपि आनन्दाश्रमे मुद्रितम् । अस्य व्याख्या अभिनवनारायणेन्द्रकृता, आनन्दगिरिकृता, शिवानन्दयतिकृता, श्रीधरशास्त्रीपाठककृता च व्याख्याः विद्यन्ते ।
४०. छान्दोग्योपनिषद् भाष्यम् - (ASS 14)
मुद्रितोऽयं ग्रन्थ आनन्दाश्रममुद्रणालये । अस्य व्याख्याः-अभिनवनारायणेन्द्रकृता, आनन्दगिरिकृता, च विद्येते । नरेन्द्रपुरीकृता काचन भाष्यटिप्पणी मद्रासराजकीय पुस्तकालये लभ्यते ।
४१. तैत्तरीयोपनिषद् भाष्यम् - (ASS 12)
ग्रन्थोऽयं आनन्दाश्रमे वाणीविलासे च मुद्रितः । अस्य व्याख्याः आनन्दगिरिकृता, अच्युतकृष्णानन्दकृता वनमालाख्या, सुरेश्वराचार्यकृतं वार्तिकम्, अभिनवद्रविडाचार्यबालकृष्णानन्दसरस्वतीकृतं भाष्यविवरणमिति व्याख्याः विद्यन्ते ।
४२. नृसिम्हतापनीय भाष्यम् - (ASS. 30)
(D. 581. MGOML) ग्रन्थोऽयं वाणीविलासेऽऽनन्दाश्रमे च सुद्रितः ।
४३. प्रश्नोपनिषद्भाष्यम् - (ASS 18)
आनन्दाश्रमे मुद्रितः । अभिनवनारायणेन्द्रकृता आनन्दगिरिकृता उप निषद्ब्रह्मेन्द्रकृता, शिवानन्दयतिकृता अज्ञातकर्तृका व्याख्याः विद्यन्ते ।।
४४. ब्रह्मसूत्रभाष्यम् - (N. S. P.)
शारीरकमीमांसाभाष्यापरनामायं ग्रन्थः निर्णयसागरमुद्रणालयेऽन्यत्र च मुद्रितः । अस्य व्याख्याः - अद्वैतानन्दकृतं ब्रह्मविद्याभरणम् , अनन्तकृष्णाशास्त्रिकृतः प्रदीपः, अनन्यानुभवकृता शारीरकन्यायमणिमाला, अनुभूतिस्वरूपाचार्यकृत प्रकटार्थविवरणम्, आनन्दगिरिकृतः न्यायनिर्णयः, उपनिषद्ब्रह्मेन्द्रकृतः भाष्यसिद्धान्तसंग्रहः, कृष्णशास्त्रिकृतानुगुण्यसिद्धिः, रामानन्दकृता रत्नप्रभा, चित्सुखाचार्यकृता भाष्यभावप्रकाशिका, ज्ञानोत्तमकृता विद्याश्रीः, त्र्यम्बकशास्त्रिकृता भाष्यानुप्रभा (भाष्यभानुप्रभा) नारायणसरस्वतीकृतं भाष्यवार्तिकम्, पद्मपादाचार्यकृता पञ्चपादिका, बालकृष्णानन्दकृतं भाष्यवार्तिकम् , ब्रह्मानन्दयतिकृतं भाष्यवार्तिकम् , ब्रह्मानन्दयतिकृतः भाष्यार्थसंग्रहः, वाचस्पतिमिश्रकृता भामती, शिवनारायणानन्दकृता सुबोधिनी, कृष्णानुभूतिकृतः भाष्यसिद्धान्तसंग्रहः, प्रकाशात्मकृता भाष्यन्यायसंग्रहाख्याश्च व्याख्याः प्रसिद्धतमाः ।।
ब्रह्मसूत्रभाष्यम् (भामती प्रस्थानम्)
%%% Chart
पञ्चपादिकाप्रस्थानम् (विवरणप्रस्थानम्)
%%% Chart
४५. बृहदारण्यकोपनिषद्भाष्यम् - (ASS 16)
आनन्दाश्रममुद्रणालये वाणीविलासमुद्रणालये च मुद्रितः । अस्य आनन्दगिरिकृता, शिवानन्दयतिकृता, महादेवेन्द्रसरस्वतीकृता, सुरेश्वराचार्यकृतं वार्तिकञ्चेति व्याख्यास्सन्ति ।
बृहदारण्यकोपनिषत्
%%% Chart
४६. भगवद्गीताभाष्यम् - (N. S. P.)
निर्णयसागर-वाणीविलास-आनन्दाश्रममुद्रणालयेषु मुद्रितः । अस्य व्याख्याः - अनुभूतिस्वरूपाचार्यकृता, आनन्दगिरिकृता, केशवसाक्षिभगवत्पादकृता, रामानन्दकृता, भागवतानन्दकृताश्च विद्यन्ते ।
४७. माण्डूक्योपनिषद्भाष्यम् - (ASS 10) 
सव्याख्योऽयं ग्रन्थः वाणीविलासमुद्रणालयेऽऽनन्दाश्रममुद्रणालये च मुद्रितः आनन्दगिरिकृता मधुरानाथशुक्लकृता, राघवानन्दकृता, अज्ञातकर्तृकाश्च व्याख्यास्सन्ति ।
४८. माण्डूक्यकारिकाभाष्यम् - (ASS 10)
गौडपादकारिकाभाष्यापरनामायं ग्रन्थ आनन्दाश्रमे मुद्रितः । अस्य आनन्दगिरिकृता, अनुमूतिस्वरूपकृताश्च व्याख्याः विद्यन्ते ।।
माण्ड्क्योपनिषत्
%%% Chart
४९. मुण्डकोपनिषद् भाष्यम् (ASS 9)
ग्रन्थोऽयं वाणीविलासआनन्दाश्रममुद्रणालययोर्मुद्रितः । अस्य व्याख्या - अभिनवनारायणेन्द्रकृता, आनन्दगिरिकृता, शिवानन्दयतिकृताश्च विद्यन्ते ।।
५०. श्वेताश्वतरोपनिषद्भाष्यम् (ASS 17)
भाष्यमिदं शङ्कराचार्यकृतमिति प्रसिद्धं आनन्दाश्रमे मुद्रितञ्च । परन्तु नेदं शङ्कराचार्यकृतमिति वर्णितं उनिषत्प्रस्थाने ।
स्तोत्रग्रन्थाः
५१. अच्युताष्टकम् ।
५२. अन्नपूर्णाष्टकम् ।
५३. अर्धनारीश्वरस्तोत्रम् ।
५४. आनन्दलहरी ।
५५. उमामहेश्वरस्तोत्रम् ।
५६. कनकधारास्तोत्रम् ।
५७. कल्याणवृष्टिस्तवः ।
५८. कालभैरवाष्टकम् ।
५९. काशीपञ्चकम् ।
६०. कृष्णाष्टकम् ।
६१. गणेशपञ्चरत्नम् ।
६२. गणेशभुजङ्गम् ।
६३. गुर्वष्टकम् ।
६४. गोविन्दाष्टकम् ।
६५. गौरीशतकम् ।
६६. गङ्गाष्टकम् ।
६७. जगन्नाथाष्टकम् ।
६८. त्रिपुरसुन्दरीवेदपादस्तोत्रम् ।
६९. त्रिपुरसुन्दरीमानसपूजास्तोत्रम् ।
७०. त्रिपुरसुन्दर्यष्टकम् ।
७१. दक्षिणामूर्तिवर्णमालास्तोत्रम् ।
७२. दक्षिणामूर्त्यष्टकम् ।
७३. दशश्लोकीस्तुतिः ।
७४. देवीचतुःषष्ठ्युपचारस्तोत्रम् ।
७५. देवीभुजङ्गम् ।
७६. नवरत्नमालिका ।
७७. नर्मदाष्टकम् ।
७८. निर्गुणमानसपूजा ।
७९. पाण्डुरङ्गाष्टकम् ।
८०. प्रातस्स्मरणस्तोत्रम् ।
८१. भवानीभुजङ्गम् ।
८२. भगवन्मानसपूजा ।
८३. भ्रमराम्बाष्टकम् ।
८४. मणिकर्णिकाष्टकम् ।
८५. मृत्युञ्जयमानसपूजास्तोत्रम् ।
८६. मन्त्रमातृकापुष्पमालास्तवः ।
८७. मोहमुद्गरः ।
८८. मीनाक्षीस्तोत्रम् ।
८९. यमुनाष्टकम् ।
९०. रामभुजङ्गप्रयातम् ।
९१. लक्ष्मीनृसिम्हकरुणास्तोत्रम् ।
९२. ललितात्रिशतीभाष्यम् ।
९३. ललितापञ्चरत्नम् ।
९४. विष्णुसहस्रनामभाष्यम् ।
९५. विष्णुभुजङ्गप्रयातम् ।
९६. विष्णुपादादिकेशान्तवर्णनम् ।
९७. वेदसारशिवस्तोत्रम् ।
९८. शारदाभुजङ्गप्रयाताष्टकम् ।
९९. शिवपञ्चाक्षरनक्षत्रमालास्तोत्रम् ।
१००. शिवनामावल्यष्टकम् ।
१०१. शिवकेशादिपादान्तवर्णनम् ।
१०२. शिवपादादिकेशान्तवर्णनम् ।
१०३. शिबभुजङ्गम् ।
१०४. शिवापराधक्षमापनस्तोत्रम् ।
१०५. शिवानन्दलहरी ।
१०६. षट्पदीस्तोत्रम् ।
१०७. सुब्रह्मण्यभुजङ्गम् ।
१०८. सौन्दर्यलहरी ।
अस्य व्याख्याः- अरिच्छित्कृता ``सुधाविद्योतिनी" । अमुद्रिता व्याख्येयं तिरुवनन्तपुरपुस्तकालये (C. O. L. 116 F.) लभ्यते । अस्यां व्याख्यायां सौन्दर्यलहरी प्रवरसेनेन कृतेति दृश्यते । प्रवरसेनाख्यः काश्चित् क्षत्रिय आसीत् । तस्य पिता द्रमिडाख्यः कश्चन राजा । माता वेदवतीनाम्नी । शुकाख्यः मन्त्री स च प्रवरसेनजननकालरीत्या द्रमिडस्य राज्यच्युतिमुवाच । भीतः द्रमिडः स्वपुत्रं प्रवरसेनं गिरिमूर्धनि तत्याज । क्षुधिताय तस्मै बालाय प्रवरसेनाय भगवती परमेश्वरी स्तन्तं पाययामास । स्तन्यपानलब्धवैदुष्यः प्रवरसेनः भगवतीस्तोत्ररूपां सौन्दर्यलहरी चकार । अत एव ``तव स्तन्यं मन्ये" इत्यादिना निर्दिश्यते । प्रवरसेनस्य कश्चित्पुत्रः वन्यायां भार्यायामुत्पन्नः । तस्य नाम अरिच्छिदिति । अनेन कृतेयं सुधाविद्योतिनीति कथापि दृश्यते ।
अन्ये तु ``तव स्तन्यं मन्ये" इति श्लोके दृश्यमानः द्रविडशिशुशब्दः तमिलभाषाशैवसाहित्यप्रसिद्धः तिरुज्ञानसम्बन्धनायनारिति वदन्तः पूर्वोक्तां कथामधिक्षिपन्ति ।
कौवल्याश्रमकृता सौभाग्यवर्धिनी, नरसिम्हकृता व्याख्या, डिण्डिमरामकविकृता व्याख्या, लक्ष्मीधरकृता लक्ष्मीधरा, अज्ञातकर्तृका विद्वान्प्ननोरमा, सदाशिवकृता व्याख्याश्च मद्रासराजकीहस्तलिखितपुस्तकालये लभ्यन्ते ।
१०९. हरिस्तुतिः ।
११०. हनूपत्पञ्चरत्नम् ।
१११. सुवर्णमालास्तोत्रम् ।
११२. द्वादशलिङ्गस्तोत्रम् ।
अन्योऽपि प्रपञ्चसाराख्यः तन्त्रशास्त्रग्रन्थः विरचितः । एवमन्येऽपि शङ्कराचार्यकृता इति मुद्रिता अमुद्रिताश्च बहवः ग्रन्था उपलभ्यन्ते । परन्तु नामान्तरेण निर्दिष्टा एत एव ग्रन्था इति विभावनीयम् ।
२७.पद्मपादाचार्यः (800 A. D.)
शङ्करभगवत्पादशिष्यतल्लजेष्वन्यतमोऽयं पद्मपादाचार्यः आश्रमस्वीकारात्पूर्वं सनन्दनापरनामा विमलनामकस्य ब्राह्मणस्य पुत्रः, नद्यास्तीरे स्थितोऽयं उत्तरतीरवर्तिना गुरुणाऽहूतः सत्वरागमनाय नदीजले पादौ निक्षिप्य पाथोरुहेषु क्रमेण पदं विनिक्षिप्याजगाम । तादृशीमनितरसाधरणीं गुरुभक्तिं दृष्ट्वा तुष्ट आचार्यस्तस्य पद्मपाद इति नाम व्यतानीदिति कथा प्रसिद्धा । स्वगुरोस्सकाशादनेन वारत्रयं भाष्यमपाठि । आचार्यानुज्ञातेनानेन भाष्यस्य पञ्चपादिका नाम्नी टीका व्यरचि ।
टीकामेतां पूर्वमीमांसापण्डितस्य कर्मनिष्ठस्य कस्यचित् स्वबन्धोर्गृहे निक्षिप्य पद्भपादाचार्यः तीर्थयात्रायै जगाम । द्वेषात् स बन्धुः लोकापवादभीतेश्च पञ्चपादिकां भस्मसात् कर्तुमिच्छन् स्वगृहमेव भस्मसात् चकार । तीर्थयात्रायास्समागतेन पद्भपादाचार्येण पुनरपि स्वगुरुणा शङ्कराचार्येण स्मृतिपथमानीतं यद्यदुक्तं तत्सर्वं विलिखितमिति निर्मूला दन्तकथा श्रूयते ।
शङ्कराचार्यशिष्योऽयं अष्टमशतकीय इति तु सामान्यसिद्धन्तः । म. म. कुप्पुखामिशास्त्रिणां तु सप्तमशतकमिति (625-705 A.D) विशेषसिद्धान्तः । अस्य सतीर्थ्याः सुरेश्वरहस्तामलकतोटकाचार्याः प्रसिद्धाः । अद्वैतवेदान्तसाहित्ये पद्भपादाचार्यः विशिष्टप्रस्थानस्य प्रवर्तकः । तच्च प्रस्थानं पञ्चपादिकाप्रस्थानमिति प्रसिद्धम् ।
१. पञ्चपादिका-(V. N. S. S. 3)
सूत्रभाष्यव्याख्यात्मकोऽयं ग्रन्थश्चतुस्सूत्र्यन्त एवोपलभ्यते । नवभिः वर्णकैपूर्णोऽयं ग्रन्थः विजयनगरसंस्कृतग्रन्थमालायां, निर्णयसागरमुद्रणालये, वाणीविलासमुद्रणालये, कल्कत्तासंस्कृतग्रन्थमालायां, च मुद्रितः । ग्रन्थस्य नामपर्यालोचने पञ्चभिः पादैर्भाव्यम् । परन्तु प्रथमेऽध्याये प्रथमे पादे चतुस्सूत्र्यन्त एवोपलभ्यते । अस्य व्याख्याः - आत्मसर्वज्ञकृता ``प्रबोधपरिशोधिनी," आनन्दपूर्वविद्यासागरकृता पञ्चपादिकाव्याख्या उत्तमज्ञयतिकृता वक्तव्यप्रकाशिका, नृसिम्हाश्रमिकृता वेदान्तरत्नकोशः, प्रकाशात्मयतिकृतं पञ्चपादिका विवरणम्, रामानन्दकृता त्रय्यन्तभावदीपिका, धर्मराजाध्वरिकृता पञ्चपादिकाव्याख्या, विज्ञानात्मकृता तात्पर्यद्योतिनी, अज्ञातकर्तृका च पञ्चपादिका व्याख्याः ग्रन्थेऽस्मिन् प्रतिपादिताः ।।
२. विज्ञानदीपिका- (A. U. S. S. J)
पद्यबद्धोऽयं ग्रन्थः । ग्रन्थेऽस्मिन् सकलवेदान्ततत्वान्यवलोङ्य विविधकर्मपाशबन्धनच्छेदेनैव मुक्तिर्भवितुमर्हतीति निश्चित्य विश्वजनीनं मोक्षमार्गं सुलभं कर्तुं केनोपायेन कर्मनिर्मुक्तिर्भविष्यतीति ज्ञापनमुखेन फलानुसन्धानरहितकर्मानुष्ठानद्वारा साक्षात् परम्परया वा विज्ञानं जायत इति तत्वज्ञानप्रदीपिकेयं प्रदीपिता । अस्य व्याक्यापि मूलकृतैव कृता विद्यते । मुद्रितश्चायं ग्रन्थः अलहाबाद विश्वविद्यालयग्रन्थमालायाम् ।
३. आत्मबोधव्याख्या-(D. 4558 MGOML)
शाङ्करात्मबोधव्याख्यात्मकोऽयं ग्रन्थः मद्रासराजकीयपुस्तकालये अमुद्रित उपलभ्यते । बरोडापुस्तकालयेऽपि लभ्यते ।
४. आत्मानात्मविवेकव्याख्या - (686 C. C. P. B.)
५. कठोपनिषद्भाष्यम् - (742 C. C. P. B.)
६. कर्मनिर्णयः - (686 C. C. P. B.)
७. तत्वमसिपञ्चकम् - (77 Nasik Vol. XXVI 52) ग्रन्थोऽयं नासिक सूच्यां दृश्यते ।
८. प्रपञ्चसारव्याख्या - (686 C. C. P. B.)
९. स्वरूपानुभवः - (7730 TSML.)
एते ग्रन्थाः पद्भपादकृता इति निर्दिष्टाः । परन्तु प्रबलप्रमाणानि नोपलभ्यन्ते । आदर्शग्रन्थाश्च नावलोकितुं पार्यन्ते च ।
२८. सुरेश्वराचार्यः (800-900 A. D.)
``श्रीमच्छङ्करपादपद्मयुगलं संसेव्य लब्ध्वोचिवान् " इति नैष्कर्म्यसिद्धौ वदन्नयं सुरेश्वराचार्यः शङ्कराचार्यशिष्यः पद्मपादतोटकहस्तामलकानां सतीर्थ्यश्च ।
पूर्वाश्रमे शोणानदीतीरवासी पञ्चगौडान्तर्गतः कुमरिलभट्टजामाता पूर्वकाण्डप्रवर्तकः मण्डनमिश्र इति ख्यातः विश्वरूप एव सन्यासस्वीकारादनन्तरं सुरेश्वर इति प्रसिद्ध इति साम्प्रदायिका वदन्ति । जागोपमहाशयेन नैष्कर्म्यसिद्धिभूमिकायां मण्डनमिश्रसुरेश्वरविश्वरूपाणामैक्यंमङ्गीक्रियते । सप्तदशशतकीयेन बालकृष्णानन्दसरस्वत्या कृते शारीरकमीमांसाभाष्यवार्तिके च त्रयाणामैक्यमेवोपवर्णितम् । विद्यारण्यैः विवरणप्रमेयसंग्रहे बृहदारण्यकवार्तिकादुद्धरणं दत्तम् । तत्रापि विश्वरूपशब्देन सुरेश्वरः निर्दिष्टः ।
दासगुप्तमहाशयस्तु सुरेश्वरविश्वरूपावभिन्नौ मण्डनमिश्रस्त्वन्य इति वदति । हिरियण्णामहाशयस्तु (J.R. A. S. 1924) पत्रिकायां सुरेश्वरः मण्डनादन्य इति निश्चिनोति । म. म कुप्पुस्वामिशास्त्रिणस्तु सुरेश्वरब्रह्मसिद्धिकारयोस्सिद्धान्तगत भेदमुपवर्ण्य ब्रह्मसिद्धिकारः सुरेश्वरादन्य इति प्रतिपादयन्ति । संक्षेपशारीरक कर्ता सर्वज्ञात्मा सुरेश्वर (देबेश्वर) शिष्य इति प्रसिद्धिः । श्रीकण्ठशास्त्री तु नायं सर्वज्ञात्मगुरुरिति (I. H. Q. Vol . XIV) वदति । दासगुप्तस्तु सर्वज्ञात्मा सुरेश्वरशिष्य इत्येव वदति । श्रीकण्ठशास्त्रिणा प्रदर्शितायां श्रृङ्गगिरिगुरुपरम्परायां नित्यबोधघनाभिघः नित्यबोधाचार्यस्सुरेश्वरशिष्य इति निर्दिष्टम् । नित्यबोधाचार्यकालश्च (773 - 848 A.D.) पर्यन्तमिति च । तस्मात् सर्वज्ञात्मन एव नित्यबोधाचार्य इत्यपि नामान्तरं स्यादिति स्वीकारोऽपि सुष्ठु लगति ।
शङ्कराचार्यकाल एव सुरेश्वरकालः । तस्मात् नवमशतकीयस्सुरेश्वरः । कुप्पुस्वामिशास्त्रिणश्च  (602 - 700 A.D.) इति वदन्ति । श्रृङ्गगिरिगुरुपरम्परायां सुरेश्वरकालः (695 - 777 A.D.) इति दृश्यते ।
१. तैत्तरीयोपनिषद्वार्तिकम् - (A. S. S. 13)
शाङ्करतैत्तरीयभाष्यस्य पद्यमयी वार्तिकनाम्नी व्याख्या आनन्दाश्रममुद्रणालये वाराणसीग्रन्थमालायाञ्च मुद्रिता । अस्या व्याख्याः - आनन्दगिरिकृता वार्तिकटीका, विश्वानुभवकृता वार्तिकसङ्गतिः, लिङ्गनसोमयाजिकृतम् - कल्याणविवरणम् , विद्यन्ते ।।
तैत्तरीयोपनिषत्
%%% Chart
२. नैष्कर्म्यसिद्धिः
३. प्रणवार्थकारिकाः
४. पञ्चीकरणवार्तिकम् 
५. बृहदारण्यकोपनिषद्भाष्यवार्तिकम् 
६. मानसोल्लासः
७. मोक्षनिर्णयः
८. वेदान्तसारवर्तिकराजसंग्रहः
९. लघुवार्तिकम्

 ``" ``" ``" ``" ``" ``" ``"

ऽ  ?
``" ``" ``" ``" ``" ``" ``" ``" ``" ``" ``" ``" ``" ``" ``" ``" ``" ``" ``" ``" ``" ``"
`' `' `' `' `' `' `' `' `' `' 

ऽ  ।   ॥ ?
