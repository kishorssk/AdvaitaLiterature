\chapter{शङ्करात्प्राचीना अद्वैताचार्याः}
द्वितीयो भागः
अद्वैताचार्याः, अद्वैतग्रन्थप्रणेतारश्च ।
अद्वैताचार्याः अद्वैतमतप्रतिपादकग्रन्थविशेषप्रणेतारश्च कालभेदेन द्वेधा विभक्तुं शक्यन्ते -``शङ्करभगवत्पादेभ्यः प्राचीनाः, शङ्करभगवत्पादेभ्य अर्वाचीनश्चेति ।" तेष्वपि शङ्करभगवत्पादेभ्यः प्राचीनेषु आचार्येषु १. ब्रह्मसूत्रकारात् शङ्करभगवत्पादेभ्यश्च प्राचीनाः, २. ब्रह्मसूत्रकारात् अर्वाचीनाः शङ्करभगवत्पादेभ्यः प्राचीनाश्चेति विभागोऽपि कर्तुं शक्यते । अद्वैतमतसिद्धान्तस्य प्राचीनतमत्वात् ब्रह्मसूत्रेषु बहूनां वृत्तिग्रन्थानां सत्वानुमानाच्च । शङ्करभगवत्पादकृतभाष्यप्रभावात् बौद्धप्रभावाच्च प्राचीनं वृत्यादिकं विनष्टम् । अथवा शाङ्करभाष्येण तेषां सम्पूर्णतया गतार्थत्वात् तेषां संरक्षणे औद सीन्यवशाच्च विनष्टं जातम् । तत्वचन्द्रिकाकारेणोमामहेश्वरेण तत्वचन्द्रिकायां (R 5156 MGOML) ``शङ्करभगवत्पादैः स्वीये भाष्ये एकोनशतं सृत्रवृत्तिग्रन्थाः परामृष्टाः, केचित् खण्डिताश्चेति" निर्दिश्यते । शाङ्करभाष्यदर्शकाणां सर्वेषाञ्च मनसि ईदृशी चिन्ता स्वाभाविकी यत् `शाङ्करभाष्ये शङ्कराचार्येण ततोऽपि प्राचीनवृत्तिकारस्य व्याख्यानं बहुषु स्थलेषु खण्डितम् । तदर्थं तत्र तत्र युक्तिरपि प्रदर्शिता । परन्तु तासां वृत्तीनां नाम, वृत्तिकारादीनां नाम वा न कुत्रापि निर्दिष्टम् । परन्तु व्याख्यानादिकर्तृभिः तेषां नाम तत्र तत्र निर्दिष्टं क्वचित् क्वचित् । शाङ्करभाष्ये पूर्वपक्षत्वेन गृहीता एव भास्कररामानुजादिभिः सिद्धान्तत्वेन स्वीकृता वर्तन्ते । तस्मात् शङ्करभाष्ये पूर्वपक्षत्वेनोपन्यस्तानां मतवादानां मध्ये कियन्तो मतवादाः पूर्वैरुद्भाविताः? कियन्तो वा बौधायनादिभिः प्रकल्पिताः? कति वा शङ्करभगवता स्वयं समुद्भाविताः? किं ते शाङ्करभाष्यानुगाः ? उत शङ्करसिद्धान्तेन अंशतस्सदृशाः ? इत्यादिविषया न निर्णेतुं शक्यन्ते । तथा च बोधायनोपवर्षादिभिस्सह भास्कररामानुजादीनां सम्बन्धो यथा दुर्निर्णेय स्तथा वृत्तिकारादीनामपीत्येवावगम्यते । एवमपि बोधायनोपवर्षभर्तृप्रपञ्चभर्तृहरि ब्रह्मनन्दिसुन्दरपाण्डयद्रविडाचार्यब्रह्मदत्तादिवेदान्ताचार्याः ब्रह्मसूत्रेषु वृत्तिग्रन्थान् प्राणैषुरिति परं ज्ञायते । व्याख्यानपरम्परैवात्र प्रमाणम् । एतेषां ग्रन्थानामिदानीमनुपलम्भात् ।'
एवं ब्रह्मसूत्रेवपि बादरायणव्यासेन केचनाचार्याः नाम्ना निर्दिष्टाः . तेषाञ्च ग्रन्था नोपलभ्यन्ते । तथापि तेषां मतवादादिकं विविधपत्रिकादिप्रमाणानुसारं यथाकथञ्चित विवृणीतुं प्रयतामहे । तेषु आचार्येषु ``अष्टावक्र - आत्रेय - आश्म रथ्य - उपवर्ष औहुलौमि - काशकृत्स्न-कार्ष्णाजिनि-जैमिनिबादर्यादयः व्यासात्प्राचीनाचार्यविभागे, आचार्यसुन्दरपाण्डूय द्रविडाचार्य ब्रह्मदत्त ब्रह्मनन्दि भर्तृप्रपञ्च भर्तृ रि बदिरायणापरव्यासाचार्याः शङ्करात्माचीनाचार्यविभागे, च विभक्तुं शक्यन्ते । एवमनिर्णीताः अज्ञातसमयाः विभिन्नविचारलक्ष्यीभूता अपरे अध्यात्मरामायणकारआदिशेषकाश्यपजाम्बवतदत्तात्रेययोगवासिष्ठ कारशु कसनत्सुजातादयश्च वर्तन्ते । तेषां सर्वेषां इतिहासादिकं प्रकरणेऽस्मिन् वर्णम लाक्रमेण निरूप्यते ।।"

१. अष्टावक्रः
अष्टावक्रोऽयं महर्षिः सुजाताकहोलयोः पुत्रः, उद्दालकस्य दौहित्रः, श्वेतकेतोस्स्वस्रीयः, वदान्यजामाता, सुपभाभर्तेति ज्ञायते । उद्दालकनामा महान् ऋषिंः वहोलाय स्वशिष्याय स्वकन्यां सुजातानाम्नीं वैवाहिकेन विधिना ददौ । शिप्यतध्ये अधीय नं स्वपितरं कहोलं सुजातागर्भस्थः अग्निकल्पशिशशुः रात्रिन्दिवं कहो रुस्य ध्ययनशीलतां च प्रति अधिचिक्षेप । क्रुद्धः मातामह उद्दालकस्स्वदौहित्रं ``यस्मात त्वं कुक्षौ वर्तमानोऽधिक्षिपसि तस्मात् त्वं अष्टकृत्वः वक्रो भवितासि" इति शशाप । तथा स वक्र एवाभ्यजायत अष्टावक्र इति प्रथितश्च । स्वपत्नीप्रेरणया वित्तार्जनार्थं जनरूपुरं गत अष्टावक्रपिता कहोलः जनकपुरद्वारपालेन बन्दिना वादे पराजितः जले मिमज्य मृतश्च । मातृपकाशात् पितृवृत्तान्तं श्रुत्वा मातुलेन श्वेतकेतुता सह अष्टावक्रः जनकपुरं गतः, वादेषु द्वारपालं जनकञ्च जित्वा ``समङ्गापरनाम्नीं मधुविलानदी स्नात्वा अष्टावक्रेदेहं समीकृत्य जरर्तरूपधारीण्या उत्त दिगभिपानिदेवतायाः धर्मो देशं स्वीकृत्य वदान्यकन्यां सुप्रभानाम्नी विधिनोपयेमे । इति कथा महाभारतादिप्रसिद्धा । अनेन कृतः ग्रन्थः-"
(क) अष्टावक्रगीता
अष्टावक्रसृक्तम्, अवधूतानुभूतिः, इत्यादिनाम्ना प्रसिद्धोऽयं ग्रन्थः ए विंशतिभिाध्यायैः पूर्ण अष्टवक्रजनकसंवादरूपेण अद्वैतवेदान्तसिद्धान्तान् ब्रह्मणोऽद्वितीयत्वं चिन्मयत्वञ्च प्रतिपादयति । ग्रन्थोऽयं आष्टेकरकम्पनि पूनानगरे मुद्रितश्च । अस्य व्य ख्याः विश्वेश्वरकृता - दीपिघख्या, पूर्णानन्दतीर्थकृता काचन व्याख्या, मुकुन्दमुनिकृता अन्या व्याख्या, भासुरानन्दकृता अपरा व्याख्या इत्येवं चतस्रः व्याख्या उपलभ्यन्ते ।
२. आचार्य सुन्दरपाण्ड्यः (600 A.D.)
शङ्करभगवत्पादेभ्यः प्राचीनोऽयं सुन्दरपाण्डयाचार्यः दक्षिणद्रविडदेशीयः, मधुरानगरवासीति ज्ञायते । पूर्वोत्तरमीमांसयोः प्रकाण्डपण्डितेनानेन ब्रह्मसूत्रणां वार्तिकं विरचितमिति, कैस्तवीय षष्ठशतकात् प्राचीनः, षष्ठशतकीयो वा इति निश्चीयते । अत्रेमानि कारणानि-
स्वर्गीयमहामहोपाध्याय कुप्पुस्वामिशास्त्रिणः ``जर्नल आफ ओरियण्टल रिसर्च मद्रास" पत्रिकायाः प्रथमे भागे (J.O.R.I.) एवमभिपयन्ति । शङ्करभगवत्पादैस्स्वोये ब्रह्मसूत्रमाष्ये सनन्वय - अधिकरणभाष्यान्ते अपिवाहुः-
``गौणमिथ्य त्मनोऽपत्वे पुत्रदेहादिबाधनत् ।
सद्ब्रह्म त्माहं इत्येव बोधकार्थं कथं भवेत् ।।
अन्वेष्टव्यात्मबिज्ञानात् प्राक् प्रमातृत्वमात्मनः ।
अन्विष्टस्स्यात् प्रमातैव पाप्मदोषादिवर्जितः ।।
देहात्मप्रत्ययो यद्वत् प्रप्ताणत्वेन कल्पितः ।
लौकिकं तद्वदेवेदं प्रनाणं त्वात्मनिश्चयात् ।।" इति 
श्लोकत्रयमुदाहृतम् । अत्र भामतीकारैः ``अत्रैव ब्रह्मविदां गाथामुदाहरति" इत्यवतारितम् । पञ्चपादिकाकारैः ``प्रसिद्धपेतत् ब्रह्मविदां पूर्वोक्तं न्याय संक्षेपतः श्लोकैस्संगृह्नणाति" इत्यवतारितम् । पञ्चपादिकाव्याख्यात्रा नरसिम्हस्वरूपशिप्येण आत्मस्वरूपेण स्वीयप्रबोवपरिशोधिन्यां `श्लोकत्रयं सुन्दरपाण्डयाचार्यप्रणीतं प्रमाणयति' इत्यवतारितम् ।
माधवनन्त्रिणा विरचितायां सूतसंहितयाः व्याख्यायां तात्पर्यदीपिकाख्यायां ``देहात्मप्रत्ययो यद्वत् प्रप्ताणत्वेन कल्पितः । लौकिकं तद्वदेवेदं प्रमाणन्त्वात्मनिश्चयात् ।" इत्ययं श्लोक उद्धृतः । एतच्छ्रलोकविवरणावसरे तथा सुन्दर पाण्ड्यवार्तिकमपीति अवतारिका कृता । 
एवं त्रयोदशशतकीयेनामलानन्देन स्वरविते कल्पतरुग्रन्थे (3 - 3 - 25 Page 755 N.S.P. Edn.) ``आह चात्र निदर्शनमाचार्यसुन्दरपाण्ड्यः" इति-
निःश्रेण्यारोहणप्राप्यं प्राप्तिमात्रोपपादि च ।
एकमेव फलं प्राप्तुं उभावारोहतो यदा ।।
एकसोपानवर्त्येको भूमिष्ठश्चापरस्तयोः ।
उभयोश्च जवस्तुल्यः प्रतिबन्धश्च नान्तरा ।।
विरोधिनोस्तदैको हि तत्फलं प्राप्नुयात्तयोः ।
प्रथमेन गृहीतेऽस्मिन् पश्चिमोऽवतरेन्मुधा ।। इति 
श्लोकत्रयमुपपादितम् । एतदेव श्लोकत्रयं कुमरिलभट्टैर्बलाबलाधिकरणे तन्त्रवार्तिके (BSS Page 852-853) आह चेत्यादिना प्रतिपादितम् । एवञ्च शङ्करभगवत्पादेभ्यः कुमरिलभट्टाच्च प्राचीन इति सिध्यति ।
आचार्यसुन्दरपाण्ड्यकृतः नीतिद्विषष्ठिकाख्यः ग्रन्थः कश्चन नीतिपरः मद्रपुर्यां प्रकाशितः । तत्रस्थाः बहवः श्लोकाः त्रयोदशशतकीयेन जल्हणेन सूक्तिमुक्तावल्यां, पञ्चदशशतकीयेन वल्लभदेवेन सुभाषितावल्यां, पञ्चदशशतकीयेन शार्ङ्गधरेण स्वीयशार्ङ्गधरपद्धत्यां पोतयार्येण प्रसङ्गरत्नावल्यां, पेद्दिभट्टेन सूक्तिवारिधौ, द्वादशशतकीयेन कलिङ्गराजापरनाम्ना सूर्यपण्डितेन कुलशेखरसूक्तिरत्नहाराख्ये ग्रन्थे चोदाहृताः । विशेषतः सूर्यपण्डितेन ``आचार्यसुन्दरपाण्ड्यकृता" इति निर्दिष्टाश्च । पञ्चतन्त्रकर्ता विष्णुशर्मा क्रैस्तवीयषष्ठशतकादर्वाचीन इति विमर्शकसिद्धान्तः । तेनापि नीतिद्विषष्ठिकास्थाः 29, 30, 48, श्लोकाः स्वीये ग्रन्थे उद्धृताः । एवं नीतिद्विषष्ठिकाग्रन्थावसाने कश्चन श्लोकः - ``इमां काञ्चनपीठस्थां समेत्य कवयो भुवि । आर्यां सुन्दरपाण्ड्यस्य स्नापयन्ति वधूमिव ।" दृश्यते । क्रैस्तवीय षष्ठशतकादारभ्य द्रविडेदेशे मधुरायां द्रविडसङ्घस्स्थापितः । अभ्यर्हिंत कविं तत्कृतिञ्च तत्सङ्घस्थाः विद्वांसः सङ्धपूजिते काञ्चनपीठे निवेश्य कनकाभिषेकमकुर्वन्निति तमिलसाहित्ये प्रसिद्धम् । तादृशस्तत्कारः ग्रन्थस्यास्यनीतिद्विषष्ठिकाख्यस्यापि प्रवृत्त इत्येवास्मात् पद्यात् ज्ञायते ।
``श्रीमच्छकाब्देऽब्धिशशिसायकसम्मिते । राजा माधववर्माभूत् विख्यातो धरणीतले ।।" इति प्रसङ्गरत्नावल्याख्ये पोतयार्यकृते ग्रन्थे दर्शनात्, पेण्डयाल सुब्रह्मण्यशास्त्रिभिः प्रकाशितात् पुलिवूरुशिलाशासनप्रमाणाच्च 514 शके 592 A.D. काले विष्णुकुण्डिनवंश्यः माधववर्मापरनामा जनाश्रयाख्यः कृष्णानदीतीरान्ध्रदेशाधीश आसीदिति ज्ञायते । तेन राज्ञा कृता कृतिः जानाश्रयीति च प्रसिद्धा । तस्मिन् जानाश्रयीत्यपरनामके छन्दोग्रन्थे नीतिद्विषष्ठिकायाः चतुर्विशतितमः ``चारित्रनिर्मल जलः सत्पुरुषनदोऽक्षयो भवतु नित्यम् । यस्य विभवारविन्दे विद्वद्भमराः कृतविनोदाः ।।" इति शलोक उद्धृतः ।
तस्मात्-कुमरिलभट्टेन, शङ्कराचार्येण, विष्णुशर्मणा, जनाश्रयेण, माधवमन्त्रिणा च प्रमाणीकृतोऽयं सुन्दरपाण्ड्याचार्य क्रैस्तवीयषष्ठशतकात् प्राचीनः, षष्ठशतकीयो वेति निश्चप्रचमभ्युपगम्यते ।
स्वर्गीय महामहोपाध्याय कुप्पुस्वामिशास्त्रिण आचार्यसुन्दरपाड्यमेनं क्रैस्तवीयाष्टमशतकीयं प्रवदन्ति । तमिलसाहित्ये प्रसिद्धः (अषकेसरी) एवायमिति जर्नल आफ ओरियण्टल पत्रिकायाः प्रथमे भागे J. O. R. I. निरूपयन्ति । इतिहासनिपुणाः K. A. नीलकण्ठशास्त्रिणस्तु जर्नल आफ ओरियण्टलरिसर्चपत्रिकायाः प्रथमे भागे (J. O. R. Madras-1) सप्तमशतकमध्यकाले सुन्दरपाण्ड्याचार्य आसीदिति प्रतिपादयन्ति । सर्वथापि शङ्करभगवत्पादेभ्यः प्राचीनोऽयमित्येवास्मत्सिद्धान्तः । ``श्रीमान् सुन्दरपाड्यः श्रुति स्मृति प्रसृत सत्पदार्थज्ञ" इति नीतिद्विषष्ठिकायां दर्शनात आचार्यसुन्दरपाण्ड्योऽयं पूर्वोत्तरमीमांसादिषु निष्णात इति ज्ञायते । शाङ्करभाष्ये उद्धृत्य प्रमाणीकृतत्वात् वेदान्तेऽनेन ब्रह्मसूत्राणां किमपि वार्तिकं पद्यबंद्ध कृतं स्यादिति निश्चीयते । कुमरिल भट्टैरुद्धृतोऽयं पूर्वमीमांसायामपि ग्रन्थप्रणेता इत्यभ्यूह्यते ।
अस्यैवाचार्यसुन्दरपाण्ड्यस्य द्रविडाचार्य इत्यपि नामान्तम्, व्यावहारिकनाम वा स्यादित्यूह्यते । अत्रेदं कारणं भवति-अभिनवद्रविडाचार्यापरनाम्ना अष्टादशशतकीयेन बालकृष्णानन्दसरस्वत्या विरचिते पद्यबद्धे शारीरकमीमांसाभाष्यवार्तिके आशुतोषग्रन्थमालामुद्रिते शङ्करभाष्यस्थस्य `अपिचे'ति ग्रन्थस्यावतरणसमये ``कथितार्थपरं द्रविडार्यकृतां अपिचेति गुरुर्वदतीह कथाम्" (ASI Page 403) निर्दिष्टम् । तस्मात् आचार्यसुन्दरपाण्ड्योऽयं द्रविडाचार्य इत्यपि व्यवहृतस्स्या दिति निश्चीयते ।
३. आत्रेयः
व्यासात्पूर्वतनेषु वेदान्ताचार्येषु अन्यतमोऽयमात्रेयः । बादरायणव्यासनिमिंतेषु ब्रह्मसूत्रेषु स्वामिनः फलश्रुतेरित्यात्रेयः 3-4-44 इति आत्रेयोऽयं निर्दिष्टः । यज्ञे अङ्गाश्रीतोपासना यज्ञस्वामिना एवं ऋत्विग्भिश्च कर्तव्या । अत्र फलविषये संशयः । किं उपासनाजन्यफलभाक् यजमानः ? उत ऋत्विक् ? इति । अत्रात्रेयमतन्तु अङ्गाश्रितोपासनाफलभाग्यजनमान एवेति । अदसीयः वेदान्तग्रन्थस्तु नोपलभ्यते ।
४. आदिशेषः (परमार्थसारकारः)
``वेदान्तशास्त्रमखिलं विलोक्य शेषस्तु जगदाधार" इति परमार्थमारे दृश्यते । साघवानन्दकृतायां व्याख्यायां ``भगवता जगदाधरेण आदिशेषेण" इति दृश्यते । तस्य त् सहस्रफणामणिमणिमण्डल आदिशेष एवास्य परमार्थसारस्य कर्तेति साम्प्रदायिकविश्व सः । परन्तु विधुशेखरभट्टाचार्यास्स्वपम्पादिवि ``गौडपदीयं आगमशास्त्र" मिति ग्रन्थे गौडपादाचार्यकालात् भास्कराचार्यकालस्य च मध्यवर्तिना केतापि आदिशेषनाम्ना ग्रन्थरचना कृतेति परमार्थसारग्रन्थकारकालः (500-800 A.D.) इति प्रवदन्ति ।
परमार्थसारः - (TSS 12)
आर्यावृत्तघटितैः पद्यैरद्वैतपरमार्थसारान् शिष्योपदेशशैल्यां प्रतिपादयन्नयं ग्रन्थः अभिनवगुप्ताचार्यकृतात् परमार्थसाराद्भिन्नः मुद्रितश्च चौखाम्बामुद्रणालये । अनन्तशयनग्रन्थावल्याञ्च सव्याख्योऽयं मुद्रितः । अस्य व्याख्याः- १ राघवानन्द मुनिकृता विवरणनाम्नी काचन २ वासुदेवयतिकृता अन्या प्रकाशि कानाम्नी व्याख्या अमुद्रिता (R. 4149 C. MGOML) लभ्यते ।
५. आश्मरथ्यः
ब्रह्मसूत्रकारात् बादरायणव्यासात् पूर्वतनोऽयं आश्मरथ्यः ब्रह्मसूत्रेषु वैश्वानराधिकरणे ``अभिव्यक्तेरित्याश्मरथ्यः 1-2-29, एवं वाक्यान्वयाधिकरणे प्रतिज्ञासिद्धेर्लिङ्गमित्याशमरथ्यः" 1-1-29 इति वारद्वयं निर्दिष्टः ।
उपनिषत्सु ईश्वरः प्रादेशमात्रः प्रतिपादितः । अस्य उपपतिरनेनैवं क्रियते ``परमेश्वर अनन्तः । भक्तानुग्रहार्थं प्रादेशमात्रादुद्भवति । हृदयादिषु उपलब्धियोग्येषु प्रदेशेषु उपलभ्यमानोऽयमिति प्रादेशमात्र इति च । भेदाभेदवाद्ययम् । कार्यावस्थायां विज्ञानात्मा परमात्मनः भिन्नः । कारणावस्थायान्तु अभिन्न इत्यस्य सिद्धान्त इति ज्ञायते ।"
६. उपवर्षाचार्यः (100 BC - 200 AD)
उपवर्षाचार्योऽयं पूर्वोत्तरमीमांसयेर्वृत्तिकारः, शङ्करभगवत्पादेभ्यः शबरस्वामिनोऽपि प्राक्तन इति ज्ञायते । शङ्करभगवत्पादैस्स्वीये सूत्रभाष्ये ऐकात्म्याधिकरणे 3-3-53 सूत्रे ``अत एव च भगवतोपवर्षेण प्रथमतन्त्रे आत्मास्तित्वाभिधानप्रसक्तौ शारीरके वक्ष्याम" इत्युद्धारः कृत इति सबहुमानमुपवर्षाचार्य आवेदितः । प्रकटार्थकारैरपि ``अतएवेत्य दि" भाष्यव्याख्यानावसरे `वृत्तिकारवचनं गमकमित्याह इत्येव शाङ्करभाष्यमवतारितम् । एवमानन्दमयाघिकरणे शङ्कराचार्यैः प्रथमं वृत्तिकारमतानुसारेण सूत्राणि व्याख्यातानि । अनन्तरं' `इदं त्विह वक्तव्यम्' इत्यादिना अधिकरणान्ते वृत्तिकारमतं पूर्वपक्षीकृत्य सिद्धान्तविधया स्वीयसिद्धन्तः प्रदर्शितः । एवमन्यत्रापि `अन्ये त्वाहुः' `अपरे त्वाहुः' इत्यादिना शङ्करभगवत्पादर्वत्तिकारमतमनूदितम् ।
शबरस्वामिना च ``वर्णा एव तु शब्दाः" इति भगवानुपवर्ष इति उपवर्षाचार्यः प्रमाणीकृतः । एवञ्च ब्रह्मसूत्राणां वृत्तिकार उपवर्षाचार्य अद्वैतमतैकदेशी प्र. चीन इति निर्णीयते । तादृशी उपवर्षीया वृत्तिस्तु नोपलभ्यते कुत्रापीदानीम् ।
ग्रन्थानुपलव्धेरेव विशिष्टाद्वैतिनः उपवर्षाचार्यस्यैव बोधायनकृतकोटिरित्यपि नामान्तरमिति वर्णयन्तस्स्वमतसंरक्षकबोधायनवृत्तिसत्यत्वसंरक्षणाय मणिमेखलादि द्रविडभाषाग्रन्थअवन्तिसुन्दरीग्रन्थप्रपञ्चहृदयग्रन्थमुखेन बोधायनकृतकोटिउपवषत्रयस्य अभिन्नतां साधयितुं प्रकटप्रयत्नमकुर्वन् । परन्तु तेषां प्रयत्नो विफल इति बोध यनकृतकोटिउपवर्षाचार्याः भिन्ना एवेति बोधायनवृत्तिस्तु नास्त्त्ये वेते ब्रह्मश्री पोलकं श्रीरामशास्त्रिभिरस्मद्गुरुचरणैः स्वीये द्रविडात्रेयदर्शने प्रतिपादितम् ।
आचार्य भगवद्दत्तैस्तु स्वीये ``भारतवर्ष का बृहद् इतिहास" नामके ग्रन्थे प्रथमभागे 84 पुटे शबरस्वामिनां कालः विक्रम चतुर्थशतकात् प्राचीन इति प्रतिपादितम् । तस्मादुपवर्षकालः 200 A.D. कालात् प्राक्तन इति तु निर्णीयते ।।
उपवर्षः वर्षोपाध्यायस्य कनिष्ठभ्राता पाणिनीयवृत्तिकृत् कात्यायनस्य श्वशुरः, उपकोशाया जनकः, महापद्भनन्दस्य प्रधानमन्त्रीति 500 A. D. काले आसीदिति च ``वृद्धत्रय्यां" गुरुपादशर्महालदारः ।
७. औहुलोमिः
व्यासात्पूर्वतनेषु आचार्येषु औडुलोमिरप्यन्यः । औडुलोमिरयं ब्रह्मसूत्रेषु वाक्यान्वयाधिकरणे ``उत्क्रमिष्यत एवं भावादौडुलौभिः" 1 - 4 - 21 इति, स्वाम्यधिकरणे ``आर्त्विज्यमौडुलोमिस्तस्मै हि परिक्रीयते" 3 - 4 - 45 इति, ब्राह्माधिकरणे ``चितिमात्रेण तदात्मकत्वादित्यौडुलोमि" 4 - 4 - 6 इति च स्थलत्रये निर्दिष्टः ।
संसार-मोक्षकालभेदेन जीवब्रह्मणोर्भेदाभेदवादी अयमौडुलोमिः । दृश्यप्रपञ्चेऽज्ञानवशात् जीवब्रह्मणोर्भेदः । मुक्तावस्थायान्तु उभयोरप्यभेद इत्यस्य मतं स्यादित्युह्यते । भामतीकारोऽपि मतमेतदीयं प्रतिपादयति । 
८. काशकृत्स्नः
अविकृतः परमेश्वरो जीवः, नान्य इति सिद्धान्तवाद्ययं काशकृत्स्नः ब्रह्मसूत्रेषु वाक्यान्वयाधिकरणे ``अवस्थितेरिति काशकृत्स्नः" 1 - 4 - 22 निर्दिष्टः ।।
९. काश्यपः
नायं ब्रह्मसूत्रकारैर्निर्दिष्टः । परन्तु ``शाण्डिल्यभक्तिसूत्रे तामैश्वर्यपरां काश्यपः परत्वात्" No. 29 इति निर्दिष्टः ।
१०. काष्णाजिनिः
छान्देग्योपनिषदां पञ्चमाध्याये श्रूयमाणस्य ``रमणीयचरणा" इति ग्रन्थस्य व्याख्यानावसरे कार्ष्णाजिनिमतं ब्रह्मसूत्रे निर्दिष्टम् । कृतात्ययाधिकरणे ``चरणादिति चेन्नोपलक्षणार्थेति कार्ष्णाजिनि" 3 - 1 - 9 सूत्रेण निर्दिष्टः ।।
११. जाम्बवान्
एतत्कृतत्वेन प्रसिद्धस्य प्रणवमहाभाष्याख्यग्रन्थस्य उपान्त्यवाक्यात् रामचन्द्रभक्तो भगवान् जाम्बवानेवायमिति प्रतीयते । यद्येवं तर्हि प्रजापति पुत्रोऽयं त्रेतायुगादारभ्य वर्तमानश्चिरञ्जीवी जाम्बवतीपिता भगवतः कृष्णस्य श्वशुरश्चेति निर्णेतुं शक्यते ।
प्रणवमहाभाष्यम् -
प्रणवार्थप्रकाशकोऽयं ग्रन्थः माण्डूत्योपनिषदन्तर्गतं ओङ्कारोपासनार्थ प्रदर्शयति । आह्निकत्रयपूर्ण अमुद्रितोऽयं ग्रन्थःतिरुवनन्तपुरपुस्तकालये 306 TCD दृश्यते ।।
१२.जैमिनिः
जैमिनिरयं ब्रह्मसूत्रेषु वैश्वानराधिकरणे देवताधिकरणे बालाक्यधिकरणे फलाधिकरणे पुरुषार्थाधिकरणे परामर्शाधिकरणे तद्भूताधिकरणे कार्याधिकरणे ब्राह्माधिकरणे अभावाधिकरणे च निर्दिष्टः । बादरायणस्य साक्षाच्छिष्यः 300 B.C कालात्पूर्वतन इति च सिद्धान्तः ।
१३. दत्तात्रेयः
पातिव्रत्यधर्मपरायणा अनसूया स्वपतिं अत्रिमुनिं स्वशिरसि वहन्ती निशीथे स्वाश्रमात् देशान्तरं जगाम । सूचिभेद्ये तमसि मध्येमार्गं गच्छन्ती सा शूलारोपितं माण्ढव्यमबुध्वा स्वपतिं माण्ढव्यशरीरे घर्षितवती । निष्कारणं पीडामुत्पादयन्तं कमित्यज्ञात्वा माण्ढव्यः ``सूर्योदयादतन्तरं पीडोत्पादकस्य मृतिर्भवतु" इति शशाप । पतिपरायणाऽनसूया ``सूर्योदय एव मा भूदिति" शशाप । अन्धकारावृते च जगति, सूर्ये च अनुदिते यज्ञक्रियादिकर्मलोपात् भीताः देवाः ब्रह्मणा प्रेरितास्सूर्योदयाय अनसूयां प्रार्थयामासुः । उदिते च सूर्ये अनसूयायाः पातिव्रत्यधर्मेण तुष्टेन इन्द्रेण प्रार्थितः भगवान् विष्णुरनसूयायां अत्रेः पुत्रत्वेनावततार । सोऽयं पुत्रः दत्तात्रेयः । दत्तात्रेयस्य प्रसादेन कार्तवीर्यार्जुतस्सचराचरं भूमण्डलं शश स । दत्तात्रेयः । दत्तात्रेयस्य प्रसादेन कार्तवीर्यार्जुतस्सचराचरं भूमण्डलं शश स । दत्तात्रेयः निमिनामकस्य पिता श्रीमतः पितामहश्चेति कथा महाभारते सभापर्वणि अनुशासनपर्वणि च प्रसिद्धा ।
(क) वेदान्तसारः -
क्वचित् क्वचित् अस्यैव अवधूतगीता इति नामान्तरमिति च दृश्यते । दत्तात्रेयकार्तिकेयंसवादरूपेऽस्मिन् ग्रन्थे प्रथमपरिच्छेदे अद्वैतब्रह्मवर्णना, द्वितीयादारभ्य सप्तमपरिच्छेदान्तं स्वात्मसंवित्युपदेशश्च दृश्यते । अमुद्रितोऽयं सम्पूर्णग्रन्थः सरस्वतीमहालये (7589 TSML) दृश्यते ।
(ख) अवधूतगीता -
दत्तगीता, जीवन्मुक्तिगीता, इत्यादिकं नाम अस्यैव ग्रन्थस्य दृश्यते । गोरक्षदत्तात्रेयसंवादरूपेऽस्मिन् ग्रन्थे जीवन्मुक्तस्य स्वरूपं सम्यगुपवर्णितम् । मुद्रितश्चायं ग्रन्थ अ ष्टेकरकम्पनि पूनानगरे गीतासङ्ग्रहे । अस्याः व्याख्या पर मानन्दगीर्थकृता अमुद्रिता वर्तते ।
(ग) प्रबोधचन्द्रिका -
ग्रन्थोऽयं मध्यप्रान्तीयबरार्ग्रन्थसूच्यां दृश्यते ।
(घ) स्वात्मसंवित्युपदेशः-
ग्रन्थोऽयममुद्रितः बरोडापुस्तकालये (996 BRD) लभ्यते । जीवन्मुक्तलक्षणमप्यस्य कृतिरिति ज्ञायते ।


  ``"



ऽ  ?
``" ``" ``" ``" ``" ``" ``" ``" ``" ``" ``"

`' `' `' `' `' `' `' `' `' `' 

ऽ  ।   ॥ ?
