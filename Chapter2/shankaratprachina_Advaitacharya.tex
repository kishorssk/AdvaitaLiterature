\chapter{शङ्करात्प्राचीना अद्वैताचार्याः}
द्वितीयो भागः
अद्वैताचार्याः, अद्वैतग्रन्थप्रणेतारश्च ।
अद्वैताचार्याः अद्वैतमतप्रतिपादकग्रन्थविशेषप्रणेतारश्च कालभेदेन द्वेधा विभक्तुं शक्यन्ते -``शङ्करभगवत्पादेभ्यः प्राचीनाः, शङ्करभगवत्पादेभ्य अर्वाचीनश्चेति ।" तेष्वपि शङ्करभगवत्पादेभ्यः प्राचीनेषु आचार्येषु १. ब्रह्मसूत्रकारात् शङ्करभगवत्पादेभ्यश्च प्राचीनाः, २. ब्रह्मसूत्रकारात् अर्वाचीनाः शङ्करभगवत्पादेभ्यः प्राचीनाश्चेति विभागोऽपि कर्तुं शक्यते । अद्वैतमतसिद्धान्तस्य प्राचीनतमत्वात् ब्रह्मसूत्रेषु बहूनां वृत्तिग्रन्थानां सत्वानुमानाच्च । शङ्करभगवत्पादकृतभाष्यप्रभावात् बौद्धप्रभावाच्च प्राचीनं वृत्यादिकं विनष्टम् । अथवा शाङ्करभाष्येण तेषां सम्पूर्णतया गतार्थत्वात् तेषां संरक्षणे औद सीन्यवशाच्च विनष्टं जातम् । तत्वचन्द्रिकाकारेणोमामहेश्वरेण तत्वचन्द्रिकायां (R 5156 MGOML) ``शङ्करभगवत्पादैः स्वीये भाष्ये एकोनशतं सृत्रवृत्तिग्रन्थाः परामृष्टाः, केचित् खण्डिताश्चेति" निर्दिश्यते । शाङ्करभाष्यदर्शकाणां सर्वेषाञ्च मनसि ईदृशी चिन्ता स्वाभाविकी यत् `शाङ्करभाष्ये शङ्कराचार्येण ततोऽपि प्राचीनवृत्तिकारस्य व्याख्यानं बहुषु स्थलेषु खण्डितम् । तदर्थं तत्र तत्र युक्तिरपि प्रदर्शिता । परन्तु तासां वृत्तीनां नाम, वृत्तिकारादीनां नाम वा न कुत्रापि निर्दिष्टम् । परन्तु व्याख्यानादिकर्तृभिः तेषां नाम तत्र तत्र निर्दिष्टं क्वचित् क्वचित् । शाङ्करभाष्ये पूर्वपक्षत्वेन गृहीता एव भास्कररामानुजादिभिः सिद्धान्तत्वेन स्वीकृता वर्तन्ते । तस्मात् शङ्करभाष्ये पूर्वपक्षत्वेनोपन्यस्तानां मतवादानां मध्ये कियन्तो मतवादाः पूर्वैरुद्भाविताः? कियन्तो वा बौधायनादिभिः प्रकल्पिताः? कति वा शङ्करभगवता स्वयं समुद्भाविताः? किं ते शाङ्करभाष्यानुगाः ? उत शङ्करसिद्धान्तेन अंशतस्सदृशाः ? इत्यादिविषया न निर्णेतुं शक्यन्ते । तथा च बोधायनोपवर्षादिभिस्सह भास्कररामानुजादीनां सम्बन्धो यथा दुर्निर्णेय स्तथा वृत्तिकारादीनामपीत्येवावगम्यते । एवमपि बोधायनोपवर्षभर्तृप्रपञ्चभर्तृहरि ब्रह्मनन्दिसुन्दरपाण्डयद्रविडाचार्यब्रह्मदत्तादिवेदान्ताचार्याः ब्रह्मसूत्रेषु वृत्तिग्रन्थान् प्राणैषुरिति परं ज्ञायते । व्याख्यानपरम्परैवात्र प्रमाणम् । एतेषां ग्रन्थानामिदानीमनुपलम्भात् ।'
एवं ब्रह्मसूत्रेवपि बादरायणव्यासेन केचनाचार्याः नाम्ना निर्दिष्टाः . तेषाञ्च ग्रन्था नोपलभ्यन्ते । तथापि तेषां मतवादादिकं विविधपत्रिकादिप्रमाणानुसारं यथाकथञ्चित विवृणीतुं प्रयतामहे । तेषु आचार्येषु ``अष्टावक्र - आत्रेय - आश्म रथ्य - उपवर्ष औहुलौमि - काशकृत्स्न-कार्ष्णाजिनि-जैमिनिबादर्यादयः व्यासात्प्राचीनाचार्यविभागे, आचार्यसुन्दरपाण्डूय द्रविडाचार्य ब्रह्मदत्त ब्रह्मनन्दि भर्तृप्रपञ्च भर्तृ रि बदिरायणापरव्यासाचार्याः शङ्करात्माचीनाचार्यविभागे, च विभक्तुं शक्यन्ते । एवमनिर्णीताः अज्ञातसमयाः विभिन्नविचारलक्ष्यीभूता अपरे अध्यात्मरामायणकारआदिशेषकाश्यपजाम्बवतदत्तात्रेययोगवासिष्ठ कारशु कसनत्सुजातादयश्च वर्तन्ते । तेषां सर्वेषां इतिहासादिकं प्रकरणेऽस्मिन् वर्णम लाक्रमेण निरूप्यते ।।"

१. अष्टावक्रः
अष्टावक्रोऽयं महर्षिः सुजाताकहोलयोः पुत्रः, उद्दालकस्य दौहित्रः, श्वेतकेतोस्स्वस्रीयः, वदान्यजामाता, सुपभाभर्तेति ज्ञायते । उद्दालकनामा महान् ऋषिंः वहोलाय स्वशिष्याय स्वकन्यां सुजातानाम्नीं वैवाहिकेन विधिना ददौ । शिप्यतध्ये अधीय नं स्वपितरं कहोलं सुजातागर्भस्थः अग्निकल्पशिशशुः रात्रिन्दिवं कहो रुस्य ध्ययनशीलतां च प्रति अधिचिक्षेप । क्रुद्धः मातामह उद्दालकस्स्वदौहित्रं ``यस्मात त्वं कुक्षौ वर्तमानोऽधिक्षिपसि तस्मात् त्वं अष्टकृत्वः वक्रो भवितासि" इति शशाप । तथा स वक्र एवाभ्यजायत अष्टावक्र इति प्रथितश्च । स्वपत्नीप्रेरणया वित्तार्जनार्थं जनरूपुरं गत अष्टावक्रपिता कहोलः जनकपुरद्वारपालेन बन्दिना वादे पराजितः जले मिमज्य मृतश्च । मातृपकाशात् पितृवृत्तान्तं श्रुत्वा मातुलेन श्वेतकेतुता सह अष्टावक्रः जनकपुरं गतः, वादेषु द्वारपालं जनकञ्च जित्वा ``समङ्गापरनाम्नीं मधुविलानदी स्नात्वा अष्टावक्रेदेहं समीकृत्य जरर्तरूपधारीण्या उत्त दिगभिपानिदेवतायाः धर्मो देशं स्वीकृत्य वदान्यकन्यां सुप्रभानाम्नी विधिनोपयेमे । इति कथा महाभारतादिप्रसिद्धा । अनेन कृतः ग्रन्थः-"
(क) अष्टावक्रगीता
अष्टावक्रसृक्तम्, अवधूतानुभूतिः, इत्यादिनाम्ना प्रसिद्धोऽयं ग्रन्थः ए विंशतिभिाध्यायैः पूर्ण अष्टवक्रजनकसंवादरूपेण अद्वैतवेदान्तसिद्धान्तान् ब्रह्मणोऽद्वितीयत्वं चिन्मयत्वञ्च प्रतिपादयति । ग्रन्थोऽयं आष्टेकरकम्पनि पूनानगरे मुद्रितश्च । अस्य व्य ख्याः विश्वेश्वरकृता - दीपिघख्या, पूर्णानन्दतीर्थकृता काचन व्याख्या, मुकुन्दमुनिकृता अन्या व्याख्या, भासुरानन्दकृता अपरा व्याख्या इत्येवं चतस्रः व्याख्या उपलभ्यन्ते ।
२. आचार्य सुन्दरपाण्ड्यः (600 A.D.)
शङ्करभगवत्पादेभ्यः प्राचीनोऽयं सुन्दरपाण्डयाचार्यः दक्षिणद्रविडदेशीयः, मधुरानगरवासीति ज्ञायते । पूर्वोत्तरमीमांसयोः प्रकाण्डपण्डितेनानेन ब्रह्मसूत्रणां वार्तिकं विरचितमिति, कैस्तवीय षष्ठशतकात् प्राचीनः, षष्ठशतकीयो वा इति निश्चीयते । अत्रेमानि कारणानि-
स्वर्गीयमहामहोपाध्याय कुप्पुस्वामिशास्त्रिणः ``जर्नल आफ ओरियण्टल रिसर्च मद्रास" पत्रिकायाः प्रथमे भागे (J.O.R.I.) एवमभिपयन्ति । शङ्करभगवत्पादैस्स्वोये ब्रह्मसूत्रमाष्ये सनन्वय - अधिकरणभाष्यान्ते अपिवाहुः-
``गौणमिथ्य त्मनोऽपत्वे पुत्रदेहादिबाधनत् ।
सद्ब्रह्म त्माहं इत्येव बोधकार्थं कथं भवेत् ।।
अन्वेष्टव्यात्मबिज्ञानात् प्राक् प्रमातृत्वमात्मनः ।
अन्विष्टस्स्यात् प्रमातैव पाप्मदोषादिवर्जितः ।।
देहात्मप्रत्ययो यद्वत् प्रप्ताणत्वेन कल्पितः ।
लौकिकं तद्वदेवेदं प्रनाणं त्वात्मनिश्चयात् ।।" इति 
श्लोकत्रयमुदाहृतम् । अत्र भामतीकारैः ``अत्रैव ब्रह्मविदां गाथामुदाहरति" इत्यवतारितम् । पञ्चपादिकाकारैः ``प्रसिद्धपेतत् ब्रह्मविदां पूर्वोक्तं न्याय संक्षेपतः श्लोकैस्संगृह्नणाति" इत्यवतारितम् । पञ्चपादिकाव्याख्यात्रा नरसिम्हस्वरूपशिप्येण आत्मस्वरूपेण स्वीयप्रबोवपरिशोधिन्यां `श्लोकत्रयं सुन्दरपाण्डयाचार्यप्रणीतं प्रमाणयति' इत्यवतारितम् ।
माधवनन्त्रिणा विरचितायां सूतसंहितयाः व्याख्यायां तात्पर्यदीपिकाख्यायां ``देहात्मप्रत्ययो यद्वत् प्रप्ताणत्वेन कल्पितः । लौकिकं तद्वदेवेदं प्रमाणन्त्वात्मनिश्चयात् ।" इत्ययं श्लोक उद्धृतः । एतच्छ्रलोकविवरणावसरे तथा सुन्दर पाण्ड्यवार्तिकमपीति अवतारिका कृता । 
एवं त्रयोदशशतकीयेनामलानन्देन स्वरविते कल्पतरुग्रन्थे (3 - 3 - 25 Page 755 N.S.P. Edn.) ``आह चात्र निदर्शनमाचार्यसुन्दरपाण्ड्यः" इति-
निःश्रेण्यारोहणप्राप्यं प्राप्तिमात्रोपपादि च ।
एकमेव फलं प्राप्तुं उभावारोहतो यदा ।।
एकसोपानवर्त्येको भूमिष्ठश्चापरस्तयोः ।
उभयोश्च जवस्तुल्यः प्रतिबन्धश्च नान्तरा ।।
विरोधिनोस्तदैको हि तत्फलं प्राप्नुयात्तयोः ।
प्रथमेन गृहीतेऽस्मिन् पश्चिमोऽवतरेन्मुधा ।। इति 
श्लोकत्रयमुपपादितम् । एतदेव श्लोकत्रयं कुमरिलभट्टैर्बलाबलाधिकरणे तन्त्रवार्तिके (BSS Page 852-853) आह चेत्यादिना प्रतिपादितम् । एवञ्च शङ्करभगवत्पादेभ्यः कुमरिलभट्टाच्च प्राचीन इति सिध्यति ।
आचार्यसुन्दरपाण्ड्यकृतः नीतिद्विषष्ठिकाख्यः ग्रन्थः कश्चन नीतिपरः मद्रपुर्यां प्रकाशितः । तत्रस्थाः बहवः श्लोकाः त्रयोदशशतकीयेन जल्हणेन सूक्तिमुक्तावल्यां, पञ्चदशशतकीयेन वल्लभदेवेन सुभाषितावल्यां, पञ्चदशशतकीयेन शार्ङ्गधरेण स्वीयशार्ङ्गधरपद्धत्यां पोतयार्येण प्रसङ्गरत्नावल्यां, पेद्दिभट्टेन सूक्तिवारिधौ, द्वादशशतकीयेन कलिङ्गराजापरनाम्ना सूर्यपण्डितेन कुलशेखरसूक्तिरत्नहाराख्ये ग्रन्थे चोदाहृताः । विशेषतः सूर्यपण्डितेन ``आचार्यसुन्दरपाण्ड्यकृता" इति निर्दिष्टाश्च । पञ्चतन्त्रकर्ता विष्णुशर्मा क्रैस्तवीयषष्ठशतकादर्वाचीन इति विमर्शकसिद्धान्तः । तेनापि नीतिद्विषष्ठिकास्थाः 29, 30, 48, श्लोकाः स्वीये ग्रन्थे उद्धृताः । एवं नीतिद्विषष्ठिकाग्रन्थावसाने कश्चन श्लोकः - ``इमां काञ्चनपीठस्थां समेत्य कवयो भुवि । आर्यां सुन्दरपाण्ड्यस्य स्नापयन्ति वधूमिव ।" दृश्यते । क्रैस्तवीय षष्ठशतकादारभ्य द्रविडेदेशे मधुरायां द्रविडसङ्घस्स्थापितः । अभ्यर्हिंत कविं तत्कृतिञ्च तत्सङ्घस्थाः विद्वांसः सङ्धपूजिते काञ्चनपीठे निवेश्य कनकाभिषेकमकुर्वन्निति तमिलसाहित्ये प्रसिद्धम् । तादृशस्तत्कारः ग्रन्थस्यास्यनीतिद्विषष्ठिकाख्यस्यापि प्रवृत्त इत्येवास्मात् पद्यात् ज्ञायते ।
``श्रीमच्छकाब्देऽब्धिशशिसायकसम्मिते । राजा माधववर्माभूत् विख्यातो धरणीतले ।।" इति प्रसङ्गरत्नावल्याख्ये पोतयार्यकृते ग्रन्थे दर्शनात्, पेण्डयाल सुब्रह्मण्यशास्त्रिभिः प्रकाशितात् पुलिवूरुशिलाशासनप्रमाणाच्च 514 शके 592 A.D. काले विष्णुकुण्डिनवंश्यः माधववर्मापरनामा जनाश्रयाख्यः कृष्णानदीतीरान्ध्रदेशाधीश आसीदिति ज्ञायते । तेन राज्ञा कृता कृतिः जानाश्रयीति च प्रसिद्धा । तस्मिन् जानाश्रयीत्यपरनामके छन्दोग्रन्थे नीतिद्विषष्ठिकायाः चतुर्विशतितमः ``चारित्रनिर्मल जलः सत्पुरुषनदोऽक्षयो भवतु नित्यम् । यस्य विभवारविन्दे विद्वद्भमराः कृतविनोदाः ।।" इति शलोक उद्धृतः ।
तस्मात्-कुमरिलभट्टेन, शङ्कराचार्येण, विष्णुशर्मणा, जनाश्रयेण, माधवमन्त्रिणा च प्रमाणीकृतोऽयं सुन्दरपाण्ड्याचार्य क्रैस्तवीयषष्ठशतकात् प्राचीनः, षष्ठशतकीयो वेति निश्चप्रचमभ्युपगम्यते ।
स्वर्गीय महामहोपाध्याय कुप्पुस्वामिशास्त्रिण आचार्यसुन्दरपाड्यमेनं क्रैस्तवीयाष्टमशतकीयं प्रवदन्ति । तमिलसाहित्ये प्रसिद्धः (अषकेसरी) एवायमिति जर्नल आफ ओरियण्टल पत्रिकायाः प्रथमे भागे J. O. R. I. निरूपयन्ति । इतिहासनिपुणाः K. A. नीलकण्ठशास्त्रिणस्तु जर्नल आफ ओरियण्टलरिसर्चपत्रिकायाः प्रथमे भागे (J. O. R. Madras-1) सप्तमशतकमध्यकाले सुन्दरपाण्ड्याचार्य आसीदिति प्रतिपादयन्ति । सर्वथापि शङ्करभगवत्पादेभ्यः प्राचीनोऽयमित्येवास्मत्सिद्धान्तः । ``श्रीमान् सुन्दरपाड्यः श्रुति स्मृति प्रसृत सत्पदार्थज्ञ" इति नीतिद्विषष्ठिकायां दर्शनात आचार्यसुन्दरपाण्ड्योऽयं पूर्वोत्तरमीमांसादिषु निष्णात इति ज्ञायते । शाङ्करभाष्ये उद्धृत्य प्रमाणीकृतत्वात् वेदान्तेऽनेन ब्रह्मसूत्राणां किमपि वार्तिकं पद्यबंद्ध कृतं स्यादिति निश्चीयते । कुमरिल भट्टैरुद्धृतोऽयं पूर्वमीमांसायामपि ग्रन्थप्रणेता इत्यभ्यूह्यते ।
अस्यैवाचार्यसुन्दरपाण्ड्यस्य द्रविडाचार्य इत्यपि नामान्तम्, व्यावहारिकनाम वा स्यादित्यूह्यते । अत्रेदं कारणं भवति-अभिनवद्रविडाचार्यापरनाम्ना अष्टादशशतकीयेन बालकृष्णानन्दसरस्वत्या विरचिते पद्यबद्धे शारीरकमीमांसाभाष्यवार्तिके आशुतोषग्रन्थमालामुद्रिते शङ्करभाष्यस्थस्य `अपिचे'ति ग्रन्थस्यावतरणसमये ``कथितार्थपरं द्रविडार्यकृतां अपिचेति गुरुर्वदतीह कथाम्" (ASI Page 403) निर्दिष्टम् । तस्मात् आचार्यसुन्दरपाण्ड्योऽयं द्रविडाचार्य इत्यपि व्यवहृतस्स्या दिति निश्चीयते ।
३. आत्रेयः
व्यासात्पूर्वतनेषु वेदान्ताचार्येषु अन्यतमोऽयमात्रेयः । बादरायणव्यासनिमिंतेषु ब्रह्मसूत्रेषु स्वामिनः फलश्रुतेरित्यात्रेयः 3-4-44 इति आत्रेयोऽयं निर्दिष्टः । यज्ञे अङ्गाश्रीतोपासना यज्ञस्वामिना एवं ऋत्विग्भिश्च कर्तव्या । अत्र फलविषये संशयः । किं उपासनाजन्यफलभाक् यजमानः ? उत ऋत्विक् ? इति । अत्रात्रेयमतन्तु अङ्गाश्रितोपासनाफलभाग्यजनमान एवेति । अदसीयः वेदान्तग्रन्थस्तु नोपलभ्यते ।
४. आदिशेषः (परमार्थसारकारः)
``वेदान्तशास्त्रमखिलं विलोक्य शेषस्तु जगदाधार" इति परमार्थमारे दृश्यते । साघवानन्दकृतायां व्याख्यायां ``भगवता जगदाधरेण आदिशेषेण" इति दृश्यते । तस्य त् सहस्रफणामणिमणिमण्डल आदिशेष एवास्य परमार्थसारस्य कर्तेति साम्प्रदायिकविश्व सः । परन्तु विधुशेखरभट्टाचार्यास्स्वपम्पादिवि ``गौडपदीयं आगमशास्त्र" मिति ग्रन्थे गौडपादाचार्यकालात् भास्कराचार्यकालस्य च मध्यवर्तिना केतापि आदिशेषनाम्ना ग्रन्थरचना कृतेति परमार्थसारग्रन्थकारकालः (500-800 A.D.) इति प्रवदन्ति ।
परमार्थसारः - (TSS 12)
आर्यावृत्तघटितैः पद्यैरद्वैतपरमार्थसारान् शिष्योपदेशशैल्यां प्रतिपादयन्नयं ग्रन्थः अभिनवगुप्ताचार्यकृतात् परमार्थसाराद्भिन्नः मुद्रितश्च चौखाम्बामुद्रणालये । अनन्तशयनग्रन्थावल्याञ्च सव्याख्योऽयं मुद्रितः । अस्य व्याख्याः- १ राघवानन्द मुनिकृता विवरणनाम्नी काचन २ वासुदेवयतिकृता अन्या प्रकाशि कानाम्नी व्याख्या अमुद्रिता (R. 4149 C. MGOML) लभ्यते ।
५. आश्मरथ्यः
ब्रह्मसूत्रकारात् बादरायणव्यासात् पूर्वतनोऽयं आश्मरथ्यः ब्रह्मसूत्रेषु वैश्वानराधिकरणे ``अभिव्यक्तेरित्याश्मरथ्यः 1-2-29, एवं वाक्यान्वयाधिकरणे प्रतिज्ञासिद्धेर्लिङ्गमित्याशमरथ्यः" 1-1-29 इति वारद्वयं निर्दिष्टः ।
उपनिषत्सु ईश्वरः प्रादेशमात्रः प्रतिपादितः । अस्य उपपतिरनेनैवं क्रियते ``परमेश्वर अनन्तः । भक्तानुग्रहार्थं प्रादेशमात्रादुद्भवति । हृदयादिषु उपलब्धियोग्येषु प्रदेशेषु उपलभ्यमानोऽयमिति प्रादेशमात्र इति च । भेदाभेदवाद्ययम् । कार्यावस्थायां विज्ञानात्मा परमात्मनः भिन्नः । कारणावस्थायान्तु अभिन्न इत्यस्य सिद्धान्त इति ज्ञायते ।"
६. उपवर्षाचार्यः (100 BC - 200 AD)
उपवर्षाचार्योऽयं पूर्वोत्तरमीमांसयेर्वृत्तिकारः, शङ्करभगवत्पादेभ्यः शबरस्वामिनोऽपि प्राक्तन इति ज्ञायते । शङ्करभगवत्पादैस्स्वीये सूत्रभाष्ये ऐकात्म्याधिकरणे 3-3-53 सूत्रे ``अत एव च भगवतोपवर्षेण प्रथमतन्त्रे आत्मास्तित्वाभिधानप्रसक्तौ शारीरके वक्ष्याम" इत्युद्धारः कृत इति सबहुमानमुपवर्षाचार्य आवेदितः । प्रकटार्थकारैरपि ``अतएवेत्य दि" भाष्यव्याख्यानावसरे `वृत्तिकारवचनं गमकमित्याह इत्येव शाङ्करभाष्यमवतारितम् । एवमानन्दमयाघिकरणे शङ्कराचार्यैः प्रथमं वृत्तिकारमतानुसारेण सूत्राणि व्याख्यातानि । अनन्तरं' `इदं त्विह वक्तव्यम्' इत्यादिना अधिकरणान्ते वृत्तिकारमतं पूर्वपक्षीकृत्य सिद्धान्तविधया स्वीयसिद्धन्तः प्रदर्शितः । एवमन्यत्रापि `अन्ये त्वाहुः' `अपरे त्वाहुः' इत्यादिना शङ्करभगवत्पादर्वत्तिकारमतमनूदितम् ।
शबरस्वामिना च ``वर्णा एव तु शब्दाः" इति भगवानुपवर्ष इति उपवर्षाचार्यः प्रमाणीकृतः । एवञ्च ब्रह्मसूत्राणां वृत्तिकार उपवर्षाचार्य अद्वैतमतैकदेशी प्र. चीन इति निर्णीयते । तादृशी उपवर्षीया वृत्तिस्तु नोपलभ्यते कुत्रापीदानीम् ।
ग्रन्थानुपलव्धेरेव विशिष्टाद्वैतिनः उपवर्षाचार्यस्यैव बोधायनकृतकोटिरित्यपि नामान्तरमिति वर्णयन्तस्स्वमतसंरक्षकबोधायनवृत्तिसत्यत्वसंरक्षणाय मणिमेखलादि द्रविडभाषाग्रन्थअवन्तिसुन्दरीग्रन्थप्रपञ्चहृदयग्रन्थमुखेन बोधायनकृतकोटिउपवषत्रयस्य अभिन्नतां साधयितुं प्रकटप्रयत्नमकुर्वन् । परन्तु तेषां प्रयत्नो विफल इति बोध यनकृतकोटिउपवर्षाचार्याः भिन्ना एवेति बोधायनवृत्तिस्तु नास्त्त्ये वेते ब्रह्मश्री पोलकं श्रीरामशास्त्रिभिरस्मद्गुरुचरणैः स्वीये द्रविडात्रेयदर्शने प्रतिपादितम् ।
आचार्य भगवद्दत्तैस्तु स्वीये ``भारतवर्ष का बृहद् इतिहास" नामके ग्रन्थे प्रथमभागे 84 पुटे शबरस्वामिनां कालः विक्रम चतुर्थशतकात् प्राचीन इति प्रतिपादितम् । तस्मादुपवर्षकालः 200 A.D. कालात् प्राक्तन इति तु निर्णीयते ।।
उपवर्षः वर्षोपाध्यायस्य कनिष्ठभ्राता पाणिनीयवृत्तिकृत् कात्यायनस्य श्वशुरः, उपकोशाया जनकः, महापद्भनन्दस्य प्रधानमन्त्रीति 500 A. D. काले आसीदिति च ``वृद्धत्रय्यां" गुरुपादशर्महालदारः ।
७. औहुलोमिः
व्यासात्पूर्वतनेषु आचार्येषु औडुलोमिरप्यन्यः । औडुलोमिरयं ब्रह्मसूत्रेषु वाक्यान्वयाधिकरणे ``उत्क्रमिष्यत एवं भावादौडुलौभिः" 1 - 4 - 21 इति, स्वाम्यधिकरणे ``आर्त्विज्यमौडुलोमिस्तस्मै हि परिक्रीयते" 3 - 4 - 45 इति, ब्राह्माधिकरणे ``चितिमात्रेण तदात्मकत्वादित्यौडुलोमि" 4 - 4 - 6 इति च स्थलत्रये निर्दिष्टः ।
संसार-मोक्षकालभेदेन जीवब्रह्मणोर्भेदाभेदवादी अयमौडुलोमिः । दृश्यप्रपञ्चेऽज्ञानवशात् जीवब्रह्मणोर्भेदः । मुक्तावस्थायान्तु उभयोरप्यभेद इत्यस्य मतं स्यादित्युह्यते । भामतीकारोऽपि मतमेतदीयं प्रतिपादयति । 
८. काशकृत्स्नः
अविकृतः परमेश्वरो जीवः, नान्य इति सिद्धान्तवाद्ययं काशकृत्स्नः ब्रह्मसूत्रेषु वाक्यान्वयाधिकरणे ``अवस्थितेरिति काशकृत्स्नः" 1 - 4 - 22 निर्दिष्टः ।।
९. काश्यपः
नायं ब्रह्मसूत्रकारैर्निर्दिष्टः । परन्तु ``शाण्डिल्यभक्तिसूत्रे तामैश्वर्यपरां काश्यपः परत्वात्" No. 29 इति निर्दिष्टः ।
१०. काष्णाजिनिः
छान्देग्योपनिषदां पञ्चमाध्याये श्रूयमाणस्य ``रमणीयचरणा" इति ग्रन्थस्य व्याख्यानावसरे कार्ष्णाजिनिमतं ब्रह्मसूत्रे निर्दिष्टम् । कृतात्ययाधिकरणे ``चरणादिति चेन्नोपलक्षणार्थेति कार्ष्णाजिनि" 3 - 1 - 9 सूत्रेण निर्दिष्टः ।।
११. जाम्बवान्
एतत्कृतत्वेन प्रसिद्धस्य प्रणवमहाभाष्याख्यग्रन्थस्य उपान्त्यवाक्यात् रामचन्द्रभक्तो भगवान् जाम्बवानेवायमिति प्रतीयते । यद्येवं तर्हि प्रजापति पुत्रोऽयं त्रेतायुगादारभ्य वर्तमानश्चिरञ्जीवी जाम्बवतीपिता भगवतः कृष्णस्य श्वशुरश्चेति निर्णेतुं शक्यते ।
प्रणवमहाभाष्यम् -
प्रणवार्थप्रकाशकोऽयं ग्रन्थः माण्डूत्योपनिषदन्तर्गतं ओङ्कारोपासनार्थ प्रदर्शयति । आह्निकत्रयपूर्ण अमुद्रितोऽयं ग्रन्थःतिरुवनन्तपुरपुस्तकालये 306 TCD दृश्यते ।।
१२.जैमिनिः
जैमिनिरयं ब्रह्मसूत्रेषु वैश्वानराधिकरणे देवताधिकरणे बालाक्यधिकरणे फलाधिकरणे पुरुषार्थाधिकरणे परामर्शाधिकरणे तद्भूताधिकरणे कार्याधिकरणे ब्राह्माधिकरणे अभावाधिकरणे च निर्दिष्टः । बादरायणस्य साक्षाच्छिष्यः 300 B.C कालात्पूर्वतन इति च सिद्धान्तः ।
१३. दत्तात्रेयः
पातिव्रत्यधर्मपरायणा अनसूया स्वपतिं अत्रिमुनिं स्वशिरसि वहन्ती निशीथे स्वाश्रमात् देशान्तरं जगाम । सूचिभेद्ये तमसि मध्येमार्गं गच्छन्ती सा शूलारोपितं माण्ढव्यमबुध्वा स्वपतिं माण्ढव्यशरीरे घर्षितवती । निष्कारणं पीडामुत्पादयन्तं कमित्यज्ञात्वा माण्ढव्यः ``सूर्योदयादतन्तरं पीडोत्पादकस्य मृतिर्भवतु" इति शशाप । पतिपरायणाऽनसूया ``सूर्योदय एव मा भूदिति" शशाप । अन्धकारावृते च जगति, सूर्ये च अनुदिते यज्ञक्रियादिकर्मलोपात् भीताः देवाः ब्रह्मणा प्रेरितास्सूर्योदयाय अनसूयां प्रार्थयामासुः । उदिते च सूर्ये अनसूयायाः पातिव्रत्यधर्मेण तुष्टेन इन्द्रेण प्रार्थितः भगवान् विष्णुरनसूयायां अत्रेः पुत्रत्वेनावततार । सोऽयं पुत्रः दत्तात्रेयः । दत्तात्रेयस्य प्रसादेन कार्तवीर्यार्जुतस्सचराचरं भूमण्डलं शश स । दत्तात्रेयः । दत्तात्रेयस्य प्रसादेन कार्तवीर्यार्जुतस्सचराचरं भूमण्डलं शश स । दत्तात्रेयः निमिनामकस्य पिता श्रीमतः पितामहश्चेति कथा महाभारते सभापर्वणि अनुशासनपर्वणि च प्रसिद्धा ।
(क) वेदान्तसारः -
क्वचित् क्वचित् अस्यैव अवधूतगीता इति नामान्तरमिति च दृश्यते । दत्तात्रेयकार्तिकेयंसवादरूपेऽस्मिन् ग्रन्थे प्रथमपरिच्छेदे अद्वैतब्रह्मवर्णना, द्वितीयादारभ्य सप्तमपरिच्छेदान्तं स्वात्मसंवित्युपदेशश्च दृश्यते । अमुद्रितोऽयं सम्पूर्णग्रन्थः सरस्वतीमहालये (7589 TSML) दृश्यते ।
(ख) अवधूतगीता -
दत्तगीता, जीवन्मुक्तिगीता, इत्यादिकं नाम अस्यैव ग्रन्थस्य दृश्यते । गोरक्षदत्तात्रेयसंवादरूपेऽस्मिन् ग्रन्थे जीवन्मुक्तस्य स्वरूपं सम्यगुपवर्णितम् । मुद्रितश्चायं ग्रन्थ अ ष्टेकरकम्पनि पूनानगरे गीतासङ्ग्रहे । अस्याः व्याख्या पर मानन्दगीर्थकृता अमुद्रिता वर्तते ।
(ग) प्रबोधचन्द्रिका -
ग्रन्थोऽयं मध्यप्रान्तीयबरार्ग्रन्थसूच्यां दृश्यते ।
(घ) स्वात्मसंवित्युपदेशः-
ग्रन्थोऽयममुद्रितः बरोडापुस्तकालये (996 BRD) लभ्यते । जीवन्मुक्तलक्षणमप्यस्य कृतिरिति ज्ञायते ।
१४. द्रविडाचार्याः
अद्वैतसम्प्रदायप्रवर्तकेषु पूर्वाचार्येषु अन्यतमा एते द्रविडाचार्या एतद्युगारम्भ एव सम्भूताः । एते भाष्यकारा इति प्रसिद्धाः । छान्दोग्योपनिषदां अर्थविवरणात्मकं सूत्ररूपवाक्यनिचयपण्डितं वाक्यनामकं ग्रन्थं ब्रह्मतन्दिनः प्राणैषु । द्रविडाचार्यैः स्वव्याख्येयवाक्यग्रन्थानुपारं सविशेषनिर्विशेषभेदेन द्विरूपं ब्रह्म न्यरूपि ।
बृहदाण्यकोपनिषद्भाष्ये द्वितीयाध्यायप्रथमब्राह्मणविवरणे शङ्कराचार्यैः द्रविडाचार्याः प्रमाणीकृताः । ``अत्र हि सम्प्रदायविदः आख्यायिकां सम्प्रचक्षते" इति । आनन्दगिरिणापि ``तत्वमस्यादिवाक्यमैक्यपरं, तच्छेषस्सूष्टय दिवाक्यम्" इत्युक्तेऽर्थे द्रविाडाचार्यसम्मतिमाह - ``अत्र चेति" इत्यवतारिका प्रदीयते । बृ दाण्यकवार्तिकेऽपि - ``आचक्षते तथाचात्र केचिदाख्यायिकां शुभाम् । यथाभिलषितार्थोऽयं यथा सम्भाव्यते स्फुटः ।" इति । अत्राप्यानन्दगिरिणा ``द्रविडाचार्य प्रणीतामाख्य यि कामवतारयति" इत्यवतारिता । एवं तोटकाचार्यः श्रुतिसारसमुद्धरणे विषयेऽस्मिन्नेव द्रविडाचार्यान् निर्दिशति -``द्रविडोऽपि च तत्वमसीति वचो विनिवर्तकमेव निरूपितवान् । शबरेण विवर्धितराजशिशोर्निजजन्मविदुक्तिनिदर्शनतः।" इति । एवं छान्दोग्योपनिषदश्शाङ्करभाष्योपक्रमे एवं दृश्यते - ``ओमित्येतदक्षरमित्यष्टाध्यायी छान्दोग्योपनिषत् । तस्याः संक्षेपत अर्थजिज्ञासुभ्यः ऋजुविवरणं अल्पग्रन्थमिदं आरभ्यते । इति । आनन्दगिरिणा " ``अथ पाठक्रममाश्रित्यापि द्राविडं भाष्यं प्रणीतम् , तत्किमनेन इत्याकाङ्क्षायामाह अल्पग्रन्थ" मिति अवतरणिका प्रदत्ता । मधुसूदनसरस्वत्या कृतायां संक्षेपशारीरकटीकायां ब्रह्मप्तन्दिविरचितवाक्यानां सूत्ररूपाणां भाष्यकर्ता द्रविडाचार्य इति निर्दिश्यते । नृसिम्हाश्रमिकृतायां संक्षेपशारीरकटीकायां ``भाष्यकृदद्रविडाचार्यवचनात्" इति निर्दिश्यते । रामतीर्थेनापि नन्दिकृतग्रन्थभाष्यकारः द्रविडाचार्य इति निर्दिश्यते । छान्दोग्यभाष्ये तृतीयाध्याये मधुविद्य विवरणे शङ्कराचार्यैः ``अत्रोक्तः परिहार आचार्यैः" इति आचार्यशब्देन द्रविडाचार्यः निर्दिष्टः । आनन्दगिरिणाऽपि द्रविडाचार्योक्तं उपपादयतीत्यवतारिका दीयते । सूत्रभाष्ये ज्योतिश्चरणाधिकरणे ``व्याचक्षत" इति शब्देन द्रविडाचार्याः निर्दिश्यन्ते । भामत्यां समन्वयाधिकरणे ``यथाहुर्दविडाचार्या इति द्रविडाचार्याः" प्रमाणीकृताः । सूत्रभाष्ये समन्वयाधिकरणभाष्यान्ते ``गौणमिथ्यात्मनोऽसत्वे पुत्रदेहादिबाधनात्" इत्यादि श्लोकत्रयमुद्धृत्य प्रप्ताणीकृतम् । बालकृष्णानन्दसरस्वत्या प्रकाशिते शारीरकमीमांसाभाष्यवार्तिके पूर्वोक्तभाष्यस्यावतारिकाप्रदानसप्तये ``कथितार्थपरां द्रविडार्यकृतां अपि चेति गुरुवर्दतीहकथाम् ।।" इति निर्दिश्यते । गौणमिथ्यात्मन इत्यादि पद्यं द्रविडाचार्यकृतमिति निर्दिष्टम् । म. ग. कुप्पुस्वामि शास्त्रिभिश्च प्राच्यभाषासंशोघनपत्रिकायाः प्रथमे भागे J. O. R. Vol I Madras पूर्वोक्तपद्यत्रयं आचार्यसुन्दरपाण्ड्यकृतमिति निरूपितम् । द्रविडाचार्यस्यैव सुन्दरपाण्ड्य इति स्यान्नाम् ।
एवञ्चाद्वैताचार्यैः निर्दिष्टाः द्रविडाचार्याः नूतनसम्प्रदायप्रवर्तकत्वेन प्रसिद्धाः गौडपादसामयिकाः तत्समानमेव पूज्याः गौडपादाचार्या इव इमेऽपि सन्यासपरम्परायाः प्रवर्तका आसन् । स च सम्प्रदायश्शङ्कराचार्यैः स्वान्तेवासिषु केषुचन उरीकृत एव परन्तु भगवत्पादादिक्रमेण नोरीकृतः । किन्तु सम्प्रदायान्तरद्वारा । अत एव गौडपादानारभ्य सङ्कलितासु स्वाचार्यपरम्पारासु द्रविडाचार्यः न निर्दिष्टः इति प्रतिभाति । एतैः । बृहदारण्यकवाक्यभाष्यं 2 छान्दोग्यभाष्वमपि कृतं स्यात् ।।
१५.बादरिः
ब्रह्मसूत्रेषु वैश्वानराधिकरणे, कृतात्ययाधिकरणे कार्याधिकरणे अभावाधिकरणे च निर्दिष्टोऽय बादरिः । वैदिककर्मणि सर्वेषामधिकारं प्रादेशमात्रे दृदये वर्तमानत्वात् प्रादेशमात्र ईश्वर इति सिद्धान्तं, ``रमणीयचरणा" इति छान्दोग्यवाक्यस्थस्य चरणशब्दस्य कर्मपरत्वं, छान्देग्यस्य 4-15-5 य एनान् ब्रह्म गमयतीति बाक्यस्थब्रह्मशब्दस्य कार्यब्रह्मवाचकत्वं ईश्वरभावापन्नस्य विदुषश्शरीरेन्द्रियमनसां असत्वञ्च वदन् अयं बादरिर्व्यासात्प्राचीनेषु प्रसिद्धाचार्येषु अन्यतमः ।।
१६. ब्रह्मदत्तः
जीवाः ब्रह्मण उत्पद्यन्ते मोक्षपर्यन्तावस्थायिनश्चेति ब्रह्मदत्तः । वेदान्तेषु अस्य मतं औपनिषदाभासशब्देन व्यवहारार्हं भवति । ज्ञानकाण्डब्रह्मकाण्डयोर्मुख्यं फलभेकमेवेति ब्रह्मदत्तसिद्धान्तः । ``अहं ब्रह्मास्मि" इति वाक्यार्थज्ञानमेव मोक्षोपयोगीति ब्रह्मदत्तः । ब्रह्मदत्तपते उपनिषदः ध्याननियोगप्रधानाः । अद्वेतिनां मते मोक्षो दृष्टफलः । ब्रह्मदत्तमते मोक्ष अदृष्टफलः । ब्रह्मदत्तमते `तत्वमसि' वाक्यात् ``आत्मा वारे द्रष्टव्य" इति वाक्यमेवोपादेयार्हम् ।
ज्ञानकर्मसमुच्चयवादी अयं ब्रह्मदत्तः ब्रह्मसूत्राणां ज्ञानकर्मसमुच्चयपरां व्याख्यां कृतवान् स्यात् । ब्रह्मदत्तश्च यामुनाचार्येण सिद्धित्रयेऽनुपादेयत्वेन निर्दिष्टः । सुरेश्वराचार्यकृतनैष्कर्म्यसिद्धेः ज्ञानामृतकृतायां विद्यासुरभिनाम्न्यां व्याख्यायां मद्रासराजकीयहस्तलिखित पुस्तकालयस्थायां (R. 3354 MGOML) ब्रह्मदत्तः निर्दिष्टः । आनन्दगिरिणापि बृहदारण्यकसम्बन्धवार्तिकव्याख्यानावसरे (P. 220 ASS 16) ब्रह्मदत्तः निर्दिष्टः 
किमयं ब्रह्मदत्तः ज्ञानकर्मसमुच्चयवादी ? उत न ? इति हिरियण्णामहाशयैः (J. O. R. Vol. 12 Madras) पत्रिकायां विमृष्टम् ।।
१७. ब्रह्मनन्दी
एते ब्रह्मनन्द्याचार्या अद्वैताचार्येषु प्राचीनेषु अन्यतमाश्श्ङ्करात् पूर्वमासन्निति ज्ञायते । अदसीयाः ग्रन्था नोपलभ्यन्ते । तथापि अद्वैताचार्यैशशङ्करादर्वाक्तनैः प्रमाणीकृताः । 
भामतीव्याख्याने कल्पतरौ ``इयञ्चोगदानपरिणामादिभावा न विकारोभिप्रायेण" इत्यादि भामतीग्रन्थस्य व्याख्यानावसरे (Page 421 VVS Edn) ब्रह्मनन्दी निर्दिष्टः । संत्रक्षपशारीरके तृतीयपरिच्छेदे (श्लोक संख्या 217-221) ``आत्रेय वाक्यमपि संव्यवहारमात्रम्" इत्यादिना आत्रेय अत्रिगोत्रजः ब्रह्मनन्दी अनूदितः । मधुसूदनसरस्वत्या स्वटीकायां ``छान्दोग्यवाक्यकारेण ब्रह्यनन्दिना" इत्यवतारितम् । ``सिद्धन्तु निवर्तकत्वात्" इति वाक्यं शङ्कराचार्येः माण्डूक्योपनिषदां भाष्ये वैतथ्यप्रकरणे ``न निरोधो न चोत्पत्तिः" इति कारिकाविवरणावसरे निर्दिष्टम् । इष्टसिद्धौ (Page 72) ``सिद्धन्तु निवर्तकत्वादिति चोक्तं वाक्यं ज्ञानोत्तमकृतायां इष्टसिद्धिटीकायां ब्रह्मनन्दीयमिति निर्दिष्टम् । आनन्दगिरिणा योगवासिष्ठव्याख्यात्रा आनन्दबोधेन च द्रविडशब्देन ब्रह्मनन्दी निर्दिश्यते । पञ्चपादिकाविवरणेऽष्टमवर्णके ``सिद्धन्तुनिवर्तकत्वात्" इति वाक्यं प्रमाणत्वेन स्वीकृतम् । खण्डनखण्डखाद्यव्याख्यायां विद्यासागर्यां सिद्धन्तु निवर्तकत्वादितिवाक्यं उदाहृतम् । नृसिम्हाश्रमिकृतायां संक्षेपपशारीरकटीकायां ``ब्रह्मतन्दिनापिछान्दोग्यषष्ठव्याख्यानावसरे उक्तम्" इति निर्दिष्टम् । रामतीर्थकृतायां संक्षेपशारीरकटीकायां ``ब्रह्मनन्दिनाप्याचार्येण छान्दोग्य भाष्ये उक्तम्" नन्दिकृत भाष्यकारः द्रविडाचार्य इति च निर्दिश्यते । यामुुुनाचार्येण ``आत्मसिद्धौ" ``आचार्यटङ्क - भर्तृप्रपञ्च - भर्तृमित्र भर्तृहरि ब्रह्मदत्त - शङ्कर-श्रीवत्साङ्क - भास्करादि विरचितसितासितविविधनिबन्धनश्रद्धाविप्रलब्धबुद्धयः न यथावत् अन्यथा च प्रति पद्यन्ते ।" इति टङ्क अपरिग्र ह्यत्वेन निर्दिष्टः । टङ्क एव ब्रह्मनन्दीति श्रुतप्रकाशिकाचार्येण वेङ्कटनाथेन तात्पर्यदीपिकायां ``अत्र भाष्यकारः ब्रह्मनन्दिवाक्यव्याक्याता द्रविडाचार्यः" इति द्रविडाचार्यव्याख्येयग्रन्थकर्त टङ्काख्यः ब्रह्मनन्दीत्युक्तम् ।।
तस्मात् ब्रह्यनन्द्यभिन्नः टङ्काख्योऽयं अपरिग्राह्यत्वेन यामुनाचार्येण निर्दिष्टः विविर्तवादावलम्बी नाद्वैतमतविरोधीति स्पष्टं प्रतीयते । अनेन निर्मितः छान्देग्यवाक्यनामा ग्रन्थस्तु न कुत्रापि लभ्यते ।।
१८. भर्तृप्रपञ्चः
भर्तृप्रपञ्चोऽयं वेदान्तसाहित्ये शनैश्शनैर्म्लानयशास्सञ्जातः . शङ्कराचार्यसिद्धान्तात् भर्तृप्रपञ्चसिद्धान्तःभिद्यते । भर्तृप्रपञ्चः भेदाभेदवादी । शङ्कर अभेदवादी । भर्तृप्रपञ्चः ज्ञानकर्मसमुच्चयवादी । शङ्कराचार्यः ज्ञानवादी । मतस्य दर्शनग्रन्थानाञ्च जीवेश्वराणांं आत्मनश्च प्रतिपादने एव तात्पर्यमिति भर्तृप्रपञ्चः । भर्तृप्तपञ्चसिद्धान्तः प्रमाणसमुच्चयताम्नापि व्यवहर्तुं शक्यते । भर्तृप्रपञ्चः भोग एव मोक्षहेतुः न वैराग्यम् , वस्तुतत्वानुभव एव विरक्तेस्मुगमः पन्था इति चाभिप्रैति । ``तत्वमसि" वाक्यात् ``आत्मानमेवलोकमुपासीत" इति वाक्यमेव भर्तृप्रपञ्चमते उपादेयतरमिति ज्ञायते ।
ज्ञानकर्मसमुच्चयवादी भर्तृप्रपञ्चोऽय द्वैताद्वैतवादीति च निश्चयः । शङ्करभगवत्पादैरयं बृहदारण्यकभाष्ये ``औपनिषदम्मन्य" इति नाम्नानूद्य खण्डितः । आनन्दगिरिणापि भाष्यव्याख्यायां तत्र तत्र निर्दिष्टः । तस्मात् नायं विशुद्धाद्वैतवादी परन्तु अद्वैतैकदेशीति परं वक्तुं अर्हः । अनेन कठोषनिषदां बृहदाण्यकस्य च भाष्यमारचितं स्यादिति ज्ञायते ।
एनमधिकृत्य ``इण्डियन् आण्डिक्वैरिपत्रिकायां (I. A. Part 53 Page 77, 1924) हिरियण्णामहाशयेन सविस्तरं प्रतिपादितम् ।"
१९. भर्तृहरिः
वेदविदामलङ्कार इति प्रसिद्धोऽयं भर्तृहरिः ``नचागमादृते धर्मस्तर्केण व्यवतिष्ठते" इति वाक्यपदीयब्रह्मकाण्डे वदन् स्वस्य वैदिकधर्मावलम्बित्वं प्रकटयति । वाक्यपदीयकर्ता भर्तृः रिः वसुरातशिष्य इति ज्ञायते । ``ईस्टिङ् भारत यात्रा" ग्रन्थानुसारं भर्तृहरिसमयस्सप्तमशतकापरार्धावधिक (600-700 A. D.) इति निर्णीयते । भारतपण्डितमण्डलीप्रसिद्धा तु कथा- ``भर्तृहरिः विक्रमादित्यभ्राता" इति । तन्त्रवार्तिके कुमरिलभट्टः वाक्यपदीयं वाक्यं खण्डयति (1 - 3 - 871) काशिकायां (4 - 3 - 88) वाक्यपदीयः निर्दिष्टः । तस्मात् ताभ्यां पूर्वतन इति न संशयः । कुन्हन राजामहाशयस्तु (I. H. Q. Vol. XIV) भर्तृहरि पञ्चमशतकीयं वदन्ति । म. म. कुप्पुस्वामि शास्त्रिणः ब्रह्मसिद्धि भूमिकायां षष्ठशतकापरार्धादारब्धे सप्तमशतकापरार्धावधिके काले भर्तृहरिरासीदिति प्रवदन्ति । शबर स्वामिनोऽपि प्राचीनोऽयमिति भगवद्दत्तजीकृत वैदिकवाङ्मयेतिहासे दृश्यते ।
यामुनाचार्येण सिद्धित्रये भर्तृहरिकृतस्य सूत्रव्याख्यानस्य अनुपादेयत्व प्रदर्शनात् भर्तृहरिणापि ब्रह्मसूत्रवृत्तिः कृताह इति ज्ञायते । केचित्तु शब्दाद्वैत वादिनमेनं वदन्ति । वाक्यपदीयब्रह्मकाण्डमेव भर्तृहरिणः वेदान्तित्वे प्रमाणम् ।
२०. वाल्मीकिः (योगवासिष्ठकारः)
अध्यात्मविद्यायाः अद्वैतवेदान्तसिद्धान्तस्य च प्राचीनतमोऽयं ग्रन्थः योगवासिष्ठमिति मतिरस्माकम् । रामतीर्थस्वामिनः ग्रन्थमेनं भूमण्डलान्तर्गतेषु ग्रन्थेषु अत्युत्तमं ब्रह्मसाक्षात्कारकरञ्चेति वदन्ति । प्रस्थानत्रथी साधनावस्थोपयोगिनी । योगवासिष्ठन्तु सिद्धावस्थायामपि पठनार्हं ग्रन्थरत्नम् । ग्रन्थश्चायं अनेकदृष्टान्तोपाख्यानादिभिर्युक्तिभिश्च अद्वैतसिद्धान्तं प्रतिपादयति ।
ग्रन्थस्यास्याद्भुतस्य रचनाविषयेऽस्ति महान् विदुषां मतभेदः । चित्तशुद्धि समुत्पादनाय पूर्वरामायणम्, सञ्जातचित्तशुद्धेः पुरुषस्य जिज्ञासाशान्त्यै आत्मानात्मविवेचनपरं अद्वैतसिद्धान्तकोशभूतमिंद उत्तररामायणापराभिघं योगवासिंष्ठ वाल्मीकिना प्रणीतमिति तु साम्प्रदायिकी पण्डितमण्डितवार्ता । ``ऋषिभिर्बहुधागीत" मिति गीताया (XIII. 3.) ऋषिभिरित्यस्य व्याख्यानावसरे ``वसिष्ठादिभिरिति" शाङ्करं भाष्यम् इति च प्रमाणं प्रवदन्ति । यदि वाल्मीकिरेवास्य कर्ता स्यात् तर्हि वाल्मीकिस्सुकन्याच्यवनयोः पुत्र इति पौराणिकी प्रसिद्धिरिति कालादिनिर्णयो न कर्तुं शक्यते ।
आधुनिकेषु प्राच्यप्रतीच्यभाषाप्रवीणेषु विमर्शकवरेषु च अस्य रचनाकाल विषये महान् मतभेद आशयभेदश्चवरीवर्ति । डाक्टर फर्कुहार प्रभृतय आधुनिकाः द्वादशत्रयोदशशतकमध्यमस्य रचनाकाल इति मन्वते । शिवप्रसाद भट्टाचार्यास्तु (900-1110 A. D.) दशमैकादशशतकमध्यमस्य रचनाकाल इति वर्णयन्ति । भारतीय साहित्येतिहासलेखकानां जर्मन पण्डितानां विण्टर्निट महाशयानां नवमशतकमिति । दिवानजी महाशयास्तु योगवासिष्ठरचनास्थानं काष्मीरदेशः, योगवासिष्ठरचनाकालः दशमशतकमिति निश्चिन्वन्ति । आत्रेयपहाशयास्तु कालिदामात् अर्वावीने भर्तृहरिगौडपादशङ्करसुरेश्वरप्रभृतिभ्य अद्वैतवेदान्ताचायभ्यः प्राचीने न काले योगवासिष्ठं प्रणीतमिति सिद्धान्तयति । डाक्टर. वे. राघवमहोदयाश्च गीतायाः योगवासिष्ठस्य च साम्यप्रतिपादकानि द्विनवतिंसख्याकानि उद्धरणानि प्रतिपाद्य राजशेखरात् अनन्तरभाविनि नवमशतकादारब्धे त्रयोदशशतकान्ते च काले योगवासिष्ठं प्रणीतमिति जर्नल आफ ओरियण्टल पत्रिकायां प्रतिपादयन्ति । दासगुप्तमहाशयस्तु नवमशतकीयेन काष्मीरिणा अभिनन्देन लघुयोगवासिष्ठनामा ग्रन्थः प्रणीत इति तत्कालात्पूर्वतनोऽयं ग्रन्थ इति (HIP Vol II) प्रतिपादयति ।
शङ्कराचार्यकृतविवेकचूडामणौ, विद्यारण्यस्वामिकृतपञ्चदश्यां, जीवन्मुक्तिविवेके च, भर्तृहरिकृतवाक्यपदीयवैराग्यशतकयोः, प्रकाशानन्दानां वेदान्त सिद्धान्तमुक्तावल्यां च योगवासिष्टीयश्लोकाः दृश्यन्ते । उपनिषत्स्वपि योगवासिष्ठीयश्लोकाः दृश्यन्ते । कतिपयोपनिषदः योगवासिष्ठश्लोकसंग्रहा एवेति च आत्रेयमहाशयेन सविस्तरमुपपादितम् ।
योगवासिष्ठग्रन्थे न कोऽपि ग्रन्थः ग्रन्थकारो वा उद्धृतः प्रमाणत्वेन निर्दिष्टश्च । द्वित्रिस्थलेषु परं `बुद्धः' जिनः इत्यादिशब्दाः (Page 33, 669, 729, Vol I NSP Edn) दृश्यन्ते । परन्तु तत्रापि व्याख्यात्रानन्दबोधेन ``प्रव्रजितः" इत्येवार्थः क्रियते । प्रथमभागे 454 तमे पुटे ``यत्प्राप्तं शङ्करादिभिः" इत्यादिना शङ्करः निर्दिष्टः । अत्र व्याख्यात्रा न व्याख्यातम् । कोऽयं शङ्करः ? किं शङ्कर भगवत्पादः ? उत भगवान् भवानीपतिः ? प्रकरणवशात्सु शङ्करभगवत्पाद इत्येवास्माकं प्रतीयते । एवस्मिन्नेव स्थले 714 तमे पुटे प्रथमभागे ``बृहदारण्यकादिषु" इति बृहदारण्यकोपनिषत् नाम्ना निर्दिश्यते । प्रथमभागे 594 तमे पुटे ``श्रीशैलाचार्यपुत्रेण" इति श्रीशैलाचार्यः निर्दिष्टः । एवमादिप्रमाणैश्शङ्करादनन्तरभावित्वमस्य ग्रन्थस्य वक्तुं शक्यते परन्तु प्रक्षिप्ता इमे श्लोका इति वज्रकुठारप्रक्षेपभीत्या न तथापि वक्तुं शक्यते ।
``अयं प्रपञ्चो मिथ्यैव सत्यं ब्रह्माहमद्वयम् । अत्र प्रनाणं वेदान्ता गुरवोऽनु भवस्तथा ।" (Page 181 Vol I) इत्यादिभिरसंख्यैः पद्यैः जगन्मिथ्यात्वं जीवब्रह्मैक्यं, ब्रह्मणो नामरूपबहिर्भूतत्वं, जीवन्मुक्ति, अजातवादः, अनिर्वचनीयतावादः, इत्यादय अद्वैतसिद्धान्ताः काव्यशैल्यां प्रतिपादिता इति ग्रन्थोऽयं अद्वैतवेदान्तसाहित्यास्यादिमं महाकाव्यामित्येव मदीयस्मिद्धान्तः । एतादृशे अद्वैतवेदान्तशास्त्रकाव्ये आध्यात्मिकमहाकाव्यापरनामके अद्वैतसिद्धान्तवर्णना, जगन्मिथ्यात्ववर्णना च एतादृशी वर्तते यया प्रभाविताः केचन विद्वासः योगवसीष्ठे बौद्धमतस्य सर्वश्न्यवादस्य प्रभावं वर्णयन्ति । परन्तु नैतत्सत्यम् । योगवासिष्ठस्य रचयिता न साधकः । परन्तु अद्वैतानान्दानुभवी महान् सिद्धः । तादृसस्य सिद्धस्य स्वीयानुभवैकप्रमाणे ग्रन्थेऽस्मिन् अद्वैतसिद्धान्ताः सिद्धावस्थानुकूला एव प्रतिपादिता इति तु निश्चयः । ग्रन्थोऽयं तात्पर्यप्रकाशव्याख्यासहितः निर्णयसागरमुद्रणालये मुद्रितः । अस्य व्याख्याः एतत्सम्बद्धाश्च ग्रन्थाः-
(क) अद्वयाख्याकृता - योगवासिष्ठपददीपिका । ग्रन्थोऽयं कल्कत्ता रायल आसियाटिका सूच्यां दृश्यते ।
(ख) आनन्दबोधकृतः - तात्पर्यप्रकाशः । ग्रन्थोऽयं निर्णयसागरमुद्रणालये मुद्रितः ।
(ग) अभिनन्दकृतः - योगवासिष्ठसंक्षेपः (लघुयोगवासिष्ठम्) ग्रन्थोऽयं आत्मसुखेन वासिष्ठचन्द्रिकाव्याख्यया व्याख्यातः । मुम्मुडिदेवेन संसारतरणि व्याख्यया च व्याख्यातः । ग्रन्थोऽयं सव्याख्यः निर्णयसागरमुद्रणालये मुद्रितः ।
(घ) महीधरकृतः - योगवासिष्ठसारः सव्याख्यः । ग्रन्थोऽयं बरोडापुस्तकालये लन्दनपुस्तकालये बाभ्बेविश्वविद्यालयहस्तलिखितपुस्तकालये च लभ्यते । 
(ङ) माधवसरस्वतीकृता - वासिष्ठपञ्चिका । ग्रन्थोऽयं अनन्तशयनपुस्तकालये लभ्यते ।
(च) काष्मीरपण्डिकृतम् - ज्ञानवासिष्ठम् ।
(छ) कृष्णय्यकृतः - ज्ञानवासिष्ठसारसमुच्चयः । इमौ द्वावपि ग्रन्थौ मद्रासराजकीयहस्तलिखितपुस्तकालये लभ्येते ।
%%% Chat
एतेषु केचन ग्रन्थास्तेलुगुलिप्यामेव सन्ति ।
२१. शुकः
शुकाचार्यापरनामानः बहवो वेदान्तिन आसन्निति ज्ञायते । कुत्रचित् शुकभगवत्पाद इति, कुत्रचित् शुकयोगीति नामानि बहुनि श्रूयन्ते । शुकाष्टाककर्ता शुक अन्यः ब्रह्मसूत्रभाष्यकर्ता शुक अन्य इत्येव ज्ञातुं पार्यते । अत्र प्रबलतरप्रमाणानि तु नैवोपलभ्यन्ते । यदि अद्वैतसम्प्रदायप्रवर्तकाचार्येषु परिगणितश्शुकाचार्यस्स्यात् तर्हि शुकाष्टककर्ताय महाभारतादिप्रसिद्धः कृष्णद्वैपायनात् शुकीरूपधारिण्यां घृताच्यांं जातः महान् ज्ञानीति सिघ्यति । एवञ्चास्य कालादिकथनं दुश्शकम् । इदन्तयाऽनिर्णीतत्वात् ।
(क) शुकाष्टकम् - व्यासपुत्राष्टकमित्यपरनामाय ग्रन्थः जीवन्मुक्तमहिमानं प्रदर्शयति । ग्रन्थश्चायं मुद्रितः । अस्य व्याख्यापि गङ्गाधरेन्द्रसरस्वतीकृता नासिकसूच्यां दृश्यते ।
(ख) ज्ञानबोधः - अमुद्रितोऽयं पूर्णग्रन्थ अडयारपुस्तकालये (9. B. 23) सरस्वतीमाहालये च लभ्यते ।
(ग) ब्रह्मसूत्रवृत्तिः - अस्य कर्ता शुकभगवत्पादाचार्य इति निर्दिष्टम् । ग्रन्थोऽयमुद्रितः पञ्जाब सूूच्यां 719 दृश्यते ।
२२. सनत्सुजातः
ब्रह्मणो मानसः पुत्रोऽयमिति महाभारतादिषु प्रसिद्धिः । तस्मादस्य कालनिर्णयो न शक्यते । महाभारतयुद्धस्य कालः क्रिस्तो पूर्वमिति विविधमतभेदेन विमर्शः कृतः । तस्मात् क्रिस्तोः पूर्ववर्ती युगान्तरीयस्सनत्सुजात इति परं वक्तुं शक्यते ।
(क) सनत्सुजातीयम् - मुद्रीतोऽयं ग्रन्थः बहुत्र मुद्रणालयेषु । अस्य व्याख्या शङ्करभगवत्कृता च वाणीविलासमुद्रणालये आनन्दश्रममुद्रणालये च मुद्रिता ।
एवं व्यासात् शङ्करभगवत्पादाच्च प्राक्तनाः प्रायशः सर्वे वेदान्ताचार्या निरूपिताः । अद्वैतसम्प्रदायस्य प्रवर्तकाचार्याणां परम्परा नारायणादारब्घेति प्रसिद्धा । प्रसिद्धा चेयं परम्परा-
``नारायणं पद्भभुवं वसिष्ठं शक्तिञ्च तत्पुत्रपराशरञ्च ।
व्यासं शुकं गौडपदं महान्तं गोविन्दयोगीन्द्रम् ।
अथास्य शिष्यं श्री शङ्कराचार्यमथास्य पद्मपादञ्च हस्तामलकञ्च शिष्यम् ।
तं तोटकं वार्तिककारमन्यान् अस्मद्गुरून् सन्ततमानतोऽस्मि ।" इति पद्येन ।
एतेषु वेदान्ताचार्येषु पराशरशक्त्योर्विषये न किमपि ज्ञातुं शक्यते । तस्मात् इतः परं बादरायणव्यासादारब्धाः क्रैस्तवीयर्विशतिशतकान्ताः अद्वैतवेदान्तसाहित्ये प्रसिद्धतमप्रमाणभूतग्रन्थप्रणेतारः मूलग्रन्थव्याख्याग्रन्थप्रणेतराश्चाद्वैताचार्याः कालक्रमेण निरूप्यन्ते ।
२३. बादरायणव्यासः (400 A. D. कलात् प्राक्)
अष्टादशपुराणानां रचयिता कृष्णद्वैपायनापरनामा व्यास एव बादरायण व्यास इति ब्रह्मसूत्राणां रचयितेति च साम्प्रदायिकाः । केचित्तु ब्रह्मसूत्रकारः बादरायणव्यासोऽयं अष्टादशपुराणकर्तुः कृष्णद्वैपायनाद्भिन्न इति वर्णयन्ति । पाणिनेरप्ययं ब्रह्मसूत्रकारः प्राक्तनः । यतः ``पाराशर्यशिलालिभ्यां भिक्षुनटसूत्रयो" रिति पाणिनिना ब्रह्मसूत्राणि भिक्षुसूत्रनाम्ना निर्दिष्टानि । ब्रह्मसूत्रपदैश्चैव हेतुमद्भिर्विनिश्चितैः (गीता. 13.4) इति बादरायणकृतान्येव ब्रह्मसूत्राणि विनिर्दिष्टानि, इति भगवद्गीताव्याख्यातृश्रीधरस्वामिनामभिप्रायः । केचित्तु गीतायां निर्दिष्टानि ब्रह्मसूत्राणि नैतानि, परन्त्वन्यानि तानि च नष्टानीति वदन्तः पुराणादिकर्तुः व्यासात् ब्रह्मसूत्रकारमन्यं मन्वते ।
सर्वकारणकारणं एकमेवाद्वितीयं सत्यज्ञानानन्दस्वरूपं अनन्तं असङ्गम्, नित्यं च ब्रह्म, तज्जानान्मोक्ष इत्यादिकं सर्वं अलौकिकविषयान्तर्गतम् । एतादृशेऽलौकिक विषये अनाद्यपौरुषेयश्रुतिरेव प्रमाणम् । व्यासाचार्यास्तु अलौकिकविषये अनाद्यपौरुषेयश्रुतिमेव प्रमाणमावेदयन्ति । बादरायणेन ब्रह्मसूत्रमुखेन तत्वान्युपदिश्यन्ते । तादृशं श्रुतिसम्मतमद्वैतमेवेति प्रदर्शयितुं भगवान्नारायण एव कृष्णद्वैपायनापरबादरायणव्यासो भूत्वा ब्रह्मसूत्राणि विरचयामास । बादरायणव्यासोऽयमद्वैतीत्यत्र प्रमाणानि कानिचन दृश्यन्ते-शाण्डिल्यभक्तिसूत्रे 30 आत्मैकपरां बादरायणः" इति दृश्यते । एवं बादरायणव्यास एव अष्टादशपुराणानां कर्ता, स च अधिकारिभेदं मनसि कृत्वा मन्दाधिकारिणां तत्वोपदेशाय अरूपं निर्गुणं अनिर्वचनीयं च आत्मस्वरूपं सरूपं सगुणं स्तुतिविषयं कृत्वा पुराणानि रचयामासेति वदन्त पद्यमिदं व्यासकृतत्वेन प्रसिद्धं प्रमाणयन्ति -
``रूपं रूपविवर्जितस्य भवतो ध्यानेन यत्कल्पितम्,
स्तुत्यानिर्वचनीयताखिलगुरो दूरीकृता यन्मया ।
व्यापित्वञ्च निराकृतं तु भवतो यत्तीर्थयात्रादिना
क्षन्तव्यं जगदीश तद्विकलतादोषत्रयं मत्कृतम् ।।" इति ।
तस्मात् अष्टादशपुराणादिकर्तुः ब्रह्मसूत्रकर्तुश्च बादरायणव्यासस्य कालादि निर्णेतुं न शक्यते । तथापि प्रो. हिरियण्णामहाशयास्तु ब्रह्मसूत्रनिर्माणकालः (400 A.D.) इति वदन्ति ।
(क) ब्रह्मसूत्राणि - शारीरकमीमांसासूत्रमित्यपरनामायं ग्रन्थः समन्वयाविरोधसानफलाख्यैश्चतुर्भिरध्यायैः पूर्णः । प्रत्यध्यायं चत्वारः पादा विद्यन्ते । प्रतिपादं बहून्यधिकरणानि । प्रत्यधिकरणं भिन्नविषयम् । एषु सूत्रेषु बादरायणेन उपनिषदां तदर्थानां च सङ्कलनं कृतम् । शङ्कराचार्यमतेन सूत्रसमष्टिसंख्या 555,अधिकरणसंख्या 192 । अस्य भाष्यं शङ्करभगवत्पादैः कृतं शारीरकमीमांसाभाष्यमित्याख्यम् । यदाधारं कृत्वा परश्शतानि ग्रन्थरत्नानि अद्वैतवेदान्तसाहित्ये प्रकाशन्ते ।
(ख) सिद्धान्तदर्शनम् - पूर्वोत्तराम्नायभेदात् द्विविधा हि मीमांसा । तत्रोत्तरमीमांसा पुनर्द्वेधा वादिबुबुत्सुप्रतिपादनभेदात् । तत्राद्या ब्रह्मसूत्राख्या, द्वितीया सिद्धान्तसूत्रात्मिका । एवञ्च ``अथातो ब्रह्मजिज्ञासा" इत्यादिकं वादिप्रतिपादनाय कृतम् । इदन्तु ``सिद्धान्तदर्शनं" बुबुत्सुप्रतिपादनाय कृतमिति विशेषः । सूत्ररूपेऽस्मिन् ग्रन्थे समग्राद्वैत्तसिद्धान्ताः प्रतिपाद्यन्ते । केचित्तु साम्प्रदायिका अपि अस्य ग्रन्थस्य बादरायणव्यासप्रणीतत्वे सन्दिह्यन्ति । ग्रन्थोऽयं आनन्दाश्रम मुद्रणालये मुद्रितः । अस्य भाष्यं विश्वदेवेन रचितं ``निरञ्जनभाष्या"ख्यमपि मुद्रितम् ।
(ग) भगवद्गीता - (महाभारतान्तर्गता) एतामधिकृत्य स्विस्तरं गीताप्रस्थाने प्रतिपादितमिति नेह प्रतन्यते ।
२४. गौडपादाचार्यः (500 A.D.)
गौडपादाचार्या इमे शुकमुनीन्द्रशिष्या इति, अद्वैताचार्यपरम्परायां महनीयतमा इति च ``नारायणं पद्भभुवमि"त्यादिश्लोकात् ज्ञायते । नृसिम्हतापनीयोपनिषदां गौडपादकृते व्याख्याने ``इति श्रीपरमहंसपरिव्राजकाचार्यश्रीमच्छुकमुनीन्द्रशिष्य गौडपादविरचिते उत्तरतापनीयोपनिषद्विवरणे प्रथमः खण्डः, नवमः खण्डः" इति (D. 581, 582 MGOML) ग्रन्थेऽमुद्रिते दृश्यते । एवं श्वेताश्वतरोपनिषदां शाङ्करभाष्ये ``तथाच शुकशिष्यो गौडपादाचार्यः" (Page 30, ASS 17) इति दृश्यते । एवमेव लक्ष्मणशास्त्रिविरचिते गुरुवंशकाव्ये (12. VSS) दृश्यते । तस्मान् शुकशिष्यो गौडपादाचार्य इति नीश्चीयते ।
गौडपादाचार्यस्य स्थानं नाद्यापि निश्चितम् । विषयेऽस्मिन् विभिन्ना एव विचारा दृश्यन्ते । परन्तु मान्डूक्यकारिकाशङ्करभाष्यव्याख्याने आनन्दगिरीये अलातशान्तिप्रकरणस्थस्य ``तं वन्दे द्विपदां वर"मिति पद्यांशस्य व्याख्याने एवं दृश्यते -``आचार्यो हि पुरा वदरिकाश्रमे नरनारायणाधिष्ठिते नारायणं भगवन्तमभिप्रेत्य तपो महदतप्यत ।" इति (Page 157 ASS 10) दृश्यते । तस्मात् बदरिकाश्रम एव गौडपादस्य स्थानमिति ज्ञायते । सप्तदशशतकीयेन बालकृष्णान्दसरस्वत्या स्वीये शारीरकमीमांसाभाष्यवार्तिके तु ``गौडचरणाः कुरुक्षेत्रगता हीरारावतीनदीतीरभवगौडजातिश्रेष्ठाः, देशविशेषभवजातिनाम्नैव प्रसिद्धाः द्वापरयुगमारभ्यैव समाधिनिष्ठत्वेन आधुनिकैरपरिज्ञातविशेषाभिधानास्सामान्यनाम्नैव लोकविख्याता" (Page 6. AS.I) इति प्रतिपादितम् । एवञ्च गौडपादः कुरुक्षेत्रवासी गौडजात्युत्पन्न इति गौडपादीयनामान्तरापरिज्ञाने च कारणं सूचितम् । केचित्तु गौडपादाचार्यं गौडदेशभवं वदन्ति । अत एव देशनाम्ना तेषां व्यवहारः । यथागौडब्रह्मानन्द इत्यादि । एवञ्चैतत्पक्षे गौडपादः बंगालदेशानां उत्तरभागवर्तीति सिध्यति ।
भारतीयाद्वैतवेदान्तपरम्पराप्रामाण्येन गौडपादश्शुकशिष्यः, शङ्करश्च गौडपादशिष्यः, गौडपादेनानुगृहीतश्चेति शङ्करात्पूर्वतनो वा शङ्करकालपर्यन्तजीवी वेति निश्चीयते । यदि वयं म. म. कुप्पुस्वामिशास्त्रिमहाशयानां सिद्धान्तमनुसृत्य 632-661 A.D. कालवर्तिनं शङ्करमभ्युपगच्छामस्तर्हि तैरेव प्रतिपादितसिद्धान्तमनुसृत्य गौडपादकालः (520-620 A.D.) इति, इच्छामात्रशरीरत्यागिनां गौडपादाचार्याणां कालश्शङ्कराचार्यानुग्रहपर्यन्तमिति वा स्वीकर्तव्यम् । एतेन शङ्करदिग्विजयादिवचनञ्च सङ्गतं भवति । विधुशेखरभट्टाचार्यास्तु स्वसम्पादिते `आगमशास्त्र' ग्रन्थोपोद्धाते एवं वदन्ति -``द्वितीयशतकादारभ्य चतुर्थशतकपर्यन्तानां बौद्धपण्डितानां ग्रन्थस्य गौडपादकारिकायाश्च शब्दसाम्यदर्शनात् गौडपादस्तदर्वाग्भव इति तथाच तन्मतरीत्या गौडपादकालः (500 A.D.) इति सिध्यति । यद्येवं गौडपादाचार्यस्य शङ्कराचार्यप्राचार्यत्वं कथम् ? किमन्योऽयं गौडपादः ? उतान्योऽयं शङ्करः ? न वा शङ्करः गौडपादेनानुगृहीतः, नापि प्रशिष्य इत्यभ्युपगम्य शङ्करदिग्विजयादिग्रन्थानामप्रामाण्यं स्वीकर्तव्यम् ? आहोस्वित् शङ्कराचार्यानुग्रहकालपर्यन्तजीवी गौडपादः ? इत्यादयस्संशयविशेषाः स्वतस्समुद्भवन्ति ।"
गुरुपादहालदारस्तु ``वृद्धत्रय्यां" (Page 307) गौडपादाचार्यः गोविन्दभगवत्पादशिष्यः शङ्कराचार्यकामदेवभूपालयोः परमगुरुः वेदान्तसम्प्रदायप्रवर्तकः, माण्डूक्यकारिकाकृदद्वैती (700 A. D.) कालवर्तीति प्रतिपादयति ।
(क) माण्डूक्यकारिका - (ASS. 10) गौडपादकारिकाभिधेऽस्मिन् माण्डूक्योपनिषदां व्याख्यात्मके ग्रन्थे चत्वारि प्रकरणानि सन्ति । तत्र प्रथमेऽऽगमाख्यप्रकरणे एकोनत्रिंशत्, द्वितीये वैतथ्याख्ये अष्टात्रिंशत, तृतीयेऽद्वैताख्ये अष्टाचत्वारिंशत्, चतुर्थेऽलातशान्तिप्रकरणे शतमिति 215 कारिकास्सन्ति । मुद्रितश्चायं ग्रन्थ आनन्दाश्रममुद्रणालये ।
आगमप्रकरणम् - इदमागमप्रकरणं माण्डूक्योपनिषदां भावार्थरूपं आगममूलकत्वात् अन्वर्थनाम । प्रकरणेऽस्मिन् अकारोकारमकारैः प्रतिपादितेभ्यः वैश्वानर हिरण्यगर्म-ईश्वरेभ्यः, जाग्रत्स्वप्नसुषुप्त्यवस्थाभ्यश्च भिन्नं तदनुगतं साक्षिरूपं च परमात्मतत्वं ``तुरीय" इति नाम्ना वर्णितम् ।
वैतथ्यप्रकरणम् - द्वितीयेऽस्मिन् प्रकरणे दृश्यप्रपञ्चस्य मायामयत्वं मिथ्यात्वञ्च सयुक्तिकं साधितम् । आत्मा एक एव नित्यः, तस्मिन् विविधकल्पनावशात् प्रपञ्चस्तोत्पत्तिरिवि विकल्पो भवति । अस्य मूलकारणं माया । मायाकल्पितजगतः गन्धर्वनगरवत् असत्यत्वमिति प्रतिपाद्य ``न निरोधोनचोत्पत्ति" रित्यादिना अखण्डचिद्धनानन्दआत्मतत्वादन्यस्यासत्वं प्रतिपादितम् ।
अद्वैतप्रकरणम्-तृतीयेऽद्वैताख्यप्रकरणेऽस्मिन् अनेकाभिस्सुदृढाभिर्युक्ति भिरद्वैतत्वं साधितम् । आत्मनि सुखदुःखभावना नितरां असङ्गता । यथा बालाः धूलिधूमादिसंसर्गेणाकाशं मलिनमामनन्ति, वस्तुतः यथा च आकाशो मालिन्यशून्यः तथैवात्मनोऽपि सुखित्वदुःखित्वकथनं बालबुद्धिविलासतुल्यमिति प्रतिपादितम् । असङ्गोह्यात्मा । माया हि द्वैतकल्पनायाः कारणम् । अमृतस्य मर्त्यत्वं, मर्त्यस्य अमृतत्वञ्चासङ्गतम् । अत अमृतस्यात्मनः यदि उत्पत्तिस्स्वीक्रियते तर्हि मर्त्यत्व धर्म आपद्येत इति आत्मनः उत्पत्ति - जातिः नास्ति इति प्रतिपादितम् । अयमेव गौडपादाचार्याणां अजातिवादः । एतच्च 1-17, 2-31, 32, 3-4, श्लोकेषु प्रतिपादितम् । अयमजातिबादः गौडपादात् प्राचीनस्य बौद्धाचार्यस्य दिङ्नागस्य माध्यमिकवृत्तौ, पालिभाषाप्रणीतबौद्धग्रन्थेषु च समुपलब्धेस्ततो गृहीत इति केचिद्वदन्ति । पालीभाषाया अपि प्राचीनासूपनिषत्सु ``अजायमानो वहुधा व्यजायत" इत्यादिदर्शनात् तेषामुक्तेरनुपपत्तौ भारतीयाः प्रमाणम् ।
``अलातशान्तिप्रकरणम्"- चतुर्थेऽस्मिन् प्रकरणे यथा अलाते भ्रमिते सति गोलाकारप्रतीतिर्जायते परन्तु सा गोलाकारभ्रमणजन्या एव न वस्तुतः, एवं जगदादि मायाकल्पितमेव । मनसो व्यापारादेव तस्योत्पत्तिः, मनसः निरोघे च स नास्त्येव । यथा च भ्रमणादिक्रियाशान्तौ गोलाकारकप्रतीतेश्शान्तिः, एवं मनस अमनीभावात् जगतश्शान्तिः । जगदुत्पत्तिलयौ प्रतीत्यप्रतीती उभावपि भ्रान्तिजनितावेव । परमार्थतः परमार्थतत्वं पारमार्थिकमिति प्रतिपादितम् । अद्वैतवेदान्तस्य प्राणभूताऽनिर्वचनीयख्यातिरपि प्रकरणेऽस्मित् प्रतिपादिता । ``विपर्यासात् यथा जाग्रदित्यादिना (4-41) एवं ``न निर्गतास्ते विज्ञानादित्या" दिना (4-52) ``उभेह्यन्योन्यं दृश्येते" इत्यादिना च (4-67) पद्येन प्रदर्शिता ।
प्रकरणस्यास्य भाषा ``विज्ञप्ति" रित्यादिपारिभाषिकशब्दैः पूर्णा । एवं मङ्गलाचरणश्लोके ``तं वन्दे द्विपदां वरम्" इत्यत्र द्विपदां वरशब्दश्च प्रयुक्तः । एते शब्दाः बुद्धमतग्रन्थेषु दृश्यन्त इति केचन बुद्धमतमेव गौडपादः वेदान्तापदेशेन प्रतिपादयतीति प्रच्छन्नबौद्धा अद्वैतिन इति वदन्ति ।
परन्तु शब्दसाम्यं नात्र प्रमाणमकिञ्चित्करञ्च । यत अध्यात्मशास्त्राणां पारिभाषिकशब्दाः न केवलं बौद्धानां स्वम् । परन्तु ते सर्वदर्शनसामान्याः । तेषां प्रयोगे यथा गौडपादस्य तथा बौद्धानां यथा बौद्धानां तथा गौडपादस्येति सर्वेषामधिकास्समस्ति । द्विपदांवर शब्दस्य पुराणादिष्वपि भूरिशः प्रयोगः दृश्यते । भारतरामायणादीनां बुद्धादपि प्राचीनत्वं प्रसिद्धमेव । नलभीमार्जुनभीष्मादिषु शब्दोऽयं प्रयुक्तः दृश्यते । महाभारते नारायणीयपर्वाध्याये द्विपदांवरार्थकं द्विपदां वरिष्ठपदं प्रयुक्तं दृश्यते । न वा एतत्पदं कोषग्रन्थेषु बुद्धपरत्वेन व्याख्यातम् । तस्मात् यौगिकश्शब्द एवैषः न तु योगरूढः ।
माण्डूक्यकारिकाचेयं शङ्करभगवत्पादैर्व्याख्यातम् । आनन्दगिरिव्याख्योपेतं भाष्यं आनन्दाश्रममुद्रणालये मुद्रितम् । स्वयम्प्रकाशानन्दसरस्वतीकृता मिताक्षरानाम्नी माण्डूक्यकारिकाव्याख्या वाराणस्यांं (BSS 48) मुद्रिता । उपनिषद्ब्रह्मकृता व्याख्या अडयारपुस्तकालये मुद्रिता । अनुभूतिस्वरूपाचार्यकृतं गौडपादीयभाष्यटिप्पणं (R. 2911 MGOML) लभ्यते । अज्ञातकर्तृकः गौडपादीयविवेकनामा ग्रन्थोऽपि (ई. 3882 d MGOML) लभ्यते । गौडपादाचार्यप्रणीतत्वेन प्रसिद्धाः ग्रन्थाः-
(ख) उत्तरगीताव्याख्या - ग्रन्थोऽयं तिरुपति सृच्यां (DCVORIT) अनन्तशयनपुस्तकालये (275 TCL) जयपुर पोटीखानासूच्यां (XXXIII 74/4) च दृश्यते ।
(ग) पञ्चीकरणवार्तिकम् - बरोडासूच्यां (13325 c BRD) दृश्यते ।
(घ) नृसिम्हतापनीयभाष्यम् - (D. 581 MGOML)
(ङ) अनुगीताभाष्यम् - ग्रन्थोऽयं नासिक सूच्यां दृश्यते ।
(च) श्रीविद्यारत्नसूत्रम् - (275 TCL)
(छ) दुर्गासप्तशती व्याख्या - ग्रन्थोऽयं तन्त्रदर्शनाचार्येण भास्कररायेण दुर्गासप्तशती व्याख्याने निर्दिष्टः ।
(ज) सुभगोदयः - (275 TCL)
(झ) सांख्याप्रवचन भाष्यम् ?
२५. मण्डनमिश्रः (750-850 A.D.)
मण्डनमिश्रोऽयं कुमरिलभट्टस्य शिष्यश्शङ्कराचार्यकाले प्रसिद्धः पूर्वमीमांसापण्डितः कर्मनिष्ठश्चेति प्रसिद्धिः । अस्यैव विश्वरूप इति नामन्तरम् । मण्डनमिश्रस्यैव शङ्कराचार्यात् आश्रमस्वीकारपूर्वकशिष्यत्वस्वीकारादनन्तरं सुरेश्वराचार्य इति नामेति सम्प्रदायविदः । ``जागोपि" महाशयेन नैष्कर्म्यसिद्धिभूमिकायां मण्डनमिश्र-विश्वरूपसुरेश्वराणां ऐक्यमङ्गीकृतम् । सप्तदशशतकीयेन बालकृष्णानन्दसरस्वत्या कृते शारीरकमीमांसाभाष्यवार्तिके च मण्डनमिश्रसुरेश्वरविश्वरूपाणामैक्यमेव वर्णितम् ।
दासगुप्तमहाशयास्तु सुरश्वरविश्वरूपावभिन्नौ मण्डनमिश्रस्तु अन्य एवेति (H. I. P. Vol. II) ग्रन्थे निर्दिशन्ति । हिरियण्णामहाशयास्तु (J. R. A. S. 1924) रायलासियाटिक सोसाइटि पत्रिकायाश्चतुर्विशतितमे भागे सुरेश्वरः मण्डनादन्य इति निश्चिन्वन्ति । म. म. कुप्पुस्वामिशास्त्रिणश्च स्वसम्पादितब्रह्मसिद्धि भूमिकायां सुरेश्वरब्रह्मसिद्धिकारयोस्सिद्धान्तगतभेदमुपवर्ण्य सुरेश्वरादन्यं ब्रह्मसिद्धिकारं मण्डनं वर्णयन्ति स्म ।
बिब्लियोथिकाइण्डिकासीरीजमुद्रितायां पराशरस्मृतिव्याख्यायां (Page 51) बृहदारण्यकवार्तिकात् उद्घृतम् । तच्चोद्धरणं विश्वरूपाचार्यकृतग्रन्थादित्युक्तम् । एवं विद्यारण्यैः विवरणप्रमेयसंग्रहे (Page 92) बृहदारण्यकवार्तिकात् (IV. 8.) उद्धरणं दत्तम् । तत्रापि सुरेश्वरः विश्वरूपशब्देनैव निर्दिष्टः । तस्मात् विश्वरूपसुरेश्वरावभिन्नौ, मण्डनस्त्वन्य इति निश्चयः ।
ब्रह्मसिद्धिकारेणानेन शङ्करात्पूर्वतनाः ग्रन्थाः प्रमाणत्वेन निर्दिष्टाः । वाचस्पतिमिश्रेण च ब्रह्मसिद्धिं प्रमाणं कृत्वा ब्रह्मतत्वसमीक्षा कृता । अत एव मण्डनपृष्ठसेवी वाचस्पतिरिति च प्रसिद्धिः । तस्मात् वाचस्पतिकालिको वा, तस्मात् पूर्वतनो वा भवितुमर्हति । शङ्कराचार्यसामयिक इति तु सम्प्रदायविदः । दासगुप्तमहाशयेन नवमशतकीयोऽयमिति बर्ण्यते । कुप्पुस्वामिशास्त्रिणस्तु (615-695 A. D.) इति सप्तमशतकीयं वर्णयन्ति ।
ब्रह्मसिद्धिः - ग्रन्थोऽयं मद्रासराजकीय पुस्तकालये (MGOMLS 4) सव्याख्यः मुद्रितः । अस्याः व्याख्या वाचस्पतिमिश्रकृता ``ब्रह्मतत्वसमीक्षा" चित्सुखाचार्यकृता ``अभिप्रायप्रकाशिका" आनन्दपूर्णविद्यासागरकृता टीकारत्नापरनामा ``भावशुद्धिः" शङ्खपाणिकृता ब्रह्मसिद्धिटीका चेति ग्रन्थाः वर्तन्ते ।
(ख) विभ्रमविवेकः - 162 पद्यैः पूर्णोऽयं ग्रन्थः पञ्चख्यातिव्याख्यात्मकः । मुद्रितश्चायं जर्नलआफ ओरियण्टल पत्रिकायां (J. O. R.) मद्रासनगरे । अन्येऽपि ग्रन्थाः मीमांसाशास्त्रे कताः ।
26. शङ्करभगवत्पादः (788-820 A. D.)
शङ्कराचार्यवतारसमये धर्मपरिस्थितिः-
सनतनकालत एव प्रबलप्रमाणपूर्वकं आत्यन्तिकनिःश्रेयसाधिगमसाधनत्वेन श्रुतिपुराणभगवद्गीतोपनिषद्भिः महद्भिराचार्यैश्च निरूपितोऽयमद्वैतसिद्धान्तः ।
 ``"


ऽ  ?
``" ``" ``" ``" ``" ``" ``" ``" ``" ``" ``" ``" ``" ``" ``" ``" ``" ``" ``" ``" ``" ``"
`' `' `' `' `' `' `' `' `' `' 

ऽ  ।   ॥ ?
